% defun library.
% Guy L. Steele Jr.
% Copyright 1985, 1986 , 1988, 1989 Guy L. Steele Jr.

%  3/18/85 12:48  Now \@defunname generates index entries.
%  5/22/85 14:16  Use \noexpand before \tt and \rm in index entries.
%		  This will delay their processing until the point
%		  when the index is read back in again.
%  5/23/85 17:01  Add special check for nil as an init value for an
%		  &optional or &key argument; make it be \tt, not \it.
% 12/17/85 12:56  Add "lisp" environment.
%  1/02/86 14:15  Base "lisp" environment on tabbing instead of raggedright.
%  1/02/86 17:00  Make \endlisp use \ignorespaces.
%  1/03/86 17:00  Put a space into the index key to make sorting work better.
%  1/31/86 12:32  Simplify "lisp" environment to let the tabbing environment
%		  take care of interparagraph spacing.
%  1/31/86 13:27  Add \smallcaps stuff.
%  1/31/86 13:38  In \@defunkeymode et al., assume the user provides a :.
%  1/31/86 16:03  Flush \smallcaps stuff.
%  1/31/86 17:03  Use \addvspace in \defun et al.
%  2/04/86 17:15  Add feature so that the syntax \begin{defun}[Frob] causes "[Frob]"
%		  to appear at the right margin, as in the Common Lisp book, and causes
%		  the index entry for function baz to be "baz frob".  If no square
%		  brackets ae used, then nothing appears at the right margin, and the
%		  index entry looks like "baz \lowercase{default}" where "default" is
%		  the definition of \defaultdefuntype, which is initialized by this file
%		  to {Function} but can be redefined by the user.
%  2/07/86 17:08  Make lisp environment use \frenchspacing.
%  3/11/86 11:42  Improve some of the breaking places in the defun headers:
%		  Introduce \hbox around every name, and make better pseudo-hyphens.
%		  Also fix a bug in \@defuninitvalue.
%  3/12/86 11:47  Hair up the defunbreak stuff some more.  Now we have different
%		  behavior depending on whether the function name is long or short.
%  3/12/86 16:09  Fix a possible bug in \@showdefuntype.
%  3/13/86 14:16  More fixes to hairy defunbreak stuff.
%  5/29/86 01:47  Add \internalroutine.
%  6/04/86 14:47  Correct spacing in \@showdefuntype.
%  6/29/88 00:31  Remove priority-box macros.

\catcode`@=11	% Make it possible to refer to some LaTeX utility macros.
\catcode`&=11

\def\lisp{\cf\tabbing}
\let\endlisp=\endtabbing

\def\@defunstart{\noindent\leavevmode     % Need this to trigger the \everypar for margin rules.
  \begingroup
  \samepage  \raggedright \cf \catcode`&=11
  \hyphenpenalty=-5  \exhyphenpenalty=\@M  \brokenpenalty=\@M}

% The addvspace used to be "plus 4pt" but that was not enough stretch relative to \parskip.
\def\defun{\@ifstar{\@defunstart{}\@ifnextchar[{\@defuntyped}{\@defununtyped}}{\relax
   \addvspace{18pt plus 12pt minus 6pt}\@defunstart    %Margin macros know this amount
   \@ifnextchar[{\@defuntyped}{\@defununtyped}}}


\def\@showdefuntype{\setbox0\hbox{\hskip1.5em$\lbrack$\it\@defuntype\/$\rbrack$}\relax
		    \setbox1\hbox{\hskip\textwidth\hskip-1\wd0}\relax
		    \leavevmode\parshape=2  0pt 1\wd1  0pt \textwidth  \relax
		    \hbox to 0pt{\hbox to \textwidth{\hss\box0}\hss}\let\@defuntype\@defunsecondtype}

\def\@defununtyped{\let\@defunshow=\relax \@defun}
\def\@defuntyped[#1]{\def\@defuntype{#1}\let\@defunsecondtype\@defuntype
                     \let\@defunshow=\@showdefuntype
                     \@ifnextchar[{\@defuntypedtwice}{\@defun}}
\def\@defuntypedtwice[#1]{\def\@defunsecondtype{#1}\let\@defunshow=\@showdefuntype
                          \@defun}

\def\@defun{\@ifnextchar?{\@defunname}{\@defunname}}	% Skips spaces before name 

\def\enddefun{\par
  %%\message{end{defun} for \currentdefun}%For debugging
  \@ifstar{}{\addvspace{6pt plus 6pt minus 2pt}}}    %Margin macros know this amount
% Used to be "plus 2pt" but that was not enough stretch relative to \parskip.

% 1: entry
% \@defuntype: \lowercase{category}
\def\@defunname#1 {\@defunshow
%  \typeout{defunname}\typeout{ 1: #1}\typeout{ type: \lowercase{\@defuntype}}
  #1\@defunindex#1:\par{\@defuntype}\relax
  %%\message{begin{defun} for #1}\global\def\defun@name{#1}%For debugging
  \setbox0\hbox{#1}\relax
  \ifdim 1\wd0 < 1.3in\relax
     \def\@defunbreak{\discretionary{\hbox{}}{\setbox0\hbox{#1 }\hbox{\hskip 1\wd0}}{\hbox{ }}}\relax
     \let\@firstdefunbreak=~\relax
  \else
     \def\@defunbreak{\discretionary{\hbox{}}{\hbox{\hskip2em}}{\hbox{ }}}\relax
     \let\@firstdefunbreak=\@defunbreak
  \fi
  \def\@defunkeywordbreak{\@firstdefunbreak\let\@defunkeywordbreak=\@defunbreak}\relax
  \def\@hairydefunbreak{\@firstdefunbreak
		        \let\@defunkeywordbreak=\@defunbreak
		        \let\@hairydefunbreak\@defunbreak}\relax
  \@defunreqmode}

% 1: entry
% 2: empty
% 3: \lowercase{category}
\def\@defunindex#1:#2\par#3{%
%  \typeout{defunindex}\typeout{ 1: #1}\typeout{ 2: #2}\typeout{ 3: #3}%
  \def\@tempa{}\def\thing{#2}\ifx\thing\@tempa
  \@idefunindex{}#1:\par{#3}\else \@idefunindex{#1:}#2\par{#3}\fi}
% 1: empty
% 2: entry
% 3: \lowercase{category}
\def\@idefunindex#1#2:\par#3{%
%  \typeout{idefunindex}\typeout{ 1: #1}\typeout{ 2: #2}\typeout{ 3: #3}%
  \index{#2!#3}}%{{#2 }{\noexpand\noexpand\noexpand\cf #1#2\ \noexpand\noexpand\noexpand\rm #3}}}


\def\@newlinecheck#1#2{\def\@tempa{#1}\def\@tempb{\\}\ifx\@tempa\@tempb \\*\let\@tempd\@defunname
  \else \def\@tempd{#2}\fi \@tempd}

\def\@ifparnext#1#2{\def\@tempa{#1}\def\@tempb{#2}\futurelet\@tempc\@ifparnx}
\def\@ifparnx{\ifx\@tempc\@@par \let\@tempd\@tempa \else \let\@tempd\@tempb \fi \@tempd}


\def\@defunreqmode{\@ifparnext{\@defundone}{\@ifnextchar &{\@defunkeyword}{\@ifnextchar({\@defunparenreqarg}{\@defunreqarg}}}}
\def\@defunreqarg#1 {\@newlinecheck{#1}{\@hairydefunbreak\hbox\textit{#1\/}\@defunreqmode}}
\def\@defunparenreqarg(#1 #2) {\@hairydefunbreak(\hbox\textit{#1\/}~#2)\@defunreqmode}  %CLOS methods

\def\@defunrestmode{\@ifparnext{\@defundone}{\@ifnextchar &{\@defunkeyword}{\@defunrestarg}}}
\def\@defunrestarg#1 {\@newlinecheck{#1}{\penalty\@m\ \hbox\textit{#1\/}\@defunrestmode}}

\def\@defunoptionalmode{\@ifparnext{\@defundone}{\@ifnextchar &{\@defunkeyword}{\@ifnextchar({\@defunparenoptionalarg}{\@defunoptionalarg}}}}
\def\@defunoptionalarg#1 {\@newlinecheck{#1}{\@hairydefunbreak\hbox\textit{#1\/}\@defunoptionalmode}}
\def\@defunparenoptionalarg(#1 #2) {\@hairydefunbreak(\hbox\textit{#1\/}~\@defuninitvalue#2)\@defunoptionalmode}

\def\@defuninitvalue#1#2){{\ifcat#1A\def\@tempa{#1#2}\def\@tempb{nil}\ifx\@tempa\@tempb
  \cf \else \it \fi \else \cf \fi #1#2})}

\def\@defunkeymode{\@ifparnext{\@defundone}{\@ifnextchar &{\@defunkeyword}{\@ifnextchar({\@defunparenkeyarg}{\@defunkeyarg}}}}
\def\@defunkeyarg#1 {\@newlinecheck{#1}{\penalty\@m\@defunkeywordbreak\hbox{#1}\@defunkeymode}}
\def\@defunparenkeyarg(#1 #2) {\penalty\@m~(\hbox{#1}~\@defuninitvalue#2)\@defunkeymode}

\def\@defunkeyword &#1 {\@defunkeywordbreak\hbox{\&#1}\csname @defun#1mode\endcsname}

\def\@defundone\par{\par\endgroup
   %%\edef\currentdefun{\defun@name}%For debugging
   \nobreak\addvspace{6pt}\noindent}



\begingroup
\catcode`\[=13 \catcode`\]=13 \catcode`$=13 \catcode`$=13
\catcode`\<=13 \catcode`\>=13 \catcode`\|=13 \catcode`\?=13
\global\def\@defmacstart#1{\relax
%  \noindent\leavevmode     % Need this to trigger the \everypar for margin rules.
  \begingroup \samepage  \topsep 0pt
  \catcode`&=11
  \def\@lbracehack{{$\,\lbrace$}\=\pushtabs\+\@backslashsetup\it}\relax
  \def\@rbracehack{\-\poptabs
       \@ifnextchar*{\@rbracesuper}{\@ifnextchar+{\@rbracesuper}{\/$\rbrace\,$\@backslashsetup\it}}}\relax
  \def[{{$\,\lbrack$}\=\pushtabs\+\@backslashsetup\it}\relax
  \def]{\-\poptabs$\/\rbrack\,$\@backslashsetup\it}\relax
  \def<{{$\,\dlbrack$}\=\pushtabs\+\@backslashsetup\it}\relax
  \def>{\-\poptabs$\/\drbrack\,$\@backslashsetup\it}\relax
  \def({{\cf \char40}\=\pushtabs\+\@backslashsetup\it}\relax
  \def){\-\poptabs{\cf \char41}\@backslashsetup\it}\relax
  \def|{{$\char124$}}\relax
  \def?{{$\downarrow$}}\relax
  \def\@finishdefmac{\-\poptabs\end{tabbing}\endgroup\everypar{}
       \addvspace{6pt}\noindent}\relax
  \expandafter\def\@carret{\expandafter\@ifnextchar\@carret
     {\@finishdefmac\@gobblecr}{\@tabcr*\@backslashsetup\it}}\relax
  \@defunhackbraces
  \def\@backslashsetup{\def\\{\-\poptabs\@tabcr*\@backslashsetup
				\@dodefmacname{#1}}}\relax
  \catcode`\[=13 \catcode`\]=13 \catcode`$=13 \catcode`$=13
  \catcode`\<=13 \catcode`\>=13 \catcode`\|=13 \catcode`\?=13 \catcode`\^^M=13
  \def\!##1!{\cd{##1}}\relax
  \begin{tabbing}\@backslashsetup\@margineveryparguts
  \@dodefmacname{#1}}
\endgroup
\def\@rbracesuper#1{{\/$\rbrace^{#1}\,$}\@backslashsetup\it}


\def\Mopt#1{{$\,\lbrack$}\textit{#1\/}{$\rbrack\,$}}
\def\Mchoice#1{{$\,\dlbrack$}\textit{#1\/}{$\drbrack\,$}}
\def\Mstar#1{{$\,\lbrace$}\textit{#1\/}{$\rbrace^*\,$}}
\def\Mplus#1{{$\,\lbrace$}\textit{#1\/}{$\rbrace^+\,$}}
\def\Mgroup#1{{$\,\lbrace$}\textit{#1\/}{$\rbrace\,$}}
\def\Mor{{$|$}}
\def\Mind#1{$\downarrow$\textit{#1\/}}

% The following is taken from /usr/local/lib/tex/macros/latex.tex
% but corrects a scoping bug.

\newdimen\@curtabmardimen
\def\@startline{\global\@curtabmar\@nxttabmar\relax
   \global\@curtabmardimen\dimen\@curtabmar
   \global\@curtab\@curtabmar\global\setbox\@curline\hbox
    {}\@startfield\strut}

\def\@stopline{\unskip\@stopfield\if@rjfield \global\@rjfieldfalse
   \@tempdima\@totalleftmargin \advance\@tempdima\linewidth
   \hbox to\@tempdima{\@itemfudge\hskip\@curtabmardimen
   \box\@curline\hfil\box\@curfield}\else\@addfield
   \hbox{\@itemfudge\hskip\@curtabmardimen\box\@curline}\fi}

% End of material from /usr/local/lib/tex/macros/latex.tex .

\def\defmac{\@defmacbegin{Macro}}
\let\enddefmac\enddefun

\def\defspec{\@defmacbegin{Special form}}
\let\enddefspec\enddefun

\def\defloop{\@defmacbegin{Loop clause}}
\let\enddefloop\enddefun

\def\@defmacbegin#1{\@ifstar{\@defmacstart{#1}}{\addvspace{18pt plus 12pt minus 6pt}\@defmacstart{#1}}}

{\catcode`\^^M=13\global\def\@carret{
}\global\def\@defmacnamecrgobble#1
{\@defmacname{#1}}}

\def\@dodefmacname#1{\cf\expandafter\@ifnextchar\@carret
   {\@defmacnamecrgobble{#1}}{\@defmacname{#1}}}

\def\@defmacname#1#2 {%\typeout{#1: #2 }\relax
  \setbox0\hbox{\hskip1.5em$\lbrack$\it#1\/$\rbrack$}\relax
  \leavevmode\hbox to 0pt{\hbox to \textwidth{\hss\box0}\hss}{\cf #2\ }\=\pushtabs\+\@backslashsetup
  \@defunindex#2:\par{#1}
  \it}

\bgroup
\catcode`\<=1 \catcode`\>=2 \catcode`\{=13 \catcode`\}=13
\global\def\@defunhackbraces<\catcode`\{=13\catcode`\}=13\let{\@lbracehack\let}\@rbracehack>
\egroup

\catcode`@=12		% Restore character codes
\catcode`&=4

%[End]
