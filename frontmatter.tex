%Part{Char, Root = "CLM.MSS"}
%%% Chapter of Common Lisp Manual.  Copyright 1984, 1988, 1989 Guy L. Steele Jr.

\input Title-page.tex

\pagenumbering{roman}

\def\pagestatus{ROUGH PAGES}

\begingroup
\makeatletter
\clearpage \global\c@page 5  % Start toc on page v

\def\numberline#1{\setbox0=\hbox{#1 }\ifdim \wd0 < \@tempdima
    \hbox to \@tempdima{\box0\hfil}\else \box0 \fi}
\makeatother
\tableofcontents
\endgroup

\cleardoublepage

\chapter*{\vtop{\LARGE \sf \noindent Preface Пролог \relax\\[5pt]
                \normalsize\sf SECOND EDITION}}


Common Lisp has succeeded.  Since publication of the first edition of
this book in 1984, many implementors have used it as a {\it de facto}
standard for Lisp implementation.  As a result, it is now much easier
to port large Lisp programs from one implementation to another.
Common Lisp has proved to be a useful and stable platform for
rapid prototyping and
systems delivery in artificial intelligence and other areas.
With experience gained in using Common Lisp for so many
applications, implementors found no shortage of opportunities for
innovation.
One of the important
characteristics of Lisp is its good support for experimental extension
of the language; while Common Lisp has been stable, it has not stagnated.

Common Lisp к успеху пришел. С момента публикации первой редакции
данной книги в 1984, много организаций использовали его как {\it
де-факто} стандарт для реализации Lisp'а. В результате сейчас гораздо
проще портировать большую Lisp программу с одной реализации на
другую. Common Lisp доказал свою полезность и стабильность, как
платформы для быстрого прототипирования и быстрой поставки систем в
области искусственного интеллекта и не только в ней. С приобретенным
опытом использования Common Lisp'а для такого большого количества
приложений, организации не нашли недостатков в возможностях для
инноваций. Одна из важных характеристик Lisp'а это его хорошая
поддержка для экспериментальных расширений языка; несмотря на то, что
Common Lisp стабилен, он не инертен.

\markboth{PREFACE (SECOND EDITION)}{PREFACE (SECOND EDITION)}

The 1984 definition of Common Lisp was imperfect and incomplete.
In some cases this was inadvertent: some odd boundary situation was
overlooked and its consequences not specified, or different
passages were in conflict, or some property of Lisp was so well-known
and traditionally relied upon that I forgot to write it down.
In other cases the informal committee that was defining
Common Lisp could not settle on a solution, and therefore agreed to leave
some important aspect of the language unspecified rather than
choose a less than satisfactory definition.  An example
is error handling; 1984 Common Lisp had plenty of ways to signal
errors but no way for a program to trap or process them.

Версия Common Lisp'а от 1984 года была несовершенной и
незавершенной. В некоторых случаях допускались неосторожности:
некоторые двузначные ситуации игнорировались и из следствия не
определялись, или разные вещи конфликтовали или некоторые свойства
Lisp'а были так хорошо известны, что на них традиционно
полагались, даже я забыл их записать. В других случаях
неофициальный коммитет, что создавал Common Lisp, не мог принять
решение и соглашался оставить некоторые важные вещи языка
неопределенными, чем выбирать менее удачный вариант. Например,
обработка ошибок; в Common Lisp 1984 года было изобилие способов
генерации сингалов об ошибках, но не было методов для их ловушек. 

Over the next year I collected reports of errors in the book and gaps in the
language.  In December 1985, a group of implementors and users met in
Boston to discuss the state of Common Lisp.  I prepared
two lists for this meeting, one of errata and clarifications
that I thought would be relatively uncontroversial (boy, was I wrong!)
and one of more substantial changes I thought should be considered and
perhaps voted upon.  Others also brought proposals to discuss.
It became clear to everyone that there was now enough interest in Common Lisp,
and dependence on its stability, that a more formal mechanism was needed for
managing changes to the language.

This realization led to the formation of X3J13, a subcommittee of
ANSI committee X3, to produce a formal American National Standard
for Common Lisp.  That process is nearing completion.  X3J13 has
completed the bulk of its technical work in rectifying the 1984
definition and codifying extensions to that definition that have
received widespread use and approval.  A draft standard is now
being prepared; it will probably be available
in 1990.  There will then be a period (required by ANSI) for
public review.  X3J13 must then consider the comments it receives
and respond appropriately.  If the comments result
in substantial changes to the draft standard, multiple public review
periods may be required before the draft can be approved as an American
National Standard.

Fortunately, X3J13 has done an outstanding job of documenting its work.
For every change that came to a formal vote, a document was prepared
that described the problem to be solved and one or more solutions.
For each solution there is a detailed proposal for changing the
language; a rationale; test cases that distinguish the proposal
from the status quo or from other proposals for solving that problem;
discussions of current practice, cost to implementors, cost to users,
cost of not adopting the proposal, benefits of adoption,
aesthetic criteria; and any relevant informal discussion that may have 
preceded creation of the formal proposal.  All of these proposal
documents were made available on-line as well as in paper form.
By my count, by June 1989 some
186 such proposals were approved as language changes.
(This count does not include many proposals that came before the committee
but were rejected.)

The purpose of this second edition is to bridge the gap between the
first edition and the forthcoming ANSI standard for Common Lisp.
Because of the requirement for formal public review,
it will be some time yet before the ANSI standard is final.
This book in no way resembles the forthcoming standard (which
is being written independently
by Kathy Chapman of Digital Equipment Corporation with assistance
from the X3J13 Drafting Subcommittee).

I have incorporated into this second edition
a great deal of material based on the votes of X3J13,
in order to give the reader a picture of where the language is heading.
My purpose here is not simply to quote the X3J13 documents verbatim
but to paraphrase them and relate them to the structure of the first
edition.  A single vote by X3J13 may be discussed in many parts of this book,
and a single passage of this book may be affected by many of the votes.

I wish to be very clear: this book is not an official document
of X3J13, though it is based on publicly available material
produced by X3J13.  In no way does this book constitute a definitive
description of the forthcoming ANSI standard.  The
committee's decisions have been remarkably stable (it has rescinded
earlier decisions only two or three times), and I do not
expect radical changes in direction.
Nevertheless, it is quite probable
that the draft standard will be substantively revised in response to
editorial review or public comment.
I have therefore reported here on the actions of X3J13 not to
inscribe them in \strut stone, but to make clear how the language
of the first edition is likely to change.
I have tried to be careful
in my wording to avoid saying ``the language has been changed''
and to state simply that
``X3J13 voted at such-and-so time to make the following change.''

Until the day when an official ANSI Common Lisp standard emerges,
it is likely that the 1984 definition of Common Lisp will
continue to be used widely.  This book has been designed
to be used as a reference both to the 1984 definition
and to the language as modified by the actions of X3J13.

It contains the entire text of the first edition
of {\it Common Lisp: The Language}, with corrections
and minor editorial changes;
however, more than half of the material in this edition is new.
All new material is
identified by solid lines
in the left margin.
Dotted lines in the left margin indicate material from the first edition
that applies to the 1984 definition but that has been modified
by a vote of X3J13.  Modifications to these outmoded
passages are explained by preceding or following text (which will
have a solid line in the margin).
In summary:
\begin{itemize}
\item To use the 1984 language definition, read all material that does not
have a solid line in the margin.
\item To use the updated language definition, read everything, but
be wary of material with a dotted line in the margin.
\end{itemize}

At the end of the book is an index of the X3J13 votes, ordered
by the committee's internal code names (included to ease cross-reference
to the X3J13 documents, which may be useful during the public review
periods).  References to this list of votes appear as numbers
in angle brackets; thus
``$\langle$14$\rangle$'' refers to the vote on issue number 14, whereas
``[14]'' refers to reference 14 in the bibliography.

I have kept changes to the wording of the first-edition material to a minimum.
Obvious spelling and typographical errors have been corrected,
and the entire text has been edited to a uniform style of
spelling and punctuation.  (Note in particular that the first edition
used the spelling ``signalling'' but this edition,
in deference to the style decision of the X3J13 Drafting
Subcommittee, uses ``signaling.'')  A few minor
changes were made to accommodate typographical or layout constraints.
(For example, the word ``also'' has been deleted from the first
sentence of chapter~1, partly to make that paragraph look better
and partly to allow a better page break at the bottom of page~2.)
In a very few cases the first edition contained substantive errors
that I could not in good conscience correct silently; these have
been flagged by paragraphs beginning with the phrase
{\it Notice of correction}.

The chapter and section numbering of this edition matches that
of the first edition, with the exception that a new
section~\ref{STRUCTURE-TRAVERSAL-SECTION}
has been interpolated.
Four new chapters (\ref{LOOP}--\ref{CONDITION})
describe substantial changes approved by X3J13: an extended
\cd{loop} macro, a pretty printer interface, the Common Lisp
Object System, and the Common Lisp Condition System.

X3J13, in the course of its work, formed a subcommittee to study
whether additional means of iteration
should be standardized for use in Common Lisp, for a great
deal of existing practice in this area was not included in the
first edition because of lack of agreement in~1984.
The X3J13 Iteration Subcommittee produced reports on three possible facilities.
One (\cd{loop}) was approved for inclusion in the forthcoming draft standard
and is described in chapter~\ref{LOOP}.

X3J13 expressed interest in the other two approaches (series and generators),
but the consensus as of January 1989
was that these other approaches were not yet sufficiently mature or
in sufficiently widespread use to warrant inclusion in the draft Common Lisp
standard at that time.  However, the subcommittee was directed to continue work
on these approaches and X3J13 is open to the possibility of standardizing
them at a later date.
Please note that I do not wish the prejudge the
question of whether X3J13 will ever choose to make the other two proposals the
subject of standardization.  Nevertheless,
I have chosen to include them in the second edition,
in cooperation with Dr.~Richard~C.~Waters,
as appendices~\ref{SERIES} and~\ref{GENERATORS},
in order to make these ideas
available to the Lisp community.  In my judgement these proposals
address an area of language design not otherwise covered by Common Lisp
and are likely to have practical value even if they are never
adopted as part of a formal standard.

Some new material in this book has nothing to do with the work of X3J13.
In many places I have added explanations, clarifications, new examples,
warnings, and tips on writing portable code.
Appendix~\ref{BACKQUOTE-SIMULATOR} contains a piece of code
that may help in understanding the backquote syntax.

This second edition,
unlike the first edition, also includes a few diagrams to pep up the text.
However, there are absolutely no new jokes, and very few outright lies.


\chapter*{\vtop{\LARGE \sf \noindent Acknowledgments Благодарности \relax\\[5pt]
                \normalsize\sf SECOND EDITION}}


First and foremost, I must thank the many people in the Lisp
community who have worked so hard to specify, implement, and use
Common Lisp.  Some of these have volunteered many hours
of effort as members of ANSI committee X3J13.  Others
have made presentations or proposals to X3J13, and yet others
have sent suggestions and corrections to the first edition directly to me.
This book builds on their efforts as well as mine.

\markboth{ACKNOWLEDGMENTS (SECOND EDITION)}{ACKNOWLEDGMENTS (SECOND EDITION)}

An early draft of this book was made available to all members
of X3J13 for their criticism.  I have also worked with
the many public documents that have been written during the course
of the committee's work (which is not over yet).
It is my hope that this book is an accurate reflection of the
committee's actions as of October 1989.
Nevertheless, any errors or inconsistencies are my responsibility.
The fact that I have made a draft available to certain persons,
received feedback from them, or thanked them in these
acknowledgments does not necessarily imply that any one of them
or any of the institutions with which they are affiliated endorse this book
or anything of its contents.

Digital Press and I gave permission to X3J13 to use any or all parts
of the first edition in the production of an ANSI Common Lisp standard.
Conversely, in writing this book I have worked with publicly available
documents produced by X3J13 in the course of its work, and in some cases
as a courtesy have obtained the consent of the authors of those documents
to quote them extensively.  This common ancestry will result in similarities
between this book and the emerging ANSI Common Lisp standard (that is the
purpose, after all).  Nevertheless, this second edition 
has no official connection whatsoever
with X3J13 or ANSI, nor is it endorsed by either of those institutions.

The following persons have been members of X3J13 or involved in its
activities at one time or another:
Jim Allard, Dave Andre, Jim Antonisse, William Arbaugh, John
Aspinall, Bob Balzer, Gerald Barber, Richard Barber, Kim Barrett,
David Bartley, Roger Bate, Alan Bawden, Michael Beckerle, Paul
Beiser, Eric Benson, Daniel Bobrow, Mary Boelk, Skona Brittain, Gary
Brown, Tom Bucken, Robert Buckley, Gary Byers, Dan Carnese, Bob
Cassels, J\'er\^ome Chailloux, Kathy Chapman, Thomas Christaller,
Will Clinger, Peter Coffee, John Cugini, Pavel Curtis, Doug Cutting,
Christopher Dabrowski, Jeff Dalton, Linda DeMichiel, Fred Discenzo,
Jerry Duggan, Patrick Dussud, Susan Ennis, Scott Fahlman, Jogn Fitch,
John Foderaro, Richard Gabriel, Steven Gadol, Nick Gall, Oscar
Garcia, Robert Gian\-sira\-cusa, Brad Goldstein, David Gray, Richard
Greenblatt, George Hadden, Steve Haflich, Dave Henderson, Carl
Hewitt, Carl Hoffman, Cheng Hu, Masayuki Ida, Takayasu Ito, Sonya
Keene, James Kempf, Gregory Jennings, Robert Kerns, Gregor Kiczales,
Kerry Kimbrough, Dieter Kolb, Timothy Koschmann, Ed Krall, Fritz
Kunze, Aaron Larson, Joachim Laubsch, Kevin Layer, Michael Levin, Ray
Lim, Thom Linden, David Loeffler, Sandra Loosemore, Barry Margolin,
Larry Masinter, David Matthews, Robert Mathis, John McCarthy, Chris
McConnell, Rob McLachlan, Jay Mendelsohn, Martin Mikelsons, Tracey
Miles, Richard Mlyarnik, David Moon, Jarl Nilsson, Leo Noordhulsen,
Ronald Ohlander, Julian Padget, Jeff Peck, Jan Pedersen, Bob
Pellegrino, Crispin Perdue, Dan Pierson, Kent Pitman, Dexter Pratt,
Christian Quiennec, B. Raghavan, Douglas Rand, Jonathan Rees, Chris
Richardson, Jeff Rininger, Walter van Roggen, Jeffrey Rosenking,  Don
Sakahara, William Scherlis, David Slater, James Smith, Alan Snyder,
Angela Sodan, Richard Soley, S. Sridhar, Bill St.\ Clair, Philip
Stanhope, Guy Steele, Herbert Stoyan, Hiroshi Torii, Dave Touretzky,
Paul Tucker, Rick Tucker, Thomas Turba, David Unietis, Mary Van
Deusen, Ellen Waldrum, Richard Waters, Allen Wechsler, Mark Wegman,
Jon~L White, Skef Wholey, Alexis Wieland, Martin Yonke, Bill York,
Taiichi Yuasa, Gail Zacharias, and Jan Zubkoff.





I must express particular gratitude and appreciation to a number
of people for their exceptional efforts:

Larry Masinter, chairman of
the X3J13 Cleanup Subcommittee, developed the standard format for
documenting all proposals to be voted upon.  The result has been
an outstanding tehcnical and historical record of all the actions
taken by X3J13 to rectify and improve Common Lisp.

Sandra Loosemore, chairwoman of the X3J13 Compiler Subcommittee,
produced many proposals for clarifying the semantics of the compilation
process.  She has been a diligent stickler for detail and has helped
to clarify many parts of Common Lisp left vague in the first edition.

Jon L White, chairman of the X3J13 Iteration Subcommittee,
supervised the consideration of several controversial
proposals, one of which (\cd{loop}) was eventually adopted by X3J13.

Thom Linden, chairman of the X3J13 Character Subcommittee,
led a team in grappling with the difficult problem of accommodating
various character sets in Common Lisp.  One result is that
Common Lisp will be more attractive for international use.

Kent Pitman, chairman of the X3J13 Error Handling Subcommittee,
plugged the biggest outstanding
hole in Common Lisp as described by the first edition.

Kathy Chapman, chairwoman of the X3J13 Drafting Subcommittee,
and principal author of the draft standard, has not only written
a great deal of text but also insisted on coherent and consistent
terminology and pushed the rest of the committee forward when necessary.

Robert Mathis, chairman of X3J13, has kept administrative matters
flowing smoothly during technical controversies.

Mary Van Deusen, secretary of X3J13, kept excellent minutes
that were a tremendous aid to me in tracing the history of
a number of complex discussions.

Jan Zubkoff, X3J13 meeting and mailing
organizer, knows what's going on, as always.
She is a master of organization and of physical arrangements.
Moreover, she once again pulled me out of the fire at the last minute.

Dick Gabriel, international representative for X3J13,
has kept information flowing smoothly between Europe, Japan,
and the United States.  He provided a great deal of the energy and drive
for the completion of the Common Lisp Object System specification.
He has also provided me with a great
deal of valuable advice and has been on call for last-minute
consultation at all hours during the final stages of preparation
for this book.

David Moon has consistently been a source of reason,
expert knowledge, and careful scrutiny.  He has read the
first edition and the X3J13 proposals perhaps more carefully
than anyone else.

David Moon, Jon~L White, Gregor Kiczales, Robert Mathis, Mary Boelk
provided extensive feedback on an early draft of this book.
I thank them as well as the many others who commented in one way
or another on the draft.

I wish to thank the authors of large proposals to X3J13
that have made material available for more or less wholesale
inclusion in this book as distinct chapters.
This material was produced primarily for the use of X3J13 in its work.
It has been included here
on a non-exclusive basis with the consent of the authors.

The author of the chapter on \cd{loop} (Jon~L White)
notes that the chapter is based on documentation
written at Lucid, Inc., by Molly~M. Miller,
Sonia Orin Lyris, and Kris Dinkel.
Glenn Burke, Scott Fahlman, Colin Meldrum,
David Moon, Cris Perdue, and Dick Waters
contributed to the design of the \cd{loop} macro.

The authors of the Common Lisp Object System specification
(Daniel G.~Bobrow, Linda G.~DeMichiel,
Richard P.~Gabriel, Sonya E.~Keene, Gregor Kiczales,
and David A.~Moon)
wish to thank Patrick Dussud, Kenneth Kahn,
Jim Kempf, Larry Masinter, Mark Stefik,
Daniel~L. Weinreb, and Jon~L White
for their contributions.

The author of the chapter on Conditions (Kent M. Pitman)
notes that there is a paper \cite{EXCEPTIONAL-SITUATIONS}
containing background information about the design of the
condition system, which is based on the condition system
of the Symbolics Lisp Machines \cite{SIGNALLING-CONDITIONS}.
The members of the X3J13 Error Handling Subcommittee
were
Andy Daniels and Kent Pitman.
Richard Mlynarik and David~A. Moon made major design contributions.
Useful comments, questions,
suggestions, and criticisms were provided by
    Paul Anagnostopoulos,
    Alan Bawden,
    William Chiles,
    Pavel Curtis,
    Mary Fontana,
    Dick Gabriel,
    Dick King,
    Susan Lander,
   David D. Loeffler,
 Ken Olum,
 David~C. Plummer,
 Alan Snyder,
   Eric Weaver, and
Daniel~L. Weinreb.
The Condition System was designed specifically to
accommodate the needs of Common Lisp.
The design is, however, most directly based on the ``New Error System''
(NES) developed at Symbolics by    David L. Andre,
    Bernard~S. Greenberg,
    Mike McMahon,
    David~A. Moon, and
    Daniel~L. Weinreb.
The NES was in turn based on experiences with the original Lisp
Machine error system (developed at MIT), which was found to be
inadequate for the needs of the modern Lisp Machine environments.
Many aspects of the NES were inspired by the (PL/I) condition
system used by the Honeywell Multics operating system. Henry Lieberman
provided
conceptual guidance and encouragement in the design of the NES.
A reimplementation of the NES for non-Symbolics Lisp Machine 
dialects (MIT, LMI, and TI) was done at MIT by Richard~M. Stallman.
During the process
of that reimplementation, some conceptual changes were made which
have significantly influenced the Common Lisp Condition System.

As for the smaller but no less important proposals,
Larry Masinter deserves recognition as an author of over half of them.
He has worked indefatigably to write up proposals and to polish
drafts by other authors.  Kent Pitman, David Moon, and Sandra Loosemore
have also been notably prolific,
as well as Jon~L White, Dan Pierson, Walter van Roggen,
Skona Brittain, Scott Fahlman, and myself.
Other authors of proposals include
David Andre,
John Aspinall,
Kim Barrett,
Eric Benson,
Daniel Bobrow,
Bob Cassels,
Kathy Chapman,
WIlliam Clinger,
Pavel Curtis,
Doug Cutting,
Jeff Dalton,
Linda DiMichiel,
Richard Gabriel,
Steven Haflich,
Sonya Keene,
James Kempf,
Gregor Kiczales,
Dieter Kolb,
Barry Margolin,
Chris McConnell,
Jeff Peck,
Jan Pedersen,
Crispin Perdue,
Jonathan Rees,
Don Sakahara,
David Touretzky,
Richard Waters, and
Gail Zacharias.

I am grateful to Donald~E. Knuth and his colleagues for producing
the \TeX\ text formatting system \cite{KNUTH-TEXBOOK},
which was used to produce
and typeset the manuscript.
Knuth did an especially good job of publishing the program for
\TeX~\cite{KNUTH-TEX-PROGRAM};
I had to consult the code about eight times while debugging particularly
complicated macros.  Thanks to the extensive indexing
and cross-references, in each case it took me less than five minutes to
find the relevant fragment of that 500-page program.

I also owe a debt
to Leslie Lamport, author of the \LaTeX\ macro package~\cite{LAMPORT-LATEX}
for \TeX,
within which I implemented the document style for this book.

Blue Sky Research sells and supports Textures, an implementation
of \TeX\ for Apple Macintosh computers; Gayla Groom and Barry Smith
of Blue Sky Research provided excellent technical support when I
needed it.  Other software tools that were invaluable
in preparing this book were QuicKeys (sold by CE Software, Inc.),
which provides keyboard macros;
G\=ofer (sold by Microlytics, Inc.), which performs rapid
text searches in multiple files; Symantec Utilities for Macintosh
(sold by Symantec Corporation), which saved me from more than one disk crash;
and the PostScript language and compatible
fonts (sold by Adobe Systems Incorporated).

Some of this software (such as \LaTeX) I obtained for free and some I bought,
but all have proved to be useful tools of excellent quality.
I am grateful to these developers for creating them.

Electronic mail has been indispensible to
the writing of this book, as well to as the work of X3J13.
(It is a humbling experience to publish a book and then for
the next five years to receive
at least one electronic mail message a week, if not twenty, pointing out
some mistake or defect.)
Kudos to those develop and maintain the Internet, which arose
from the Arpanet and other networks.

Chase Duffy, George Horesta, and Will Buddenhagen of Digital Press have given me
much encouragement and support.  David Ford designed the book and
provided specifications that I could code into \TeX.
Alice Cheyer and Kate Schmit edited the copy for style
and puzzled over the more obscure jokes with great patience.
Marilyn Rowland created the index; Tim Evans and I did some polishing.
Laura Fillmore and her colleagues at Editorial, Inc., have
tirelessly and meticulously checked one draft after another and
has kept the paperwork flowing smoothly during the last hectic weeks
of proofreading, page makeup, and typesetting.

Thinking Machines Corporation has supported all my work with X3J13.
I thank all my colleagues there for their encouragement and help.

Others who provided indispensible encouragement and support include
Guy and Nalora Steele; David Steele; Cordon and Ruth Kerns;
David, Patricia, Tavis, Jacob, Nicholas, and Daniel Auwerda;
Donald and Denise Kerns; and David, Joyce, and Christine Kerns.

Most of the writing of this book took place between
10 P.M.~and 3 A.M.~(I'm not as young as I used to be).
I am grateful to Barbara,
Julia, Peter, and Matthew for putting up with it, and for their love.

\begin{tabbing}
Guy L. Steele Jr. \\
Lexington, Massachusetts \\
All Saints' Day, 1989
\end{tabbing}


\chapter*{\vtop{\LARGE \sf \noindent Acknowledgments Благодарности \relax\\[5pt]
                \normalsize\sf FIRST EDITION (1984)}}

Common Lisp was designed
by a diverse group of people affiliated with many institutions.

\markboth{ACKNOWLEDGMENTS (FIRST EDITION, 1984)}{ACKNOWLEDGMENTS (FIRST EDITION, 1984)}

Contributors to the
design and implementation of Common Lisp and to the polishing of this book
are hereby gratefully acknowledged:
\vskip 0pt plus 10pt
\hrule width 0pt\relax

\begin{tabbing}
\hskip8.5pc\=\kill
Paul Anagnostopoulos\>Digital Equipment Corporation \\
Dan Aronson\>Carnegie-Mellon University \\
Alan Bawden\>Massachusetts Institute of Technology \\
Eric Benson\>University of Utah, Stanford University, and Symbolics,\\
           \>Incorporated \\
Jon Bentley\>Carnegie-Mellon University and Bell
Laboratories \\ Jerry Boetje\>Digital Equipment Corporation \\
Gary Brooks\>Texas Instruments \\
Rodney A. Brooks\>Stanford University \\
Gary L. Brown\>Digital Equipment Corporation \\
Richard L. Bryan\>Symbolics, Incorporated \\
Glenn S. Burke\>Massachusetts Institute of Technology \\
Howard I. Cannon\>Symbolics, Incorporated \\
George J. Carrette\>Massachusetts Institute of Technology \\
Robert Cassels\>Symbolics, Incorporated \\
Monica Cellio\>Carnegie-Mellon University \\
David Dill\>Carnegie-Mellon University \\
Scott E. Fahlman\>Carnegie-Mellon University \\
Richard J. Fateman\>University of California, Berkeley \\
Neal Feinberg\>Carnegie-Mellon University \\
Ron Fischer\>Rutgers University \\
John Foderaro\>University of California, Berkeley \\
Steve Ford\>Texas Instruments
\end{tabbing}

\penalty-10000

\begin{tabbing}
\hskip8.5pc\=\kill
Richard P. Gabriel\>Stanford University and Lawrence Livermore National \\
                  \>Laboratory \\
Joseph Ginder\>Carnegie-Mellon University and Perq Systems Corp. \\
Bernard S. Greenberg\>Symbolics, Incorporated \\
Richard Greenblatt\>Lisp Machines Incorporated (LMI) \\
Martin L. Griss\>University of Utah and Hewlett-Packard Incorporated \\
Steven Handerson\>Carnegie-Mellon University \\
Charles L. Hedrick\>Rutgers University \\
Gail Kaiser\>Carnegie-Mellon University \\
Earl A. Killian\>Lawrence Livermore National Laboratory \\
Steve Krueger\>Texas Instruments \\
John L. Kulp\>Symbolics, Incorporated \\
Jim Large\>Carnegie-Mellon University \\
Rob Maclachlan\>Carnegie-Mellon University \\
William Maddox\>Carnegie-Mellon University \\
Larry M. Masinter\>Xerox Corporation, Palo Alto Research Center \\
John McCarthy\>Stanford University \\
Michael E. McMahon\>Symbolics, Incorporated \\
Brian Milnes\>Carnegie-Mellon University \\
David A. Moon\>Symbolics, Incorporated \\
Beryl Morrison\>Digital Equipment Corporation \\
Don Morrison\>University of Utah \\
Dan Pierson\>Digital Equipment Corporation \\
Kent M. Pitman\>Massachusetts Institute of Technology \\
Jonathan Rees\>Yale University \\
Walter van Roggen\>Digital Equipment Corporation \\
Susan Rosenbaum\>Texas Instruments \\
William L. Scherlis\>Carnegie-Mellon University \\
Lee Schumacher\>Carnegie-Mellon University \\
Richard M. Stallman\>Massachusetts Institute of Technology \\
Barbara K. Steele\>Carnegie-Mellon University \\
Guy L. Steele Jr.\>Carnegie-Mellon University and Tartan Laboratories \\
                 \>Incorporated \\
Peter Szolovits\>Massachusetts Institute of Technology \\
William vanMelle\>Xerox Corporation, Palo Alto Research Center \\ Ellen
Waldrum\>Texas Instruments \\ Allan C. Wechsler\>Symbolics, Incorporated \\
Daniel L. Weinreb\>Symbolics, Incorporated \\
Jon L White\>Xerox Corporation, Palo Alto Research Center \\
Skef Wholey\>Carnegie-Mellon University
\end{tabbing}
\begin{tabbing}
\hskip8.5pc\=\kill
Richard Zippel\>Massachusetts Institute of Technology \\
Leonard Zubkoff\>Carnegie-Mellon University and Tartan Laboratories \\
               \>Incorporated
\end{tabbing}
Some contributions were relatively small; others involved enormous
expenditures of effort and great dedication.  A few of the contributors
served more as worthy adversaries than as benefactors (and do not
necessarily endorse the final design reported here),
but their pointed criticisms were just as important to the polishing of Common Lisp
as all the positively phrased suggestions.
All of the people named above were helpful in one way or another,
and I am grateful for the interest and spirit of cooperation
that allowed most decisions to be made by consensus after due discussion.

Considerable encouragement and moral support were also provided by:
\begin{tabbing}
\hskip1.5in\=\kill
Norma Abel\>Digital Equipment Corporation \\
Roger Bate\>Texas Instruments \\
Harvey Cragon\>Texas Instruments \\
Dennis Duncan\>Digital Equipment Corporation \\
Sam Fuller\>Digital Equipment Corporation \\
A. Nico Habermann\>Carnegie-Mellon University \\
Berthold K. P. Horn\>Massachusetts Institute of Technology \\
Gene Kromer\>Texas Instruments \\
Gene Matthews\>Texas Instruments \\
Allan Newell\>Carnegie-Mellon University \\
Dana Scott\>Carnegie-Mellon University \\
Harry Tennant\>Texas Instruments \\
Patrick H. Winston\>Massachusetts Institute of Technology \\
Lowell Wood\>Lawrence Livermore National Laboratory \\
William A. Wulf\>Carnegie-Mellon University and Tartan Laboratories \\
               \>Incorporated
\end{tabbing}
I am very grateful to each of them.

Jan Zubkoff of Carnegie-Mellon University
provided a great deal of organization,
secretarial support, and unfailing good cheer in the face of adversity.

The development of Common Lisp would most probably not have been possible
without the electronic message system provided by the ARPANET.
Design decisions were made on several hundred distinct points, for the
most part by consensus, and by simple majority vote when necessary.
Except for two one-day face-to-face meetings, all of the language design
and discussion was done through the {ARPANET} message system, which
permitted effortless dissemination of messages to dozens of people, and
several interchanges per day.  The message system also provided
automatic archiving of the entire discussion, which has proved
invaluable in the preparation of this reference manual.  Over the course
of thirty months, approximately 3000 messages were sent (an average of
three per day), ranging in length from one line to twenty pages.
Assuming 5000 characters per printed page of text, the entire
discussion totaled about 1100 pages.  It would have been substantially
more difficult to have conducted this discussion by any other means,
and would have required much more time.

The ideas in Common Lisp have come from many sources and been polished by
much discussion.  I am responsible for the form of this
book, and for any errors or inconsistencies that may remain;
but the credit for the design and support of Common Lisp lies with
the individuals named above, each of whom has made significant
contributions.

The organization and content
of this book were inspired in large part by the
{\it MacLISP Reference Manual} by David A. Moon and others \cite{MOONUAL},
and by the {\it LISP Machine Manual} (fourth edition)
by Daniel Weinreb and David Moon \cite{BLUE-LISPM},
which in turn acknowledges the efforts of Richard Stallman, Mike McMahon,
Alan Bawden, Glenn Burke, and ``many people too numerous to list.''

I thank Phyllis Keenan, Chase Duffy,
Virginia Anderson,
John Osborn,
and Jonathan Baker of Digital Press for their
help in preparing this book for publication.
Jane Blake did an admirable job of copy-editing.
James Gibson and Katherine Downs of Waldman Graphics were most cooperative
in typesetting this book from my on-line manuscript files.

I am grateful to Carnegie-Mellon University and to
Tartan Laboratories Incorporated for supporting me in the writing
of this book over the last three years.

Part of the work on this book was
done in conjunction with the Carnegie-Mellon University Spice Project,
an effort to construct an advanced scientific software development
environment for personal computers.
The Spice Project is
supported by the Defense Advanced Research Projects Agency, Department   
of Defense, ARPA Order 3597, monitored by the Air Force Avionics   
Laboratory under contract F33615-78-C-1551.  The views   
and conclusions contained in this book are those of the author
and should not be interpreted as representing the official policies,   
either expressed or implied, of the Defense Advanced Research   
Projects Agency or the U.S. Government.

Most of the writing of this book took place between
midnight and 5 A.M.  I am grateful to Barbara, Julia, and Peter
for putting up with it, and for their love.

\begin{tabbing}
Guy L. Steele Jr. \\
Pittsburgh, Pennsylvania \\
March 1984
\end{tabbing}


\newpage

\null
\vskip 1in

\thispagestyle{empty}

\begingroup
\raggedright \small
\list{}{\rightmargin=8pc \leftmargin=8pc}\item[]
Would it be wonderful if, under the
pressure of all these difficulties, the
Convention should have been forced
into some deviations from that artifi-
cial structure and regular symmetry 
which an abstract view of the subject 
might lead an ingenious theorist to 
bestow on a constitution planned in 
his~closet or in his imagination?
\par\vskip 4pt
\begin{tabbing}
---\={\it James Madison, The Federalist} \\
\>{\it No. 37, January 11, 1788}
\end{tabbing}
\endlist
\endgroup
