%% Document style TMreport: report for Thinking Machines Corporation.
%% Produced from /usr/local/lib/tex/macros/report.sty by Guy Steele.
% 3/18/85 10:28:39  Created.
% 3/18/85 10:29:39  Uses TMrep10.sty instead of rep10.sty, etc.
% 3/18/85 11:12:44  New \maketitle puts in copyright notice and other text.
% 3/19/85 12:38:40  Added \formattedindex and \indexentry.
% 3/21/85 11:39:27  Make index be in \small font.  Adjust spacing.
% 5/22/85 12:38:42  Need to make @ be alphabetic (catcode 11) when
%		    reading in the index.  Things like \z@ show up.
% 1/31/86 15:24:51  Add hackery to put blank lines in index where
%		    initial letter changes.
% 2/03/86 16:01:52  Use \indexspace, not \bigskip.
% 2/04/86 14:10:52  Added \registered and \trademark.
% 3/07/86 14:21:25  Make \theindex use \raggedright.
% 3/11/86 16:32:19  Add new document option "onecolumnindex".
% 6/28/88 23:40:11  Add stuff to reset the \jobname for TeXtures.  Yuk.
% 8/27/88 17:17:17  Add flushdesc environment.

\def\PostScriptFile#1{\special{psfile=#1}}

\newif\ifchiron \chironfalse

\makeatletter

%\newif\if@onecolumnindex \@onecolumnindexfalse
%\newif\if@externaldocument \@externaldocumentfalse
\newif\if@draft \@draftfalse
\newif\if@permutemonospace   \@permutemonospacefalse
%\typeout{Document Style 'TMreport'. Version 0.91a - released 18 March 1985}
\def\@ptsize{0} \@namedef{ds@11pt}{\def\@ptsize{1}}
\@namedef{ds@12pt}{\def\@ptsize{2}} 
\def\ds@twoside{\@twosidetrue \@mparswitchtrue}
\def\ds@draft{\overfullrule 18pt\@drafttrue}
%\def\ds@external{\@externaldocumenttrue}
%\def\ds@onecolumnindex{\@onecolumnindextrue}
\def\ds@permutemonospace{\@permutemonospacetrue}
\@options
%\input CLTL1\@ptsize.sty\relax
\newskip\ruletonoteskip \ruletonoteskip=6pt plus 3pt minus 1pt
\newskip\ruletotextskip \ruletotextskip=10pt plus 12pt minus 2pt
\newdimen\thinrule   \thinrule=0.4pt

%\clubpenalty=1000
%\widowpenalty=1000

%%% defun.tex

% defun library.
% Guy L. Steele Jr.
% Copyright 1985, 1986 , 1988, 1989 Guy L. Steele Jr.

%  3/18/85 12:48  Now \@defunname generates index entries.
%  5/22/85 14:16  Use \noexpand before \tt and \rm in index entries.
%		  This will delay their processing until the point
%		  when the index is read back in again.
%  5/23/85 17:01  Add special check for nil as an init value for an
%		  &optional or &key argument; make it be \tt, not \it.
% 12/17/85 12:56  Add "lisp" environment.
%  1/02/86 14:15  Base "lisp" environment on tabbing instead of raggedright.
%  1/02/86 17:00  Make \endlisp use \ignorespaces.
%  1/03/86 17:00  Put a space into the index key to make sorting work better.
%  1/31/86 12:32  Simplify "lisp" environment to let the tabbing environment
%		  take care of interparagraph spacing.
%  1/31/86 13:27  Add \smallcaps stuff.
%  1/31/86 13:38  In \@defunkeymode et al., assume the user provides a :.
%  1/31/86 16:03  Flush \smallcaps stuff.
%  1/31/86 17:03  Use \addvspace in \defun et al.
%  2/04/86 17:15  Add feature so that the syntax \begin{defun}[Frob] causes "[Frob]"
%		  to appear at the right margin, as in the Common Lisp book, and causes
%		  the index entry for function baz to be "baz frob".  If no square
%		  brackets ae used, then nothing appears at the right margin, and the
%		  index entry looks like "baz \lowercase{default}" where "default" is
%		  the definition of \defaultdefuntype, which is initialized by this file
%		  to {Function} but can be redefined by the user.
%  2/07/86 17:08  Make lisp environment use \frenchspacing.
%  3/11/86 11:42  Improve some of the breaking places in the defun headers:
%		  Introduce \hbox around every name, and make better pseudo-hyphens.
%		  Also fix a bug in \@defuninitvalue.
%  3/12/86 11:47  Hair up the defunbreak stuff some more.  Now we have different
%		  behavior depending on whether the function name is long or short.
%  3/12/86 16:09  Fix a possible bug in \@showdefuntype.
%  3/13/86 14:16  More fixes to hairy defunbreak stuff.
%  5/29/86 01:47  Add \internalroutine.
%  6/04/86 14:47  Correct spacing in \@showdefuntype.
%  6/29/88 00:31  Remove priority-box macros.

\catcode`@=11	% Make it possible to refer to some LaTeX utility macros.
\catcode`&=11

\newcounter{lispcounter}
\newcounter{defuncounter}

\newenvironment{lisp}{\stepcounter{lispcounter}\begin{tabbing}}{\end{tabbing}}

\def\@defunstart{\noindent\leavevmode     % Need this to trigger the \everypar for margin rules.
  \begingroup
  \catcode`&=11}

% The addvspace used to be "plus 4pt" but that was not enough stretch relative to \parskip.
\def\defun{\@ifstar{\@defunstart{}\@ifnextchar[{\@defuntyped}{\@defununtyped}}{\relax\@defunstart    %Margin macros know this amount
   \@ifnextchar[{\@defuntyped}{\@defununtyped}}}


\def\@showdefuntype{\emph{[\@defuntype\/]}\relax
  \let\@defuntype\@defunsecondtype}

\def\@defununtyped{\let\@defunshow=\relax \@defun}
\def\@defuntyped[#1]{\def\@defuntype{#1}\let\@defunsecondtype\@defuntype
                     \let\@defunshow=\@showdefuntype
                     \@ifnextchar[{\@defuntypedtwice}{\@defun}}
\def\@defuntypedtwice[#1]{\def\@defunsecondtype{#1}\let\@defunshow=\@showdefuntype
                          \@defun}

\def\@defun{\@ifnextchar?{\@defunname}{\@defunname}}	% Skips spaces before name 

\def\enddefun{\par
  %%\message{end{defun} for \currentdefun}%For debugging
  \@ifstar{}{}}    %Margin macros know this amount
% Used to be "plus 2pt" but that was not enough stretch relative to \parskip.

% 1: entry
% \@defuntype: \lowercase{category}
\def\@defunname#1 {\@defunshow
%  \typeout{defunname}\typeout{ 1: #1}\typeout{ type: \lowercase{\@defuntype}}
  \@defunindex#1:\par{\@defuntype}\relax
  %%\message{begin{defun} for #1}\global\def\defun@name{#1}%For debugging
  \textbf{ #1}
  \ifdim 1\wd0 < 1.3in\relax
     \def\@defunbreak{ }\relax
     \let\@firstdefunbreak=~\relax
  \else
     \def\@defunbreak{ }\relax
     \let\@firstdefunbreak=\@defunbreak
  \fi
  \def\@defunkeywordbreak{\@firstdefunbreak\let\@defunkeywordbreak=\@defunbreak}\relax
  \def\@hairydefunbreak{\@firstdefunbreak
		        \let\@defunkeywordbreak=\@defunbreak
		        \let\@hairydefunbreak\@defunbreak}\relax
  \@defunreqmode}

% 1: entry
% 2: empty
% 3: \lowercase{category}
\def\@defunindex#1:#2\par#3{%
  \typeout{defunindex}\typeout{ 1: #1}\typeout{ 2: #2}\typeout{ 3: #3}%
  \def\@tempa{}\def\thing{#2}\ifx\thing\@tempa
  \@idefunindex{}#1:\par{#3}\else \@idefunindex{#1:}#2\par{#3}\fi}
% 1: empty
% 2: entry
% 3: \lowercase{category}
\def\@idefunindex#1#2:\par#3{%
%  \typeout{idefunindex}\typeout{ 1: #1}\typeout{ 2: #2}\typeout{ 3: #3}%
  \index{symbols}{#2!#3}\refstepcounter{defuncounter}\label{#2}}


\def\@newlinecheck#1#2{\def\@tempa{#1}\def\@tempb{\\}\ifx\@tempa\@tempb \\*\let\@tempd\@defunname
  \else \def\@tempd{#2}\fi \@tempd}

\def\@ifparnext#1#2{\def\@tempa{#1}\def\@tempb{#2}\futurelet\@tempc\@ifparnx}
\def\@ifparnx{\ifx\@tempc\@@par \let\@tempd\@tempa \else \let\@tempd\@tempb \fi \@tempd}


\def\@defunreqmode{\@ifparnext{\@defundone}{\@ifnextchar &{\@defunkeyword}{\@ifnextchar({\@defunparenreqarg}{\@defunreqarg}}}}
\def\@defunreqarg#1 {\@newlinecheck{#1}{\@hairydefunbreak\textit{#1\/}\@defunreqmode}}
\def\@defunparenreqarg(#1 #2) {\@hairydefunbreak(\textit{#1\/}~#2)\@defunreqmode}  %CLOS methods

\def\@defunrestmode{\@ifparnext{\@defundone}{\@ifnextchar &{\@defunkeyword}{\@defunrestarg}}}
\def\@defunrestarg#1 {\@newlinecheck{#1}{\@hairydefunbreak\textit{#1\/}\@defunrestmode}}

\def\@defunoptionalmode{\@ifparnext{\@defundone}{\@ifnextchar &{\@defunkeyword}{\@ifnextchar({\@defunparenoptionalarg}{\@defunoptionalarg}}}}
\def\@defunoptionalarg#1 {\@newlinecheck{#1}{\@hairydefunbreak\textit{#1\/}\@defunoptionalmode}}
\def\@defunparenoptionalarg(#1 #2) {\@hairydefunbreak(\textit{#1\/}~\@defuninitvalue#2)\@defunoptionalmode}

\def\@defuninitvalue#1#2){{\ifcat#1A\def\@tempa{#1#2}\def\@tempb{nil}\ifx\@tempa\@tempb
  \cf \else \it \fi \else \cf \fi #1#2})}

\def\@defunkeymode{\@ifparnext{\@defundone}{\@ifnextchar &{\@defunkeyword}{\@ifnextchar({\@defunparenkeyarg}{\@defunkeyarg}}}}
\def\@defunkeyarg#1 {\@newlinecheck{#1}{\@defunkeywordbreak\textit{#1\/}\@defunkeymode}}
\def\@defunparenkeyarg(#1 #2) {~(#1~\@defuninitvalue#2)\@defunkeymode}

\def\@defunkeyword &#1 {\@defunkeywordbreak\&#1 \csname @defun#1mode\endcsname}

\def\@defundone\par{\par\endgroup
  \nobreak\noindent}

%% Macros, special forms
\begingroup
%\catcode`\[=13 \catcode`\]=13
%\catcode`$=13 \catcode`$=13
\catcode`\<=13 \catcode`\>=13 %\catcode`\|=13
\catcode`\?=13
\global\def\@defmacstart#1{\relax
%  \noindent\leavevmode     % Need this to trigger the \everypar for margin rules.
  \begingroup
   \catcode`&=11
   \def\@lbracehack{{\,\{}\@backslashsetup}\relax
   \def\@rbracehack{\@ifnextchar*{\@rbracesuper}{\@ifnextchar+{\@rbracesuper}{\/\}\,\@backslashsetup}}}\relax
  % %\def[{{$\,\lbrack$}\=\pushtabs\+\@backslashsetup}\relax
  % %\def]{\-\poptabs$\/\rbrack\,$\@backslashsetup}\relax
  \def<{\,[[\@backslashsetup}\relax
  \def>{\/]]\,\@backslashsetup}\relax
  % \def({{\cf \char40}\=\pushtabs\+\@backslashsetup}\relax
  % \def){\-\poptabs{\cf \char41}\@backslashsetup}\relax
  %\def|{{$\char124$}}\relax
   \def?{{$\downarrow$}}\relax
   \def\@finishdefmac{%\-
     \end{tabbing}\endgroup%
     \everypar{}\paragraph{}\noindent}\relax
   \expandafter\def\@carret{\expandafter\@ifnextchar\@carret%
     {\@finishdefmac\@gobblecr}{\@tabcr*\@backslashsetup}}\relax
  \@defunhackbraces
  \def\@backslashsetup{\def\\{\@tabcr*\@backslashsetup\@dodefmacname{#1}}}\relax
  % \catcode`\[=13 \catcode`\]=13 \catcode`$=13 \catcode`$=13
  \catcode`\<=13 \catcode`\>=13 %\catcode`\|=13 
  \catcode`\?=13 \catcode`\^^M=13
  %\def\!##1!{\cd{##1}}\relax
  \begin{tabbing}
  \@backslashsetup\@margineveryparguts
  \@dodefmacname{#1}}
\endgroup
\def\@rbracesuper#1{{\/\}#1\,}\@backslashsetup}


\def\Mopt#1{\,[\textit{#1\/}]\,}
\def\Mchoice#1{\,[[\textit{#1\/}]]\,}
\def\Mstar#1{\,\{\textit{#1\/}\}*\,}
\def\Mplus#1{\,\{\textit{#1\/}\}+\,}
\def\Mgroup#1{\,\{\textit{#1\/}\}\,}
\def\Mor{|}
\def\Mind#1{$\downarrow$\textit{#1\/}}

\ifx \rulang\Undef
\def\defmac{\@defmacbegin{Macro}}
\else %RUSSIAN
\def\defmac{\@defmacbegin{Макрос}}
\fi
\let\enddefmac\enddefun

\ifx \rulang\Undef%
\def\defspec{\@defmacbegin{Special operator}} %
\else %
\def\defspec{\@defmacbegin{Специальный оператор}}% 
\fi
\let\enddefspec\enddefun

\ifx \rulang\Undef%
\def\defloop{\@defmacbegin{Loop clause}} %
\else %
\def\defloop{\@defmacbegin{Выражение цикла}}% 
\fi
\let\enddefloop\enddefun

\def\@defmacbegin#1{\@ifstar{\@defmacstart{#1}}{\@defmacstart{#1}}}

{\catcode`\^^M=13\global\def\@carret{
}\global\def\@defmacnamecrgobble#1
{\@defmacname{#1}}}

\def\@dodefmacname#1{
  \expandafter\@ifnextchar\@carret
  {\@defmacnamecrgobble{#1}}{\@defmacname{#1}}}

\def\@defmacname#1#2 {
  \emph{[#1\/]}\relax
  \leavevmode
  \textbf{ #2 }%\=\pushtabs\+
  \@backslashsetup
  \@defunindex#2:\par{#1}
  }

\bgroup
\catcode`\<=1 \catcode`\>=2 \catcode`\{=13 \catcode`\}=13
\global\def\@defunhackbraces<\catcode`\{=13\catcode`\}=13\let{\@lbracehack\let}\@rbracehack>
\egroup

\catcode`@=12		% Restore character codes
\catcode`&=4


%[End]

%%%

%%% marginal.tex

\makeatletter
\let\@margineverypar\relax
\let\@margineveryparguts\relax
\makeatother

\newenvironment{new}{%
}{%
}
\newenvironment{newer}{%
}{%
}
\newenvironment{obsolete}{%
}{%
}

%%% 

\makeatletter

% The \null in the following is intended to suppress hyphenation
% in code words not already containing hyphens.  The \leavevmode
% is needed to prevent vertical mode from swallowing the \null.
%%%\def\cd#1{\leavevmode{\cf \null#1}}
%\def\cd{\leavevmode\begingroup\cf\@cd} 
%\def\cd{\leavevmode\begingroup\@cd}
%\def\@cd#1{\texttt{#1}\endgroup}
%\def\@cd#1{\textit{\null #1}}
\newcommand{\cd}[1]{\texttt{#1}}

% if label exists, then automatically make hyperref
% else insert plain text
\newcommand{\cdf}[1]{\texttt{\testlabel{#1}}}

\newcommand{\testlabel}[1]{%
  \@ifundefined{r@#1}%
     {#1}%
     {\hyperref[#1]{#1}}}%


\def\Xtilde{\char"7E}
\def\Xbq{\char"60}
\def\Xquote{\char"27}
\def\Xequal{\char"3D}
\def\Xcircumflex{\char"5E}
\def\cf{} %\frenchspacing}

\ifchiron
  \def\Xarrowright{\char"A2}
  \def\Xarrowdown{\char"A4}
\else
   \def\Xarrowright{$ \Rightarrow $}
   \def\Xarrowdown{$ \Downarrow $}
\fi
\def\Xatsign{\char"40}
\def\Xbackslash{\char"5C}
\def\Xunderscore{\char"5F}
\def\Xlbracket{\char"5B}
\def\Xrbracket{\char"5D}
\def\Xlbrace{\char"7B}
\def\Xrbrace{\char"7D}
\def\dlbrack{\lbrack\lbrack}
\def\drbrack{\rbrack\rbrack}
\def\Xdquote{\char"22}

\def\EV{$\Rightarrow$}
\def\EX{$\rightarrow$}
\def\EQ{$\equiv$}

\newcommand{\indexterm}[1]{\index{symbols}{#1}}

\def\emptylist{()}
\def\false{\cdf{nil}}
\def\true{\cdf{t}}
\def\nil{\cdf{nil}}

\ifx \rulang\Undef
\newenvironment{implementation}{\par\small\noindent\textbf{Implementation note:}}{\par}
\else %RUSSIAN
\newenvironment{implementation}{\par\small\noindent\textbf{Заметка для реализации:}}{\par}
\fi 

\ifx \rulang\Undef
\newenvironment{rationale}{\par\small\noindent\textbf{Rationale:}}{\par}
\else %RUSSIAN
\newenvironment{rationale}{\par\small\noindent\textbf{Обоснование:}}{\par}
\fi

\ifx \rulang\Undef
\newenvironment{sideremark}{\par\small\noindent\textbf{Remark:}}{\par}
\else %RUSSIAN
\newenvironment{sideremark}{\par\small\noindent\textbf{Примечание:}}{\par}
\fi

\ifx \rulang\Undex
\newenvironment{incompatibility}{\par\small\noindent\textbf{Compatibility note:}}{\par}
\else %RUSSIAN
\newenvironment{incompatibility}{\par\small\noindent\textbf{Несовместимость:}}{\par}
\fi

\def\beforenoterule{%
}

\def\betweennoterule{%
}

\def\afternoterule{%
}


\def\issue#1{\index{issues}{#1}}

\def\prefaceword{\noindent preface:}

%%% For use in "little tables"; leaves some space after the rule (16pt b/b).
\def\hlinesp{%
}

\def\quotation{\list{}{
}\item[]}
\let\endquotation=\endlist


\def\indentdesc#1{
  \@ifstar{\list{}{
    % \leftmargin=#1\relax \topsep\z@ \labelwidth\z@
    % \itemindent-\leftmargin \labelsep\z@
    % \def\makelabel####1{\hbox to #1{####1\hss}}
  }}{\list{}{
    % \leftmargin=#1\relax \labelwidth\z@
    % \itemindent-\leftmargin \labelsep\z@
    % \def\makelabel####1{\hbox to #1{####1\hss}}
  }}}
\let\endindentdesc\endlist

\def\flushdesc{\@ifstar{\list{}{}}{\list{}{}}}
\let\endflushdesc\endlist

\def\titlepage{%\@restonecolfalse\if@twocolumn\@restonecoltrue\onecolumn\else
 \newpage 
%\fi
\thispagestyle{empty}\c@page\z@}
\def\endtitlepage{%\if@restonecol\twocolumn \else
\newpage
% \fi
}
%%%%%% Corrections to some internal LaTeX macros.

%%%%%%%%%%% End of internal LaTeX corrections.

%%%%%%%%%%% Corrections to lplain.tex

\def\neq{\not\,=} \let\ne=\neq

\makeatother
