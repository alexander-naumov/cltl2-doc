%% Document style TMreport: report for Thinking Machines Corporation.
%% Produced from /usr/local/lib/tex/macros/report.sty by Guy Steele.
% 3/18/85 10:28:39  Created.
% 3/18/85 10:29:39  Uses TMrep10.sty instead of rep10.sty, etc.
% 3/18/85 11:12:44  New \maketitle puts in copyright notice and other text.
% 3/19/85 12:38:40  Added \formattedindex and \indexentry.
% 3/21/85 11:39:27  Make index be in \small font.  Adjust spacing.
% 5/22/85 12:38:42  Need to make @ be alphabetic (catcode 11) when
%		    reading in the index.  Things like \z@ show up.
% 1/31/86 15:24:51  Add hackery to put blank lines in index where
%		    initial letter changes.
% 2/03/86 16:01:52  Use \indexspace, not \bigskip.
% 2/04/86 14:10:52  Added \registered and \trademark.
% 3/07/86 14:21:25  Make \theindex use \raggedright.
% 3/11/86 16:32:19  Add new document option "onecolumnindex".
% 6/28/88 23:40:11  Add stuff to reset the \jobname for TeXtures.  Yuk.
% 8/27/88 17:17:17  Add flushdesc environment.

\def\PostScriptFile#1{\special{psfile=#1}}

\newif\ifchiron \chironfalse

\makeatletter

%\newif\if@onecolumnindex \@onecolumnindexfalse
%\newif\if@externaldocument \@externaldocumentfalse
\newif\if@draft \@draftfalse
\newif\if@permutemonospace   \@permutemonospacefalse
%\typeout{Document Style 'TMreport'. Version 0.91a - released 18 March 1985}
\def\@ptsize{0} \@namedef{ds@11pt}{\def\@ptsize{1}}
\@namedef{ds@12pt}{\def\@ptsize{2}} 
\def\ds@twoside{\@twosidetrue \@mparswitchtrue}
\def\ds@draft{\overfullrule 18pt\@drafttrue}
%\def\ds@external{\@externaldocumenttrue}
%\def\ds@onecolumnindex{\@onecolumnindextrue}
\def\ds@permutemonospace{\@permutemonospacetrue}
\@options
%\input CLTL1\@ptsize.sty\relax
\newskip\ruletonoteskip \ruletonoteskip=6pt plus 3pt minus 1pt
\newskip\ruletotextskip \ruletotextskip=10pt plus 12pt minus 2pt
\newdimen\thinrule   \thinrule=0.4pt

%\clubpenalty=1000
%\widowpenalty=1000

%%% defun.tex

% defun library.
% Guy L. Steele Jr.
% Copyright 1985, 1986 , 1988, 1989 Guy L. Steele Jr.

%  3/18/85 12:48  Now \@defunname generates index entries.
%  5/22/85 14:16  Use \noexpand before \tt and \rm in index entries.
%		  This will delay their processing until the point
%		  when the index is read back in again.
%  5/23/85 17:01  Add special check for nil as an init value for an
%		  &optional or &key argument; make it be \tt, not \it.
% 12/17/85 12:56  Add "lisp" environment.
%  1/02/86 14:15  Base "lisp" environment on tabbing instead of raggedright.
%  1/02/86 17:00  Make \endlisp use \ignorespaces.
%  1/03/86 17:00  Put a space into the index key to make sorting work better.
%  1/31/86 12:32  Simplify "lisp" environment to let the tabbing environment
%		  take care of interparagraph spacing.
%  1/31/86 13:27  Add \smallcaps stuff.
%  1/31/86 13:38  In \@defunkeymode et al., assume the user provides a :.
%  1/31/86 16:03  Flush \smallcaps stuff.
%  1/31/86 17:03  Use \addvspace in \defun et al.
%  2/04/86 17:15  Add feature so that the syntax \begin{defun}[Frob] causes "[Frob]"
%		  to appear at the right margin, as in the Common Lisp book, and causes
%		  the index entry for function baz to be "baz frob".  If no square
%		  brackets ae used, then nothing appears at the right margin, and the
%		  index entry looks like "baz \lowercase{default}" where "default" is
%		  the definition of \defaultdefuntype, which is initialized by this file
%		  to {Function} but can be redefined by the user.
%  2/07/86 17:08  Make lisp environment use \frenchspacing.
%  3/11/86 11:42  Improve some of the breaking places in the defun headers:
%		  Introduce \hbox around every name, and make better pseudo-hyphens.
%		  Also fix a bug in \@defuninitvalue.
%  3/12/86 11:47  Hair up the defunbreak stuff some more.  Now we have different
%		  behavior depending on whether the function name is long or short.
%  3/12/86 16:09  Fix a possible bug in \@showdefuntype.
%  3/13/86 14:16  More fixes to hairy defunbreak stuff.
%  5/29/86 01:47  Add \internalroutine.
%  6/04/86 14:47  Correct spacing in \@showdefuntype.
%  6/29/88 00:31  Remove priority-box macros.

\catcode`@=11	% Make it possible to refer to some LaTeX utility macros.
\catcode`&=11

%\def\lisp{\cf\tabbing}
%\let\endlisp=\endtabbing

\newcounter{lispcounter}
\newcounter{defuncounter}

\newenvironment{lisp}{\stepcounter{lispcounter}
  \begin{tabbing}}{\end{tabbing}}

\def\@defunstart{\catcode`&=11}

\def\defun{
  \typeout{\value{defuncounter}}
  \refstepcounter{defuncounter}
  \@ifstar{\@defunstart\@ifnextchar[{\@defuntyped}{\@defununtyped}}
  {\@defunstart\@ifnextchar[{\@defuntyped}{\@defununtyped}}}


\def\@showdefuntype{\textit{[\@defuntype\/]}\relax\let\@defuntype\@defunsecondtype}

\def\@defununtyped{\let\@defunshow=\relax \@defun}

\def\@defuntyped[#1]{%
  \def\@defuntype{#1}
  \let\@defunsecondtype\@defuntype
  \let\@defunshow=\@showdefuntype
  \@ifnextchar[{\@defuntypedtwice}{\@defun}}

\def\@defuntypedtwice[#1]{
  \def\@defunsecondtype{#1}
  \let\@defunshow=\@showdefuntype 
  \@defun}

\def\@defun{\@ifnextchar?{\@defunname}{\@defunname}}	% Skips spaces before name 

\def\enddefun{\par
  \typeout{enddefun currentdefun}
  \@ifstar{}{}
}    %Margin macros know this amount
% Used to be "plus 2pt" but that was not enough stretch relative to \parskip.

% 1: entry
% \@defuntype: \lowercase{category}
\def\@defunname#1 {\@defunshow
  \typeout{defunname}
  \typeout{1: #1} 
  \typeout{ type:  \lowercase{\@defuntype}}
  \textbf{#1}\@defunindex#1:\par{\@defuntype}\relax
  \ifdim 1\wd0 < 1.3in\relax
     \def\@defunbreak{ }\relax
     \let\@firstdefunbreak=~\relax
  \else
     \def\@defunbreak{ }\relax
     \let\@firstdefunbreak=\@defunbreak
  \fi
  \def\@defunkeywordbreak{\@firstdefunbreak\let\@defunkeywordbreak=\@defunbreak}\relax
  \def\@hairydefunbreak{\@firstdefunbreak
		        \let\@defunkeywordbreak=\@defunbreak
		        \let\@hairydefunbreak\@defunbreak}\relax
  \@defunreqmode}

% 1: entry
% 2: empty
% 3: \lowercase{category}
\def\@defunindex#1:#2\par#3{%
%  \typeout{defunindex}\typeout{ 1: #1}\typeout{ 2: #2}\typeout{ 3: #3}%
  \def\@tempa{}\def\thing{#2}\ifx\thing\@tempa
  \@idefunindex{}#1:\par{#3}\else \@idefunindex{#1:}#2\par{#3}\fi}
% 1: empty
% 2: entry
% 3: \lowercase{category}
\def\@idefunindex#1#2:\par#3{%
  \typeout{idefunindex}\typeout{ 1: #1}\typeout{ 2: #2}\typeout{ 3: #3}%
  \index{#2!#3}%
  \label{#2}
  }%{{#2 }{\noexpand\noexpand\noexpand\cf #1#2\ \noexpand\noexpand\noexpand\rm #3}}}


\def\@newlinecheck#1#2{\def\@tempa{#1}\def\@tempb{\\}\ifx\@tempa\@tempb \\*\let\@tempd\@defunname
  \else \def\@tempd{#2}\fi \@tempd}

\def\@ifparnext#1#2{\def\@tempa{#1}\def\@tempb{#2}\futurelet\@tempc\@ifparnx}
\def\@ifparnx{\ifx\@tempc\@@par \let\@tempd\@tempa \else \let\@tempd\@tempb \fi \@tempd}


\def\@defunreqmode{\@ifparnext{\@defundone}{\@ifnextchar &{\@defunkeyword}{\@ifnextchar({\@defunparenreqarg}{\@defunreqarg}}}}
%\def\@defunreqarg#1 {\@newlinecheck{#1}{\@hairydefunbreak\hbox{\it #1\/}\@defunreqmode}}
\def\@defunreqarg#1 {\@newlinecheck{#1}{\@hairydefunbreak \textit{#1\/} \@defunreqmode}}
%\def\@defunparenreqarg(#1 #2) {\@hairydefunbreak(\hbox{\it #1\/}~#2)\@defunreqmode}  %CLOS methods
\def\@defunparenreqarg(#1 #2) {\@hairydefunbreak(t \textit{#1\/} ~#2)\@defunreqmode}  %CLOS methods

\def\@defunrestmode{\@ifparnext{\@defundone}{\@ifnextchar &{\@defunkeyword}{\@defunrestarg}}}
\def\@defunrestarg#1 {\@newlinecheck{#1}{\penalty\@m\ #1 \@defunrestmode}}

\def\@defunoptionalmode{\@ifparnext{\@defundone}{\@ifnextchar &{\@defunkeyword}{\@ifnextchar({\@defunparenoptionalarg}{\@defunoptionalarg}}}}
\def\@defunoptionalarg#1 {\@newlinecheck{#1}{\@hairydefunbreak\textit{#1\/}\@defunoptionalmode}}
\def\@defunparenoptionalarg(#1 #2) {\@hairydefunbreak(\textit{#1\/}~\@defuninitvalue#2)\@defunoptionalmode}

\def\@defuninitvalue#1#2){{\ifcat#1A\def\@tempa{#1#2}\def\@tempb{nil}\ifx\@tempa\@tempb
  \cf \else \it \fi \else \cf \fi #1#2})}

\def\@defunkeymode{\@ifparnext{\@defundone}{\@ifnextchar &{\@defunkeyword}{\@ifnextchar({\@defunparenkeyarg}{\@defunkeyarg}}}}
\def\@defunkeyarg#1 {\@newlinecheck{#1}{\penalty\@m\@defunkeywordbreak #1 \@defunkeymode}}
\def\@defunparenkeyarg(#1 #2) {\penalty\@m~(\textit{#1} ~\@defuninitvalue#2)\@defunkeymode}

\def\@defunkeyword &#1 {\@defunkeywordbreak \textbf{\&#1} \csname @defun#1mode\endcsname}
\def\@defundone\par{%
  \par 
  \noindent
 }

\begingroup
\catcode`\[=13 \catcode`\]=13 \catcode`\(=13 \catcode`\)=13
\catcode`\<=13 \catcode`\>=13 \catcode`\|=13 \catcode`\?=13
\global\def\@defmacstart#1{\relax
%  \noindent\leavevmode     % Need this to trigger the \everypar for margin rules.
  \begingroup
  %\samepage  
  %\topsep 0pt
  \catcode`&=11
  \def\@lbracehack{{$\,\lbrace$}\@backslashsetup\begin{itshape}}\relax
  \def\@rbracehack{\end{itshape}
       \@ifnextchar*{\@rbracesuper}{\@ifnextchar+{\@rbracesuper}{\/$\rbrace\,$\@backslashsetup}}}\relax
  \def[{{$\,\lbrack$}\@backslashsetup\begin{itshape}}\relax
  \def]{\end{itshape}$\/\rbrack\,$\@backslashsetup}\relax
  \def<{{$\,\dlbrack$}\@backslashsetup\begin{itshape}}\relax
  \def>{\end{itshape}$\/\drbrack\,$\@backslashsetup}\relax
  \def({\texttt{\char40}\@backslashsetup}\relax
  \def){\texttt{\char41}\@backslashsetup}\relax
  \def|{{$\char124$}}\relax
  \def?{{$\downarrow$}}\relax
  \def\@finishdefmac{
    \end{tabbing}
    \endgroup
    \everypar{}\noindent
    }\relax
  \expandafter\def\@carret{\expandafter\@ifnextchar\@carret
     {\@finishdefmac\@gobblecr}{\@backslashsetup}}\relax
  \@defunhackbraces
  \def\@backslashsetup{\def\\{\@backslashsetup
				\@dodefmacname{#1}}
                            }\relax
  \catcode`\[=13 \catcode`\]=13 \catcode`\(=13 \catcode`\)=13
  \catcode`\<=13 \catcode`\>=13 \catcode`\|=13 \catcode`\?=13 \catcode`\^^M=13
  \def\!##1!{\cd{##1}}\relax
  \begin{tabbing}
    \@backslashsetup\@margineveryparguts
  \@dodefmacname{#1}}
\endgroup
% \begingroup
% \catcode`\[=13 \catcode`\]=13 \catcode`\(=13 \catcode`\)=13
% \catcode`\<=13 \catcode`\>=13 \catcode`\|=13 \catcode`\?=13
% \global\def\@defmacstart#1{\relax
% %  \noindent\leavevmode     % Need this to trigger the \everypar for margin rules.
%   \begingroup
%   %\samepage  
%   %\topsep 0pt
%   \catcode`&=11
%   \def\@lbracehack{{$\,\lbrace$}\=\pushtabs\+\@backslashsetup\begin{itshape}}\relax
%   \def\@rbracehack{\end{itshape}\-\poptabs
%        \@ifnextchar*{\@rbracesuper}{\@ifnextchar+{\@rbracesuper}{\/$\rbrace\,$\@backslashsetup}}}\relax
%   \def[{{$\,\lbrack$}\=\pushtabs\+\@backslashsetup\begin{itshape}}\relax
%   \def]{\end{itshape}\-\poptabs$\/\rbrack\,$\@backslashsetup}\relax
%   \def<{{$\,\dlbrack$}\=\pushtabs\+\@backslashsetup\begin{itshape}}\relax
%   \def>{\end{itshape}\-\poptabs$\/\drbrack\,$\@backslashsetup}\relax
%   \def({\texttt{\char40}\=\pushtabs\+\@backslashsetup}\relax
%   \def){\-\poptabs\texttt{\char41}\@backslashsetup}\relax
%   \def|{{$\char124$}}\relax
%   \def?{{$\downarrow$}}\relax
%   \def\@finishdefmac{\-\poptabs\end{tabbing}
%     \endgroup
%     \everypar{}\noindent}\relax
%   \expandafter\def\@carret{\expandafter\@ifnextchar\@carret
%      {\@finishdefmac\@gobblecr}{\@tabcr*\@backslashsetup}}\relax
%   \@defunhackbraces
%   \def\@backslashsetup{\def\\{\-\poptabs\@tabcr*\@backslashsetup
% 				\@dodefmacname{#1}}}\relax
%   \catcode`\[=13 \catcode`\]=13 \catcode`\(=13 \catcode`\)=13
%   \catcode`\<=13 \catcode`\>=13 \catcode`\|=13 \catcode`\?=13 \catcode`\^^M=13
%   \def\!##1!{\cd{##1}}\relax
%   \begin{tabbing}\@backslashsetup\@margineveryparguts
%   \@dodefmacname{#1}}
% \endgroup
\def\@rbracesuper#1{{\/$\rbrace^{#1}\,$}\@backslashsetup}


\def\Mopt#1{{$\,\lbrack$}\textit{#1\/}{$\rbrack\,$}}
\def\Mchoice#1{{$\,\dlbrack$}\textit{#1\/}{$\drbrack\,$}}
\def\Mstar#1{{$\,\lbrace$}\textit{#1\/}{$\rbrace^*\,$}}
\def\Mplus#1{{$\,\lbrace$}\textit{#1\/}{$\rbrace^+\,$}}
\def\Mgroup#1{{$\,\lbrace$}\textit{#1\/}{$\rbrace\,$}}
\def\Mor{{$|$}}
\def\Mind#1{$\downarrow$\textit{#1\/}}

% The following is taken from /usr/local/lib/tex/macros/latex.tex
% but corrects a scoping bug.

% \newdimen\@curtabmardimen
% \def\@startline{\global\@curtabmar\@nxttabmar\relax
%    \global\@curtabmardimen\dimen\@curtabmar
%    \global\@curtab\@curtabmar\global\setbox\@curline\hbox
%     {}\@startfield\strut}

% \def\@stopline{\unskip\@stopfield\if@rjfield \global\@rjfieldfalse
%    \@tempdima\@totalleftmargin \advance\@tempdima\linewidth
%    \hbox to\@tempdima{\@itemfudge\hskip\@curtabmardimen
%    \box\@curline\hfil\box\@curfield}\else\@addfield
%    \hbox{\@itemfudge\hskip\@curtabmardimen\box\@curline}\fi}

% End of material from /usr/local/lib/tex/macros/latex.tex .

\def\defmac{\@defmacbegin{Macro}}
\let\enddefmac\enddefun

\def\defspec{\@defmacbegin{Special form}}
\let\enddefspec\enddefun

\def\defloop{\@defmacbegin{Loop clause}}
\let\enddefloop\enddefun

\def\@defmacbegin#1{
  \refstepcounter{defuncounter}
  \@ifstar{\@defmacstart{#1}}{\@defmacstart{#1}}}

{\catcode`\^^M=13\global\def\@carret{
}\global\def\@defmacnamecrgobble#1
{\@defmacname{#1}}}

\def\@dodefmacname#1{\expandafter\@ifnextchar\@carret
   {\@defmacnamecrgobble{#1}}{\@defmacname{#1}}}

\def\@defmacname#1#2 {
  \typeout{defmacname #1: #2 }\relax
  %\setbox0\hbox{\hskip1.5em$\lbrack$\it#1\/$\rbrack$}\relax
  \textit{[#1\/]}\relax
  %\leavevmode\hbox to 0pt{\hbox to \textwidth{\hss\box0}\hss}{\cf #2\ }\=\pushtabs\+\@backslashsetup
  \textbf{ #2\ }\@backslashsetup
  \@defunindex#2:\par{#1}
  %\it
}

\bgroup
\catcode`\<=1 \catcode`\>=2 \catcode`\{=13 \catcode`\}=13
\global\def\@defunhackbraces<\catcode`\{=13\catcode`\}=13\let{\@lbracehack\let}\@rbracehack>
\egroup

\catcode`@=12		% Restore character codes
\catcode`&=4

%[End]

%%%

%%% marginal.tex

\makeatletter
\let\@margineverypar\relax
\let\@margineveryparguts\relax
\makeatother

\newenvironment{new}{%
}{%
}
\newenvironment{newer}{%
}{%
}
\newenvironment{obsolete}{%
}{%
}

%%% 

\makeatletter

% The \null in the following is intended to suppress hyphenation
% in code words not already containing hyphens.  The \leavevmode
% is needed to prevent vertical mode from swallowing the \null.
%%%\def\cd#1{\leavevmode{\cf \null#1}}
%\def\cd{\leavevmode\begingroup\cf\@cd} 
\def\cd{\@cd}
%\def\@cd#1{\null#1\endgroup}
% if label exists, then automatically make hyperref
% else insert plain text
\def\@cd#1{\textit{\null #1}}

\if@permutemonospace
  \typeout{PERMUTE}
  \def\undo@permute{\catcode"2D=12  %hyphen
                    \catcode"27=12  %quote
                    \catcode"3D=12  %equals
                   }
  \ifchiron
    \def\Xhyphen{{\tt\char"2D}}
  \else
    \def\Xhyphen{{\tt\char"2D}}       % use endash for hyphen
  \fi
  % use acute of next larger size for quote
  \def\Xquote{{\tt\@tempdima=-1.36ex\setbox0\hbox to 1em{\hss\biggersize\tt
         \advance\@tempdima by 1.36ex
         \lower\@tempdima\hbox{\ifchiron\char"27\else\char"27\fi}\hss}\dp0=0pt\box0}}
  % use grave of next larger size for backquote
  \def\Xbq{{\tt\@tempdima=-1.36ex\setbox0\hbox to 1em{\hss\biggersize\tt
         \advance\@tempdima by 1.36ex
         \lower\@tempdima\hbox{\ifchiron\char"60\else\char"60\fi}\hss}\dp0=0pt\box0}}
  % use tilde of next larger size for asciitilde
  \def\Xtilde{{\tt\@tempdima=-1.36ex\setbox0\hbox to 1em{\hss\biggersize\tt
         \advance\@tempdima by 1.36ex
         \lower\@tempdima\hbox{\ifchiron\char"7E\else\char"7E\fi}\hss}\dp0=0pt\box0}}
  % use circumflex of next larger size for asciicircumflex
  \def\Xcircumflex{{\tt\@tempdima=-1.36ex\setbox0\hbox to 1em{\hss\biggersize\tt
         \advance\@tempdima by 1.36ex
         \lower\@tempdima\hbox{\ifchiron\char"5E\else\char"5E\fi}\hss}\dp0=0pt\box0}}
  \def\Xsharp{\hbox{\tt\hbox to 0pt{\char35\hss}\hbox to 0pt
                       {\raise0.189ex\hbox{\char45}\hss}\lower0.189ex\hbox{\char45}}}
  % Make equals sign from two em dashes
  \def\Xequal{\hbox{\tt\hbox to 0pt
                       {\raise0.189ex\hbox{\ifchiron\char"A3\else\char"55\fi}\hss}\relax
                        \lower0.189ex\hbox{\ifchiron\char"A3\else\char"55\fi}}}
  \def\index@hyphen{-}
  \def\index@quote{'}
  \def\index@equal{=}
  {\catcode"2D=13  %hyphen
   \catcode"27=13  %quote
   \catcode"3D=13  %equals
   \global\def\index@permute{\let-\index@hyphen
                             \let'\index@quote
                             \let=\index@equal}
   \global\def\@permute{\catcode"2D 13  %hyphen
                        \catcode"27 13  %quote
                        \catcode"3D 13  %equals
                        \let-\Xhyphen
                        \let'\Xquote
                        \let=\Xequal
                        \let\#\Xsharp
                        \def\rm{\undo@permute\protect\prm}\relax
                        \def\it{\undo@permute\protect\pit}\relax
                        \def\bf{\undo@permute\protect\pbf}\relax
                       }
  }
  \def\cf{\@permute\tt\frenchspacing}
\else
  \def\Xtilde{\texttt{\char"7E}}
  \def\Xbq{\texttt{\char"60}}
  \def\Xquote{\texttt{\char"27}}
  \def\Xequal{\texttt{\char"3D}}
  \def\Xcircumflex{\texttt{\char"5E}}
  \def\cf{} %\frenchspacing}
\fi
\ifchiron
  \def\Xarrowright{\texttt{\char"A2}}
  \def\Xarrowdown{\texttt{\char"A4}}
\else
   \def\Xarrowright{\begin{math} \Rightarrow \end{math}}
   \def\Xarrowdown{\begin{math} \Downarrow \end{math}}
\fi
\def\Xatsign{\texttt{\char"40}}
\def\Xbackslash{\texttt{\char"5C}}
\def\Xunderscore{\texttt{\char"5F}}
\def\Xlbracket{\texttt{\char"5B}}
\def\Xrbracket{\texttt{\char"5D}}
\def\Xlbrace{\texttt{\char"7B}}
\def\Xrbrace{\texttt{\char"7D}}
\def\dlbrack{\lbrack\mskip-\thinmuskip\lbrack\mskip\thinmuskip}
\def\drbrack{\mskip\thinmuskip\rbrack\mskip-\thinmuskip\rbrack}
\def\Xdquote{\texttt{\char"22}}

\def\EV{$\Rightarrow$}
\def\EX{$\rightarrow$}
\def\EQ{$\equiv$}

\newcommand{\indexterm}[1]{\index{#1}}

%\def\empty{\cd{(}\cd{)}}
\def\emptylist{\cd{(}\cd{)}}
\def\false{\cd{nil}}
\def\true{\cd{t}}
\def\nil{\cd{nil}}

% \renewcommand\it[1]{\textit{#1}}

\def\implementation{\par\small\noindent\textbf{Implementation note:}}
\def\endimplementation{\par}
\def\rationale{\par\small\noindent\textbf{Rationale:}}
\def\endrationale{\par}
\def\sideremark{\par\small\noindent\textbf{Remark:}}
\def\endsideremark{\par}
\def\incompatibility{\par\small\noindent\textbf{Compatibility note:}}
\def\endincompatibility{\par}

\def\beforenoterule{
% \par\@tempskipb\lastskip \@tempdima\prevdepth
%   \ifdim 0pt>\@tempdima \@tempdima=0pt\fi
%   \ifdim 14pt>\@tempskipb \vskip-\@tempskipb \vskip 14pt \fi
%   \kern-\@tempdima\nointerlineskip
%        \hrule height \thinrule width 12pc
%   \nobreak\vskip 16pt\vskip-\baselineskip\hbox{}\nobreak
}

\def\betweennoterule{
% \par\@tempskipb\lastskip \@tempdima\prevdepth
%   \ifdim 0pt>\@tempdima \@tempdima=0pt\fi
%   \nobreak
%   \vskip-\@tempskipb \vskip 11pt
%   \kern-\@tempdima\nointerlineskip
% 		     \hrule height \thinrule width 12pc\penalty-500\vskip -\thinrule
%        % The game here is to have *two* rules that normally coincide exactly,
%        % but if a page break occurs between them then you get one at the
%        % bottom of the first page and one at the top of the second page.
%        \hrule height \thinrule width 12pc
% 		\nobreak\vskip 16pt\vskip-\baselineskip\hbox{}\nobreak
}

\def\afternoterule{
  %\par\@tempskipb\lastskip \@tempdima \prevdepth
  %\ifdim 0pt>\@tempdima \@tempdima=0pt\fi
  %\nobreak
  %\vskip-\@tempskipb \vskip 11pt
  %\nobreak\kern-\@tempdima\nointerlineskip
  %     \hrule height \thinrule width 12pc
  %\vskip 16pt
}

\def\afternoterule{
  % \par\nobreak\vskip\ruletonoteskip
  % \hrule height \thinrule width 12pc
  % \vskip\ruletotextskip
}


\def\issue#1{{$\langle$}\@ifundefined{i@#1}{\textbf{?}\@warning
   {Issue `#1' on page \thepage \space undefined}}{\csname
       i@#1\endcsname}{$\rangle$}\if@draft
  \actualmarginfrob{\setbox0=\hbox{\parbox[t]{12pc}{\raggedright\rm
         Issue \cf #1}}\ht0=0pt\dp0=0pt\box0}\fi}

\def\issueitem#1{\@ifnextchar[{\@issueitem{#1}{:}}{\@issueitem{#1}{}[]}}

\def\@issueitem#1#2[#3]{\item\if@filesw \immediate\write\@auxout
       {\string\issuecite{#1}{\the\c@enumi}}\fi
       \cd{#1#2#3~~}\unskip}
\def\issuecite#1#2{\global\@namedef{i@#1}{#2}}

\def\prefaceword{\noindent{\sc p{\kern1pt}r{\kern1pt}e{\kern1pt}f{\kern1pt}a{\kern1pt}c{\kern1pt}e{\kern1pt}:} }

%%% For use in "little tables"; leaves some space after the rule (16pt b/b).
\def\hlinesp{
% \noalign{{\@tempdima \prevdepth
%                        \hrule \@height \arrayrulewidth
%                        \vskip-\arrayrulewidth
%                         \ifdim 0pt<\prevdepth \vskip-\prevdepth\fi
%                        % Instead of undoing a baselineskip and then skipping 16pt,
%                        % we note that a baselineskip is 12.5pt, and 16 is 1.28 times that.
%                        % This makes it work in \small size as well.
%                        \vskip 0.28\normalbaselineskip \hbox{}}}
}

\def\quotation{\list{}{\listparindent 1.5em
\itemindent\z@
\labelwidth\z@ \labelsep\z@
%\rightmargin\leftmargin
\parsep 0pt plus 1pt}\item[]}
\let\endquotation=\endlist
%\def\quote{\list{}{\rightmargin\leftmargin}\item[]}
%\let\endquote=\endlist
%\def\descriptionlabel#1{\hspace\labelsep \bf #1}
%\def\description{\relax
%  \list{}{\labelwidth\z@ \itemindent-\leftmargin
%  \let\makelabel\descriptionlabel}}
%\let\enddescription\endlist
%\newdimen\descriptionmargin
%\descriptionmargin=3em
\def\indentdesc#1{\@ifstar{\list{}{\leftmargin=#1\relax \topsep\z@ \labelwidth\z@
                                   \itemindent-\leftmargin \labelsep\z@
                                   \def\makelabel####1{\hbox to #1{####1\hss}}}}{\list
        {}{\leftmargin=#1\relax \labelwidth\z@
           \itemindent-\leftmargin \labelsep\z@
           \def\makelabel####1{\hbox to #1{####1\hss}}}}}
\let\endindentdesc\endlist

% Add 3.5 pt depth to yield 16pt baseline-to-baseline.
\def\flushdesclabel#1{{\setbox0\hbox to \linewidth{\rm\strut#1\hfill}\relax
                       \@tempdima\dp0 \advance\@tempdima by 3.5pt \dp0\@tempdima \box0}}
\def\flushdesc{\@ifstar{\list{}{\topsep\z@
                       \itemsep=\topsep  \parsep 0pt plus 0.25pt
                       \labelwidth\z@ \labelsep\z@
                       \leftmargin\z@ \itemindent\z@
                       \listparindent 1em
                       \let\makelabel\flushdesclabel}}{\list
                       {}{\topsep=12pt plus 2pt minus 2pt
                       \itemsep=\topsep  \parsep 0pt plus 0.25pt
                       \labelwidth\z@ \labelsep\z@
                       \leftmargin\z@ \itemindent\z@
                       \listparindent 1em
                       \let\makelabel\flushdesclabel}}}
\let\endflushdesc\endlist

%? \def\@begintheorem#1#2{\it \trivlist \item[\hskip \labelsep{\bf #1\ #2}]}
%? \def\@endtheorem{\endtrivlist}
%? \@addtoreset{equation}{chapter} 
%? \def\theequation{\thechapter.\arabic{equation}}

\def\titlepage{%\@restonecolfalse\if@twocolumn\@restonecoltrue\onecolumn\else
 \newpage 
%\fi
\thispagestyle{empty}\c@page\z@}
\def\endtitlepage{%\if@restonecol\twocolumn \else
\newpage
% \fi
}
%\arraycolsep 5pt \tabcolsep 6pt \arrayrulewidth \thinrule \doublerulesep 2pt 
%\tabbingsep \labelsep 
%\skip\@mpfootins = \skip\footins
%\fboxsep = 3pt \fboxrule = \thinrule 

\def\@pnumwidth{1.55em}
\def\@tocrmarg {2.55em}
\def\@dotsep{4.5}
\setcounter{tocdepth}{3}

\def\numberline#1{\setbox0=\hbox{#1. }\ifdim \wd0 < \@tempdima
    \hbox to \@tempdima{\box0\hfil}\else \box0 \fi}

\def\tableofcontents{\begingroup\columnseprule \z@
   \columnsep 1.5pc\twocolumn[\@makeschapterhead{Contents}]\relax
   \markboth{CONTENTS}{CONTENTS}\thispagestyle{plain}\raggedright\small
   \parskip\z@ plus .4pt\@starttoc{toc}
   \l@chapter{References}{\pageref{REFERENCES}}
   \l@chapter{Index of X3J13 Votes}{\pageref{ISSUES}}
   \l@chapter{Other Indexes}{\pageref{INDEX}}
   \endgroup
   \vfill    % remove this if within 2 lines of bottom of page?
   \onecolumn}

\newdimen\tocindent   \tocindent=2.5pc
\def\l@chapter#1#2{\vskip 12pt plus 1fil
{\samepage \hangindent 1.5pc \noindent
 \@tempdima 0pt{\normalsize\bf \ignorespaces#1}\nobreak\hskip 1em{#2}\par}
 \nobreak\vskip 4pt}
\def\l@section#1#2{{\samepage \noindent \hangindent\tocindent
                   \@tempdima\tocindent
                   #1\nobreak\hskip 1em\relax #2\par}}
\let\l@subsection=\l@section
\let\l@subsubsection\l@subsection
%? \def\l@paragraph{\@dottedtocline{4}{10em}{5em}}
%? \def\l@subparagraph{\@dottedtocline{5}{12em}{6em}}
%? \def\listoftables{\@restonecolfalse\if@twocolumn\@restonecoltrue\onecolumn
%? \fi\chapter*{List of Tables}\markright
%? {List of Tables}\@starttoc{lot}\if@restonecol\twocolumn
%? \fi}
%? \def\l@table{\@dottedtocline{1}{1.5em}{2.3em}}
%? \def\listoffigures{\@restonecolfalse\if@twocolumn\@restonecoltrue\onecolumn
%? \fi\chapter*{List of Figures}\markright
%? {List of Figures}\@starttoc{lof}\if@restonecol\twocolumn
%? \fi}
%? \let\l@figure\l@table

\def\thebibliography#1{\chapter*{References}\markboth
{REFERENCES}{REFERENCES}\label{REFERENCES}\small\list
{[\arabic{enumi}]}{\settowidth\labelwidth{[#1] }\leftmargin\labelwidth
\advance\leftmargin\labelsep \parsep 0pt\itemsep 0pt plus 2pt\usecounter{enumi}}}
\let\endthebibliography=\endlist


%\def\formattedindex{\openin1 \jobname.fdx \ifeof1
%  \typeout{No file \jobname.fdx.}\else
%  \closein1\newpage\begin{theindex}\catcode`@=11\@@input \jobname.fdx\end{theindex}\fi}
%\def\indexentry#1#2{\@indexentry#1{#2}}
%\let\@indexlastinitial=\relax
%\def\@indexsetinitial#1#2\par{\let\@indexlastinitial=#1}
%\def\@indexentry#1#2#3{\if\@indexlastinitial\relax
%		          \@indexsetinitial#1\par
%                       \else
%			  \let\@indexoldinitial=\@indexlastinitial
%			  \@indexsetinitial#1\par
%			  \if\@indexlastinitial\@indexoldinitial
%			  \else
%			     \indexspace
%			  \fi
%		       \fi
%		       \item {#2}\ \ \relax#3}

%? \def\footnoterule{\kern-1\p@ 
%? \hrule width .4\columnwidth 
%? \kern .6\p@} 
%? \@addtoreset{footnote}{chapter} 
%? \long\def\@makefntext#1{\@setpar{\@@par\@tempdima \hsize 
%? \advance\@tempdima-10pt\parshape \@ne 10pt \@tempdima}\par
%? \parindent 1em\noindent \hbox to \z@{\hss$^{\@thefnmark}$}#1}

\setcounter{topnumber}{2}
\def\topfraction{.9}
\setcounter{bottomnumber}{1}
\def\bottomfraction{.3}
\setcounter{totalnumber}{3}
\def\textfraction{.1}
\def\floatpagefraction{.1}
\setcounter{dbltopnumber}{2}
\def\dbltopfraction{.7}
\def\dblfloatpagefraction{.5}
\long\def\@makecaption#1#2{
\noindent {\small {\bf #1:} #2}\par\addvspace{12pt}
%   \setbox\@tempboxa\hbox{#1: #2}
%   \ifdim \wd\@tempboxa >\hsize \unhbox\@tempboxa\par \else \hbox
%   to\hsize{\hfil\box\@tempboxa\hfil} 
%   \fi
}
\def\@qfloat#1[#2]{\@xfloat{#1}[#2]\bgroup \hrule height \thinrule \small
       \vskip 20pt\vskip-\baselineskip\hbox{}}
\def\@qpagefloat#1[#2]{\@xfloat{#1}[#2]\vbox to \textheight
                                   \bgroup \hrule height \thinrule \small
       \vskip 20pt\vskip-\baselineskip\hbox{}}

\def\headrule{%
  %\hbox to 0pt{\vbox to 0pt{\vskip-13pt\hrule height 0pt depth \thinrule width 12pc
  %\vskip-5pc\hbox to 0pt{\hskip-3pc\small\rm\tagline\hss}\vss}\hss}}
}

\def\runningtitle#1{\markboth{#1}{}}
\markboth{}{}
\mark{{}{}} 

%%%%%% Corrections to some internal LaTeX macros.

%%%%%%%%%%% End of internal LaTeX corrections.

%%%%%%%%%%% Corrections to lplain.tex

\def\neq{\not\,=} \let\ne=\neq

%%%%%%%%%%% End of corrections to lplain.tex

%\def\makeindex{\if@filesw \newwrite\@indexfile
%  \immediate\openout\@indexfile=\jobname.idx
%  \def\index{\@bsphack\begingroup\@sanitize\@wrindex\@indexfile}\typeout
%  {Writing index file \jobname.idx }\fi}

%\def\@wrindex#1#2{\let\thepage\relax \index@permute
%   \xdef\@gtempa{\write#1{\string
%      \indexentry{#2}{\thepage}}}\endgroup\@gtempa
%   \if@nobreak \ifvmode\nobreak\fi\fi\@esphack}

%\def\index{\@bsphack\begingroup\@sanitize\@index}

%\def\@index#1{\endgroup\@esphack}


\makeatother
