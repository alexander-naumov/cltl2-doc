\documentclass{article}

% !!! Эту комбинацию не менять
\usepackage[T2A]{fontenc} % включаем русские шрифты
\usepackage[utf8]{inputenc} % включаем поддержку UTF8
\usepackage[russian,english]{babel} % включаем пакет для поддержки русского языка

\usepackage{color,graphicx}
\usepackage{hyperref}

\pagestyle{headings}

\title{Common Lisp the Languange, 2nd Edition}

\begin{document}

\begin{figure}
\href{./en/clm.html}{\logo{cltl_sml}} \section*{Common Lisp the Language, 2nd Edition. Guy Steele}
\end{figure}

This document contains book with some corrections from
\url{http://bc.tech.coop/cltl2-ansi.htm}.

To use it, start with \href{./en/clm.html}{Title page}. A searchable index
interface to the book is \href{./en/symbols.html}{here}.

You can also download \href{./enpdf/cltl2.pdf}{pdf version}.

Please report errors in the book to filonenko.mikhail<at>gmail.com.

\begin{figure}
\href{./ru/clm.html}{\logo{cltl_sml}} \section*{Язык Common Lisp, 2-ое издание. Гай Стил мл.}
\end{figure}

Данный сайт содержит перевод книги с правками, взятыми из следующего документа
\url{http://bc.tech.coop/cltl2-ansi.htm}.

Для чтения онлайн версии перейдите на  \href{./ru/clm.html}{первую страницу}.
Для поиска документации по предментному указателю используйте данную \href{./en/symbols.html}{страницу}.

Вы также можете скачать \href{./rupdf/cltl2.pdf}{pdf версию} книги.

Ошибки, замечания и пожелания можно присылать по адресу filonenko.mikhail<at>gmail.com.


\end{document}
%%% Local Variables: 
%%% mode: latex
%%% TeX-master: t
%%% End: 
