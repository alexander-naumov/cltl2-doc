\begin{titlepage}

\makeatletter
\if@draft
\vbox to 0pt{\vss
\begin{center}
\Huge DRAFT
\end{center}
\vskip 16pt}
\fi
\makeatother

\hrule width 16pc height 1.5pt

\vskip 10pt
\noindent\textbf{\huge Common Lisp}
\vskip 20pt
\noindent\textbf{\LARGE The Language}
\vskip 10pt

\hrule width 16pc

\vskip 8pt
\noindent\textbf{\Large Second Edition}
\vskip 8pt
\hrule width 16pc
\vskip 10pt
\begin{flushleft}
\textbf{\large Guy L. Steele Jr.} \\
\emph{Thinking Machines Corporation} \\[10pt]
\emph{with contributions by} \\[5pt]
\textbf{Scott E. Fahlman} \\
\emph{Carnegie-Mellon University} \\[5pt]
\textbf{Richard P. Gabriel} \\
\emph{Lucid, Inc.} \\
\emph{Stanford University} \\[5pt]
\textbf{David A. Moon} \\
\emph{Symbolics, Incorporated} \\[5pt]
\textbf{Daniel L. Weinreb} \\
\emph{Symbolics, Incorporated} \\[10pt]
\emph{and with contributions to the second edition by} \\[5pt]
\textbf{Kent M. Pitman} \\
\emph{Symbolics, Incorporated} \\[5pt]
\textbf{Richard C. Waters} \\
\emph{Massachusetts Institute of Technology} \\[5pt]
\textbf{Jon L White} \\
\emph{Lucid, Inc.}
\end{flushleft}

\vfill
\begin{center}
\copyright{} 1984, 1989 Guy L. Steele Jr. All rights reserved.
\end{center}
\vfill
\begin{flushright}
To be published by Digital Press.
\end{flushright}
\end{titlepage}

\let\titlepage=\relax

\newpage

\makeatletter
\if@draft
\vbox to 0pt{\vss
\begin{center}
\Huge Черновик
\end{center}
\vskip 16pt}
\fi
\makeatother

\hrule width 16pc height 1.5pt
\vskip 10pt
\noindent\textbf{\huge Common Lisp}
\vskip 20pt
\noindent\textbf{\LARGE Язык}
\vskip 10pt
\hrule width 16pc
\vskip 8pt
\noindent\textbf{\Large Второе издание}
\vskip8pt
\hrule width 16pc
\vskip 10pt
\begin{flushleft}
\textbf{\large Guy L. Steele Jr.} \\
\emph{Thinking Machines Corporation} \\[10pt]
\emph{with contributions by} \\[5pt]
\textbf{Scott E. Fahlman} \\
\emph{Carnegie-Mellon University} \\[5pt]
\textbf{Richard P. Gabriel} \\
\emph{Lucid, Inc.} \\
\emph{Stanford University} \\[5pt]
\textbf{David A. Moon} \\
\emph{Symbolics, Incorporated} \\[5pt]
\textbf{Daniel L. Weinreb} \\
\emph{Symbolics, Incorporated} \\[10pt]
\emph{and with contributions to the second edition by} \\[5pt]
\textbf{Kent M. Pitman} \\
\emph{Symbolics, Incorporated} \\[5pt]
\textbf{Richard C. Waters} \\
\emph{Massachusetts Institute of Technology} \\[5pt]
\textbf{Jon L White} \\
\emph{Lucid, Inc.}
\end{flushleft}

\vfill
\begin{center}
\copyright{} 1984, 1989 Guy L. Steele Jr. Все права защищены.
\end{center}
\vfill
\begin{flushright}
Опубликовано Digital Press.
\end{flushright}

\let\titlepage=\relax

\newpage 

\null
\vskip 1in

\begingroup
\raggedright
\list{}{\rightmargin=8pc \leftmargin=8pc}\item[] \small
Would it be wonderful if, under the pressure of all these difficulties,
the Convention should have been forced into some deviations from that
artificial structure and regular symmetry which an abstract view of the
subject might lead an ingenious theorist to bestow on a constitution
planned in his closet or in his imagination?
\par\vskip 4pt
\begin{tabbing}
---\=\emph{James Madison, The Federalist} \\
\>\emph{No. 37, January 11, 1788}
\end{tabbing}
\endlist
\endgroup

\newpage 

\null
\vskip 1in

\begingroup
\raggedright
\list{}{\rightmargin=8pc \leftmargin=8pc}\item[] \small
Стоит ли удивляться, если под давлением всех упомянутых трудностей конвент был
вынужден отойти от искусственной структуры и полной симметрии, которые
многомудрый теоретик в угоду своим отвлеченным взглядам предпослал составленной
им в кабинете или в воображении конституции?

\par\vskip 4pt
\begin{tabbing}
---\=\emph{Джеймс Мэдисон, Журнал <<Федералист>>} \\
\>\emph{№ 37, 11 января, 1788} \\
\>\emph{\href{http://grachev62.narod.ru/Fed/Fed\_37.htm}{http://grachev62.narod.ru/Fed/Fed\_37.htm}
  
}
\end{tabbing}
\endlist
\endgroup

