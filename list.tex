%Part{List, Root = "CLM.MSS"}
%%% Chapter of Common Lisp Manual.  Copyright 1984, 1988, 1989 Guy L. Steele Jr.

\clearpage\def\pagestatus{FINAL PROOF}

\chapter{Lists}

A {\it cons}, or dotted pair, is a compound data object having two components
called the {\it car} and {\it cdr}.  Each component may be any Lisp object.
A {\it list} is a chain of conses
linked by {\it cdr} fields; the chain is terminated by some atom
(a non-cons object).
An ordinary list is terminated by {\nil}, the empty list
(also written {\emptylist}).
A list whose {\it cdr} chain is terminated by some non-{\nil} atom is called
a {\it dotted list}.

The recommended predicate for testing for the end of a list is \cd{endp}.

\section{Conses}

These are the basic operations on conses viewed as pairs rather than
as the constituents of a list.

\begin{defun}[Function]
car list

This returns the {\it car} of {\it list}, which must be a cons or {\emptylist};
that is, {\it list} must satisfy the predicate \cd{listp}.
By definition, the {\it car} of {\emptylist} is {\emptylist}.
If the cons is regarded as the first cons of a list, then \cd{car}
returns the first element of the list.
For example:
\begin{lisp}
(car '(a b c)) \EV\ a
\end{lisp}
See \cd{first}.
The {\it car} of a cons may be altered by using \cd{rplaca} or \cd{setf}.
\end{defun}

\begin{defun}[Function]
cdr list

This returns the {\it cdr} of {\it list}, which must be a cons or {\emptylist};
that is, {\it list} must satisfy the predicate \cd{listp}.
By definition, the {\it cdr} of {\emptylist} is {\emptylist}.
If the cons is regarded as the first cons of a list, then \cd{cdr}
returns the rest of the list, which is a list with all elements
but the first of the original list.
For example:
\begin{lisp}
(cdr '(a b c)) \EV\ (b c)
\end{lisp}
See \cd{rest}.
The {\it cdr} of a cons may be altered by using \cd{rplacd} or \cd{setf}.
\end{defun}

\begin{defun}[Function]
caar list \\
cadr list \\
cdar list \\
cddr list \\
caaar list \\
caadr list \\
cadar list \\
caddr list \\
cdaar list \\
cdadr list \\
cddar list \\
cdddr list \\
caaaar list \\
caaadr list \\
caadar list \\
caaddr list \\
cadaar list \\
cadadr list \\
caddar list \\
cadddr list \\
cdaaar list \\
cdaadr list \\
cdadar list \\
cdaddr list \\
cddaar list \\
cddadr list \\
cdddar list \\
cddddr list

All of the compositions of up to four \cd{car} and \cd{cdr} operations
are defined as separate Common Lisp functions.
The names of these functions begin with \cd{c} and end with \cd{r},
and in between is a sequence of \cd{a} and \cd{d} letters
corresponding to
the composition performed by the function. 
For example:
\begin{lisp}
(cddadr x) {\rm is the same as} (cdr (cdr (car (cdr x))))
\end{lisp}
If the argument is regarded as a list, then \cd{cadr} returns
the second element of the list, \cd{caddr} the third, and \cd{cadddr}
the fourth.  If the first element of a list is a list, then
\cd{caar} is the first element of the sublist, \cd{cdar} is the
rest of that sublist, and \cd{cadar} is the second element of the sublist,
and so on.

As a matter of style, it is often preferable to define a function or
macro to access part of a complicated data structure, rather than to use
a long \cd{car}/\cd{cdr} string.  For example, one might define
a macro to extract the list of parameter variables from a lambda-expression:
\begin{lisp}
(defmacro lambda-vars (lambda-exp) `(cadr ,lambda-exp))
\end{lisp}
and then use \cd{lambda-vars} for this purpose instead of \cd{cadr}.
See also \cd{defstruct}, which will automatically define
new record data types and access functions for instances of them.

Any of these functions may be used to specify a {\it place} for \cd{setf}.
\end{defun}
	
\begin{defun}[Function]
cons x y

\cd{cons} is the primitive function to create a new {\it cons} whose
{\it car} is {\it x} and whose {\it cdr} is {\it y}.
For example:
\begin{lisp}
(cons 'a 'b) \EV\ (a . b) \\
(cons 'a (cons 'b (cons 'c '{\emptylist}))) \EV\ (a b c) \\
(cons 'a '(b c d)) \EV\ (a b c d)
\end{lisp}
\cd{cons} may be thought of as creating a {\it cons}, or as adding a new element
to the front of a list.
\end{defun}

\begin{defun}[Function]
tree-equal x y &key :test :test-not

This is a predicate that is true if {\it x} and {\it y} are
isomorphic trees with identical leaves, that is, if {\it x} and {\it y}
are atoms that satisfy the test (by default \cd{eql}),
or if they are both conses and their {\it car}'s are \cd{tree-equal}
and their {\it cdr}'s are \cd{tree-equal}.
Thus \cd{tree-equal} recursively compares conses (but not any other objects
that have components).  See \cd{equal}, which does recursively
compare certain other structured objects, such as strings.

\begin{new}
X3J13 voted in January 1989
\issue{MAPPING-DESTRUCTIVE-INTERACTION}
to restrict user side effects; see section~\ref{STRUCTURE-TRAVERSAL-SECTION}.
\end{new}
\end{defun}

\section{Lists}


The following functions perform various operations on lists.
\begin{newer}
The list is one of the original Lisp data types.  The very name ``Lisp''
is an abbreviation for ``LISt Processing.''
\end{newer}

\goodbreak


\begin{defun}[Function]
endp object

The predicate \cd{endp} is the recommended way to test for the end
of a list.  It is false of conses, true of {\nil}, and an error for
all other arguments.

\beforenoterule
\begin{implementation}
Implementations are encouraged to signal an
error, especially in the interpreter, for a non-list argument.
The \cd{endp} function is defined so as to allow compiled code
to perform simply an atom check or a null check if speed is more
important than safety.
\end{implementation}
\afternoterule
\end{defun}

\begin{defun}[Function]
list-length list

\cd{list-length} returns, as an integer, the length of {\it list}.
\cd{list-length} differs from \cd{length} when the {\it list} is
circular; \cd{length} may fail to return, whereas \cd{list-length}
will return {\nil}.
For example:
\begin{lisp}
(list-length '{\emptylist}) \EV\ 0 \\
(list-length '(a b c d)) \EV\ 4 \\
(list-length '(a (b c) d)) \EV\ 3 \\
(let ((x (list 'a b c))) \\
~~(rplacd (last x) x) \\
~~(list-length x)) \EV\ {\nil}
\end{lisp}
\cd{list-length} could be implemented as follows:
\begin{lisp}
(defun list-length (x) \\
~~(do ((n 0 (+ n 2))~~~~~~~~~~~~;{\rm Counter} \\
~~~~~~~(fast x (cddr fast))~~~~~;{\rm Fast pointer: leaps by 2} \\
~~~~~~~(slow x (cdr slow)))~~~~~;{\rm Slow pointer: leaps by 1} \\
~~~~~~(nil) \\
~~~~;; If fast pointer hits the end, return the count. \\
~~~~(when (endp fast) (return n)) \\
~~~~(when (endp (cdr fast)) (return (+ n 1))) \\
~~~~;; If fast pointer eventually equals slow pointer, \\
~~~~;;  then we must be stuck in a circular list. \\
~~~~;; (A deeper property is the converse: if we are \\
~~~~;;  stuck in a circular list, then eventually the \\
~~~~;;  fast pointer will equal the slow pointer. \\
~~~~;;  That fact justifies this implementation.) \\
~~~~(when (and (eq fast slow) (> n 0)) (return nil))))
\end{lisp}
See \cd{length}, which will return the length of any sequence.
\end{defun}

\begin{defun}[Function]
nth n list

\cd{(nth {\it n} {\it list})} returns the {\it n}th element of {\it list}, where
the {\it car} of the list is the ``zeroth'' element.
The argument {\it n} must be a non-negative integer.
If the length of the list is not greater than {\it n}, then the result
is {\emptylist}, that is, {\false}.
(This is consistent with the idea that the {\it car} and {\it cdr}
of {\emptylist} are each {\emptylist}.)
For example:
\begin{lisp}
(nth 0 '(foo bar gack)) \EV\ foo \\
(nth 1 '(foo bar gack)) \EV\ bar \\
(nth 3 '(foo bar gack)) \EV\ {\emptylist}
\end{lisp}

\beforenoterule
\begin{incompatibility}
This is not
the same as the Interlisp function called \cd{nth},
which is similar to but not exactly the same as the Common Lisp function
\cd{nthcdr}.  This definition of \cd{nth} is compatible
with Lisp Machine Lisp and NIL (New Implementation of Lisp).
Also, some people have used macros and functions called \cd{nth} of their own in
their old MacLisp programs, which may not work the same way.
\end{incompatibility}
\afternoterule

\cd{nth} may be used to specify a {\it place} to \cd{setf};
when \cd{nth} is used in this way, the argument {\it n} must be less
than the length of the {\it list}.

Note that the arguments to \cd{nth} are reversed from the order
used by most other sequence selector functions such as \cd{elt}.
\end{defun}


\begin{defun}[Function]
first list \\
second list \\
third list \\
fourth list \\
fifth list \\
sixth list \\
seventh list \\
eighth list \\
ninth list \\
tenth list

These functions are sometimes convenient for accessing particular
elements of a list.  \cd{first} is the same as \cd{car},
\cd{second} is the same as \cd{cadr}, \cd{third} is the
same as \cd{caddr}, and so on.
Note that the ordinal numbering used here is one-origin,
as opposed to the zero-origin numbering used by \cd{nth}:
\begin{lisp}
(fifth x) \EQ\ (nth 4 x)
\end{lisp}

\cd{setf} may be used with each of these functions to store
into the indicated position of a list.
\end{defun}

\begin{defun}[Function]
rest list

\cd{rest} means the same as \cd{cdr} but mnemonically complements \cd{first}.
\cd{setf} may be used with \cd{rest} to replace the {\it cdr} of a list
with a new value.
\end{defun}

\begin{defun}[Function]
nthcdr n list

\cd{(nthcdr {\it n} {\it list})} performs the \cd{cdr} operation {\it n} times
on {\it list}, and returns the result.
For example:
\begin{lisp}
(nthcdr 0 '(a b c)) \EV\ (a b c) \\
(nthcdr 2 '(a b c)) \EV\ (c) \\
(nthcdr 4 '(a b c)) \EV\ {\emptylist}
\end{lisp}
In other words, it returns the {\it n}th {\it cdr} of the list.

\beforenoterule
\begin{incompatibility}
This is similar to the Interlisp function \cd{nth},
except that the Interlisp function is one-based instead of zero-based.
\end{incompatibility}
\afternoterule

\begin{lisp}
(car (nthcdr n x)) \EQ\ (nth n x)
\end{lisp}
\begin{new}
X3J13 voted in January 1989
\issue{ARGUMENTS-UNDERSPECIFIED}
to clarify that the argument {\it n}
must be a non-negative integer.
\end{new}

\end{defun}

\begin{obsolete}
\begin{defun}[Function]
last list

\cd{last} returns the last cons ({\it not} the last element!) of {\it list}.
If {\it list} is {\emptylist}, it returns {\emptylist}.
For example:
\begin{lisp}
(setq x '(a b c d)) \\
(last x) \EV\ (d) \\
(rplacd (last x) '(e f)) \\
x \EV\ '(a b c d e f) \\
(last '(a b c . d)) \EV\ (c . d)
\end{lisp}
\end{defun}
\end{obsolete}

\begin{new}
X3J13 voted in June 1988
\issue{LAST-N}
to extend the \cd{last} function to accept
an optional second argument.  The effect is to make \cd{last}
complementary in operation to \cd{butlast}.
The new description (with some additional examples) would be as follows.

\begin{defun}[Function]
last list &optional (n 1)

\cd{last} returns the tail of the {\it list}
consisting of the last {\it n} conses of {\it list}.  The {\it list} may
be a dotted list.  It is an error if the {\it list} is circular.

The argument {\it n} must be a non-negative integer.
If {\it n} is zero, then the atom that terminates the {\it list}
is returned.  If {\it n} is not less than the number of cons cells
making up the {\it list}, then the {\it list} itself is returned.

For example:
\begin{lisp}
(setq x '(a b c d)) \\
(last x) \EV\ (d) \\
(rplacd (last x) '(e f)) \\
x \EV\ '(a b c d e f) \\
(last x 3) \EV\ (d e f) \\
(last '()) \EV\ () \\
(last '(a b c . d)) \EV\ (c . d) \\
(last '(a b c . d) 0) \EV\ d \\
(last '(a b c . d) 2) \EV\ (b c . d) \\
(last '(a b c . d) 1729) \EV\ (a b c . d)
\end{lisp}
\end{defun}
\end{new}

\begin{defun}[Function]
list &rest args

\cd{list} constructs and returns a list of its arguments.
For example:
\begin{lisp}
(list 3 4 'a (car '(b . c)) (+ 6 -2)) \EV\ (3 4 a b 4)
\end{lisp}

\begin{newer}
\begin{lisp}
(list) \EV\ () \\
(list (list 'a 'b) (list 'c 'd 'e)) \EV\ ((a b) (c d e))
\end{lisp}
\end{newer}
\end{defun}

\begin{defun}[Function]
list* arg &rest others

\cd{list*} is like \cd{list} except that the last {\it cons}
of the constructed list is ``dotted.''  The last argument to \cd{list*}
is used as the {\it cdr} of the last cons constructed;
this need not be an atom.  If it is not an atom,
then the effect is to add several new elements to the front of a list.
For example:
\begin{lisp}
(list* 'a 'b 'c 'd) \EV\ (a b c . d)
\end{lisp}
This is like
\begin{lisp}
(cons 'a (cons 'b (cons 'c 'd)))
\end{lisp}
Also:
\begin{lisp}
(list* 'a 'b 'c '(d e f)) \EV\ (a b c d e f) \\*
(list* x) \EQ\ x
\end{lisp}
\end{defun}

\begin{defun}[Function]
make-list size &key :initial-element

This creates and returns a list containing {\it size} elements, each
of which is initialized to the \cd{:initial-element}
argument (which defaults to {\false}).
{\it size} should be a non-negative integer.
For example:
\begin{lisp}
(make-list 5) \EV\ ({\false} {\false} {\false} {\false} {\false}) \\
(make-list 3 \cd{:initial-element} 'rah) \EV\ (rah rah rah)
\end{lisp}
\end{defun}

\begin{defun}[Function]
append &rest lists

The arguments to \cd{append} are lists.  The result is a list that is the
concatenation of the arguments.
The arguments are not destroyed.
For example:
\begin{lisp}
(append '(a b c) '(d e f) '{\emptylist} '(g)) \EV\ (a b c d e f g)
\end{lisp}
Note that \cd{append} copies the top-level list structure of each of its
arguments {\it except} the last.
The function \cd{concatenate} can perform a similar operation, but always
copies all its arguments.  See also \cd{nconc}, which is like \cd{append}
but destroys all arguments but the last.

The last argument actually need not be a list but may be any Lisp object,
which becomes the tail end of the constructed list.
For example, \cd{(append '(a b c) 'd)} \EV\ \cd{(a b c . d)}.

\cd{(append {\it x} '{\emptylist})} is an idiom once frequently used to copy the
list {\it x}, but the \cd{copy-list} function is more appropriate to this
task.
\end{defun}

\begin{defun}[Function]
copy-list list

This returns a list that is \cd{equal} to {\it list}, but not \cd{eq}.
Only the top level of list structure is copied; that is, \cd{copy-list}
copies in the {\it cdr} direction but not in the {\it car} direction.
If the list is ``dotted,'' that is, \cd{(cdr (last {\it list}))}
is a non-{\nil} atom, this will be true of the returned list also.
See also \cd{copy-seq} and \cd{copy-tree}.
\end{defun}

\begin{defun}[Function]
copy-alist list

\cd{copy-alist} is for copying association lists.  The top level of
list structure of {\it list} is copied, just as for \cd{copy-list}.
In addition, each element of {\it list} that is a cons is replaced
in the copy by a new cons with the same {\it car} and {\it cdr}.
\end{defun}

\begin{defun}[Function]
copy-tree object

\cd{copy-tree} is for copying trees of conses.
The argument {\it object} may be any Lisp object.
If it is not a cons, it is returned; otherwise
the result is a new cons of the results of calling \cd{copy-tree}
on the {\it car} and {\it cdr} of the argument.  In other words,
all conses in the tree are copied recursively, stopping
only when non-conses are encountered.
Circularities and the sharing of substructure are {\it not} preserved.

\beforenoterule
\begin{incompatibility}
This function is called \cd{copy} in Interlisp.
\end{incompatibility}
\afternoterule
\end{defun}

\begin{defun}[Function]
revappend x y

\cd{(revappend {\it x} {\it y})} is exactly the same as 
\cd{(append (reverse {\it x}) {\it y})} except that it is potentially more
efficient.  Both {\it x} and {\it y} should be lists.
The argument {\it x} is copied, not destroyed.
Compare this with \cd{nreconc}, which destroys its first argument.
\end{defun}

\begin{defun}[Function]
nconc &rest lists

\cd{nconc} takes lists as arguments.  It returns a list that is the arguments
concatenated together.  The arguments are changed rather than copied.
(Compare this with \cd{append}, which copies arguments rather than
destroying them.)
For example:
\begin{lisp}
(setq x '(a b c)) \\
(setq y '(d e f)) \\
(nconc x y) \EV\ (a b c d e f) \\
x \EV\ (a b c d e f)
\end{lisp}
Note, in the example, that the value of \cd{x} is now different,
since its last cons has been \cd{rplacd}'d to the value of \cd{y}.
If one were then to evaluate \cd{(nconc x y)} again,
it would yield a piece of ``circular'' list
structure, whose printed representation would be
\cd{(a b c d e f d e f d e f ...)}, repeating forever;
if the \cd{*print-circle*} switch were non-{\nil},
it would be printed as \cd{(a b c . \#1=(d e f . \#1\#))}.

\begin{newer}
X3J13 voted in March 1989 \issue{REMF-DESTRUCTION-UNSPECIFIED}
to clarify the permissible side effects of certain operations.
The side-effect behavior of \cd{nconc} is specified by a recursive relationship
outlined in the following table, in which a call to \cd{nconc} matching
the earliest possible
pattern on the left is required to have side-effect behavior
equivalent to the corresponding expression on the right.
\begin{flushleft}
\begin{tabular}{@{}ll@{}}
\cd{(nconc)}&\cd{nil~~~~~;}{\rm No side effects} \\
\cd{(nconc nil . {\it r})~~~~}&\cd{(nconc . {\it r})} \\
\cd{(nconc {\it x})}&{\it x} \\
\cd{(nconc {\it x} {\it y})}&\cd{(let ((p {\it x}) (q {\it y}))} \\
                                  &\cd{~~(rplacd (last p) q)} \\
                                  &\cd{~~p)} \\
\cd{(nconc {\it x} {\it y} . {\it r})}&\cd{(nconc (nconc {\it x} {\it y}) . {\it
r})} \end{tabular}
\end{flushleft}
\end{newer}
\end{defun}

\penalty-10000 %manual

\begin{defun}[Function]
nreconc x y

\cd{(nreconc {\it x} {\it y})} is exactly the same as 
\cd{(nconc (nreverse {\it x}) {\it y})} except that it is potentially more
efficient.  Both {\it x} and {\it y} should be lists.
The argument {\it x} is destroyed.
Compare this with \cd{revappend}.

\begin{newer}
\begin{lisp}
(setq planets '(jupiter mars earth venus mercury)) \\
(setq more-planets '(saturn uranus pluto neptune)) \\
(nreconc more-planets planets) \\
\`\EV\ (neptune pluto uranus saturn jupiter mars earth venus mercury) \\
~~{\rm and now the value of \cd{more-planets} is not well defined}
\end{lisp}
\end{newer}

\begin{newer}
X3J13 voted in March 1989 \issue{REMF-DESTRUCTION-UNSPECIFIED}
to clarify the permissible side effects of certain operations;
\cd{(nreconc {\it x} {\it y})} is permitted and
required to have side-effect behavior
equivalent to that of \cd{(nconc (nreverse {\it x})~{\it y})}.
\end{newer}

\end{defun}


\begin{defmac}
push item place

The form {\it place} should be the name of a generalized variable
containing a list; {\it item} may refer to any Lisp object.  The {\it item}
is consed onto the front of the list, and the augmented list is stored
back into {\it place} and returned.
The form {\it place} may be any form acceptable as a
generalized variable to \cd{setf}.  If the list held in {\it place} is
viewed as a push-down stack, then \cd{push} pushes an element onto the top
of the stack.
For example:
\begin{lisp}
(setq x '(a (b c) d)) \\
(push 5 (cadr x)) \EV\ (5 b c)  {\rm and now} x \EV\ (a (5 b c) d)
\end{lisp}
The effect of \cd{(push {\it item} {\it place})}
is roughly equivalent to
\begin{lisp}
(setf {\it place} (cons {\it item} {\it place}))
\end{lisp}
except that the latter would evaluate any subforms of {\it place}
twice, while \cd{push} takes care to evaluate them only once.
Moreover, for certain {\it place} forms \cd{push} may be
significantly more efficient than the \cd{setf} version.
\begin{newer}
X3J13 voted in March 1988 \issue{PUSH-EVALUATION-ORDER}
to clarify order of evaluation (see section~\ref{SETF-SECTION}).
Note that {\it item} is fully evaluated before any part of {\it place}
is evaluated.
\end{newer}
\end{defmac}

\begin{defmac}
pushnew item place \hbox\bgroup\cf &key :test :test-not :key\egroup

The form {\it place} should be the name of a generalized variable
containing a list; {\it item} may refer to any Lisp object.  If the
{\it item} is not already a member of the list (as determined by
comparisons using the \cd{:test} predicate, which defaults to \cd{eql}),
then the {\it item} is consed onto the front of the list, and
the augmented list is stored back into {\it place} and returned; otherwise
the unaugmented list is returned.  The form {\it place} may be
any form acceptable as a generalized variable to \cd{setf}.  If the
list held in {\it place} is viewed as a set, then \cd{pushnew} adjoins an
element to the set; see \cd{adjoin}.

The keyword arguments to \cd{pushnew}
follow the conventions for the generic sequence
functions.  See chapter~\ref{KSEQUE}.
In effect, these keywords are simply passed on to the \cd{adjoin} function.

\cd{pushnew} returns the new contents of the {\it place}.
For example:
\begin{lisp}
(setq x '(a (b c) d)) \\
(pushnew 5 (cadr x)) \EV\ (5 b c)   {\rm and now} x \EV\ (a (5 b c) d) \\
(pushnew 'b (cadr x)) \EV\ (5 b c)  {\rm and \cd{x} is unchanged}
\end{lisp}
The effect of
\begin{lisp}
(pushnew {\it item} {\it place} \cd{:test} {\it p})
\end{lisp}
is roughly equivalent to
\begin{lisp}
(setf {\it place} (adjoin {\it item} {\it place} \cd{:test} {\it p}))
\end{lisp}
except that the latter would evaluate any subforms of
{\it place} twice, while \cd{pushnew} takes care to evaluate them only once.
Moreover, for certain {\it place} forms \cd{pushnew} may be
significantly more efficient than the \cd{setf} version.
\begin{newer}
X3J13 voted in March 1988 \issue{PUSH-EVALUATION-ORDER}
to clarify order of evaluation (see section~\ref{SETF-SECTION}).
Note that {\it item} is fully evaluated before any part of {\it place}
is evaluated.
\end{newer}
\end{defmac}

\begin{defmac}
pop place

The form {\it place} should be the name of a generalized variable
containing a list.  The result of \cd{pop} is the \cd{car} of the contents
of {\it place}, and as a side effect the \cd{cdr} of the contents is stored
back into {\it place}.  The form {\it place} may be any form acceptable as a
generalized variable to \cd{setf}.  If the list held in {\it place} is
viewed as a push-down stack, then \cd{pop} pops an element from the top of
the stack and returns it.
For example:
\begin{lisp}
(setq stack '(a b c)) \\
(pop stack) \EV\ a  {\rm and now} stack \EV\ (b c)
\end{lisp}
The effect of \cd{(pop {\it place})} is roughly equivalent to
\begin{lisp}
(prog1 (car {\it place}) (setf {\it place} (cdr {\it place})))
\end{lisp}
except that the latter would evaluate any subforms of {\it place}
three times, while \cd{pop} takes care to evaluate them only once.
Moreover, for certain {\it place} forms \cd{pop} may be
significantly more efficient than the \cd{setf} version.
\begin{newer}
X3J13 voted in March 1988 \issue{PUSH-EVALUATION-ORDER}
to clarify order of evaluation (see section~\ref{SETF-SECTION}).
\end{newer}
\end{defmac}

\begin{defun}[Function]
butlast list &optional n

This creates and returns a list with the same elements as {\it list},
excepting the last {\it n} elements.
{\it n} defaults to 1.  The argument is not destroyed.
If the {\it list} has fewer than {\it n} elements, then {\emptylist} is returned.
For example:
\begin{lisp}
(butlast '(a b c d)) \EV\ (a b c) \\
(butlast '((a b) (c d))) \EV\ ((a b)) \\
(butlast '(a)) \EV\ {\emptylist} \\
(butlast nil) \EV\ {\emptylist}
\end{lisp}
The name is from the phrase ``all elements but the last.''
\end{defun}

\begin{defun}[Function]
nbutlast list &optional n

This is the destructive version of \cd{butlast}; it changes the {\it cdr} of
the cons {\it n}+1 from the end of the {\it list} to {\nil}.  {\it n} defaults to 1.
If the {\it list} has fewer than {\it n} elements, then \cd{nbutlast}
returns {\emptylist}, and the argument is not modified.  (Therefore
one normally writes \cd{(setq a (nbutlast a))} rather than simply
\cd{(nbutlast a)}.)
For example:
\begin{lisp}
(setq foo '(a b c d)) \\
(nbutlast foo) \EV\ (a b c) \\
foo \EV\ (a b c) \\
(nbutlast '(a)) \EV\ {\emptylist} \\
(nbutlast '{\nil}) \EV\ {\emptylist}
\end{lisp}
\end{defun}


\begin{defun}[Function]
ldiff list sublist

{\it list} should be a list, and {\it sublist} should be a sublist
of {\it list}, that is, one of the conses that make up {\it list}.
\cd{ldiff} (meaning ``list difference'') will return a new (freshly consed)
list, whose elements are those elements of {\it list} that appear before
{\it sublist}.  If {\it sublist} is not a tail of {\it list}
(and in particular if {\it sublist} is {\nil}),
then a copy of the entire {\it list} is returned.
The argument {\it list} is not destroyed.
For example:
\begin{lisp}
(setq x '(a b c d e)) \\
(setq y (cdddr x)) \EV\ (d e) \\
(ldiff x y) \EV\ (a b c) \\[4pt]
{\rm but} (ldiff '(a b c d) '(c d)) \EV\ (a b c d)
\end{lisp}
since the sublist was not \cd{eq} to any part of the list.
\end{defun}


\section{Alteration of List Structure}

The functions \cd{rplaca} and \cd{rplacd}
may be used to make alterations in already existing
list structure, that is, to change the {\it car} or {\it cdr} of an
existing cons.
One may also use \cd{setf} in conjunction with \cd{car} and \cd{cdr}.

The structure is not copied but is destructively altered;
hence caution should be exercised when using these functions, as
strange side effects can occur if portions of list structure become
shared.
The \cd{nconc}, \cd{nreverse}, \cd{nreconc},
and \cd{nbutlast} functions, already
described,
have the same property, as do certain of the generic sequence
functions such as \cd{delete}.
However, they are normally not
used for this side effect; rather, the list-structure modification
is purely for efficiency, and compatible non-modifying functions
are provided.

\begin{defun}[Function]
rplaca x y

\cd{(rplaca {\it x} {\it y})} changes the {\it car} of {\it x} to {\it y} and returns
(the modified) {\it x}.  {\it x} must be a cons, but {\it y} may be any
Lisp object.
For example:
\begin{lisp}
(setq g '(a b c)) \\
(rplaca (cdr g) 'd) \EV\ (d c) \\
{\rm Now} g \EV\ (a d c)
\end{lisp}
\end{defun}

\begin{defun}[Function]
rplacd x y

\cd{(rplacd {\it x} {\it y})} changes the {\it cdr} of {\it x} to {\it y} and returns
(the modified) {\it x}.  {\it x} must be a cons, but {\it y} may be
any Lisp object.
For example:
\begin{lisp}
(setq x '(a b c)) \\
(rplacd x 'd) \EV\ (a . d) \\
{\rm Now} x \EV\ (a . d)
\end{lisp}
\end{defun}

\begin{newer}
The functions \cd{rplaca} and \cd{rplacd} go back to the earliest
origins of Lisp, along with \cd{car}, \cd{cdr}, and \cd{cons}.
Nowadays, however, they seem to be falling by the wayside.
More and more Common Lisp programmers use \cd{setf} for nearly
all structure modifications: \cd{(rplaca x~y)} is rendered
as \cd{(setf (car x)~y)} or perhaps as \cd{(setf (first x)~y)}.
Even more likely is that a \cd{defstruct} structure or a CLOS class
is used in place of a list, if the data structure is at all complicated;
in this case \cd{setf} is used with a slot accessor.
\end{newer}
\newpage%manual

\section{Substitution of Expressions}

\indexterm{substitution}
A number of functions are provided for performing substitutions
within a tree.  All take a tree and a description
of old subexpressions to be replaced by new ones.
They come in non-destructive and destructive varieties
and specify substitution either by two arguments or by an association list.

The naming conventions for these functions and for their keyword
arguments generally follow the conventions for the generic sequence
functions.  See chapter~\ref{KSEQUE}.

\begin{defun}[Function]
subst new old tree &key :test :test-not :key \\
subst-if new test tree &key :key \\
subst-if-not new test tree &key :key

\cd{(subst {\it new} {\it old} {\it tree})} makes a copy of {\it tree},
substituting {\it new} for every subtree or leaf of {\it tree}
(whether the subtree or leaf is a {\it car} or a {\it cdr} of its parent)
such that {\it old} and the subtree or leaf satisfy the test.  It
returns the modified copy of {\it tree}.  The original {\it tree} is
unchanged, but the result tree may share with parts of the argument
{\it tree}.

\beforenoterule
\begin{incompatibility}
In MacLisp, \cd{subst} is guaranteed {\it not} to share with
the {\it tree} argument, and the idiom \cd{(subst {\nil} {\nil} x)} was
used to copy a tree \cd{x}.  In Common Lisp, the function \cd{copy-tree} should
be used to copy a tree, as the \cd{subst} idiom will not work.
\end{incompatibility}
\afternoterule

For example:
\begin{lisp}
(subst 'tempest 'hurricane \\
~~~~~~~'(shakespeare wrote (the hurricane))) \\
~~~\EV\ (shakespeare wrote (the tempest)) \\
\\
(subst 'foo '{\nil} '(shakespeare wrote (twelfth night))) \\
~~~\EV\ (shakespeare wrote (twelfth night . foo) . foo) \\
\\
(subst '(a . cons) '(old . pair) \\
~~~~~~~'((old . spice) ((old . shoes) old . pair) (old . pair)) \\
~~~~~~~\cd{:test} \#'equal) \\
~~~\EV\ ((old . spice) ((old . shoes) a . cons) (a . cons))
\end{lisp}
This function is not destructive; that is, it does not change
the {\it car} or {\it cdr} of any already existing list structure.
One possible definition of \cd{subst}:
\begin{lisp}
(defun subst (old new tree \cd{\&rest} x \cd{\&key} test test-not key) \\*
~~(cond ((satisfies-the-test old tree :test test \\*
~~~~~~~~~~~~~~~~~~~~~~~~~~~~~:test-not test-not :key key) \\*
~~~~~~~~~new) \\*
~~~~~~~~((atom tree) tree) \\
~~~~~~~~(t (let ((a (apply \#'subst old new (car tree) x)) \\*
~~~~~~~~~~~~~~~~~(d (apply \#'subst old new (cdr tree) x))) \\
~~~~~~~~~~~~~(if (and (eql a (car tree)) \\*
~~~~~~~~~~~~~~~~~~~~~~(eql d (cdr tree))) \\*
~~~~~~~~~~~~~~~~~tree \\*
~~~~~~~~~~~~~~~~~(cons a d))))))
\end{lisp}
See also \cd{substitute}, which substitutes for top-level elements
of a sequence.

\begin{new}
X3J13 voted in January 1989
\issue{MAPPING-DESTRUCTIVE-INTERACTION}
to restrict user side effects; see section~\ref{STRUCTURE-TRAVERSAL-SECTION}.
\end{new}
\end{defun}

\begin{defun}[Function]
nsubst new old tree &key :test :test-not :key \\
nsubst-if new test tree &key :key \\
nsubst-if-not new test tree &key :key

\cd{nsubst} is a destructive version of \cd{subst}.  The list structure of
{\it tree} is altered by destructively replacing with {\it new}
each leaf or subtree of the {\it tree} such that {\it old} and the leaf
or subtree satisfy the test.

\begin{new}
X3J13 voted in January 1989
\issue{MAPPING-DESTRUCTIVE-INTERACTION}
to restrict user side effects; see section~\ref{STRUCTURE-TRAVERSAL-SECTION}.
\end{new}
\end{defun}

\begin{defun}[Function]
sublis alist tree &key :test :test-not :key

\cd{sublis} makes substitutions for objects in a tree
(a structure of conses).
The first argument to \cd{sublis} is an association list.
The second argument is the tree in which
substitutions are to be made, as for \cd{subst}.
\cd{sublis} looks at all subtrees and leaves of the tree;
if a subtree or leaf appears as a key in the association
list (that is, the key and the subtree or leaf satisfy the test),
it is replaced by the object with which it is associated.
This operation is non-destructive.  In effect, \cd{sublis} can
perform several \cd{subst} operations simultaneously.
For example:
\begin{lisp}
(sublis '((x . 100) (z . zprime)) \\*
~~~~~~~~'(plus x (minus g z x p) 4 . x)) \\*
~~~\EV\ (plus 100 (minus g zprime 100 p) 4 . 100) \\*
 \\*
(sublis '(((+ x y) . (- x y)) ((- x y) . (+ x y))) \\*
~~~~~~~~'(* (/ (+ x y) (+ x p)) (- x y)) \\*
~~~~~~~~:test \#'equal) \\*
~~~\EV\ (* (/ (- x y) (+ x p)) (+ x y))
\end{lisp}

\begin{new}
X3J13 voted in January 1989
\issue{MAPPING-DESTRUCTIVE-INTERACTION}
to restrict user side effects; see section~\ref{STRUCTURE-TRAVERSAL-SECTION}.
\end{new}
\end{defun}

\begin{defun}[Function]
nsublis alist tree &key :test :test-not :key

\cd{nsublis} is like \cd{sublis} but destructively modifies the relevant
parts of the {\it tree}.

\begin{new}
X3J13 voted in January 1989
\issue{MAPPING-DESTRUCTIVE-INTERACTION}
to restrict user side effects; see section~\ref{STRUCTURE-TRAVERSAL-SECTION}.
\end{new}
\end{defun}

\section{Using Lists as Sets}

Common Lisp includes functions that allow a list of items to be
treated as a {\it set}.
There are functions to add, remove, and search for items in a list,
based on various criteria.
There are also set union, intersection, and difference functions.

The naming conventions for these functions and for their keyword
arguments generally follow the conventions that apply to the generic sequence
functions.  See chapter~\ref{KSEQUE}.

\begin{defun}[Function]
member item list &key :test :test-not :key \\
member-if predicate list &key :key \\
member-if-not predicate list &key :key

The {\it list} is searched for an element that satisfies the test.
If none is found, {\false} is returned;
otherwise, the tail of {\it list} beginning
with the first element that satisfied the test is returned.
The {\it list} is searched on the top level only. 
These functions are suitable for use as predicates.

For example:
\begin{lisp}
(member 'snerd '(a b c d)) \EV\ {\false} \\
(member-if \#'numberp '(a \#{\Xbackslash}Space 5/3 foo)) \EV\ (5/3 foo) \\
(member 'a '(g (a y) c a d e a f)) \EV\ (a d e a f)
\end{lisp}
Note, in the last example,
that the value returned by \cd{member} is \cd{eq} to the portion of the list
beginning with \cd{a}.
Thus \cd{rplaca} on the result of \cd{member} may be used
to alter the found list element,
if a check is first made that \cd{member} did not return {\false}.

See also \cd{find} and \cd{position}.

\begin{new}
X3J13 voted in January 1989
\issue{MAPPING-DESTRUCTIVE-INTERACTION}
to restrict user side effects; see section~\ref{STRUCTURE-TRAVERSAL-SECTION}.
\end{new}

\beforenoterule
\begin{incompatibility}
In MacLisp, the \cd{member} function uses
an \cd{equal} comparison rather than \cd{eql}, which is the default
test for \cd{member} in Common Lisp.  Where in MacLisp one would write
\cd{(member x y)}, in Common Lisp one must write \cd{(member x y :test \#'equal)}
to get a completely identical effect.  Similarly, one can get the
precise effect, and no more, of the MacLisp \cd{(memq x y)}
by writing in Common Lisp \cd{(member x y :test \#'eq)}.
\end{incompatibility}
\afternoterule
\end{defun}

\begin{defun}[Function]
tailp sublist list

This predicate is true if {\it sublist} is a sublist of {\it list} (that is,
one of the conses that makes up {\it list}); otherwise it is false.
Another way to look at this is that \cd{tailp} is true if
\cd{(nthcdr {\it n} {\it list})} is {\it sublist}, for some value of {\it n}.
See \cd{ldiff}.

\begin{new}
X3J13 voted in January 1989
\issue{TAILP-NIL}
to strike the parenthetical remark that suggests
that the {\it sublist} must be a cons, to clarify that \cd{tailp}
is true if and only if there exists an integer {\it n} such that
\begin{lisp}
(eql {\it sublist} (nthcdr {\it n} {\it list}))
\end{lisp}
and to specify that {\it list} may be a dotted list (implying that
implementations must use \cd{atom} and not \cd{endp} to check for
the end of the {\it list}).
\end{new}
\end{defun}

\begin{defun}[Function]
adjoin item list &key :test :test-not :key

\cd{adjoin} is used to add an element to a set, provided that
it is not already a member.  The equality test defaults to \cd{eql}.
\begin{lisp}
(adjoin {\it item} {\it list}) \EQ\ (if (member {\it item} {\it list}) {\it list} (cons {\it item} {\it list}))
\end{lisp}
In general, the test may be any predicate; the {\it item} is added to the
list only if there is no element of the list that ``satisfies the test.''

\cd{adjoin} deviates from the usual rules described in chapter~\ref{KSEQUE}
for the treatment of arguments named {\it item} and \cd{:key}.
If a \cd{:key} function is specified, it is applied to {\it item}
as well as to each element of the list.  The rationale is that
if the {\it item} is not yet in the list, it soon will be, and so
the test is more properly viewed as being between two elements
rather than between a separate {\it item} and an element.
\begin{lisp}
(adjoin {\it item} {\it list} :key {\it fn}) \\
~~\EQ\ (if (member (funcall {\it fn} {\it item}) {\it list}
  :key {\it fn}) {\it list} (cons {\it item} {\it list})) \end{lisp}
See \cd{pushnew}.
\begin{new}%CORR
{\it Notice of correction.}
In the first edition, the form \cd{({\it fn} {\it item})} appeared
in this example without the required \cd{funcall}.
\end{new}

\begin{new}
X3J13 voted in January 1989
\issue{MAPPING-DESTRUCTIVE-INTERACTION}
to restrict user side effects; see section~\ref{STRUCTURE-TRAVERSAL-SECTION}.
\end{new}
\end{defun}

\begin{defun}[Function]
union list1 list2 &key :test :test-not :key \\
nunion list1 list2 &key :test :test-not :key

\cd{union} takes two lists and returns a new list containing
everything that is an element of either of the {\it lists}.
If there is a duplication between two lists,
only one of the duplicate instances will be in the result.
If either of the arguments has duplicate entries within it,
the redundant entries
may or may not appear in the result.
For example:
\begin{lisp}
(union '(a b c) '(f a d)) \\
~~~\EV\ (a b c f d) {\rm or} (b c f a d) {\rm or} (d f a b c) {\rm or} ... \\
 \\
(union '((x 5) (y 6)) '((z 2) (x 4)) :key \#'car) \\
~~~\EV\ ((x 5) (y 6) (z 2)) {\rm or} ((x 4) (y 6) (z 2)) {\rm or} ...
\end{lisp}

There is no guarantee that the order of elements in the result will
reflect the ordering of the arguments in any particular way.
The implementation is therefore free to use any of a variety of strategies.
The result list may share cells with, or be \cd{eq} to, either of the arguments
if appropriate.

In general, the test may be any predicate, and the union operation may be
described as follows.  For all possible ordered pairs consisting of one
element from {\it list1} and one element from {\it list2}, the test is used
to determine whether they ``match.''  For every matching pair, at least
one of the two elements of the pair will be in the result.  Moreover, any
element from either list that matches no element of the other will appear
in the result.  All this is very general, but probably not particularly
useful unless the test is an equivalence relation.

The \cd{:test-not} argument can be useful when the test function
is the logical negation of an equivalence test.  A good example
of this is the function \cd{mismatch}, which is logically inverted
so that possibly useful information can be returned if the arguments do not
match.  This additional ``useful information'' is discarded in the following
example; \cd{mismatch} is used purely as a predicate.
\begin{lisp}
(union '(\#(a b) \#(5 0 6) \#(f 3)) \\
~~~~~~~'(\#(5 0 6) (a b) \#(g h)) \\
~~~~~~~:test-not \\
~~~~~~~\#'mismatch) \\
~~~\EV\ (\#(a b) \#(5 0 6) \#(f 3) \#(g h))~~~~~;{\rm One possible result} \\
~~~\EV\ ((a b) \#(f 3) \#(5 0 6) \#(g h))~~~~~~;{\rm Another possible result}
\end{lisp}
Using \cd{\cd{:test-not} \#'mismatch} differs from using
\cd{\cd{:test} \#'equalp}, for example, because \cd{mismatch}
will determine that \cd{\#(a b)} and \cd{(a b)} are the same,
while \cd{equalp} would regard them as not the same.

\cd{nunion} is the destructive version of \cd{union}.
It performs the same operation but may destroy the argument lists,
perhaps in order to use their cells to construct the result.

\begin{new}
X3J13 voted in January 1989
\issue{MAPPING-DESTRUCTIVE-INTERACTION}
to restrict user side effects; see section~\ref{STRUCTURE-TRAVERSAL-SECTION}.
\end{new}

\begin{newer}
X3J13 voted in March 1989 \issue{REMF-DESTRUCTION-UNSPECIFIED}
to clarify the permissible side effects of certain operations;
\cd{nunion} is permitted to perform a \cd{setf} on any part,
{\it car} or {\it cdr}, of the top-level list structure of 
any of the argument lists.
\end{newer}
\end{defun}

\begin{defun}[Function]
intersection list1 list2 &key :test :test-not :key \\
nintersection list1 list2 &key :test :test-not :key

\cd{intersection} takes two lists and returns a new list containing
everything that is an element of both argument lists.
If either list has duplicate entries, the redundant entries
may or may not appear in the result.
For example:
\begin{lisp}
(intersection '(a b c) '(f a d)) \EV\ (a)
\end{lisp}
There is no guarantee that the order of elements in the result will
reflect the ordering of the arguments in any particular way.
The implementation is therefore free to use any of a variety of strategies.
The result list may share cells with, or be \cd{eq} to, either of the arguments
if appropriate.

In general, the test may be any predicate, and the intersection operation
may be described as follows.  For all possible ordered pairs consisting of
one element from {\it list1} and one element from {\it list2}, the test is
used to determine whether they ``match.''  For every matching pair,
exactly one of the two elements of the pair will be put in the result.
No element from either list appears in the result that does not match
an element from the other list.
All this is very general, but probably
not particularly useful unless the test is an equivalence relation.

\cd{nintersection} is the destructive version of \cd{intersection}.
It performs the same operation, but may destroy {\it list1},
perhaps in order to use its cells to construct the result.
(The argument {\it list2} is {\it not} destroyed.)

\begin{new}
X3J13 voted in January 1989
\issue{MAPPING-DESTRUCTIVE-INTERACTION}
to restrict user side effects; see section~\ref{STRUCTURE-TRAVERSAL-SECTION}.
\end{new}

\begin{newer}
X3J13 voted in March 1989 \issue{REMF-DESTRUCTION-UNSPECIFIED}
to clarify the permissible side effects of certain operations;
\cd{nintersection} is permitted to perform a \cd{setf} on any part,
{\it car} or {\it cdr}, of the top-level list structure of 
any of the argument lists.
\end{newer}
\end{defun}

\begin{defun}[Function]
set-difference list1 list2 &key :test :test-not :key \\
nset-difference list1 list2 &key :test :test-not :key

\cd{set-difference} returns a list of elements of {\it list1}
that do not appear in {\it list2}.  This operation is
not destructive.

There is no guarantee that the order of elements in the result will
reflect the ordering of the arguments in any particular way.
The implementation is therefore free to use any of a variety of strategies.
The result list may share cells with, or be \cd{eq} to, either of the arguments
if appropriate.

In general, the test may be any predicate, and the set difference operation
may be described as follows.  For all possible ordered pairs consisting of
one element from {\it list1} and one element from {\it list2}, the test is
used to determine whether they ``match.''  An element of {\it list1}
appears in the result if and only if it does not match any element
of {\it list2}.  This is very general and permits interesting applications.
For example, one can remove from a list of strings all those strings
containing one of a given list of characters:
\begin{lisp}
;; Remove all flavor names that contain "c" or "w". \\
(set-difference '("strawberry" "chocolate" "banana" \\
~~~~~~~~~~~~~~~~~~"lemon" "pistachio" "rhubarb") \\
~~~~~~~~~~~~~~~~'(\#{\Xbackslash}c \#{\Xbackslash}w) \\
~~~~~~~~~~~~~~~~:test \\
~~~~~~~~~~~~~~~~\#'(lambda (s c) (find c s))) \\
~~~\EV\ ("banana" "rhubarb" "lemon")~~~~~;{\rm One possible ordering}
\end{lisp}

\cd{nset-difference} is the destructive version of \cd{set-difference}.
This operation may destroy {\it list1}.

\begin{new}
X3J13 voted in January 1989
\issue{MAPPING-DESTRUCTIVE-INTERACTION}
to restrict user side effects; see section~\ref{STRUCTURE-TRAVERSAL-SECTION}.
\end{new}

\beforenoterule
\begin{incompatibility}
An approximately equivalent Interlisp function
is \cd{ldifference}.
\end{incompatibility}
\afternoterule
\end{defun}

\begin{defun}[Function]
set-exclusive-or list1 list2 &key :test :test-not :key \\
nset-exclusive-or list1 list2 &key :test :test-not :key

\cd{set-exclusive-or} returns a list of elements that appear
in exactly one of {\it list1} and {\it list2}.
This operation is not destructive.

There is no guarantee that the order of elements in the result will
reflect the ordering of the arguments in any particular way.
The implementation is therefore free to use any of a variety of strategies.
The result list may share cells with, or be \cd{eq} to, either of the arguments
if appropriate.

In general, the test may be any predicate, and the set-exclusive-or operation
may be described as follows.  For all possible ordered pairs consisting of
one element from {\it list1} and one element from {\it list2}, the test is
used to determine whether they ``match.''  The result contains precisely
those elements of {\it list1} and {\it list2} that appear in no matching pair.

\newpage%manual

\cd{nset-exclusive-or} is the destructive version of \cd{set-exclusive-or}.
Both lists may be destroyed in producing the result.

\begin{new}
X3J13 voted in January 1989
\issue{MAPPING-DESTRUCTIVE-INTERACTION}
to restrict user side effects; see section~\ref{STRUCTURE-TRAVERSAL-SECTION}.
\end{new}

\begin{newer}
X3J13 voted in March 1989 \issue{REMF-DESTRUCTION-UNSPECIFIED}
to clarify the permissible side effects of certain operations;
\cd{nset-exclusive-or} is permitted to perform a \cd{setf} on any part,
{\it car} or {\it cdr}, of the top-level list structure of 
any of the argument lists.
\end{newer}
\end{defun}

\begin{defun}[Function]
subsetp list1 list2 &key :test :test-not :key

\cd{subsetp} is a predicate that is true if every element of {\it list1}
appears in (``matches'' some element of) {\it list2}, and false otherwise.

\begin{new}
X3J13 voted in January 1989
\issue{MAPPING-DESTRUCTIVE-INTERACTION}
to restrict user side effects; see section~\ref{STRUCTURE-TRAVERSAL-SECTION}.
\end{new}
\end{defun}


\section{Association Lists}

An {\it association list}, or {\it a-list}, is a data structure
used very frequently in Lisp.  An a-list is a list of pairs (conses);
each pair is an association.  The {\it car} of a pair is called the {\it key},
and the {\it cdr} is called the {\it datum}.

An advantage of the a-list representation is that an a-list can be
incrementally augmented simply by adding new entries to the front.
Moreover, because the searching function \cd{assoc} searches the
a-list in order, new entries can ``shadow'' old entries.  If an a-list is
viewed as a mapping from keys to data, then the mapping can be not only
augmented but also altered in a non-destructive manner by adding new
entries to the front of the a-list.

Sometimes an a-list represents a bijective mapping, and it is desirable
to retrieve a key given a datum.  For this purpose, the ``reverse'' searching
function \cd{rassoc} is provided.  Other variants of a-list searches
can be constructed using the function \cd{find} or \cd{member}.

It is permissible to let {\false} be an element of an a-list in place of
a pair.  Such an element is not considered to be a pair but is simply
passed over when the a-list is searched by \cd{assoc}.

\begin{defun}[Function]
acons key datum a-list

\cd{acons} constructs a new association list by adding the pair
\cd{({\it key} . {\it datum})} to the old {\it a-list}.
\begin{lisp}
(acons x y a) \EQ\ (cons (cons x y) a)
\end{lisp}
\begin{newer}
This is a trivial convenience function, but I find I use it a lot.
\end{newer}
\end{defun}

\newpage%manual

\begin{defun}[Function]
pairlis keys data &optional a-list

\cd{pairlis} takes two lists and makes an association list that associates
elements of the first list to corresponding elements of the second
list.  It is an error if the two lists {\it keys} and {\it data} are not of
the same length.  If the optional argument {\it a-list} is provided, then the
new pairs are added to the front of it.

The new pairs may appear in the resulting a-list in any order;
in particular, either forward or backward order is permitted.
Therefore the result of the call
\begin{lisp}
(pairlis '(one two) '(1 2) '((three . 3) (four . 19)))
\end{lisp}
might be
\begin{lisp}
((one . 1) (two . 2) (three . 3) (four . 19))
\end{lisp}
but could equally well be
\begin{lisp}
((two . 2) (one . 1) (three . 3) (four . 19))
\end{lisp}
\end{defun}

\begin{defun}[Function]
assoc item a-list &key :test :test-not :key \\
assoc-if predicate a-list \\
assoc-if-not predicate a-list

\begin{new}
X3J13 voted in March 1988
\issue{ASSOC-RASSOC-IF-KEY}
to allow \cd{assoc-if} and \cd{assoc-if-not}
also to take a keyword argument named \cd{:key}, to be used
to determine whether a pair ``satisfies the test'' in the same manner as
for sequence functions.  The new function descriptions are therefore as follows:

\begin{defun}[Function]
assoc-if predicate a-list &key :key \\
assoc-if-not predicate a-list &key :key

\end{defun}
The omission of \cd{:key} arguments for these
functions in the first edition was probably an oversight.
\end{new}

Each of these searches the association list
{\it a-list}.  The value is the first pair in the a-list such that
the {\it car} of the pair satisfies the test, or {\false} if there is
no such pair in the a-list.
For example:
\begin{lisp}
(assoc 'r '((a . b) (c . d) (r . x) (s . y) (r . z))) \\
~~~~~~~~\EV\  (r . x) \\
(assoc 'goo '((foo . bar) (zoo . goo))) \EV\ {\false} \\
(assoc '2 '((1 a b c) (2 b c d) (-7 x y z))) \EV\ (2 b c d)
\end{lisp}
It is possible to \cd{rplacd} the result of \cd{assoc} {\it provided}
that it is not {\false},
in order to ``update'' the ``table'' that was \cd{assoc}'s second argument.
(However, it is often better to update an a-list by adding new pairs
to the front, rather than altering old pairs.)
For example:
\begin{lisp}
(setq values '((x . 100) (y . 200) (z . 50))) \\
(assoc 'y values) \EV\ (y . 200) \\
(rplacd (assoc 'y values) 201) \\
(assoc 'y values) \EV\ (y . 201) {\rm now}
\end{lisp}
A typical trick is to say
\cd{(cdr (assoc x y))}.
Because the {\it cdr} of {\false} is guaranteed to be {\false},
this yields {\false} if no pair is found {\it or} if a pair is
found whose {\it cdr} is {\false}.  This is useful if {\false} serves
its usual role as a ``default value.''

The two expressions
\begin{lisp}
(assoc {\it item} {\it list} :test {\it fn})
\end{lisp}
and
\begin{lisp}
(find {\it item} {\it list} :test {\it fn} :key \#'car)
\end{lisp}
are equivalent in meaning with one important exception:
if {\nil} appears in the a-list in place of a pair,
and the {\it item} being searched for is {\nil},
\cd{find} will blithely compute the {\it car} of the {\nil} in the a-list,
find that it is equal to the {\it item}, and return {\nil},
whereas \cd{assoc} will ignore the {\nil} in the a-list and continue
to search for an actual pair (cons) whose {\it car} is {\nil}.
See \cd{find} and \cd{position}.

\begin{new}
X3J13 voted in January 1989
\issue{MAPPING-DESTRUCTIVE-INTERACTION}
to restrict user side effects; see section~\ref{STRUCTURE-TRAVERSAL-SECTION}.
\end{new}

\beforenoterule
\begin{incompatibility}
In MacLisp, the \cd{assoc} function uses
an \cd{equal} comparison rather than \cd{eql}, which is the default
test for \cd{assoc} in Common Lisp.  Where in MacLisp one would write
\cd{(assoc x y)}, in Common Lisp one must write \cd{(assoc x y :test \#'equal)}
to get the completely identical effect.  Similarly, one can get the
precise effect, and no more, of the MacLisp \cd{(assq x y)}
by writing in Common Lisp \cd{(assoc x y :test \#'eq)}.

In Interlisp, \cd{assoc} uses an \cd{eq} test, and \cd{sassoc}
uses an Interlisp \cd{equal} test.
\end{incompatibility}
\afternoterule
\end{defun}

\begin{defun}[Function]
rassoc item a-list &key :test :test-not :key \\
rassoc-if predicate a-list \\
rassoc-if-not predicate a-list

\begin{new}
X3J13 voted in March 1988
\issue{ASSOC-RASSOC-IF-KEY}
to allow \cd{rassoc-if} and \cd{rassoc-if-not}
also to take a keyword argument named \cd{:key}, to be used
to determine whether a pair ``satisfies the test'' in the same manner as
for sequence functions.  The new function descriptions are therefore as follows:

\begin{defun}[Function]
rassoc-if predicate a-list &key :key \\
rassoc-if-not predicate a-list &key :key

\end{defun}
The omission of \cd{:key} arguments for these
functions in the first edition was probably an oversight.
\end{new}

\cd{rassoc} is the reverse form of \cd{assoc}; it searches for
a pair whose {\it cdr} satisfies the test, rather than the {\it car}.
If the {\it a-list} is considered to be a mapping, then \cd{rassoc}
treats the {\it a-list} as representing the inverse mapping.
For example:
\begin{lisp}
(rassoc 'a '((a . b) (b . c) (c . a) (z . a))) \EV\ (c . a)
\end{lisp}

The expressions
\begin{lisp}
(rassoc {\it item} {\it list} :test {\it fn})
\end{lisp}
and
\begin{lisp}
(find {\it item} {\it list} :test {\it fn} :key \#'cdr)
\end{lisp}
are equivalent in meaning, except when the {\it item} is {\nil}
and {\nil} appears in place of a pair in the a-list.  See the discussion
of the function \cd{assoc}.

\begin{new}
X3J13 voted in January 1989
\issue{MAPPING-DESTRUCTIVE-INTERACTION}
to restrict user side effects; see section~\ref{STRUCTURE-TRAVERSAL-SECTION}.
\end{new}
\end{defun}
