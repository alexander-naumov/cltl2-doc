%Part{String, Root = "CLM.MSS"}
%%%Chapter of Common Lisp Manual.  Copyright 1984, 1988, 1989 Guy L. Steele Jr.


\chapter{Strings Строки}
\def\pagestatus{FINAL PROOF}

A string is a specialized vector (one-dimensional array)
whose elements are characters.

Строка является специализированным вектором (или одномерным массивом), элементы
которого --- строковые символы.

\begin{obsolete}
Specifically, the type \cdf{string}
is identical to the type \cd{(vector string-char)}, which in turn
is the same as \cd{(array string-char (*))}.
\end{obsolete}

\begin{newer}
X3J13 voted in March 1989 \issue{CHARACTER-PROPOSAL}
to eliminate the type \cdf{string-char} and to redefine the type
\cdf{string} to be the union of one or more specialized vector
types, the types of whose elements are subtypes of the type \cdf{character}.
\end{newer}

Any string-specific function defined in this chapter
whose name begins with the prefix \cdf{string}
will accept a symbol instead of a string
as an argument \emph{provided} that the operation never modifies that argument;
the print name of the symbol is used.
\indexterm{print name}
In this respect the string-specific sequence operations are not
simply specializations of generic versions; the generic sequence
operations described in chapter \ref{KSEQUE} never accept symbols as sequences.
This slight inelegance is permitted in Common Lisp in the name of pragmatic utility.
One may get the effect of having a generic sequence function
operate on either symbols or strings by applying the coercion
function \cdf{string} to any argument whose data type is in doubt.

Любая функция, определенная в данной главе, имя которой имеет префикс
\cdf{string}, в качестве аргумента может принимать символ, при
условии, что операция не модифицирует этот аргумент.
При этом использоваться будет выводимое имя символа.
Таким образом операции над последовательностями из строковый символов, не
являются специализированными версиями обобщенных функций; обобщенные операции
над последовательностями, описанные в главе~\ref{KSEQUE} не принимают символ в
качестве последовательности. Такая <<неизящность>> сделана в Common Lisp'е в
целях прагматичности. Для достижения унификации функций, предлагается использовать
функцию \cdf{string} применительно ко всем аргументам, тип которых не известен заранее.

\begin{new}
Note that this remark, predating the design of the Common Lisp Object System,
uses the term ``generic'' in a generic sense and not necessarily
in the technical sense used by CLOS
(see chapter \ref{DTYPES}).
\end{new}

Also, there is a slight non-parallelism in the names of string functions.
Where the suffixes \cdf{equalp} and \cdf{eql} would be more appropriate,
for historical compatibility the suffixes \cdf{equal} and \cdf{=} are used instead
to indicate case-insensitive and case-sensitive character comparison,
respectively.

Также, существует небольшая расхождение в именах строковых функций.
Тогда как суффиксы \cdf{equalp} и \cdf{eql} были бы более подходящими, вместо
них все же для исторической совместимости используются суффиксы \cdf{equal} и
\cdf{=}, для указания, соотвественно, регистронезависимого и регистрозависимого
сравнения строковых символов.

Any Lisp object may be tested for being a string by
the predicate \cdf{stringp}.

Любой Lisp'овый объект может быть проверен на принадлежность строковому типу с
помощью предиката \cdf{stringp}.

Note that strings, like all vectors, may have fill pointers
(though such strings are not necessarily \emph{simple}).
String operations generally operate only on the active portion of the string
(below the fill pointer).  See \cdf{fill-pointer} and related
functions.

Следует отметить, что строки, как и все вектора, могут иметь указатель
заполнения (хотя такие строки необязательно простыеу).
Строковые операции в основном взаимодействуют только с активной частью строки
(перед указателем заполнения). Смотрите \cdf{fill-pointer} и связанные с ним
функции.

\section{String Access Доступ к строковым символам}

The following functions access a single character element of a string.

Следующие функции предоставляют доступ к одиночным символам в строке.

\begin{defun}[Function]
char string index \\
schar simple-string index

The given \emph{index} must be a non-negative integer less than
the length of \emph{string}, which must be a
string.  The character at position \emph{index}
of the string is returned as a character object.

Указываемый индекс \emph{index} должен быть неотрицательным целым меньшим, чем длина
строки \emph{string}. Символ из данной позиции строки возвращается, как символьный
объект. FIXME
\begin{obsolete}
(This character will necessarily satisfy the predicate \cdf{string-char-p}.)
\end{obsolete}
\begin{newer}
X3J13 voted in March 1989 \issue{CHARACTER-PROPOSAL}
to eliminate \cdf{string-char-p}.
\end{newer}
As with all sequences in Common Lisp, indexing is zero-origin.
For example:
\begin{lisp}
(char "Floob-Boober-Bab-Boober-Bubs" 0) \EV\ \#{\Xbackslash}F \\
(char "Floob-Boober-Bab-Boober-Bubs" 1) \EV\ \#{\Xbackslash}l
\end{lisp}
See \cdf{aref} and \cdf{elt}.  In effect,
\begin{lisp}
(char s j) \EQ\ (aref (the string s) j)
\end{lisp}
\cdf{setf} may be used with \cdf{char} to destructively replace
a character within a string.

Как и во всех Common Lisp'овых последовательностях, индексация начинается с
нуля.
Например:
\begin{lisp}
(char "Floob-Boober-Bab-Boober-Bubs" 0) \EV\ \#{\Xbackslash}F \\
(char "Floob-Boober-Bab-Boober-Bubs" 1) \EV\ \#{\Xbackslash}l
\end{lisp}
Смотрите \cdf{aref} и \cdf{elt}. Фактически, 
\begin{lisp}
(char s j) \EQ\ (aref (the string s) j)
\end{lisp}
\cdf{setf} может использоваться с \cdf{char} для деструктивной замены символа в
строке.

For \cdf{char}, the string may be any string;
for \cdf{schar}, it must be a simple string.
In some implementations of Common Lisp, the function \cdf{schar} may
be faster than \cdf{char} when it is applicable.

Для \cdf{char} строка может быть любой.
Для \cdf{schar} строка должна быть простой.
В некоторых реализациях Common Lisp'а, функция \cdf{schar} может быть быстрее,
чем \cdf{char}.
\end{defun}

\section{String Comparison Сравнение строк}

The naming conventions for these functions and for their keyword
arguments generally follow the conventions for the generic sequence
functions (see chapter~\ref{KSEQUE}).

Правило именования этих функций и их аргументов следует из правил именования
общих функций для последовательностей (смотрите главу~\ref{KSEQUE}).

\begin{new}
Note that this remark, predating the design of the Common Lisp Object System,
uses the term ``generic'' in a generic sense and not necessarily
in the technical sense used by CLOS
(see chapter \ref{DTYPES}).
\end{new}

\begin{defun}[Function]
string= string1 string2 &key :start1 :end1 :start2~:end2

\cdf{string=} compares two strings and is true if
they are the same (corresponding characters are identical)
but is false if they are not.
The function \cdf{equal} calls \cdf{string=} if
applied to two strings.

\cdf{string=} сравнивает две строки и возвращает истину, если они одинаковые
(все символы одинаковые), и ложь, если это не так.
Функция \cdf{equal} вызывает \cdf{string=}, если применяется к двум строкам.

The keyword arguments \cd{:start1} and \cd{:start2} are the places
in the strings to start the comparison.
The arguments \cd{:end1} and \cd{:end2} are the
places in the strings to stop comparing; comparison stops just
\emph{before} the position specified by a limit.
The ``start'' arguments default to zero (beginning of string),
and the ``end'' arguments (if either omitted or {\false})
default to the lengths of the strings (end of string),
so that by default the entirety of each string is examined.
These arguments are provided so that substrings can be compared
efficiently.

Именованные аргументы \cd{:start1} и \cd{:start2} указывают на то, с каких
позиций начинать сравнение.
Именованные аргументы \cd{:end1} и \cd{:end2} указывают на то, на каких
позициях заканчивать сравнение.
Сравнение заканичвается \emph{перед} указанными позициями.
Аргументы <<start>> по-умолчанию равны нулю (указывают на начало строки), и 
<<end>> (если опущены или равны {\false}) по-умолчанию указывают на конец
строки. Таким образом, по-умолчанию строки сравниваются целиком.
Эти аргументы позволяют удобно сравнивать подстроки.

\cdf{string=} is necessarily false if the (sub)strings
being compared are of unequal length; that is, if
\begin{lisp}
(not (= (- end1 start1) (- end2 start2)))
\end{lisp}
is true, then \cdf{string=} is false.

\begin{lisp}
(string= "foo" "foo") \textrm{is true} \\
(string= "foo" "Foo") \textrm{is false} \\
(string= "foo" "bar") \textrm{is false} \\
(string= "together" "frog" :start1 1 :end1 3 :start2 2) \\
~~~\textrm{is true}
\end{lisp}

\cdf{string=} возвращает ложь, если сравниваемые (под)строки неодинаковой
длины. То есть, если 
\begin{lisp}
(not (= (- end1 start1) (- end2 start2)))
\end{lisp}
истинно, то \cdf{string=} ложно.

\begin{lisp}
(string= "foo" "foo") \textrm{истина} \\
(string= "foo" "Foo") \textrm{ложь} \\
(string= "foo" "bar") \textrm{ложь} \\
(string= "together" "frog" :start1 1 :end1 3 :start2 2) \\
~~~\textrm{истина}
\end{lisp}

\begin{newer}
X3J13 voted in June 1989 \issue{STRING-COERCION}
to clarify string coercion (see \cdf{string}).
\end{newer}

\beforenoterule
\begin{incompatibility}
\cdf{string=} is called \cdf{strequal} in Interlisp.
\end{incompatibility}
\afternoterule
\end{defun}

\begin{defun}[Function]
string-equal string1 string2 &key :start1 :end1 :start2~:end2

\cdf{string-equal} is just like \cdf{string=} except that differences
in case are ignored; two characters are considered to be the same
if \cdf{char-equal} is true of them.
For example:
\begin{lisp}
(string-equal "foo" "Foo") \textrm{is true}
\end{lisp}
\begin{newer}
X3J13 voted in June 1989 \issue{STRING-COERCION}
to clarify string coercion (see \cdf{string}).
\end{newer}

\cdf{string-equal} похож на \cdf{string=} за исключением того, что игнорирует
регистр символов. Два символа равны если \cdf{char-equal} для них истинно.
Например:
\begin{lisp}
(string-equal "foo" "Foo") \textrm{is true}
\end{lisp}
\end{defun}

\begin{defun}[Function]
string< string1 string2 &key :start1 :end1 :start2~:end2 \\
string> string1 string2 &key :start1 :end1 :start2~:end2 \\
string<= string1 string2 &key :start1 :end1 :start2~:end2 \\
string>= string1 string2 &key :start1 :end1 :start2~:end2 \\
string/= string1 string2 &key :start1 :end1 :start2~:end2

These functions compare the two string arguments lexicographically,
and the result is {\false} unless \emph{string1} is respectively
less than, greater than,
less than or equal to, greater than or equal to, or not equal to \emph{string2}.
If the condition is satisfied, however, then
the result is the index within the strings of the first character
position at which the strings fail to match; put another way,
the result is the length of the longest common prefix of the strings.

A string \emph{a} is less than a string \emph{b} if
in the first position in which they differ the character of \emph{a}
is less than the corresponding character of \emph{b} according to
the function \cdf{char<}, or
if string \emph{a} is a proper prefix of string \emph{b}
(of shorter length and matching in all the characters of \emph{a}).

The keyword arguments \cd{:start1} and \cd{:start2} are the places
in the strings to start the comparison.
The keyword arguments \cd{:end1} and \cd{:end2}
are the places in the strings to stop comparing; comparison stops just
\emph{before} the position specified by a limit.
The ``start'' arguments default to zero (beginning of string),
and the ``end'' arguments (if either omitted or {\false})
default to the lengths of the strings (end of string),
so that by default the entirety of each string is examined.
These arguments are provided so that substrings can be compared
efficiently.  The index returned in case of a mismatch
is an index into \emph{string1}.

\begin{newer}
X3J13 voted in June 1989 \issue{STRING-COERCION}
to clarify string coercion (see \cdf{string}).
\end{newer}

Эти функции сравнивают две строки лексикографически.
Если не выполняется условие, что \emph{string1} соответственно меньше, больше,
меньше либо равно, больше либо равно, не равно чем \emph{string2}, то результат
{\false}.
Однако, если условие выполняется, то результатом будет первая позиция символа,
на которой произошло несовпадение строк.
Другими словами, результат это длина префикса строки, удовлетворяющего условию.

Строка \emph{a} меньше, чем строка \emph{b}, если в первой позиции, в которой
они различаются, символ из \emph{a} меньше, чем соотвествующий символ из
\emph{b} в соответсвии с функцией \cdf{char<}, или строка \emph{a} является
префиксом строки \emph{b} (меньшей длины и совпадением всех символов). FIXME

Именованные аргументы \cd{:start1} и \cd{:start2} указывают на то, с каких
позиций начинать сравнение.
Именованные аргументы \cd{:end1} и \cd{:end2} указывают на то, на каких
позициях заканчивать сравнение.
Сравнение заканичвается \emph{перед} указанными позициями.
Аргументы <<start>> по-умолчанию равны нулю (указывают на начало строки), и 
<<end>> (если опущены или равны {\false}) по-умолчанию указывают на конец
строки. Таким образом, по-умолчанию строки сравниваются целиком.
Эти аргументы позволяют удобно сравнивать подстроки. Индекс, возвращаемый в
случае несовпадения строк, находится в \emph{string1}. FIXME
\end{defun}

\begin{defun}[Function]
string-lessp string1 string2 &key :start1 :end1 :start2~:end2 \\
string-greaterp string1 string2 &key :start1 :end1 :start2~:end2 \\
string-not-greaterp string1 string2 &key :start1 :end1 :start2~:end2 \\
string-not-lessp string1 string2 &key :start1 :end1 :start2~:end2 \\
string-not-equal string1 string2 &key :start1 :end1 :start2~:end2

These are exactly like \cdf{string<}, \cdf{string>}, \cdf{string<=},
\cdf{string>=}, and \cdf{string/=}, respectively, except that distinctions between
uppercase and lowercase letters are ignored.  It is as if
\cdf{char-lessp} were used instead of \cdf{char<}
for comparing characters.

Эти функции такие же, как и соответственно  \cdf{string<}, \cdf{string>}, \cdf{string<=},
\cdf{string>=} и \cdf{string/=} за исключением того, что различия между
регистрами букв игнорируются. Для сравнения символов в таком случае вместо
\cdf{char<} используется \cdf{char-lessp}.

\begin{newer}
X3J13 voted in June 1989 \issue{STRING-COERCION}
to clarify string coercion (see \cdf{string}).
\end{newer}
\end{defun}

\section{String Construction and Manipulation Создание и манипулирование строками}

Most of the interesting operations on strings may be performed
with the generic sequence functions described in chapter \ref{KSEQUE}.
The following functions perform additional operations that are specific
to strings.
\begin{new}
Note that this remark, predating the design of the Common Lisp Object System,
uses the term ``generic'' in a generic sense and not necessarily
in the technical sense used by CLOS
(see chapter \ref{DTYPES}).
\end{new}

Большинство необходимых операций над строками могут быть выполнены с помощью
общих функций для последовательностей, описанных в главе~\ref{KSEQUE}.
Следующие функций производят дополнительные операции специфичные для строк.

\begin{obsolete}
\begin{defun}[Function]
make-string size &key :initial-element

This returns a string (in fact a simple string)
of length \emph{size}, each of whose characters
has been initialized to the \cd{:initial-element} argument.
If an \cd{:initial-element} argument is not specified, then the string will
be initialized in an implementation-dependent way.

\beforenoterule
\begin{implementation}
It may be convenient to initialize the string
to null characters, or to spaces, or to garbage (``whatever was there'').
\end{implementation}
\afternoterule

A string is really just a one-dimensional array of ``string
characters'' (that is, those characters that are members of type
\cdf{string-char}).  More complex character arrays may be constructed using the
function \cdf{make-array}.
\end{defun}
\end{obsolete}

\begin{newer}
X3J13 voted in March 1989 \issue{CHARACTER-PROPOSAL}
to eliminate the type \cdf{string-char} and to add a keyword
argument \cd{:element-type} to \cdf{make-string}.  The new function
description is as follows.

\begin{defun}[Function]
make-string size &key :initial-element :element-type

This returns a simple string
of length \emph{size}, each of whose characters
has been initialized to the \cd{:initial-element} argument.
If an \cd{:initial-element} argument is not specified, then the string will
be initialized in an implementation-dependent way.

The \cd{:element-type} argument names the type of the elements
of the string; a string is constructed of the most specialized type
that can accommodate elements of the given type.  If \cd{:element-type}
is omitted, the type \cdf{character} is the default.

X3J13 voted in January 1989
\issue{ARGUMENTS-UNDERSPECIFIED}
to clarify that the \emph{size} argument
must be a non-negative integer less than the value of
\cdf{array-dimension-limit}.

Функция возвращает простую строку длинной \emph{size}, каждый из символов
которой равен аргументу \cd{:initial-element}.
Если аргумент \cd{:initial-element} не задан, тогда строка инициализируется в
зависимости от реализации.

Аргумент \cd{:element-type} задает тип элементов строки. 
Строка создается с наиболее специализированным типов элементов, позволяющем
хранить элементы указанного типа.
Если \cd{:element-type} опущен, то по-молчанию используется тип \cdf{character}.
\end{defun}
\end{newer}

\begin{defun}[Function]
string-trim character-bag string \\
string-left-trim character-bag string \\
string-right-trim character-bag string

\cdf{string-trim} returns a substring of \emph{string}, with all characters in
\emph{character-bag} stripped off the beginning and end.
The function \cdf{string-left-trim} is similar but strips characters
off only the beginning; \cdf{string-right-trim} strips off only the end.
The argument \emph{character-bag} may be any sequence containing
characters.
For example:
\begin{lisp}
(string-trim '(\#{\Xbackslash}Space \#{\Xbackslash}Tab \#{\Xbackslash}Newline) " garbanzo beans \\
~~~~~~~~") \EV\ "garbanzo beans" \\
(string-trim " (*)" " ( *three (silly) words* ) ") \\
~~~\EV\ "three (silly) words" \\
(string-left-trim " (*)" " ( *three (silly) words* ) ") \\
~~~\EV\ "three (silly) words* ) " \\
(string-right-trim " (*)" " ( *three (silly) words* ) ") \\
~~~\EV\ " ( *three (silly) words"
\end{lisp}
If no characters need to be trimmed from the \emph{string},
then either the argument \emph{string} itself or a copy of it may
be returned, at the discretion of the implementation.

\begin{newer}
X3J13 voted in June 1989 \issue{STRING-COERCION}
to clarify string coercion (see \cdf{string}).
\end{newer}

\cdf{string-trim} возвращает подстроку от \emph{string}, с удаленными в начале и
в конце символами из \emph{character-bag}.
Функция \cdf{string-left-trim} похожа на предыдущую, но удаляет символы только в
начале.
\cdf{string-right-trim} удаляет символы только в
конце.
Аргумент \emph{character-bag} может быть любой последовательностью, содержащей
строковые символы.
Например:
\begin{lisp}
(string-trim '(\#{\Xbackslash}Space \#{\Xbackslash}Tab \#{\Xbackslash}Newline) " garbanzo beans \\
~~~~~~~~") \EV\ "garbanzo beans" \\
(string-trim " (*)" " ( *three (silly) words* ) ") \\
~~~\EV\ "three (silly) words" \\
(string-left-trim " (*)" " ( *three (silly) words* ) ") \\
~~~\EV\ "three (silly) words* ) " \\
(string-right-trim " (*)" " ( *three (silly) words* ) ") \\
~~~\EV\ " ( *three (silly) words"
\end{lisp}
В зависимости от реализации, в случае, если строка не претерпевала модификаций,
может возвратиться сама строка или ее копия.

\end{defun}

\begin{defun}[Function]
string-upcase string &key :start :end \\
string-downcase string &key :start :end \\
string-capitalize string &key :start :end

\cdf{string-upcase} returns a string just like \emph{string} with all lowercase
characters replaced by the corresponding uppercase characters.  More
precisely, each character of the result string is produced by applying
the function \cdf{char-upcase} to the corresponding character of
\emph{string}.

\cdf{string-upcase} возвращает сроку, в которой все символы в нижнем регистре
изменены на соответствующие в верхнем. А точнее, все символы строки получаются
применением к соотвествующим символам входной строки функции \cdf{char-upcase}.

\cdf{string-downcase} is similar, except that uppercase characters are
converted to lowercase characters (using \cdf{char-downcase}).

\cdf{string-downcase} похожа на предыдущю функию за исключением того, что
символы в верхнем регистре заменяются на соответствующие символы в нижнем
(используя \cdf{char-downcase}).

The keyword arguments \cd{:start} and \cd{:end} delimit the portion
of the string to be affected.  The result is always of the same length
as \emph{string}, however.

Именованные аргументы \cd{:start} и \cd{:end} указывают на то, какая часть
строки будет обрабатываться. Однако, результат всегда той же длины, что и
входная строка. 

The argument is not destroyed.  However, if no characters in the argument
require conversion, the result may be either the argument or a copy of it,
at the implementation's discretion.
For example:
\begin{lisp}
(string-upcase "Dr. Livingstone, I presume?") \\
~~~\EV\ "DR. LIVINGSTONE, I PRESUME?" \\
(string-downcase "Dr. Livingstone, I presume?") \\
~~~\EV\ "dr. livingstone, i presume?" \\
(string-upcase "Dr. Livingstone, I presume?" \cd{:start} 6 \cd{:end} 10) \\
~~~\EV\ "Dr. LiVINGstone, I presume?"
\end{lisp}

Аргумент деструктивно не преобразуется. Однако, если символов для перевода
регистра в аргументе не оказалось, то на усмотрение реализации результат может
быть входной строкой или ее копией.
Например:
\begin{lisp}
(string-upcase "Dr. Livingstone, I presume?") \\
~~~\EV\ "DR. LIVINGSTONE, I PRESUME?" \\
(string-downcase "Dr. Livingstone, I presume?") \\
~~~\EV\ "dr. livingstone, i presume?" \\
(string-upcase "Dr. Livingstone, I presume?" \cd{:start} 6 \cd{:end} 10) \\
~~~\EV\ "Dr. LiVINGstone, I presume?"
\end{lisp}

\cdf{string-capitalize} produces a copy of \emph{string} such that,
for every word in the copy, the first character of the word,
if case-modifiable, is uppercase and
any other case-modifiable characters in the word are lowercase.
For the purposes of \cdf{string-capitalize},
a word is defined to be a
consecutive subsequence consisting of alphanumeric characters or digits,
delimited at each end either by a non-alphanumeric character
or by an end of the string.
For example:
\begin{lisp}
(string-capitalize " hello ") \EV\ " Hello " \\
(string-capitalize \\
~~~~\="occlUDeD cASEmenTs FOreSTAll iNADVertent DEFenestraTION") \\
\EV\>"Occluded Casements Forestall Inadvertent Defenestration" \\
(string-capitalize 'kludgy-hash-search) \EV\ "Kludgy-Hash-Search" \\
(string-capitalize "DON'T!") \EV\ "Don'T!"~~~~~;\emph{not} "Don't!" \\
(string-capitalize "pipe 13a, foo16c") \EV\ "Pipe 13a, Foo16c"
\end{lisp}

\cdf{string-capitalize} возвращает копию строки \emph{string}, в которой первые
буквы всех слов возводятся в верхний регистр, а остальные буквы в нижний.
Слова состоят из алфавитно-цифровых символов, разделенных друг от друга
неалфавитно-цифровыми символами.
Например:
\begin{lisp}
(string-capitalize " hello ") \EV\ " Hello " \\
(string-capitalize \\
~~~~\="occlUDeD cASEmenTs FOreSTAll iNADVertent DEFenestraTION") \\
\EV\>"Occluded Casements Forestall Inadvertent Defenestration" \\
(string-capitalize 'kludgy-hash-search) \EV\ "Kludgy-Hash-Search" \\
(string-capitalize "DON'T!") \EV\ "Don'T!"~~~~~;\emph{not} "Don't!" \\
(string-capitalize "pipe 13a, foo16c") \EV\ "Pipe 13a, Foo16c"
\end{lisp}


\begin{newer}
X3J13 voted in June 1989 \issue{STRING-COERCION}
to clarify string coercion (see \cdf{string}).
\end{newer}

\beforenoterule
\begin{incompatibility}
Some very approximate Interlisp equivalents to
\cdf{string-upcase}, \cdf{string-downcase}, and \cdf{string-capitalize}
are \cdf{u-case}, \cdf{l-case} with second argument {\nil},
and \cdf{l-case} with second argument {\true}.
\end{incompatibility}
\afternoterule
\end{defun}

\begin{defun}[Function]
nstring-upcase string &key :start :end \\
nstring-downcase string &key :start :end \\
nstring-capitalize string &key :start :end

These three functions are just like \cdf{string-upcase},
\cdf{string-downcase}, and \cdf{string-capitalize}
but destructively modify the argument \emph{string} by altering
case-modifiable characters as necessary.

The keyword arguments \cd{:start} and \cd{:end} delimit the portion
of the string to be affected.  The argument \emph{string} is returned as
the result.

Эти три функции похожи на \cdf{string-upcase}, \cdf{string-downcase} и
\cdf{string-capitalize}, но в отличие от последних деструктивно модифицируют
аргумент \emph{string} изменяя где надо регистр символов.
\end{defun}

\begin{defun}[Function]
string x

Most of the string
functions effectively apply \cdf{string}
to such of their arguments as are supposed to be
strings.
If \emph{x} is a string, it is returned.
If \emph{x} is a symbol, its print name is returned.

Большинство строковых фукнций применяют \cdf{string} к своим аргументам.
Если \emph{x} является строкой, то они и возвращается.
Если \emph{x} является символом, то возвращается его выводимое имя.

\begin{obsolete}
If \emph{x} is a string character (a character of type \cdf{string-char}),
then a string containing that one character is returned.
\end{obsolete}
\begin{newer}
X3J13 voted in March 1989 \issue{CHARACTER-PROPOSAL}
to eliminate the type \cdf{string-char} and to redefine the type
\cdf{string} to be the union of one or more specialized vector
types, the types of whose elements are subtypes of the type \cdf{character}.
Presumably converting a character to a string always works according
to this vote.
\end{newer}
In any other situation, an error is signaled.

To convert a sequence of characters to a string, use \cdf{coerce}.
(Note that \cd{(coerce x 'string)} will not succeed if \cdf{x} is a symbol.
Conversely, \cdf{string} will not convert a list or other sequence
to be a string.)

To get the string representation of a number or any other Lisp
object, use \cd{prin1-to-string}, \cdf{princ-to-string},
or \cdf{format}.

В любой другой ситуации, сигнализируется ошибка.

Для преобразования последовательности символов в строку используйте
\cdf{coerce}.
(Следует отметить, что \cd{(coerce x 'string)} не сработает, если \cd{x}
является Lisp'овым символом. И наоборот, \cdf{string} не будет конвертировать
список или другую последовательность в строку.)

Для получения строкового отображения для числа или любого другого Lisp'ового
объекта используйте \cdf{prin1-to-string}, \cdf{princ-to-string} или
\cdf{format}.

\begin{newer}
X3J13 voted in June 1989 \issue{STRING-COERCION}
to specify that the following functions perform coercion
on their \emph{string} arguments identical to that performed
by the function \cdf{string}.

\begin{flushleft}
\cf
\begin{tabular*}{\textwidth}{@{}l@{\extracolsep{\fill}}ll@{}}
string= & string-equal & string-trim \\
string< & string-lessp &  string-left-trim \\
string> & string-greaterp &  string-right-trim \\
string<= & string-not-greaterp & string-upcase \\
string>= & string-not-lessp & string-downcase \\
string/= & string-not-equal & string-capitalize
\end{tabular*}
\end{flushleft}
Note that \cdf{nstring-upcase}, \cdf{nstring-downcase}, and
\cdf{nstring-capitalize} are absent from this list; because they modify destructively,
the argument must be a string.

As part of the same vote X3J13 specified that \cdf{string}
may perform additional implementation-dependent coercions
but the returned value must be of type \cdf{string}.
Only when no coercion is defined, whether standard or implementation-dependent,
is \cdf{string} required to signal an error, in which case the error condition
must be of type \cdf{type-error}.
\end{newer}
\end{defun}
