%Part{Number, Root = "CLM.MSS"}
%Chapter of Common Lisp Manual.  Copyright 1984, 1988, 1989 Guy L. Steele Jr.

\clearpage\def\pagestatus{FINAL PROOF}

\begingroup
\def\arcsinh{\mathop\textrm{arcsinh}\nolimits}
\def\arccosh{\mathop\textrm{arccosh}\nolimits}
\def\arctanh{\mathop\textrm{arctanh}\nolimits}
\def\cis{\mathop\textrm{cis}\nolimits}
\def\phase{\mathop\textrm{phase}\nolimits}

\ifx \rulang\Undef

\chapter{Numbers}
\label{NUMBER}

Common Lisp provides several different representations for numbers.
These representations may be divided into four categories: integers,
ratios, floating-point numbers, and complex numbers.  Many numeric
functions will accept any kind of number; they are \emph{generic}.  Other
functions accept only certain kinds of numbers.

\begin{new}
Note that this remark, predating the design of the Common Lisp Object System,
uses the term ``generic'' in a generic sense and not necessarily
in the technical sense used by CLOS
(see chapter~\ref{DTYPES}).
\end{new}

In general, numbers in Common Lisp are not true objects; \cdf{eq} cannot
be counted upon to operate on them reliably.  In particular,
it is possible that the expression
\begin{lisp}
(let ((x z) (y z)) (eq x y))
\end{lisp}
may be false rather than true if the value of \cdf{z} is a number.

\beforenoterule
\begin{rationale}
This odd breakdown of \cdf{eq} in the case of numbers
allows the implementor enough design freedom to produce exceptionally
efficient numerical code on conventional architectures.
MacLisp requires this freedom, for example, in order to produce compiled
numerical code equal in speed to Fortran.
Common Lisp makes this same restriction,
if not for this freedom, then at least for the sake of compatibility.
\end{rationale}
\afternoterule

If two objects are to be compared for ``identity,'' but either might be
a number, then the predicate \cdf{eql} is probably appropriate;
if both objects are known to be numbers, then \cdf{=}
may be preferable.

\section{Precision, Contagion, and Coercion}
\label{PRECISION-CONTAGION-COERCION-SECTION}

In general,
computations with floating-point numbers are only approximate.
The \emph{precision} of a floating-point number is not necessarily
correlated at all with the \emph{accuracy} of that number.
For instance, 3.142857142857142857 is a more precise approximation
to $\pi$ than 3.14159, but the latter is more accurate.
The precision refers to the number of bits retained in the representation.
When an operation combines a short floating-point number with a long one,
the result will be a long floating-point number.  This rule is made
to ensure that as much accuracy as possible is preserved; however,
it is by no means a guarantee.
Common Lisp numerical routines do assume, however, that the accuracy of
an argument does not exceed its precision.  Therefore
when two small floating-point numbers
are combined, the result will always be a small floating-point number.
This assumption can be overridden by first explicitly converting
a small floating-point number to a larger representation.
(Common Lisp never converts automatically from a larger size to a smaller one.)

Rational computations cannot overflow in the usual sense
(though of course there may not be enough storage
to represent one), as integers and ratios may in principle be of any magnitude.
Floating-point computations may get exponent overflow or underflow;
this is an error.

\vskip 0pt plus 24pt

\begin{newer}
X3J13 voted in June 1989 \issue{FLOAT-UNDERFLOW}
to address certain problems relating to floating-point overflow and
underflow, but certain parts of the proposed solution were not adopted, namely
to add the macro \cdf{without-floating-underflow-traps} to the language and to
require certain behavior of floating-point overflow and underflow.
The committee agreed that this area of the language requires more
discussion before a solution is standardized.

For the record, the proposal that was considered and rejected
(for the nonce) introduced a macro
\cdf{without-floating-underflow-traps}
that would execute its body in such a way that, within its dynamic extent,
a floating-point underflow
must not signal an error but instead must produce
either a denormalized number or zero as the result.
The rejected proposal also specified the following treatment of overflow and underflow:
\begin{itemize}
\item A floating-point computation that overflows should signal
  an error of type \cdf{floating-point-overflow}.
\item Unless the dynamic extent of a use of
  \cdf{without-floating-underflow-traps}, a floating-point computation that
  underflows should signal an error of type \cdf{floating-point-underflow}.  A
  result that can be represented only in denormalized form must be considered an
  underflow in implementations that support denormalized floating-point
  numbers.
\end{itemize}
These points refer to conditions \cdf{floating-point-overflow}
and \cdf{floating-point-underflow}
that were approved by X3J13
and are described in section~\ref{PREDEFINED-CONDITIONS-SECTION}.
\end{newer}

When rational and floating-point numbers are compared or combined by
a numerical function, the rule of \emph{floating-point contagion}
is followed: when a rational meets a floating-point number,
the rational is first converted to a floating-point number of
the same format.  For functions such as \cdf{+}
that take more than two arguments,
it may be that part of the operation is carried out exactly using
rationals and then the rest is done using floating-point arithmetic.

\begin{new}
X3J13 voted in January 1989
\issue{CONTAGION-ON-NUMERICAL-COMPARISONS}
to apply the rule of floating-point
contagion stated above to the case of \emph{combining} rational and floating-point numbers.
For \emph{comparing}, the following rule is to be used instead:
When a rational number and a floating-point number are to be compared
by a numerical function, in effect the floating-point number is first
converted to a rational number as if by the function \cdf{rational},
and then an exact comparison of two rational numbers is performed.
It is of course valid to use a more efficient implementation than
 actually calling the function \cdf{rational}, as long as the result
of the comparison is the same.  In the case of complex numbers, the
real and imaginary parts are handled separately.

\beforenoterule
\begin{rationale}
In general, accuracy cannot be preserved in combining operations, but
it can be preserved in comparisons, and preserving it makes that part
of Common Lisp algebraically a bit more tractable.  In particular,
this change prevents the breakdown of transitivity.
Let \cdf{a} be the result of \cd{(/~10.0 single-float-epsilon)}, and
let \cdf{j} be the result of \cd{(floor~a)}.  (Note that \cd{(=~a~(+~a~1.0))}
is true, by the definition of \cdf{single-float-epsilon}.)
Under the old rules,
all of \cd{(<=~a~j)}, \cd{(<~j~(+~j~1))}, and \cd{(<=~(+~j~1)~a)}
would be true; transitivity would then imply that \cd{(<~a~a)} ought to be
true, but of course it is false, and therefore transitivity fails.
Under the new rule, however, \cd{(<=~(+~j~1)~a)} is false.
\end{rationale}
\afternoterule
\end{new}

For functions that are mathematically associative (and possibly commutative),
a Common Lisp implementation may process the arguments in any manner consistent
with associative (and possibly commutative) rearrangement.
This does not affect the order in which the argument forms
are evaluated, of course; that order is always left to right,
as in all Common Lisp function calls.  What is left loose is the
order in which the argument values are processed.
The point of all this is that implementations may differ in 
which automatic coercions are applied because of differing
orders of argument processing.  As an example, consider this
expression:
\begin{lisp}
(+ 1/3 2/3 1.0D0 1.0 1.0E-15)
\end{lisp}
One implementation might process the arguments from left to right,
first adding \cd{1/3} and \cd{2/3} to get \cd{1}, then converting that
to a double-precision floating-point number for combination
with \cd{1.0D0}, then successively converting and adding \cd{1.0} and
\cd{1.0E-15}.  Another implementation might process the arguments
from right to left, first performing a single-precision floating-point addition
of \cd{1.0} and \cd{1.0E-15} (and probably losing some accuracy
in the process!), then converting the sum to double precision
and adding \cd{1.0D0}, then converting \cd{2/3} to double-precision
floating-point and adding it, and then converting \cd{1/3} and adding that.
A third implementation might first scan all the arguments, process
all the rationals first to keep that part of the computation exact,
then find an argument of the largest floating-point format among all
the arguments and add that, and then add in all other arguments,
converting each in turn (all in a perhaps misguided attempt to make
the computation as accurate as possible).  In any case, all three
strategies are legitimate.  The user can of course control the order of
processing explicitly by writing several calls; for example:
\begin{lisp}
(+ (+ 1/3 2/3) (+ 1.0D0 1.0E-15) 1.0)
\end{lisp}
The user can also control all coercions simply by writing calls
to coercion functions explicitly.

In general, then, the type of the result of a numerical function
is a floating-point number of the largest format among all the
floating-point arguments to the function; but if the arguments
are all rational, then the result is rational (except for functions
that can produce mathematically irrational results, in which case
a single-format floating-point number may result).

There is a separate rule of complex contagion.
As a rule, complex numbers never result from a numerical function
unless one or more of the
arguments is complex.  (Exceptions to this
rule occur among the irrational and transcendental functions,
specifically \cdf{expt}, \cdf{log}, \cdf{sqrt},
\cdf{asin}, \cdf{acos}, \cdf{acosh}, and \cdf{atanh};
see section~\ref{TRANSCENDENTAL-SECTION}.)
When a non-complex number meets a complex number, the non-complex
number is in effect first converted to a complex number by providing an
imaginary part of zero.

If any computation produces a result that is a ratio of
two integers such that the denominator evenly divides the
numerator, then the result is immediately converted to the equivalent
integer.  This is called the rule of \emph{rational canonicalization}.

If the result of any computation would be a complex rational
with a zero imaginary part, the result is immediately
converted to a non-complex rational number by taking the
real part.  This is called the rule of \emph{complex canonicalization}.
Note that this rule does \emph{not} apply to complex numbers whose components
are floating-point numbers.  Whereas \cd{\#C(5 0)} and \cd{5} are not
distinct values in Common Lisp (they are always \cdf{eql}),
\cd{\#C(5.0 0.0)} and \cd{5.0} are always distinct values in Common Lisp
(they are never \cdf{eql}, although they are \cdf{equalp}).

\section{Predicates on Numbers}

Each of the following functions tests a single number for
a specific property.
Each function requires that its argument be
a number; to call one with a non-number is an error.

\begin{defun}[Function]
zerop number

This predicate is true if \emph{number} is zero (the integer zero,
a floating-point zero, or a complex zero), and is false otherwise.
Regardless of whether an implementation provides distinct representations
for positive and negative floating-point zeros,
\cd{(zerop -0.0)} is always true.
It is an error if the argument \emph{number} is not a number.
\end{defun}

\begin{defun}[Function]
plusp number

This predicate is true if \emph{number} is strictly greater than zero,
and is false otherwise.
It is an error if the argument \emph{number} is not a non-complex number.
\end{defun}

\begin{defun}[Function]
minusp number

This predicate is true if \emph{number} is strictly less than zero,
and is false otherwise.
Regardless of whether an implementation provides distinct representations
for positive and negative floating-point zeros,
\cd{(minusp -0.0)} is always false.
(The function \cdf{float-sign} may be used to distinguish a negative zero.)
It is an error if the argument \emph{number} is not a non-complex number.
\end{defun}

\begin{defun}[Function]
oddp integer

This predicate is true if the argument \emph{integer} is odd (not divisible
by 2), and otherwise is false.  It is an error if the argument is not
an integer.
\end{defun}

\begin{defun}[Function]
evenp integer

This predicate is true if the argument \emph{integer} is even (divisible
by 2), and otherwise is false.  It is an error if the argument is not
an integer.
\end{defun}

See also the data-type predicates \cdf{integerp},
\cdf{rationalp}, \cdf{floatp}, \cdf{complexp}, and \cdf{numberp}.

\section{Comparisons on Numbers}

Each of the functions in this section requires that its arguments all be
numbers; to call one with a non-number is an error.  Unless otherwise
specified, each works on all types of numbers, automatically performing
any required coercions when arguments are of different types.

\begin{defun}[Function]
= number &rest more-numbers \\
/= number &rest more-numbers \\
< number &rest more-numbers \\
> number &rest more-numbers \\
<= number &rest more-numbers \\
>= number &rest more-numbers

These functions each take one or more arguments.  If the sequence
of arguments satisfies a certain condition:
\begin{tabbing}
\hskip 5pc\=\kill
\cdf{=}\>all the same \\
\cdf{/=}\>all different \\
\cdf{<}\>monotonically increasing \\
\cdf{>}\>monotonically decreasing \\
\cdf{<=}\>monotonically nondecreasing \\
\cdf{>=}\>monotonically nonincreasing
\end{tabbing}
then the predicate is true, and otherwise is false.
Complex numbers may be compared using \cdf{=} and \cdf{/=},
but the others require non-complex arguments.
Two complex numbers are considered equal by \cdf{=}
if their real parts are equal and their imaginary parts are equal
according to \cdf{=}.
A complex number may be compared with a non-complex number with \cdf{=} or \cdf{/=}.
For example:
\begin{lisp}
\hskip 0.5\textwidth\=\kill
(= 3 3) \textrm{is true.}\>(/= 3 3) \textrm{is false.} \\
(= 3 5) \textrm{is false.}\>(/= 3 5) \textrm{is true.} \\
(= 3 3 3 3) \textrm{is true.}\>(/= 3 3 3 3) \textrm{is false.} \\
(= 3 3 5 3) \textrm{is false.}\>(/= 3 3 5 3) \textrm{is false.} \\
(= 3 6 5 2) \textrm{is false.}\>(/= 3 6 5 2) \textrm{is true.} \\
(= 3 2 3) \textrm{is false.}\>(/= 3 2 3) \textrm{is false.} \\
(< 3 5) \textrm{is true.}\>(<= 3 5) \textrm{is true.} \\
(< 3 -5) \textrm{is false.}\>(<= 3 -5) \textrm{is false.} \\
(< 3 3) \textrm{is false.}\>(<= 3 3) \textrm{is true.} \\
(< 0 3 4 6 7) \textrm{is true.}\>(<= 0 3 4 6 7) \textrm{is true.} \\
(< 0 3 4 4 6) \textrm{is false.}\>(<= 0 3 4 4 6) \textrm{is true.} \\
(> 4 3) \textrm{is true.}\>(>= 4 3) \textrm{is true.} \\
(> 4 3 2 1 0) \textrm{is true.}\>(>= 4 3 2 1 0) \textrm{is true.} \\
(> 4 3 3 2 0) \textrm{is false.}\>(>= 4 3 3 2 0) \textrm{is true.} \\
(> 4 3 1 2 0) \textrm{is false.}\>(>= 4 3 1 2 0) \textrm{is false.} \\
(= 3) \textrm{is true.}\>(/= 3) \textrm{is true.} \\
(< 3) \textrm{is true.}\>(<= 3) \textrm{is true.} \\
(= 3.0 \#C(3.0 0.0)) \textrm{is true.}\>(/= 3.0 \#C(3.0 1.0)) \textrm{is true.} \\
(= 3 3.0) \textrm{is true.}\>(= 3.0s0 3.0d0) \textrm{is true.} \\
(= 0.0 -0.0) \textrm{is true.}\>(= 5/2 2.5) \textrm{is true.} \\
(> 0.0 -0.0) \textrm{is false.}\>(= 0 -0.0) \textrm{is true.}
\end{lisp}
With two arguments, these functions perform the usual arithmetic
comparison tests.
With three or more arguments, they are useful for range checks,
as shown in the following example:
\begin{lisp}
(<= 0 x 9)~~~~~~~~~~~~~~~~~~~~~~;\textrm{true if \cdf{x} is between 0 and 9, inclusive} \\
(< 0.0 x 1.0)~~~~~~~~~~~~~~~~~~~;\textrm{true if \cdf{x} is between 0.0 and 1.0, exclusive} \\
(< -1 j (length s))~~~~~~~~~~~~~;\textrm{true if \cdf{j} is a valid index for \cdf{s}} \\
(<= 0 j k (- (length s) 1))~~~~~;\textrm{true if \cdf{j} and \cdf{k} are each valid} \\
~~~~~~~~~~~~~~~~~~~~~~~~~~~~~~~~;\textrm{indices for \cdf{s} and $\cdf{j}\leq\cdf{k}$}
\end{lisp}


\beforenoterule
\begin{rationale}
The ``unequality'' relation is called \cdf{/=} rather than
\cd{<>}
(the name used in Pascal) for two reasons.  First, \cdf{/=} of more than two
arguments is not the same as the \cdf{or} of \cdf{<} and \cdf{>} of those same
arguments.  Second, unequality is meaningful for complex numbers even though
\cdf{<} and \cdf{>} are not.  For both reasons it would be misleading to
associate unequality with the names of \cdf{<} and \cdf{>}.
\end{rationale}
\afternoterule
\end{defun}

\begin{defun}[Function]
max number &rest more-numbers \\
min number &rest more-numbers

The arguments may be any non-complex numbers.
\cdf{max} returns the argument that is greatest (closest
to positive infinity).
\cdf{min} returns the argument that is least (closest to
negative infinity).

For \cdf{max},
if the arguments are a mixture of rationals and floating-point
numbers, and the largest argument
is a rational, then the implementation is free to
produce either that rational or its floating-point approximation;
if the largest argument is a floating-point number of a smaller format
than the largest format of any floating-point argument,
then the implementation is free to
return the argument in its given format or expanded to the larger format.
More concisely, the implementation has the choice of returning the largest
argument as is or applying the rules of floating-point contagion,
taking all the arguments into consideration for contagion purposes.
Also, if two or more of the arguments are equal, then any one
of them may be chosen as the value to return.
Similar remarks apply to \cdf{min} (replacing ``largest argument'' by
``smallest argument'').

\begin{lisp}
\hskip 0.5\textwidth\=\kill
(max 6 12) \EV\ 12\>(min 6 12) \EV\ 6 \\
(max -6 -12) \EV\ -6\>(min -6 -12) \EV\ -12 \\
(max 1 3 2 -7) \EV\ 3\>(min 1 3 2 -7) \EV\ -7 \\
(max -2 3 0 7) \EV\ 7\>(min -2 3 0 7) \EV\ -2 \\
(max 3) \EV\ 3\>(min 3) \EV\ 3 \\
(max 5.0 2) \EV\ 5.0\>(min 5.0 2) \EV\ 2 \emph{or} 2.0 \\
(max 3.0 7 1) \EV\ 7 \emph{or} 7.0\>(min 3.0 7 1) \EV\ 1 \emph{or} 1.0 \\
(max 1.0s0 7.0d0) \EV\ 7.0d0 \\
(min 1.0s0 7.0d0) \EV\ 1.0s0 \emph{or} 1.0d0 \\
(max 3 1 1.0s0 1.0d0) \EV\ 3 \emph{or} 3.0d0 \\
(min 3 1 1.0s0 1.0d0) \EV\ 1 \emph{or} 1.0s0 \emph{or} 1.0d0
\end{lisp}
\end{defun}

\section{Arithmetic Operations}

Each of the functions in this section requires that its arguments all be
numbers; to call one with a non-number is an error.  Unless otherwise
specified, each works on all types of numbers, automatically performing
any required coercions when arguments are of different types.

\begin{defun}[Function]
+ &rest numbers

This returns the sum of the arguments.  If there are no arguments, the result
is \cd{0}, which is an identity for this operation.
\end{defun}

\begin{defun}[Function]
- number &rest more-numbers

The function \cdf{-}, when given one argument, returns the negative
of that argument.

The function \cdf{-}, when given more than one argument, successively subtracts
from the first argument all the others, and returns the result.
For example, \cd{(- 3 4 5)} \EV\ \cd{-6}.
\end{defun}

\begin{defun}[Function]
* &rest numbers

This returns the product of the arguments.
If there are no arguments, the result
is \cd{1}, which is an identity for this operation.
\end{defun}

\begin{defun}[Function]
/ number &rest more-numbers

The function \cdf{/}, when given more than one argument, successively divides
the first argument by all the others and returns the result.

\begin{new}%CORR
It is generally accepted that it is an error for any argument other than the
first to be zero.
\end{new}

With one argument, \cdf{/} reciprocates the argument.

\begin{new}%CORR
It is generally accepted that it is an error in this case for the argument
to be zero.
\end{new}

\cdf{/} will produce a ratio if the mathematical quotient of two integers
is not an exact integer.  For example:
\begin{lisp}
(/ 12 4) \EV\ 3 \\
(/ 13 4) \EV\ 13/4 \\
(/ -8) \EV\ -1/8 \\
(/ 3 4 5) \EV\ 3/20
\end{lisp}
To divide one integer by another producing an integer result,
use one of the functions \cdf{floor}, \cdf{ceiling}, \cdf{truncate},
or \cdf{round}.

If any argument is a floating-point number,
then the rules of floating-point contagion apply.
\end{defun}

\begin{defun}[Function]
1+ number \\
1- number

\cd{(1+ \emph{x})} is the same as \cd{(+ \emph{x} 1)}.

\cd{(1- \emph{x})} is the same as \cd{(- \emph{x} 1)}.
Note that the short name may be confusing: \cd{(1- \emph{x})} does \emph{not} mean
$1-\emph{x}$; rather, it means $\emph{x}-1$.

\beforenoterule
\begin{implementation}
Compiler writers are very strongly encouraged to ensure
that \cd{(1+ x)} and \cd{(+ x 1)} compile into identical code, and
similarly for \cd{(1- x)} and \cd{(- x 1)}, to avoid pressure on a Lisp
programmer to write possibly less clear code for the sake of efficiency.
This can easily be done as a source-language transformation.
\end{implementation}
\afternoterule
\end{defun}

\begin{defmac}
incf place [delta] \\
decf place [delta]

The number produced by the form \emph{delta}
is added to (\cdf{incf}) or subtracted from (\cdf{decf})
the number in the generalized variable named by \emph{place},
and the sum is stored back into \emph{place} and returned.
The form \emph{place} may be any form acceptable
as a generalized variable to \cdf{setf}.
If \emph{delta} is not supplied, then the number in \emph{place} is changed
by \cd{1}.
For example:
\begin{lisp}
\hskip 9pc\=\kill
(setq n 0) \\
(incf n) \EV\ 1\>\textrm{and now} n \EV\ 1 \\
(decf n 3) \EV\ -2\>\textrm{and now} n \EV\ -2 \\
(decf n -5) \EV\ 3\>\textrm{and now} n \EV\ 3 \\
(decf n) \EV\ 2\>\textrm{and now} n \EV\ 2
\end{lisp}
The effect of \cd{(incf \emph{place} \emph{delta})}
is roughly equivalent to
\begin{lisp}
(setf \emph{place} (+ \emph{place} \emph{delta}))
\end{lisp}
except that the latter would evaluate any subforms of \emph{place}
twice, whereas \cdf{incf} takes care to evaluate them only once.
Moreover, for certain \emph{place} forms \cdf{incf} may be
significantly more efficient than the \cdf{setf} version.
\begin{newer}
X3J13 voted in March 1988 \issue{PUSH-EVALUATION-ORDER}
to clarify order of evaluation (see section~\ref{SETF-SECTION}).
\end{newer}
\end{defmac}

\begin{defun}[Function]
conjugate number

This returns the complex conjugate of \emph{number}.  The conjugate
of a non-complex number is itself.  For a complex number \cdf{z},
\begin{lisp}
(conjugate z) \EQ\ (complex (realpart z) (- (imagpart z)))
\end{lisp}
For example:
\begin{lisp}
(conjugate \#C(3/5 4/5)) \EV\ \#C(3/5 -4/5) \\
(conjugate \#C(0.0D0 -1.0D0)) \EV\ \#C(0.0D0 1.0D0) \\
(conjugate 3.7) \EV\ 3.7
\end{lisp}
\end{defun}

\begin{defun}[Function]
gcd &rest integers

This returns the greatest common divisor of all the arguments,
which must be integers.  The result of \cdf{gcd} is always a non-negative
integer.
If one argument is given, its absolute value is returned.
If no arguments are given, \cdf{gcd} returns \cd{0},
which is an identity for this operation.
For three or more arguments,
\begin{lisp}
(gcd \emph{a} \emph{b} \emph{c} ... \emph{z}) \EQ\ (gcd (gcd \emph{a} \emph{b}) \emph{c} ... \emph{z})
\end{lisp}

Here are some examples of the use of \cdf{gcd}:
\begin{lisp}
(gcd 91 -49) \EV\ 7 \\
(gcd 63 -42 35) \EV\ 7 \\
(gcd 5) \EV\ 5 \\
(gcd -4) \EV\ 4 \\
(gcd) \EV\ 0
\end{lisp}
\end{defun}

\begin{defun}[Function]
lcm integer &rest more-integers

This returns the least common multiple of its arguments,
which must be integers.
The result of \cdf{lcm} is always a non-negative integer.
For two arguments that are not both zero,
\begin{lisp}
(lcm \emph{a} \emph{b}) \EQ\ (/ (abs (* \emph{a} \emph{b})) (gcd \emph{a} \emph{b}))
\end{lisp}
If one or both arguments are zero,
\begin{lisp}
(lcm \emph{a} 0) \EQ\ (lcm 0 \emph{a}) \EQ\ 0
\end{lisp}

For one argument, \cdf{lcm} returns the absolute value of that argument.
For three or more arguments,
\begin{lisp}
(lcm \emph{a} \emph{b} \emph{c} ... \emph{z}) \EQ\ (lcm (lcm \emph{a} \emph{b}) \emph{c} ... \emph{z})
\end{lisp}

Some examples:
\begin{lisp}
(lcm 14 35) \EV\ 70 \\
(lcm 0 5) \EV\ 0 \\
(lcm 1 2 3 4 5 6) \EV\ 60
\end{lisp}

\begin{lisp}
\cd{(lcm)~\EV~1}.  
\end{lisp}

\cd{(lcm)} ought to have been defined to return 1.
\end{defun}

\section{Irrational and Transcendental Functions}
\label{TRANSCENDENTAL-SECTION}

Common Lisp provides no data type that can accurately represent irrational
numerical values.
The functions in this section are described as if the results
were mathematically accurate, but actually they all produce floating-point
approximations to the true mathematical result in the general case.
In some places
mathematical identities are set forth that are intended to elucidate the
meanings of the functions; however, two mathematically identical
expressions may be computationally different because of errors
inherent in the floating-point approximation process.

When the arguments to
a function in this section are all rational and the true mathematical result
is also (mathematically) rational, then unless otherwise noted
an implementation is free to return either an accurate result of
type \cdf{rational} or a single-precision floating-point approximation.
If the arguments are all rational but the result cannot be expressed
as a rational number, then a single-precision floating-point
approximation is always returned.

If the arguments to a function are all of type
\cd{(or~rational (complex rational))} and the true mathematical result
is (mathematically) a complex number with rational real
and imaginary parts, then unless otherwise noted
an implementation is free to return either an accurate result of
type \cd{(or rational (complex rational))}
or a single-precision floating-point approximation
of type \cdf{single-float} (permissible only if the imaginary part
of the true mathematical result is zero) or \cd{(complex single-float)}.
If the arguments are all of type
\cd{(or~rational (complex rational))} but the result cannot be expressed
as a rational or complex rational number, then the returned value
will be of type \cdf{single-float} (permissible only if the imaginary part
of the true mathematical result is zero) or \cd{(complex single-float)}.

The rules of floating-point contagion and complex contagion are 
effectively obeyed by all the functions in this section except \cdf{expt},
which treats some cases of rational exponents specially.
When, possibly after contagious conversion, all of the arguments are of
the same floating-point or complex floating-point type,
then the result will be of that same type unless otherwise noted.

\beforenoterule
\begin{implementation}
There is a ``floating-point cookbook'' by
Cody and Waite~\cite{CODY-AND-WAITE} that may be a useful aid
in implementing the functions defined in this section.
\end{implementation}
\afternoterule

\subsection{Exponential and Logarithmic Functions}

Along with the usual one-argument and two-argument exponential and
logarithm functions, \cdf{sqrt} is considered to be an exponential
function, because it raises a number to the power 1/2.

\begin{defun}[Function]
exp number

Returns \emph{e} raised to the power \emph{number},
where \emph{e} is the base of the natural logarithms.
\end{defun}

\begin{defun}[Function]
expt base-number power-number

Returns \emph{base-number} raised to the power \emph{power-number}.
If the \emph{base-number} is of type \cdf{rational} and the \emph{power-number} is
an \cdf{integer},
the calculation will be exact and the result will be of type \cdf{rational};
otherwise a floating-point approximation may result.

\begin{newer}
X3J13 voted in March 1989
\issue{COMPLEX-RATIONAL-RESULT}
to clarify that provisions similar to those of the previous paragraph apply to complex
numbers.  If the \emph{base-number} is of type \cd{(complex~rational)}
and the \emph{power-number} is
an \cdf{integer},
the calculation will also be exact and the result will be of type
\cd{(or~rational (complex rational))};
otherwise a floating-point or complex floating-point approximation may result.
\end{newer}

When \emph{power-number} is \cd{0} (a zero of type integer),
then the result is always the value 1 in the type of \emph{base-number},
even if the \emph{base-number} is zero (of any type).  That is:
\begin{lisp}
(expt \emph{x} 0) \EQ\ (coerce 1 (type-of \emph{x}))
\end{lisp}
If the \emph{power-number} is a zero of any other data type,
then the result is also the value 1, in the type of the arguments
after the application of the contagion rules, with one exception:
it is an error if \emph{base-number} is zero when the \emph{power-number}
is a zero not of type integer.

Implementations of \cdf{expt} are permitted to use different algorithms
for the cases of a rational \emph{power-number} and a floating-point
\emph{power-number}; the motivation is that in many cases greater accuracy
can be achieved for the case of a rational \emph{power-number}.
For example, \cd{(expt pi 16)} and \cd{(expt pi 16.0)} may yield
slightly different results if the first case is computed by repeated squaring
and the second by the use of logarithms.  Similarly, an implementation
might choose to compute \cd{(expt x 3/2)} as if it had
been written \cd{(sqrt (expt x 3))}, perhaps producing a more accurate
result than would \cd{(expt x 1.5)}.  It is left to the implementor
to determine the best strategies.

The result of \cdf{expt} can be a complex number, even when neither argument
is complex, if \emph{base-number} is negative and \emph{power-number}
is not an integer.  The result is always the principal complex value.
Note that \cd{(expt -8 1/3)} is not permitted to return \cd{-2};
while \cd{-2} is indeed one of the cube roots of \cd{-8}, it is
not the principal cube root, which is a complex number
approximately equal to \cd{\#C(1.0 1.73205)}.
\end{defun}

\begin{defun}[Function]
log number &optional base

Returns the logarithm of \emph{number} in the base \emph{base},
which defaults to \emph{e}, the base of the natural logarithms.
For example:
\begin{lisp}
(log 8.0 2) \EV\ 3.0 \\
(log 100.0 10) \EV\ 2.0
\end{lisp}
The result of \cd{(log 8 2)} may be either \cd{3} or \cd{3.0}, depending on the
implementation.

Note that \cdf{log} may return a complex result when given a non-complex
argument if the argument is negative.  For example:
\begin{lisp}
(log -1.0) \EQ\ (complex 0.0 (float pi 0.0))
\end{lisp}

\begin{new}
X3J13 voted in January 1989
\issue{IEEE-ATAN-BRANCH-CUT}
to specify certain floating-point behavior when minus zero is supported.
As a part of that vote it approved a mathematical definition of complex logarithm
in terms of real logarithm, absolute value,
arc tangent of two real arguments, and the phase function as
\begin{tabbing}
\hskip 10pc\=\kill
Logarithm\>$ \log \left| \emph{z} \right| + \emph{i} \phase \emph{z} $
\end{tabbing}
This specifies the branch cuts precisely whether minus zero is supported or not;
see \cdf{phase} and \cdf{atan}.
\end{new}
\end{defun}

\begin{defun}[Function]
sqrt number

Returns the principal square root of \emph{number}.
If the \emph{number} is not complex but is negative, then the result
will be a complex number.
For example:
\begin{lisp}
(sqrt 9.0) \EV\ 3.0 \\
(sqrt -9.0) \EV\ \#c(0.0 3.0)
\end{lisp}
The result of \cd{(sqrt 9)} may be either \cd{3} or \cd{3.0}, depending on the
implementation.  The result of \cd{(sqrt -9)} may be either \cd{\#c(0 3)}
or \cd{\#c(0.0 3.0)}.

\begin{new}
X3J13 voted in January 1989
\issue{IEEE-ATAN-BRANCH-CUT}
to specify certain floating-point behavior when minus zero is supported.
As a part of that vote it approved a mathematical definition of complex square root
in terms of complex logarithm and exponential functions as
\begin{tabbing}
\hskip 10pc\=\kill
Square root\> $ \emph{e}^{(\log z)/2} $
\end{tabbing}
This specifies the branch cuts precisely whether minus zero is supported or not;
see \cdf{phase} and \cdf{atan}.
\end{new}
\end{defun}

\begin{defun}[Function]
isqrt integer

Integer square root: the argument must be a non-negative integer, and the
result is the greatest integer less than or equal to the exact positive
square root of the argument.
For example:
\begin{lisp}
(isqrt 9) \EV\ 3 \\
(isqrt 12) \EV\ 3 \\
(isqrt 300) \EV\ 17 \\
(isqrt 325) \EV\ 18
\end{lisp}
\end{defun}

\subsection{Trigonometric and Related Functions}

Some of the functions in this section, such as \cdf{abs}
and \cdf{signum}, are apparently unrelated
to trigonometric functions when considered as functions of
real numbers only.  The way in which they are extended to
operate on complex numbers makes the trigonometric connection clear.

\begin{defun}[Function]
abs number

Returns the absolute value of the argument.  For a non-complex number \emph{x},
\begin{lisp}
(abs \emph{x}) \EQ\ (if (minusp \emph{x}) (- \emph{x}) \emph{x})
\end{lisp}
and the result is always of the same type as the argument.

For a complex number \emph{z}, the absolute value may be computed as
\begin{lisp}
(sqrt (+ (expt (realpart \emph{z}) 2) (expt (imagpart \emph{z}) 2)))
\end{lisp}

\beforenoterule
\begin{implementation}
The careful implementor will not use this formula directly
for all complex numbers
but will instead handle very large or very small components specially
to avoid intermediate overflow or underflow.
\end{implementation}
\afternoterule

For example:
\begin{lisp}
(abs \#c(3.0 -4.0)) \EV\ 5.0
\end{lisp}
The result of \cd{(abs \#c(3 4))} may be either \cd{5} or \cd{5.0},
depending on the implementation.
\end{defun}

\begin{defun}[Function]
phase number

The phase of a number is the angle part of its polar representation
as a complex number.  That is,
\begin{lisp}
(phase \emph{z}) \EQ\ (atan (imagpart \emph{z}) (realpart \emph{z}))
\end{lisp}

\begin{new}
X3J13 voted in January 1989
\issue{IEEE-ATAN-BRANCH-CUT}
to specify certain floating-point behavior when minus zero is supported;
\cdf{phase} is still defined in terms of \cdf{atan} as above,
but thanks to a change in \cdf{atan} the range of \cdf{phase}
becomes $-\pi$ \emph{inclusive} to $\pi$ inclusive.  The value $-\pi$
results from an argument\vadjust{\penalty-10000}
whose real part is negative and whose imaginary
part is minus zero.  The \cdf{phase} function therefore has a branch cut
along the negative real axis.  The phase of $+0+0\emph{i}$ is $+0$, of $+0-0\emph{i}$ is $-0$,
of $-0+0\emph{i}$ is $+\pi$, and of $-0-0\emph{i}$ is $-\pi$.
\end{new}

If the argument is a complex floating-point number, the result
is a floating-point number of the same type as the components of
the argument.
If the argument is a floating-point number, the result is a
floating-point number of the same type.
If the argument is a rational number or complex rational number, the result
is a single-format floating-point number.
\end{defun}

\begin{defun}[Function]
signum number

By definition,
\begin{lisp}
(signum \emph{x}) \EQ\ (if (zerop \emph{x}) \emph{x} (/ \emph{x} (abs \emph{x})))
\end{lisp}
For a rational number, \cdf{signum} will return one of \cd{-1}, \cd{0}, or \cd{1}
according to whether the number is negative, zero, or positive.
For a floating-point number, the result will be a floating-point number
of the same format whose value is $-1$, $0$, or $1$.
For a complex number \emph{z}, \cd{(signum \emph{z})} is a complex number of
the same phase but with unit magnitude, unless \emph{z} is a complex zero,
in which case the result is \emph{z}.
For example:
\begin{lisp}
(signum 0) \EV\ 0 \\
(signum -3.7L5) \EV\ -1.0L0 \\
(signum 4/5) \EV\ 1 \\
(signum \#C(7.5 10.0)) \EV\ \#C(0.6 0.8) \\
(signum \#C(0.0 -14.7)) \EV\ \#C(0.0 -1.0)
\end{lisp}
For non-complex rational numbers, \cdf{signum} is a rational function,
but it may be irrational for complex arguments.
\end{defun}

\begin{defun}[Function]
sin radians \\
cos radians \\
tan radians

\cdf{sin} returns the sine of the argument, \cdf{cos} the cosine,
and \cdf{tan} the tangent.  The argument is in radians.
The argument may be complex.
\end{defun}

\begin{defun}[Function]
cis radians

This computes $ \emph{e} ^ {i \cdot radians} $.
The name \cdf{cis} means ``cos + \emph{i} sin,'' because
$ \emph{e} ^{ i \theta } = \cos \theta + \emph{i} \sin \theta $.
The argument is in
radians and may be any non-complex number.  The result is a complex
number whose real part is the cosine of the argument and whose imaginary
part is the sine.  Put another way, the result is a complex number whose
phase is the equal to the argument (mod $2\pi$)
and whose magnitude is unity.

\beforenoterule
\begin{implementation}
Often it is cheaper to calculate the sine and cosine
of a single angle together than to perform two disjoint calculations.
\end{implementation}
\afternoterule
\end{defun}

\begin{defun}[Function]
asin number \\
acos number

\cdf{asin} returns the arc sine of the argument, and \cdf{acos} the arc cosine.
The result is in radians.  The argument may be complex.

The arc sine and arc cosine functions may be defined mathematically for
an argument \emph{z} as follows:
\begin{tabbing}
\hskip 10pc\=\kill
Arc sine\>$ -\emph{i} \log \left(\emph{i}\emph{z} + \sqrt{1-\emph{z}^2}\right) $ \\[2pt]
Arc cosine\>$ -\emph{i} \log \left(\emph{z} + \emph{i}\sqrt{1-\emph{z}^2}\right) $
\end{tabbing}
Note that the result of \cdf{asin} or \cdf{acos} may be
complex even if the argument is not complex; this occurs
when the absolute value of the argument is greater than 1.

\begin{newer}
Kahan \cite{KAHAN-COMPLEX-FNS} suggests for \cdf{acos} the
defining formula
\begin{tabbing}
\hskip 10pc\=\kill
Arc cosine\>$  \displaystyle { 2 \log \left( \sqrt{{1+\emph{z} \over 2}} + \emph{i} \sqrt{{1-\emph{z} \over 2}} \right) \over i}$
\end{tabbing}
or even the much simpler $ (\pi/2)-\arcsin \emph{z} $.  Both equations are mathematically
equivalent to the formula shown above.
\end{newer}

\beforenoterule
\begin{implementation}
These formulae are mathematically correct, assuming
completely accurate computation.  They may be terrible methods for
floating-point computation.  Implementors should consult a good text on
numerical analysis.  The formulae given above are not necessarily
the simplest ones for real-valued computations, either; they are chosen
to define the branch cuts in desirable ways for the complex case.
\end{implementation}
\afternoterule
\end{defun}

\begin{defun}[Function]
atan y &optional x

An arc tangent is calculated and the result is returned in radians.

With two arguments \emph{y} and \emph{x}, neither argument may be complex.
The result is the arc tangent of the quantity \emph{y/x}.
The signs of \emph{y} and \emph{x} are used to derive quadrant
information; moreover, \emph{x} may be zero provided
\emph{y} is not zero.  The value of \cdf{atan} is always between
$-\pi$ (exclusive) and $\pi$ (inclusive).
The following table details various special cases.

\begin{flushleft}
\begin{tabular*}{\linewidth}{@{}l@{\extracolsep{\fill}}llc@{}}
\multicolumn{2}{c}{Condition}&Cartesian Locus&Range of Result \\
$y=+0$&$x>0$&Just above positive x-axis&$+0$ \\
$y>0$&$x>0$&Quadrant I&$+0 < result < \pi/2$ \\
$y>0$&$x=\pm 0$&Positive y-axis&$\pi/2$ \\
$y>0$&$x<0$&Quadrant II&$\pi/2 < result < \pi$ \\
$y=+0$&$x<0$&Just below negative x-axis&$\pi$ \\
$y=-0$&$x<0$&Just above negative x-axis&$\pi$ \\
$y<0$&$x<0$&Quadrant III&$-\pi < result < -\pi/2$ \\
$y<0$&$x=\pm 0$&Negative y-axis&$-\pi/2$ \\
$y<0$&$x>0$&Quadrant IV&$-\pi/2 < result < -0$ \\
$y=-0$&$x>0$&Just below positive x-axis&$-0$ \\
$y=+0$&$x=+0$&Near origin&$+0$ \\
$y=-0$&$x=+0$&Near origin&$-0$ \\
$y=+0$&$x=-0$&Near origin&$\pi$ \\
$y=-0$&$x=-0$&Near origin&$-\pi$ \\
\end{tabular*}
\end{flushleft}

Note that the case $y=0,x=0$ is an error in the absence of minus zero,
but the four cases $y=\pm 0,x=\pm 0$ are defined in the presence
of minus zero.

With only one argument \emph{y}, the argument may be complex.
The result is the arc tangent of \emph{y}, which may be defined by
the following formula:
\begin{tabbing}
\hskip 10pc\=\kill
Arc tangent\>$\log (1+\emph{i}\emph{y}) - \log (1-\emph{i}\emph{y}) \over 2\emph{i}$
\end{tabbing}

\beforenoterule
\begin{implementation}
This formula is mathematically correct, assuming
completely accurate computation.  It may be a terrible method for
floating-point computation.  Implementors should consult a good text on
numerical analysis.  The formula given above is not necessarily
the simplest one for real-valued computations, either; it is chosen
to define the branch cuts in desirable ways for the complex case.
\end{implementation}
\afternoterule

For a non-complex argument \emph{y}, the result is non-complex and lies between
$-\pi/2$ and $\pi/2$ (both exclusive).
\end{defun}

\begin{defun}[Constant]
pi

This global variable has as its value the best possible approximation to
$\pi$ in \emph{long} floating-point format.
For example:
\begin{lisp}
(defun sind (x)~~~~~;\textrm{The argument is in degrees} \\
~~(sin (* x (/ (float pi x) 180))))
\end{lisp}
An approximation to $\pi$ in some other precision can
be obtained by writing \cd{(float pi \emph{x})}, where \emph{x} is a
floating-point number of the desired precision,
or by writing \cd{(coerce pi \emph{type})}, where \emph{type} is the
name of the desired type, such as \cdf{short-float}.
\end{defun}

\penalty-10000\relax

\begin{defun}[Function]
sinh number \\
cosh number \\
tanh number \\
asinh number \\
acosh number \\
atanh number

These functions compute the hyperbolic sine, cosine, tangent,
arc sine, arc cosine, and arc tangent functions, which are mathematically
defined for an argument \emph{z} as follows:

\begin{tabbing}
\hskip 10pc\=\kill
Hyperbolic sine\>$ (\emph{e}^{z}-\emph{e}^{-z})/2 $ \\
Hyperbolic cosine\>$ (\emph{e}^{z}+\emph{e}^{-z})/2 $ \\
Hyperbolic tangent\>$ (\emph{e}^{z}-\emph{e}^{-z})/(\emph{e}^{z}+\emph{e}^{-z}) $ \\[2pt]
Hyperbolic arc sine\>$ \log \left(\emph{z}+\sqrt{1+\emph{z}^2}\right) $ \\[2pt]
Hyperbolic arc cosine\>$ \log
\left(\emph{z}+(\emph{z}+1)\sqrt{(\emph{z}-1)/(\emph{z}+1)}\right) $ \\[2pt]
Hyperbolic arc tangent\>$ \log \left((1+\emph{z})\sqrt{1/(1-\emph{z}^2)}\right) $
\end{tabbing}

Note that the result of \cdf{acosh} may be
complex even if the argument is not complex; this occurs
when the argument is less than 1.
Also, the result of \cdf{atanh} may be
complex even if the argument is not complex; this occurs
when the absolute value of the argument is greater than 1.

\beforenoterule
\begin{implementation}
These formulae are mathematically correct, assuming
completely accurate computation.  They may be terrible methods for
floating-point computation.  Implementors should consult a good text on
numerical analysis.  The formulae given above are not necessarily
the simplest ones for real-valued computations, either; they are chosen
to define the branch cuts in desirable ways for the complex case.
\end{implementation}
\afternoterule
\end{defun}

\subsection{Branch Cuts, Principal Values, and Boundary Conditions in the
  Complex Plane}
\label{BRANCH-CUTS-SECTION}

Many of the irrational and transcendental functions are multiply defined
in the complex domain; for example, there are in general an infinite
number of complex values for the logarithm function.  In each such
case, a principal value must be chosen for the function to return.
In general, such values cannot be chosen so as to make the range
continuous; lines in the domain
called \emph{branch cuts} must be defined, which in turn
define the discontinuities in the range.

Common Lisp defines the branch cuts, principal values, and boundary
conditions for the complex functions following
a proposal for complex functions in APL \cite{APL-BRANCH-CUTS}.
The contents of this section are borrowed largely from that proposal.

\begin{new}
Indeed, X3J13 voted in January 1989
\issue{COMPLEX-ATAN-BRANCH-CUT}
to alter the direction of continuity for
the branch cuts of \cdf{atan}, and also
\issue{IEEE-ATAN-BRANCH-CUT}
to address the treatment of branch cuts
in implementations that have a distinct floating-point minus zero.

The treatment of minus zero centers in two-argument \cdf{atan}.
If there is no minus zero, then the branch cut runs just below the negative real
axis as before, and the range of two-argument \cdf{atan} is $(-\pi,\pi]$.
If there is a minus zero, however, then the branch cut runs precisely on the negative real
axis, skittering between pairs of numbers of the form $-\emph{x}\pm 0\emph{i}$,
and the range of two-argument \cdf{atan} is $[-\pi,\pi]$.

The treatment of minus zero by all other irrational and transcendental functions
is then specified by defining those functions in terms of two-argument \cdf{atan}.
First, \cdf{phase} is defined in terms of two-argument \cdf{atan}, and
complex \cdf{abs} in terms of real \cdf{sqrt};
then complex \cdf{log} is defined in terms of \cdf{phase}, \cdf{abs}, and real \cdf{log};
then complex \cdf{sqrt} in terms of complex \cdf{log};
and finally all others are defined in terms of these.

Kahan \cite{KAHAN-COMPLEX-FNS} treats these matters in some detail and also
suggests specific algorithms for implementing irrational and transcendental functions
in IEEE standard floating-point arithmetic \cite{IEEE-PROPOSED-FLOATING-POINT-STANDARD}.

Remarks in the first edition about the direction of the continuity of branch
cuts continue to hold in the absence of minus zero and may be ignored if minus zero
is supported; since all branch cuts happen to run along the principal axes, they
run \emph{between} plus zero and minus zero, and so each sort of zero is associated
with the obvious quadrant.
\end{new}

\begin{flushdesc}
\item[\cdf{sqrt}]
The branch cut for square root lies along the negative real axis,
continuous with quadrant II.
The range consists of the right half-plane, including the non-negative
imaginary axis and excluding the negative imaginary axis.

\begin{new}
X3J13 voted in January 1989
\issue{IEEE-ATAN-BRANCH-CUT}
to specify certain floating-point behavior when minus zero is supported.
As a part of that vote it approved a mathematical definition of complex square root:
\begin{tabbing}
$ \sqrt{\emph{z}} = \emph{e}^{(\log z)/2} $
\end{tabbing}
This defines the branch cuts precisely, whether minus zero is supported or not.
\end{new}

\item[\cdf{phase}]
The branch cut for the phase function lies along the negative real
axis, continuous with quadrant II.  The range consists of that portion of
the real axis between $-\pi$ (exclusive) and $\pi$ (inclusive).

\begin{new}
X3J13 voted in January 1989
\issue{IEEE-ATAN-BRANCH-CUT}
to specify certain floating-point behavior when minus zero is supported.
As a part of that vote it approved a mathematical definition of phase:
\begin{tabbing}
$ \phase \emph{z} = \arctan (\Im \emph{z}, \Re \emph{z}) $
\end{tabbing}
where $\Im \emph{z}$ is the imaginary part of $\emph{z}$ and $\Re \emph{z}$ the real part of $\emph{z}$.
This defines the branch cuts precisely, whether minus zero is supported or not.
\end{new}

\item[\cdf{log}]
The branch cut for the logarithm function of one argument (natural
logarithm) lies along the negative real axis, continuous with quadrant II.
The domain excludes the origin.  For a complex number $\emph{z}$,
$\log \emph{z}$ is defined to be
\begin{tabbing}
$ \log \emph{z} = \left(\log \left|\emph{z}\right|\right)+\emph{i} (\phase \emph{z}) $
\end{tabbing}
Therefore the range of the one-argument logarithm function is that strip
of the complex plane containing numbers with imaginary parts between
$-\pi$ (exclusive) and $\pi$ (inclusive).

\begin{new}
The X3J13 vote on minus zero
\issue{IEEE-ATAN-BRANCH-CUT}
would alter that exclusive bound of $-\pi$  to be inclusive if minus zero is supported.
\end{new}

The two-argument logarithm function is defined as $ \log_{b} \emph{z}=(\log \emph{z})/(\log \emph{b}) $.
This defines the principal values precisely.  The range of the two-argument
logarithm function is the entire complex plane.
It is an error if $\emph{z}$ is zero.  If $\emph{z}$ is non-zero and $\emph{b}$ is zero,
the logarithm is taken to be zero.

\item[\cdf{exp}]
The simple exponential function has no branch cut.

\item[\cdf{expt}]
The two-argument exponential function is defined
as $ \emph{b}^{x}=\emph{e}^{x \log b } $.
This defines the principal values precisely.  The range of the
two-argument exponential function is the entire complex plane.  Regarded
as a function of $\emph{x}$, with $\emph{b}$ fixed, there is no branch cut.
Regarded as a function of \emph{b}, with $\emph{x}$ fixed, there is in general
a branch cut along the negative real axis, continuous with quadrant II.
The domain excludes the origin.
By definition, $0^0=1$.  If $\emph{b}=0$ and the real part of $\emph{x}$ is strictly
positive, then $\emph{b}^{x}=0$.
For all other values of $\emph{x}$, $0^{x}$
is an error.

\item[\cdf{asin}]
The following definition for arc sine determines the range and
branch cuts:
\begin{tabbing}
$ \arcsin \emph{z}=-\emph{i} \log \left(\emph{i}\emph{z}+\sqrt{1-\emph{z}^2}\right) $
\end{tabbing}

\begin{newer}\noindent
This is equivalent to the formula
\begin{tabbing}
$ \displaystyle \arcsin \emph{z} = { \arcsinh \emph{i}\emph{z} \over \emph{i} }$
\end{tabbing}
recommended by Kahan \cite{KAHAN-COMPLEX-FNS}.
\end{newer}

The branch cut for the arc sine function is in two pieces:
one along the negative real axis to the left of $-1$
(inclusive), continuous with quadrant II, and one along the positive real
axis to the right of 1 (inclusive), continuous with quadrant IV.  The
range is that strip of the complex plane containing numbers whose real
part is between $-\pi/2$ and $\pi/2$.  A number with real
part equal to $-\pi/2$ is in the range if and only if its imaginary
part is non-negative; a number with real part equal to $\pi/2$ is in
the range if and only if its imaginary part is non-positive.

\item[\cdf{acos}]
The following definition for arc cosine determines the range and
branch cuts:
\begin{tabbing}
$ \arccos \emph{z}=-\emph{i} \log \left(\emph{z}+\emph{i} \sqrt{1-\emph{z}^2}\right) $
\end{tabbing}
or, which is equivalent,
\begin{tabbing}
$ \arccos \emph{z}={\pi \over 2}-\arcsin \emph{z} $
\end{tabbing}
The branch cut for the arc cosine function is in two pieces:
one along the negative real axis to the left of $-1$
(inclusive), continuous with quadrant II, and one along the positive real
axis to the right of 1 (inclusive), continuous with quadrant IV.  
This is the same branch cut as for arc sine.
The range is that strip of the complex plane containing numbers whose real
part is between zero and $\pi$.  A number with real
part equal to zero is in the range if and only if its imaginary
part is non-negative; a number with real part equal to $\pi$ is in
the range if and only if its imaginary part is non-positive.

\item[\cdf{atan}]
The following definition for (one-argument) arc tangent determines the
range and branch cuts:

\begin{new}
X3J13 voted in January 1989
\issue{COMPLEX-ATAN-BRANCH-CUT}
to replace the formula shown above with the formula
\begin{tabbing}
$ \displaystyle \arctan \emph{z} = { \log (1+\emph{i}\emph{z}) - \log (1-\emph{i}\emph{z}) \over 2\emph{i} }$
\end{tabbing}
This is equivalent to the formula
\begin{tabbing}
$ \displaystyle \arctan \emph{z} = { \arctanh \emph{i}\emph{z} \over \emph{i} }$
\end{tabbing}
recommended by Kahan \cite{KAHAN-COMPLEX-FNS}.
It causes the upper branch cut to be continuous with
quadrant I rather than quadrant II, and the lower branch cut to
be continuous with quadrant III rather than quadrant IV; otherwise it agrees with the
formula of the first edition.  Therefore this change alters the result returned by \cdf{atan}
only for arguments on the positive imaginary axis that
are of magnitude greater than 1.  The full description for this new formula is as follows.

The branch cut for the arc tangent function is in two pieces:
one along the positive imaginary axis above \emph{i}
(exclusive), continuous with quadrant I, and one along the negative imaginary
axis below $-\emph{i}$ (exclusive), continuous with quadrant III.  
The points \emph{i} and $-\emph{i}$ are excluded from the domain.
The range is that strip of the complex plane containing numbers whose real
part is between $-\pi/2$ and $\pi/2$.  A number with real
part equal to $-\pi/2$ is in the range if and only if its imaginary
part is strictly negative; a number with real part equal to $\pi/2$ is in
the range if and only if its imaginary part is strictly positive.  Thus the range of
the arc tangent function is \emph{not} identical to that of the arc sine function.
\end{new}

\item[\cdf{asinh}]
The following definition for the inverse hyperbolic sine determines
the range and branch cuts:
\begin{tabbing}
$ \arcsinh \emph{z}=\log \left(\emph{z}+\sqrt{1+\emph{z}^2}\right)$
\end{tabbing}
The branch cut for the inverse hyperbolic sine function is in two pieces:
one along the positive imaginary axis above \emph{i}
(inclusive), continuous with quadrant I, and one along the negative imaginary
axis below $-\emph{i}$ (inclusive), continuous with quadrant III.
The range is that strip of the complex plane containing numbers whose imaginary
part is between $-\pi/2$ and $\pi/2$.  A number with imaginary
part equal to $-\pi/2$ is in the range if and only if its real
part is non-positive; a number with imaginary part equal to $\pi/2$ is in
the range if and only if its real part is non-negative.

\item[\cdf{acosh}]
The following definition for the inverse hyperbolic cosine
determines the range and branch cuts:
\begin{tabbing}
$ \arccosh \emph{z}=\log \left(\emph{z}+(\emph{z}+1)\sqrt{(\emph{z}-1)/(\emph{z}+1)}\right) $
\end{tabbing}

\begin{newer}
Kahan \cite{KAHAN-COMPLEX-FNS} suggests the formula
\begin{tabbing}
$ \arccosh \emph{z}=2 \log \left(  \sqrt{(\emph{z}+1)/2} + \sqrt{(\emph{z}-1)/2} \right) $
\end{tabbing}
pointing out that it yields the same principal value but eliminates
a gratuitous removable singularity at $\emph{z}=-1$.
A proposal was submitted to X3J13 in September 1989 to replace the
formula \cdf{acosh} with that recommended by Kahan.
There is a good possibility that it will be adopted.
\end{newer}

The branch cut for the inverse hyperbolic cosine function
lies along the real axis to the left of 1 (inclusive), extending
indefinitely along the negative real axis, continuous with quadrant II
and (between 0 and 1) with quadrant I.
The range is that half-strip of the complex plane containing numbers whose
real part is non-negative and whose imaginary
part is between $-\pi$ (exclusive) and $\pi$ (inclusive).
A number with real part zero is in the range 
if its imaginary part is between zero (inclusive) and $\pi$ (inclusive).

\item[\cdf{atanh}]
The following definition for the inverse hyperbolic tangent
determines the range and branch cuts:

\begin{newer}
WARNING!  \emph{The formula shown above for hyperbolic arc tangent is incorrect.}
It is not a matter of incorrect branch cuts; it simply does not compute anything
like a hyperbolic arc tangent.  This unfortunate error in the first edition
was the result of mistranscribing a (correct) APL formula from Penfield's paper
\cite{APL-BRANCH-CUTS}.  The formula should have been transcribed as
\begin{tabbing}
$ \arctanh \emph{z}=\log \left((1+\emph{z})\sqrt{1/(1-\emph{z}^2)}\right) $
\end{tabbing}
\end{newer}

\begin{newer}
A proposal was submitted to X3J13 in September 1989 to replace the
formula \cdf{atanh} with that recommended by Kahan \cite{KAHAN-COMPLEX-FNS}:
\begin{tabbing}
$ \displaystyle \arctanh \emph{z}= { \left(\log(1+\emph{z}) - \log(1-\emph{z})\right) \over 2 } $
\end{tabbing}
There is a good possibility that it will be adopted.  If it is, the complete
description of the branch cuts of \cdf{atanh} will then be as follows.

The branch cut for the inverse hyperbolic tangent function
is in two pieces: one along the negative real axis to the left of
$-1$ (inclusive), continuous with quadrant II, and one along
the positive real axis to the right of 1 (inclusive), continuous with
quadrant IV.  The points $-1$ and 1 are excluded from the
domain.
The range is that strip of the complex plane containing
numbers whose imaginary part is between $-\pi/2$ and
$\pi/2$.  A number with imaginary part equal to $-\pi/2$
is in the range if and only if its real part is strictly positive; a number with
imaginary part equal to $\pi/2$ is in the range if and only if its real
part is strictly negative.  Thus the range of the inverse
hyperbolic tangent function is \emph{not} the same as
that of the inverse hyperbolic sine function.
\end{newer}
\end{flushdesc}

With these definitions, the following useful identities are obeyed
throughout the applicable portion of the complex domain, even on
the branch cuts:


\begin{flushleft}
\begin{tabular*}{\textwidth}{@{}l@{\extracolsep{\fill}}ll@{}}
$\sin \emph{i}\emph{z} = \emph{i} \sinh \emph{z}$&$\sinh \emph{i}\emph{z} = \emph{i} \sin \emph{z}$&$\arctan \emph{i}\emph{z} = \emph{i} \arctanh \emph{z}$ \\
$\cos \emph{i}\emph{z} = \cosh \emph{z}$&$\cosh \emph{i}\emph{z} = \cos \emph{z}$&$\arcsinh \emph{i}\emph{z} = \emph{i} \arcsin \emph{z}$ \\
$\tan \emph{i}\emph{z} = \emph{i} \tanh \emph{z}$&$\arcsin \emph{i}\emph{z} = \emph{i} \arcsinh \emph{z}$&$\arctanh \emph{i}\emph{z} = \emph{i} \arctan \emph{z}$
\end{tabular*}
\end{flushleft}

\begin{new}
I thought it would be useful to provide some graphs illustrating the behavior
of the irrational and transcendental functions in the complex plane.
It also provides an opportunity to show off the Common Lisp code that
was used to generate them.

Imagine the complex plane to be decorated
as follows.  The real and imaginary axes are painted with thick lines.
Parallels from the axes on both sides at distances of 1, 2, and 3 are painted
with thin lines; these parallels are doubly infinite lines, as are the axes.
Four annuli (rings) are painted in gradated shades of gray.  Ring 1, the inner ring,
consists of points whose radial distances from the origin lie in the range
$[1/4,1/2]$; ring 2 is in the radial range
$[3/4, 1]$; ring 3, in the range
$[\pi /2, 2]$; and ring 4, in the range $[3, \pi]$.
Ring \emph{j} is divided into $2^{j+1}$ equal sectors, with each sector
painted a different shade of gray, darkening as one proceeds counterclockwise
from the positive real axis.

We can illustrate the behavior of a numerical function $\emph{f}$ by considering how
it maps the complex plane to itself.  More specifically, consider each
point $\emph{z}$ of the decorated plane.  We decorate a new plane by coloring
the point $\emph{f}(\emph{z})$ with the same color that point $\emph{z}$ had in the original
decorated plane.  In other words, the newly decorated plane illustrates
how the $\emph{f}$ maps the axes, other horizontal and vertical lines, and annuli.

In each figure we will show only a fragment of the complex plane,
with the real axis horizontal in the usual manner ($-\infty$ to the left, $+\infty$
to the right) and the imaginary axis vertical ($-\infty \emph{i}$ below, $+\infty \emph{i}$
above).  Each fragment shows a region containing points whose real and imaginary
parts are in the range $[-4.1, 4.1]$.  The axes of the new plane are shown as very
thin lines, with large tick marks at integer coordinates and somewhat smaller
tick marks at multiples of $\pi/2$.

Figure~\ref{IDENTITY-PLOT} shows the result of plotting the \cdf{identity} function
(quite literally); the graph exhibits the decoration of the original plane.

Figures~\ref{SECOND-PLOT} through~\ref{LAST-PLOT} show the graphs for the functions
\cdf{sqrt}, \cdf{exp}, \cdf{log}, \cdf{sin}, \cdf{asin}, \cdf{cos}, \cdf{acos}, \cdf{tan}, \cdf{atan},
\cdf{sinh}, \cdf{asinh}, \cdf{cosh}, \cdf{acosh}, \cdf{tanh}, and \cdf{atanh}, and
as a bonus, the graphs for the functions $\sqrt{1-\emph{z}^2}$,
$\sqrt{1+\emph{z}^2}$, $(\emph{z}-1)/(\emph{z}+1)$, and $(1+\emph{z})/(1-\emph{z})$.  All of these are related
to the trigonometric functions in various ways.  For example, if
$\emph{f}(\emph{z})=(\emph{z}-1)/(\emph{z}+1)$, then $\tanh \emph{z} = \emph{f}(\emph{e}^{2z})$, and if $\emph{g}(\emph{z})=\sqrt{1-\emph{z}^2}$, then
$\cos \emph{z} = \emph{g}(\sin \emph{z})$.  It is instructive to examine the graph for $\sqrt{1-\emph{z}^2}$
and try to visualize how it transforms the graph for \cdf{sin} into the graph for~\cdf{cos}.

Each figure is accompanied by a commentary on what maps to what and other interesting
features.  None of this material is terribly new; much of it may be found in any
good textbook on complex analysis.  I believe that the particular form in which the
graphs are presented is novel, as well as the fact that the graphs have been generated
as PostScript \cite{ADOBE-POSTSCRIPT} code by Common Lisp code.  This PostScript
code was then fed directly to the typesetting equipment that set the pages for this book.
Samples of this PostScript code follow the figures themselves,
after which the code for the entire program is presented.

In the commentaries that accompany the figures I
sometimes speak of mapping the points $\pm\infty$ or $\pm\infty i$.
When I say that function $f$ maps $+\infty$ to a certain point $z$, I mean that
\begin{tabbing}
$ z=\lim_{x\rightarrow +\infty} f(x+0\, i\/) $
\end{tabbing}
Similarly, when I say that $f$ maps
$-\infty i$ to $z$, I mean that
\begin{tabbing}
$ z = \lim_{y\rightarrow -\infty} f(0+yi\/) $
\end{tabbing}
In other words, I am considering a limit as one travels out along one of the
main axes.  I~also speak in a similar manner of mapping \emph{to} one of these
infinities.
\end{new}
\clearpage

\begingroup

\ifx \HCode\Undef
\newcommand{\numplot}[1]{\includegraphics{#1-plot}\\}
\else
\newcommand{\numplot}[1]{\includegraphics{#1-plot.png}\\}
\fi

\newdimen\foodimen
\foodimen=\textheight
\setbox0=\vbox{\hrule}
\advance\foodimen by -2\ht0
\advance\foodimen by -16pt


\begin{figure}
\caption{Initial Decoration of the Complex Plane (Identity Function)}
\label{IDENTITY-PLOT}
\numplot{identity}
\small\noindent
This figure was produced in exactly the same manner as succeeding figures,
simply by plotting the function \cdf{identity} instead of a numerical function.
Thus the first of these figures was produced by the last function of the first edition.
I knew it would come in handy someday!
\end{figure}


\clearpage

\begin{figure}
\caption{Illustration of the Range of the Square Root Function}
\label{SECOND-PLOT}
\numplot{sqrt}
\small\noindent
The \cdf{sqrt} function maps the complex plane into the right half
of the plane by slitting it along the negative real axis and then sweeping
it around as if half-closing a folding fan.
The fan also shrinks, as if it were made of cotton and had gotten
wetter at the periphery than at the center.
The positive real axis is mapped onto itself.  The negative real axis is mapped
onto the positive imaginary axis (but if minus zero is supported, then
$-\emph{x}+0\emph{i}$ is mapped onto the positive imaginary axis and $-\emph{x}-0\emph{i}$ onto
the negative imaginary axis, assuming $\emph{x}>0$).  The positive imaginary axis
is mapped onto the northeast diagonal, and the negative imaginary axis
onto the southeast diagonal.  More generally, lines are mapped to rectangular hyperbolas
(or fragments thereof\,) centered on the origin;
lines through the origin are mapped to degenerate
hyperbolas (perpendicular lines through the origin).
\end{figure}

\clearpage

\begin{figure}
\caption{Illustration of the Range of the Exponential Function}
\numplot{exp}
\small\noindent
The \cdf{exp} function maps horizontal lines to radii and maps vertical
lines to circles centered at the origin.
The origin is mapped to 1.  (It is instructive to compare
this graph with those of other functions
that map the origin to 1, for example $(1+\emph{z})/(1-\emph{z})$, $\cos \emph{z}$, and $\sqrt{1-\emph{z}^2}$.)
The entire real axis is mapped to the positive
real axis, with $-\infty$ mapping to the origin and $+\infty$ to itself.
The imaginary axis is mapped to the unit circle with infinite multiplicity (period $2\pi$);
therefore the mapping of the imaginary infinities $\pm\infty \emph{i}$ is indeterminate.
It follows that the entire left half-plane is mapped to the interior of the unit circle,
and the right half-plane is mapped to the exterior of the unit circle.
A line at any angle other than horizontal or vertical is mapped to a
logarithmic spiral (but this is not illustrated here).
\end{figure}

\clearpage

\begin{figure}
\caption{Illustration of the Range of the Natural Logarithm Function}
\numplot{log}
\small\noindent
The \cdf{log} function, which is the inverse of \cdf{exp}, naturally maps radial lines to
horizontal lines and circles centered at the origin to vertical lines.
The interior of the unit circle is thus mapped to the entire left half-plane,
and the exterior of the unit circle is mapped to the right half-plane.
The positive real axis is mapped to the entire real axis, and the negative
real axis to a horizontal line of height $\pi$.  The positive and negative
imaginary axes are mapped to horizontal lines of height $\pm\pi/2$.
The origin is mapped to $-\infty$.
\end{figure}

\clearpage

\begin{figure}
\caption{Illustration of the Range of the Function $(\emph{z}-1)/(\emph{z}+1)$}
\numplot{minus-over-plus}
\small\noindent
A line is a degenerate circle with infinite radius;
when I say ``circles'' here I also mean lines.
Then $(\emph{z}-1)/(\emph{z}+1)$ maps circles into circles.
All circles through $-1$ become lines; all lines become
circles through $1$.
The real axis is mapped onto itself: 1 to
the origin, the origin to $-1$, $-1$ to infinity, and infinity to 1.
The imaginary axis becomes the unit circle; \emph{i} is mapped to itself,
as is $-\emph{i}$.  Thus the entire right half-plane is mapped to the interior
of the unit circle, the unit circle interior to the left half-plane,
the left half-plane to the unit circle exterior, and the unit circle exterior
to the right half-plane.  Imagine the complex plane to be a vast sea.
The Colossus of Rhodes straddles the origin, its left foot on \emph{i} and its right foot on $-\emph{i}$.
It bends down and briefly paddles water between its legs so furiously that the water
directly beneath is pushed out into the entire area behind it; much that was
behind swirls forward to either side; and all that was before is
sucked in to lie between its feet.
\end{figure}

\clearpage

\begin{figure}
\caption{Illustration of the Range of the Function $(1+\emph{z})/(1-\emph{z})$}
\numplot{plus-over-minus}
\small\noindent
The function $\emph{h}(\emph{z})=(1+\emph{z})/(1-\emph{z})$
is the inverse of $\emph{f}(\emph{z})=(\emph{z}-1)/(\emph{z}+1)$; that is,
$\emph{h}(\emph{f}(\emph{z}))=\emph{f}(\emph{h}(\emph{z}))=\emph{z}$. At first glance,
the graph of $\emph{h}$ appears to be that of $\emph{f}$
flipped left-to-right, or perhaps reflected in the origin, but careful
consideration of the shaded annuli reveals that this is not so; something more
subtle is going on.  Note that
$\emph{f}(\emph{f}(\emph{z}))=\emph{h}(\emph{h}(\emph{z}))=\emph{g}(\emph{z})=-1/\emph{z}$.
The functions $\emph{f}$, $\emph{g}$, $\emph{h}$, and the identity function
thus form a group under composition, isomorphic to the group
of the cyclic permutations of the points $-1$, $0$, $1$, and $\infty$, as
indeed these functions accomplish the four possible cyclic permutations
on those points.  This function group is a subset of the group of bilinear
transformations $(\emph{a}\emph{z}+\emph{b})/(\emph{c}\emph{z}+\emph{d})$,
all of which are conformal (angle-preserving) and map circles
onto circles.  Now, doesn't that tangle of circles through $-1$ look like something
the cat got into?
\end{figure}

\clearpage

\begin{figure}
\caption{Illustration of the Range of the Sine Function}
\numplot{sin}
\small\noindent
We are used to seeing \cdf{sin} looking like a wiggly ocean wave,
graphed vertically as a function of the real axis only.  Here is a different view.
The entire real axis is mapped to the segment $[-1, 1]$ of the real axis
with infinite multiplicity (period $2\pi$).  The imaginary axis is mapped to itself
as if by \cdf{sinh} considered as a real function.  The origin is mapped to itself.
Horizontal lines are mapped to ellipses with foci at $\pm1$ (note that two horizontal
lines equidistant from the real axis will map onto the same ellipse).
Vertical lines are mapped to hyperbolas with the same foci.  There is a curious accident:
the ellipse for horizontal
lines at distance $\pm1$ from the real axis appears to intercept the real axis at
$\pm\pi/2\approx \pm1.57\ldots$ but this is not so; the intercepts are actually at
$\pm(\emph{e}+1/\emph{e})/2\approx \pm1.54\ldots\,\hbox{}$.
\end{figure}

\clearpage

\begin{figure}
\caption{Illustration of the Range of the Arc Sine Function}
\numplot{asin}
\small\noindent
Just as \cdf{sin} grabs horizontal lines and bends them into elliptical loops around
the origin, so its inverse \cdf{asin} takes annuli and yanks them more
or less horizontally straight.  Because sine is not injective,
its inverse as a function cannot be surjective.  This is just a highfalutin
way of saying that the range of the \cdf{asin} function doesn't
cover the entire plane but only a strip $\pi$ wide; arc sine as a one-to-many relation
would cover the plane with an infinite number of copies of this strip side by side,
looking for all the world like the tail of a peacock with an infinite number of feathers.
The imaginary axis is mapped to itself as if by \cdf{asinh} considered as a real function.
The real axis is mapped to a bent path, turning corners at $\pm\pi/2$ (the points to
which $\pm 1$ are mapped); $+\infty$ is mapped to $\pi/2-\infty \emph{i}$, and
$-\infty$ to $-\pi/2+\infty \emph{i}$.
\end{figure}

\clearpage

\begin{figure}
\caption{Illustration of the Range of the Cosine Function}
\numplot{cos}
\small\noindent
We are used to seeing \cdf{cos} looking exactly like \cdf{sin}, a wiggly ocean wave,
only displaced.  Indeed the complex mapping of \cdf{cos} is also similar
to that of \cdf{sin}, with horizontal and vertical lines mapping to the same ellipses
and hyperbolas with foci at $\pm 1$, although mapping to them in a different
manner, to be sure.
The entire real axis is again mapped to the segment $[-1, 1]$ of the real axis,
but each half of the imaginary axis is mapped to the real axis to the right of 1
(as if by \cdf{cosh} considered as a real function).  Therefore $\pm\infty \emph{i}$
both map to $+\infty$.
The origin is mapped to 1.  Whereas \cdf{sin} is an odd function, \cdf{cos} is an
even function; as a result \emph{two} points in each annulus, one the negative
of the other, are mapped to the same shaded point in this graph; the shading shown here
is taken from points in the original upper half-plane.
\end{figure}

\clearpage

\begin{figure}
\caption{Illustration of the Range of the Arc Cosine Function}
\numplot{acos}
\small\noindent
The graph of \cdf{acos} is very much like that of \cdf{asin}.
One might think that our nervous peacock has shuffled half a step
to the right, but the shading on the annuli shows that we have instead caught
the bird exactly in mid-flight while doing a cartwheel.
This is easily understood if we recall that $\arccos \emph{z}=(\pi/2)-\arcsin \emph{z}$;
negating $\arcsin \emph{z}$ rotates it upside down, and adding the result to $\pi/2$
translates it $\pi/2$ to the right.
The imaginary axis is mapped upside down to the vertical line at $\pi/2$.
The point $+1$ is mapped to the origin, and $-1$ to $\pi$.
The image of the real axis is again cranky; $+\infty$ is mapped to $+\infty \emph{i}$,
and $-\infty$ to $\pi-\infty \emph{i}$.
\end{figure}

\clearpage

\begin{figure}
\caption{Illustration of the Range of the Tangent Function}
\numplot{tan}
\small\noindent
The usual graph of \cdf{tan} as a real function looks like an infinite chorus
line of disco dancers, left hands pointed skyward and right hands to the floor.
The \cdf{tan} function is the quotient of \cdf{sin} and \cdf{cos}
but it doesn't much look like either except for having period $2\pi$.
This goes for the complex plane as well, although the swoopy loops produced
from the annulus between $\pi/2$ and $2$ look vaguely like
those from the graph of \cdf{sin} inside out.
The real axis is mapped onto itself with infinite multiplicity (period $2\pi$).
The imaginary axis is mapped backwards onto $[-\emph{i},\emph{i}]$:
$+\infty \emph{i}$ is mapped to $-\emph{i}$ and $-\infty \emph{i}$ to $+\emph{i}$.
Horizontal lines below or above the real axis
become circles surrounding $+\emph{i}$ or $-\emph{i}$, respectively.
Vertical lines become circular arcs from $+\emph{i}$ to $-\emph{i}$;
two vertical lines separated by $(2\emph{k}+1)\pi$ for integer $\emph{k}$
together become a complete circle.  It seems that two arcs
shown hit the real axis at $\pm\pi/2=\pm 1.57\ldots$ but that is a coincidence;
they really hit the axis at $\pm\tan 1= 1.55\ldots\,\hbox{}$.
\end{figure}

\clearpage

\begin{figure}
\caption{Illustration of the Range of the Arc Tangent Function}
\numplot{xatan}
\small\noindent
All I can say is that this peacock is a horse of another color.
At first glance, the axes seem to map in the same way as for \cdf{asin} and
\cdf{acos}, but look again: this time it's the imaginary axis doing weird things.
All infinities map multiply to the points
$(2\emph{k}+1)\pi/2$; within the strip of principal values we may say that
the real axis is mapped to the interval $[-\pi/2,+\pi/2]$ and therefore
$-\infty$ is mapped to $-\pi/2$ and $+\infty$ to $+\pi/2$.
The point $+\emph{i}$ is mapped to $+\infty \emph{i}$, and $-\emph{i}$ to $-\infty \emph{i}$, and
so the imaginary axis is mapped into three pieces: the segment
$[-\infty \emph{i},-\emph{i}]$ is mapped to $[\pi/2,\pi/2-\infty \emph{i}]$; the segment
$[-\emph{i},\emph{i}]$ is mapped to the imaginary axis $[-\infty \emph{i},+\infty \emph{i}]$; and the segment
$[+\emph{i},+\infty \emph{i}]$ is mapped to $[-\pi/2+\infty \emph{i},-\pi/2]$.
\end{figure}

\clearpage

\begin{figure}
\caption{Illustration of the Range of the Hyperbolic Sine Function}
\numplot{sinh}
\small\noindent
It would seem that the graph of \cdf{sinh} is merely that of \cdf{sin}
rotated 90 degrees.  If that were so, then we would have $\sinh \emph{z} = \emph{i} \sin \emph{z}$.
Careful inspection of the shading, however, reveals that this is not quite the case;
in both graphs the lightest and darkest shades, which initially are adjacent to
the positive real axis, remain adjacent to the positive real axis in both cases.
To derive the graph of \cdf{sinh} from \cdf{sin} we must therefore first
rotate the complex plane by $-90$ degrees, then apply \cdf{sin}, then
rotate the result by 90 degrees. In other words, $\sinh \emph{z} = \emph{i} \sin (-\emph{i\/})\emph{z}$;
consistently replacing $\emph{z}$ with $\emph{i}\emph{z}$ in this formula yields the familiar identity
$\sinh \emph{i}\emph{z} = \emph{i} \sin \emph{z}$.
\end{figure}

\clearpage

\begin{figure}
\caption{Illustration of the Range of the Hyperbolic Arc Sine Function}
\numplot{asinh}
\small\noindent
The peacock sleeps.  Because $\arcsinh \emph{i}\emph{z} = \emph{i} \arcsin \emph{z}$,
the graph of \cdf{asinh}
is related to that of \cdf{asin} by pre- and post-rotations of the complex plane
in the same way as for \cdf{sinh} and \cdf{sin}.
\end{figure}

\clearpage

\begin{figure}
\caption{Illustration of the Range of the Hyperbolic Cosine Function}
\numplot{cosh}
\small\noindent
The graph of \cdf{cosh} does \emph{not} look like that of \cdf{cos} rotated
90 degrees; instead it looks like that of \cdf{cos} unrotated.
That is because $\cosh \emph{i}\emph{z}$ is not equal to $\emph{i} \cos \emph{z}$;
rather, $\cosh \emph{i}\emph{z} = \cos \emph{z}$.
Interpreted, that means that the shading is pre-rotated but there is no
post-rotation.
\end{figure}

\clearpage

\begin{figure}
\caption{Illustration of the Range of the Hyperbolic Arc Cosine Function}
\numplot{acosh}
\small\noindent
Hmm---I'd rather not say what happened to this peacock.
This feather looks a bit mangled.  Actually it is all right---the principal
value for \cdf{acosh} is so chosen that its graph does not look simply
like a rotated version of the graph of \cdf{acos}, but if all values were
shown, the two graphs would fill the plane in repeating patterns related
by a rotation.
\end{figure}

\clearpage

\begin{figure}
\caption{Illustration of the Range of the Hyperbolic Tangent Function}
\numplot{tanh}
\small\noindent
The diagram for \cdf{tanh} is simply that of \cdf{tan} turned on its ear:
$\emph{i} \tan \emph{z} = \tanh \emph{i}\emph{z}$.
The imaginary axis is mapped onto itself with infinite multiplicity (period $2\pi$),
and the real axis is mapped onto the segment $[-1,+1]$:
$+\infty$ is mapped to $+1$, and $-\infty$ to $-1$.
Vertical lines to the left or right of the real axis
are mapped to circles surrounding $-1$ or $1$, respectively.
Horizontal lines are mapped to circular arcs anchored at $-1$ and $+1$;
two horizontal lines separated by a distance $(2\emph{k}+1)\pi$ for integer $\emph{k}$ are
together mapped into a complete circle.  How do we know these really are circles?
Well, $\tanh \emph{z} = ((\exp 2\emph{z})-1)/((\exp 2\emph{z})+1)$, which is the composition of the
bilinear transform $(\emph{z}-1)/(\emph{z}+1)$, the exponential $\exp \emph{z}$,
and the magnification $2\emph{z}$.
Magnification maps lines to lines of the same slope; the exponential maps
horizontal lines to circles and vertical lines to radial lines;
and a bilinear transform maps generalized circles (including lines) to
generalized circles.  Q.E.D.
\end{figure}

\clearpage

\begin{figure}
\caption{Illustration of the Range of the Hyperbolic Arc Tangent Function}
\label{ATANH-PLOT}
\numplot{really-good-atanh}
\small\noindent
A sleeping peacock of another color: $\arctanh \emph{i}\emph{z} = \emph{i} \arctan \emph{z}$.
\end{figure}

\clearpage

\begin{figure}
\caption{Illustration of the Range of the Function $\protect\sqrt{1-\emph{z}^2}$}
\numplot{sqrt-one-minus-sq}
\small\noindent
Here is a curious graph indeed for so simple a function!
The origin is mapped to 1.  The real axis segment $[0,1]$ is
mapped backwards (and non-linearly) into itself; the segment $[1,+\infty]$
is mapped non-linearly onto the positive imaginary axis.
The negative real axis is mapped to the same points as the positive real axis.
Both halves of the imaginary axis are mapped into $[1,+\infty]$ on the real axis.
Horizontal lines become vaguely vertical, and
vertical lines become vaguely horizontal.
Circles centered at the origin are transformed into Cassinian \hbox{(half-)ovals}; the unit
circle is mapped to a \hbox{(half-)lemniscate} of Bernoulli.  The outermost annulus appears
to have its \emph{inner} edge at $\pi$ on the real axis and its \emph{outer}
edge at 3 on the imaginary axis, but this is another accident; the intercept
on the real axis, for example, is not really at $\pi\approx 3.14\ldots$ but
at $\sqrt{1-(3\emph{i\/})^2}=\sqrt{10}\approx 3.16\ldots\,\hbox{}$.
\end{figure}

\clearpage

\begin{figure}
\caption{Illustration of the Range of the Function $\protect\sqrt{1+\emph{z}^2}$}
\label{LAST-PLOT}
\numplot{sqrt-one-plus-sq}
\small\noindent
The graph of $\emph{q}(\emph{z})=\sqrt{1+\emph{z}^2}$ looks like that
of $\emph{p}(\emph{z})=\sqrt{1-\emph{z}^2}$ except for
the shading.  You might not expect $\emph{p}$ and $\emph{q}$ to be related in the same
way that \cdf{cos} and \cdf{cosh} are, but after a little reflection (or perhaps
I should say, after turning it around in one's mind) one can see that
$\emph{q}(\emph{i}\emph{z})=\emph{p}(\emph{z})$.  This formula is indeed of exactly the same form as
$\cosh \emph{i}\emph{z} = \cos \emph{z}$.  The function
$\sqrt{1+\emph{z}^2}$ maps both halves of the real axis into $[1,+\infty]$ on the real axis.
The segments $[0,\emph{i}]$ and $[0,-\emph{i}]$ of the imaginary axis are each mapped
backwards onto segment $[0,1]$ of the real axis; $[\emph{i},+\infty \emph{i}]$ and
$[-\emph{,}-\infty \emph{i}]$ are each mapped onto the positive imaginary axis
(but if minus zero is supported then opposite sides of the
imaginary axis map to opposite halves of the imaginary axis---for example,
$q(+0+2\emph{i\/})=\sqrt{5}\emph{i}$ but $q(-0+2\emph{i\/})=-\sqrt{5}\emph{i}$).
\end{figure}

\clearpage
\endgroup
\begin{new}
Here is a sample of the PostScript code that generated
figure~\ref{IDENTITY-PLOT}, showing the initial scaling,
translation, and clipping parameters; the code for one
sector of the innermost annulus; and the code for the negative
imaginary axis.  Comment lines indicate how path or boundary
segments were generated separately and then spliced (in order to
allow for the places that a singularity might lurk, in which case
the generating code can ``inch up'' to the problematical argument
value).

The size of the entire PostScript file for the
\cdf{identity} function was about 68 kilobytes (2757 lines, including comments).
The smallest files
were the plots for \cdf{atan} and \cdf{atanh}, about 65 kilobytes apiece;
the largest were the plots for \cdf{sin}, \cdf{cos}, \cdf{sinh}, and \cdf{cosh},
about 138 kilobytes apiece.

\begingroup\small  \topsep 0pt plus 6pt \relax
\begin{lisp}
\\
\% PostScript file for plot of function IDENTITY \\
\% Plot is to fit in a region 4.666666666666667 inches square \\
\% ~showing axes extending 4.1 units from the origin. \\
 \\
40.97560975609756 40.97560975609756 scale \\
4.1 4.1 translate \\
newpath \\
~~-4.1 -4.1 moveto \\
~~4.1 -4.1 lineto \\
~~4.1 4.1 lineto \\
~~-4.1 4.1 lineto \\
~~closepath \\
clip \\
\% Moby grid for function IDENTITY \\
\% Annulus 0.25 0.5 4 0.97 0.45 \\
\% Sector from 4.7124 to 6.2832 (quadrant 3) \\
newpath \\
~~0.0 -0.25 moveto \\
~~0.0 -0.375 lineto \\
~~\%middle radial \\
~~0.0 -0.375 lineto \\
~~0.0 -0.5 lineto \\
~~\%end radial \\
~~0.0 -0.5 lineto \\
~~0.092 -0.4915 lineto \\
~~0.1843 -0.4648 lineto \\
~~0.273 -0.4189 lineto \\
~~0.3536 -0.3536 lineto \\
~~\%middle circumferential \\
~~0.3536 -0.3536 lineto \\
~~0.413 -0.2818 lineto \\
~~0.4594 -0.1974 lineto \\
~~0.4894 -0.1024 lineto \\
~~0.5 0.0 lineto \\
~~\%end circumferential \\
~~0.5 0.0 lineto \\
~~0.375 0.0 lineto \\
~~\%middle radial \\
~~0.375 0.0 lineto \\
~~0.25 0.0 lineto \\
~~\%end radial \\
~~0.25 0.0 lineto \\
~~0.2297 -0.0987 lineto \\
~~0.1768 -0.1768 lineto \\
~~\%middle circumferential \\
~~0.1768 -0.1768 lineto \\
~~0.0922 -0.2324 lineto \\
~~0.0 -0.25 lineto \\
~~\%end circumferential \\
~~closepath \\
currentgray~~~0.45 setgray~~~fill~~~setgray
\end{lisp}
\textrm{{\lbrack}2598 lines omitted{\rbrack}}
\begin{lisp}
\% Vertical line from (0.0, -0.5) to (0.0, 0.0) \\
newpath \\
~~0.0 -0.5 moveto \\
~~0.0 0.0 lineto \\
0.05 setlinewidth~~~1 setlinecap~~stroke \\
\% Vertical line from (0.0, -0.5) to (0.0, -1.0) \\
newpath \\
~~0.0 -0.5 moveto \\
~~0.0 -1.0 lineto \\
0.05 setlinewidth~~~1 setlinecap~~stroke \\
\% Vertical line from (0.0, -2.0) to (0.0, -1.0) \\
newpath \\
~~0.0 -2.0 moveto \\
~~0.0 -1.0 lineto \\
0.05 setlinewidth~~~1 setlinecap~~stroke \\
\% Vertical line from (0.0, -2.0) to (0.0, -1.1579208923731617E77) \\
newpath \\
~~0.0 -2.0 moveto \\
~~0.0 -6.3553 lineto \\
~~0.0 -6.378103166302659 lineto \\
~~0.0 -6.378103166302659 lineto \\
~~0.0 -6.378103166302659 lineto \\
0.05 setlinewidth   1 setlinecap  stroke
\end{lisp}
\textrm{{\lbrack}84 lines omitted{\rbrack}}
\begin{lisp}
\% End of PostScript file for plot of function IDENTITY
\end{lisp}
\endgroup

Here is the program that generated the PostScript code for
the graphs shown in figures~\ref{IDENTITY-PLOT} through~\ref{LAST-PLOT}.
It contains a mixture of fairly general mechanisms and \emph{ad hoc} kludges
for plotting functions of a single complex argument while gracefully handling
extremely large and small values,
branch cuts, singularities, and periodic behavior.
The aim was to provide a simple user interface that would not
require the caller to provide special advice for each function
to be plotted.
The file for figure~\ref{IDENTITY-PLOT}, for example, was generated
by the call \cd{(picture~'identity)}, which resulted in the writing of
a file named \cd{identity-plot.ps}.

The program assumes that any periodic behavior will have a period that is a multiple of
$2\pi$; that branch cuts will fall along the real or imaginary axis;
and that singularities or very large or small values
will occur only at the origin, at $\pm 1$ or $\pm \emph{i}$,
or on the boundaries of the annuli (particularly those with radius $\pi/2$ or $\pi$).
The central function is \cdf{parametric-path}, which accepts four arguments:
two real numbers that are the endpoints of an interval of real numbers,
a function that maps this interval into a path in the complex plane,
and the function to be plotted; the task of \cdf{parametric-path} is to
generate PostScript code (a series of \cdf{lineto} operations)
that will plot an approximation to the image of the parametric path
as transformed by the function to be plotted.
Each of the functions \cdf{hline}, \cdf{vline}, \cdf{-hline}, \cdf{-vline}, \cdf{radial},
and \cdf{circumferential} takes appropriate parameters
and returns a function suitable for use as the third argument
to \cdf{parametric-path}.
There is some code that defends against errors
(by using \cdf{ignore-errors}) and against certain peculiarities of
IEEE floating-point arithmetic (the code that checks for not-a-number (NaN) results).

The program is offered here without further comment or apology.

\vskip 0pt plus 10pt%manual
\hrule width 0pt\relax

\begingroup\small \topsep 0pt plus 10pt \relax
\begin{lisp}
\\
(defparameter units-to-show 4.1) \\*
(defparameter text-width-in-picas 28.0) \\*
(defparameter device-pixels-per-inch 300) \\*
(defparameter pixels-per-unit \\*
~~(* (/ (/ text-width-in-picas 6) \\*
~~~~~~~~(* units-to-show 2)) \\*
~~~~~device-pixels-per-inch)) \\
\\
(defparameter big (sqrt (sqrt most-positive-single-float))) \\*
(defparameter tiny (sqrt (sqrt least-positive-single-float))) \\
\\
(defparameter path-really-losing 1000.0) \\
(defparameter path-outer-limit (* units-to-show (sqrt 2) 1.1)) \\
(defparameter path-minimal-delta (/ 10 pixels-per-unit)) \\
(defparameter path-outer-delta (* path-outer-limit 0.3)) \\
(defparameter path-relative-closeness 0.00001) \\*
(defparameter back-off-delta 0.0005)
\end{lisp}
\newpage%manual
\begin{lisp}
(defun comment-line (stream \&rest stuff) \\*
~~(format stream "{\Xtilde}\%\% ") \\*
~~(apply \#'format stream stuff) \\*
~~(format t "{\Xtilde}\%\% ") \\*
~~(apply \#'format t stuff)) \\
\\
(defun parametric-path (from to paramfn plotfn) \\*
~~(assert (and (plusp from) (plusp to))) \\*
~~(flet ((domainval (x) (funcall paramfn x)) \\*
~~~~~~~~~(rangeval (x) (funcall plotfn (funcall paramfn x))) \\*
~~~~~~~~~(losing (x) (or (null x) \\*
~~~~~~~~~~~~~~~~~~~~~~~~~(/= (realpart x) (realpart x))~~;NaN? \\*
~~~~~~~~~~~~~~~~~~~~~~~~~(/= (imagpart x) (imagpart x))~~;NaN? \\*
~~~~~~~~~~~~~~~~~~~~~~~~~(> (abs (realpart x)) path-really-losing) \\*
~~~~~~~~~~~~~~~~~~~~~~~~~(> (abs (imagpart x)) path-really-losing)))) \\
~~~~(when (> to 1000.0) \\*
~~~~~~(let ((f0 (rangeval from)) \\*
~~~~~~~~~~~~(f1 (rangeval (+ from 1))) \\*
~~~~~~~~~~~~(f2 (rangeval (+ from (* 2 pi)))) \\*
~~~~~~~~~~~~(f3 (rangeval (+ from 1 (* 2 pi)))) \\*
~~~~~~~~~~~~(f4 (rangeval (+ from (* 4 pi))))) \\
~~~~~~~~(flet ((close (x y) \\*
~~~~~~~~~~~~~~~~~(or (< (careful-abs (- x y)) path-minimal-delta) \\*
~~~~~~~~~~~~~~~~~~~~~(< (careful-abs (- x y)) \\*
~~~~~~~~~~~~~~~~~~~~~~~~(* (+ (careful-abs x) (careful-abs y)) \\*
~~~~~~~~~~~~~~~~~~~~~~~~~~~path-relative-closeness))))) \\
~~~~~~~~~~(when (and (close f0 f2) \\*
~~~~~~~~~~~~~~~~~~~~~(close f2 f4) \\*
~~~~~~~~~~~~~~~~~~~~~(close f1 f3) \\*
~~~~~~~~~~~~~~~~~~~~~(or (and (close f0 f1) \\*
~~~~~~~~~~~~~~~~~~~~~~~~~~~~~~(close f2 f3)) \\*
~~~~~~~~~~~~~~~~~~~~~~~~~(and (not (close f0 f1)) \\*
~~~~~~~~~~~~~~~~~~~~~~~~~~~~~~(not (close f2 f3))))) \\*
~~~~~~~~~~~~(format t "{\Xtilde}\&Periodicity detected.") \\*
~~~~~~~~~~~~(setq to (+ from (* (signum (- to from)) 2 pi))))))) \\
~~~~~(let ((fromrange (ignore-errors (rangeval from))) \\*
~~~~~~~~~~(torange (ignore-errors (rangeval to)))) \\*
~~~~~~(if (losing fromrange) \\*
~~~~~~~~~~(if (losing torange) \\*
~~~~~~~~~~~~~~'() \\*
~~~~~~~~~~~~~~(parametric-path (back-off from to) to paramfn plotfn)) \\
~~~~~~~~~~(if (losing torange) \\*
~~~~~~~~~~~~~~(parametric-path from (back-off to from) paramfn plotfn) \\*
~~~~~~~~~~~~~~(expand-path (refine-path (list from to) \#'rangeval) \\*
~~~~~~~~~~~~~~~~~~~~~~~~~~~\#'rangeval))))))
\end{lisp}
\vskip 0pt plus 10pt \newpage%manual
\begin{lisp}
(defun back-off (point other) \\*
~~(if (or (> point 10.0) (< point 0.1)) \\*
~~~~~~(let ((sp (sqrt point))) \\*
~~~~~~~~(if (or (> point sp other) (< point sp other)) \\*
~~~~~~~~~~~~sp \\*
~~~~~~~~~~~~(* sp (sqrt other)))) \\*
~~~~~~(+ point (* (signum (- other point)) back-off-delta)))) \\
\\
(defun careful-abs (z) \\*
~~(cond ((or (> (realpart z) big) \\*
~~~~~~~~~~~~~(< (realpart z) (- big)) \\*
~~~~~~~~~~~~~(> (imagpart z) big) \\*
~~~~~~~~~~~~~(< (imagpart z) (- big))) \\*
~~~~~~~~~big) \\
~~~~~~~~((complexp z) (abs z)) \\*
~~~~~~~~((minusp z) (- z)) \\*
~~~~~~~~(t z))) \\      %intentional extra \\
\end{lisp}
\begin{lisp}
(defparameter max-refinements 5000) \\
\\
(defun refine-path (original-path rangevalfn) \\*
~~(flet ((rangeval (x) (funcall rangevalfn x))) \\*
~~~~(let ((path original-path)) \\*
~~~~~~(do ((j 0 (+ j 1))) \\*
~~~~~~~~~~((null (rest path))) \\
~~~~~~~~(when (zerop (mod (+ j 1) max-refinements)) \\*
~~~~~~~~~~~~~~(break "Runaway path")) \\
~~~~~~~~(let* ((from (first path)) \\*
~~~~~~~~~~~~~~~(to (second path)) \\*
~~~~~~~~~~~~~~~(fromrange (rangeval from)) \\*
~~~~~~~~~~~~~~~(torange (rangeval to)) \\*
~~~~~~~~~~~~~~~(dist (careful-abs (- torange fromrange))) \\*
~~~~~~~~~~~~~~~(mid (* (sqrt from) (sqrt to))) \\*
~~~~~~~~~~~~~~~(midrange (rangeval mid))) \\
~~~~~~~~~~(cond ((or (and (far-out fromrange) (far-out torange)) \\*
~~~~~~~~~~~~~~~~~~~~~(and (< dist path-minimal-delta) \\*
~~~~~~~~~~~~~~~~~~~~~~~~~~(< (abs (- midrange fromrange)) \\*
~~~~~~~~~~~~~~~~~~~~~~~~~~~~~path-minimal-delta) \\*
~~~~~~~~~~~~~~~~~~~~~~~~~~;; Next test is intentionally asymmetric to \\*
~~~~~~~~~~~~~~~~~~~~~~~~~~;;~~avoid problems with periodic functions. \\*
~~~~~~~~~~~~~~~~~~~~~~~~~~(< (abs (- (rangeval (/ (+ to (* from 1.5)) \\*
~~~~~~~~~~~~~~~~~~~~~~~~~~~~~~~~~~~~~~~~~~~~~~~~~~2.5)) \\*
~~~~~~~~~~~~~~~~~~~~~~~~~~~~~~~~~~~~~fromrange)) \\*
~~~~~~~~~~~~~~~~~~~~~~~~~~~~~path-minimal-delta))) \\*
~~~~~~~~~~~~~~~~~(pop path)) \\
~~~~~~~~~~~~~~~~((= mid from) (pop path)) \\*
~~~~~~~~~~~~~~~~((= mid to) (pop path)) \\*
~~~~~~~~~~~~~~~~(t (setf (rest path) (cons mid (rest path))))))))) \\*
~~original-path) \\
\\
(defun expand-path (path rangevalfn) \\*
~~(flet ((rangeval (x) (funcall rangevalfn x))) \\*
~~~~(let ((final-path (list (rangeval (first path))))) \\*
~~~~~~(do ((p (rest path) (cdr p))) \\*
~~~~~~~~~~((null p) \\*
~~~~~~~~~~~(unless (rest final-path) \\*
~~~~~~~~~~~~~(break "Singleton path")) \\*
~~~~~~~~~~~(reverse final-path)) \\
~~~~~~~~(let ((v (rangeval (car p)))) \\*
~~~~~~~~~~(cond ((and (rest final-path) \\*
~~~~~~~~~~~~~~~~~~~~~~(not (far-out v)) \\*
~~~~~~~~~~~~~~~~~~~~~~(not (far-out (first final-path))) \\*
~~~~~~~~~~~~~~~~~~~~~~(between v (first final-path) \\*
~~~~~~~~~~~~~~~~~~~~~~~~~~~~~~~~~(second final-path))) \\*
~~~~~~~~~~~~~~~~~(setf (first final-path) v)) \\
~~~~~~~~~~~~~~~~((null (rest p))~~~;Mustn't omit last point \\*
~~~~~~~~~~~~~~~~~(push v final-path)) \\
~~~~~~~~~~~~~~~~((< (abs (- v (first final-path))) path-minimal-delta)) \\
~~~~~~~~~~~~~~~~((far-out v) \\*
~~~~~~~~~~~~~~~~~(unless (and (far-out (first final-path)) \\*
~~~~~~~~~~~~~~~~~~~~~~~~~~~~~~(< (abs (- v (first final-path))) \\*
~~~~~~~~~~~~~~~~~~~~~~~~~~~~~~~~~path-outer-delta)) \\*
~~~~~~~~~~~~~~~~~~~(push (* 1.01 path-outer-limit (signum v)) \\*
~~~~~~~~~~~~~~~~~~~~~~~~~final-path))) \\*
~~~~~~~~~~~~~~~~(t (push v final-path)))))))) \\
\\
(defun far-out (x) \\*
~~(> (careful-abs x) path-outer-limit)) \\
\\
(defparameter between-tolerance 0.000001) \\
\\
(defun between (p q r) \\*
~~(let ((px (realpart p)) (py (imagpart p)) \\*
~~~~~~~~(qx (realpart q)) (qy (imagpart q)) \\*
~~~~~~~~(rx (realpart r)) (ry (imagpart r))) \\
~~~~(and (or (<= px qx rx) (>= px qx rx)) \\*
~~~~~~~~~(or (<= py qy ry) (>= py qy ry)) \\*
~~~~~~~~~(< (abs (- (* (- qx px) (- ry qy)) \\*
~~~~~~~~~~~~~~~~~~~~(* (- rx qx) (- qy py)))) \\*
~~~~~~~~~~~~between-tolerance))))
\end{lisp}
\vskip 0pt plus 10pt \newpage%manual
\begin{lisp}
(defun circle (radius) \\*
~~\#'(lambda (angle) (* radius (cis angle)))) \\
\\
(defun hline (imag) \\*
~~\#'(lambda (real) (complex real imag))) \\
\\
(defun vline (real) \\*
~~\#'(lambda (imag) (complex real imag))) \\
\\
(defun -hline (imag) \\*
~~\#'(lambda (real) (complex (- real) imag))) \\
\\
(defun -vline (real) \\*
~~\#'(lambda (imag) (complex real (- imag)))) \\
\\
(defun radial (phi quadrant) \\*
~~\#'(lambda (rho) (repair-quadrant (* rho (cis phi)) quadrant))) \\
\\
(defun circumferential (rho quadrant) \\*
~~\#'(lambda (phi) (repair-quadrant (* rho (cis phi)) quadrant))) \\
\\
;;; Quadrant is 0, 1, 2, or 3, meaning I, II, III, or IV. \\
\\
(defun repair-quadrant (z quadrant) \\*
~~(complex (* (+ (abs (realpart z)) tiny) \\*
~~~~~~~~~~~~~~(case quadrant (0 1.0) (1 -1.0) (2 -1.0) (3 1.0))) \\*
~~~~~~~~~~~(* (+ (abs (imagpart z)) tiny) \\*
~~~~~~~~~~~~~~(case quadrant (0 1.0) (1 1.0) (2 -1.0) (3 -1.0))))) \\
\\
(defun clamp-real (x) \\*
~~(if (far-out x) \\*
~~~~~~(* (signum x) path-outer-limit) \\*
~~~~~~(round-real x))) \\
\\
(defun round-real (x) \\*
~~(/ (round (* x 10000.0)) 10000.0)) \\
\\
(defun round-point (z) \\*
~~(complex (round-real (realpart z)) (round-real (imagpart z)))) \\
\\
(defparameter hiringshade 0.97) \\*
(defparameter loringshade 0.45) \\
\\
(defparameter ticklength 0.12) \\*
(defparameter smallticklength 0.09)
\end{lisp}
\vskip 0pt plus 10pt \newpage%manual
\begin{lisp}
;;; This determines the pattern of lines and annuli to be drawn. \\*
(defun moby-grid (\&optional (fn 'sqrt) (stream t)) \\*
~~(comment-line stream "Moby grid for function {\Xtilde}S" fn) \\
~~(shaded-annulus 0.25 0.5 4 hiringshade loringshade fn stream) \\*
~~(shaded-annulus 0.75 1.0 8 hiringshade loringshade fn stream) \\*
~~(shaded-annulus (/ pi 2) 2.0 16 hiringshade loringshade fn stream) \\*
~~(shaded-annulus 3 pi 32 hiringshade loringshade fn stream) \\
~~(moby-lines :horizontal 1.0 fn stream) \\*
~~(moby-lines :horizontal -1.0 fn stream) \\*
~~(moby-lines :vertical 1.0 fn stream) \\*
~~(moby-lines :vertical -1.0 fn stream) \\
~~(let ((tickline 0.015) \\*
~~~~~~~~(axisline 0.008)) \\
~~~~(flet ((tick (n) (straight-line (complex n ticklength) \\*
~~~~~~~~~~~~~~~~~~~~~~~~~~~~~~~~~~~~(complex n (- ticklength)) \\*
~~~~~~~~~~~~~~~~~~~~~~~~~~~~~~~~~~~~tickline \\*
~~~~~~~~~~~~~~~~~~~~~~~~~~~~~~~~~~~~stream)) \\
~~~~~~~~~~~(smalltick (n) (straight-line (complex n smallticklength) \\*
~~~~~~~~~~~~~~~~~~~~~~~~~~~~~~~~~~~~~~~~~(complex n (- smallticklength)) \\*
~~~~~~~~~~~~~~~~~~~~~~~~~~~~~~~~~~~~~~~~~tickline \\*
~~~~~~~~~~~~~~~~~~~~~~~~~~~~~~~~~~~~~~~~~stream))) \\
~~~~~~(comment-line stream "Real axis") \\*
~~~~~~(straight-line \#c(-5 0) \#c(5 0) axisline stream) \\*
~~~~~~(dotimes (j (floor units-to-show)) \\*
~~~~~~~~(let ((q (+ j 1))) (tick q) (tick (- q)))) \\
~~~~~~(dotimes (j (floor units-to-show (/ pi 2))) \\*
~~~~~~~~(let ((q (* (/ pi 2) (+ j 1)))) \\*
~~~~~~~~~~(smalltick q) \\*
~~~~~~~~~~(smalltick (- q))))) \\
~~~~(flet ((tick (n) (straight-line (complex ticklength n) \\*
~~~~~~~~~~~~~~~~~~~~~~~~~~~~~~~~~~~~(complex (- ticklength) n) \\*
~~~~~~~~~~~~~~~~~~~~~~~~~~~~~~~~~~~~tickline \\*
~~~~~~~~~~~~~~~~~~~~~~~~~~~~~~~~~~~~stream)) \\
~~~~~~~~~~~(smalltick (n) (straight-line (complex smallticklength n) \\*
~~~~~~~~~~~~~~~~~~~~~~~~~~~~~~~~~~~~~~~~~(complex (- smallticklength) n) \\*
~~~~~~~~~~~~~~~~~~~~~~~~~~~~~~~~~~~~~~~~~tickline \\*
~~~~~~~~~~~~~~~~~~~~~~~~~~~~~~~~~~~~~~~~~stream))) \\
~~~~~~(comment-line stream "Imaginary axis") \\*
~~~~~~(straight-line \#c(0 -5) \#c(0 5) axisline stream) \\*
~~~~~~(dotimes (j (floor units-to-show)) \\*
~~~~~~~~(let ((q (+ j 1))) (tick q) (tick (- q)))) \\
~~~~~~(dotimes (j (floor units-to-show (/ pi 2))) \\*
~~~~~~~~(let ((q (* (/ pi 2) (+ j 1)))) \\*
~~~~~~~~~~(smalltick q) \\*
~~~~~~~~~~(smalltick (- q)))))))
\end{lisp}
\vskip 0pt plus 10pt \newpage%manual
\begin{lisp}
(defun straight-line (from to wid stream) \\*
~~(format stream \\*
~~~~~~~~~~"{\Xtilde}\%newpath~~{\Xtilde}S {\Xtilde}S moveto~~{\Xtilde}S {\Xtilde}S lineto~~{\Xtilde}S {\Xtilde} \\
~~~~~~~~~~~setlinewidth~~1~~setlinecap~~stroke" \\*
~~~~~~~~~~(realpart from) \\*
~~~~~~~~~~(imagpart from) \\*
~~~~~~~~~~(realpart to) \\*
~~~~~~~~~~(imagpart to) \\*
~~~~~~~~~~wid)) \\
\end{lisp}
\begin{lisp}
;;; This function draws the lines for the pattern. \\
(defun moby-lines (orientation signum plotfn stream) \\*
~~(let ((paramfn (ecase orientation \\*
~~~~~~~~~~~~~~~~~~~(:horizontal (if (< signum 0) \#'-hline \#'hline)) \\*
~~~~~~~~~~~~~~~~~~~(:vertical (if (< signum 0) \#'-vline \#'vline))))) \\
~~~~(flet ((foo (from to other wid) \\*
~~~~~~~~~~~~~(ecase orientation \\*
~~~~~~~~~~~~~~~(:horizontal \\*
~~~~~~~~~~~~~~~~(comment-line stream \\*
~~~~~~~~~~~~~~~~~~~~~~~~~~~~~~"Horizontal line from ({\Xtilde}S, {\Xtilde}S) to ({\Xtilde}S, {\Xtilde}S)" \\*
~~~~~~~~~~~~~~~~~~~~~~~~~~~~~~(round-real (* signum from)) \\*
~~~~~~~~~~~~~~~~~~~~~~~~~~~~~~(round-real other) \\*
~~~~~~~~~~~~~~~~~~~~~~~~~~~~~~(round-real (* signum to)) \\*
~~~~~~~~~~~~~~~~~~~~~~~~~~~~~~(round-real other))) \\
~~~~~~~~~~~~~~~(:vertical \\*
~~~~~~~~~~~~~~~~(comment-line stream \\*
~~~~~~~~~~~~~~~~~~~~~~~~~~~~~~"Vertical line from ({\Xtilde}S, {\Xtilde}S) to ({\Xtilde}S, {\Xtilde}S)" \\*
~~~~~~~~~~~~~~~~~~~~~~~~~~~~~~(round-real other) \\*
~~~~~~~~~~~~~~~~~~~~~~~~~~~~~~(round-real (* signum from)) \\*
~~~~~~~~~~~~~~~~~~~~~~~~~~~~~~(round-real other) \\*
~~~~~~~~~~~~~~~~~~~~~~~~~~~~~~(round-real (* signum to))))) \\
~~~~~~~~~~~~~(postscript-path \\*
~~~~~~~~~~~~~~~stream \\*
~~~~~~~~~~~~~~~(parametric-path from \\*
~~~~~~~~~~~~~~~~~~~~~~~~~~~~~~~~to \\*
~~~~~~~~~~~~~~~~~~~~~~~~~~~~~~~~(funcall paramfn other) \\*
~~~~~~~~~~~~~~~~~~~~~~~~~~~~~~~~plotfn)) \\*
~~~~~~~~~~~~~(postscript-penstroke stream wid))) \\
~~~~~~(let* ((thick 0.05) \\*
~~~~~~~~~~~~~(thin 0.02)) \\*
~~~~~~~~;; Main axis \\*
~~~~~~~~(foo 0.5 tiny 0.0 thick) \\*
~~~~~~~~(foo 0.5 1.0 0.0 thick) \\*
~~~~~~~~(foo 2.0 1.0 0.0 thick) \\*
~~~~~~~~(foo 2.0 big 0.0 thick) \\
~~~~~~~~;; Parallels at 1 and -1 \\*
~~~~~~~~(foo 2.0 tiny 1.0 thin) \\*
~~~~~~~~(foo 2.0 big 1.0 thin) \\*
~~~~~~~~(foo 2.0 tiny -1.0 thin) \\*
~~~~~~~~(foo 2.0 big -1.0 thin) \\
~~~~~~~~;; Parallels at 2, 3, -2, -3 \\*
~~~~~~~~(foo tiny big 2.0 thin) \\*
~~~~~~~~(foo tiny big -2.0 thin) \\*
~~~~~~~~(foo tiny big 3.0 thin) \\*
~~~~~~~~(foo tiny big -3.0 thin))))) \\
\\
(defun splice (p q) \\*
~~(let ((v (car (last p))) \\*
~~~~~~~~(w (first q))) \\
~~~~(and (far-out v) \\*
~~~~~~~~~(far-out w) \\*
~~~~~~~~~(>= (abs (- v w)) path-outer-delta) \\
~~~~~~~~~;; Two far-apart far-out points.~~Try to walk around \\*
~~~~~~~~~;;~~outside the perimeter, in the shorter direction. \\*
~~~~~~~~~(let* ((pdiff (phase (/ v w))) \\*
~~~~~~~~~~~~~~~~(npoints (floor (abs pdiff) (asin .2))) \\*
~~~~~~~~~~~~~~~~(delta (/ pdiff (+ npoints 1))) \\*
~~~~~~~~~~~~~~~~(incr (cis delta))) \\
~~~~~~~~~~~(do ((j 0 (+ j 1)) \\*
~~~~~~~~~~~~~~~~(p (list w "end splice") (cons (* (car p) incr) p))) \\*
~~~~~~~~~~~~~~~((= j npoints) (cons "start splice" p))))))) \\
\end{lisp}
\begin{lisp}
;;; This function draws the annuli for the pattern. \\
(defun shaded-annulus (inner outer sectors firstshade lastshade fn stream) \\*
~~(assert (zerop (mod sectors 4))) \\*
~~(comment-line stream "Annulus {\Xtilde}S {\Xtilde}S {\Xtilde}S {\Xtilde}S {\Xtilde}S" \\*
~~~~~~~~~~~~~~~~(round-real inner) (round-real outer) \\*
~~~~~~~~~~~~~~~~sectors firstshade lastshade) \\
~~(dotimes (jj sectors) \\*
~~~~(let ((j (- sectors jj 1))) \\
~~~~~~(let* ((lophase (+ tiny (* 2 pi (/ j sectors)))) \\*
~~~~~~~~~~~~~(hiphase (* 2 pi (/ (+ j 1) sectors))) \\*
~~~~~~~~~~~~~(midphase (/ (+ lophase hiphase) 2.0)) \\*
~~~~~~~~~~~~~(midradius (/ (+ inner outer) 2.0)) \\*
~~~~~~~~~~~~~(quadrant (floor (* j 4) sectors))) \\
~~~~~~~~(comment-line stream "Sector from {\Xtilde}S to {\Xtilde}S (quadrant {\Xtilde}S)" \\*
~~~~~~~~~~~~~~~~~~~~~~(round-real lophase) \\*
~~~~~~~~~~~~~~~~~~~~~~(round-real hiphase) \\*
~~~~~~~~~~~~~~~~~~~~~~quadrant) \\
~~~~~~~~(let ((p0 (reverse (parametric-path midradius \\*
~~~~~~~~~~~~~~~~~~~~~~~~~~~~~~~~~~~~~~~~~~~~inner \\*
~~~~~~~~~~~~~~~~~~~~~~~~~~~~~~~~~~~~~~~~~~~~(radial lophase quadrant) \\*
~~~~~~~~~~~~~~~~~~~~~~~~~~~~~~~~~~~~~~~~~~~~fn))) \\
~~~~~~~~~~~~~~(p1 (parametric-path midradius \\*
~~~~~~~~~~~~~~~~~~~~~~~~~~~~~~~~~~~outer \\*
~~~~~~~~~~~~~~~~~~~~~~~~~~~~~~~~~~~(radial lophase quadrant) \\*
~~~~~~~~~~~~~~~~~~~~~~~~~~~~~~~~~~~fn)) \\*
~~~~~~~~~~~~~~(p2 (reverse (parametric-path midphase \\*
~~~~~~~~~~~~~~~~~~~~~~~~~~~~~~~~~~~~~~~~~~~~lophase \\*
~~~~~~~~~~~~~~~~~~~~~~~~~~~~~~~~~~~~~~~~~~~~(circumferential outer \\*
~~~~~~~~~~~~~~~~~~~~~~~~~~~~~~~~~~~~~~~~~~~~~~~~~~~~~~~~~~~~~quadrant) \\*
~~~~~~~~~~~~~~~~~~~~~~~~~~~~~~~~~~~~~~~~~~~~fn))) \\
~~~~~~~~~~~~~~(p3 (parametric-path midphase \\*
~~~~~~~~~~~~~~~~~~~~~~~~~~~~~~~~~~~hiphase \\*
~~~~~~~~~~~~~~~~~~~~~~~~~~~~~~~~~~~(circumferential outer quadrant) \\*
~~~~~~~~~~~~~~~~~~~~~~~~~~~~~~~~~~~fn)) \\
~~~~~~~~~~~~~~(p4 (reverse (parametric-path midradius \\*
~~~~~~~~~~~~~~~~~~~~~~~~~~~~~~~~~~~~~~~~~~~~outer \\*
~~~~~~~~~~~~~~~~~~~~~~~~~~~~~~~~~~~~~~~~~~~~(radial hiphase quadrant) \\*
~~~~~~~~~~~~~~~~~~~~~~~~~~~~~~~~~~~~~~~~~~~~fn))) \\
~~~~~~~~~~~~~~(p5 (parametric-path midradius \\*
~~~~~~~~~~~~~~~~~~~~~~~~~~~~~~~~~~~inner \\*
~~~~~~~~~~~~~~~~~~~~~~~~~~~~~~~~~~~(radial hiphase quadrant) \\*
~~~~~~~~~~~~~~~~~~~~~~~~~~~~~~~~~~~fn)) \\
~~~~~~~~~~~~~~(p6 (reverse (parametric-path midphase \\*
~~~~~~~~~~~~~~~~~~~~~~~~~~~~~~~~~~~~~~~~~~~~hiphase \\*
~~~~~~~~~~~~~~~~~~~~~~~~~~~~~~~~~~~~~~~~~~~~(circumferential inner \\*
~~~~~~~~~~~~~~~~~~~~~~~~~~~~~~~~~~~~~~~~~~~~~~~~~~~~~~~~~~~~~quadrant) \\*
~~~~~~~~~~~~~~~~~~~~~~~~~~~~~~~~~~~~~~~~~~~~fn))) \\
~~~~~~~~~~~~~~(p7 (parametric-path midphase \\*
~~~~~~~~~~~~~~~~~~~~~~~~~~~~~~~~~~~lophase \\*
~~~~~~~~~~~~~~~~~~~~~~~~~~~~~~~~~~~(circumferential inner quadrant) \\*
~~~~~~~~~~~~~~~~~~~~~~~~~~~~~~~~~~~fn))) \\
~~~~~~~~~~(postscript-closed-path stream \\*
~~~~~~~~~~~~(append \\*
~~~~~~~~~~~~~~p0 (splice p0 p1) '("middle radial") \\*
~~~~~~~~~~~~~~p1 (splice p1 p2) '("end radial") \\
~~~~~~~~~~~~~~p2 (splice p2 p3) '("middle circumferential") \\
~~~~~~~~~~~~~~p3 (splice p3 p4) '("end circumferential") \\
~~~~~~~~~~~~~~p4 (splice p4 p5) '("middle radial") \\
~~~~~~~~~~~~~~p5 (splice p5 p6) '("end radial") \\
~~~~~~~~~~~~~~p6 (splice p6 p7) '("middle circumferential") \\
~~~~~~~~~~~~~~p7 (splice p7 p0) '("end circumferential") \\*
~~~~~~~~~~~~~~)))
\end{lisp}
\vskip 0pt plus 10pt \newpage%manual
\begin{lisp}
~~~~~~~~(postscript-shade stream \\*
~~~~~~~~~~~~~~~~~~~~~~~~~~(/ (+ (* firstshade (- (- sectors 1) j)) \\*
~~~~~~~~~~~~~~~~~~~~~~~~~~~~~~~~(* lastshade j)) \\*
~~~~~~~~~~~~~~~~~~~~~~~~~~~~~(- sectors 1))))))) \\
\\
(defun postscript-penstroke (stream wid) \\*
~~(format stream "{\Xtilde}\%{\Xtilde}S setlinewidth~~~1 setlinecap~~stroke" \\
~~~~~~~~~~wid)) \\
\\
(defun postscript-shade (stream shade) \\*
~~(format stream "{\Xtilde}\%currentgray~~~{\Xtilde}S setgray~~~fill~~~setgray" \\
~~~~~~~~~~shade)) \\
\\
(defun postscript-closed-path (stream path) \\*
~~(unless (every \#'far-out (remove-if-not \#'numberp path)) \\*
~~~~(postscript-raw-path stream path) \\*
~~~~(format stream "{\Xtilde}\%~~closepath"))) \\
\\
(defun postscript-path (stream path) \\*
~~(unless (every \#'far-out (remove-if-not \#'numberp path)) \\*
~~~~(postscript-raw-path stream path))) \\
\\
;;; Print a path as a series of PostScript "lineto" commands. \\*
(defun postscript-raw-path (stream path) \\*
~~(format stream "{\Xtilde}\%newpath") \\*
~~(let ((fmt "{\Xtilde}\%~~{\Xtilde}S {\Xtilde}S moveto")) \\*
~~~~(dolist (pt path) \\*
~~~~~~(cond ((stringp pt) \\*
~~~~~~~~~~~~~(format stream "{\Xtilde}\%~~\%{\Xtilde}A" pt)) \\
~~~~~~~~~~~~(t (format stream \\*
~~~~~~~~~~~~~~~~~~~~~~~fmt \\*
~~~~~~~~~~~~~~~~~~~~~~~(clamp-real (realpart pt)) \\*
~~~~~~~~~~~~~~~~~~~~~~~(clamp-real (imagpart pt))) \\*
~~~~~~~~~~~~~~~(setq fmt "{\Xtilde}\%~~{\Xtilde}S {\Xtilde}S lineto")))))) \\
\\
;;; Definitions of functions to be plotted that are not \\*
;;; standard Common Lisp functions. \\*
\\*
(defun one-plus-over-one-minus (x) (/ (+ 1 x) (- 1 x))) \\
\\
(defun one-minus-over-one-plus (x) (/ (- 1 x) (+ 1 x))) \\
\\
(defun sqrt-square-minus-one (x) (sqrt (- 1 (* x x)))) \\
\\
(defun sqrt-one-plus-square (x) (sqrt (+ 1 (* x x))))
\end{lisp}
\vskip 0pt plus 10pt \newpage%manual
\begin{lisp}
;;; Because X3J13 voted for a new definition of the atan function, \\*
;;; the following definition was used in place of the atan function \\*
;;; provided by the Common Lisp implementation I was using. \\*
\\*
(defun good-atan (x) \\*
~~(/ (- (log (+ 1 (* x \#c(0 1)))) \\*
~~~~~~~~(log (- 1 (* x \#c(0 1))))) \\*
~~~~~\#c(0 2))) \\
\\
;;; Because the first edition had an erroneous definition of atanh, \\*
;;; the following definition was used in place of the atanh function \\*
;;; provided by the Common Lisp implementation I was using. \\*
\\*
(defun really-good-atanh (x) \\*
~~(/ (- (log (+ 1 x)) \\
~~~~~~~~(log (- 1 x))) \\
~~~~~2)) \\
\end{lisp}
\begin{lisp}
;;; This is the main procedure that is intended to be called by a user. \\*
(defun picture (\&optional (fn \#'sqrt)) \\*
~~(with-open-file (stream (concatenate 'string \\*
~~~~~~~~~~~~~~~~~~~~~~~~~~~~~~~~~~~~~~~(string-downcase (string fn)) \\*
~~~~~~~~~~~~~~~~~~~~~~~~~~~~~~~~~~~~~~~"-plot.ps") \\*
~~~~~~~~~~~~~~~~~~~~~~~~~~:direction :output) \\
~~~~(format stream "\% PostScript file for plot of function {\Xtilde}S{\Xtilde}\%" fn) \\*
~~~~(format stream "\% Plot is to fit in a region {\Xtilde}S inches square{\Xtilde}\%" \\*
~~~~~~~~~~~~(/ text-width-in-picas 6.0)) \\
~~~~(format stream \\*
~~~~~~~~~~~~"\%~~showing axes extending {\Xtilde}S units from the origin.{\Xtilde}\%" \\*
~~~~~~~~~~~~units-to-show) \\
~~~~(let ((scaling (/ (* text-width-in-picas 12) (* units-to-show 2)))) \\*
~~~~~~(format stream "{\Xtilde}\%{\Xtilde}S {\Xtilde}:*{\Xtilde}S scale" scaling)) \\
~~~~(format stream "{\Xtilde}\%{\Xtilde}S {\Xtilde}:*{\Xtilde}S translate" units-to-show) \\
~~~~(format stream "{\Xtilde}\%newpath") \\
~~~~(format stream "{\Xtilde}\%~~{\Xtilde}S {\Xtilde}S moveto" (- units-to-show) (- units-to-show)) \\
~~~~(format stream "{\Xtilde}\%~~{\Xtilde}S {\Xtilde}S lineto" units-to-show (- units-to-show)) \\
~~~~(format stream "{\Xtilde}\%~~{\Xtilde}S {\Xtilde}S lineto" units-to-show units-to-show) \\
~~~~(format stream "{\Xtilde}\%~~{\Xtilde}S {\Xtilde}S lineto" (- units-to-show) units-to-show) \\
~~~~(format stream "{\Xtilde}\%~~closepath") \\
~~~~(format stream "{\Xtilde}\%clip") \\
~~~~(moby-grid fn stream) \\
~~~~(format stream \\*
~~~~~~~~~~~~"{\Xtilde}\%\% End of PostScript file for plot of function {\Xtilde}S" \\*
~~~~~~~~~~~~fn) \\
~~~~(terpri stream)))
\end{lisp}
\endgroup
\end{new}

\section{Type Conversions and Component Extractions on Numbers}

While most arithmetic functions will operate on any kind of number,
coercing types if necessary, the following functions are provided to
allow specific conversions of data types to be forced when desired.

\begin{defun}[Function]
float number &optional other

This converts any non-complex number to a floating-point number.
With no second argument, if \emph{number} is already a floating-point
number, then \emph{number} is returned;
otherwise a \cdf{single-float} is produced.
If the argument \emph{other} is provided, then it must be a floating-point
number, and \emph{number} is converted to the same format as \emph{other}.
See also \cdf{coerce}.
\end{defun}

\begin{defun}[Function]
rational number \\
rationalize number

Each of these functions converts any non-complex number to a rational
number.  If the argument is already rational, it is returned.
The two functions differ in their treatment of floating-point numbers.

\cdf{rational} assumes that the floating-point number is completely accurate
and returns a rational number mathematically equal to the precise
value of the floating-point number.

\cdf{rationalize} assumes that the
floating-point number is accurate only to the precision of the
floating-point representation and may return any rational number for
which the floating-point number is the best available approximation of
its format; in doing this it attempts to keep both numerator and
denominator small.

It is always the case that
\begin{lisp}
(float (rational \emph{x}) \emph{x}) \EQ\ \emph{x}
\end{lisp}
and
\begin{lisp}
(float (rationalize \emph{x}) \emph{x}) \EQ\ \emph{x}
\end{lisp}
That is, rationalizing a floating-point number by either method
and then converting it back
to a floating-point number of the same format produces the original number.
What distinguishes the two functions is that \cdf{rational} typically
has a simple, inexpensive implementation, whereas \cdf{rationalize} goes
to more trouble to produce a result that is more pleasant to view and
simpler to compute with for some purposes.
\end{defun}

\begin{defun}[Function]
numerator rational \\
denominator rational

These functions take a rational number (an integer or ratio)
and return as an integer the numerator or denominator of the canonical
reduced form of the rational.  The numerator of an integer is that integer;
the denominator of an integer is \cd{1}.  Note that
\begin{lisp}
(gcd (numerator \emph{x}) (denominator \emph{x})) \EV\ 1
\end{lisp}
The denominator will always be a strictly positive integer;
the numerator may be any integer.
For example:
\begin{lisp}
(numerator (/ 8 -6)) \EV\ -4 \\
(denominator (/ 8 -6)) \EV\ 3
\end{lisp}
\end{defun}

There is no \cdf{fix} function in Common Lisp because there are several
interesting ways to convert non-integral values to integers.
These are provided by the functions below, which perform not only
type conversion but also some non-trivial calculations as well.

\begin{defun}[Function]
floor number &optional divisor \\
ceiling number &optional divisor \\
truncate number &optional divisor \\
round number &optional divisor

In the simple one-argument case,
each of these functions converts its argument \emph{number}
(which must not be complex) to an integer.
If the argument is already an integer, it is returned directly.
If the argument is a ratio or floating-point number, the functions use
different algorithms for the conversion.

\cdf{floor} converts its argument by truncating toward negative
infinity; that is, the result is the largest integer that is not larger
than the argument.

\cdf{ceiling} converts its argument by truncating toward positive
infinity; that is, the result is the smallest integer that is not smaller
than the argument.

\cdf{truncate} converts its argument by truncating toward zero;
that is, the result is the integer of the same sign as the argument
and which has the greatest integral
magnitude not greater than that of the argument.

\cdf{round} converts its argument by rounding to the nearest
integer; if \emph{number} is exactly halfway between two integers
(that is, of the form $integer+0.5$), then it is rounded to the one that
is even (divisible by 2).

The following table shows what the four functions produce when given
various arguments.

\begin{flushleft}
\cf
\begin{tabular}{@{}ccccc@{}}
\textrm{Argument}&floor&ceiling&truncate&round\\
\hlinesp
~2.6& ~2& ~3& ~2& ~3 \\
~2.5& ~2& ~3& ~2& ~2 \\
~2.4& ~2& ~3& ~2& ~2 \\
~0.7& ~0& ~1& ~0& ~1 \\
~0.3& ~0& ~1& ~0& ~0 \\
-0.3& -1& ~0& ~0& ~0 \\
-0.7& -1& ~0& ~0& -1 \\
-2.4& -3& -2& -2& -2 \\
-2.5& -3& -2& -2& -2 \\
-2.6& -3& -2& -2& -3 \\
\hline
\end{tabular}
\end{flushleft}

If a second argument \emph{divisor} is supplied, then the result
is the appropriate type of rounding or truncation applied to the
result of dividing the \emph{number} by the \emph{divisor}.
For example, \cd{(floor 5~2)}~\EQ~\cd{(floor (/~5~2))} but is potentially more
efficient.
\begin{new}%CORR
This statement is not entirely accurate; one should instead say that
\cd{(values (floor 5~2))}~\EQ~\cd{(values (floor (/~5~2)))},
because there is a second value to consider, as discussed below.
In other words, the first values returned by the two forms will be the same, but
in general the second values will differ.  Indeed, we have
\begin{lisp}
(floor 5 2) \EV\ 2 \textrm{and} 1 \\
(floor (/ 5 2)) \EV\ 2 \textrm{and} 1/2
\end{lisp}
for this example.
\end{new}
The \emph{divisor} may be any non-complex number.
\begin{new}%CORR
It is generally accepted that it is an error for the \emph{divisor} to be zero.
\end{new}
The one-argument case is exactly like the two-argument case where the second
argument is \cd{1}.

\begin{newer}
In other words, the one-argument case returns an integer and fractional part
for the \emph{number}: \cd{(truncate 5.3) \EV\ 5.0 \textrm{and} 0.3}, for example.
\end{newer}
Each of the functions actually returns \emph{two} values,
whether given one or two arguments.  The second
result is the remainder and may be obtained using
\cdf{multiple-value-bind} and related constructs.
If any of these functions is given two arguments $\emph{x}$ and $\emph{y}$
and produces results $\emph{q}$ and $\emph{r}$, then $\emph{q} \cdot \emph{y}+\emph{r}=\emph{x}$.
The first result $\emph{q}$ is always an integer.
The remainder $\emph{r}$ is an integer if both arguments are integers,
is rational if both arguments are rational,
and is floating-point if either argument is floating-point.
One consequence is that
in the one-argument case the remainder is always a number of the same type
as the argument.

When only one argument is given, the two results are exact;
the mathematical sum of the two results is always equal to the
mathematical value of the argument.
\end{defun}

\begin{defun}[Function]
mod number divisor \\
rem number divisor

\cdf{mod} performs the operation \cdf{floor} on its two arguments
and returns the \emph{second} result of \cdf{floor} as its only result.
Similarly,
\cdf{rem} performs the operation \cdf{truncate} on its arguments
and returns the \emph{second} result of \cdf{truncate} as its only result.

\cdf{mod} and \cdf{rem} are therefore the usual modulus
and remainder functions when applied to two integer arguments.
In general, however, the arguments may be integers or floating-point
numbers.
\begin{lisp}
\hskip 0.5\textwidth\=\kill
(mod 13 4) \EV\ 1\>(rem 13 4) \EV\ 1 \\
(mod -13 4) \EV\ 3\>(rem -13 4) \EV\ -1 \\
(mod 13 -4) \EV\ -3\>(rem 13 -4) \EV\ 1 \\
(mod -13 -4) \EV\ -1\>(rem -13 -4) \EV\ -1 \\
(mod 13.4 1) \EV\ 0.4\>(rem 13.4 1) \EV\ 0.4 \\
(mod -13.4 1) \EV\ 0.6\>(rem -13.4 1) \EV\ -0.4
\end{lisp}
\end{defun}

\begin{defun}[Function]
ffloor number &optional divisor \\
fceiling number &optional divisor \\
ftruncate number &optional divisor \\
fround number &optional divisor

These functions are just like \cdf{floor}, \cdf{ceiling}, \cdf{truncate}, and
\cdf{round}, except that the result (the first result of two) is always a
floating-point number rather than an integer.  It is roughly as if
\cdf{ffloor} gave its arguments to \cdf{floor}, and then applied \cdf{float} to
the first result before passing them both back.  In practice, however,
\cdf{ffloor} may be implemented much more efficiently.  Similar remarks
apply to the other three functions.  If the first argument is a
floating-point number, and the second argument is not a floating-point
number of longer format, then the first result will be a floating-point
number of the same type as the first argument.
For example:
\begin{lisp}
(ffloor -4.7) \EV\ -5.0 and 0.3 \\
(ffloor 3.5d0) \EV\ 3.0d0 and 0.5d0
\end{lisp}
\end{defun}

\begin{defun}[Function]
decode-float float \\
scale-float float integer \\
float-radix float \\
float-sign float1 &optional float2 \\
float-digits float \\
float-precision float \\
integer-decode-float float

The function \cdf{decode-float} takes a floating-point number
and returns three values.

The first value is a new floating-point number of the same format
representing the significand; the second value is an integer
representing the exponent; and the third value is a floating-point
number of the same format indicating the sign ($-1.0$ or $1.0$).
Let \emph{b} be the radix for the floating-point representation;
then \cdf{decode-float} divides the argument by an integral power of \emph{b}
so as to bring its value between 1/\emph{b} (inclusive) and 1 (exclusive)
and returns the quotient as the first value.
If the argument is zero, however, the result
is equal to the absolute value of the argument (that is, if there is a negative
zero, its significand is considered to be a positive~zero).

The second value of \cdf{decode-float} is
the integer exponent \emph{e} to which \emph{b} must be raised
to produce the appropriate power for the division.
If the argument is zero, any integer value may be returned, provided
that the identity shown below for \cdf{scale-float} holds.

The third value of \cdf{decode-float} is a floating-point number,
of the same format as the argument, whose absolute value is 1
and whose sign matches that of the argument.

The function \cdf{scale-float} takes a floating-point number \emph{f}
(not necessarily between 1/\emph{b} and 1) and
an integer \emph{k}, and returns \cd{(* \emph{f} (expt (float \emph{b} \emph{f}) \emph{k}))}.
(The use of \cdf{scale-float} may be much more efficient than using
exponentiation and multiplication and avoids intermediate
overflow and underflow if the final result is representable.)

Note that
\begin{lisp}
(multiple-value-bind (signif expon sign) \\*
~~~~~~~~~~~~~~~~~~~~~(decode-float \emph{f}) \\*
~~(scale-float signif expon)) \\*
\EQ\ (abs \emph{f})
\end{lisp}
and
\begin{lisp}
(multiple-value-bind (signif expon sign) \\*
~~~~~~~~~~~~~~~~~~~~~(decode-float \emph{f}) \\*
~~(* (scale-float signif expon) sign)) \\*
\EQ\ \emph{f}
\end{lisp}

The function \cdf{float-radix} returns (as an integer)
the radix \emph{b} of the floating-point argument.

The function \cdf{float-sign} returns a floating-point number \emph{z} such
that \emph{z} and \emph{float1} have the same sign and also such that
\emph{z} and \emph{float2} have the same absolute value.
The argument \emph{float2} defaults to the value of \cd{(float 1 \emph{float1})};
\cd{(float-sign x)} therefore always produces a \cd{1.0} or \cd{-1.0}
of appropriate format
according to the sign of \cdf{x}.  (Note that if an implementation
has distinct representations for negative zero and positive zero,
then \cd{(float-sign -0.0)} \EV\ \cd{-1.0}.)

The function \cdf{float-digits} returns, as a non-negative integer,
the number of radix-\emph{b} digits
used in the representation of its argument (including any implicit
digits, such as a ``hidden bit'').
The function \cdf{float-precision}
returns, as a non-negative integer,
the number of significant radix-\emph{b} digits present in the
argument; if the argument is (a floating-point)
zero, then the result is (an integer) zero.
For normalized floating-point numbers, the results of \cdf{float-digits}
and \cdf{float-precision}
will be the same, but the precision will be less than the
number of representation digits for a denormalized or zero number.

The function \cdf{integer-decode-float} is similar to \cdf{decode-float}
but for its first value returns,
as an \cdf{integer}, the significand scaled so as to be an integer.
For an argument \emph{f}, this integer will be strictly less than
\begin{lisp}
\cd{(expt \emph{b} (float-precision \emph{f}))}
\end{lisp}
but no less than
\begin{lisp}
\cd{(expt \emph{b} (- (float-precision \emph{f}) 1))}
\end{lisp}
except that if \emph{f} is zero, then the integer value will be zero.

The second value bears the same relationship to the first value
as for \cdf{decode-float}:
\begin{lisp}
(multiple-value-bind (signif expon sign) \\*
~~~~~~~~~~~~~~~~~~~~~(integer-decode-float \emph{f}) \\*
~~(scale-float (float signif \emph{f}) expon)) \\*
\EQ\ (abs \emph{f})
\end{lisp}

The third value of \cdf{integer-decode-float} will be \cd{1} or \cd{-1}.

\beforenoterule
\begin{rationale}
These functions allow the writing of machine-independent,
or at least machine-parameterized, floating-point software of reasonable
efficiency.
\end{rationale}
\afternoterule
\end{defun}

\begin{defun}[Function]
complex realpart &optional imagpart

The arguments must be non-complex numbers; a number is returned
that has \emph{realpart} as its real part and \emph{imagpart} as its imaginary
part, possibly converted according to the rule of floating-point
contagion (thus both components will be of the same type).
If \emph{imagpart} is not specified,
then \cd{(coerce 0 (type-of \emph{realpart}))} is
effectively used.  Note that if both the \emph{realpart} and \emph{imagpart} are
rational and the \emph{imagpart} is zero, then the result is just the
\emph{realpart} because of the rule of canonical representation
for complex rationals.  It follows that the result of \cdf{complex}
is not always a complex number; it may be simply a \cdf{rational}.
\end{defun}

\begin{defun}[Function]
realpart number \\
imagpart number

These return the real and imaginary parts of a complex number.  If
\emph{number} is a non-complex number, then \cdf{realpart} returns its
argument \emph{number} and \cdf{imagpart}
returns \cd{(* 0 \emph{number})}, which
has the effect that the imaginary part of a rational is \cd{0} and that of
a floating-point number is a floating-point zero of the same format.

\begin{newer}
A clever way to multiply a complex number \emph{z} by \emph{i} is to write
\begin{lisp}
(complex (- (imagpart \emph{z})) (realpart \emph{z}))
\end{lisp}
instead of \cd{(*~\emph{z} \#c(0~1))}.  This cleverness is not always
gratuitous; it may be of particular importance in the presence of minus
zero.  For example, if we are using IEEE standard floating-point arithmetic
and $z=4+0\emph{i}$, the result of the clever expression is $-0+4\emph{i}$, a true
$90^\circ$ rotation of $\emph{z}$, whereas the result of \cd{(*~\emph{z} \#c(0~1))}
is likely to be
\begin{tabbing}
$ (4+0\emph{i\/})(+0+\emph{i\/}) = ((4)(+0) - (+0)(1)) + ((4)(1) + (+0)(+0))\emph{i} $ \\
\hskip2pc$ = ((+0)-(+0))+((4)+(+0))\emph{i} = +0+4\emph{i} $
\end{tabbing}
which could
land on the wrong side of a branch cut, for example.
\end{newer}
\end{defun}

\section{Logical Operations on Numbers}

The logical operations in this section require integers
as arguments; it is an error to supply a non-integer as an argument.
The functions all treat integers as if
they were represented in two's- complement notation.

\beforenoterule
\begin{implementation}
Internally, of course, an implementation of
Common Lisp may or may not use a two's-complement representation.
All that is necessary is that the logical operations
perform calculations so as to give this appearance to the user.
\end{implementation}
\afternoterule

The logical operations provide a convenient way to represent
an infinite vector of bits.  Let such a conceptual vector be
indexed by the non-negative integers.  Then bit $j$ is assigned
a ``weight'' $2^{j}$.
Assume that only a finite number of bits are 1's
or only a finite number of bits are 0's.
A vector with only a finite number of one-bits is represented
as the sum of the weights of the one-bits, a positive integer.
A vector with only a finite number of zero-bits is represented
as \cd{-1} minus the sum of the weights of the zero-bits, a negative integer.

This method of using integers to represent bit-vectors can in turn
be used to represent sets. Suppose that some (possibly countably
infinite) universe of discourse
for sets is mapped into the non-negative integers.
Then a set can be represented as a bit vector; an element is in the
set if the bit whose index corresponds to that element is a one-bit.
In this way all finite sets can be represented (by positive
integers), as well as all sets whose complement are finite
(by negative integers).  The functions \cdf{logior}, \cdf{logand},
and \cdf{logxor} defined below then compute the union,
intersection, and symmetric difference operations on sets
represented in this way.

\begin{defun}[Function]
logior &rest integers

This returns the bit-wise logical \emph{inclusive or} of its arguments.
If no argument is given, then the result is zero,
which is an identity for this operation.
\end{defun}

\begin{defun}[Function]
logxor &rest integers

This returns the bit-wise logical \emph{exclusive or} of its arguments.
If no argument is given, then the result is zero,
which is an identity for this operation.
\end{defun}

\begin{defun}[Function]
logand &rest integers

This returns the bit-wise logical \emph{and} of its arguments.
If no argument is given, then the result is \cd{-1},
which is an identity for this operation.
\end{defun}

\begin{defun}[Function]
logeqv &rest integers

This returns the bit-wise logical \emph{equivalence} (also known as \emph{exclusive nor})
of its arguments.
If no argument is given, then the result is \cd{-1},
which is an identity for this operation.
\end{defun}

\begin{defun}[Function]
lognand integer1 integer2 \\
lognor integer1 integer2 \\
logandc1 integer1 integer2 \\
logandc2 integer1 integer2 \\
logorc1 integer1 integer2 \\
logorc2 integer1 integer2

These are the other six non-trivial bit-wise logical operations
on two arguments.  Because they are not associative,
they take exactly two arguments rather than any non-negative number
of arguments.
\begin{lisp}
(lognand \emph{n1} \emph{n2}) \EQ\ (lognot (logand \emph{n1} \emph{n2})) \\[2pt]
(lognor \emph{n1} \emph{n2}) \EQ\ (lognot (logior \emph{n1} \emph{n2})) \\[2pt]
(logandc1 \emph{n1} \emph{n2}) \EQ\ (logand (lognot \emph{n1}) \emph{n2}) \\[2pt]
(logandc2 \emph{n1} \emph{n2}) \EQ\ (logand \emph{n1} (lognot \emph{n2})) \\[2pt]
(logorc1 \emph{n1} \emph{n2}) \EQ\ (logior (lognot \emph{n1}) \emph{n2}) \\[2pt]
(logorc2 \emph{n1} \emph{n2}) \EQ\ (logior \emph{n1} (lognot \emph{n2}))
\end{lisp}
\end{defun}

The ten bit-wise logical operations on two integers are summarized
in the following table:
\begin{flushleft}
\cf
\begin{tabular}{@{}rlllll@{}}
\emph{integer1}&0&0&1&1 \\
\emph{integer2}&0&1&0&1&\textrm{Operation Name} \\
\hlinesp
logand~~&0&0&0&1&\textrm{and} \\
logior~~&0&1&1&1&\textrm{inclusive or} \\
logxor~~&0&1&1&0&\textrm{exclusive or} \\
logeqv~~&1&0&0&1&\textrm{equivalence (exclusive nor)} \\
lognand~&1&1&1&0&\textrm{not-and} \\
lognor~~&1&0&0&0&\textrm{not-or} \\
logandc1&0&1&0&0&\textrm{and complement of \emph{integer1} with \emph{integer2}} \\
logandc2&0&0&1&0&\textrm{and \emph{integer1} with complement of \emph{integer2}} \\
logorc1~&1&1&0&1&\textrm{or complement of \emph{integer1} with \emph{integer2}} \\
logorc2~&1&0&1&1&\textrm{or \emph{integer1} with complement of \emph{integer2}} \\
\hline
\end{tabular}
\end{flushleft}

\begin{defun}[Function][Constant]
boole op integer1 integer2 \\
boole-clr \\
boole-set \\
boole-1 \\
boole-2 \\
boole-c1 \\
boole-c2 \\
boole-and \\
boole-ior \\
boole-xor \\
boole-eqv \\
boole-nand \\
boole-nor \\
boole-andc1 \\
boole-andc2 \\
boole-orc1 \\
boole-orc2

The function \cdf{boole} takes an operation \emph{op} and two integers,
and returns an integer produced by performing the logical operation
specified by \emph{op} on the two integers.  The precise values of
the sixteen constants are implementation-dependent, but they are
suitable for use as the first argument to \cdf{boole}:
\begin{flushleft}
\cf
\begin{tabular}{@{}rlllll@{}}
\emph{integer1}&0&0&1&1 \\
\emph{integer2}&0&1&0&1&\textrm{Operation Performed} \\
\hlinesp
boole-clr~~&0&0&0&0&\textrm{always 0} \\
boole-set~~&1&1&1&1&\textrm{always 1} \\
boole-1~~~~&0&0&1&1&\emph{integer1} \\
boole-2~~~~&0&1&0&1&\emph{integer2} \\
boole-c1~~~&1&1&0&0&\textrm{complement of \emph{integer1}} \\
boole-c2~~~&1&0&1&0&\textrm{complement of \emph{integer2}} \\
boole-and~~&0&0&0&1&\textrm{and} \\
boole-ior~~&0&1&1&1&\textrm{inclusive or} \\
boole-xor~~&0&1&1&0&\textrm{exclusive or} \\
boole-eqv~~&1&0&0&1&\textrm{equivalence (exclusive nor)} \\
boole-nand~&1&1&1&0&\textrm{not-and} \\
boole-nor~~&1&0&0&0&\textrm{not-or} \\
boole-andc1&0&1&0&0&\textrm{and complement of \emph{integer1} with \emph{integer2}} \\
boole-andc2&0&0&1&0&\textrm{and \emph{integer1} with complement of \emph{integer2}} \\
boole-orc1~&1&1&0&1&\textrm{or complement of \emph{integer1} with \emph{integer2}} \\
boole-orc2~&1&0&1&1&\textrm{or \emph{integer1} with complement of \emph{integer2}} \\
\hline
\end{tabular}
\end{flushleft}

\cdf{boole} can therefore compute all sixteen logical functions on two
arguments.  In general,
\begin{lisp}
(boole boole-and x y) \EQ\ (logand x y)
\end{lisp}
and the latter is more perspicuous.  However, \cdf{boole} is useful when it
is necessary to parameterize a procedure so that it can use
one of several logical operations.
\end{defun}

\begin{defun}[Function]
lognot integer

This returns the bit-wise logical \emph{not} of its argument.
Every bit of the result is the complement of the corresponding bit
in the argument.
\begin{lisp}
(logbitp \emph{j} (lognot \emph{x})) \EQ\ (not (logbitp \emph{j} \emph{x}))
\end{lisp}
\end{defun}

\begin{defun}[Function]
logtest integer1 integer2

\cdf{logtest} is a predicate that is true if any of
the bits designated by the 1's in \emph{integer1} are 1's in \emph{integer2}.
\begin{lisp}
(logtest \emph{x} \emph{y}) \EQ\ (not (zerop (logand \emph{x} \emph{y})))
\end{lisp}
\end{defun}

\begin{defun}[Function]
logbitp index integer

\cdf{logbitp} is true if the bit in \emph{integer} whose index
is \emph{index} (that is, its weight is $2^{index}$) is a one-bit;
otherwise it is false.
For example:
\begin{lisp}
(logbitp 2 6) \textrm{is true} \\
(logbitp 0 6) \textrm{is false} \\
(logbitp \emph{k} \emph{n}) \EQ\ (ldb-test (byte 1 \emph{k}) \emph{n})
\end{lisp}

The \emph{index} must be a non-negative integer.
\end{defun}

\begin{defun}[Function]
ash integer count

This function shifts \emph{integer} arithmetically left by \emph{count} bit
positions if \emph{count} is positive,
or right by $-\emph{count}$ bit positions if \emph{count} is negative.
The sign of the result is always the same as the sign of \emph{integer}.

Mathematically speaking, this operation performs the computation
$\emph{floor}(\emph{integer}\cdot2^{count}$).

Logically, this moves all of the bits in \emph{integer} to the left,
adding zero-bits at the bottom, or moves them to the right,
discarding bits.  (In this context the question of what gets shifted
in on the left is irrelevant; integers, viewed as strings of bits,
are ``half-infinite,'' that is, conceptually extend infinitely far to the left.)
For example:
\begin{lisp}
(logbitp \emph{j} (ash \emph{n} \emph{k})) \EQ\ (and (>= \emph{j} \emph{k}) (logbitp (- \emph{j} \emph{k}) \emph{n}))
\end{lisp}
\end{defun}

\begin{defun}[Function]
logcount integer

The number of bits in \emph{integer} is determined and returned.
If \emph{integer} is positive, the \cd{1}-bits in its binary
representation are counted.  If \emph{integer} is negative,
the \cd{0}-bits in its two's-complement binary representation are counted.
The result is always a non-negative integer.
For example:
\begin{lisp}
(logcount 13) \EV\ 3~~~~~~;\textrm{Binary representation is} ...0001101 \\
(logcount -13) \EV\ 2~~~~~;\textrm{Binary representation is} ...1110011 \\
(logcount 30) \EV\ 4~~~~~~;\textrm{Binary representation is} ...0011110 \\
(logcount -30) \EV\ 4~~~~~;\textrm{Binary representation is} ...1100010
\end{lisp}
The following identity always holds:
\begin{lisp}
(logcount x) \EQ\ (logcount (- (+ x 1))) \\
~~~~~~~~~~~~~\EQ\ (logcount (lognot x))
\end{lisp}
\end{defun}

\begin{defun}[Function]
integer-length integer

This function performs the computation
\begin{tabbing}
$ \emph{ceiling}(\log_2 (\textbf{if}\; \emph{integer} < 0 \;\textbf{then} \;
    -\emph{integer} \;\textbf{else}\; \emph{integer}+1)) $
\end{tabbing}
This is useful in two different ways.
First, if \emph{integer} is non-negative, then its value can be represented
in unsigned binary form in a field whose width in bits is
no smaller than \cd{(integer-length \emph{integer})}.
Second, regardless of the sign of \emph{integer}, its value can be
represented in signed binary two's-complement form in a field
whose width in bits is no smaller than \cd{(+ (integer-length \emph{integer}) 1)}.
For example:
\begin{lisp}
(integer-length 0) \EV\ 0 \\
(integer-length 1) \EV\ 1 \\
(integer-length 3) \EV\ 2 \\
(integer-length 4) \EV\ 3 \\
(integer-length 7) \EV\ 3 \\
(integer-length -1) \EV\ 0 \\
(integer-length -4) \EV\ 2 \\
(integer-length -7) \EV\ 3 \\
(integer-length -8) \EV\ 3
\end{lisp}
\end{defun}


\section{Byte Manipulation Functions}

Several functions are provided for dealing with an arbitrary-width field of
contiguous bits appearing anywhere in an integer.
Such a contiguous set of bits is called a \emph{byte}.
Here the term \emph{byte} does not imply some fixed number of bits
(such as eight), rather a field of arbitrary and user-specifiable width.

The byte-manipulation functions use objects called \emph{byte specifiers} to
designate a specific byte position within an integer.
The representation of a byte specifier is implementation-dependent;
in particular, it may or may not be a number.
It is sufficient to know that the function \cdf{byte} will construct one,
and that the byte-manipulation functions will accept them.
The function \cdf{byte} accepts two integers representing
the \emph{position} and \emph{size} of the byte and returns
a byte specifier.
Such a specifier designates a byte whose width is \emph{size}
and whose bits have weights $ 2^{position+size-1} $
through $ 2^{position} $.

\begin{defun}[Function]
byte size position

\cdf{byte} takes two integers representing the size and position
of a byte and returns a byte specifier suitable for use
as an argument to byte-manipulation functions.
\end{defun}

\begin{defun}[Function]
byte-size bytespec \\
byte-position bytespec

Given a byte specifier, \cdf{byte-size} returns the size specified as an
integer; \cdf{byte-position} similarly returns the position.
For example:
\begin{lisp}
(byte-size (byte \emph{j} \emph{k})) \EQ\ \emph{j} \\
(byte-position (byte \emph{j} \emph{k})) \EQ\ \emph{k}
\end{lisp}
\end{defun}

\begin{defun}[Function]
ldb bytespec integer

\emph{bytespec} specifies a byte of \emph{integer} to be extracted.
The result is returned as a non-negative integer.
For example:
\begin{lisp}
(logbitp \emph{j} (ldb (byte \emph{s} \emph{p}) \emph{n})) \EQ\ (and (< \emph{j} \emph{s}) (logbitp (+ \emph{j} \emph{p}) \emph{n}))
\end{lisp}
The name of the function \cdf{ldb} means ``load byte.''

If the argument \emph{integer} is specified by a form that is a \emph{place} form
acceptable to \cdf{setf}, then
\cdf{setf} may be used with \cdf{ldb} to modify
a byte within the integer that is stored
in that \emph{place}.
The effect is to perform a \cdf{dpb} operation
and then store the result back into the \emph{place}.
\end{defun}

\begin{defun}[Function]
ldb-test bytespec integer

\cdf{ldb-test} is a predicate that is true if any of
the bits designated by the byte specifier \emph{bytespec} are 1's in \emph{integer};
that is, it is true if the designated field is non-zero.
\begin{lisp}
(ldb-test \emph{bytespec} \emph{n}) \EQ\ (not (zerop (ldb \emph{bytespec} \emph{n})))
\end{lisp}
\end{defun}

\begin{defun}[Function]
mask-field bytespec integer

This is similar to \cdf{ldb}; however, the result contains
the specified byte
of \emph{integer} in the position specified by \emph{bytespec},
rather than in position 0 as with \cdf{ldb}.
The result therefore agrees with \emph{integer} in the byte specified
but has zero-bits everywhere else.
For example:
\begin{lisp}
(ldb \emph{bs} (mask-field \emph{bs} \emph{n})) \EQ\ (ldb \emph{bs} \emph{n}) \\
 \\
(logbitp \emph{j} (mask-field (byte \emph{s} \emph{p}) \emph{n})) \\
~~~\EQ\ (and (>= \emph{j} \emph{p}) (< \emph{j} (+ \emph{p} \emph{s})) (logbitp \emph{j} \emph{n})) \\
 \\
(mask-field \emph{bs} \emph{n}) \EQ\ (logand \emph{n} (dpb -1 \emph{bs} 0))
\end{lisp}

If the argument \emph{integer} is specified by a form that is a \emph{place} form
acceptable to \cdf{setf},
then \cdf{setf} may be used with \cdf{mask-field}
to modify a byte within the integer that is stored
in that \emph{place}.
The effect is to perform a \cdf{deposit-field} operation
and then store the result back into the \emph{place}.
\end{defun}

\begin{defun}[Function]
dpb newbyte bytespec integer

This returns a number that is the same as \emph{integer} except in the
bits specified by \emph{bytespec}.  Let \emph{s} be the size specified
by \emph{bytespec}; then the low \emph{s} bits of \emph{newbyte} appear in
the result in the byte specified by \emph{bytespec}.
The integer \emph{newbyte} is therefore interpreted as
being right-justified, as if it were the result of \cdf{ldb}.
For example:
\begin{lisp}
(logbitp \emph{j} (dpb \emph{m} (byte \emph{s} \emph{p}) \emph{n})) \\
~~\EQ\ (if \=(and (>= \emph{j} \emph{p}) (< \emph{j} (+ \emph{p} \emph{s}))) \\
\>(logbitp (- \emph{j} \emph{p}) \emph{m}) \\
\>(logbitp \emph{j} \emph{n}))
\end{lisp}
The name of the function \cdf{dpb} means ``deposit byte.''
\end{defun}

\begin{defun}[Function]
deposit-field newbyte bytespec integer

This function is to \cdf{mask-field} as \cdf{dpb} is to \cdf{ldb}.
The result is an integer that contains the bits of \emph{newbyte}
within the byte specified by \emph{bytespec}, and elsewhere contains the bits
of \emph{integer}.
For example:
\begin{lisp}
(logbitp \emph{j} (deposit-field \emph{m} (byte \emph{s} \emph{p}) \emph{n})) \\
~~~\EQ\ (if \=(and (>= \emph{j} \emph{p}) (< \emph{j} (+ \emph{p} \emph{s}))) \\
\>(logbitp \emph{j} \emph{m}) \\
\>(logbitp \emph{j} \emph{n}))
\end{lisp}

\beforenoterule
\begin{implementation}
If the \emph{bytespec} is a constant, one may of course
construct, at compile time, an equivalent mask \emph{m}, for example
by computing \cd{(deposit-field -1 \emph{bytespec} 0)}.  Given
this mask \emph{m}, one may then compute
\begin{lisp}
(deposit-field \emph{newbyte} \emph{bytespec} \emph{integer})
\end{lisp}
by computing
\begin{lisp}
(logior (logand \emph{newbyte} \emph{m}) (logand \emph{integer} (lognot \emph{m})))
\end{lisp}
where the result of \cd{(lognot \emph{m})} can of course also be computed
at compile time.  However, the following expression
may also be used and may require fewer
temporary registers in some situations:
\begin{lisp}
(logxor \emph{integer} (logand \emph{m} (logxor \emph{integer} \emph{newbyte})))
\end{lisp}
A related, though possibly less useful, trick is that
\begin{lisp}
(let ((z (logand (logxor x y) m))) \\
~~(setq x (logxor z x)) \\
~~(setq y (logxor z y)))
\end{lisp}
interchanges those bits of \cdf{x} and \cdf{y} for which the mask \cdf{m} is
\cd{1}, and leaves alone those bits of \cdf{x} and \cdf{y} for which \cdf{m} is
\cd{0}.
\end{implementation}
\afternoterule
\end{defun}

\section{Random Numbers}
\label{RANDOM}

The Common Lisp facility for generating pseudo-random numbers has
been carefully defined to make its use reasonably portable.
While two implementations may produce different series
of pseudo-random numbers, the distribution of values should
be relatively independent of such machine-dependent aspects
as word size.

\begin{defun}[Function]
random number &optional state

\cd{(random \emph{n})} accepts a positive number \emph{n} and returns
a number of the same kind between zero (inclusive) and \emph{n} (exclusive).
The number \emph{n} may be an integer or a floating-point number.
An approximately uniform choice distribution is used.
If \emph{n} is an integer, each of the possible results
occurs with (approximate) probability 1/\emph{n}.
(The qualifier ``approximate'' is used because of implementation
considerations; in practice, the deviation from uniformity should be
quite small.)

The argument \emph{state} must be an object of type \cdf{random-state};
it defaults to the value of the variable \cdf{*random-state*}.
This object is used to maintain the state of the pseudo-random-number
generator and is altered as a side effect of the \cdf{random} operation.

\beforenoterule
\begin{implementation}
In general, even if \cdf{random} of zero arguments
were defined as in MacLisp,
it is not adequate to define \cd{(random \emph{n})} for integral \emph{n}
to be simply \cd{(mod (random) \emph{n})}; this fails to be uniformly distributed
if \emph{n} is larger than the largest number produced by \cdf{random},
or even if \emph{n} merely approaches this number.
This is another reason for omitting \cdf{random} of zero arguments in Common Lisp.
Assuming that the underlying mechanism produces ``random bits''
(possibly in chunks such as fixnums), the best approach is to produce
enough random bits to construct an integer \emph{k} some number \emph{d} of bits
larger than \cd{(integer-length \emph{n})} (see \cdf{integer-length}), and
then compute \cd{(mod \emph{k} \emph{n})}.  The quantity \emph{d} should be at
least 7, and preferably 10 or more.

To produce random floating-point numbers in the half-open
range $[\emph{A},\emph{B})$,
accepted practice (as determined by a look through the
\emph{Collected Algorithms from the ACM}, particularly algorithms
133, 266, 294, and 370) is to compute $\emph{X}\cdot(\emph{B}-\emph{A})+\emph{A}$,
where \emph{X} is a floating-point number uniformly distributed over
$[0.0, 1.0)$
and computed by calculating a random integer $\emph{N}$ in the range
$[0,\emph{M})$
(typically by a multiplicative-congruential or linear-congruential method
mod $\emph{M}$) and then setting $\emph{X}=\emph{N}/\emph{M}$.  See also \cite{KNUTH-VOLUME-2}.
If one takes $\emph{M}=2^{f}$, where $\emph{f}$ is the length of the significand
of a floating-point number (and it is in fact common to choose $\emph{M}$
to be a power of 2), then this method is equivalent to the following
assembly-language-level procedure.  Assume the representation
has no hidden bit.  Take a floating-point 0.5,
and clobber its entire significand with random bits.  Normalize the
result if necessary.

For example, on the DEC PDP-10, assume that accumulator \cdf{T} is completely random
(all 36 bits are random).  Then the code sequence
\begin{lisp}
LSH T,-9~~~~~~~~~~~~~~~~~;\textrm{Clear high 9 bits; low 27 are random} \\
FSC T,128.~~~~~~~~~~~~~~~;\textrm{Install exponent and normalize}
\end{lisp}
will produce in \cdf{T} a random floating-point number uniformly distributed
over $[0.0, 1.0)$.  (Instead of the \cdf{LSH} instruction,
one could do
\begin{lisp}
TLZ T,777000~~~~~~~~~~~~~;\textrm{That's 777000 octal}
\end{lisp}
but if the 36 random bits came from a congruential random-number generator,
the high-order bits tend to be ``more random'' than the low-order ones,
and so the \cdf{LSH} would be better for uniform distribution.
Ideally all the bits would be the result of high-quality randomness.)

With a hidden-bit representation, normalization is not a problem,
but dealing with the hidden bit is.  The method can be adapted as follows.
Take a floating-point 1.0 and clobber the explicit significand bits with
random bits; this produces a random floating-point number in
the range $[1.0, 2.0)$.  Then simply subtract 1.0.  In effect, we
let the hidden bit creep in and then subtract it away again.

For example, on the DEC VAX, assume that register \cdf{T} is
completely random (but a little less random than on the PDP-10, as
it has only 32 random bits).  Then the code sequence
\begin{lisp}
INSV \#{\Xcircumflex}X81,\#7,\#9,T~~~~~;\textrm{Install correct sign bit and exponent} \\
SUBF \#{\Xcircumflex}F1.0,T~~~~~~~~~~;\textrm{Subtract 1.0}
\end{lisp}
will produce in \cdf{T} a random floating-point number uniformly distributed
over $[0.0, 1.0)$.  Again, if the low-order bits are not random enough,
then the instruction
\begin{lisp}
ROTL \#7,T
\end{lisp}
should be performed first.

Implementors may wish to consult reference \cite{ADDITIVE-RANDOMS} for
a discussion of some efficient methods of generating pseudo-random numbers.
\end{implementation}
\afternoterule
\end{defun}

\begin{defun}[Variable]
*random-state*

This variable holds a data structure,
an object of type \cdf{random-state}, that encodes the internal state
of the random-number generator that \cdf{random} uses by default.
The nature
of this data structure is implementation-dependent.  It may be
printed out and successfully read back in, but may or may not function
correctly as a random-number state object in another implementation.
A call to \cdf{random} will perform a side effect on this data structure.
Lambda-binding this variable to a different random-number state object
will correctly save and restore the old state object.
\end{defun}

\begin{defun}[Function]
make-random-state &optional state

This function returns a new object of type \cdf{random-state},
suitable for use as the value of the variable \cdf{*random-state*}.
If \emph{state} is {\false} or omitted, \cdf{make-random-state} returns a \emph{copy}
of the current random-number state object (the value of
the variable \cdf{*random-state*}).  If \emph{state} is a state object,
a copy of that state object is returned.  If \emph{state} is {\true},
then a new state object is returned that has been ``randomly''
initialized by some means (such as by a time-of-day clock).

\beforenoterule
\begin{rationale}
Common Lisp purposely provides no way to initialize a \cdf{random-state}
object from a user-specified ``seed.''  The reason for this is that
the number of bits of state information in a \cdf{random-state} object
may vary widely from one implementation to another, and there is no
simple way to guarantee that any user-specified seed value will be
``random enough.''  Instead, the initialization of \cdf{random-state}
objects is left to the implementor in the case where the argument {\true}
is given to \cdf{make-random-state}.

To handle the common situation of executing the same program many times
in a reproducible manner, where that program uses \cdf{random}, the following
procedure may be used:
\begin{enumerate}
\item
Evaluate \cd{(make-random-state t)} to create a \cdf{random-state} object.

\item
Write that object to a file, using \cdf{print}, for later use.

\item
Whenever the program is to be run, first use \cdf{read} to create
a copy of the \cdf{random-state} object from the printed representation
in the file.
Then use the \cdf{random-state} object newly created by the \cdf{read} operation
to initialize the random-number generator for the program.
\end{enumerate}
It is for the sake of this procedure for reproducible execution that
implementations are required to provide a read/print syntax for objects
of type \cdf{random-state}.

It is also possible to make copies of a \cdf{random-state} object
directly without going through the print/read process, simply by
using the \cdf{make-random-state} function to copy the object; this allows
the same sequence of random numbers to be generated many times
within a single program.
\end{rationale}
\betweennoterule
\begin{implementation}
A recommended way to implement the type \cdf{random-state}
is effectively to use the machinery for \cdf{defstruct}.
The usual structure syntax may then be used for printing \cdf{random-state}
objects; one might look something like
\begin{lisp}
\#S(RANDOM-STATE DATA \#(14 49 98436589 786345 8734658324 ...))
\end{lisp}
where the components are of course completely implementation-dependent.
\end{implementation}
\afternoterule
\end{defun}

\begin{defun}[Function]
random-state-p object

\cdf{random-state-p} is true if its argument is a random-state object,
and otherwise is false.
\begin{lisp}
(random-state-p \emph{x}) \EQ\ (typep \emph{x} 'random-state)
\end{lisp}
\end{defun}

\section{Implementation Parameters}

The values of the named constants defined in this section are
implementation-dependent.  They may be useful for parameterizing
code in some situations.

\begin{defun}[Constant]
most-positive-fixnum \\
most-negative-fixnum

The value of \cdf{most-positive-fixnum} is that fixnum closest in value to
positive infinity provided by the implementation.

The value of \cdf{most-negative-fixnum} is that fixnum closest in value to
negative infinity provided by the implementation.

\begin{new}
X3J13 voted in January 1989
\issue{FIXNUM-NON-PORTABLE}
to specify that \cdf{fixnum} must be a supertype
of the type \cd{(signed-byte 16)}, and additionally that the value
of \cdf{array-dimension-limit} must be a fixnum.  This implies that the value
of \cdf{most-negative-fixnum} must be less than or equal to $-2^{15}$,
and the value of \cdf{most-positive-fixnum} must be greater than or equal to
both $2^{15}-1$ and the value of \cdf{array-dimension-limit}.
\end{new}
\end{defun}

\begin{defun}[Constant]
most-positive-short-float \\
least-positive-short-float \\
least-negative-short-float \\
most-negative-short-float

The value of \cdf{most-positive-short-float} is that short-format
floating-point number closest in value to (but not equal to)
positive infinity provided by the implementation.

The value of \cdf{least-positive-short-float} is that positive short-format
floating-point number closest in value to (but not equal to) zero provided by
the implementation.

The value of \cdf{least-negative-short-float} is that negative short-format
floating-point number closest in value to (but not equal to) zero provided by
the implementation.  (Note that even if an implementation supports
minus zero as a distinct short floating-point value,
\cdf{least-negative-short-float} must not be minus zero.)

\begin{newer}
X3J13 voted in June 1989 \issue{FLOAT-UNDERFLOW}
to clarify that these definitions are to be taken quite literally.
In implementations that support denormalized numbers,
the values of \cdf{least-positive-short-float} and
\cdf{least-negative-short-float} may be denormalized.
\end{newer}

The value of \cdf{most-negative-short-float} is that short-format
floating-point number closest in value to (but not equal to)
negative infinity provided by the implementation.
\end{defun}

\begin{defun}[Constant]
most-positive-single-float \\
least-positive-single-float \\
least-negative-single-float \\
most-negative-single-float \\
most-positive-double-float \\
least-positive-double-float \\
least-negative-double-float \\
most-negative-double-float \\
most-positive-long-float \\
least-positive-long-float \\
least-negative-long-float \\
most-negative-long-float

These are analogous to the constants defined above for short-format
floating-point numbers.
\end{defun}

\bigskip

\begin{defun}[Constant]
least-positive-normalized-short-float \\
least-negative-normalized-short-float

X3J13 voted in June 1989 \issue{FLOAT-UNDERFLOW}
to add these constants to the language.

The value of \cdf{least-positive-normalized-short-float} is that positive normalized
short-format
floating-point number closest in value to (but not equal to) zero provided by
the implementation.  In implementations that do not support denormalized numbers
this may be the same as the value of
\cdf{least-positive-short-float}.

The value of \cdf{least-negative-normalized-short-float} is that negative normalized short-format
floating-point number closest in value to (but not equal to) zero provided by
the implementation.
(Note that even if an implementation supports
minus zero as a distinct short floating-point value,
\cdf{least-negative-normalized-short-float} must not be minus zero.)
In implementations that do not support denormalized numbers
this may be the same as the value of \cdf{least-positive-short-float}.
\end{defun}

\begin{defun}[Constant]
least-positive-normalized-single-float \\
least-negative-normalized-single-float \\
least-positive-normalized-double-float \\
least-negative-normalized-double-float \\
least-positive-normalized-long-float \\
least-negative-normalized-long-float

These are analogous to the constants defined above for short-format
floating-point numbers.
\end{defun}

\begin{defun}[Constant]
short-float-epsilon \\
single-float-epsilon \\
double-float-epsilon \\
long-float-epsilon

These constants have as value, for each floating-point format,
the smallest positive floating-point number \emph{e} of that format such that
the expression
\begin{lisp}
(not (= (float 1 \emph{e}) (+ (float 1 \emph{e}) \emph{e})))
\end{lisp}
is true when actually evaluated.
\end{defun}

\begin{defun}[Constant]
short-float-negative-epsilon \\
single-float-negative-epsilon \\
double-float-negative-epsilon \\
long-float-negative-epsilon

These constants have as value, for each floating-point format,
the smallest positive floating-point number \emph{e} of that format such that
the expression
\begin{lisp}
(not (= (float 1 \emph{e}) (- (float 1 \emph{e}) \emph{e})))
\end{lisp}
is true when actually evaluated.
\end{defun}

\else % RUSSIAN

\chapter{Числа}
\label{NUMBER}

Common Lisp содержит несколько различных представлений для чисел.
Эти представления могут быть разделены на четыре категории: целые, дробные, с
плавающей точкой и комплексные. Большинство функций принимают любой тип
числа. Некоторые другие --- только определённый тип.

В целом числа в Common Lisp'е не является объектами в прямом понимании
слова. Нельзя полагаться, например, на использование \cdf{eq} для чисел. В
частности, возможно, что значение выражения
\begin{lisp}
(let ((x z) (y z)) (eq x y))
\end{lisp}
может быть ложным, а не истинным, если значение \cdf{z} является числом.

Если два объекта сравниваются для <<идентичности>>, но один из может быть
числом, то использование предиката \cdf{eql} является более подходящим.
Если оба объекта являются числами, то преимущество стоит отдать \cdf{=}.

\section{Точность, неявное приведение и явное приведение}
\label{PRECISION-CONTAGION-COERCION-SECTION}

Вычисления с числами с плавающей точкой являются приблизительными.
\emph{Точность} чисел с плавающей точкой не обязательно коррелирует с
\emph{<<аккуратностью>>} числа.
Например, 3.142857142857142857 имеет более точное приближение к $pi$ чем
3.14159, но последнее число более <<аккуратно>>.
Точность указывает на количество бит используемых при представлении числа.
Если операция объединяла числа с плавающей точкой короткого формата и длинного,
то результат будет иметь длинный формат. Это правило создано для уверенности,
что при вычислениях аккуратности будет как можно больше. Однако это не
гарантированно.
Однако, численные процедуры Common Lisp'а предполагают, что аккуратность
аргумента не превышает его точность. Поэтому, когда объединяются два 
числа с плавающей точкой небольшой точности, результатом всегда будет число с
плавающей точкой небольшой точности.
Это предположение может быть изменено первым же явным преобразованием
число с плавающей точкой в более точное представление.
(Common Lisp никогда не преобразует числа из более точного формата в менее точный.)

Вычисления рациональных чисел не может вызвать переполнения в обычном смысле
этого слова (хотя, конечно, может быть возникнуть недостаток места для
представления), целые и дробные числа в принципе могут быть любой величины.
Вычисления с плавающей точкой могут вызвать переполнение экспоненты. Это
является ошибкой.

Когда числовой функцией между собой сравниваются или объединяются рациональное и
число с плавающей точкой, то вступает в силу правило \emph{неявного приведения к
 плавающей точке}.
Когда рациональное встречает число с плавающей точкой, то рациональное
преобразуется в тот же формат второго числа.
Для функций, например \cdf{+}, которые принимает более двух аргументов, может
быть, что все рациональные будут вычислены и результат будет преобразован в
число с плавающей точкой.

Для функций, которые математически ассоциативны (и возможно коммутативны),
реализация Common Lisp'а может обрабатывать аргументы любым подходящим методом с
ассоциативной (и возможно коммутативной) перестановкой.
Это конечно не влияет на порядок вычисления форм, данный порядок всегда слева
направо, как и во всех Common Lisp'овых вызовах функций. Порядок, который может
быть изменён, это обработка значений аргументов.
Смысл всего этого в том, что реализация может отличаться в том, какие
автоматические приведения типов и в каком порядке производятся. Например,
рассмотрим выражение:
\begin{lisp}
(+ 1/3 2/3 1.0D0 1.0 1.0E-15)
\end{lisp}
Одна реализация может обрабатывать аргументы слева направо, складывая сначала
\cd{1/3} и \cd{2/3} для получения \cd{1}, затем преобразовывая результат в число
с плавающей точкой двойной точности для сложения с \cd{1.0D0}, затем
преобразовывая и добавляя \cd{1.0} и \cd{1.0E-15}. Другая реализация может
обрабатывать значения аргументов справа налево, сначала выполняя сложение чисел
с плавающей точкой \cdf{1.0} и \cd{1.0E-15} (и возможно теряя аккуратность в
процессе!), затем преобразовывая результат в число двойной точности и прибавляя
\cd{1.0D0}, затем преобразовывая \cd{2/3} к числу с плавающей точкой двойной
точности, и затем преобразовывая \cd{1/3} и добавляя его.
Третья реализация может сначала просканировать все аргументы, и сгруппировав их
по типам, выполнить сложения сначала одинаковых типов, затем преобразовать
результаты к наиболее точному типу и сложить их. В этом случае все три стратегии
являются допустимыми. Пользователь конечно может контролировать порядок
обработки аргументов явно задавая вызовы вычислений, например:
\begin{lisp}
(+ (+ 1/3 2/3) (+ 1.0D0 1.0E-15) 1.0)
\end{lisp}
Пользователь может также контролировать приведения, явно используя  для этого
функцию.

В целом, тип результата числовой функции является числом с плавающей точкой
наиболее точного формата, который был в аргументах данной функции.
Но если все аргументы были рационального типа, тогда результат будет
рациональным (за исключением функций, который математически возвращают
иррациональные результаты, в случае который используется одинарная точность с
плавающей точкой) 

Другое правило для комплексных чисел.
Комплексные числа никогда не возвращаются из числовых функций, если только в
аргументах не было использовано хоть одно комплексное число. (Исключением из
этого правила являются иррациональные и трансцендентальные функции, в частности
\cdf{expt}, \cdf{log}, \cdf{sqrt}, \cdf{asin}, \cdf{acos}, \cdf{acosh} и
\cdf{atanh}. Смотрите раздел~\ref{TRANSCENDENTAL-SECTION}.)
Когда некомплексное число встречает комплексно, то первое сначала конвертируется
во второе с нулевой мнимой частью, а потом вычисляется результат.

Если любое вычисление привело к дробному результату, в котором числитель
нацело делится на знаменатель, то результат немедленно преобразуется к
эквивалентному целому числу. Это правило называется \emph{канонизацией дробей}.

Если результат любого вычисления должен быть комплексным числом с рациональными
частями и нулевой мнимой частью, то результат немедленно преобразуется в
некомплексное рациональное число и равен действительной части исходного.
Это называется правилом \emph{канонизации комплексного числа}.
Следует отметить, что это правило \emph{не} применяется к комплексным числам с
компонентами из чисел с плавающими точками. Таким образом \cd{\#C(5 0)} и \cd{5}
равны \cdf{eql}, а \cd{\#C(5.0 0.0)} и \cd{5.0} не равны \cdf{eql}, но равны
\cdf{equalp}.

\section{Предикаты для чисел}

Каждая из следующих функций проверяет одиночное число на наличие некоторого
свойства.
Каждая функция требует числовой аргумент. Вызов с аргументом другого типа
является ошибкой.

\begin{defun}[Функция]
zerop number

Если \emph{number} является нулём (целым нулём, нулём с
плавающей точкой, комплексным нулём), этот предикат истинен, иначе ложен.
Вне зависимости от того, предоставляет различные представления для
положительного и отрицательного нулей система, \cd{(zerop -0.0)} всегда истинно.
Если аргумент \emph{number} не является числом, то сигнализируется ошибка.
\end{defun}

\begin{defun}[Функция]
plusp number

Если \emph{number} строго больше нуля, то предикат истинен, иначе ложен.
Аргумент \emph{number} должен быть любым числом, кроме комплексного, иначе
сигнализируется ошибка. 
\end{defun}

\begin{defun}[Функция]
minusp number

Если \emph{number} строго меньше нуля, то предикат истинен, иначе ложен.
Вне зависимости от того, предоставляет ли система различные отображения для
отрицательного и положительного нулей с плавающей точкой, \cd{(minusp -0.0)}
всегда ложно.
(Для проверки отрицательного нуля, может использоваться функция
\cdf{float-sign}.)
Аргумент \emph{number} должен быть любым числом, кроме комплексного, иначе
сигнализируется ошибка. 
\end{defun}

\begin{defun}[Функция]
oddp integer

Если аргумент \emph{integer} является нечётным числом (то есть не делится на два
нацело), то предикат истинен, иначе ложен.
Если аргумент не целое число, сигнализируется ошибка.
\end{defun}

\begin{defun}[Функция]
evenp integer

Если аргумент \emph{integer} является чётным числом (то есть делится на два
нацело), то предикат истинен, иначе ложен.
Если аргумент не целое число, сигнализируется ошибка.
\end{defun}

Смотрите также предикаты типов данных \cdf{integerp},
\cdf{rationalp}, \cdf{floatp}, \cdf{complexp} и \cdf{numberp}.

\section{Сравнение чисел}

Каждая функция из данного раздела требует, чтобы все аргументы были
числами. Использование в качестве аргумента не числа является ошибкой. Если не
указано иное, то каждая функция работает со всеми типами чисел, автоматически
выполняя необходимые приведения, когда типы аргументы различаются.

\begin{defun}[Функция]
= number &rest more-numbers \\
/= number &rest more-numbers \\
< number &rest more-numbers \\
> number &rest more-numbers \\
<= number &rest more-numbers \\
>= number &rest more-numbers
Каждая из этих функций принимает один и более аргументов. Если последовательно
аргументов удовлетворяет следующему условию:

\begin{tabbing}
\hskip 5pc\=\kill
\cdf{=}\>все равны \\
\cdf{/=}\>все различны \\
\cdf{<}\>монотонно возрастают \\
\cdf{>}\>монотонно убывают \\
\cdf{<=}\>монотонно не убывают \\
\cdf{>=}\>монотонно не возрастают
\end{tabbing}

тогда предикат истинен, иначе ложен.
Комплексные числа могут сравниваться с помощью \cdf{=} и \cdf{/=}, но остальные
предикаты требуют некомплексных аргументов.
Два комплексных числа равны \cdf{=}, если их действительные части равны между
собой и мнимые равны между собой с помощью предиката \cdf{=}.
Комплексное число может быть сравнено с некомплексным с помощью \cdf{=} или
\cdf{/=}.
Например
\begin{lisp}
\hskip 0.5\textwidth\=\kill
(= 3 3) \textrm{истина}\>(/= 3 3) \textrm{ложь} \\
(= 3 5) \textrm{ложь}\>(/= 3 5) \textrm{истина} \\
(= 3 3 3 3) \textrm{истина}\>(/= 3 3 3 3) \textrm{ложь} \\
(= 3 3 5 3) \textrm{ложь}\>(/= 3 3 5 3) \textrm{ложь} \\
(= 3 6 5 2) \textrm{ложь}\>(/= 3 6 5 2) \textrm{истина} \\
(= 3 2 3) \textrm{ложь}\>(/= 3 2 3) \textrm{ложь} \\
(< 3 5) \textrm{истина}\>(<= 3 5) \textrm{истина} \\
(< 3 -5) \textrm{ложь}\>(<= 3 -5) \textrm{ложь} \\
(< 3 3) \textrm{ложь}\>(<= 3 3) \textrm{истина} \\
(< 0 3 4 6 7) \textrm{истина}\>(<= 0 3 4 6 7) \textrm{истина} \\
(< 0 3 4 4 6) \textrm{ложь}\>(<= 0 3 4 4 6) \textrm{истина} \\
(> 4 3) \textrm{истина}\>(>= 4 3) \textrm{истина} \\
(> 4 3 2 1 0) \textrm{истина}\>(>= 4 3 2 1 0) \textrm{истина} \\
(> 4 3 3 2 0) \textrm{ложь}\>(>= 4 3 3 2 0) \textrm{истина} \\
(> 4 3 1 2 0) \textrm{ложь}\>(>= 4 3 1 2 0) \textrm{ложь} \\
(= 3) \textrm{истина}\>(/= 3) \textrm{истина} \\
(< 3) \textrm{истина}\>(<= 3) \textrm{истина} \\
(= 3.0 \#C(3.0 0.0)) \textrm{истина}\>(/= 3.0 \#C(3.0 1.0)) \textrm{истина} \\
(= 3 3.0) \textrm{истина}\>(= 3.0s0 3.0d0) \textrm{истина} \\
(= 0.0 -0.0) \textrm{истина}\>(= 5/2 2.5) \textrm{истина} \\
(> 0.0 -0.0) \textrm{ложь}\>(= 0 -0.0) \textrm{истина}
\end{lisp}
С двумя аргументами, эти функции выполняют обычные арифметические сравнения.
С тремя и более аргументами, они полезны для проверок рядов, как показано в
следующем примере:
\begin{lisp}
(<= 0 x 9)~~~~~~~~~~~~~~~~~~~~~~;\textrm{истина, если \cdf{x} между 0 и 9, включительно} \\
(< 0.0 x 1.0)~~~~~~~~~~~~~~~~~~~;\textrm{истина, если \cdf{x} между 0.0 и 1.0} \\
(< -1 j (length s))~~~~~~~~~~~~~;\textrm{истина, если \cdf{j} корректный индекс для \cdf{s}} \\
(<= 0 j k (- (length s) 1))~~~~~;\textrm{истина, если \cdf{j} и \cdf{k} оба корректные} \\
~~~~~~~~~~~~~~~~~~~~~~~~~~~~~~~~;\textrm{индексы \cdf{s} и $\cdf{j}\leq\cdf{k}$}
\end{lisp}

\beforenoterule
\begin{rationale}
Отношение <<неравенства>> называется \cdf{/=}, а не \cd{<>} (как в Pascal'е) по
двум причинам. Первое, \cdf{/=} для более чем двух аргументов не означает то же,
что и \cd{или} \cd{<}, или \cd{>}. Второе, неравенство для комплексная не
означает то же, что и \cd{<} и \cd{>}. По этим двум причинам неравенство не
означает больше или меньше.
\end{rationale}
\afternoterule
\end{defun}

\begin{defun}[Функция]
max number &rest more-numbers \\
min number &rest more-numbers

Аргументы могут быть некомплексными числами.
\cdf{max} возвращает наибольший аргумент (ближайший к положительной
бесконечности).
\cdf{max} возвращает наименьший аргумент (ближайший к отрицательной
бесконечности).

Для \cdf{max}, если аргументы являются смесью из дробных и с плавающей точкой
чисел, и наибольший является дробным, реализация может вернуть как дробное
число, так и его аппроксимацию с плавающей точкой.
Если наибольший аргумент является числом с плавающей точкой меньшего формата в
сравнении с числом большего формата, то реализация может вернуть число без
изменения, либо конвертировать его в больший формат.
More concisely, the implementation has the choice of returning the largest
argument as is or applying the rules of floating-point contagion,
taking all the arguments into consideration for contagion purposes.
Если среди наибольших два аргумента равны, то реализация может вернуть любой из
них.
Те же правила применяются и ко второй функции \cdf{min} (только <<наибольший
аргумент>> нужно заменить на <<наименьший аргумент>>).

\begin{lisp}
\hskip 0.5\textwidth\=\kill
(max 6 12) \EV\ 12\>(min 6 12) \EV\ 6 \\
(max -6 -12) \EV\ -6\>(min -6 -12) \EV\ -12 \\
(max 1 3 2 -7) \EV\ 3\>(min 1 3 2 -7) \EV\ -7 \\
(max -2 3 0 7) \EV\ 7\>(min -2 3 0 7) \EV\ -2 \\
(max 3) \EV\ 3\>(min 3) \EV\ 3 \\
(max 5.0 2) \EV\ 5.0\>(min 5.0 2) \EV\ 2 \emph{или} 2.0 \\
(max 3.0 7 1) \EV\ 7 \emph{или} 7.0\>(min 3.0 7 1) \EV\ 1 \emph{или} 1.0 \\
(max 1.0s0 7.0d0) \EV\ 7.0d0 \\
(min 1.0s0 7.0d0) \EV\ 1.0s0 \emph{или} 1.0d0 \\
(max 3 1 1.0s0 1.0d0) \EV\ 3 \emph{или} 3.0d0 \\
(min 3 1 1.0s0 1.0d0) \EV\ 1 \emph{или} 1.0s0 \emph{или} 1.0d0
\end{lisp}
\end{defun}

\section{Арифметические операции}

Каждая из функций в этом разделе в качестве аргументов принимает числа. Применение
какой-либо из них к нечисловому аргументу является ошибкой. Каждая из них
работает со всеми типами чисел, автоматически приводя типы для различных
аргументов, если не указано иное.

\begin{defun}[Функция]
+ &rest numbers

Данная функция возвращает сумму всех аргументов. Если аргументов не было,
результатом является \cd{0}.
\end{defun}

\begin{defun}[Функция]
- number &rest more-numbers

Функция \cdf{-} при использовании с одним аргументом возвращает отрицательное
значение для этого аргумента.

Функция \cdf{-} при использовании с более чем одним аргументом вычитает из
первого аргумента все остальные и возвращает результат.
Например, \cd{(- 3 4 5)} \EV\ \cd{-6}.
\end{defun}

\begin{defun}[Функция]
* &rest numbers

Функция возвращает произведение всех аргументов.
Если вызывается без аргументов, тогда возвращает \cd{1}.
\end{defun}

\begin{defun}[Функция]
/ number &rest more-numbers

Функция \cdf{/} при использовании с более чем одним аргументом производит
деление первого аргумента на все остальные значения и возвращает полученный
результат.

Ноль может использоваться только для первого аргумента.

Для одного аргумента \cdf{/} возвращает обратное значение. В таком случае ноль в
первом аргументе недопустим.

Если математическое отношение двух целых
чисел не является целым числом, \cdf{/} вернёт дробный тип. 
Например:
\begin{lisp}
(/ 12 4) \EV\ 3 \\
(/ 13 4) \EV\ 13/4 \\
(/ -8) \EV\ -1/8 \\
(/ 3 4 5) \EV\ 3/20
\end{lisp}

Для целочисленного деления используйте одну из функций  
\cdf{floor}, \cdf{ceiling}, \cdf{truncate}
или \cdf{round}.

Если один из аргументов является числом с плавающей точкой, тогда применяются
правила обработки чисел с плавающей точкой.
\end{defun}

\begin{defun}[Функция]
1+ number \\
1- number

\cd{(1+ \emph{x})} является тем же самым, что \cd{(+ \emph{x} 1)}.
\cd{(1- \emph{x})} является тем же самым, что \cd{(- \emph{x} 1)}.
Следует отметить, что короткое имя может ввести в заблуждение: 
\cd{(1- \emph{x})} \emph{не значит} то же, что и $1-\emph{x}$.
\end{defun}

\begin{defmac}
incf place [delta] \\
decf place [delta]

Значение обобщённой переменной \emph{place} увеличивается (\cdf{incf}) или
уменьшается (\cdf{decf}) на значение формы
\emph{delta} и затем присваивается переменной \emph{place}.
Результатом формы является присвоенное значение.
Форма \emph{place} может быть любой формой подходящей для \cdf{setf}.
Если форма \emph{delta} не указана, тогда число в \emph{place} изменяется на
\cd{1}.
Например:
\begin{lisp}
\hskip 9pc\=\kill
(setq n 0) \\
(incf n) \EV\ 1\>\textrm{and now} n \EV\ 1 \\
(decf n 3) \EV\ -2\>\textrm{and now} n \EV\ -2 \\
(decf n -5) \EV\ 3\>\textrm{and now} n \EV\ 3 \\
(decf n) \EV\ 2\>\textrm{and now} n \EV\ 2
\end{lisp}
Результат \cd{(incf \emph{place} \emph{delta})} примерно эквивалентен
\begin{lisp}
(setf \emph{place} (+ \emph{place} \emph{delta}))
\end{lisp}
за исключением того, что в последнем случае \emph{place} вычисляется дважды,
тогда как \cdf{incf} гарантирует, что вычисление \emph{place} будет выполнено
только один раз.
Более того, для некоторых форма \emph{place} \cdf{incf} может быть
эффективнее чем \cdf{setf}.
\end{defmac}

\begin{defun}[Функция]
conjugate number

Функция возвращает комплексное сопряжение с числа \emph{number}. 
Сопряжением некомплексного числа является само число. Для комплексного числа
\cdf{z},
\begin{lisp}
(conjugate z) \EQ\ (complex (realpart z) (- (imagpart z)))
\end{lisp}
Например,
\begin{lisp}
(conjugate \#C(3/5 4/5)) \EV\ \#C(3/5 -4/5) \\
(conjugate \#C(0.0D0 -1.0D0)) \EV\ \#C(0.0D0 1.0D0) \\
(conjugate 3.7) \EV\ 3.7
\end{lisp}
\end{defun}

\begin{defun}[Функция]
gcd &rest integers

Функция возвращает наибольший общий делитель всех аргументов, которые в свою
очередь должны быть целыми числами. Результатом \cdf{gcd} всегда является
неотрицательное целое число.
Если передан только один параметр, то возвращается его абсолютное значение
(модуль).
Если ни один параметр не был передан, \cdf{gcd} возвращает \cd{0}.
Для трёх и более аргументов,
\begin{lisp}
(gcd \emph{a} \emph{b} \emph{c} ... \emph{z}) \EQ\ (gcd (gcd \emph{a} \emph{b}) \emph{c} ... \emph{z})
\end{lisp}

Несколько примеров использования \cdf{gcd}:
\begin{lisp}
(gcd 91 -49) \EV\ 7 \\
(gcd 63 -42 35) \EV\ 7 \\
(gcd 5) \EV\ 5 \\
(gcd -4) \EV\ 4 \\
(gcd) \EV\ 0
\end{lisp}
\end{defun}

\begin{defun}[Функция]
lcm integer &rest more-integers

Функция возвращает наименьшее общее кратное для аргументов, которые в свою
очередь должны быть целыми числами.
Результатом \cdf{lcm} всегда является неотрицательное целое число.
Для двух аргументов, не являющихся нулями,
\begin{lisp}
(lcm \emph{a} \emph{b}) \EQ\ (/ (abs (* \emph{a} \emph{b})) (gcd \emph{a} \emph{b}))
\end{lisp}
Если один или оба аргумента является нулём.
\begin{lisp}
(lcm \emph{a} 0) \EQ\ (lcm 0 \emph{a}) \EQ\ 0
\end{lisp}

Для одного аргумента, \cdf{lcm} возвращает абсолютное значение этого аргумента
(модуль). 
Для трёх и более аргументов,
\begin{lisp}
(lcm \emph{a} \emph{b} \emph{c} ... \emph{z}) \EQ\ (lcm (lcm \emph{a} \emph{b}) \emph{c} ... \emph{z})
\end{lisp}

Примеры использования:
\begin{lisp}
(lcm 14 35) \EV\ 70 \\
(lcm 0 5) \EV\ 0 \\
(lcm 1 2 3 4 5 6) \EV\ 60
\end{lisp}

%Математически \cd{(lcm)} должна возвращать бесконечность. Но так как Common Lisp
%не имеет представления для бесконечности, \cdf{lcm}, в отличие от \cdf{gcd},
%всегда требует как минимум одного аргумента.

\cdf{lcm}, вызванная без аргументов, возвращает \cd{1}.
\begin{lisp}
(lcm)~\EV~1
\end{lisp}
\end{defun}

\section{Иррациональные и трансцендентные функции}
\label{TRANSCENDENTAL-SECTION}

Common Lisp не содержит тип данных, который точно отображает иррациональные
числовые значения.
В данном разделе функции описаны так, как если бы результаты были математически
точными, но фактически все они возвращают число с плавающей точкой
приблизительно равное настоящему математическому значению.
В некоторых местах изложены математические тождества, связанные со значениями
функций, однако, два математически идентичных выражения могут быть различными
по причине ошибок процесса аппроксимации при вычислениях чисел с плавающей
точкой.

Когда все аргументы для функции данного раздела являются рациональными и Ъ
математический результат также является (математически) рациональным, тогда,
если не указано иное, реализация может вернуть или точный результат типа
\cdf{rational} или приближенное значение с плавающей точкой одинарной точности.
Если все аргументы рациональны, но результат не может быть рациональным, тогда
возвращается число с плавающей точкой одинарной точности.

Если все переданные в функцию аргументы принадлежат типу \cd{(or~rational
  (complex rational))} и Ъ математический результат является (математически)
комплексным числом с рациональными действительной и мнимой частями, тогда, если
не указано иное, реализация может вернуть или точный результат типа \cd{(or
  rational (complex rational))} или приближенный типа с плавающей точкой с
одинарной точностью \cdf{single-float} (только если мнимая часть равна нулю) или 
 \cd{(complex single-float)}.
Если все аргументы типа \cd{(or~rational (complex rational))}, но результат не
может быть выражен рациональными или комплексным рациональным числом, тогда он
будет принадлежать типу \cdf{single-float} (только если мнимая часть равна нулю)
или \cd{(complex single-float)}.

Все функции за исключением \cdf{expt} подчиняются правилам неявного приведения
плавающей точки или комплексного числа. Когда, возможно после приведения типов,
все аргументы становятся с плавающей точкой или комплексным числом с плавающей
точкой, тогда результат будет такого же типа, что и аргумента, если не указано
иное.

\beforenoterule
\begin{implementation}
Для понимания работы функций из данного раздела может быть полезной <<поваренная
книга чисел с плавающей точкой>> от Cody и Waite~\cite{CODY-AND-WAITE}.
\end{implementation}
\afternoterule

\subsection{Экспоненциальные и логарифмические функции}

Наряду с обычными одно-аргументными и двух-аргументными экспоненциальными и
логарифмическими функциями, \cdf{sqrt} рассматривается как экспоненциальная
функция, потому что она возводит число в степень 1/2.

\begin{defun}[Функция]
exp number

Возвращает \emph{е}, возведённое в степень \emph{number},
где \emph{е} является основанием натурального логарифма.
\end{defun}

\begin{defun}[Функция]
expt base-number power-number

Возвращает \emph{base-number}, возведённый в степень \emph{power-number}.
Если \emph{base-number} принадлежит типу \cdf{rational} и \emph{power-number}
--- \cdf{integer}, тогда результат вычислений будет принадлежать типу
\cdf{rational}, в противном случае результат будет приближенным числом с
плавающей точкой.

Если \emph{power-number} равен \cd{0} (ноль целочисленного типа), тогда
результат всегда будет значение 1, такого же типа что и \emph{base-number}, даже
если \emph{base-number} равен нулю (любого типа). То есть:
\begin{lisp}
(expt \emph{x} 0) \EQ\ (coerce 1 (type-of \emph{x}))
\end{lisp}
Если \emph{power-number} является нулём любого другого типа данных, тогда
результат также равен 1 такого же типа, что и аргументы после применения правил
приведения числовых типов, с одним условием:
если \emph{base-number} равен нулю, когда \emph{power-number} является
нецелочисленным нулём, то это является ошибкой.

Реализация \cdf{expt} может использовать различные алгоритмы для случаев с
рациональным и с плавающей точкой аргументом \emph{power-number}. Суть в том,
что в большинстве случаев боле аккуратный результат может быть достигнут для
рационального \emph{power-number}.
Например, \cd{(expt pi 16)} и  \cd{(expt pi 16.0)} могут вернуть слегка разные
результаты, если в первом случае алгоритм <<повторных квадратов>>, а во втором
использование логарифмов.

Результатом \cdf{expt} может быть комплексное число, даже если ни один аргумент
не был комплексным. Такой результат получается если \emph{base-number}
отрицательное число и \emph{power-number} не является целочисленным.
Следует отметить, что  \cd{(expt -8 1/3)} не может вернуть \cd{-2}, хотя и
\cd{-2} несомненно является одним из кубических корней для \cd{-8}, но основным
корнем является аппроксимированное комплексное число \cd{\#C(1.0 1.73205)}.
\end{defun}

\begin{defun}[Функция]
log number &optional base

Функция возвращает логарифм числа \emph{number} c основанием \emph{base},
которое по умолчанию равно \emph{e} (эпсилон, основание для натурального
логарифма).
Например:
\begin{lisp}
(log 8.0 2) \EV\ 3.0 \\
(log 100.0 10) \EV\ 2.0
\end{lisp}
В зависимости от реализации, результатом \cd{(log 8 2)} может быть как \cd{3},
так и \cd{3.0}.
Например:
\begin{lisp}
(log -1.0) \EQ\ (complex 0.0 (float pi 0.0))
\end{lisp}
\end{defun}

\begin{defun}[Функция]
sqrt number

Функция возвращает квадратный корень числа \emph{number}.
Если \emph{number} не является комплексным числом и отрицательно, тогда
результат будет комплексным числом.
Например:
\begin{lisp}
(sqrt 9.0) \EV\ 3.0 \\
(sqrt -9.0) \EV\ \#c(0.0 3.0)
\end{lisp}
В зависимости от реализации результат \cd{(sqrt 9)} может быть как \cd{3}, так и
\cd{3.0}. Результат \cd{(sqrt -9)} может быть как \cd{\#(0 3)} или \cd{\#c(0.0
  3.0)}.
\end{defun}

\begin{defun}[Функция]
isqrt integer

Целочисленный квадратный корень: аргумент должен быть неотрицательным целым, и
результат является наибольшим целым числом, которое меньше или равно точному
положительному квадратному корню аргумента.
\begin{lisp}
(isqrt 9) \EV\ 3 \\
(isqrt 12) \EV\ 3 \\
(isqrt 300) \EV\ 17 \\
(isqrt 325) \EV\ 18
\end{lisp}
\end{defun}

\subsection{Тригонометрические и связанные с ними функции}

Некоторые из функций в данном разделе, такие как \cdf{abs} и \cdf{signum},
несомненно не относятся к тригонометрическим функциям, когда рассматриваются
как функции только для действительных чисел. Однако, путь, с помощью которого
они расширены для операций на комплексных числах, делает их связь с
тригонометрическими функциями явной.

\begin{defun}[Функция]
abs number

Возвращает абсолютное значение аргумента. Для некомплексных чисел \emph{x}, 
\begin{lisp}
(abs \emph{x}) \EQ\ (if (minusp \emph{x}) (- \emph{x}) \emph{x})
\end{lisp}
и результат всегда имеет тот же тип, что и аргумент.

Для комплексных чисел \emph{z}, абсолютное значение может быть вычислено как
\begin{lisp}
(sqrt (+ (expt (realpart \emph{z}) 2) (expt (imagpart \emph{z}) 2)))
\end{lisp}

\beforenoterule
\begin{implementation}
Аккуратные разработчики не будут напрямую использовать эту формулу для всех
комплексных чисел. Очень большие и очень маленькие части комплексных чисел будут
обрабатываться специализированно для избежания выходов за верхние и нижние
границы значений.
\end{implementation}
\afternoterule

Например:
\begin{lisp}
(abs \#c(3.0 -4.0)) \EV\ 5.0
\end{lisp}
Результатом \cd{(abs \#(3 4))} может быть или \cd{5} или \cd{5.0}, в зависимости
от реализации.
\end{defun}

\begin{defun}[Функция]
phase number

Аргументом (так называется функция phase) числа z (arg(z) \EQ (phase z))
является угол $\varphi$ (в радианах) радиус-вектора точки, соответствующей
комплексному числу z.

Если аргумент является комплексным числом с частями из чисел с плавающей точкой,
то результатом является число с плавающей точкой того же типа.
Если аргумент является числом с плавающей точкой,
то результатом является число с плавающей точкой того же типа.
Если аргумент является комплексным числом с частями из рациональных чисел,
то результатом является число с плавающей точкой одинарного типа.
\end{defun}

\begin{defun}[Функция]
signum number

По определению,
\begin{lisp}
(signum \emph{x}) \EQ\ (if (zerop \emph{x}) \emph{x} (/ \emph{x} (abs \emph{x})))
\end{lisp}
Для рационального числа, \cdf{signum} будет возвращать один из вариантов:
\cd{-1}, \cd{0} или \cd{1}, в соответствии с тем, является ли число
отрицательным, нулём или положительным.
Для числа с плавающей точкой, результатом будет число с плавающей точкой того же
типа и значением: $-1$, $0$ или $1$.
Для комплексного числа \emph{z}, \cd{(signum \emph{z})} является комплексным
числом с таким же аргументом (phase), но с единичным модулем. Но если \emph{z}
равен комплексному нулю, результатом является само число \emph{z}.
Например:
\begin{lisp}
(signum 0) \EV\ 0 \\
(signum -3.7L5) \EV\ -1.0L0 \\
(signum 4/5) \EV\ 1 \\
(signum \#C(7.5 10.0)) \EV\ \#C(0.6 0.8) \\
(signum \#C(0.0 -14.7)) \EV\ \#C(0.0 -1.0)
\end{lisp}
Для некомплексных рациональных чисел \cdf{signum} является рациональной
функцией, но для комплексных чисел может быть иррациональной.
\end{defun}

\begin{defun}[Функция]
sin radians \\
cos radians \\
tan radians

\cdf{sin} возвращает синус аргумента, \cdf{cos} --- косинус, \cdf{tan} ---
тангенс. Аргумент рассматривается как угол в радианах. Аргумент может быть
комплексным числом.
\end{defun}

\begin{defun}[Функция]
cis radians

Функция вычисляет $ \emph{e} ^ {i \cdot radians} $.
Имя \cdf{cis} обозначает <<cos + \emph{i} sin>>, потому что 
$ \emph{e} ^{ i \theta } = \cos \theta + \emph{i} \sin \theta $.
Аргумент рассматривается как угол в радианах и может быть любым некомплексным
числом. Результатом является комплексное число, у которого действительная часть
это косинус аргумента, и мнимая --- синус аргумента. Другими словами, результат
это комплексное число, у которого аргумент (фаза) равна (mod $2/pi$) и модуль
равен единице.

\beforenoterule
\begin{implementation}
Чаще всего дешевле вычислить синус и косинус угла вместе, чем выполнять два
раздельных вычисления.
\end{implementation}
\afternoterule
\end{defun}

\begin{defun}[Функция]
asin number \\
acos number

\cdf{asin} возвращаем арксинус аргумента, и \cdf{acos} --- арккосинус аргумента.
Результат будет в радианах. Аргумент может быть комплексным числом.

Функции арксинус и арккосинус могут быть математически определены для аргумента
\emph{z} следующим образом:
\begin{tabbing}
\hskip 10pc\=\kill
Арксинус\>$ -\emph{i} \log \left(\emph{i}\emph{z} + \sqrt{1-\emph{z}^2}\right) $ \\[2pt]
Арккосинус\>$ -\emph{i} \log \left(\emph{z} + \emph{i}\sqrt{1-\emph{z}^2}\right) $
\end{tabbing}
Следует отметить, что результат \cdf{asin} или \cd{acos} может быть комплексным
числом, даже если аргумент не являлся комплексным. Такое происходит, когда
абсолютное значение аргумента превышает 1.
\end{defun}

\begin{defun}[Функция]
atan y &optional x

Функция вычисляет арктангенс и возвращает результат в радианах.

Ни один из двух аргументов \emph{y} и \emph{x} не может быть комплексным.
Знаки \emph{y} и \emph{x} используются для определения квадранта. \emph{x} может
быть нулём при условии, что \emph{y} не ноль.
Значение \cdf{atan} лежит между $-pi$ (невключительно) и $pi$ (включительно).
Следующая таблица описывает различные специальные случаи.

\begin{flushleft}
\begin{tabular*}{\linewidth}{@{}l@{\extracolsep{\fill}}llc@{}}
\multicolumn{2}{c}{Условие}&Декартово местоположение&Промежуток результата \\
$y=+0$&$x>0$&Точно выше положительной оси x&$+0$ \\
$y>0$&$x>0$&Квадрант I&$+0 < result < \pi/2$ \\
$y>0$&$x=\pm 0$&Положительная ось y&$\pi/2$ \\
$y>0$&$x<0$&Квадрант II&$\pi/2 < result < \pi$ \\
$y=+0$&$x<0$&Точно ниже негативной оси x&$\pi$ \\
$y=-0$&$x<0$&Точно выше отрицательной оси x&$\pi$ \\
$y<0$&$x<0$&Квадрант III&$-\pi < result < -\pi/2$ \\
$y<0$&$x=\pm 0$&Отрицательная ось y&$-\pi/2$ \\
$y<0$&$x>0$&Квадрант IV&$-\pi/2 < result < -0$ \\
$y=-0$&$x>0$&Точно ниже положительной оси x&$-0$ \\
$y=+0$&$x=+0$&Рядом с центром&$+0$ \\
$y=-0$&$x=+0$&Рядом с центром&$-0$ \\
$y=+0$&$x=-0$&Рядом с центром&$\pi$ \\
$y=-0$&$x=-0$&Рядом с центром&$-\pi$ \\
\end{tabular*}
\end{flushleft}

Следует отметить, что случай $y=0,x=0$ при отсутствии минус ноля
является ошибкой, но четыре случая $y=\pm 0,x=\pm 0$ определены при условии
существования минус ноля.

Если не указывать \emph{x}, \emph{y} может быть комплексным.
Результатом функции является арктангенс \emph{y}, который может быть определён
следующей формулой:
\begin{tabbing}
\hskip 10pc\=\kill
Арктангенс\>$\log (1+\emph{i}\emph{y}) - \log (1-\emph{i}\emph{y}) \over 2\emph{i}$
\end{tabbing}
 
Для некомплексного аргумента \emph{y}, результат будет некомплексным и будет
лежать между $-\pi/2$ и $\pi/2$ (оба невключительно).
\end{defun}

\begin{defun}[Константа]
pi

Данная глобальная переменная содержит значение наиболее приближенное к $pi$ в
\emph{длинном} формате числа с плавающей точкой.
Например:
\begin{lisp}
(defun sind (x)~~~~~;\textrm{Аргумент в градусах} \\
~~(sin (* x (/ (float pi x) 180))))
\end{lisp}
Приближение к $pi$ с другими точностями может быть выполнено с помощью
\cd{(float pi \emph{x})}, где \emph{x} является числом с плавающей точкой
необходимой точности, или с помощью  \cd{(coerce pi \emph{type})}, где
\emph{type} является именем необходимого типа, как например \cdf{short-float}.
\end{defun}

\penalty-10000\relax

\begin{defun}[Функция]
sinh number \\
cosh number \\
tanh number \\
asinh number \\
acosh number \\
atanh number

Данные функции вычисляют гиперболический синус, косинус, тангенс, арксинус,
арккосинус и арктангенс, которые для аргумента \emph{z} математически определены
следующим образом:
\begin{tabbing}
\hskip 10pc\=\kill
Гиперболический синус~~\>$ (\emph{e}^{z}-\emph{e}^{-z})/2 $ \\
Гиперболический косинус~~\>$ (\emph{e}^{z}+\emph{e}^{-z})/2 $ \\
Гиперболический тангенс~~\>$ (\emph{e}^{z}-\emph{e}^{-z})/(\emph{e}^{z}+\emph{e}^{-z}) $ \\[2pt]
Гиперболический арксинус~~\>$ \log\left(\emph{z}+\sqrt{1+\emph{z}^2}\right) $ \\[2pt]
Гиперболический арккосинус~~\>$ \log\left(\emph{z}+(\emph{z}+1)\sqrt{(\emph{z}-1)/(\emph{z}+1)}\right) $ \\[2pt]
Гиперболический арктангенс~~\>$ \log\left((1+\emph{z})\sqrt{1/(1-\emph{z}^2)}\right) $
\end{tabbing}

Следует отметить, что результат \cdf{acosh} может быть комплексным, даже если
аргумент комплексным числом не был. Это происходит если аргумент меньше 1.
Также результат \cdf{atanh} может быть комплексным, даже если аргумент не был
комплексным. Это происходит если абсолютное значение аргумента было больше 1.
\end{defun}

\subsection{Точки ветвления, главные значения и краевые условия на комплексной плоскости}
\label{BRANCH-CUTS-SECTION}

Многие иррациональные и трансцендентные функции на комплексной
плоскости многозначны. Например, в общем случае, логарифмическая функция имеет 
бесконечное количество комплексных значений. В каждом таком случае для возврата 
должно быть выбрано одно главное значение.

Common Lisp определяет точки ветвления, главные значения и краевые условия для
комплексных функции в соотвествие с предложениями для комплексных функций в APL
\cite{APL-BRANCH-CUTS}.
Содержимое раздела по большей части скопировано из этих предложений.

[Однако перевод не сделан. Не могу назвать себя знатоком ТФКП]


\section{Приведение типов и доступ к компонентам чисел}

Тогда как большинство арифметических функций будут оперировать любым типом
чисел, выполняя при необходимости приведения, следующие функции позволяют явно
преобразовывать типы данных.

\begin{defun}[Функция]
float number &optional other

Преобразует любое некомплексное число в число с плавающей точкой.
При отсутствии необязательного параметра, если \emph{number} уже
является числом с плавающей точкой, то оно и будет возвращено, иначе число будет
преобразовано в \cdf{single-float}.
Если аргумент \emph{other} указан, тогда он должен быть числом с плавающей
точкой, и \emph{number} будет конвертирован в такой же формат как у
\emph{other}.

Смотрите также \cdf{coerce}.
\end{defun}

\begin{defun}[Функция]
rational number \\
rationalize number

Каждая из этих функций преобразует любое некомплексное число в рациональное.
Если аргумент уже является рациональным, он возвращается как есть.
Две функции различаются в том, как они обрабатывают числа с плавающей точкой.

\cdf{rational} предполагает, что число с плавающей точкой совершенно точно, и
возвращает рациональное число математически эквивалентное значению числа с
плавающей точкой.

\cdf{rationalize} предполагает, что число с плавающей точкой приближенное, и
может возвращать любое рациональное число, для которого исходное число является
наилучшим приближением. Функция пытается сохранить числитель и знаменатель
наименьшими насколько это возможно.

Следующие тождества всегда справедливы
\begin{lisp}
(float (rational \emph{x}) \emph{x}) \EQ\ \emph{x}
\end{lisp}
и
\begin{lisp}
(float (rationalize \emph{x}) \emph{x}) \EQ\ \emph{x}
\end{lisp}
То есть, преобразование числа с плавающей точкой любым из методов туда и обратно
даёт исходное число.
Различие в том, что \cdf{rational} обычно имеет более простую недорогую
реализацию, тогда как \cdf{rationalize} представляет более <<красивое>> число.
\end{defun}

\begin{defun}[Функция]
numerator rational \\
denominator rational

Эти функции принимают рациональное число (целое или дробное)
и возвращают в качестве целого числа числитель или знаменатель дроби,
приведённое к каноническому виду. Числитель целого числа и является этим
числом. Знаменатель целого числа \cd{1}. Следует отметить, что
\begin{lisp}
(gcd (numerator \emph{x}) (denominator \emph{x})) \EV\ 1
\end{lisp}
Знаменатель будет всегда строго положительным числом. Числитель может быть любым
целым числом.
Например:
\begin{lisp}
(numerator (/ 8 -6)) \EV\ -4 \\
(denominator (/ 8 -6)) \EV\ 3
\end{lisp}
\end{defun}

В Common Lisp'е нет функции \cdf{fix}, потому что есть несколько интересных
способов преобразовать нецелое число к целому.
Эти способы представлены функциями ниже, которые выполняются не только
преобразование типа, но также некоторые нетривиальные вычисления.

\begin{defun}[Функция]
floor number &optional divisor \\
ceiling number &optional divisor \\
truncate number &optional divisor \\
round number &optional divisor

При вызове с одним аргументом, каждая из этих функций преобразует аргумент
\emph{number} (который не может быть комплексным числом) в целое число.
Если аргумент уже является целым числом, то он немедленно возвращается в
качестве результата.
Если аргумент дробь или число с плавающей точкой, для конвертации функции используют различные
алгоритмы.

\cdf{floor} преобразовывает аргумент путём отсечения к отрицательной
бесконечности, то есть, результатом является наибольшее целое число, которое не
больше чем аргумент.

\cdf{ceiling} преобразовывает аргумент путём отсечения к положительной
бесконечности, то есть, результатом является наименьшее целое число, которое не
меньше чем аргумент.

\cdf{ceiling} преобразовывает аргумент путём отсечения к нулю, то есть,
результатом является целое число с таким же знаком, которое имеет наибольшую
целую величину, но не большую чем аргумент.

Следующая таблица содержит то, что возвращают четыре функции для разных
аргументов.

\begin{flushleft}
\cf
\begin{tabular}{@{}ccccc@{}}
\textrm{Аргумент}&floor&ceiling&truncate&round\\
\hlinesp
~2.6& ~2& ~3& ~2& ~3 \\
~2.5& ~2& ~3& ~2& ~2 \\
~2.4& ~2& ~3& ~2& ~2 \\
~0.7& ~0& ~1& ~0& ~1 \\
~0.3& ~0& ~1& ~0& ~0 \\
-0.3& -1& ~0& ~0& ~0 \\
-0.7& -1& ~0& ~0& -1 \\
-2.4& -3& -2& -2& -2 \\
-2.5& -3& -2& -2& -2 \\
-2.6& -3& -2& -2& -3 \\
\hline
\end{tabular}
\end{flushleft}

Если указан второй аргумент \emph{divisor}, тогда аргумент \emph{number} будет
разделен на \emph{divisor}, а затем уже будут проведены вышеописанные действия.
Например, \cd{(floor 5~2)}~\EQ~\cd{(values (floor (/~5~2)))}, но первый вариант
потенциально эффективнее.
\emph{divisor} может любым числом, кроме комплексного.
\emph{divisor} не может равняться нулю.
Случай вызова функции с одним аргументом эквивалентен вызову с двумя
аргументами, последний из которых равен \cd{1}.

Каждая из функций возвращает \emph{два} значения. Второе значение является
остатком и может быть получено с помощью \cdf{multiple-value-bind} или другими
подобными конструкциями.
Если любая из этих функций получает два аргумента \emph{x} и \emph{y} и
возвращает \emph{q} и \emph{r}, тогда  $\emph{q} \cdot
\emph{y}+\emph{r}=\emph{x}$.
Первое значение результата \emph{q} всегда целочисленное.
Остаток \emph{r} целочисленный, если оба аргумента были целочисленными, и
рациональное, если оба аргумента были рациональными,
и с плавающей точкой, если один из аргументов был с плавающей точкой.
Если при вызове был указан один аргумент, то тип данных остатка всегда такой же
как и этот аргумент.

Когда указывается только один аргумент, сумма двух значений результата точно
равняется переданному в параметре значению.
\end{defun}

\begin{defun}[Функция]
mod number divisor \\
rem number divisor

\cdf{mod} выполняет операцию \cdf{floor} для двух аргументов и возвращает
\emph{второй} результат \cdf{floor}.
Таким же образом, \cdf{rem} выполняет операцию \cdf{truncate} для двух аргументов и возвращает
\emph{второй} результат \cdf{truncate}.

Таким образом \cdf{mod} и \cdf{rem} являются обычными функциями вычисления
остатка от деления двух чисел. Аргументы могут быть также числами с плавающей
точкой.
\begin{lisp}
\hskip 0.5\textwidth\=\kill
(mod 13 4) \EV\ 1\>(rem 13 4) \EV\ 1 \\
(mod -13 4) \EV\ 3\>(rem -13 4) \EV\ -1 \\
(mod 13 -4) \EV\ -3\>(rem 13 -4) \EV\ 1 \\
(mod -13 -4) \EV\ -1\>(rem -13 -4) \EV\ -1 \\
(mod 13.4 1) \EV\ 0.4\>(rem 13.4 1) \EV\ 0.4 \\
(mod -13.4 1) \EV\ 0.6\>(rem -13.4 1) \EV\ -0.4
\end{lisp}
\end{defun}

\begin{defun}[Функция]
ffloor number &optional divisor \\
fceiling number &optional divisor \\
ftruncate number &optional divisor \\
fround number &optional divisor

Эти функции похожи на \cdf{floor}, \cdf{ceiling}, \cdf{truncate} и \cdf{round}
за исключением того, что результат (первый из двух) всегда целое число, а не
число с плавающей точкой. Это примерно, как если бы \cdf{ffloor} передала
аргументы в \cdf{floor}, а затем применила к первому результату \cdf{float} и
вернула полученную пару значений. Однако, на практике \cdf{ffloor} может быть
реализована более эффективно. Такое же описание подходит к остальным трём
функциям. Если первый аргумент является числом с плавающей точкой, и второй
аргумент не точнее типа первого, тогда первый результат будет такого же типа как
первый аргумент.
Например:
\begin{lisp}
(ffloor -4.7) \EV\ -5.0 and 0.3 \\
(ffloor 3.5d0) \EV\ 3.0d0 and 0.5d0
\end{lisp}
\end{defun}

\begin{defun}[Функция]
decode-float float \\
scale-float float integer \\
float-radix float \\
float-sign float1 &optional float2 \\
float-digits float \\
float-precision float \\
integer-decode-float float

Функция \cdf{decode-float} принимает числа с плавающей точкой и возвращает три
значения.

Первое значение является мантиссой и числом того же типа, что и
аргумент. Второе значение является целочисленной экспонентой.
Третье значение отображает знак аргумента (-1.0 или 1.0) и является и числом
того же типа, что и аргумент .
Пусть \emph{b} есть система счисления для отображения чисел с плавающей точкой,
тогда \cdf{decode-float} делит аргумент на \emph{b} в некоторой степени, чтобы
привести значение в промежуток включая 1/\emph{b} и не включая 1 и возвращает
частное в качестве первого значения FIXME. Однако, если аргумент равен нулю
результат равен абсолютному значению аргумента (то есть, если существует
отрицательный ноль, то для него возвращается положительный ноль).

Второе значение \cdf{decode-float} является целочисленным экспонентой \emph{e},
которая равняется степени, в которую было возведено \emph{b}.
Если аргумент равен нулю, то может быть возвращено любое целое число, при
условии, что тожество, описанное ниже для \cdf{scale-format}, имеет место быть.

Третье значение \cdf{decode-float} является числом с плавающей точкой в том же
формате, что и аргумент, абсолютное значение которого равно 1, и знак совпадает
со знаком аргумента.

Функция \cdf{scale-float} принимает число с плавающей точкой \emph{f} (не
обязательно между 1/\emph{b} и 1) и целое число \emph{k}, и возвращает \cd{(*
  \emph{f} (expt (float \emph{b} \emph{f}) \emph{k}))}.
(Использование \cdf{scale-float} может быть более эффективным, чем использование
возведения в степень или умножения и позволяет избежать переполнений).

Следует отметить, что
\begin{lisp}
(multiple-value-bind (signif expon sign) \\*
~~~~~~~~~~~~~~~~~~~~~(decode-float \emph{f}) \\*
~~(scale-float signif expon)) \\*
\EQ\ (abs \emph{f})
\end{lisp}
и
\begin{lisp}
(multiple-value-bind (signif expon sign) \\*
~~~~~~~~~~~~~~~~~~~~~(decode-float \emph{f}) \\*
~~(* (scale-float signif expon) sign)) \\*
\EQ\ \emph{f}
\end{lisp}

Функция \cdf{float-radix} возвращает (в качестве целого числа)
основание \emph{b} для числа с плавающей точкой.

Функция \cdf{float-sign} возвращает такое число с плавающей точкой \emph{z}, что
\emph{z} и \emph{float1} имеют одинаковый знак, и \emph{z} и \emph{float2} имеют
равное абсолютное значение.
Аргумент \emph{float2} по-умолчанию имеет значение \cd{(float 1 \emph{float1})}.
Таким образом \cd{(float-sign x)} всегда возвращает \cd{1.0} или \cd{-1.0} в
таком же формате и с тем же знаком, что и \emph{x}. (Следует отметить, что если
реализация содержит различные представления для отрицательного и положительного
нулей, тогда \cd{(float-sign -0.0)} \EV\ \cd{-1.0}.)

Функция \cdf{float-digits} возвращает целочисленное количество
цифр используемых в представлении аргумента. Функция \cdf{float-precision}
возвращает целочисленное количество цифр в мантиссе аргумента. Если аргумент
равен нулю, то результатом также будет ноль.
Для нормализованных чисел с плавающей точкой, результаты \cdf{float-digits} и
\cdf{float-precision} будут такими же. Но в случае с денормализованными числами
или нулём точность будет меньше чем количество цифр в представлении.

Функция \cdf{integer-decode-float} похожа на \cdf{decode-float}, но в качестве
первого значение возвращает масштабированную целочисленную мантиссу.
Для аргумента \emph{f}, это число будет строго меньше чем
\begin{lisp}
\cd{(expt \emph{b} (float-precision \emph{f}))}
\end{lisp}
на не менее чем 
\begin{lisp}
\cd{(expt \emph{b} (- (float-precision \emph{f}) 1))}
\end{lisp}
за исключением того, что если \emph{f} равно нулю, тогда целочисленное значение
также будет равно нулю.

Второе значение имеет такую же связь с первым значением, как и для
\cdf{decode-float}: 
\begin{lisp}
(multiple-value-bind (signif expon sign) \\*
~~~~~~~~~~~~~~~~~~~~~(integer-decode-float \emph{f}) \\*
~~(scale-float (float signif \emph{f}) expon)) \\*
\EQ\ (abs \emph{f})
\end{lisp}

Третье значение \cdf{integer-decode-float} будет \cd{1} или \cd{-1}.

\beforenoterule
\begin{rationale}
Эти функции позволяют писать машиннонезависиммые или как минимум
машиннопараметризированные приложения с вычислениями чисел с плавающими точками
с необходимой эффективностью.
\end{rationale}
\afternoterule
\end{defun}

\begin{defun}[Функция]
complex realpart &optional imagpart

Аргументы должны быть некомплексными числами. Результатом является число,
которое имеет действительную часть \emph{realpart} и мнимую \emph{imagpart},
возможно конвертированные к одному типу.
Если \emph{imagpart} не указаны, тогда используется \cd{(coerce 0 (type-of
  \emph{realpart}))}.
Необходимо отметить, что если обе части комплексного число рациональны, и мнимая
часть равна нуля, то результатом будет только действительная часть
\emph{realpart} так как в силу вступят правила канонизации.
Таким образом результат \cdf{complex} не всегда является комплексным числом. Он
может быть просто рациональным числом (\cdf{rational}).
\end{defun}

\begin{defun}[Функция]
realpart number \\
imagpart number

Эти функции возвращают действительную и мнимую части комплексного числа. Если
\emph{number} не является комплексным числом, тогда \cdf{realpart} возвращает
это число, а \cdf{imagpart} возвращает \cd{(* 0 \emph{number})}, другими
словами, \cd{0} того типа, каким был аргумент.
\end{defun}

\section{Логические операции над числами}

Логические операции в данном разделе в качестве аргументов требуют целых
чисел. Передача числа любого другого формата является ошибкой.
Функции обрабатывают целые числа, как если бы они были представлены в двух
системах счисления. FIXME

\beforenoterule
\begin{implementation}
Внутренне, конечно, реализация Common Lisp'а может использовать и может и нет
представление числа с дополнительным кодом. Все, что необходимо это чтобы
логические операции выполняли вычисление так, как описано в разделе.
\end{implementation}
\afternoterule

Логические операции предоставляют удобный способ для представления бесконечного
вектора битов. Пусть такой концептуальный вектор будет индексироваться с помощью
неотрицательного целого. Тогда биту $j$ присваивается <<вес (weight)>> $2^{j}$.
Предположим, что лишь конечное число битов являются 1 или только конечное
число битов являются 0.
Вектор, у которого конечное число битов 1, представлен как сумма всех весов этих
битов, и является положительным числом.
Вектор, у которого конечное число битов 0, представлен как \cd{-1} минус сумма
всех весов этих битов, и является отрицательным числом. FIXME

Данный метод использования целых чисел для представления битовых векторов может
в свою очередь использоваться для представления множеств. Предположим, что
некоторая (возможно бесконечная) совокупность рассуждений для множеств отображается
в неотрицательные целые числа. FIXME
Тогда множество может быть представлено как битовый вектор. Элемент принадлежит
множеству, если бит, индекс которого соответствует элементу, является 1.
Таким образом все конечные множества могут быть представлены с помощью
положительных целых, и множества, дополнения которых конечны, с помощью
отрицательных чисел. Функции \cdf{logior}, \cdf{logand} и \cdf{logxor},
определённые ниже, вычисляют объединение, пересечение и симметричную разность для
таких множеств.

\begin{defun}[Функция]
logior &rest integers

Функция возвращает побитовое логическое \emph{или} для аргументов.
Если не задан ни один аргумент, возвращается ноль.
\end{defun}

\begin{defun}[Функция]
logxor &rest integers

Функция возвращает побитовое логическое \emph{исключающее или} для аргументов.
Если не задан ни один аргумент, возвращается ноль.
\end{defun}

\begin{defun}[Функция]
logand &rest integers

Функция возвращает побитовое логическое \emph{и} для аргументов.
Если ни один аргумент не задан, возвращается ноль.
\end{defun}

\begin{defun}[Функция]
logeqv &rest integers

Функция возвращает побитовую логическую \emph{эквивалентность} (также известную
как \emph{исключающее отрицающее или} для аргументов.
Если не задан ни один аргумент, возвращается ноль.
\end{defun}

\begin{defun}[Функция]
lognand integer1 integer2 \\
lognor integer1 integer2 \\
logandc1 integer1 integer2 \\
logandc2 integer1 integer2 \\
logorc1 integer1 integer2 \\
logorc2 integer1 integer2

Данные функции служат для шести нетривиальные битовый логический операций для
двух аргументов. Так как они не ассоциативны, они принимают только два
аргумента.
\begin{lisp}
(lognand \emph{n1} \emph{n2}) \EQ\ (lognot (logand \emph{n1} \emph{n2})) \\[2pt]
(lognor \emph{n1} \emph{n2}) \EQ\ (lognot (logior \emph{n1} \emph{n2})) \\[2pt]
(logandc1 \emph{n1} \emph{n2}) \EQ\ (logand (lognot \emph{n1}) \emph{n2}) \\[2pt]
(logandc2 \emph{n1} \emph{n2}) \EQ\ (logand \emph{n1} (lognot \emph{n2})) \\[2pt]
(logorc1 \emph{n1} \emph{n2}) \EQ\ (logior (lognot \emph{n1}) \emph{n2}) \\[2pt]
(logorc2 \emph{n1} \emph{n2}) \EQ\ (logior \emph{n1} (lognot \emph{n2}))
\end{lisp}
\end{defun}

В следующей таблице перечислены десять битовых логических операций для двух
целых чисел.
\begin{flushleft}
\cf
\begin{tabular}{@{}rlllll@{}}
\emph{integer1}&0&0&1&1 \\
\emph{integer2}&0&1&0&1&\textrm{Имя операции} \\
\hlinesp
logand~~&0&0&0&1&\textrm{и} \\
logior~~&0&1&1&1&\textrm{или} \\
logxor~~&0&1&1&0&\textrm{исключающее или} \\
logeqv~~&1&0&0&1&\textrm{эквивалентность (исключающее отрицающее или)} \\
lognand~&1&1&1&0&\textrm{не и} \\
lognor~~&1&0&0&0&\textrm{не или} \\
logandc1&0&1&0&0&\textrm{не \emph{integer1} и \emph{integer2}} \\
logandc2&0&0&1&0&\textrm{\emph{integer1} и не \emph{integer2}} \\
logorc1~&1&1&0&1&\textrm{не \emph{integer1} или \emph{integer2}} \\
logorc2~&1&0&1&1&\textrm{\emph{integer1} или не \emph{integer2}} \\
\hline
\end{tabular}
\end{flushleft}

\begin{defun}[Функция][Константа]
boole op integer1 integer2 \\
boole-clr \\
boole-set \\
boole-1 \\
boole-2 \\
boole-c1 \\
boole-c2 \\
boole-and \\
boole-ior \\
boole-xor \\
boole-eqv \\
boole-nand \\
boole-nor \\
boole-andc1 \\
boole-andc2 \\
boole-orc1 \\
boole-orc2

Функция \cdf{boole} принимает операцию \emph{op} и два целых числа, и возвращает
целое число полученное применением операции \emph{op} к этим двум числам. Точные
значения шестнадцати констант зависят от реализации, но они подходят для
использования в качестве первого аргумента для \cdf{boole}:
\begin{flushleft}
\cf
\begin{tabular}{@{}rlllll@{}}
\emph{integer1}&0&0&1&1 \\
\emph{integer2}&0&1&0&1&\textrm{Выполняемая операция} \\
\hlinesp
boole-clr~~&0&0&0&0&\textrm{всегда 0} \\
boole-set~~&1&1&1&1&\textrm{всегда 1} \\
boole-1~~~~&0&0&1&1&\emph{integer1} \\
boole-2~~~~&0&1&0&1&\emph{integer2} \\
boole-c1~~~&1&1&0&0&\textrm{дополнение \emph{integer1}} \\
boole-c2~~~&1&0&1&0&\textrm{дополнение \emph{integer2}} \\
boole-and~~&0&0&0&1&\textrm{и} \\
boole-ior~~&0&1&1&1&\textrm{или} \\
boole-xor~~&0&1&1&0&\textrm{исключающее или} \\
boole-eqv~~&1&0&0&1&\textrm{эквивалентность (исключительное отрицающее или)} \\
boole-nand~&1&1&1&0&\textrm{не и} \\
boole-nor~~&1&0&0&0&\textrm{не или} \\
boole-andc1&0&1&0&0&\textrm{не \emph{integer1} и \emph{integer2}} \\
boole-andc2&0&0&1&0&\textrm{\emph{integer1} и не \emph{integer2}} \\
boole-orc1~&1&1&0&1&\textrm{не \emph{integer1} или \emph{integer2}} \\
boole-orc2~&1&0&1&1&\textrm{\emph{integer1} или не \emph{integer2}} \\
\hline
\end{tabular}
\end{flushleft}

Таким образом \cdf{boole} может вычислять все шестнадцать логических функций для
двух аргументов. В целом,
\begin{lisp}
(boole boole-and x y) \EQ\ (logand x y)
\end{lisp}
и далее по аналогии. \cdf{boole} полезна, когда необходимо
параметризировать процедуру так, что они может использовать одну из нескольких
логических операций. 
\end{defun}

\begin{defun}[Функция]
lognot integer

Функция возвращает битовое логическое \emph{отрицание} аргумента.
Каждый бит результата является дополнение соответствующего исходного бита
аргумента.
\begin{lisp}
(logbitp \emph{j} (lognot \emph{x})) \EQ\ (not (logbitp \emph{j} \emph{x}))
\end{lisp}
\end{defun}

\begin{defun}[Функция]
logtest integer1 integer2

\cdf{logtest} является предикатом, который истинен, если любой бит определённый
как 1 в \emph{integer1} также является соответствующим битом 1 в
\emph{integer2}.
\begin{lisp}
(logtest \emph{x} \emph{y}) \EQ\ (not (zerop (logand \emph{x} \emph{y})))
\end{lisp}
\end{defun}

\begin{defun}[Функция]
logbitp index integer

Предикат \cdf{logbitp} является истиной, если бит в позиции \emph{index} в целом
числе \emph{integer} (то есть, его вес $2^{index}$), является 1, иначе предикат
ложен.
Например:
\begin{lisp}
(logbitp 2 6) \textrm{is true} \\
(logbitp 0 6) \textrm{is false} \\
(logbitp \emph{k} \emph{n}) \EQ\ (ldb-test (byte 1 \emph{k}) \emph{n})
\end{lisp}

\emph{index} должен быть неотрицательным целым числом.
\end{defun}

\begin{defun}[Функция]
ash integer count

Если \emph{count} положительное число, данная функция арифметически сдвигает
число \emph{integer} влево на количество бит \emph{count}. Если \emph{count}
отрицательное число, функция сдвигает число \emph{integer} вправо.  
Знак результата всегда такое же как и исходного числа \emph{integer}.

Говоря математически, эта операция выполняет вычисления
$\emph{floor}(\emph{integer}\cdot2^{count}$).

Логически, функция перемещает все биты числа \emph{integer} влево, добавляя
нулевые биты в конец, или вправо отбрасывая биты в конце числа. (В этом
контексте вопрос о том, что сдвигается налево, не относится к делу. Целые
числа, рассматриваемые как строки битов, являются <<полубесконечными>>,
то есть концептуально расширяются бесконечно влево FIXME.)
Например:
\begin{lisp}
(logbitp \emph{j} (ash \emph{n} \emph{k})) \EQ\ (and (>= \emph{j} \emph{k}) (logbitp (- \emph{j} \emph{k}) \emph{n}))
\end{lisp}
\end{defun}

\begin{defun}[Функция]
logcount integer

Функция возвращает количество бит в числе \emph{integer}.
Если \emph{integer} положительное число, тогда подсчитаны будут биты
\cd{1}. Если число \emph{integer} отрицательно, то в два раза дополненной
(two's-complement) бинарной форме будут подсчитаны биты
\cd{0}. FIXME 
Результатом всегда является неотрицательное целое число.
Например:
\begin{lisp}
(logcount 13) \EV\ 3~~~~~~;\textrm{Бинарное представление} ...0001101 \\
(logcount -13) \EV\ 2~~~~~;\textrm{Бинарное представление} ...1110011 \\
(logcount 30) \EV\ 4~~~~~~;\textrm{Бинарное представление} ...0011110 \\
(logcount -30) \EV\ 4~~~~~;\textrm{Бинарное представление} ...1100010
\end{lisp}
Следующее тождество всегда верно:
\begin{lisp}
(logcount x) \EQ\ (logcount (- (+ x 1))) \\
~~~~~~~~~~~~~\EQ\ (logcount (lognot x))
\end{lisp}
\end{defun}

\begin{defun}[Функция]
integer-length integer

Данная функция выполняет следующие вычисления
\begin{tabbing}
$ \emph{ceiling}(\log_2 ({\bf if}\; \emph{integer} < 0 \;{\bf then} \;
    -\emph{integer} \;{\bf else}\; \emph{integer}+1)) $
\end{tabbing}
Эта функция полезна в двух случаях.
Первое, если целое число \emph{integer} не отрицательно, тогда его значение
может быть представлено в беззнаковой бинарной форме в поле, ширина которого в
битах не меньше чем \cd{(integer-length \emph{integer})}.
Второе, независимо от знака числа \emph{integer}, его значение может быть
представлено как знаковая бинарная два раза дополненная (two's-complement) форма
в поле, ширина которого в битах не меньше чем
 \cd{(+ (integer-length \emph{integer}) 1)}. FIXME
Например:
\begin{lisp}
(integer-length 0) \EV\ 0 \\
(integer-length 1) \EV\ 1 \\
(integer-length 3) \EV\ 2 \\
(integer-length 4) \EV\ 3 \\
(integer-length 7) \EV\ 3 \\
(integer-length -1) \EV\ 0 \\
(integer-length -4) \EV\ 2 \\
(integer-length -7) \EV\ 3 \\
(integer-length -8) \EV\ 3
\end{lisp}
\end{defun}


\section{Функции для манипуляции с байтами}

Common Lisp содержит несколько функций для работы с полями смежных битов
произвольной длины, находящимися где-нибудь в целом числе.
Такое множество смежных битов называется \emph{байт}.
В Common Lisp'е термин \emph{байт} не означает некоторое определённое количество
бит (как например восемь), а обозначает поле произвольной, определяемой
пользователем длины.

Функции обработки байтом используют объекты, называющиеся \emph{спецификаторы
  байта} для определения заданной позиции байта в целом числе.
Представление спецификатора байта зависит от реализации. В частности оно может
быть или не быть числом.
Достаточно знать, что функция \cdf{byte} будет создавать такой спецификатор, и
что функция для обработки байта будет принимать его.
Функция \cdf{byte} принимает два целых числа указывающих на \emph{позицию} и
\emph{размер} байта и возвращает его спецификатор.
Такой спецификатор определяет байт, у которого ширина равна \emph{размеру}, и
биты которого имеют веса с $ 2^{position+size-1} $
по $ 2^{position} $.

\begin{defun}[Функция]
byte size position

\cdf{byte} принимает два целых числа, указывающих на размер и позицию байта и
возвращает спецификатор, который может использоваться в функциях работы с
байтами.
\end{defun}

\begin{defun}[Функция]
byte-size bytespec \\
byte-position bytespec

Принимая спецификатор байта, \cdf{byte-size} возвращает размер указанного
байта, а \cdf{byte-position} --- позицию этого байта.
Например:
\begin{lisp}
(byte-size (byte \emph{j} \emph{k})) \EQ\ \emph{j} \\
(byte-position (byte \emph{j} \emph{k})) \EQ\ \emph{k}
\end{lisp}
\end{defun}

\begin{defun}[Функция]
ldb bytespec integer

Спецификатор байта \emph{bytespec} указывает на байт, который будет извлечён из
целого числа \emph{integer}.
Результат возвращается в качестве неотрицательного целого числа.
Например:
\begin{lisp}
(logbitp \emph{j} (ldb (byte \emph{s} \emph{p}) \emph{n})) \EQ\ (and (< \emph{j} \emph{s}) (logbitp (+ \emph{j} \emph{p}) \emph{n}))
\end{lisp}
Имя функции \cdf{ldb} означает <<load byte (загрузить байт)>>.

Если аргумент \emph{integer} задан формой, которая может использоваться в
\cdf{setf}, тогда \cdf{setf} может использоваться вместе с \cdf{ldb} для
изменения байта в целом числе указанном в форме \emph{integer}.
Это действие выполняется с помощью операции \cdf{dpb}.
\end{defun}

\begin{defun}[Функция]
ldb-test bytespec integer

\cdf{ldb-test} является предикатом, который истинен, если любой из битов,
указанных в спецификаторе байта \emph{bytespec}, в целом числе \emph{integer}
имеет значение 1.
Другими словами, предикат истинен, если указанное поле не равно нулю.
\begin{lisp}
(ldb-test \emph{bytespec} \emph{n}) \EQ\ (not (zerop (ldb \emph{bytespec} \emph{n})))
\end{lisp}
\end{defun}

\begin{defun}[Функция]
mask-field bytespec integer

Функция похожа а \cdf{ldb}. Однако результат содержит указанный байт целого
числа \emph{integer} в той же позиции, что указана в спецификаторе
\emph{bytespec}, а не в нулевой позиции, как делает \cdf{ldb}.
Таким образом результат в указанном байте совпадает с целым числом
\emph{integer} , но в остальных местах имеет нулевые биты.
Например:
\begin{lisp}
(ldb \emph{bs} (mask-field \emph{bs} \emph{n})) \EQ\ (ldb \emph{bs} \emph{n}) \\
 \\
(logbitp \emph{j} (mask-field (byte \emph{s} \emph{p}) \emph{n})) \\
~~~\EQ\ (and (>= \emph{j} \emph{p}) (< \emph{j} (+ \emph{p} \emph{s})) (logbitp \emph{j} \emph{n})) \\
 \\
(mask-field \emph{bs} \emph{n}) \EQ\ (logand \emph{n} (dpb -1 \emph{bs} 0))
\end{lisp}

Если аргумент \emph{integer} задан формой, которая может использоваться в
\cdf{setf}, тогда \cdf{setf} может использоваться вместе с \cdf{mask-field} для
изменения байта в целом числе указанном в форме \emph{integer}.
Это действие выполняется с помощью операции \cdf{deposit-field}.
\end{defun}

\begin{defun}[Функция]
dpb newbyte bytespec integer

Данная функция возвращает число, которое похоже на \emph{integer} за исключением
 битов, указанных спецификатором \emph{bytespec.}. Пусть \emph{s} будет
 размером, указанным в спецификаторе \emph{bytespec}.
Тогда целое число \emph{newbyte} будет интерпретировано, будучи выровненным вправо,
как если бы было результатом \cdf{ldb}.
Например:
\begin{lisp}
(logbitp \emph{j} (dpb \emph{m} (byte \emph{s} \emph{p}) \emph{n})) \\
~~\EQ\ (if \=(and (>= \emph{j} \emph{p}) (< \emph{j} (+ \emph{p} \emph{s}))) \\
\>(logbitp (- \emph{j} \emph{p}) \emph{m}) \\
\>(logbitp \emph{j} \emph{n}))
\end{lisp}
Имя функции \cdf{dpb} означает <<deposit byte (сохранить байт)>>.
\end{defun}

\begin{defun}[Функция]
deposit-field newbyte bytespec integer

Данная функция относится к \cdf{mask-field} так же, как \cdf{dpb} относится к
\cdf{ldb}.
Результатом является целое число, которое содержит биты \emph{newbyte} в том
месте, куда указывает спецификатор \emph{bytespec}, и биты числа \emph{integer}
во всех остальных местах.
\begin{lisp}
(logbitp \emph{j} (deposit-field \emph{m} (byte \emph{s} \emph{p}) \emph{n})) \\
~~~\EQ\ (if \=(and (>= \emph{j} \emph{p}) (< \emph{j} (+ \emph{p} \emph{s}))) \\
\>(logbitp \emph{j} \emph{m}) \\
\>(logbitp \emph{j} \emph{n}))
\end{lisp}
\end{defun}

\section{Случайные числа}
\label{RANDOM}

Функциональность Common Lisp'а для генерации псевдослучайных чисел была
аккуратно проработана, чтобы быть переносимой. Тогда как две реализации могут
создавать различные ряды псевдослучайных чисел, распределение значений должно
быть независимо от таких характеристик компьютера как размер слова.

\begin{defun}[Функция]
random number &optional state

\cd{(random \emph{n})} принимает положительное число \emph{n} и возвращает число
такого же типа между нулём (включительно) и \emph{n} (невключительно).
Число \emph{n} может быть целым или с плавающей точкой.
При этом используется приблизительно равномерное распределения. FIXME
Если \emph{n} целое число, каждый из возможных результатов встречается с
(примерной) вероятностью 1/\emph{n}.
(Слово <<примерно>> используется из-за различий реализаций. На практике отклонение
от однородности должно быть весьма мало.)

Аргумент \emph{state} должен быть объектом типа \cdf{random-state}.
По-умолчанию он равен значению переменной \cdf{*random-state*}.
Объект используется для сохранения состояния генератора псевдослучайных чисел и
изменяется как побочное действие от операции \cdf{random}.
\end{defun}

\begin{defun}[Переменная]
*random-state*

Данная переменная содержит объект типа \cdf{random-state}, который хранит
состояние для генератора случайных чисел, которое по-умолчанию используется 
функцией \cdf{random}.
Природа этой структуры данных зависит от реализации. Она может быть выведена на
печать и прочитана обратно, но не обязательно это может быть сделано между
реализациями.
Вызов \cdf{random} изменяет этот объект.
Связывание этой переменной с другим значением будет корректно сохраняться и
восстанавливаться.
\end{defun}

\begin{defun}[Функция]
make-random-state &optional state

Данная функция возвращает новый объект типа \cdf{random-state}, подходящий для
использования в качестве значения переменной \cdf{*random-state*}.
Если \emph{state} не указано или равно {\false}, тогда \cdf{make-random-state}
возвращает \emph{копию} текущего объекта состояния генератора псевдослучайных чисел
(значение переменной \cdf{*random-state*}). Если \emph{state} является объектом
состояния, то возвращается копия данного объекта.
Если \emph{state} равен {\true}, тогда новый возвращается новый объект
состояния, который в некотором смысле <<случайно>> инициализирован (текущим
временем например).
\end{defun}

\begin{defun}[Функция]
random-state-p object

\cdf{random-state-p} истинен, если аргумент является объектом состояния
генератора случайных чисел, иначе --- ложен.
\begin{lisp}
(random-state-p \emph{x}) \EQ\ (typep \emph{x} 'random-state)
\end{lisp}
\end{defun}

\section{Параметры реализации}

Значение констант, определённый в этом разделе, зависят от реализации. В
некоторых случаях они могут быть полезны для параметризации кода.

\begin{defun}[Константа]
most-positive-fixnum \\
most-negative-fixnum

Значением \cdf{most-positive-fixnum} является такое число, которое ближе всего
к положительной бесконечности.

Значением \cdf{most-negative-fixnum} является такое число, которое ближе всего
к отрицательной бесконечности.
\end{defun}

\begin{defun}[Константа]
most-positive-short-float \\
least-positive-short-float \\
least-negative-short-float \\
most-negative-short-float

Значением \cdf{most-positive-short-float} является такое значение числа
короткого формата с плавающей точкой, которое ближе всего к (но не равно)
положительной бесконечности. 

Значением \cdf{least-positive-short-float} является такое значение числа
короткого формата с плавающей точкой, которое ближе всего к (но не равно) нулю. 

Значением \cdf{least-negative-short-float} является такое значение числа
короткого формата с плавающей точкой, которое ближе всего к (но не равно) нулю
слева. (Следует отметить, что если разработчики поддерживают минус ноль, то
значение данной константы не может ему равняться.) 

Значением \cdf{most-negative-short-float} является такое значение числа
короткого формата с плавающей точкой, которое ближе всего к (но не равно)
отрицательной бесконечности. 
\end{defun}

\begin{defun}[Константа]
most-positive-single-float \\
least-positive-single-float \\
least-negative-single-float \\
most-negative-single-float \\
most-positive-double-float \\
least-positive-double-float \\
least-negative-double-float \\
most-negative-double-float \\
most-positive-long-float \\
least-positive-long-float \\
least-negative-long-float \\
most-negative-long-float

Данные константы аналогичны предыдущим, но указывают на другие форматы чисел с
плавающей точкой.
\end{defun}

\bigskip

\begin{defun}[Константа]
least-positive-normalized-short-float \\
least-negative-normalized-short-float

Значением \cdf{least-positive-normalized-short-float} является такое
положительное нормализованное число в коротком формате с плавающей запятой,
которое ближе всего к (но не равно) нулю. В реализации, которая не поддерживает
денормализованные числа, может равняться тому же, что и
\cdf{least-positive-short-float}. 

Значением \cdf{least-positive-normalized-short-float} является такое
отрицательное нормализованное число в коротком формате с плавающей запятой,
которое ближе всего к (но не равно) нулю. 
(Следует отметить, что если реализация поддерживает минус ноль, то данная
константа должна ему равняться.)
В реализации, которая не поддерживает
денормализованные числа, может равняться тому же, что и
\cdf{least-negative-short-float}. 
\end{defun}

\begin{defun}[Константа]
least-positive-normalized-single-float \\
least-negative-normalized-single-float \\
least-positive-normalized-double-float \\
least-negative-normalized-double-float \\
least-positive-normalized-long-float \\
least-negative-normalized-long-float

Данные константы аналогичны предыдущим, но указывают на другие форматы чисел с
плавающей точкой.
\end{defun}

\begin{defun}[Константа]
short-float-epsilon \\
single-float-epsilon \\
double-float-epsilon \\
long-float-epsilon

Эти константы для каждого формата с плавающей точкой содержат наименьшее
натуральное число с плавающей точкой \emph{e}, такое что при вычислении
выражение
\begin{lisp}
(not (= (float 1 \emph{e}) (+ (float 1 \emph{e}) \emph{e})))
\end{lisp}
истинно.
\end{defun}

\begin{defun}[Константа]
short-float-negative-epsilon \\
single-float-negative-epsilon \\
double-float-negative-epsilon \\
long-float-negative-epsilon

Эти константы для каждого формата с плавающей точкой содержат наименьшее
натуральное число с плавающей точкой \emph{e}, такое что при вычислении выражение
\begin{lisp}
(not (= (float 1 \emph{e}) (- (float 1 \emph{e}) \emph{e})))
\end{lisp}
истинно.
\end{defun}

\fi

\endgroup
