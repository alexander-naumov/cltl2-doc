%Part{MISC, Root = "CLM.MSS"}
%%%Chapter of Common Lisp Manual.  Copyright 1984, 1988, 1989 Guy L. Steele Jr.

\clearpage\def\pagestatus{FINAL PROOF}

\chapter{Miscellaneous Features}
%\chapter{Разнообразные дополнительные возможности}

In this chapter are described various things that don't
seem to fit neatly anywhere else in this book:
the compiler, the \cdf{documentation}
function, debugging aids, environment inquiries (including facilities
for calculating and measuring time), and the \cdf{identity} function.

% В этой главе описываются различные вещи, которые не могли быть описаны в
% каком-либо другом месте книги:
% компилятор, функция поиска документации \cdf{documentation}, отладочный
% функционал, информация о среде выполнения (включая подсчет и измерение времени)
% и функция идентичности \cdf{identity}. 

\section{The Compiler}
%\section{Компилятор}
\label{COMPILER-SECTION} 

The compiler is a program that may make code run faster by translating
programs into an implementation-dependent form that can
be executed more efficiently by the computer.  Most of the time
you can write programs without worrying about the compiler;
compiling a file of code should produce an equivalent but more
efficient program.  When doing more esoteric things, you may need to
think carefully about what happens at ``compile time'' and what happens
at ``load time.''  Then the \cdf{eval-when} construct
becomes particularly useful.

% Компилятор --- это программа, которая выполняет преобразование кода для того,
% чтобы его выполнение было более быстрым. Способ преобразования зависит от
% реализации. Обычно вы можете писать программы, не беспокоясь о
% компиляторе. Компиляция файла с кодом должна создавать эквивалентную, но более
% быструю программу. При выполнении эзотерических вещей, вам необходимо подумать о
% том, что случается во <<время компиляции>> и, что случает во <<время
% загрузки>>. В таких случаях бывает полезна конструкция \cdf{eval-when}.

Most declarations are not used by the Common Lisp interpreter;
they may be used to give advice to the compiler.  The compiler may attempt
to check your advice and warn you if it is inconsistent.

% Большинство деклараций не используются интерпретатором Common Lisp'а, они могут
% использования для советов компилятору. Компилятор может попытаться проверить
% ваш совет и уведомить вас, если совет противоречивый.

Unlike most other Lisp dialects, Common Lisp recognizes \cdf{special}
declarations in interpreted code as well as compiled code.

% В отличие от других диалектов Lisp'a, Common Lisp распознает декларацию
% \cdf{special} как в интерпретируемом, так и в скомпилированном коде.

The internal workings of a compiler will of course be highly
implementation-dependent.  The following functions provide a standard
interface to the compiler, however.

% Внутренняя работа компилятора конечно же будет будет во многом зависеть от
% конкретной реализации. Однако, следующие функции предоставляют стандартный
% интерфейс доступа к компилятору.

\begin{defun}[Function]
compile name &optional definition

\begin{obsolete}\noindent
If \emph{definition} is supplied, it should be a lambda-expression,
the interpreted function to be compiled.  If it is not supplied,
then \emph{name} should be a symbol with a definition that is a
lambda-expression; that definition is compiled
and the resulting compiled code is put back into the symbol
as its function definition.
\end{obsolete}

\emph{name} may be any function-name (a symbol or a list
whose car is \cdf{setf}---see section~\ref{FUNCTION-NAME-SECTION}).
One may write \cd{(compile '(setf cadr))} to compile the \cdf{setf}
expansion function for \cdf{cadr}.

If the optional \emph{definition} argument is supplied,
it may be either a lambda-expression (which is coerced to a function)
or a function to be compiled; if no \emph{definition} is supplied,
the \cdf{symbol-function} of the symbol is extracted and compiled.
It is permissible for the symbol to have a macro definition rather than
a function definition; both macros and functions may be compiled.

It is an error if the function to be compiled was defined interpretively
in a non-null lexical environment.  (An implementation is free to extend
the behavior of \cdf{compile} to compile such functions properly, but
portable programs may not depend on this capability.)  The consequences
of calling \cdf{compile} on a function that is already compiled
are unspecified.

\begin{obsolete}
The definition is compiled and a compiled-function object produced.
If \emph{name} is a non-{\nil}
symbol, then the compiled-function object is installed as the
global function definition of the symbol and the symbol is returned.
If \emph{name} is {\false}, then the compiled-function object itself is returned.
For example:
\begin{lisp}
\\
(defun foo ...) \EV\ foo~~~~~~~~\=;\textrm{A function definition} \\
(compile 'foo) \EV\ foo\>;\textrm{Compile it} \\
\>;\textrm{Now \cdf{foo} runs faster (maybe)} \\[4pt]
(compile {\false} \\
~~~~~~~~~'(lambda (a b c) (- (* b b) (* 4 a c)))) \\
~~~\EV\ \textrm{a compiled function of three arguments that computes $b^2-4ac$}
\end{lisp}
\end{obsolete}

\begin{newer}
X3J13 voted in June 1989 \issue{COMPILER-DIAGNOSTICS} to specify that
\cdf{compile} returns two additional values
indicating whether the compiler issued any diagnostics
(see section~\ref{COMPILER-DIAGNOSTICS-SECTION}).
\end{newer}
\end{defun}

X3J13 voted in March 1989 \issue{COMPILER-VERBOSITY} to add two new
keyword arguments \cd{:verbose} and \cd{:print}
to \cdf{compile-file} by analogy with \cdf{load}.
The new function definition is as follows.

\begin{defun}[Function]
compile-file input-pathname &key :output-file :verbose :print

The \emph{input-pathname} must be a valid file specifier, such as a pathname.
The defaults for \emph{input-filename} are taken from the variable
\cd{*default-pathname-defaults*}.
The file should be a Lisp source file;
its contents are compiled and written as a binary object file.

The \cd{:verbose} argument (which defaults to the value of
\cd{*compile-verbose*}), if true, permits \cdf{compile-file} to print a message
in the form of a comment to \cdf{*standard-output*} indicating what file is
being compiled and other useful information.

The \cd{:print} argument (which defaults to the value of \cd{*compile-print*}),
if true, causes information about top-level forms in the file being
compiled to be printed to \cdf{*standard-output*}.  Exactly what is printed
is implementation-dependent; nevertheless something will be printed.
\end{defun}

\begin{new}
X3J13 voted in March 1988
\issue{PATHNAME-STREAM}
to specify exactly which streams may be used as pathnames
(see section~\ref{PATHNAME-FUNCTIONS}).
\end{new}
\begin{newer}
X3J13 voted in June 1989 \issue{PATHNAME-WILD}
to clarify that supplying a wild pathname
as the \emph{input-pathname} argument to \cdf{compile-file} has implementation-dependent consequences;
\cdf{compile-file} might signal an error, for example,
or might compile all files that match the wild pathname.
\end{newer}

\begin{newer}
X3J13 voted in June 1989 \issue{PATHNAME-LOGICAL} to require \cdf{compile-file}
to accept logical pathnames (see section~\ref{LOGICAL-PATHNAMES-SECTION}).
\end{newer}

The \cd{:output-file} argument may be used to specify an output pathname;
it defaults in a manner
appropriate to the implementation's file system conventions.


\begin{newer}
X3J13 voted in June 1989 \issue{COMPILER-DIAGNOSTICS} to specify that
\cdf{compile-file} returns three values: the \cdf{truename} of the output
file (or \cdf{nil} if the file could not be created) and two values
indicating whether the compiler issued any diagnostics
(see section~\ref{COMPILER-DIAGNOSTICS-SECTION}).
\end{newer}

\begin{newer}
X3J13 voted in October 1988 \issue{COMPILE-FILE-PACKAGE} to specify that
\cdf{compile-file}, like \cdf{load}, rebinds \cdf{*package*} to its current value.  If
some form in the file changes the value of \cdf{*package*},
the old value will be restored when compilation is completed.
\end{newer}


\begin{newer}
X3J13 voted in June 1989 \issue{COMPILE-FILE-SYMBOL-HANDLING} to specify
restrictions on conforming programs to ensure consistent handling of symbols
and packages.

  In order to guarantee that compiled files can be loaded correctly,
  the user must ensure that the packages referenced in the file are defined
  consistently at compile and load time.  Conforming Common Lisp programs
  must satisfy the following requirements.
\begin{itemize}
\item The value of \cdf{*package*} when a top-level form in the file is processed
      by \cdf{compile-file} must be the same as the value of \cdf{*package*} when the
      code corresponding to that top-level form in the compiled file is
      executed by the loader.  In particular,
      any top-level form in a file that alters the value of \cdf{*package*}
          must change it to a package of the same name at both compile and
          load time; moreover, if the first non-atomic top-level form
          in the file is not a call to
          \cdf{in-package}, then the value of \cdf{*package*} at the time \cdf{load} is
          called must be a package with the same name as the package that
          was the value of \cdf{*package*} at the time \cdf{compile-file} was called.

\item For every symbol appearing lexically within a top-level form that
      was accessible in the package that was the value of \cdf{*package*}
      during processing of that top-level form at compile time, but
      whose home package was another package, at load time there must
      be a symbol with the same name that is accessible in both the
      load-time \cdf{*package*} and in the package with the same name as the
      compile-time home package. 
  
\item For every symbol in the compiled file that was an external symbol in
      its home package at compile time, there must be a symbol with the
      same name that is an external symbol in the package with the same name
      at load time.
\end{itemize}
  If any of these conditions do not hold, the package in which \cdf{load} looks
  for the affected symbols is unspecified.  Implementations are permitted 
  to signal an error or otherwise define this behavior.

    These requirements are merely an explicit statement of the status quo,
    namely that users cannot depend on any particular behavior if the
    package environment at load time is inconsistent with what existed
    at compile time. 
\end{newer}


\begin{newer}
X3J13 voted in March 1989 \issue{IN-SYNTAX}
to specify that \cdf{compile-file} must bind \cd{*readtable*} to its current value
at the time \cdf{compile-file} is called; the dynamic extent of the binding
should encompass all of the file-loading activity.
This allows a portable program to include forms such as
\begin{lisp}
(in-package "FOO") \\*
\\*
(eval-when (:execute :load-toplevel :compile-toplevel) \\*
~~(setq *readtable* foo:my-readtable))
\end{lisp}
without performing a net global side effect on the loading environment.
Such statements allow the remainder of such a file to be read either as
interpreted code or by \cdf{compile-file} in a syntax determined by
an alternative readtable.
\end{newer}

\begin{newer}
X3J13 voted in June 1989 \issue{LOAD-TRUENAME}
to require that \cdf{compile-file} bind two new variables
\cd{*compile-file-pathname*} and \cd{*compile-file-truename*}; the dynamic extent of the bindings
should encompass all of the file-compiling activity.
\end{newer}

\begin{defun}[Variable]
*compile-verbose*

This variable provides the default for the \cd{:verbose} argument
to \cdf{compile-file}.  Its initial value is implementation-dependent.
\end{defun}

\begin{defun}[Variable]
*compile-print*

This variable provides the default for the \cd{:print} argument
to \cdf{compile-file}.  Its initial value is implementation-dependent.
\end{defun}

\begin{defun}[Variable]
*compile-file-pathname*

X3J13 voted in June 1989 \issue{LOAD-TRUENAME} to introduce \cd{*compile-file-pathname*};
it is initially \cdf{nil} but \cdf{compile-file} binds it to a pathname that
represents the file name given as the first argument to \cdf{compile-file} merged
with the defaults (see \cdf{merge-pathname}).
\end{defun}

\begin{defun}[Variable]
*compile-file-truename*

Variable is initially \cdf{nil} but \cdf{compile-file} binds it to the ``true
name'' of the pathname of the file being compiled.  See \cdf{truename}.
\end{defun}

\begin{defspec}
  load-time-value form [read-only-p]

  This is a mechanism for delaying evaluation of a \emph{form} until it can be
  done in the run-time environment.

  If a \cdf{load-time-value} expression is seen by \cdf{compile-file}, the compiler
  performs its normal semantic processing (such as macro expansion and
  translation into machine code) on the form, but arranges for the
  execution of the \emph{form} to occur at load time in a null
  lexical environment, with the result of this evaluation then being
  treated as an immediate quantity (that is, as if originally quoted)
  at run time.  It is guaranteed that 
  the evaluation of the \emph{form} will take place only once when the file is 
  loaded, but the order of evaluation with respect to the execution
  of top-level forms in the file is unspecified.

  If a \cdf{load-time-value} expression appears within a function compiled
  with \cdf{compile}, the \emph{form} is evaluated at compile time in a null lexical
  environment.  The result of this compile-time evaluation is treated as 
  an immediate quantity in the compiled code.  

  In interpreted code, \emph{form} is evaluated (by \cdf{eval}) in a null
  lexical environment and one value is returned.  Implementations that
  implicitly compile (or partially compile) expressions passed to
  \cdf{eval} may evaluate the \emph{form} only once, at the time this
  compilation is performed.  This is intentionally similar to the
  freedom that implementations are given for the time of expanding
  macros in interpreted code.

  If the same (as determined by \cdf{eq}) list \cd{(load-time-value \emph{form})} is
  evaluated or compiled more than once, it is unspecified whether the \emph{form}
  is evaluated only once or is evaluated more than once.  This can
  happen both when an expression being evaluated or compiled shares
  substructure and when the same expression is passed to \cdf{eval} or to
  \cdf{compile} multiple times.  Since a \cdf{load-time-value} expression may be
  referenced in more than one place and may be evaluated multiple times
  by the interpreter, it is unspecified whether each execution returns
  a ``fresh'' object or returns the same object as some other execution.
  Users must use caution when destructively modifying the resulting
  object.

  If two lists \cd{(load-time-value \emph{form})} are \cdf{equal} but not \cdf{eq}, their
  values always come from distinct evaluations of \emph{form}.  Coalescing
  of these forms is not permitted.

  The optional \emph{read-only-p} argument designates whether the result
  may be considered a
  read-only constant. If \cdf{nil} (the default), the result must be considered
  ordinary, modifiable data. If \cdf{t}, the result is a read-only quantity
  that may, as appropriate, be copied into read-only space and may,
  as appropriate, be shared
  with other programs.  The \emph{read-only-p} argument is
  not evaluated and only the literal symbols \cdf{t} and \cdf{nil} are permitted.

  This new feature addresses the same set of needs as the now-defunct
  \cd{\#,} reader syntax but in a cleaner and more general manner.
  Note that \cd{\#,} syntax was reliably useful only inside quoted structure
  (though this was not explicitly mentioned in the first edition),
  whereas a \cdf{load-time-value} form must appear outside quoted structure in a
  for-evaluation position.

  See \cdf{make-load-form}.
\end{defspec}


\begin{defun}[Function]
disassemble name-or-compiled-function

The argument should be a function object, a lambda-expression, or
a symbol with a function definition.  If the relevant function is not a
compiled function, it is first compiled.  In any case, the compiled code
is then ``reverse-assembled'' and printed out in a symbolic format.  This
is primarily useful for debugging the compiler, but also often of use to
the novice who wishes to understand the workings of compiled code.

\beforenoterule
\begin{implementation}
Implementors are encouraged to make the output
readable, preferably with helpful comments.
\end{implementation}
\afternoterule

When \cdf{disassemble} compiles a function, it never
installs the resulting compiled-function object in the
\cdf{symbol-function} of a symbol.

\emph{name} may be any function-name (a symbol or a list
whose car is \cdf{setf}---see section~\ref{FUNCTION-NAME-SECTION}).
Thus one may write \cd{(disassemble '(setf cadr))} to disassemble the \cdf{setf}
expansion function for \cdf{cadr}.
\end{defun}

\begin{defun}[Function]
function-lambda-expression fn

This function allows the
source code for a defined function to be recovered.
(The committee noted that the first edition provided no
portable way to recover a lambda-expression once it had
been compiled or evaluated to produce a function.)

This function takes one argument, which must be a function, and returns
three values.

The first value is the defining lambda-expression for the
function, or {\false} if that information is not available.
The lambda-expression may have been preprocessed in some ways
but should nevertheless be of a form suitable as an argument
to the function \cdf{compile} or for use in the \cdf{function} special operator.

The second value is {\false} if the function was definitely
produced by closing
a lambda-expression in the null lexical environment; it is some
non-{\false} value if the function might have been closed in some
non-null lexical environment.

The third value is the ``name'' of the function; this is {\false} if the
name is not available or if the function had no name.
The name is intended for debugging purposes only and may be
any Lisp object (not necessarily one that would be valid for use as a name
in a \cdf{defun} or \cdf{function} special operator, for example).

\beforenoterule
\begin{implementation}
An implementation is always free to return the values
{\false}, \cdf{t}, {\false} from this function but is encouraged to
make more useful information available as appropriate.
For example, it may not be desirable for files of compiled code
to retain the source lambda-expressions for use after the file is loaded,
but it is probably desirable for
functions produced by ``in-core'' calls to \cdf{eval},
\cdf{compile}, or \cdf{defun} to retain the defining lambda-expression
for debugging purposes.  The function \cdf{function-lambda-expression}
makes this information, if retained, accessible in a standard and portable
manner.
\end{implementation}
\afternoterule
\end{defun}

\begin{defmac}
with-compilation-unit ({option-name option-value}*) {form}*

\cdf{with-compilation-unit} executes the body forms as an implicit \cdf{progn}. Within the dynamic context
   of this form, warnings deferred by the compiler until ``the end of
   compilation'' will be deferred until the end of the outermost call
   to \cdf{with-compilation-unit}. The results are the same as those of
   the last of the forms (or \cdf{nil} if there is no \emph{form}).

   Each \emph{option-name} is an unevaluated keyword; each \emph{option-value}
   is evaluated. The set of keywords permitted may be extended by the
   implementation, but the only standard option keyword is \cd{:override};
   the default value for this option is \cdf{nil}.
   If \cdf{with-compilation-unit} forms are nested dynamically, only the outermost
   such call has any effect unless the \cd{:override} value of an
   inner call is true.

  The function \cdf{compile-file} should
  provide the effect of
  \begin{lisp}
  (with-compilation-unit (:override nil) ...)
  \end{lisp}
  around its code.

  Any implementation-dependent extensions to this behavior may be provided only
  as the result of an explicit programmer request by use of 
  an implementation-dependent keyword.  It is forbidden for an implementation
  to attach additional meaning to a conforming use of this
  macro.

  Note that not all compiler warnings are deferred. In some implementations,
  it may be that none are deferred. This macro only creates an
  interface to the capability where it exists, it does not require the
  creation of the capability. An implementation that does not 
  defer any compiler warnings may correctly implement this macro
  as an expansion into a simple \cdf{progn}.
\end{defmac}

\subsection{Compiler Diagnostics}
\label{COMPILER-DIAGNOSTICS-SECTION}

\cdf{compile} and \cdf{compile-file}
may output warning messages; any such messages should
go to the stream that is the value of \cdf{*error-output*}.

First, note that
\cdf{error} and \cdf{warning} conditions may be signaled either by the compiler itself
or by code being processed by the compiler (for example, arbitrary errors may 
    occur during compile-time macro expansion or processing of \cdf{eval-when}
    forms).
Considering only those conditions signaled \emph{by the compiler} (as
    opposed to \emph{during compilation}):
\begin{itemize}

\item   Conditions of type \cdf{error} may be signaled by the compiler in
        situations where the compilation cannot proceed without
        intervention.  Examples of such situations may include errors when opening
        a file or syntax errors.

\item  Conditions of type \cdf{warning} may be signaled by the compiler in 
        situations where the standard explicitly states that a warning must,
        should, or may be signaled.  They may also be signaled
        when the compiler can determine 
        that a situation would result at runtime that would have
        undefined consequences or would cause
        an error to be signaled.
        Examples of such situations may include
            violations of type declarations,
            altering or rebinding a constant defined with \cdf{defconstant},
            calls to built-in Lisp functions with too few or too many arguments
                or with malformed keyword argument lists,
            referring to a variable declared \cdf{ignore}, or
            unrecognized declaration specifiers.

\item  The compiler is permitted to signal diagnostics about matters of
        programming style as conditions of type \cdf{style-warning}, a subtype
    of \cdf{warning}.  Although 
        a \cdf{style-warning} condition \emph{may} be signaled in these situations, no 
        implementation is \emph{required} to do so.  However, if an 
        implementation does choose to signal a condition, that condition 
        will be of type \cdf{style-warning} and will be signaled by a call to 
        the function \cdf{warn}.
        Examples of such situations may include
            redefinition of a function with an incompatible argument list,
            calls to functions (other than built-in functions)
                with too few or too many arguments
                or with malformed keyword argument lists,
            unreferenced local variables not declared \cdf{ignore}, or
            standard declaration specifiers that are ignored by 
                the particular compiler in question.
\end{itemize}

Both \cdf{compile} and \cdf{compile-file} are permitted (but not
    required) to establish a handler for conditions of type \cdf{error}.
    Such a handler
    might, for example, issue a warning and restart compilation from some
    implementation-dependent point in order to let the compilation
    proceed without manual intervention.

The functions \cdf{compile} and \cdf{compile-file} each return three values.
See the definitions of these functions for descriptions of the first value.
    The second value is \cdf{nil} if no compiler diagnostics were issued, and
    true otherwise.
    The third value is \cdf{nil} if no compiler diagnostics other than style
    warnings were issued; a non-\cdf{nil} value indicates that there were 
    ``serious'' compiler diagnostics issued or that other conditions of
    type \cdf{error} or \cdf{warning} (but not \cdf{style-warning}) were signaled during
    compilation.


\subsection{Compiled Functions}

Certain requirements are imposed on the functions produced by the compilation
process.


If a function is of type \cdf{compiled-function}, then
all macro calls appearing lexically within the function have 
        already been expanded and will not be expanded again when the
        function is called.  The process of
        compilation effectively turns every \cdf{macrolet} or \cdf{symbol-macrolet}
        construct into a \cdf{progn} (or a \cdf{locally}) with all
        instances of the local macros in the body fully expanded.

If a function is of type \cdf{compiled-function}, then
all \cdf{load-time-value} forms appearing lexically within the function have
        already been pre-evaluated and will not be evaluated
        again when the function is called.
  
Implementations are free to classify every function as 
   a \cdf{compiled-function} provided that all functions
satisfy the preceding requirements.
Conversely, it is permissible for a function that is
      not a \cdf{compiled-function} to satisfy the preceding requirements.
  
If one or more functions are defined in a file that is compiled
      with \cdf{compile-file} and the compiled file is subsequently loaded
by the function \cdf{load},
the resulting loaded function definitions must be of
    type \cdf{compiled-function}.
  
The function \cdf{compile} must produce an object of type
      \cdf{compiled-function} as the value that is either returned
or stored into the \cdf{symbol-function} of a symbol argument.

Note that none of these restrictions addresses questions of the compilation technology or
target instruction set.  For example, a compiled function does not necessarily consist of
native machine instructions.  These requirements merely specify the behavior of the type
system with respect to certain actions taken by \cdf{compile}, \cdf{compile-file}, and
\cdf{load}.

\subsection{Compilation Environment}

Following information must be available at compile time
for correct compilation
and what need not be available until run time.

The following information must be present in the compile-time
environment for a program to be compiled correctly.  This
information need not also be present in the run-time environment.
\begin{itemize}
\item In conforming code, macros referenced in the code being compiled
        must have been previously defined in the compile-time environment.
	The compiler must treat as a function call any form that is a list whose \emph{car} is
	a symbol that does not name a macro or special operator.
  (This implies that \cdf{setf} methods must also be available at
	compile time.)

\item In conforming code, proclamations for \cdf{special} variables must
        be made in the compile-time environment before any bindings of
        those variables are processed by the compiler.  The compiler
        must treat any binding of an undeclared variable as a lexical
        binding.
\end{itemize}


The compiler may incorporate the following kinds of information
into the code it produces, if the information is present in the
compile-time environment and is referenced within the code being
compiled; however, the compiler is not required to do so.
When compile-time and run-time definitions differ, it is
unspecified which will prevail within the compiled code
(unless some other behavior is explicitly specified below).  It is also
permissible for an implementation to signal an error at run time on
detecting such a discrepancy.  In all cases, the absence of the
information at compile time is not an error, but its presence may
enable the compiler to generate more efficient code.
\begin{itemize}
\item The compiler may assume that functions that are defined and
	declared \cdf{inline} in the compile-time environment will retain the
        same definitions at run time.

\item The compiler may assume that, within a named function, a
	recursive call to a function of the same name refers to the
	same function, unless that function has been declared \cdf{notinline}.
(This permits tail-recursive calls of a function to itself
to be compiled as jumps, for example, thereby turning certain recursive
schemas into efficient loops.)

\item In the absence of \cdf{notinline}
	declarations to the contrary,
 \cdf{compile-file} may assume that a call within the file being compiled to a named
	function that is defined in that file refers to that function.
	(This rule permits \emph{block compilation} of files.)  The behavior of
	the program is unspecified if functions are redefined individually 
	at run time.

\item The compiler may assume that the signature (or ``interface contract'') of
	all built-in Common Lisp functions will not change.  In addition,
	the compiler may treat all built-in Common Lisp functions as if
	they had been proclaimed \cdf{inline}.

\item The compiler may assume that the signature (or ``interface contract'') of
	functions with \cdf{ftype} information available will not change.

\item The compiler may ``wire in'' (that is, open-code or inline)
the values of symbolic constants
	that have been defined with \cdf{defconstant} in the compile-time
	environment.

\item The compiler may assume that any type definition made with \cdf{defstruct} 
        or \cdf{deftype} in the compile-time environment will retain the same 
        definition in the run-time environment.  It may also assume that
        a class defined by \cdf{defclass} in the compile-time environment will
        be defined in the run-time environment in such a way as to have
        the same superclasses and metaclass.  This implies that
        subtype/supertype relationships of type specifiers will not 
        change between compile time and run time.  (Note that it is not 
        an error for an	unknown type to appear in a declaration at
        compile time, although it is reasonable for the compiler to 
        emit a warning in such a case.)

\item The compiler may assume that if type declarations are present
	in the compile-time environment, the corresponding variables and 
	functions present in the run-time environment will actually be of
	those types.  If this assumption is violated, the run-time behavior of the program is 
	undefined.
\end{itemize}

The compiler must not make any additional assumptions about
consistency between the compile-time and run-time environments.  In 
particular, the compiler may not assume that functions that are defined
	in the compile-time environment will retain either the
	same definition or the same signature at run time, except
as described above.
Similarly,
the compiler may not signal an error if it sees a call to a
	function that is not defined at compile time, since that function
	may be provided at run time.

X3J13 voted in January 1989 \issue{COMPILE-FILE-HANDLING-OF-TOP-LEVEL-FORMS}
to specify the compile-time side effects of processing various macro forms.

Calls to defining macros such as \cdf{defmacro} or \cdf{defvar} appearing
    within a file being processed by \cdf{compile-file} normally have
    compile-time side effects that affect how subsequent forms in the
    same file are compiled.  A convenient model for explaining how these
    side effects happen is that each defining macro expands into one or
    more \cdf{eval-when} forms and that compile-time
    side effects are caused by calls occurring in the body of an
    \cd{(eval-when (:compile-toplevel) ...)} form.

The affected defining macros and their specific side effects are
    as follows.  In each case, it is identified what a user must do to
    ensure that a program is conforming, and what a compiler must do
    in order to correctly process a conforming program.

\begin{flushdesc}
\item[\cdf{deftype}]
The user must ensure that the body of a \cdf{deftype} form is
    evaluable at compile time if the type is referenced in subsequent type
    declarations.  The compiler must ensure that a type
    specifier defined by \cdf{deftype}
    is recognized in subsequent type declarations.  If the
    expansion of a type specifier is not defined fully at compile time
    (perhaps because it expands into an unknown type specifier or a
    \cdf{satisfies} of a named function that isn't defined in the compile-time
    environment), an implementation may ignore any references to this type
    in declarations and may signal a warning.

\item[\cdf{defmacro} and \cdf{define-modify-macro}]   
The compiler must store macro
    definitions at compile time, so that occurrences of the macro later on
    in the file can be expanded correctly.  The user must ensure that the
    body of the macro is evaluable at compile time if it is referenced
    within the file being compiled.

\item[\cdf{defun}]
No required compile-time side effects are associated with \cdf{defun} forms.
    In particular, \cdf{defun} does not make the function definition available
    at compile time.  An implementation may choose to store information
    about the function for the purposes of compile-time error checking
    (such as checking the number of arguments on calls) or to permit later
    \cdf{inline} expansion of the function.

\item[\cdf{defvar} and \cdf{defparameter}]
The compiler must recognize that the variables
    named by these forms have been proclaimed \cdf{special}.  However, it must
    not evaluate the \emph{initial-value} form or \cdf{set} the variable at compile
    time.

\item[\cdf{defconstant}]
The compiler must recognize that the symbol names a
    constant.  An implementation may choose to evaluate the \emph{value-form} at
    compile time, load time, or both.  Therefore the user must ensure that
    the \emph{value-form} is evaluable at compile time (regardless of whether or
    not references to the constant appear in the file) and that it always
    evaluates to the same value.  
    (There has been considerable
variance among implementations on this point.  The effect of this specification is
to legitimize all of the implementation variants by requiring care of the user.)

\item[\cdf{defsetf} and \cdf{define-setf-method}]
The compiler must make \cdf{setf} methods
    available so that they may be used to expand calls to \cdf{setf} later on in
    the file.  Users must ensure that the body of a call
    to \cdf{define-setf-method} or
    the complex form of \cdf{defsetf} is evaluable at compile time if the
    corresponding place is referred to in a subsequent \cdf{setf} in the same
    file.  The compiler must make these \cdf{setf} methods available to 
    compile-time calls to \cdf{get-setf-method} when its environment argument is
    a value received as the \cd{\&environment} parameter of a macro.
     
\item[\cdf{defstruct}]
The compiler must make the structure type name recognized
    as a valid type name in subsequent declarations (as described above
    for \cdf{deftype}) and
    make the structure slot accessors known to \cdf{setf}.
    In addition, the
    compiler must save enough information so that
    further \cdf{defstruct} definitions can include (with the \cd{:include}
    option) a structure type defined
    earlier in the file being compiled.  The functions that \cdf{defstruct}
    generates are not defined in the compile-time environment, although
    the compiler may save enough information about the functions to allow
    \cdf{inline} expansion of
    subsequent calls to these functions.  The \cd{\#S} reader syntax may or may not be 
    available for that structure type at compile time.

\item[\cdf{define-condition}]
The rules are essentially the same as those for
    \cdf{defstruct}. The compiler must make the condition type recognizable as a
    valid type name, and it must be possible to reference the condition
    type as the \emph{parent-type} of another condition type in a subsequent
    \cdf{define-condition} form in the file being compiled.

\item[\cdf{defpackage}]
 All of the actions normally performed by the \cdf{defpackage} macro at load
    time must also be performed at compile time.
\end{flushdesc}

Compile-time side effects may cause information about a
    definition to be stored in a different manner from
information about definitions
    processed either interpretively or by loading
    a compiled file.
    In particular, the information stored by a defining macro at
    compile time may or may not be available to the interpreter (either
    during or after compilation) or during subsequent calls to \cdf{compile} or
    \cdf{compile-file}.  For example, the following code is not portable because
    it assumes that the compiler stores the macro definition of \cdf{foo} where
    it is available to the interpreter.
\begin{lisp}
(defmacro foo (x) {\Xbq}(car ,x)) \\
\\
(eval-when (:execute :compile-toplevel :load-toplevel) \\*
~~(print (foo '(a b c))))~~~~~;\textrm{Wrong}
\end{lisp}
    The goal may be accomplished portably by including the macro
    definition within the \cdf{eval-when} form:
\begin{lisp}  
(eval-when (eval compile load) \\*
~~(defmacro foo (x) {\Xbq}(car ,x)) \\*
~~(print (foo '(a b c))))~~~~~;\textrm{Right}
\end{lisp}

\begin{flushdesc}
\item[\cdf{declaim}]

X3J13 voted in June 1989 \issue{PROCLAIM-ETC-IN-COMPILE-FILE}
to add a new macro \cdf{declaim} for making proclamations recognizable
at compile time.  The declaration specifiers in the \cdf{declaim} form
are effectively proclaimed at compile time so as to affect
compilation of subsequent forms.  (Note that compiler processing
of a call to \cdf{proclaim}
does not have any compile-time side effects, for \cdf{proclaim}
is a function.)
\end{flushdesc}

\begin{flushdesc}
\item[\cdf{in-package}]

X3J13 voted in March 1989 \issue{IN-PACKAGE-FUNCTIONALITY} to specify that
all of the actions normally performed by the \cdf{in-package} macro at load
time must also be performed at compile time.
\end{flushdesc}

X3J13 voted in June 1989 \issue{CLOS-MACRO-COMPILATION}
to specify the compile-time side effects of processing various CLOS-related
macro forms.  Top-level calls to the CLOS defining macros have the
 following compile-time side effects; any other compile-time behavior
 is explicitly left unspecified.

\begin{flushdesc}
\item[\cdf{defclass}]
The class name may appear in subsequent type declarations and
can be used as a specializer in subsequent \cdf{defmethod} forms.
Thus the compile-time behavior of \cdf{defclass} is similar to that of
\cdf{deftype} or \cdf{defstruct}.

\item[\cdf{defgeneric}]
The generic function can be referenced in subsequent \cdf{defmethod} forms,
but the compiler does not arrange for the generic function to be callable
    at compile time.

\item[\cdf{defmethod}]  
The compiler does not arrange for the method to be callable at compile
    time.  If there is a generic function with the same name defined at
    compile time, compiling a \cdf{defmethod} form does not add the method to that 
    generic function; the method is added to the generic
    function only when the \cdf{defmethod} form is actually executed.

    The error-signaling behavior described in the specification of
    \cdf{defmethod} in chapter~\ref{CLOS} (if the function isn't a generic function
    or if the lambda-list is not congruent) occurs only when the defining
    form is executed, not at compile time.

    The forms in \cdf{eql} parameter specializers are evaluated when the \cdf{defmethod}
    form is executed.  The compiler is permitted to build in knowledge
    about what the form in an \cdf{eql} specializer will evaluate to in cases
    where the ultimate result can be syntactically inferred without
    actually evaluating it.

\item[\cdf{define-method-combination}]
The method combination can be used in subsequent \cdf{defgeneric} forms.  

    The body of a \cdf{define-method-combination} form is evaluated no earlier 
    than when the defining macro is executed and possibly as late as 
    generic function invocation time.  The compiler may attempt to
    evaluate these forms at compile time but must not depend on being able
    to do so.
\end{flushdesc}

\subsection{Similarity of Constants}
\label{SIMILAR-AS-A-CONSTANT-SECTION}

Following paragraphs specifies what objects can be in compiled constants and
what relationship there must be between a constant
passed to the compiler and the one that is established by compiling it
and then loading its file.

The key is a definition of an equivalence relationship called
``similarity as constants''
between Lisp
objects.  Code passed through the file
compiler and then loaded must behave as though quoted constants in it
are similar in this sense to quoted constants in the corresponding source code.
An object may be used as a quoted constant processed by \cdf{compile-file}
if and only if the compiler can guarantee that the resulting constant established
by loading the compiled file is ``similar as a constant'' to the
original.  Specific requirements are spelled out below.

Some types of objects, such as streams, are not supported in constants
processed by the file compiler.  Such objects may not portably appear
as constants in code processed with \cdf{compile-file}.  Conforming
implementations are required to handle such objects either by having
the compiler or loader reconstruct an equivalent copy of the
object in some implementation-specific manner or by having the
compiler signal an error.

Of the types supported in constants, some are treated as aggregate
objects.  For these types, being similar as constants is defined
recursively.  We say that an object of such a type has certain ``basic
attributes''; to be similar as a constant to another object, the
values of the corresponding attributes of the two objects must also be
similar as constants.

A definition of this recursive form has problems with any circular or infinitely
recursive object such as a list that is an element of itself.  We use
the idea of depth-limited comparison and say that two objects are
similar as constants if they are similar at all finite levels.  This
idea is implicit in the definitions below, and it applies in all the
places where attributes of two objects are required to be similar as
constants.  The question of handling circular constants is the subject
of a separate vote by X3J13 (see below).

The following terms are used throughout this section.
  The term \emph{constant} refers to a quoted or self-evaluating constant,
  not a named constant defined by \cdf{defconstant}.
  The term \emph{source code} is used to refer to the objects constructed
  when \cdf{compile-file} calls \cdf{read} (or the equivalent) and to
  additional objects constructed by
  macro expansion during file compilation.
  The term \emph{compiled code} is used to refer to objects constructed by 
  \cdf{load}.

Two objects are \emph{similar as a constant} if and only if
they are both of one of the types listed below and satisfy the
additional requirements listed for that type.

\begin{flushdesc}
\item[\cdf{number}]

  Two numbers are similar as constants if they are of the same type
  and represent the same mathematical value.
  
\item[\cdf{character}]

  Two characters are similar as constants if they both represent
  the same character.  (The intent is that this be compatible with
  how \cdf{eql} is defined on characters.)

\item[\cdf{symbol}]
  X3J13 voted in June 1989 \issue{COMPILE-FILE-SYMBOL-HANDLING}
  to define similarity as a constant for interned symbols.
  A symbol $\emph{S}$ appearing in the source code is similar as a constant to 
  a symbol $\emph{S}'$ in the compiled code if their print names are similar as constants
   and either of the following conditions holds:
\begin{itemize}
\item  $\emph{S}$ is accessible in \cdf{*package*} at compile time and $\emph{S}'$ is accessible in
       \cdf{*package*} at load time.
\item  $\emph{S}'$ is accessible in the package that is similar as a constant to the
       home package of symbol \emph{S}.
\end{itemize}
  The ``similar as constants'' relationship for interned symbols has nothing
  to do with \cd{*readtable*} or how the function \cdf{read} would parse the 
  characters in the print name of the symbol.

  An uninterned symbol in the source code is similar as a constant
  to an uninterned symbol in the compiled code if their print names
  are similar as constants.

\item[\cdf{package}]

  A package in the source code is similar as a constant to a package in
  the compiled code if their names are similar as constants.  Note that
  the loader finds the corresponding package object as if by calling
  \cdf{find-package} with the package name as an argument.  An error is
  signaled if no package of that name exists at load time.

\item[\cdf{random-state}]

 We say that two \cdf{random-state} objects are \emph{functionally equivalent} if 
  applying \cdf{random} to them repeatedly always produces the same 
  pseudo-random numbers in the same order.  
  
  Two random-states are similar as constants if and only if copies of
  them made via \cdf{make-random-state} are functionally equivalent.
  (Note that a constant \cdf{random-state} object cannot be used as the \emph{state}
  argument to the function \cdf{random} because \cdf{random} performs
  a side effect on that argument.)

\item[\cdf{cons}]

  Two conses are similar as constants if the values of their respective
  \emph{car} and \emph{cdr} attributes are similar as constants.

\item[\cdf{array}]

  Two arrays are similar as constants if the corresponding values of each
  of the following attributes are similar as constants:
  for vectors (one-dimensional arrays), the \cdf{length} and \cdf{element-type}
  and the result of \cdf{elt} for all valid indices;
  for all other arrays, the \cdf{array-rank}, the result of \cdf{array-dimension}
  for all valid axis numbers, the \cdf{array-element-type},
  and the result of \cdf{aref} for all valid indices.  (The point of
distinguishing vectors is to take any fill pointers into account.)

  If the array in the source code is a \cdf{simple-array}, then
  the corresponding array in the compiled code must also be a
  \cdf{simple-array}, but if the array in the source code is displaced, has a
  fill pointer, or is adjustable, the corresponding array in the
  compiled code is permitted to lack any or all of these qualities.

\item[\cdf{hash-table}]

  Two hash tables are similar as constants if they meet
  three requirements.
  First, they must have the same test (for example, both are \cdf{eql} hash tables
  or both are \cdf{equal} hash tables).
  Second, there must be a unique bijective correspondence between the keys of
      the two tables, such that the corresponding keys are similar as
      constants.
  Third, for all keys, the values associated with two corresponding keys
      must be similar as constants.

  If there is more than one possible one-to-one correspondence between
  the keys of the two tables, it is unspecified whether the two
  tables are similar as constants.  A conforming
  program cannot use such a table as a constant.

\item[\cdf{pathname}]

  Two pathnames are similar as constants if all corresponding pathname
  components are similar as constants.

\item[\cdf{stream}, \cdf{readtable}, and \cdf{method}]

  Objects of these types are not supported in compiled constants.

\item[\cdf{function}]


   X3J13 voted in June 1989 \issue{CONSTANT-FUNCTION-COMPILATION}
   to specify that objects of type \cdf{function}
   are not supported in compiled constants.

\item[\cdf{structure} and \cdf{standard-object}]

   X3J13 voted in March 1989 \issue{LOAD-OBJECTS} to introduce a facility
based on the Common Lisp Object System
whereby a user can specify how \cdf{compile-file} and \cdf{load}
must cooperate to reconstruct compile-time constant objects at load time
(see \cdf{make-load-form}).
\end{flushdesc}

   X3J13 voted in March 1989 \issue{CONSTANT-COLLAPSING} to specify
the circumstances under which constants may be coalesced in compiled code.

Suppose $\emph{A}$ and $\emph{B}$ are two
objects used as quoted constants in the source code, and that $\emph{A}'$ and
$\emph{B}'$ are the corresponding objects in the compiled code.  If $\emph{A}'$ and $\emph{B}'$
are \cdf{eql} but $\emph{A}$ and $\emph{B}$ were not \cdf{eql}, then we say that $\emph{A}$ and $\emph{B}$ have been
\emph{coalesced} by the compiler.

An implementation is permitted to coalesce constants
appearing in code to be compiled if and only if they are similar as
constants, except that objects of type \cdf{symbol}, \cdf{package},
\cdf{structure}, or \cdf{standard-object} obey their own rules
and may not be coalesced by a separate mechanism.

\beforenoterule
\begin{rationale}
Objects of type \cdf{symbol} and \cdf{package} cannot be coalesced because the fact
that they are named, interned objects means they are already as
coalesced as it is useful for them to be.  Uninterned symbols could
perhaps be coalesced, but that was thought to be more dangerous than
useful.  Structures and objects could be
coalesced if a ``similar as a constant'' predicate were defined for them;
it would be a generic function.  However, at present there is no such
predicate.  Currently \cdf{make-load-form} provides a protocol by which
\cdf{compile-file} and \cdf{load} work together to construct an object in the
compiled code that is equivalent to the object in the source code;
a different mechanism would have to be added to permit coalescing.
\end{rationale}
\afternoterule

Note that coalescing is possible only because it is forbidden to
destructively modify constants \issue{CONSTANT-MODIFICATION} (see \cdf{quote}).

Objects containing circular or infinitely recursive references may legitimately
appear as constants to be compiled.  The compiler is
required to preserve \cdf{eql}-ness of substructures within a file compiled
by \cdf{compile-file}.

\section{Debugging Tools}

%\section{Отладочные средства}

The utilities described in this section are sufficiently complex
and sufficiently dependent on the host environment that their
complete definition is beyond the scope of this book.
However, they are also sufficiently
useful to warrant mention here.  It is expected that
every implementation will
provide some version of these utilities, however clever or however simple.

Коммунальные услуги, описанные в этом разделе достаточно сложны
и достаточно зависит от внешней среды, что их
полное описание выходит за рамки этой книги.
Тем не менее, они также достаточно
полезно, чтобы оправдать упомянуть здесь. Ожидается, что
каждая реализация
предоставить некоторые версии этих программ, однако умный или же просто.

Описанные в этом разделе утилиты достаточно сложны и зависят от внешней среды
ОС, что их полное описание выходит за рамки книги. Тем не менее их описание
будет полезным. Предполагается, что каждая реализация будет представлять
некоторую версию этих утилит.

\begin{defmac}
trace {\,function-name}* \\
untrace {\,function-name}*

Invoking \cdf{trace} with one or more function-names (symbols or lists, whose
\emph{car} is \cdf{setf}---see section~\ref{FUNCTION-NAME-SECTION}),
 causes
the functions named to be traced.  Henceforth, whenever such
a function is invoked, information about the call, the arguments
passed, and the eventually returned values, if any, will be printed
to the stream that is the value of \cdf{*trace-output*}.
For example:
\begin{lisp}
(trace fft gcd string-upcase)
\end{lisp}
If a function call is open-coded (possibly as a result of an \cdf{inline}
declaration), then such a call may not produce trace output.

Invoking \cdf{untrace} with one or more function names will cause those
functions not to be traced any more.

Tracing an already traced function, or untracing a function not
currently being traced, should produce no harmful effects but may
produce a warning message.

Calling \cdf{trace} with no argument forms will return a list of functions
currently being traced.

Calling \cdf{untrace} with no argument forms will cause all currently
traced functions to be no longer traced.

The values returned by \cdf{trace} and \cdf{untrace} when
given argument forms are implementation-dependent.

\cdf{trace} and \cdf{untrace} may also accept additional
implementation-dependent argument formats.  The format of the trace
output is implementation-dependent.
\end{defmac}

\begin{defmac}
step form

This evaluates \emph{form} and returns what \emph{form} returns.
However, the user is allowed to interactively
``single-step'' through the evaluation of \emph{form}, at least
through those evaluation steps that are performed interpretively.
The nature of the interaction is implementation-dependent.
However, implementations are encouraged to respond to the typing
of the character \cd{?} by providing help, including a list
of commands.

\cdf{step} evaluates its argument \emph{form}
in the current lexical environment (not simply a null environment),
and that calls to \cdf{step} may be compiled, in which case
an implementation may step through only those parts of the
evaluation that are interpreted.  (In other words, the \emph{form}
itself is unlikely to be stepped, but if executing it happens to
invoke interpreted code, then that code may be stepped.)
\end{defmac}

\begin{defmac}
time form

This evaluates \emph{form} and returns what \emph{form} returns.  However, as
a side effect, various timing data and other information are printed to
the stream that is the value of \cdf{*trace-output*}.  The nature and
format of the printed information is implementation-dependent.  However,
implementations are encouraged to provide such information as elapsed
real time, machine run time, storage management statistics, and so on.

\cdf{time} evaluates its argument \emph{form}
in the current lexical environment (not simply a null environment),
and that calls to \cdf{time} may be compiled.
\end{defmac}

\begin{defun}[Function]
describe object &optional stream

\cdf{describe} prints, to the stream information about the \emph{object}.
Sometimes it will describe something that it finds inside something else;
such recursive descriptions are indented appropriately.  For instance,
\cdf{describe} of a symbol will exhibit the symbol's value,
its definition, and each of its properties.  \cdf{describe} of a
floating-point number will exhibit its internal representation in a way
that is useful for tracking down round-off errors and the like.
The nature and format of the output is implementation-dependent.

\cdf{describe} returns no values (that is, it returns what the expression
\cd{(values)} returns: zero values).

The output is sent to the specified \emph{stream}, which
 defaults to the value of \cdf{*standard-output*};
 the \emph{stream} may also be \cdf{nil} (meaning \cdf{*standard-output*})
 or \cdf{t} (meaning \cdf{*terminal-io*}).

The behavior of \cdf{describe} depends on the generic function
\cdf{describe-object} (see below).
\end{defun}

That \cdf{describe} is forbidden
to prompt for or require user input when given exactly one argument;
It is permitted implementations
to extend \cdf{describe} to accept keyword arguments that may cause
it to prompt for or to require user input.

\begin{defun}[Generic function][Primary method]
describe-object object stream \\
describe-object (object standard-object) stream

The generic function \cdf{describe-object} writes a description of an object to a
  stream.  The function \cdf{describe-object} is called by the \cdf{describe} function; it
  should not be called by the user.

  Each implementation must provide a method on the class
  \cdf{standard-object} and methods on enough other classes to ensure that
  there is always an applicable method.  Implementations are free to add
  methods for other classes.  Users can write methods for \cdf{describe-object} for
  their own classes if they do not wish to inherit an implementation-supplied
  method.

   The first argument may be any Lisp object.  The second argument is a stream; it
   cannot be \cdf{t} or \cdf{nil}.
   The values returned by \cdf{describe-object} are unspecified.

   Methods on \cdf{describe-object} may recursively call \cdf{describe}.  Indentation,
   depth limits, and circularity detection are all taken care of automatically,
   provided that each method handles exactly one level of structure and calls
   \cdf{describe} recursively if there are more structural levels.
   If this rule is not obeyed, the results are undefined.

   In some implementations the \emph{stream} argument passed to a \cdf{describe-object}
   method is not the original stream but is an intermediate stream that
   implements parts of \cdf{describe}.  Methods should therefore not depend on the
   identity of this stream.

\beforenoterule
\begin{rationale}
 This proposal was closely modeled on the CLOS description of \cdf{print-object},
 which was well thought out and provides a great deal of functionality and
 implementation freedom.  Implementation techniques for
 \cdf{print-object} are applicable to \cdf{describe-object}.

 The reason for making the return values for \cdf{describe-object} unspecified is to
 avoid forcing users to write \cd{(values)} explicitly in all their methods;
 \cdf{describe} should take care of that.
\end{rationale}
\afternoterule
\end{defun}

\begin{defun}[Function]
inspect object

\cdf{inspect} is an interactive version of \cdf{describe}.
The nature of the interaction is implementation-dependent,
but the purpose of \cdf{inspect} is to make it easy to wander
through a data structure, examining and modifying parts of it.
Implementations are encouraged to respond to the typing
of the character \cd{?} by providing help, including a list
of commands.

The values returned by \cdf{inspect}
are implementation-dependent.
\end{defun}

\begin{defun}[Function]
room &optional x

\cdf{room} prints, to the stream in the variable \cdf{*standard-output*},
information about the state of internal storage and its management.  This
might include descriptions of the amount of memory in use and the degree
of memory compaction, possibly broken down by internal data type if that
is appropriate.  The nature and format of the printed information is
implementation-dependent.  The intent is to provide information that may
help a user to tune a program to a particular implementation.

\cd{(room nil)} prints out a minimal amount of information.
\cd{(room t)} prints out a maximal amount of information.
Simply \cd{(room)} prints out an intermediate amount
of information that is likely to be useful.

The argument \emph{x} may also be the keyword \cd{:default},
which has the same effect as passing no argument at all.
\end{defun}

\begin{defun}[Function]
ed &optional x

If the implementation provides a resident editor, this function
should invoke it.

\cd{(ed)} or \cd{(ed nil)} simply enters the editor, leaving you in the same
state as the last time you were in the editor.

\cd{(ed \emph{pathname})} edits the contents of the file specified
by \emph{pathname}.  The \emph{pathname} may be an actual pathname
or a string.

\cdf{ed} accepts logical pathnames (see
section~\ref{LOGICAL-PATHNAMES-SECTION}).

\cd{(ed \emph{symbol})} tries to let you edit the text for the function
named \emph{symbol}.  The means by which the function text is obtained
is implementation-dependent; it might involve searching the file system,
or pretty printing resident interpreted code, for example.

Function name may be any function-name (a symbol or a list
whose \emph{car} is \cdf{setf}---see section~\ref{FUNCTION-NAME-SECTION}).
Thus one may write \cd{(ed '(setf cadr))} to edit the \cdf{setf}
expansion function for \cdf{cadr}.
\end{defun}


\begin{defun}[Function]
dribble &optional pathname

\cd{(dribble \emph{pathname})} may rebind \cdf{*standard-input*}
and \cdf{*standard-output*}, and may take other appropriate
action, so as to send a record of the
input/output interaction to a file named by \emph{pathname}.
The primary purpose of this is to create a readable record of an interactive
session.

\cd{(dribble)} terminates the recording of input and output and
closes the dribble file.

\cdf{dribble} also accepts logical pathnames (see
section~\ref{LOGICAL-PATHNAMES-SECTION}).

\cdf{dribble} is intended primarily
for interactive debugging and that its effect cannot be
relied upon for use in portable
programs.

Different implementations of Common Lisp have used radically different
techniques for implementing \cdf{dribble}.  All are reasonable interpretations
of the original specification, and all behave in approximately the same
way if \cdf{dribble} is called only from the interactive top level.
However, they may have quite different behaviors if \cdf{dribble} is
called from within compound forms.

Consider two models of the operation of \cdf{dribble}.  In the ``redirecting''
model, a call to \cdf{dribble} with a pathname argument
alters certain global variables such as \cdf{*standard-output*},
perhaps by constructing a broadcast stream directed to both the original
value of \cdf{*standard-output*} and to the dribble file; other streams
may be affected as well.  A call to \cdf{dribble} with no arguments
undoes these side effects.

In the ``recursive'' model, by contrast, a call to \cdf{dribble} with a
pathname argument creates a new interactive command loop and calls it
recursively.  This new command loop is just like an ordinary
read-eval-print loop except that it also echoes the interaction to
the dribble file.  A call to \cdf{dribble} with no arguments
does a \cdf{throw} that exits the recursive command loop and returns
to the original caller of \cdf{dribble} with an argument.

The two models may be distinguished by this test case:
\begin{lisp}
(progn (dribble "basketball") \\
~~~~~~~(print "Larry") \\
~~~~~~~(dribble) \\
~~~~~~~(princ "Bird"))
\end{lisp}
If this form is input to the Lisp top level, in either model
a newline (provided by the function \cdf{print}) and the words
\cd{Larry Bird} will be printed to the standard output.
The redirecting dribble model will additionally print all but the word
\cdf{Bird} to a file named \cdf{basketball}.

By contrast, the recursive dribble model will enter a recursive command
loop and not print anything until \cd{(dribble)} is executed from within
the new interactive command loop.  At that time the file named
\cdf{basketball} will be closed, and then execution of the
\cdf{progn} form will be resumed.  A newline and ``\cd{Larry~}'' (note the trailing space)
will be printed to the standard output, and then the call
\cd{(dribble)} may complain that there is no active dribble file.
Once this error is resolved, the word \cdf{Bird} may be printed
to the standard output.

Here is a slightly different test case:
\begin{lisp}
(dribble "baby-food")
\end{lisp}
\begin{lisp}
(progn (print "Mashed banana") \\*
~~~~~~~(dribble) \\*
~~~~~~~(princ "and cream of rice"))
\end{lisp}
If this form is input to the Lisp top level, in the redirecting model
a newline and the words
\cd{Mashed banana and cream of rice} will be printed to the standard output
and all but the words
\cd{and cream of rice} will be sent to a file named \cdf{baby-food}.

The recursive model will direct exactly the same output to the file
named \cdf{baby-food} but will never print the words
\cd{and cream of rice} to the standard output because the call
\cd{(dribble)} does not return normally; it throws.

The redirecting model may be intuitively more appealing to some.
The recursive model, however, may be more robust; it carefully limits
the extent of the dribble operation and disables dribbling if a
throw of any kind occurs.  The vote by X3J13 was an explicit decision
not to decide which model to use.  Users are advised to call \cdf{dribble}
only interactively, at top level.
\end{defun}


\begin{defun}[Function]
apropos string &optional package \\
apropos-list string &optional package

\cd{(apropos \emph{string})} tries to find all available symbols whose print names
contain \emph{string} as a substring.  (A symbol may be supplied for
the \emph{string}, in which case the print name of the symbol is used.)
Whenever \cdf{apropos} finds a symbol, it prints
out the symbol's name; in addition,
information about the function definition and dynamic value of the symbol,
if any, is printed.
If \emph{package} is specified and not {\nil}, then only symbols
available in that package are examined;
otherwise ``all'' packages are searched, as if by \cdf{do-all-symbols}.
Because a symbol may be available by way of more than one inheritance
path, \cdf{apropos} may print information about the same symbol more than once.
The information is printed to the stream that is the value
of \cdf{*standard-output*}.
\cdf{apropos} returns no values (that is, it returns what the expression
\cd{(values)} returns: zero values).

\cdf{apropos-list} performs the same search that \cdf{apropos} does but
prints nothing.  It returns a list of the symbols whose print names
contain \emph{string} as a substring.
\end{defun}

\section{Environment Inquiries}

% \section{Справка о среде}

Environment inquiry functions provide information about the
environment in which a Common Lisp program is being executed.
They are described here in two categories: first, those dealing with
determination and measurement of time, and second, all the others,
most of which deal with identification of the computer hardware
and software.

Справочные функции представляют информацию о среде, в которой исполняется Common
Lisp'овая программа.
Функции разделены на две категории: первые для работы со временем, и остальные
для получения имен, версий, типов программ и оборудования.

\subsection{Time Functions}
\label{TIME-SECTION}

Time is represented in three different ways in Common Lisp:
Decoded Time, Universal Time, and Internal Time.
The first two representations
are used primarily to represent calendar time and are
precise only to one second.
Internal Time is used primarily to represent measurements of computer
time (such as run time) and is precise to some implementation-dependent
fraction of a second, as specified by \cdf{internal-time-units-per-second}.
Decoded Time format is used only for absolute time indications.
Universal Time and Internal Time formats are used for both absolute
and relative times.

Decoded Time format represents calendar time as a number of components:
\begin{itemize}
\item
\emph{Second}: an integer between 0 and 59, inclusive.

\item
\emph{Minute}: an integer between 0 and 59, inclusive.

\item
\emph{Hour}: an integer between 0 and 23, inclusive.

\item
\emph{Date}: an integer between 1 and 31, inclusive (the upper limit actually
depends on the month and year, of course).

\item
\emph{Month}: an integer between 1 and 12, inclusive; 1 means January,
12 means December.

\item
\emph{Year}: an integer indicating the year A.D.  However, if this integer
is between 0 and 99, the ``obvious'' year is used; more precisely,
that year is assumed that is equal to the integer modulo 100 and
within fifty years of the current year (inclusive backwards
and exclusive forwards).  Thus, in the year 1978, year 28 is 1928
but year 27 is 2027.  (Functions that return time in this format always return
a full year number.)
\end{itemize}

\begin{itemize}
\item
\emph{Day-of-week}: an integer between 0 and 6, inclusive;
0 means Monday, 1 means Tuesday, and so on; 6 means Sunday.

\item
\emph{Daylight-saving-time-p}: a flag that, if not {\false}, indicates that
daylight saving time is in effect.

\item
\emph{Time-zone}: an integer specified as the number of hours west of GMT
(Greenwich Mean Time).  For example, in Massachusetts the time zone is 5,
and in California it is 8.  Any adjustment for daylight saving time is
separate from this.
\end{itemize}

Time zone part of Decoded Time need not be an integer,
but may be any rational number (either an integer or a ratio)
in the range -24 to 24 (inclusive on both ends)
that is an integral multiple of \cd{1/3600}.

\beforenoterule
\begin{rationale}
For all possible time designations to be accommodated, it is
    necessary to allow the time zone to be non-integral, for some places
    in the world have time standards offset from Greenwich Mean Time
    by a non-integral number of hours.

    There appears to be no user demand for floating-point time zones.  Since such
    zones would introduce inexact arithmetic, X3J13 did not consider
    adding them at this time.

This specification does require time zones to be represented as integral multiples
    of 1 second (rather than 1 hour).  This prevents problems that could otherwise
occur in converting Decoded Time to Universal Time.
\end{rationale}
\afternoterule

Universal Time represents time as a single non-negative integer.
For relative time
purposes, this is a number of seconds.  For absolute time, this is the
number of seconds since midnight, January 1, 1900 {GMT}.  Thus the time 1
is 00:00:01 (that is, 12:00:01 A.M.) on January 1, 1900 {GMT}.
Similarly, the time 2398291201 corresponds to time 00:00:01 on January 1,
1976 {GMT}.
Recall that the year 1900 was \emph{not} a leap year; for the purposes of
Common Lisp, a year is a leap year if and only if its number is divisible by 4, except
that years divisible by 100 are \emph{not} leap years, except that years
divisible by 400 \emph{are} leap years.  Therefore the year 2000 will
be a leap year.
(Note that the ``leap seconds'' that
are sporadically inserted by the world's official timekeepers as an additional
correction are ignored; Common Lisp assumes that every day is exactly 86400
seconds long.)
Universal Time format is used as a standard time
representation within the {ARPANET}; see reference \cite{KLH-TIME-SERVER}.
Because the Common Lisp Universal Time representation uses only
non-negative integers, times before the base time of midnight,
January 1, 1900 {GMT} cannot be processed by Common Lisp.

Internal Time also represents time as a single integer, but
in terms of an implementation-dependent unit.
Relative time is measured as a number of these units.
Absolute time is relative to an arbitrary time base, typically
the time at which the system began running.

\begin{defun}[Function]
get-decoded-time 

The current time is returned in Decoded Time format.  Nine values
are returned: \emph{second}, \emph{minute}, \emph{hour}, \emph{date}, \emph{month},
\emph{year}, \emph{day-of-week}, \emph{daylight-saving-time-p}, and \emph{time-zone}.
\end{defun}

\begin{defun}[Function]
get-universal-time 

The current time of day is returned as a single integer
in Universal Time format.

Функция возвращает текущее время всемирное время в 
\end{defun}

\begin{defun}[Function]
decode-universal-time universal-time &optional time-zone

The time specified by \emph{universal-time} in Universal Time format
is converted to Decoded Time format.  Nine values
are returned: \emph{second}, \emph{minute}, \emph{hour}, \emph{date}, \emph{month},
\emph{year}, \emph{day-of-week}, \emph{daylight-saving-time-p}, and \emph{time-zone}.

The \emph{time-zone} argument defaults to the current time zone.

\cdf{decode-universal-time},
like \cdf{encode-universal-time}, ignores daylight saving time information
if a \emph{time-zone} is explicitly specified; in this case
the returned \emph{daylight-saving-time-p} value will necessarily be
\cdf{nil} even if daylight saving time happens to be in effect in that
time zone at the specified time.
\end{defun}

\begin{defun}[Function]
encode-universal-time second minute hour date month year &optional time-zone

The time specified by the given components of Decoded Time format is
encoded into Universal Time format and returned.  If you do not specify
\emph{time-zone}, it defaults to the current time zone adjusted for daylight
saving time.  If you provide \emph{time-zone} explicitly, no adjustment for
daylight saving time is performed.

% Функция преобразует время, заданное компонентами формата декодированного
% времени, в формат всемирного времени. Если вы не укажете часовой пояс
% \emph{time-zone}, то он будет по-умолчанию равен текущему часовому поясу с
% учетом перехода на летнее время. Если вы укажете явно часовой пояс
% \emph{time-zon}, учет летнего времени производиться не будет.
\end{defun}

\begin{defun}[Constant]
internal-time-units-per-second

This value is an integer, the implementation-dependent
number of internal time units in a second.  (The internal time unit must
be chosen so that one second is an integral multiple of it.)

% Значение константы является целым числом и зависит от того, сколько единиц
% внутреннего времени в секунде для данной реализации Common Lisp'а.
% (Единица внутреннего времени должна быть выбрана так, чтобы в секунде помещалось
% целое число таких единиц.)

\beforenoterule
\begin{rationale}
The reason for allowing the internal time
units to be implementation-dependent is so that
\cdf{get-internal-run-time} and \cdf{get-internal-real-time}
can execute with minimum overhead.
The idea is that it should be very likely that a fixnum will suffice as the
returned value from these functions.  This probability can be
tuned to the implementation by trading off the speed of the machine
against the word size.  Any particular unit will
be inappropriate for some implementations: a microsecond is too long
for a very fast machine, while a much smaller unit would
force many implementations to return bignums for most calls
to \cdf{get-internal-time}, rendering that function less useful for accurate
timing measurements.
\end{rationale}
\afternoterule
\end{defun}

\begin{defun}[Function]
get-internal-run-time 

The current run time is returned as a single integer in Internal Time
format.
The precise meaning of this quantity is implementation-dependent;
it may measure real time, run time, CPU cycles, or some other quantity.
The intent is that the difference between the values of two calls
to this function be the amount of time between the two calls
during which computational effort was expended on behalf of the
executing program.

% Функция возвращает целое число в формате Внутреннего Времени. 
% Точное значение этой величины зависит от реализации и может измерятся в реальном
% времени, времени выполнения, циклов центрального процессора, или в чем-либо
% другом.
% Суть в том, что разница между значениями двух вызовов будет количеством времени,
% потраченным на исполнение программы между этими вызовами.
\end{defun}

\begin{defun}[Function]
get-internal-real-time 

The current time is returned as a single integer in Internal Time
format.  This time is relative to an arbitrary time base,
but the difference between the values of two calls
to this function will be the amount of elapsed real time between the two calls,
measured in the units defined by \cdf{internal-time-units-per-second}.

% Функция возвращает целое число в формате Внутреннего Времени. Данное время
% относительно некоторого базового значения, но разница между значениями двух
% вызовов этой функции будет количеством прошедшего (между этими вызовами)
% реального времени, выраженным в \cdf{internal-time-units-per-second}.
\end{defun}

\begin{defun}[Function]
sleep seconds

\cd{(sleep \emph{n})} causes execution to cease and become dormant for
approximately \emph{n} seconds of real time, whereupon execution is resumed.
The argument may be any non-negative non-complex number.
\cdf{sleep} returns {\nil}.

% \cd{(sleep \emph{n})} приостанавливает исполнение примерно на \emph{n} секунд в
% реальном времени.
% Аргумент может быть любым неотрицательным некомплексным числом.
% \cdf{sleep} возвращает {\nil}.
\end{defun}

\subsection{Other Environment Inquiries Справочные функции о среде}

For any of the following functions, if no appropriate
and relevant result can be produced, {\nil} is returned instead
of a string.

% Любая из этих функций вместо строки может возвращать результат {\nil}, если
% подходящей информации нет.

\beforenoterule
\begin{rationale}
These inquiry facilities are functions rather than variables
against the possibility that a Common Lisp process might migrate from
machine to machine.  This need not happen in a distributed
environment; consider, for example, dumping a core image file
containing a compiler and then shipping it to another site.
\end{rationale}
\afternoterule

% \beforenoterule
% \begin{rationale}
% Эти справочные данные возвращаются функциями, а не хранятся переменных, так как
% процесс Common Lisp'а может мигрировать между компьютерами (машинами).
% Это необязтально случается в распределенной среде.
% Например может служить сохранение образа содержащего компилятор и затем
% восстановление данного образа на другой машине. 
% \end{rationale}
% \afternoterule

\begin{defun}[Function]
lisp-implementation-type 

A string is returned that identifies the generic name of
the particular Common Lisp implementation.
Examples: \cd{"Spice LISP"}, \cd{"Zetalisp"}, \cd{"SBCL"}.

% Функция возвращают имя текущей реализации Common Lisp'а.
% Примеры: \cd{"Spice LISP"}, \cd{"Zetalisp"}, \cd{"SBCL"}.
\end{defun}

\begin{defun}[Function]
lisp-implementation-version 

A string is returned that identifies the version of
the particular Common Lisp implementation; this information
should be of use to maintainers of the implementation.
Examples: \cd{"1192"}, \cd{"53.7 with complex numbers"},
\cd{"1746.9A, NEWIO 53, ETHER 5.3"}.

% Функция возвращает версию текущей реализации Common Lisp'а.
% Эта информация должно быть использована сопровождающими реализацию.
% Примеры: \cd{"1192"}, \cd{"53.7 with complex numbers"},
% \cd{"1746.9A, NEWIO 53, ETHER 5.3"}.
\end{defun}

\begin{defun}[Function]
machine-type 

A string is returned that identifies the generic name of
the computer hardware on which Common Lisp is running.
Examples: \cd{"IMLAC"}, \cd{"DEC PDP-10"}, \cd{"DEC VAX-11/780"}, \cd{"X86-64"}.

% Функция возвращает имя типа аппаратного обеспечения, на котором запущен Common
% Lisp.
% Примеры: \cd{"IMLAC"}, \cd{"DEC PDP-10"}, \cd{"DEC VAX-11/780"}, \cd{"X86-64"}.
\end{defun}

\begin{defun}[Function]
machine-version 

A string is returned that identifies the version of
the computer hardware on which Common Lisp is running.
Example: \cd{"KL10, microcode 9"}, \cd{"AMD Athlon(tm) 64 X2 Dual Core Processor 3600+"}.

% Функция возвращает версию аппаратного обеспечения, на котором запущен Common
% Lisp.
% Примеры: \cd{"KL10, microcode 9"}, \cd{"AMD Athlon(tm) 64 X2 Dual Core Processor 3600+"}.
\end{defun}

\begin{defun}[Function]
machine-instance 

A string is returned that identifies the particular
instance of the computer hardware on which Common Lisp is running;
this might be a local nickname, for example, or a serial number.
Examples: \cd{"MIT-MC"}, \cd{"CMU GP-VAX"}.

% Функция возвращает имя компьютера, на котором запущена реализация Common Lisp'а.
% Примеры: \cd{"MIT-MC"}, \cd{"CMU GP-VAX"}.
\end{defun}

\begin{defun}[Function]
software-type 

A string is returned that identifies the generic name of
any relevant supporting software.
Examples: \cd{"Spice"}, \cd{"TOPS-20"}, \cd{"ITS"}, \cd{Linux}.

% Функция возвращает имя типа текущей операционной системы.
% Примеры: \cd{"Spice"}, \cd{"TOPS-20"}, \cd{"ITS"}, \cd{Linux}.
\end{defun}

\begin{defun}[Function]
software-version 

A string is returned that identifies the version of
any relevant supporting software; this information
should be of use to maintainer of the implementation.

% Функция возвращает версию текущей операционной системы. Данная информация должно
% использоваться сопровождающими ОС.
% Пример для ArchLinux'а: \cd{"3.2.13-1-ARCH"}.
\end{defun}

\begin{defun}[Function]
short-site-name  \\
long-site-name 

A string is returned that identifies the physical location
of the computer hardware.
Examples of short names: \cd{"MIT AI Lab"}, \cd{"CMU-CSD"}.
Examples of long names:
\begin{lisp}
"MIT Artificial Intelligence Laboratory" \\
"Massachusetts Institute of Technology \\
Artificial Intelligence Laboratory" \\
"Carnegie-Mellon University Computer Science Department"
\end{lisp}

% Функции возвращают строки, обозначающие физическое расположение аппаратной части
% компьютера.
% Примеры для кратких имен: \cd{"MIT AI Lab"}, \cd{"CMU-CSD"}.
% Примеры для длинных имен:
% \begin{lisp}
% "MIT Artificial Intelligence Laboratory" \\
% "Massachusetts Institute of Technology \\
% Artificial Intelligence Laboratory" \\
% "Carnegie-Mellon University Computer Science Department"
% \end{lisp}
\end{defun}

\noindent
See also \cdf{user-homedir-pathname}.

% \noindent
% Смотрите также \cd{user-homedir-pathname}.


\begin{defun}[Variable]
*features*

The value of the variable \cdf{*features*} should be a list of symbols
that name ``features'' provided by the implementation.
Most such names will be implementation-specific; typically
a name for the implementation will be included.

% Значение переменной \cdf{*features} должно быть списком символов, которые
% указывают на имена <<возможностей>> данной реализации.
% Большинство этих имен специфичны. 

The value of this variable is used by the \cd{\#+} and \cd{\#-}
reader syntax.

% Значение этой переменной используется с помощью синтаксических конструкций
% считывателя: \cd{\#+} и \cd{\#-}.

Feature names used with \cd{\#+} and \cd{\#-}
are read in the \cdf{keyword} package unless an explicit prefix
designating some other package appears.  The standard
feature name \cdf{ieee-floating-point} is therefore actually the
keyword \cd{:ieee-floating-point}, though one need not write the colon
when using it with \cd{\#+} or \cd{\#-}; thus \cd{\#+ieee-floating-point}
and \cd{\#+:ieee-floating-point} mean the same thing.

% По-умолчанию символы для имен <<возможностей>>, использованные в конструкциях
% \cd{\#+} и \cd{\#-}, ищутся в пакете
% \cdf{keyword}. Таким образом \cd{\#+ieee-floating-point}
% и \cd{\#+:ieee-floating-point} означают одно и то же.
\end{defun}


\section{Identity Function}

%\section{Функция идентичности (identity)}

This function is occasionally useful as an argument to
other functions that require functions as arguments.  (Got that?)

% Эта функция иногда бывает полезна для использования в качестве аргумента других
% функций, которые требуют функции в качестве аргументов. (Понятно?)

\begin{defun}[Function]
identity object

The \emph{object} is returned as the value of \cdf{identity}.

The \cdf{identity} function is the default value for the \cd{:key}
argument to many sequence functions (see chapter~\ref{KSEQUE}).

Table~\ref{IDENTITY-PLOT} illustrates the behavior in the complex plane
of the
\cdf{identity} function regarded as a function of a complex numerical argument.

Many other constructs in Common Lisp have the behavior of \cdf{identity}
when given a single argument.  For example, one might well use \cdf{values}
in place of \cdf{identity}.  However, writing \cdf{values} of a single
argument conventionally indicates that the argument form might deliver
multiple values and that the intent is to pass on only the first of
those values.

% Результатом функции является переданный объект \emph{object}.

% Функция \cdf{identity} используется по-умолчанию для аргумента \cd{:key} для
% большинства функций для последовательностей (смотрите главу~\ref{KSEQUE}).

% Поведение функции \cdf{identity} для комплексного числа проиллюстрировано в
% таблице~\ref{IDENTITY-PLOT}.

% Множество других Common Lisp'овых конструкций с одним аргументом ведут себя так
% же как и \emph{identity}. Например, можно использовать \cdf{values} вместо
% \cdf{identity}. Однако, запись \cdf{values} с одним аргументом означает, что
% форма аргумента возвращает несколько значений, но необходимо вернуть только одно
% из них.
\end{defun}

\begin{defun}[Function]
constantly object

Returns a function that will always return the \emph{object}. The returned function
takes any number of arguments.
\end{defun}

% \begin{defun}[Функция]
% constantly object

% Функция возвращает другую функцию, которая всегда возвращает объект
% \emph{object}. Возвращенная функция принимает любое количество аргументов.
% \end{defun}

\begin{defmac}
lambda lambda-list [[ {declaration}* | [doc-string]] {form}*

A dubious shortcut for \cd{(function (lambda ...))} or \cd{\#'(lambda ...)}.

% Двусмысленное сокращение для \cd{(function (lambda ...))} или \cd{\#'(lambda
%   ...)}.
\end{defmac}
