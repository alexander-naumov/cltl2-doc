\documentclass[12pt]{book}

\def\rulang{}

% !!! Эту комбинацию не менять
\usepackage[T2A]{fontenc} % включаем русские шрифты
\usepackage[utf8]{inputenc} % включаем поддержку UTF8
\usepackage[russian,english]{babel} % включаем пакет для поддержки русского языка

\ifx \HCode\Undef
  \input commands_html.tex
\else
  \input commands_html.tex

\ifx \rulang\Undef
  \title{Common Lisp The Language Second Edition}
\else %RUSSIAN
  \title{Язык Common Lisp. Второе издание.}
\fi

\fi

% !!!
%%% Root file for Common Lisp Manual.  Copyright 1984, 1987 Guy L. Steele Jr.

\sloppy
\def\pagestatus{\error}

\ifchiron
  \let\tagline\relax
\else
  \def\tagline{\ifcase\month\or
    January\or February\or March\or April\or May\or June\or
    July\or August\or September\or October\or November\or December\fi~\number\day,~\number\year
    ---\pagestatus---Common Lisp: The Language 2/E---Digital Press---\copyright~\number\year
    ~Guy L. Steele Jr. All rights reserved.}
\fi

\def\tagline{}

\pagestyle{headings}

\usepackage{multind}

%% See issues.tex for the \printindex calls
%% terms are stored using \indexterm; most of the rest use the defun
%% environment.

\makeindex{symbols}
\makeindex{issues}

%\ifx \HCode\Undef
% not tex4ht ...
%\typeout{Not tex4ht}
%  \usepackage[pdftex]{color,graphicx}
%  \usepackage[pdftex]{hyperref}
%\else
%... tex4ht ...
   \usepackage{color,graphicx}
   \usepackage{hyperref}
%\fi

\ifx \rulang\Undef

%RUSSIAN
\else
\AtBeginDocument{\renewcommand\contentsname{Содержание}
  \renewcommand{\chaptername}{Глава}
  \renewcommand{\appendixname}{Приложение}
  \renewcommand{\bibname}{Библиография}
  \renewcommand{\tablename}{Таблица}
  \renewcommand{\figurename}{Изображение}
}
\fi

\begin{document}

\input Title-page.tex

\pagenumbering{roman}

%Part{Char, Root = "CLM.MSS"}
%%% Chapter of Common Lisp Manual.  Copyright 1984, 1988, 1989 Guy L. Steele Jr.

\input Title-page.tex

\pagenumbering{roman}

\def\pagestatus{ROUGH PAGES}

\begingroup
\makeatletter
\clearpage \global\c@page 5  % Start toc on page v

\def\numberline#1{\setbox0=\hbox{#1 }\ifdim \wd0 < \@tempdima
    \hbox to \@tempdima{\box0\hfil}\else \box0 \fi}
\makeatother
\tableofcontents
\endgroup

\cleardoublepage

\chapter*{\vtop{\LARGE \sf \noindent Preface Пролог \relax\\[5pt]
                \normalsize\sf SECOND EDITION ВТОРОЕ ИЗДАНИЕ}}


Common Lisp has succeeded.  Since publication of the first edition of
this book in 1984, many implementors have used it as a {\it de facto}
standard for Lisp implementation.  As a result, it is now much easier
to port large Lisp programs from one implementation to another.
Common Lisp has proved to be a useful and stable platform for
rapid prototyping and
systems delivery in artificial intelligence and other areas.
With experience gained in using Common Lisp for so many
applications, implementors found no shortage of opportunities for
innovation.
One of the important
characteristics of Lisp is its good support for experimental extension
of the language; while Common Lisp has been stable, it has not stagnated.

Common Lisp к успеху пришел. С момента публикации первой редакции
данной книги в 1984, много организаций использовали его как {\it
де-факто} стандарт для реализации Lisp'а. В результате сейчас гораздо
проще портировать большую Lisp программу с одной реализации на
другую. Common Lisp доказал свою полезность и стабильность, как
платформы для быстрого прототипирования и быстрой поставки систем в
области искусственного интеллекта и не только в ней. С приобретенным
опытом использования Common Lisp'а для такого большого количества
приложений, организации не нашли недостатков в возможностях для
инноваций. Одна из важных характеристик Lisp'а это его хорошая
поддержка для экспериментальных расширений языка; несмотря на то, что
Common Lisp стабилен, он не инертен.

\markboth{PREFACE (SECOND EDITION)}{PREFACE (SECOND EDITION)}

The 1984 definition of Common Lisp was imperfect and incomplete.
In some cases this was inadvertent: some odd boundary situation was
overlooked and its consequences not specified, or different
passages were in conflict, or some property of Lisp was so well-known
and traditionally relied upon that I forgot to write it down.
In other cases the informal committee that was defining
Common Lisp could not settle on a solution, and therefore agreed to leave
some important aspect of the language unspecified rather than
choose a less than satisfactory definition.  An example
is error handling; 1984 Common Lisp had plenty of ways to signal
errors but no way for a program to trap or process them.

Версия Common Lisp'а от 1984 года была несовершенной и
незавершенной. В некоторых случаях допускались неосторожности:
некоторые двузначные ситуации игнорировались и из следствия не
определялись, или разные вещи конфликтовали или некоторые свойства
Lisp'а были так хорошо известны, что на них традиционно
полагались, даже я забыл их записать. В других случаях
неофициальный коммитет, что создавал Common Lisp, не мог принять
решение и соглашался оставить некоторые важные вещи языка
неопределенными, чем выбирать менее удачный вариант. Например,
обработка ошибок; в Common Lisp 1984 года было изобилие способов
генерации сингалов об ошибках, но не было методов для их ловушек. 

Over the next year I collected reports of errors in the book and gaps in the
language.  In December 1985, a group of implementors and users met in
Boston to discuss the state of Common Lisp.  I prepared
two lists for this meeting, one of errata and clarifications
that I thought would be relatively uncontroversial (boy, was I wrong!)
and one of more substantial changes I thought should be considered and
perhaps voted upon.  Others also brought proposals to discuss.
It became clear to everyone that there was now enough interest in Common Lisp,
and dependence on its stability, that a more formal mechanism was needed for
managing changes to the language.

This realization led to the formation of X3J13, a subcommittee of
ANSI committee X3, to produce a formal American National Standard
for Common Lisp.  That process is nearing completion.  X3J13 has
completed the bulk of its technical work in rectifying the 1984
definition and codifying extensions to that definition that have
received widespread use and approval.  A draft standard is now
being prepared; it will probably be available
in 1990.  There will then be a period (required by ANSI) for
public review.  X3J13 must then consider the comments it receives
and respond appropriately.  If the comments result
in substantial changes to the draft standard, multiple public review
periods may be required before the draft can be approved as an American
National Standard.

Fortunately, X3J13 has done an outstanding job of documenting its work.
For every change that came to a formal vote, a document was prepared
that described the problem to be solved and one or more solutions.
For each solution there is a detailed proposal for changing the
language; a rationale; test cases that distinguish the proposal
from the status quo or from other proposals for solving that problem;
discussions of current practice, cost to implementors, cost to users,
cost of not adopting the proposal, benefits of adoption,
aesthetic criteria; and any relevant informal discussion that may have 
preceded creation of the formal proposal.  All of these proposal
documents were made available on-line as well as in paper form.
By my count, by June 1989 some
186 such proposals were approved as language changes.
(This count does not include many proposals that came before the committee
but were rejected.)

The purpose of this second edition is to bridge the gap between the
first edition and the forthcoming ANSI standard for Common Lisp.
Because of the requirement for formal public review,
it will be some time yet before the ANSI standard is final.
This book in no way resembles the forthcoming standard (which
is being written independently
by Kathy Chapman of Digital Equipment Corporation with assistance
from the X3J13 Drafting Subcommittee).

I have incorporated into this second edition
a great deal of material based on the votes of X3J13,
in order to give the reader a picture of where the language is heading.
My purpose here is not simply to quote the X3J13 documents verbatim
but to paraphrase them and relate them to the structure of the first
edition.  A single vote by X3J13 may be discussed in many parts of this book,
and a single passage of this book may be affected by many of the votes.

I wish to be very clear: this book is not an official document
of X3J13, though it is based on publicly available material
produced by X3J13.  In no way does this book constitute a definitive
description of the forthcoming ANSI standard.  The
committee's decisions have been remarkably stable (it has rescinded
earlier decisions only two or three times), and I do not
expect radical changes in direction.
Nevertheless, it is quite probable
that the draft standard will be substantively revised in response to
editorial review or public comment.
I have therefore reported here on the actions of X3J13 not to
inscribe them in \strut stone, but to make clear how the language
of the first edition is likely to change.
I have tried to be careful
in my wording to avoid saying ``the language has been changed''
and to state simply that
``X3J13 voted at such-and-so time to make the following change.''

Until the day when an official ANSI Common Lisp standard emerges,
it is likely that the 1984 definition of Common Lisp will
continue to be used widely.  This book has been designed
to be used as a reference both to the 1984 definition
and to the language as modified by the actions of X3J13.

It contains the entire text of the first edition
of {\it Common Lisp: The Language}, with corrections
and minor editorial changes;
however, more than half of the material in this edition is new.
All new material is
identified by solid lines
in the left margin.
Dotted lines in the left margin indicate material from the first edition
that applies to the 1984 definition but that has been modified
by a vote of X3J13.  Modifications to these outmoded
passages are explained by preceding or following text (which will
have a solid line in the margin).
In summary:
\begin{itemize}
\item To use the 1984 language definition, read all material that does not
have a solid line in the margin.
\item To use the updated language definition, read everything, but
be wary of material with a dotted line in the margin.
\end{itemize}

At the end of the book is an index of the X3J13 votes, ordered
by the committee's internal code names (included to ease cross-reference
to the X3J13 documents, which may be useful during the public review
periods).  References to this list of votes appear as numbers
in angle brackets; thus
``$\langle$14$\rangle$'' refers to the vote on issue number 14, whereas
``[14]'' refers to reference 14 in the bibliography.

I have kept changes to the wording of the first-edition material to a minimum.
Obvious spelling and typographical errors have been corrected,
and the entire text has been edited to a uniform style of
spelling and punctuation.  (Note in particular that the first edition
used the spelling ``signalling'' but this edition,
in deference to the style decision of the X3J13 Drafting
Subcommittee, uses ``signaling.'')  A few minor
changes were made to accommodate typographical or layout constraints.
(For example, the word ``also'' has been deleted from the first
sentence of chapter~1, partly to make that paragraph look better
and partly to allow a better page break at the bottom of page~2.)
In a very few cases the first edition contained substantive errors
that I could not in good conscience correct silently; these have
been flagged by paragraphs beginning with the phrase
{\it Notice of correction}.

The chapter and section numbering of this edition matches that
of the first edition, with the exception that a new
section~\ref{STRUCTURE-TRAVERSAL-SECTION}
has been interpolated.
Four new chapters (\ref{LOOP}--\ref{CONDITION})
describe substantial changes approved by X3J13: an extended
\cd{loop} macro, a pretty printer interface, the Common Lisp
Object System, and the Common Lisp Condition System.

X3J13, in the course of its work, formed a subcommittee to study
whether additional means of iteration
should be standardized for use in Common Lisp, for a great
deal of existing practice in this area was not included in the
first edition because of lack of agreement in~1984.
The X3J13 Iteration Subcommittee produced reports on three possible facilities.
One (\cd{loop}) was approved for inclusion in the forthcoming draft standard
and is described in chapter~\ref{LOOP}.

X3J13 expressed interest in the other two approaches (series and generators),
but the consensus as of January 1989
was that these other approaches were not yet sufficiently mature or
in sufficiently widespread use to warrant inclusion in the draft Common Lisp
standard at that time.  However, the subcommittee was directed to continue work
on these approaches and X3J13 is open to the possibility of standardizing
them at a later date.
Please note that I do not wish the prejudge the
question of whether X3J13 will ever choose to make the other two proposals the
subject of standardization.  Nevertheless,
I have chosen to include them in the second edition,
in cooperation with Dr.~Richard~C.~Waters,
as appendices~\ref{SERIES} and~\ref{GENERATORS},
in order to make these ideas
available to the Lisp community.  In my judgement these proposals
address an area of language design not otherwise covered by Common Lisp
and are likely to have practical value even if they are never
adopted as part of a formal standard.

Some new material in this book has nothing to do with the work of X3J13.
In many places I have added explanations, clarifications, new examples,
warnings, and tips on writing portable code.
Appendix~\ref{BACKQUOTE-SIMULATOR} contains a piece of code
that may help in understanding the backquote syntax.

This second edition,
unlike the first edition, also includes a few diagrams to pep up the text.
However, there are absolutely no new jokes, and very few outright lies.


\chapter*{\vtop{\LARGE \sf \noindent Acknowledgments Благодарности \relax\\[5pt]
                \normalsize\sf SECOND EDITION ВТОРОЕ ИЗДАНИЕ}}


First and foremost, I must thank the many people in the Lisp
community who have worked so hard to specify, implement, and use
Common Lisp.  Some of these have volunteered many hours
of effort as members of ANSI committee X3J13.  Others
have made presentations or proposals to X3J13, and yet others
have sent suggestions and corrections to the first edition directly to me.
This book builds on their efforts as well as mine.

\markboth{ACKNOWLEDGMENTS (SECOND EDITION)}{ACKNOWLEDGMENTS (SECOND EDITION)}

An early draft of this book was made available to all members
of X3J13 for their criticism.  I have also worked with
the many public documents that have been written during the course
of the committee's work (which is not over yet).
It is my hope that this book is an accurate reflection of the
committee's actions as of October 1989.
Nevertheless, any errors or inconsistencies are my responsibility.
The fact that I have made a draft available to certain persons,
received feedback from them, or thanked them in these
acknowledgments does not necessarily imply that any one of them
or any of the institutions with which they are affiliated endorse this book
or anything of its contents.

Digital Press and I gave permission to X3J13 to use any or all parts
of the first edition in the production of an ANSI Common Lisp standard.
Conversely, in writing this book I have worked with publicly available
documents produced by X3J13 in the course of its work, and in some cases
as a courtesy have obtained the consent of the authors of those documents
to quote them extensively.  This common ancestry will result in similarities
between this book and the emerging ANSI Common Lisp standard (that is the
purpose, after all).  Nevertheless, this second edition 
has no official connection whatsoever
with X3J13 or ANSI, nor is it endorsed by either of those institutions.

The following persons have been members of X3J13 or involved in its
activities at one time or another:
Jim Allard, Dave Andre, Jim Antonisse, William Arbaugh, John
Aspinall, Bob Balzer, Gerald Barber, Richard Barber, Kim Barrett,
David Bartley, Roger Bate, Alan Bawden, Michael Beckerle, Paul
Beiser, Eric Benson, Daniel Bobrow, Mary Boelk, Skona Brittain, Gary
Brown, Tom Bucken, Robert Buckley, Gary Byers, Dan Carnese, Bob
Cassels, J\'er\^ome Chailloux, Kathy Chapman, Thomas Christaller,
Will Clinger, Peter Coffee, John Cugini, Pavel Curtis, Doug Cutting,
Christopher Dabrowski, Jeff Dalton, Linda DeMichiel, Fred Discenzo,
Jerry Duggan, Patrick Dussud, Susan Ennis, Scott Fahlman, Jogn Fitch,
John Foderaro, Richard Gabriel, Steven Gadol, Nick Gall, Oscar
Garcia, Robert Gian\-sira\-cusa, Brad Goldstein, David Gray, Richard
Greenblatt, George Hadden, Steve Haflich, Dave Henderson, Carl
Hewitt, Carl Hoffman, Cheng Hu, Masayuki Ida, Takayasu Ito, Sonya
Keene, James Kempf, Gregory Jennings, Robert Kerns, Gregor Kiczales,
Kerry Kimbrough, Dieter Kolb, Timothy Koschmann, Ed Krall, Fritz
Kunze, Aaron Larson, Joachim Laubsch, Kevin Layer, Michael Levin, Ray
Lim, Thom Linden, David Loeffler, Sandra Loosemore, Barry Margolin,
Larry Masinter, David Matthews, Robert Mathis, John McCarthy, Chris
McConnell, Rob McLachlan, Jay Mendelsohn, Martin Mikelsons, Tracey
Miles, Richard Mlyarnik, David Moon, Jarl Nilsson, Leo Noordhulsen,
Ronald Ohlander, Julian Padget, Jeff Peck, Jan Pedersen, Bob
Pellegrino, Crispin Perdue, Dan Pierson, Kent Pitman, Dexter Pratt,
Christian Quiennec, B. Raghavan, Douglas Rand, Jonathan Rees, Chris
Richardson, Jeff Rininger, Walter van Roggen, Jeffrey Rosenking,  Don
Sakahara, William Scherlis, David Slater, James Smith, Alan Snyder,
Angela Sodan, Richard Soley, S. Sridhar, Bill St.\ Clair, Philip
Stanhope, Guy Steele, Herbert Stoyan, Hiroshi Torii, Dave Touretzky,
Paul Tucker, Rick Tucker, Thomas Turba, David Unietis, Mary Van
Deusen, Ellen Waldrum, Richard Waters, Allen Wechsler, Mark Wegman,
Jon~L White, Skef Wholey, Alexis Wieland, Martin Yonke, Bill York,
Taiichi Yuasa, Gail Zacharias, and Jan Zubkoff.





I must express particular gratitude and appreciation to a number
of people for their exceptional efforts:

Larry Masinter, chairman of
the X3J13 Cleanup Subcommittee, developed the standard format for
documenting all proposals to be voted upon.  The result has been
an outstanding tehcnical and historical record of all the actions
taken by X3J13 to rectify and improve Common Lisp.

Sandra Loosemore, chairwoman of the X3J13 Compiler Subcommittee,
produced many proposals for clarifying the semantics of the compilation
process.  She has been a diligent stickler for detail and has helped
to clarify many parts of Common Lisp left vague in the first edition.

Jon L White, chairman of the X3J13 Iteration Subcommittee,
supervised the consideration of several controversial
proposals, one of which (\cd{loop}) was eventually adopted by X3J13.

Thom Linden, chairman of the X3J13 Character Subcommittee,
led a team in grappling with the difficult problem of accommodating
various character sets in Common Lisp.  One result is that
Common Lisp will be more attractive for international use.

Kent Pitman, chairman of the X3J13 Error Handling Subcommittee,
plugged the biggest outstanding
hole in Common Lisp as described by the first edition.

Kathy Chapman, chairwoman of the X3J13 Drafting Subcommittee,
and principal author of the draft standard, has not only written
a great deal of text but also insisted on coherent and consistent
terminology and pushed the rest of the committee forward when necessary.

Robert Mathis, chairman of X3J13, has kept administrative matters
flowing smoothly during technical controversies.

Mary Van Deusen, secretary of X3J13, kept excellent minutes
that were a tremendous aid to me in tracing the history of
a number of complex discussions.

Jan Zubkoff, X3J13 meeting and mailing
organizer, knows what's going on, as always.
She is a master of organization and of physical arrangements.
Moreover, she once again pulled me out of the fire at the last minute.

Dick Gabriel, international representative for X3J13,
has kept information flowing smoothly between Europe, Japan,
and the United States.  He provided a great deal of the energy and drive
for the completion of the Common Lisp Object System specification.
He has also provided me with a great
deal of valuable advice and has been on call for last-minute
consultation at all hours during the final stages of preparation
for this book.

David Moon has consistently been a source of reason,
expert knowledge, and careful scrutiny.  He has read the
first edition and the X3J13 proposals perhaps more carefully
than anyone else.

David Moon, Jon~L White, Gregor Kiczales, Robert Mathis, Mary Boelk
provided extensive feedback on an early draft of this book.
I thank them as well as the many others who commented in one way
or another on the draft.

I wish to thank the authors of large proposals to X3J13
that have made material available for more or less wholesale
inclusion in this book as distinct chapters.
This material was produced primarily for the use of X3J13 in its work.
It has been included here
on a non-exclusive basis with the consent of the authors.

The author of the chapter on \cd{loop} (Jon~L White)
notes that the chapter is based on documentation
written at Lucid, Inc., by Molly~M. Miller,
Sonia Orin Lyris, and Kris Dinkel.
Glenn Burke, Scott Fahlman, Colin Meldrum,
David Moon, Cris Perdue, and Dick Waters
contributed to the design of the \cd{loop} macro.

The authors of the Common Lisp Object System specification
(Daniel G.~Bobrow, Linda G.~DeMichiel,
Richard P.~Gabriel, Sonya E.~Keene, Gregor Kiczales,
and David A.~Moon)
wish to thank Patrick Dussud, Kenneth Kahn,
Jim Kempf, Larry Masinter, Mark Stefik,
Daniel~L. Weinreb, and Jon~L White
for their contributions.

The author of the chapter on Conditions (Kent M. Pitman)
notes that there is a paper \cite{EXCEPTIONAL-SITUATIONS}
containing background information about the design of the
condition system, which is based on the condition system
of the Symbolics Lisp Machines \cite{SIGNALLING-CONDITIONS}.
The members of the X3J13 Error Handling Subcommittee
were
Andy Daniels and Kent Pitman.
Richard Mlynarik and David~A. Moon made major design contributions.
Useful comments, questions,
suggestions, and criticisms were provided by
    Paul Anagnostopoulos,
    Alan Bawden,
    William Chiles,
    Pavel Curtis,
    Mary Fontana,
    Dick Gabriel,
    Dick King,
    Susan Lander,
   David D. Loeffler,
 Ken Olum,
 David~C. Plummer,
 Alan Snyder,
   Eric Weaver, and
Daniel~L. Weinreb.
The Condition System was designed specifically to
accommodate the needs of Common Lisp.
The design is, however, most directly based on the ``New Error System''
(NES) developed at Symbolics by    David L. Andre,
    Bernard~S. Greenberg,
    Mike McMahon,
    David~A. Moon, and
    Daniel~L. Weinreb.
The NES was in turn based on experiences with the original Lisp
Machine error system (developed at MIT), which was found to be
inadequate for the needs of the modern Lisp Machine environments.
Many aspects of the NES were inspired by the (PL/I) condition
system used by the Honeywell Multics operating system. Henry Lieberman
provided
conceptual guidance and encouragement in the design of the NES.
A reimplementation of the NES for non-Symbolics Lisp Machine 
dialects (MIT, LMI, and TI) was done at MIT by Richard~M. Stallman.
During the process
of that reimplementation, some conceptual changes were made which
have significantly influenced the Common Lisp Condition System.

As for the smaller but no less important proposals,
Larry Masinter deserves recognition as an author of over half of them.
He has worked indefatigably to write up proposals and to polish
drafts by other authors.  Kent Pitman, David Moon, and Sandra Loosemore
have also been notably prolific,
as well as Jon~L White, Dan Pierson, Walter van Roggen,
Skona Brittain, Scott Fahlman, and myself.
Other authors of proposals include
David Andre,
John Aspinall,
Kim Barrett,
Eric Benson,
Daniel Bobrow,
Bob Cassels,
Kathy Chapman,
WIlliam Clinger,
Pavel Curtis,
Doug Cutting,
Jeff Dalton,
Linda DiMichiel,
Richard Gabriel,
Steven Haflich,
Sonya Keene,
James Kempf,
Gregor Kiczales,
Dieter Kolb,
Barry Margolin,
Chris McConnell,
Jeff Peck,
Jan Pedersen,
Crispin Perdue,
Jonathan Rees,
Don Sakahara,
David Touretzky,
Richard Waters, and
Gail Zacharias.

I am grateful to Donald~E. Knuth and his colleagues for producing
the \TeX\ text formatting system \cite{KNUTH-TEXBOOK},
which was used to produce
and typeset the manuscript.
Knuth did an especially good job of publishing the program for
\TeX~\cite{KNUTH-TEX-PROGRAM};
I had to consult the code about eight times while debugging particularly
complicated macros.  Thanks to the extensive indexing
and cross-references, in each case it took me less than five minutes to
find the relevant fragment of that 500-page program.

I also owe a debt
to Leslie Lamport, author of the \LaTeX\ macro package~\cite{LAMPORT-LATEX}
for \TeX,
within which I implemented the document style for this book.

Blue Sky Research sells and supports Textures, an implementation
of \TeX\ for Apple Macintosh computers; Gayla Groom and Barry Smith
of Blue Sky Research provided excellent technical support when I
needed it.  Other software tools that were invaluable
in preparing this book were QuicKeys (sold by CE Software, Inc.),
which provides keyboard macros;
G\=ofer (sold by Microlytics, Inc.), which performs rapid
text searches in multiple files; Symantec Utilities for Macintosh
(sold by Symantec Corporation), which saved me from more than one disk crash;
and the PostScript language and compatible
fonts (sold by Adobe Systems Incorporated).

Some of this software (such as \LaTeX) I obtained for free and some I bought,
but all have proved to be useful tools of excellent quality.
I am grateful to these developers for creating them.

Electronic mail has been indispensible to
the writing of this book, as well to as the work of X3J13.
(It is a humbling experience to publish a book and then for
the next five years to receive
at least one electronic mail message a week, if not twenty, pointing out
some mistake or defect.)
Kudos to those develop and maintain the Internet, which arose
from the Arpanet and other networks.

Chase Duffy, George Horesta, and Will Buddenhagen of Digital Press have given me
much encouragement and support.  David Ford designed the book and
provided specifications that I could code into \TeX.
Alice Cheyer and Kate Schmit edited the copy for style
and puzzled over the more obscure jokes with great patience.
Marilyn Rowland created the index; Tim Evans and I did some polishing.
Laura Fillmore and her colleagues at Editorial, Inc., have
tirelessly and meticulously checked one draft after another and
has kept the paperwork flowing smoothly during the last hectic weeks
of proofreading, page makeup, and typesetting.

Thinking Machines Corporation has supported all my work with X3J13.
I thank all my colleagues there for their encouragement and help.

Others who provided indispensible encouragement and support include
Guy and Nalora Steele; David Steele; Cordon and Ruth Kerns;
David, Patricia, Tavis, Jacob, Nicholas, and Daniel Auwerda;
Donald and Denise Kerns; and David, Joyce, and Christine Kerns.

Most of the writing of this book took place between
10 P.M.~and 3 A.M.~(I'm not as young as I used to be).
I am grateful to Barbara,
Julia, Peter, and Matthew for putting up with it, and for their love.

\begin{tabbing}
Guy L. Steele Jr. \\
Lexington, Massachusetts \\
All Saints' Day, 1989
\end{tabbing}


\chapter*{\vtop{\LARGE \sf \noindent Acknowledgments Благодарности \relax\\[5pt]
                \normalsize\sf FIRST EDITION ПЕРВОЕ ИЗДАНИЕ (1984)}}

Common Lisp was designed
by a diverse group of people affiliated with many institutions.

\markboth{ACKNOWLEDGMENTS (FIRST EDITION, 1984)}{ACKNOWLEDGMENTS (FIRST EDITION, 1984)}

Contributors to the
design and implementation of Common Lisp and to the polishing of this book
are hereby gratefully acknowledged:
\vskip 0pt plus 10pt
\hrule width 0pt\relax

\begin{tabbing}
\hskip8.5pc\=\kill
Paul Anagnostopoulos\>Digital Equipment Corporation \\
Dan Aronson\>Carnegie-Mellon University \\
Alan Bawden\>Massachusetts Institute of Technology \\
Eric Benson\>University of Utah, Stanford University, and Symbolics,\\
           \>Incorporated \\
Jon Bentley\>Carnegie-Mellon University and Bell
Laboratories \\ Jerry Boetje\>Digital Equipment Corporation \\
Gary Brooks\>Texas Instruments \\
Rodney A. Brooks\>Stanford University \\
Gary L. Brown\>Digital Equipment Corporation \\
Richard L. Bryan\>Symbolics, Incorporated \\
Glenn S. Burke\>Massachusetts Institute of Technology \\
Howard I. Cannon\>Symbolics, Incorporated \\
George J. Carrette\>Massachusetts Institute of Technology \\
Robert Cassels\>Symbolics, Incorporated \\
Monica Cellio\>Carnegie-Mellon University \\
David Dill\>Carnegie-Mellon University \\
Scott E. Fahlman\>Carnegie-Mellon University \\
Richard J. Fateman\>University of California, Berkeley \\
Neal Feinberg\>Carnegie-Mellon University \\
Ron Fischer\>Rutgers University \\
John Foderaro\>University of California, Berkeley \\
Steve Ford\>Texas Instruments
\end{tabbing}

\penalty-10000

\begin{tabbing}
\hskip8.5pc\=\kill
Richard P. Gabriel\>Stanford University and Lawrence Livermore National \\
                  \>Laboratory \\
Joseph Ginder\>Carnegie-Mellon University and Perq Systems Corp. \\
Bernard S. Greenberg\>Symbolics, Incorporated \\
Richard Greenblatt\>Lisp Machines Incorporated (LMI) \\
Martin L. Griss\>University of Utah and Hewlett-Packard Incorporated \\
Steven Handerson\>Carnegie-Mellon University \\
Charles L. Hedrick\>Rutgers University \\
Gail Kaiser\>Carnegie-Mellon University \\
Earl A. Killian\>Lawrence Livermore National Laboratory \\
Steve Krueger\>Texas Instruments \\
John L. Kulp\>Symbolics, Incorporated \\
Jim Large\>Carnegie-Mellon University \\
Rob Maclachlan\>Carnegie-Mellon University \\
William Maddox\>Carnegie-Mellon University \\
Larry M. Masinter\>Xerox Corporation, Palo Alto Research Center \\
John McCarthy\>Stanford University \\
Michael E. McMahon\>Symbolics, Incorporated \\
Brian Milnes\>Carnegie-Mellon University \\
David A. Moon\>Symbolics, Incorporated \\
Beryl Morrison\>Digital Equipment Corporation \\
Don Morrison\>University of Utah \\
Dan Pierson\>Digital Equipment Corporation \\
Kent M. Pitman\>Massachusetts Institute of Technology \\
Jonathan Rees\>Yale University \\
Walter van Roggen\>Digital Equipment Corporation \\
Susan Rosenbaum\>Texas Instruments \\
William L. Scherlis\>Carnegie-Mellon University \\
Lee Schumacher\>Carnegie-Mellon University \\
Richard M. Stallman\>Massachusetts Institute of Technology \\
Barbara K. Steele\>Carnegie-Mellon University \\
Guy L. Steele Jr.\>Carnegie-Mellon University and Tartan Laboratories \\
                 \>Incorporated \\
Peter Szolovits\>Massachusetts Institute of Technology \\
William vanMelle\>Xerox Corporation, Palo Alto Research Center \\ Ellen
Waldrum\>Texas Instruments \\ Allan C. Wechsler\>Symbolics, Incorporated \\
Daniel L. Weinreb\>Symbolics, Incorporated \\
Jon L White\>Xerox Corporation, Palo Alto Research Center \\
Skef Wholey\>Carnegie-Mellon University
\end{tabbing}
\begin{tabbing}
\hskip8.5pc\=\kill
Richard Zippel\>Massachusetts Institute of Technology \\
Leonard Zubkoff\>Carnegie-Mellon University and Tartan Laboratories \\
               \>Incorporated
\end{tabbing}
Some contributions were relatively small; others involved enormous
expenditures of effort and great dedication.  A few of the contributors
served more as worthy adversaries than as benefactors (and do not
necessarily endorse the final design reported here),
but their pointed criticisms were just as important to the polishing of Common Lisp
as all the positively phrased suggestions.
All of the people named above were helpful in one way or another,
and I am grateful for the interest and spirit of cooperation
that allowed most decisions to be made by consensus after due discussion.

Considerable encouragement and moral support were also provided by:
\begin{tabbing}
\hskip1.5in\=\kill
Norma Abel\>Digital Equipment Corporation \\
Roger Bate\>Texas Instruments \\
Harvey Cragon\>Texas Instruments \\
Dennis Duncan\>Digital Equipment Corporation \\
Sam Fuller\>Digital Equipment Corporation \\
A. Nico Habermann\>Carnegie-Mellon University \\
Berthold K. P. Horn\>Massachusetts Institute of Technology \\
Gene Kromer\>Texas Instruments \\
Gene Matthews\>Texas Instruments \\
Allan Newell\>Carnegie-Mellon University \\
Dana Scott\>Carnegie-Mellon University \\
Harry Tennant\>Texas Instruments \\
Patrick H. Winston\>Massachusetts Institute of Technology \\
Lowell Wood\>Lawrence Livermore National Laboratory \\
William A. Wulf\>Carnegie-Mellon University and Tartan Laboratories \\
               \>Incorporated
\end{tabbing}
I am very grateful to each of them.

Jan Zubkoff of Carnegie-Mellon University
provided a great deal of organization,
secretarial support, and unfailing good cheer in the face of adversity.

The development of Common Lisp would most probably not have been possible
without the electronic message system provided by the ARPANET.
Design decisions were made on several hundred distinct points, for the
most part by consensus, and by simple majority vote when necessary.
Except for two one-day face-to-face meetings, all of the language design
and discussion was done through the {ARPANET} message system, which
permitted effortless dissemination of messages to dozens of people, and
several interchanges per day.  The message system also provided
automatic archiving of the entire discussion, which has proved
invaluable in the preparation of this reference manual.  Over the course
of thirty months, approximately 3000 messages were sent (an average of
three per day), ranging in length from one line to twenty pages.
Assuming 5000 characters per printed page of text, the entire
discussion totaled about 1100 pages.  It would have been substantially
more difficult to have conducted this discussion by any other means,
and would have required much more time.

The ideas in Common Lisp have come from many sources and been polished by
much discussion.  I am responsible for the form of this
book, and for any errors or inconsistencies that may remain;
but the credit for the design and support of Common Lisp lies with
the individuals named above, each of whom has made significant
contributions.

The organization and content
of this book were inspired in large part by the
{\it MacLISP Reference Manual} by David A. Moon and others \cite{MOONUAL},
and by the {\it LISP Machine Manual} (fourth edition)
by Daniel Weinreb and David Moon \cite{BLUE-LISPM},
which in turn acknowledges the efforts of Richard Stallman, Mike McMahon,
Alan Bawden, Glenn Burke, and ``many people too numerous to list.''

I thank Phyllis Keenan, Chase Duffy,
Virginia Anderson,
John Osborn,
and Jonathan Baker of Digital Press for their
help in preparing this book for publication.
Jane Blake did an admirable job of copy-editing.
James Gibson and Katherine Downs of Waldman Graphics were most cooperative
in typesetting this book from my on-line manuscript files.

I am grateful to Carnegie-Mellon University and to
Tartan Laboratories Incorporated for supporting me in the writing
of this book over the last three years.

Part of the work on this book was
done in conjunction with the Carnegie-Mellon University Spice Project,
an effort to construct an advanced scientific software development
environment for personal computers.
The Spice Project is
supported by the Defense Advanced Research Projects Agency, Department   
of Defense, ARPA Order 3597, monitored by the Air Force Avionics   
Laboratory under contract F33615-78-C-1551.  The views   
and conclusions contained in this book are those of the author
and should not be interpreted as representing the official policies,   
either expressed or implied, of the Defense Advanced Research   
Projects Agency or the U.S. Government.

Most of the writing of this book took place between
midnight and 5 A.M.  I am grateful to Barbara, Julia, and Peter
for putting up with it, and for their love.

\begin{tabbing}
Guy L. Steele Jr. \\
Pittsburgh, Pennsylvania \\
March 1984
\end{tabbing}
   % Title page, toc, acknowledgements

\pagenumbering{arabic}

%Part{Intro, Root = "CLM.MSS"}
% Chapter of Common Lisp Manual.  Copyright 1984, 1988, 1989 Guy L. Steele Jr.

\clearpage\def\pagestatus{FINAL PROOF}

\ifx \rulang\Undef

\chapter{Introduction}

Common Lisp is a new dialect of Lisp, a
successor to MacLisp \cite{MOONUAL,PITMANUAL}, influenced strongly by
Zetalisp \cite{BLUE-LISPM,GREEN-LISPM} and to some extent by Scheme
\cite{SCHEME-REVISED-REPORT} and Interlisp \cite{INTERLISP}.

\section{Purpose}

Common Lisp is intended to meet these goals:

\begin{flushdesc}
\item[\emph{Commonality}]
Common Lisp originated in an attempt to focus the
work of several implementation groups, each of which was constructing successor
implementations of MacLisp for different computers.  These
implementations had begun to diverge because of the differences in the
implementation environments: microcoded personal computers (Zetalisp,
Spice Lisp), commercial timeshared computers (NIL---the ``New Implementation of
Lisp''), and supercomputers (S-1 Lisp).  While the differences among the several
implementation environments of necessity will continue to force
certain incompatibilities among the
implementations, Common Lisp serves as a common dialect to
which each implementation makes any necessary extensions.

\item[\emph{Portability}]
\begingroup\looseness=1
Common Lisp intentionally excludes features
that cannot be implemented easily on a broad class of machines.
On the one hand, features that are difficult or expensive
to implement on hardware without special microcode are avoided
or provided in a more abstract and efficiently implementable form.
(Examples of this are the invisible forwarding pointers
and locatives of Zetalisp.  Some of the problems that they solve
are addressed in different ways in Common Lisp.)
On the other hand, features that are useful only on certain ``ordinary''
or ``commercial'' processors are avoided or made optional.  (An example of
this is the type declaration facility, which is useful in some
implementations and completely ignored in others.  Type declarations are
completely optional and for correct programs
affect only efficiency, not semantics.)
Common Lisp is designed to make it easy to write programs
that depend as little as possible on machine-specific
characteristics, such as word length, while allowing some variety of
implementation \hbox{techniques}.
\par\endgroup

\item[\emph{Consistency}]
Most Lisp implementations are internally inconsistent
in that by default the interpreter and compiler may assign different
semantics to correct programs.
This semantic difference stems primarily from the fact
that the interpreter assumes all variables to be dynamically scoped,
whereas the compiler assumes all variables to be local unless explicitly
directed otherwise.  This difference has been the usual practice in Lisp
for the sake of convenience
and efficiency but can lead to very subtle bugs.  The definition of
Common Lisp avoids such anomalies by explicitly requiring the interpreter
and compiler to impose identical semantics on correct programs
so far as possible.

\item[\emph{Expressiveness}]
Common Lisp culls what
experience has shown to be the most useful and understandable constructs
from not only MacLisp but also
Interlisp, other Lisp dialects, and other programming languages.
Constructs judged to be awkward or less useful have been
excluded.   (An example is the \cdf{store} construct of MacLisp.)

\item[\emph{Compatibility}]
Unless there is a good reason to the contrary,
Common Lisp strives to be compatible with Lisp Machine Lisp, MacLisp, and
Interlisp, roughly in that order.

\item[\emph{Efficiency}]
Common Lisp has a number of features designed to
facilitate the production of high-quality compiled code in those
implementations whose developers
care to invest effort in an optimizing compiler.
One implementation of Common Lisp, namely S-1 Lisp, already has a compiler
that produces code for numerical computations that is competitive
in execution speed to that produced by a Fortran compiler \cite{S1-COMPILER}.
The S-1 Lisp compiler
extends the work done in MacLisp to produce extremely efficient
numerical code \cite{MACLISP-BEATS-FORTRAN}.

\item[\emph{Power}]
Common Lisp is a descendant of MacLisp, which has
traditionally placed emphasis on providing system-building tools.
Such tools may in turn be used to build the user-level packages
such as Interlisp provides; these packages are not, however, part
of the Common Lisp core specification.  It is expected such packages will
be built on top of the Common Lisp core.

\item[\emph{Stability}]
It is intended that Common Lisp
will change only slowly and with due deliberation.  The various dialects
that are supersets of Common Lisp may serve as laboratories within which to
test language extensions, but such extensions will be added to
Common Lisp only after careful examination and experimentation.
\end{flushdesc}

\vskip 0pt plus 2pt%manual

The goals of Common Lisp are thus very close to those of Standard Lisp
\cite{STANDARD-LISP-REPORT} and Portable Standard Lisp \cite{PSL-MANUAL}.
Common Lisp differs from Standard Lisp
primarily in incorporating more features, including a
richer and more complicated set of data types and more complex
control structures.

This book is intended to be a language specification
rather than an implementation specification
(although implementation notes are scattered throughout the text).
It defines a set of
standard language concepts and constructs that may be used
for communication of data structures and algorithms in the Common Lisp
dialect.  This set of concepts
and constructs is sometimes referred to as the ``core Common Lisp language''
because it contains conceptually necessary or important features.
It is not necessarily implementationally minimal.
While many features could be defined in terms of others
by writing Lisp code, and indeed may be implemented that way,
it was felt that these features should be conceptually primitive
so that there might be agreement among all users as to their usage.
(For example, bignums and rational numbers could be implemented as
Lisp code given operations on fixnums.  However, it is important
to the conceptual integrity of the language that they be regarded
by the user as primitive, and they are useful enough to warrant
a standard definition.)

For the most part, this book defines a programming language, not a
programming environment.  A few interfaces are defined for
invoking such standard programming tools as a compiler, an editor,
a program trace facility, and a debugger, but very little is said
about their nature or operation.  It is expected that one or more
extensive programming environments will be built using Common Lisp as a
foundation, and will be documented separately.

There are now many implementations of Common Lisp,
some programmed by research groups in universities
and some by companies that sell them commercially,
and a number of useful
programming environments have indeed grown up around
these implementations.
What is more, all the goals stated above have been achieved,
most notably that of portability.  Moving large bodies
of Lisp code from one computer to another is now routine.

\section{Notational Conventions}

A number of special notational conventions are used throughout this book
for the sake of conciseness.

\subsection{Decimal Numbers}

All numbers in this book are in decimal notation unless
there is an explicit indication to the contrary.
(Decimal notation is normally taken for granted, of course.
Unfortunately, for certain other dialects of Lisp, MacLisp in particular,
the default notation for numbers is octal (base 8) rather than decimal,
and so the use of decimal notation for describing Common Lisp is,
taken in its historical context, a bit unusual!)

\subsection{Nil, False, and the Empty List}

In Common Lisp, as in most Lisp dialects, the symbol \cdf{nil}
 is used to represent both the empty list and the ``false'' value
for Boolean tests.  An empty list may, of course, also be written
\cd{()}; this normally denotes the same object as \cdf{nil}.
(It is possible, by extremely perverse manipulation of the package system,
to cause the sequence of letters \cdf{nil} to be recognized
not as the symbol that represents the empty list but as another
symbol with the same name.  This obscure possibility will be ignored
in this book.)
These two notations may be used interchangeably as far as the Lisp
system is concerned.  However, as a matter of style,
this book
uses the notation {\emptylist} when it is desirable to emphasize
the use of an empty list, and uses the notation {\false}
when it is desirable to emphasize the use of the Boolean ``false''.
The notation \cd{'nil} (note the explicit quotation mark) is used to emphasize
the use of a symbol.
For example:
\begin{lisp}
~~~~~~~~~~~~~~~~~~~~~~~~~~~\=\kill
(defun three () 3)\>;\textrm{Emphasize empty parameter list} \\
(append '{\emptylist} '{\emptylist}) \EV\ {\emptylist}\>;\textrm{Emphasize use of empty lists} \\
(not {\false}) \EV\ {\true}\>;\textrm{Emphasize use as Boolean ``false''} \\
(get '{\nil} 'color)\>;\textrm{Emphasize use as a symbol}
\end{lisp}

Any data object other than {\false} is construed to be Boolean
``not false'', that is, ``true''.  The symbol {\true} is conventionally
used to mean ``true'' when no other value is more appropriate.
When a function is said to ``return \emph{false}'' or to ``be \emph{false}''
in some circumstance, this means that it returns {\false}.
However, when a function is said to ``return \emph{true}'' or to ``be \emph{true}''
in some circumstance, this means that it returns some value other
than {\false}, but not necessarily {\true}.

\subsection{Evaluation, Expansion, and Equivalence}

Execution of code in Lisp is called \emph{evaluation} because executing a
piece of code normally results in a data object called the \emph{value}
produced by the code.  The symbol \EV\ is used in examples to
indicate evaluation.
For example,
\begin{lisp}
(+ 4 5) \EV\ 9
\end{lisp}
means ``the result of evaluating the code \cd{(+ 4 5)} is (or would be,
or would have been) \cd{9}.'' 

The symbol \EX\ is used in examples to indicate macro expansion.
For example,
\begin{lisp}
(push x v) \EX\ (setf v (cons x v))
\end{lisp}
means ``the result of expanding the macro-call form \cd{(push x v)}
is \cd{(setf v (cons x v))}.''  This implies that the two pieces
of code do the same thing; the second piece of code is
the definition of what the first does.

The symbol \EQ\ is used in examples to indicate code equivalence.
For example,
\begin{lisp}
(gcd x (gcd y z)) \EQ\ (gcd (gcd x y) z)
\end{lisp}
means ``the value and effects of evaluating the form
\cd{(gcd x (gcd y z))} are always the same as the value
and effects of
\cd{(gcd (gcd x y) z)} for any values of the
variables \cdf{x}, \cdf{y}, and \cdf{z}.''
This implies that the two pieces
of code do the same thing; however, neither directly defines
the other in the way macro expansion does.

\subsection{Errors}
\label{INTRO-ERRORS}

When this book specifies that it ``is an error'' for some situation
to occur, this means that:

\begin{itemize}
\item No valid Common Lisp program should cause this situation to occur.

\item If this situation occurs, the effects and results are completely
undefined as far as adherence to the Common Lisp specification is concerned.

\item No Common Lisp implementation is required to detect such an error.
Of course, implementors are encouraged to provide for detection
of such errors wherever reasonable.
\end{itemize}
This is not to say that some particular implementation might not define
the effects and results for such a situation; the point is that no program
conforming to the Common Lisp specification may correctly depend on such
effects or results.

On the other hand, if it is specified in this book that in some situation
``an error is \emph{signaled},'' this means that:

\begin{itemize}
\item If this situation occurs, an error will be signaled
(see \cdf{error} and \cdf{cerror}).

\item Valid Common Lisp programs may rely on the fact that an error will be
signaled.

\item Every Common Lisp implementation is required to detect such an error.
\end{itemize}

In places where it is stated that so-and-so ``must'' or ``must not''
or ``may not''  be the case, then it ``is an error'' if the stated requirement
is not met.  For example, if an argument ``must be a symbol,'' then it
``is an error'' if the argument is not a symbol.  In all cases where
an error is to be \emph{signaled}, the word ``signaled'' is always used
explicitly in this book.

\begin{newer}
X3J13 has adopted a more elaborate terminology for errors,
and has made some effort to specify the type of error to be signaled
in situations where signaling is appropriate.  This effort
was not complete as of September 1989, and I have made little
attempt to incorporate the new error terminology or
error type specifications in this book.  However, the new terminology
is described and used in the specification of the
Common Lisp Object System appearing in chapter~\ref{CLOS}; this gives
the flavor of how erroneous situations will be described,
and appropriate actions prescribed, in the forthcoming ANSI Common
Lisp standard.
\end{newer}

\begin{table}[t]
\caption{Sample Function Description}
\label{Sample-Function-Description}
\begingroup\normalsize
\begin{defun}*[Function]
sample-function arg1 arg2 &optional arg3 arg4

The function \cdf{sample-function} adds together \emph{arg1} and \emph{arg2}, and
then multiplies the result by \emph{arg3}.  If \emph{arg3} is not provided or
is {\nil}, the multiplication isn't done.  \cdf{sample-function} then returns
a list whose first element is this result and whose second element is
\emph{arg4} (which defaults to the symbol \cdf{foo}).
For example: 
\begin{lisp}
(sample-function 3 4) \EV\ (7 foo) \\
(sample-function 1 2 2 'bar) \EV\ (6 bar)
\end{lisp}
In general,
\cd{(sample-function \emph{x} \emph{y})} \EQ\ \cd{(list (+ \emph{x} \emph{y}) 'foo)}.
\end{defun}
\endgroup
\vskip\ruletonoteskip

\hrule

\vskip\ruletonoteskip\null
\caption{Sample Variable Description}
\label{Sample-Variable-Description}
\begingroup\normalsize
\begin{defun}*[Variable]
*sample-variable*

The variable \cd{*sample-variable*} specifies how many times
the special form \cdf{sample-special-form} should iterate.
The value should always be a non-negative integer or {\nil}
(which means iterate indefinitely many times).  The initial value is \cd{0}
(meaning no iterations).
\end{defun}
\endgroup
\vskip\ruletonoteskip
\hrule
\vskip\ruletonoteskip\null
\caption{Sample Constant Description}
\label{Sample-Constant-Description}
\begingroup\normalsize
\begin{defun}*[Constant]
sample-constant

The named constant \cdf{sample-constant} has as its value
the height of the terminal screen in furlongs times
the base-2 logarithm of the implementation's total disk capacity in bytes,
as a floating-point number.
\end{defun}
\endgroup
\end{table}

\begin{table}[t]
\caption{Sample Special Form Description}
\label{Sample-Special-Form-Description}

\begingroup\normalsize

\begin{defspec}*
sample-special-form [name] ({var}*) {form}+

This evaluates each form in sequence as an implicit \cdf{progn}, and does this
as many times as specified by
the global variable \cd{*sample-variable*}.  Each variable \emph{var} is bound
and initialized to \cd{43} before the first iteration, and unbound after
the last iteration.
The name \emph{name}, if supplied, may be used in a \cdf{return-from} form
to exit from the loop prematurely.  If the loop ends normally,
\cdf{sample-special-form} returns {\nil}.
For example:
\begin{lisp}
(setq *sample-variable* 3) \\
(sample-special-form {\emptylist} \emph{form1} \emph{form2})
\end{lisp}
This evaluates \emph{form1}, \emph{form2}, \emph{form1}, \emph{form2}, \emph{form1}, \emph{form2}
in that order.
\end{defspec}
\endgroup
\vskip\ruletonoteskip
\hrule
\vskip\ruletonoteskip\null
\caption{Sample Macro Description}
\label{Sample-Macro-Description}
\begingroup\normalsize
\begin{defmac}*
sample-macro var <declaration* | doc-string> {tag | statement}*

This evaluates the statements as a \cdf{prog} body,
with the variable \emph{var} bound to \cd{43}.
\begin{lisp}
(sample-macro x (return (+ x x))) \EV\ 86 \\
(sample-macro \emph{var} . \emph{body}) \EX\ (prog ((\emph{var} 43)) . \emph{body})
\end{lisp}
\end{defmac}
\endgroup
\end{table}

\subsection{Descriptions of Functions and Other Entities}
\label{FUNCTION-HEADER-NOTATION-SECTION}

Functions, variables, named constants, special forms, and macros are described
using a distinctive typographical format.
Table~\ref{Sample-Function-Description} illustrates the manner
in which Common Lisp functions are documented.
The first line specifies the name of the function,
the manner in which it accepts arguments,
and the fact that it is a function.
If the function takes many arguments, then the names of the arguments
may spill across two or three lines.
The paragraphs following this standard header
explain the definition and uses of the function and often
present examples or related functions.

Sometimes two or more related functions are explained in a single
combined description.  In this situation the headers for all the
functions appear together, followed by the combined description.

In general, actual code (including actual names of functions)
appears in this typeface: \cd{(cons a b)}.
Names that stand for pieces of code (metavariables) are written in
\emph{italics}.  In a function description, the names of the parameters appear
in italics for expository purposes.  The word \cd{\&optional} in the
list of parameters indicates that all arguments past that point are
optional; the default values for the parameters are described in the
text.  Parameter lists may also contain \cd{\&rest}, indicating that an
indefinite number of arguments may appear, or \cd{\&key}, indicating
that keyword arguments are accepted.
(The \cd{\&optional}/\cd{\&rest}/\cd{\&key}
syntax is actually used in Common Lisp function definitions for these purposes.)

Table~\ref{Sample-Variable-Description} illustrates the manner in
which a global variable is documented.  The first line specifies the
name of the variable and the fact that it is a variable.
Purely as a matter of convention, all global variables used
by Common Lisp have names beginning and ending with an asterisk.

Table~\ref{Sample-Constant-Description} illustrates the manner in
which a named constant is documented.  The first line specifies the
name of the constant and the fact that it is a constant.
(A constant is just like a global variable, except that it is
an error ever to alter its value or to bind it to a new value.)

Tables~\ref{Sample-Special-Form-Description}
and~\ref{Sample-Macro-Description} illustrate the documentation
of special forms and macros, which are closely related in purpose.
These are very different from functions.
Functions are called according to a single, specific, consistent syntax;
the \cd{\&optional}/\cd{\&rest}/\cd{\&key} syntax specifies how the function uses its arguments
internally but does not affect the syntax of a call.
In contrast, each special form or macro can have its own idiosyncratic syntax.
It is by special forms and macros that the syntax of Common Lisp is defined
and extended.

In the description of a special form or macro, an italicized word names a
corresponding part of the form that invokes the special form or macro.
Parentheses stand for themselves and should be
written as such when invoking the special form or macro.
Brackets, braces, stars, plus signs, and vertical bars are metasyntactic
marks. 
Brackets,
$\lbrack$ and $\rbrack$, indicate that what they enclose is optional
(may appear zero times or one time in that place); the square
brackets should not be written in code.
Braces, $\lbrace$ and $\rbrace$, simply parenthesize what they enclose
but may be followed by a star, ${}*$, or a plus sign, ${}+$;
a star indicates that what the braces enclose may appear any number of times
(including zero, that is, not at all), whereas a plus sign indicates
that what the braces enclose may appear any non-zero number of times
(that is, must appear at least once).  Within braces or brackets,
a vertical bar, $|$, separates mutually exclusive choices.
In summary, the notation \Mstar{x} means zero or more occurrences
of \emph{x}, the notation \Mplus{x} means one or more occurrences
of \emph{x}, and the notation \Mopt{x} means zero or one occurrence
of \emph{x}.  These notations are also used for syntactic
descriptions expressed as BNF-like productions, as
in table~\ref{NUMBER-SYNTAX-TABLE}.

\begin{newer}
Double brackets, $\dlbrack$ and $\drbrack$, indicate that any number of the
alternatives enclosed may be used, and those used
may occur in any order, but each
alternative may be used at most once unless followed by a star.
For example,
\begin{tabbing}
\emph{p} \Mchoice{x {\Mor} \Mstar{y} {\Mor} z} \emph{q}
\end{tabbing}
means that
at most one \emph{x}, any number of \emph{y}'s, and at most one \emph{z}
may appear between the mandatory occurrences of \emph{p}
and \emph{q}, and those that appear may be in any order.

A downward arrow, \Mind{}, indicates a form of syntactic indirection
that helps to make \Mchoice{~} notation more readable.  If \emph{X} is
some non-terminal symbol occurring on the left-hand side of some BNF
production, then the right-hand
side of that production is to be textually substituted for any occurrence
of \Mind{X}.  Thus the two fragments
\begin{tabbing}
\emph{p} \Mchoice{\Mind{xyz-mixture}} \emph{q} \\
\emph{xyz-mixture} ::= \emph{x\/} {\Mor} \Mstar{y} {\Mor} \emph{z\/}
\end{tabbing}
are together equivalent to the previous example.
\end{newer}

In the last example in table~\ref{Sample-Macro-Description}, notice the
use of dot notation.  The dot appearing in the expression
\cd{(sample-macro \emph{var} . \emph{body})} means that the name \emph{body} stands
for a list of forms, not just a single form, at the end of a list.  This
notation is often used in examples.

\begin{newer}
In the heading line in table~\ref{Sample-Macro-Description}, notice the
use of \Mchoice{~} notation to indicate that any number of declarations
may appear but at most one documentation string (which may appear before,
after, or somewhere in the middle of any declarations).
\end{newer}

\subsection{The Lisp Reader}

The term ``Lisp reader'' refers not to you, the reader of this book,
nor to some person reading Lisp code, but specifically
to a Lisp procedure, namely the function \cdf{read},
which reads characters from an input stream and interprets them by parsing
as representations of Lisp objects.

\subsection{Overview of Syntax}

Certain characters are used in special ways in the syntax of Common Lisp.
The complete syntax is explained in detail in chapter~\ref{IO},
but a quick summary here may be useful:

\begin{indentdesc}{1.2pc}
\item[\cd{(}]
A left parenthesis begins a list of items.  The list may
contain any number of items, including zero.  Lists may be nested.
For example, \cd{(cons (car x) (cdr y))} is a list of three things,
of which the last two are themselves lists.

\item[\cd{)}] A right parenthesis ends a list of items.

\item[\cd{\Xquote}] An acute accent (also called single
quote or apostrophe) followed by an expression \emph{form}
is an abbreviation for \cd{(quote \emph{form})}.  Thus \cd{'foo} means
\cd{(quote foo)} and \cd{'(cons 'a 'b)} means \cd{(quote (cons (quote a) (quote b)))}.

\item[\cd{;}] Semicolon is the comment character.  It and all
characters up to the end of the line are discarded.

\item[\cd{"}] Double quotes surround character strings: \\
\cd{"This is a thirty-nine-character string."}

\item[\cd{{\Xbackslash}}] Backslash is an escape character. 
It causes the next character to be treated as a letter rather than for its usual
syntactic purpose.  For example, \cd{A{\Xbackslash}(B} denotes a symbol whose
name consists of the three characters \cdf{A}, \cd{(}, and \cdf{B}. Similarly,
\cd{"{\Xbackslash}""} denotes a character string containing one character, a
double quote, because the first and third double quotes serve to delimit the
string, and the second double quote serves as the contents of the string.  The
backslash causes the second double quote to be taken literally and prevents it
from being interpreted as the terminating delimiter of the string.

\item[\cd{|}] Vertical bars are used in pairs
to surround the name (or part of the name) of a symbol that has
many special characters in it.  It is roughly equivalent to putting a
backslash in front of every character so surrounded.  For example,
\cd{|A(B)|}, \cd{A|(|B|)|}, and \cd{A{\Xbackslash}(B{\Xbackslash})} all mean the symbol whose name
consists of the four characters \cdf{A}, \cd{(}, \cdf{B}, and \cd{)}.

\item[\cd{\#}] The number sign signals the beginning of a
complicated syntactic structure.
The next character designates the precise syntax to follow.
For example, \cd{\#o105} means $105_{8}$ (105 in octal notation);
\cd{\#x105} means $105_{16}$ (105 in hexadecimal notation);
\cd{\#b1011} means $1011_{2}$ (1011 in binary notation);
\cd{\#{\Xbackslash}L} denotes a character object for the character \cdf{L}; and
\cd{\#(a b c)} denotes a vector of three elements \cdf{a}, \cdf{b}, and \cdf{c}.
A particularly important case is that \cd{\#'\emph{fn}} means \cd{(function \emph{fn})},
in a manner analogous to \cd{'\emph{form}} meaning \cd{(quote \emph{form})}.

\item[\cd{{\Xbq}}] Grave accent (``backquote'') signals that
the next expression is a template that may contain commas.  The backquote syntax
represents a program that will construct a data structure
according to the template.

\item[\cd{,}] Commas are used within the backquote syntax.

\item[\cd{:}] Colon is used to indicate which package a
symbol belongs to. For example, \cd{network:reset} denotes the symbol named
\cdf{reset} in the package named \cdf{network}.  A leading colon indicates a {\it
keyword}, a symbol that always evaluates to itself.
The colon character is not actually part of the print name
of the symbol.
This is all explained in chapter~\ref{XPACK}; until you read
that, just keep in mind that a symbol notated with a leading colon
is in effect a constant that evaluates to itself.
\end{indentdesc}

\vskip 0pt plus 2pt%manual

\begin{new}%CORR
\emph{Notice of correction.}
In the first edition, the characters ``\cd{,}'' and ``\cd{:}'' at the
left margin above were inadvertently omitted.
\end{new}

Brackets, braces, question mark, and exclamation point
(that is, \cd{{\Xlbracket}}, \cd{{\Xrbracket}}, \cd{{\Xlbrace}}, \cd{{\Xrbrace}},
\cd{?}, and \cd{!}) are not used for any purpose in standard Common Lisp syntax.
These characters are explicitly reserved to the user, primarily
for use as \emph{macro characters} for user-defined lexical syntax extensions
(see section~\ref{MACRO-CHARACTERS-SECTION}).

\begin{obsolete}
All code in this book is written using lowercase letters.
Common Lisp is generally insensitive to the case in which code
is written.  Internally, names of symbols are ordinarily
converted to and stored in uppercase form.
There are ways to force case conversion on output if desired;
see \cd{*print-case*}.
In this book, wherever an interactive exchange between a user
and the Lisp system is shown, the input is exhibited with lowercase
letters and the output with uppercase letters.
\end{obsolete}

\begin{newer}
X3J13 voted in June 1989 \issue{READ-CASE-SENSITIVITY} to introduce
\cdf{readtable-case}.  Certain settings allow the names of symbols
to be case-sensitive.  The default behavior, however, is as described
in the previous paragraph.  In any event, only uppercase letters
appear in the internal print names of symbols naming the
standard Common Lisp facilities described in this book.
\end{newer}

%RUSSIAN
\else

\chapter{Вступление}

Common Lisp это новый диалект Lisp'а, наследник
MacLisp'а \cite{MOONUAL,PITMANUAL}, под влиянием
ZetaLisp'а \cite{BLUE-LISPM,GREEN-LISPM}, в некоторой мере расширенный Schema'ой
\cite{SCHEME-REVISED-REPORT} и Interlisp'ом \cite{INTERLISP}. 

\section{Цель}

Common Lisp предназначен для достижения следующих целей: 

\begin{flushdesc}
\item[\emph{Объединение}]
Common Lisp создан в попытке сфокусировать работу нескольких групп
разработчиков, каждая из которых создавала потомка MacLisp для
различных компьютеров. Это реализации начинали отличаться из-за различий в
платформах: персональные компьютеры (Zetalisp, 
Spice Lisp), коммерческие компьютеры с разделением времени
(NIL--- <<Новая реализация Lisp'а>>) и суперкомпьютеры (S-1
Lisp). Тогда как различия между платформам приводят к несовместимостям
между реализациями, Common Lisp предоставляет общий диалект,
который каждая реализация будет расширять для своих потребностей.

\item[\emph{Переносимость}]
\begingroup\looseness=1
Переносимость
Common Lisp умышленно исключает функционал, который не может быть легко
реализован на  широком  спектре  машин. С одной
стороны, сложный или дорогой в аппаратной реализации без специальных
микрокодов функционал исключается или
представляется в более абстрактной и эффективно реализуемой форме.
(Примером тому являются невидимые ссылочные указатели и локативы ZetaLisp'а.
Некоторые из решаемых ими проблем в CommonLisp разрешаются другими путями.)
С другой стороны, функционал полезный только на некоторых <<обычных>> или
<<коммерческих>> процессорах исключается или делается опциональными. (Примером
тому является система декларации типа, которая на некоторых реализациях
полезна, а на других полностью игнорируется. Декларации типов полностью
опциональны и в правильных программах влияют только на эффективность, а не на
семантику.) Common Lisp спроектирован для упрощения построения программ,
которые как можно меньше зависят от машинно-специфичных характеристик, таких,
как, например, длина слова, но при этом допускает некоторые различия реализаций.

\par\endgroup

\item[\emph{Согласованность}]
Многие реализации Lisp'а внутренне не согласованы в том, что семантика одной и
той же корректной программы может различаться для интерпретатора и
компилятора. Эти семантические различия преимущественно вытекают
из факта, что интерпретатор считает все переменные динамическими, тогда как
компилятор считает все переменные лексическими, если иное не указано явно. Такое
различие было обычной практикой в Lisp'е для 
достижения удобства и эффективности, но могло быть причиной скрытых, очень
"тонких" ошибок. Определение Common Lisp'а исключает такие аномалии явным
требованием к интерпретатору и компилятору реализовывать
идентичные семантики для корректных программ настолько, насколько
это возможно.

\item[\emph{Выразительность}]
Common Lisp собрал конструкции, которые, как показывает опыт, наиболее удобны и
понятны, не только из MacLisp'а, но также из других диалектов, и языков
программирования. Конструкции, оценённые как неуклюжие или бесполезные были
исключены. 

\item[\emph{Совместимость}]
Common Lisp старается быть совместимым с Lisp Machine Lisp'ом,
MacLisp'ом и Interlisp'ом, примерно в таком порядке.

\item[\emph{Эффективность}]
Common Lisp содержит много функционала, созданного для облегчения
производства высококачественного скомпилированного кода в тех
реализациях, разработчики которых заинтересованы в создании
эффективного компилятора. Одна реализация Common Lisp'а называемая
S-1 Lisp, уже содержит компилятор, который производит код для
математических вычислений, который конкурирует в скорости выполнения
с кодом, произведенных компилятором Fortran'а \cite{S1-COMPILER}. Компилятор S-1
Lisp дополняет по созданию наиболее эффективных численных вычислений,
проделанную в MacLisp'е \cite{MACLISP-BEATS-FORTRAN}.

\item[\emph{Мощность}]
Common Lisp является потомком MacLisp'а, который традиционно делал
акцент на предоставлении иструментов для построения систем. Такие инструменты в
свою очередь могли быть использованы для создания пользовательских
пакетов, вроде тех, что предоставлял Interlisp. Эти пакеты, однако, не
являются частью спецификации Common Lisp'а. Ожидается, что
такие пакеты будут построены на основе Common Lisp ядра.

\item[\emph{Стабильность}]
Предполагается, что Common Lisp будет изменяться медленно с
должным обдумыванием. 
Различные диалекты, которые являются надмножествами Common Lisp'а, могут служить
лабораториями для тестирования расширений языка, но такие расширения будут
добавляться в Common Lisp только после внимательного изучения и экспериментов.
\end{flushdesc}

Цели Common Lisp'а, таким образом, очень близки к Standard Lisp'у \cite{STANDARD-LISP-REPORT}
и Portable Standard Lisp'у \cite{PSL-MANUAL}. Common Lisp отличается от Standard
Lisp'а преимущественно тем, что содержит больше возможностей, включая более
богатую и более сложную систему типов и более сложные управляющие конструкции. 

Эта книга прежде всего предназначена быть спецификацией языка, а не
описанием реализации (однако, примечания для реализаций встречаются в тексте).
Книга определяет набор стандартных языковых концепций и конструкций, которые
могут использоваться для связи данных и алгоритмов в диалекте Common Lisp. Этот
набор концепций и конструкций иногда называют <<ядром языка Common
Lisp>>, потому что он содержит концептуально необходимые или важные вещи. 
Это ядро не является необходимым  минимумом для реализации.
есмотря на то, что  многие из его конструкций
могут быть определены через другие, просто написанием
Lisp кода, все же кажется, что они должны быть
концептуальными примитивами, чтобы было согласие между пользователями по поводу
того, как ими пользоваться. (Например, bignums и
рациональные числа могут быть реализованы как Lisp код, оперирующий 
fixnum. Тем не менее, для концептуальной целостности языка важно то, что
пользователи считают их  примитивами, и они достаточно полезны для того, чтобы
внести их в стандарт.)

По большей части, данная книга описывает язык программирования, но не средства
программирования. Для обращения к таким
стандартным программным средствам, как компилятор, редактор, функции
трассировки и отладчик, определены несколько интерфейсов, но об их природе и
функционировании сказано очень мало. Предполагается, что на основе Common Lisp'а
будут построены одна или несколько обширных сред программирования, и они будут
содержать отдельную документацию. 

Теперь есть много реализаций Common Lisp'а. Некоторые были запрограммированы
исследовательскими группами в университетах, а некоторые --- компаниями, которые
продают их в коммерческих целях, и вокруг
этих реализаций фактически вырос ряд полезных сред для программирования.
Более того, все вышеуказанные цели были достигнуты, и прежде всего ---
переносимость. Перемещение большого количества Lisp-кода с одного компьютера на
другой теперь является рутинной операцией.

\section{Условные обозначения}

Для выразительности в книге используется некоторое количество условных обозначений.

\subsection{Десятичные числа}

Все числа в данной книге представлены в десятичной системе счисления кроме мест,
где система счисления указывается явно. 
(Конечно, десятичная система обычно и используется в работе.
К несчастью, в некоторых других диалектах Lisp'а, в частности в MacLisp'е,
нотацией по умолчанию является восьмеричная (основание 8), вместо десятичной, и
использование десятичной системы в описании Common Lisp'а в историческом
контексте слегка необычно!) 

\subsection{Nil, False и пустой список}

В Common Lisp'е, как и во многих диалектах Lisp'а, символ \cdf{nil} используется
для представления пустого списка и булева значения <<ложь>>. Пустой список,
конечно, может, также быть записан так: \cd{()}; это обычно означает то же, что
и \cdf{nil}.
(Конечно, существует возможность крайне извращённым способом переопределить
 с помощью системы пакетов значение последовательности букв \cdf{nil}, которое будет 
обозначать не пустой список, а другой символ с этим именем. Эта мутная
возможность игнорируется в данной книге.)
Эти два обозначения могут использоваться взаимозаменяемо настолько, насколько
позволяет Lisp. В данной книге используется обозначение {\emptylist}, когда необходимо
подчеркнуть использование пустого списка, и {\false}, когда обозначается булево
значение <<ложь>>. Запись \cd{'nil} (обратите внимание на явный знак кавычки)
используется для подчеркивания, что обозначение используется как символ.
Например:
\begin{lisp}
~~~~~~~~~~~~~~~~~~~~~~~~~~~\=\kill
(defun three () 3)\>;\textrm{Обозначает пустой список параметров} \\
(append '{\emptylist} '{\emptylist}) \EV\ {\emptylist}\>;\textrm{Обозначает использование
пустых списков} \\
(not {\false}) \EV\ {\true}\>;\textrm{Подчёркивает использование как булева значения <<ложь>>} \\
(get '{\nil} 'color)\>;\textrm{Подчёркивает использование как символа}
\end{lisp}

Любой объект данных, не являющийся {\false} истолковывается как булево значение
<<не ложь>>, которое является <<истиной>>. Символ {\true} обычно используется для
обозначения <<истины>>, когда нет более подходящего значения.
Когда говорится, что функция <<возвращает \emph{ложь}>> или <<\emph{ложна}>>
в некоторых случаях, это значит, что она возвращает {\false}.
Однако если говорится, что функция <<возвращает \emph{истину}>> или
<<\emph{истинна}>> в некоторых случаях, это значит, что она возвращает некоторое
значение отличное от {\false}, но необязательно {\true}.

\subsection{Вычисление, Раскрытие и Равенство}

Выполнение Lisp кода называется \emph{вычисление}, так как выполнение части кода
обычно возвращает некоторый объект данных, называемый \emph{значением}, созданным
этим кодом. Для обозначения вычисления в примерах используется символ \EV\ .
Например,
\begin{lisp}
(+ 4 5) \EV\ 9
\end{lisp}
означает <<результатом вычисления кода \cd{(+ 4 5)} является (или будет, или был)
\cd{9}>>.

Символ \EX\ используется в примерах для обозначения раскрытия макросов.
Например,
\begin{lisp}
(push x v) \EX\ (setf v (cons x v))
\end{lisp}
означает <<результатом раскрытия макроса формы \cd{(push x v)}
является \cd{(setf v (cons x v))}>>. Это подразумевает, что две части кода
делают одно и то же действие; вторая часть кода является определением того, что
делает первая часть.

Символ \EQ\ используется в примерах для обозначения эквивалентности
(тождественности).
Например,
\begin{lisp}
(gcd x (gcd y z)) \EQ\ (gcd (gcd x y) z)
\end{lisp}
означает <<значение и побочные эффекты вычисления формы \cd{(gcd x (gcd y z))} всегда
являются такими же, как и значение и побочные эффекты \cd{(gcd (gcd x y) z)} для любых
значений переменных \cdf{x}, \cdf{y} и \cdf{z}>>.
Это подразумевает, что две части кода делают одинаковые вещи. Однако ни одна из
них не определяет другую напрямую --- так, как это делается при раскрытии макроса.

\subsection{Ошибки}
\label{INTRO-ERRORS}

Когда в книге для некоторых возникающих ситуаций указывается, что <<это
ошибка>>, это значит: 

\begin{itemize}
\item Корректная Common Lisp программа не должна вызывать данную ситуацию.

\item Если данная ситуация случилась, побочные эффекты и результаты получатся совершенно неопределенными по спецификации Common Lisp'a.

\item От реализации Common Lisp'а не требуется обнаруживать такие ошибки.
 Конечно, разработчикам рекомендуется реализовывать детектирование подобных ошибок,
  когда это необходимо.
\end{itemize}

Имеется в виду не то, что какая-то реализация может и не задать побочные эффекты и
результаты для данных ситуаций, а иное --- то, что программа, полностью
соотвествующая спецификации Common Lisp'а, не должна быть зависима от подобных
побочных эффектов и результатов.

Однако в некоторых ситуациях, если это обозначено в книге,
<<\emph{сигнализируется} ошибка>>, и это значит: 
\begin{itemize}
\item Если данная ситуация случилась, будет сигнализирована ошибка
(см. \cdf{error} и \cdf{cerror}).

\item Корректная Common Lisp программа может полагаться на тот факт, что ошибка будет сигнализирована.

\item Каждая реализация Common Lisp'а должна детектировать такую ошибку.
\end{itemize}

В местах, где встречаются выражения <<должен>> быть или <<не должен>> быть, или что-то
<<не может>> быть, подразумевается, что если указанные условия не выполняются, то <<это ошибка>>. Например: если аргумент <<должен
быть символом>>, а аргумент не символ, тогда <<это
ошибка>>. Во всех случаях, где ошибка \emph{сигнализируется}, всегда явно
используется слово <<сигнализируется (генерируется)>>.

\begin{table}[t]
\caption{Образец описания функций}
\label{Sample-Function-Description}
\begingroup\normalsize
\begin{defun}*[Функция]
sample-function arg1 arg2 &optional arg3 arg4

Функция \cdf{sample-function} складывает вместе \emph{arg1} и \emph{arg2} и
полученную сумму умножает на \emph{arg3}. Если \emph{arg3} не задан или равен
{\nil}, умножения не производится. \cdf{sample-function} затем возвращает список,
в котором первый элемент содержит результат, а второй элемент равен \emph{arg4}
(который по умолчанию равен символу \cdf{foo}).
Например:
\begin{lisp}
(sample-function 3 4) \EV\ (7 foo) \\
(sample-function 1 2 2 'bar) \EV\ (6 bar)
\end{lisp}
В целом,
\cd{(sample-function \emph{x} \emph{y})} \EQ\ \cd{(list (+ \emph{x} \emph{y}) 'foo)}.
\end{defun}
\endgroup
\vskip\ruletonoteskip
\hrule
\vskip\ruletonoteskip\null
\caption{Образец описания переменной}
\label{Sample-Variable-Description}
\begingroup\normalsize
\begin{defun}*[Переменная]
*sample-variable*

Переменная \cd{*sample-variable*} задаёт, сколько раз специальная
форма \cdf{sample-special-form} должна выполняться. Значение должно быть всегда
неотрицательным числом или {\nil} (что значит, выполнение бесконечно много
раз). Начальное значение \cd{0} (означает отсутствие выполнения).
\end{defun}
\endgroup
\vskip\ruletonoteskip
\hrule
\vskip\ruletonoteskip\null
\caption{Образец описания константы}
\label{Sample-Constant-Description}
\begingroup\normalsize
\begin{defun}*[Константа]
sample-constant

Именованная константа \cdf{sample-constant} имеет своим значением число с плавающей точкой, равное высоте экрана в ярдах, умноженной на логарифм по основанию 2 от общего
объёма в байтах, занимемого реализацией Common Lisp'а на диске.
\end{defun}
\endgroup
\end{table}

\begin{table}[t]
\caption{Образец описания специальной формы}
\label{Sample-Special-Form-Description}

\begingroup\normalsize
\begin{defspec}*
sample-special-form [name] ({var}*) {form}+

Производит вычисление каждой формы в последовательности, как неявный \cdf{progn},
и делает это столько раз, сколько обозначено в глобальной
переменной \cd{*sample-variable*}. Каждая переменная \emph{var} связывается и
инициализируется значением \cd{43} перед первой итерацией, и
освобождается после последней итерации.
Имя \emph{name}, если задано, может быть использовано в \cdf{return-from} форме
для преждевременного выхода из цикла. Если цикл завершился
нормально, \cdf{sample-special-form} возвращает {\nil}.
Например:
\begin{lisp}
(setq *sample-variable* 3) \\
(sample-special-form {\emptylist} \emph{form1} \emph{form2})
\end{lisp}
Здесь вычисляется \emph{form1}, \emph{form2}, \emph{form1},
\emph{form2}, \emph{form1}, \emph{form2} в указанном порядке.  
\end{defspec}
\endgroup
\vskip\ruletonoteskip
\hrule
\vskip\ruletonoteskip\null
\caption{Образец описания макроса}
\label{Sample-Macro-Description}
\begingroup\normalsize
\begin{defmac}*
sample-macro var <declaration* | doc-string> {tag | statement}*

Вычисляет выражения как тело \cdf{prog} с переменной \emph{var} связанной со
значением \cd{43}.
\begin{lisp}
(sample-macro x (return (+ x x))) \EV\ 86 \\
(sample-macro \emph{var} . \emph{body}) \EX\ (prog ((\emph{var} 43)) . \emph{body})
\end{lisp}
\end{defmac}
\endgroup
\end{table}

\subsection{Описания функций и других объектов}
\label{FUNCTION-HEADER-NOTATION-SECTION}

Функции, переменные, именованные константы, специальные формы и макросы
описываются с помощью особого типографского формата.
Таблица~\ref{Sample-Function-Description} показывает способ, которым
документируются Common Lisp функции.
Первая строка определяет имя функции, способ передачи аргументов, и тот факт,
что это функция.
Если функция принимает много аргументов, тогда имена аргументов могут быть
разнесены на две или три строки.
Параграф, следующий за этим стандартным заголовком, разъясняет определение и
использование данной функции, а зачастую также предоставляет примеры или связанные
функции.

Иногда две и более связанных функций даются в одном комбинированном
описании. В такой ситуации заголовки для всех функций отображаются совместно, с
последующим описанием.

Текущий код (включая текущие имена функций) представляется в данном
шрифте: \cd{(cons a b)}.
Имена, встречающиеся в частях кода (метапеременные) пишутся \emph{наклонным
шрифтом}. В описании функции имена параметров предоставляются в наклонном
шрифте. Слово \cd{\&optional} в списке параметров указывает на то, что все
последующие аргументы являются необязательными. Значения по умолчанию для
параметров описываются в последующем тексте. Список параметров может также
включать \cd{\&rest}, указывающий на возможность бесконечного количества
аргументов, или \cd{\&key}, указывающий на то, что допустимы именованные аргументы.
(Синтаксис \cd{\&optional}/\cd{\&rest}/\cd{\&key} и в самом деле используется для
этих целей при определении функций Common Lisp).

Таблица~\ref{Sample-Variable-Description} показывает способ, с помощью которого
документируются глобальные переменные. Первая строка определяет имя переменной
и тот факт, что это переменная.  Исключительно для удобства было принято
соглашение, что все глобальные переменные Common Lisp'а имеют имена,
начинающиеся и заканчивающиеся звёздочкой (asterisk).

Таблица~\ref{Sample-Constant-Description} отображает способ, с помощью которого
документируются константы. Первая строка определяет имя константы и тот факт, что это
константа.
(Константа является просто глобальной переменной за исключением того, что
 при попытке связывания этой переменной с другим значением возникает ошибка.)

Таблицы~\ref{Sample-Special-Form-Description} и~\ref{Sample-Macro-Description}
показывают документирование специальных форм и макросов, предназначения которых
тесно связаны.
Они очень сильно отличаются от функций.
Функции вызываются в соответствии с одним определённым неизменным механизмом.
Синтаксис \cd{\&optional}/\cd{\&rest}/\cd{\&key} задаёт то, как функция
внутренне использует свои аргументы, но не влияет на механизм вызова.
В отличие от этого, каждая специальная форма или макрос может иметь свой
особенный, индивидуальный механизм. Синтаксис Common Lisp'а задаётся и
расширяется с помощью специальных форм и макросов.

В описании специальных форм или макросов слова, записанные наклонным шрифтом, обозначают
соответствующую часть формы, которая вызывает специальную форму или макрос.
Круглые скобки означают сами себя, и должны быть указаны как есть при
вызове специальной формы или макроса.
Квадратные скобки, фигурные скобки, звездочки, знаки плюса и вертикальные скобки
является метасинтаксическими знаками.
Квадратные скобки,
[ и ], показывают, что заключённое в них выражение является
необязательным (может встречаться ноль или один раз в данном месте); квадратные
скобки не должны записываться в коде.
Фигурные скобки, \{ и \}, просто отображают заключённое в них
выражение, однако после закрывающей скобки может следовать звёздочка, * или
знак плюс +. Звёздочка показывает, что выражение в скобках может
встречаться НОЛЬ и более раз, тогда как плюс показывает, что выражение может
встречаться ОДИН и более раз. Внутри скобок, может использоваться вертикальная
черта |, она разделяет взаимоисключаемые элементы выбора.
В целом, запись \Mstar{x} значит, что \emph{x} может встречаться ноль и
более раз, запись \Mplus{x} значит, что \emph{x} может встречаться один и
более раз, и запись \Mopt{x} значит, что \emph{x} может встречаться ноль или
один раз. Такие записи также используются для описания выражений в стиле БНФ,
как в таблице~\ref{NUMBER-SYNTAX-TABLE}.

Двойные скобки, [[ и ]], показывают, что может использоваться любое количество
альтернатив, перечисленных в скобках, и в любом порядке, но каждая альтернатива
может использоваться не более одного раза, если только за ней нет звёздочки.
Например,
\begin{tabbing}
\emph{p} \Mchoice{x {\Mor} \Mstar{y} {\Mor} z} \emph{q}
\end{tabbing}
означает, что максимум один \emph{x}, любое количество \emph{y}, и максимум один
\emph{z} могут в любом порядке использоваться между обязательными \emph{p} и
\emph{q}.

Стрелочка вниз, \Mind{}, показывает, что данная форма будет раскрываться
ниже. Это делает запись \Mchoice{~} более читаемой. Если \emph{X} является
некоторым нетерминальным символом, стоящим слева в некоторой БНФ форме, правая
ее часть должна быть подставлена вместо символа \Mind{X} во всех случаях его
использования. Таким образом, два фрагмента

\begin{tabbing}
\emph{p} \Mchoice{\Mind{xyz-mixture}} \emph{q} \\
\emph{xyz-mixture} ::= \emph{x\/} {\Mor} \Mstar{y} {\Mor} \emph{z\/}
\end{tabbing}
вместе составляют эквивалент для предыдущего примера.

В последнем примере в таблице~\ref{Sample-Macro-Description}, рассматривается
использование записи с точкой. Точка, встречающаяся в выражении
\cd{(sample-macro \emph{var} . \emph{body})}, означает то, что имя \emph{body}
является списком форм, а не одиночной формой в конце списка. Эта запись 
часто используется в примерах.

В заглавной строке в таблице~\ref{Sample-Macro-Description}, запись \Mchoice{~}
означает, что может указываться любое количество определений (declaration), 
но максимум одна строка документации doc-string (которая может указываться перед, после,
или между определениями).

\subsection{Лисповый считыватель}

Термин <<Лисповый считыватель (читатель лиспового кода)>> относится не к вам,
читатель этой книги, и не к какому-либо человеку, читающему код на Lisp'е, а к
Lisp процедуре, которая называется \cdf{read}. Она читает символы из входного
потока и интерпретирует их с помощью парсинга как представления Lisp объектов.

\subsection{Обзор синтаксиса}

В Common Lisp'е некоторые строковые символы используется в определённых целях. Полное
описание синтаксиса можно прочесть в главе~\ref{IO}, но небольшой обзор здесь
может быть также полезен:
\begin{indentdesc}{1.2pc}
\item[\cd{(}]
Левая круглая скобка начинает список элементов. Список может содержать любое
количество элементов, включая ноль элементов (пустой список). Списки могут быть
вложенными. Например, \cd{(cons (car x) (cdr y))} список из трёх элементов, в
котором два последних также являются списками.

\item[\cd{)}] Правая круглая скобка завершает список элементов.

\item[\cd{\Xquote}] Одинарная кавычка (апостроф) с последующим выражением
  \emph{form} является сокращением для \cd{(quote \emph{form})}.

\item[\cd{;}] Точка с запятой обозначает комментарий. Она и все символы после
неё до конца строки игнорируются.

\item[\cd{"}] Двойная кавычка окружает символьные строки: \\
\cd{"This is a thirty-nine-character string."}

\item[\cd{{\Xbackslash}}] Обратная наклонная черта является экранирующим
символом. Она показывает, что следующий символ считывается как буква, и не несет
того синтаксического смысла, который приписывается ему правилами Common
Lisp'а. Например, \cd{A{\Xbackslash}(B} означает символ, имя которого содержит
три знака: \cdf{A}, \cd{(} и \cdf{B}. Подобным образом \cd{"{\Xbackslash}""}
означает строку, которая содержит один знак - двойную кавычку. Первая и
последняя двойные кавычки обозначают начало и конец строки. Обратная наклонная
черта обозначает, что вторая двойная кавычка будет интерпретирована как знак, а
не как синтаксическая конструкция для обозначения начала и конца строки.

\item[\cd{|}] Вертикальные черты используются попарно, для окружения имени (или
части имени) символа, которое содержит много специальных знаков. Это примерно равнозначно
тому, что перед каждым из них ставилась бы обратная косая
черта. Например, \cd{|A(B)|}, \cd{A|(|B|)|}
и \cd{A{\Xbackslash}(B{\Xbackslash})} все означают одно и то же имя.

\item[\cd{\#}] Знак решётки (диез) обозначает начало сложной синтаксической
конструкции. Символ после решетки определяет последующий синтаксис.
Например, \cd{\#o105} означает $105_{8}$ (105 в восьмеричной системе счисления);
\cd{\#x105} означает $105_{16}$ (105 в шестнадцатеричной системе счисления);
\cd{\#b1011} означает $1011_{2}$ (1011 в двоичной системе счисления);
\cd{\#{\Xbackslash}L} определяет строковый символ \cdf{L}; и \cd{\#(a b c)}
обозначает вектор из трёх элементов \cdf{a}, \cdf{b} и \cdf{c}. В частности, важным
случаем является то, что \cd{\#'\emph{fn}} означает \cd{(function \emph{fn})}, на
манер использования одинарной кавычки \cd{'\emph{form}}, обозначающей \cd{(quote
\emph{form})}.

\item[\cd{{\Xbq}}] Обратная одинарная кавычка показывает, что следуемое
выражение является шаблоном, который может содержать запятые. Синтаксис этой
обратной кавычки представляет программу, которая может создавать структуры
данных в соответствии с шаблоном.

\item[\cd{,}] Запятые используются внутри конструкции с обратной кавычкой.

\item[\cd{:}] Двоеточие используется для обозначения принадлежности символа к пакету. Например, \cd{network:reset} показывает, что символ с
именем \cdf{reset} принадлежит пакету \cdf{network}. Двоеточие в начале обозначает 
keyword (примечания переводчика: это не ключевое слово в понимании других языков 
программирования), символ, который вычисляется сам в себя. Двоеточие не является
частью печатаемого имени символа. 
Это все объясняется в главе~\ref{XPACK}; пока вы её не прочли, просто держите в
голове, что символ с двоеточием в начале являлется константой, которая
вычисляется сама в себя.
\end{indentdesc}

Квадратные и фигурные скобки, вопросительный и восклицательные знаки,
(\cd{{\Xlbracket}}, \cd{{\Xrbracket}}, \cd{{\Xlbrace}}, \cd{{\Xrbrace}}, \cd{?}
и \cd{!}) не используются ни для каких целей в стандартном синтаксисе Common
Lisp. Эти символы явно зарезервированы для пользователей, преимущественно для
использования в качестве \emph{макро-символов} для пользовательских расширений
синтаксиса (см. раздел~\ref{MACRO-CHARACTERS-SECTION}).

В Common Lisp есть сущность \cdf{readtable-case}, которая позволяет именам
символов быть регистрозависимыми путем некоторых настроек. Однако, поведение по
умолчанию такое, как описано в предыдущем параграфе. В любом случае, внутреннее
представление описанных в книге имен символов состоит из букв в верхнем
регистре.  \fi
       % General info, notational conventions
%Part{Dtypes, Root = "CLM.MSS"}
% Chapter of Common Lisp Manual.  Copyright 1984, 1988, 1989 Guy L. Steele Jr.

\clearpage\def\pagestatus{FINAL PROOF}

\chapter{Data Types Типы данных}
\label{DTYPES}

Common Lisp provides a variety of types of data objects.  It is important to
note that in Lisp it is data objects that are typed, not variables.
Any variable can have any Lisp object as its value.
(It is possible to make an explicit declaration that a variable will
in fact take on one of only a limited set of values.  However, such
a declaration may always be omitted, and the program will still run correctly.
Such a declaration merely constitutes advice from the user
that may be useful in gaining efficiency.  See \cd{declare}.)

Common Lisp предоставляет множество типов для объектов
данных. Необходимо подчеркнуть, что в Lisp'е типизированы данные,
а не переменные. Любая переменная может содержать данные любого
типа. (Можно указать явно, что некоторая переменная фактически
может содержать только один или конечное множество типов
объектов. Однако, такая декларация может быть опущена, и программа
будет выполняться корректно. Такая декларация содержит
рекомендация от пользователя, и это может быть полезным при
оптимизации. См. \cd{declare}.)

In Common Lisp, a data type is a (possibly infinite) set of
Lisp objects.  Many Lisp objects belong to more than one
such set, and so it doesn't always make sense to ask what is \emph{the} type
of an object; instead, one usually asks only whether an object belongs
to a given type.  The predicate \cd{typep} may be used to ask
whether an object belongs to a given type,
and the function \cd{type-of} returns \emph{a} type
to which a given object belongs.

В Common Lisp'е тип данных является (возможно бесконечным)
множеством Lisp объектов. Многие объекты Lisp'а принадлежат к
более чем одному множеству типов, так что иногда не имеет смысла
спрашивать тип объекта; вместо этого задается вопрос о
принадлежности объекта к нужному типу. Предикат \cd{typep} может
использоваться для определения принадлежности объекта к заданному
типу, а функция \cd{type-of} возвращает тип, к которому
принадлежит заданный объект.

The data types defined in Common Lisp are arranged into a hierarchy (actually
a partial order) defined by the subset relationship.
Certain sets of objects, such as the set of numbers or the
set of strings, are interesting enough to deserve labels.
Symbols are used for most
such labels (here, and throughout this book, the word ``symbol''
refers to atomic symbols, one kind of Lisp object,
elsewhere known as literal atoms).  See chapter~\ref{DTSPEC}
for a complete description of type specifiers.

Типы данных в Common Lisp сложены в иерархию (фактически в порядке
убывания объема) определенную отношениями подмножеств. Несомненно
множества объектов, такие как множество чисел и множество строк
заслуживают идентификаторы. Для многих этих идентификаторов
используются символы (здесь и далее, слово <<символ>> ссылается на
тип Lisp объектов символ, известный также как literal
atom). См. главу~\ref{DTSPEC} подробно описывающую определения
типов. 

The set of all objects is specified
by the symbol {\true}.  The empty data type, which contains no objects, is
denoted by {\nil}.

Множество все объектов определяется символом {\true}. Пустой тип
данных, который не содержит объектов обозначается с помощью
{\nil}. 

\begin{obsolete}
A type called \cd{common} encompasses all the data
objects required by the Common Lisp language.  A Common Lisp implementation
is free to provide other data types that are not subtypes of \cd{common}.
\end{obsolete}

\begin{newer}
X3J13 voted in March 1989
\issue{COMMON-TYPE}
to remove the type \cd{common} (and the predicate \cd{commonp})
from the language, on the grounds that it has
not proved to be useful in practice and that it could be difficult to redefine in the
face of other changes to the Common Lisp type system (such as the introduction
of CLOS classes).
\end{newer}

The following categories of Common Lisp objects are of particular interest:
numbers, characters, symbols, lists, arrays, structures, and functions.
There are others as well.
Some of these categories
have many subdivisions.  There are also standard types defined to
be the union
of two or more of these categories.  The categories listed above, while they
are data types, are neither more nor less ``real'' than other data types;
they simply constitute a particularly useful slice across
the type hierarchy for expository purposes.

Следующие категории объектов Common Lisp'а в особенности
интересны: числа (numbers), знаки (characters), символы (symbols),
списки (lists), массивы (arrays), структуры (structures) и функции
(functions). Другие типы тоже, конечно, интересны. Некоторые из
этих категорий имеют много подразделов. Так же есть стандартные
типы, которые определены, как объединение двух и более данных
категорий. Вышеупомянутые категории, являясь типами объектов, are
neither more nor less <<real>> than other data types; они просто
составляют объединения типов для наглядности. 

Here are brief descriptions of various Common Lisp data types.
The remaining sections of this chapter go into more detail
and also describe notations for objects
of each type.  Descriptions of Lisp functions that operate
on data objects of each type appear in later chapters.

Вот краткое изложение о различных Common Lisp'овых типах
данных. Оставшиеся разделы данной главы рассматривают типы более
детально, а также описывают нотации для объектов для каждого
типа. Описание Lisp'овых функций, что оперируют объектами данных
каждого типа будет даваться в следующих главах. 

\begin{itemize}
\item
\emph{Numbers} are provided in various forms and representations.
Common Lisp provides a true integer data type: any integer,
positive or negative, has in principle a representation as a
Common Lisp data object, subject only to total memory limitations (rather than
machine word width).
A true rational data type is provided: the quotient of two integers,
if not an integer, is a ratio.
Floating-point numbers of various ranges and precisions are also
provided, as well as
Cartesian complex numbers.

\item
\emph{Числа} имеют различные формы и представления. Common Lisp
предоставляет целочисленный (integer) тип данных: любое целое
число, положительное или отрицательное ограничено размерами памяти
(преимущественно равными ширине машинного слова). Также
предоставляется рациональный или дробный (rational) тип данных:
это отношение двух целых чисел, не являющееся целым числом. Также
предоставляются числа с плавающей точкой различных интервалов и
точностей. И наконец в языке также есть комплексные числа. 

\item
\emph{Characters} represent printed glyphs such as letters
or text formatting operations.  Strings are one-dimensional
arrays of characters.
Common Lisp provides for a rich character set, including ways to
represent characters of various type styles.

\item
\emph{Строковые} символы представляют печатные символы такие, как
буквы или управляющие форматированием символы. Строки являются
одномерными массивами символов. Common Lisp предоставляет богатое
множество символов, включая пути для представления различных
стилей печати. 

\item
\emph{Symbols} (sometimes called \emph{atomic symbols} for emphasis
or clarity) are named data objects.  Lisp provides machinery
for locating a symbol object, given its name (in the form
of a string).  Symbols have \emph{property lists}, which in effect
allow symbols to be treated as record structures with an extensible
set of named components, each of which may be any Lisp object.
Symbols also serve to name functions and variables within programs.

\item
\emph{Символы} (иногда называемые \emph{atomic symbols} для ясности)
являются именованными оpбъектами данных. Lisp предоставляет
механизм определеющий местоположение объекта символа по заданному
имени (в форме строки). У символов есть \emph{списки свойств},
которые фактически позволяют использовать символы в качестве
структур, с расширяемым множеством имен полей, каждое из которых
может быть любым Lisp объектом. Символы также служат для
именования функций и переменных в программе. 

\item
\emph{Lists} are sequences represented in the form of linked cells
called \emph{conses}.  There is a special object (the symbol {\nil})
that is the empty list.  All other lists are built recursively by adding a new
element to the front of an existing list.  This is done by
creating a new \emph{cons}, which is an object having two components
called the \emph{car} and the \emph{cdr}.  The \emph{car} may hold anything,
and the \emph{cdr} is made to point to the previously existing list.
(Conses may actually be used completely generally as two-element
record structures, but their most important use is to represent
lists.)

\item
\emph{Списки} (прим. автора: те самые, из которых и заварилась вся
каша) это последовательности представленная в форме связанных
ячеек, называемых \emph{cons-ячейками}. Для обозначения пустого
списка служит специальный объект (обозначаемые символом
{\nil}). Все остальные списки создаются рекурсивно, с помощью
добавления новых элементов в начало существующего списка. Это
происходит так: создается новая cons-ячейка, которая является
объектом, имеющим два компонента, называемых \emph{car} и {\it
cdr}. Car может хранить, что угодно, а cdr создан для хранения
указателя на существующий ранее список. (Cons-ячейки могут
использоваться для хранения храненеия записи структуры из двух
элементов, но это не главное их предназначение.) 

\item
\emph{Arrays} are dimensioned collections of objects.
An array can have any non-negative number of dimensions and is indexed
by a sequence of integers.  A general array can have any Lisp object as
a component; other types of arrays are specialized for efficiency
and can hold only certain types of Lisp objects.
It is possible for two arrays, possibly with differing dimension information,
to share the same set of elements (such that modifying one array modifies
the other also) by causing one to be \emph{displaced} to the other.
One-dimensional arrays of any kind are called \emph{vectors}.
One-dimensional arrays of characters are called \emph{strings}.
One-dimensional arrays of bits (that is, of integers whose values are 0 or 1)
are called \emph{bit-vectors}.

\item
\emph{Массивы} - это n-мерные коллекции объектов. Массив может
иметь любое неотрицательное количество измерений и индексироваться
с помощью последовательности целых чисел. Общий тип массива может
содержать любой Lisp объект; другие типы массивов специализируются
для эффективности и могут содержать только определенные типы Lisp
объектов. Также существует возможность того, что два массива,
возможно с разным количеством измерений, указывают на одно и то же
подмножество объектов (если изменить первый массив, изменится и
второй). Это достигается с помощью указания для одного массива
\emph{быть относительным} для другого массива. Одномерные массивы
любого типа называются \emph{векторами (vectors)}. Одномерные
массивы строковых символов называются \emph{строки}. Одномерные
массивы битов (это целое число, которое может содержать 0 или 1)
называются \emph{битовыми векторами (bit-vectors)}. 

\item
\emph{Hash tables} provide an efficient way of mapping any
Lisp object (a \emph{key}) to an associated object.

\item
\emph{Хеш-таблицы} предоставляют эффективный путь для связывания
любого Lisp объекта (\emph{ключа}) с другим объектом (значением). 

\item
\emph{Readtables} are used to control the built-in expression parser
\cd{read}.

\item
\emph{Таблицы чтения (readtables)} используется для управления
парсером выражений \cd{read}. (прим. автора: это та знаменитая
штука для создания макроридеров для изменения синтаксиса языка) 

\item
\emph{Packages} are collections of symbols that serve as name spaces.
The parser recognizes symbols by looking up character sequences
in the current package.

\item
\emph{Пакеты} являются коллекциями символов и служат для разделения
пространств имен. Парсер распознает символы с помощью поиска
последовательностей строковых символов в текущем пакете. 

\item
\emph{Pathnames} represent names of files in a fairly implementation-independent
manner.  They are used to interface to the external file system.

\item
\emph{Pathnames} представляют имена файлов на кроссплатформенный
лад. Они используются для взаимодействия с внешней файловой
системой. 

\item
\emph{Streams} represent sources or sinks of data, typically characters
or bytes.  They are used to perform I/O, as well as for internal
purposes such as parsing strings.

\item
\emph{Потоки} представляют источники данных, обычно строковых
символов или байт. Они используются для ввода/вывода, а также для
внутренних нужд, например для парсинга строк. 

\item
\emph{Random-states} are data structures used to encapsulate the state
of the built-in random-number generator.

\item
\emph{Random-states} - это структуры данных, используемые для
хранения состояния встроенного генератора случайных чисел (ГСЧ).

\item
\emph{Structures} are user-defined record structures, objects that
have named components.  The \cd{defstruct} facility is used
to define new structure types.  Some Common Lisp implementations may
choose to implement certain system-supplied data types,
such as \emph{bignums}, \emph{readtables}, \emph{streams},
\emph{hash tables}, and \emph{pathnames}, as structures,
but this fact will be invisible to the user.

\item
\emph{Структуры} - это определенные пользователем объекты, имеющие
именованные поля. \cd{defstruct} используется для определения
новых типов структур. Некоторые реализации Common Lisp могут
реализовывать некоторые системные типы такие, как \emph{bignums},
\emph{readtables}, \emph{streams}, \emph{hash tables} и {\it
pathnames} как структуры, но фактически это не будет видно
пользователю. 
\end{itemize}

\begin{obsolete}
\begin{itemize}
\item
\emph{Functions} are objects that can be invoked as procedures;
these may take arguments and return values.  (All Lisp procedures
can be construed to return values and therefore every procedure is
a function.)
Such objects include \emph{compiled-functions} (compiled code objects).
Some functions are represented as a list whose \emph{car} is a particular
symbol such as \cd{lambda}.
Symbols may also be used as functions.
\end{itemize}
\end{obsolete}

\begin{newer}
X3J13 voted in June 1988 \issue{FUNCTION-TYPE}
to specify that symbols are not of type \cd{function},
but are automatically coerced to functions
in certain situations (see section~\ref{FUNCTION-TYPE-SECTION}).
\end{newer}

\begin{new}
X3J13 voted in June 1988
\issue{CONDITION-SYSTEM}
to adopt the Common Lisp Condition System,
thereby introducing a new category of data objects:

Для адаптации Системы Условий Common Lisp, вводятся следующие
категории объектов данных: 

\begin{itemize}
\item
\emph{Conditions} are objects used to affect control flow in certain
conventional ways by means of signals and handlers that intercept those signals.
In particular, errors are signaled by raising particular conditions,
and errors may be trapped by establishing handlers for those conditions.

\item
\emph{Условия (conditions)} - это объекты, используемые для
управления ходом выполнения программы, с помощью сигналов и
обработчиков этих самых сигналов. В частности, ошибки
сигнализируются с помощью генерации условия, и эти ошибки могут
быть обработаны с помощью установки обработчиков для генерируемых
условий.
\end{itemize}
\end{new}

\begin{new}
X3J13 voted in June 1988
\issue{CLOS}
to adopt the Common Lisp Object System,
thereby introducing additional categories of data objects:

Для адаптации Объектной Системы Common Lisp, вводятся следующие
категории объектов данных: 

\begin{itemize}
\item
\emph{Classes} determine the structure and behavior of other
objects, their \emph{instances}.  Every Common Lisp data object
belongs to some class.  (In some ways the CLOS class system is
a generalization of the system of type specifiers of the first edition of this book,
but the class system augments the type system rather than supplanting it.)

\item
\emph{Классы} определяют структуру и поведение других объектов,
являющихся \emph{экземплярами} данных классов. Каждый объект данных
принадлежит некоторому классу. 

\item
\emph{Methods} are chunks of code that operate on arguments
satisfying a particular pattern of classes.  Methods are
not functions; they are not invoked directly on arguments
but instead are bundled into generic functions.

\item
\emph{Методы} - это код, который оперирует аргументами, которые
соотвествуют некоторому шаблону. Методы не являются функциями; они
не вызваются напрямую, а объединяются в дженерик-функции (generic
functions). 

\item
\emph{Generic functions} are functions that contain, among other
information, a set of methods.  When invoked, a generic function
executes a subset of its methods.  The subset chosen for execution
depends in a specific way on the classes or identities of the arguments
to which it is applied.

\item
\emph{Generic функции} - это функции, которые содержат, кроме всего
прочего, множество методов. При вызове generic функция вызывает
подмножество ее методов. Подмножество для выполнения выделяется с
помощью определения классов аргументов и выбора им соответстующих
методов. 
\end{itemize}
\end{new}

These categories are not always mutually exclusive.
The required relationships among the various data types are
explained in more detail in section~\ref{DATA-TYPE-RELATIONSHIPS}.

Эти категории не всегда взаимоисключаемы. Указанные отношения
между различными типами данных более детально описано в
разделе~\ref{DATA-TYPE-RELATIONSHIPS}. 


\section{Numbers Числа}

Several kinds of numbers are defined in Common Lisp.
They are divided into \emph{integers}; \emph{ratios};
\emph{floating-point numbers}, with names provided for
up to four different floating-point representations; and \emph{complex numbers}.

В Common Lisp'е определены некоторые виды чисел. Они
подразделяются на \emph{целочисленные (integer)}; \emph{дробные
(ratios)}; \emph{плавающей точкой (floating-point)}, с
четырьмя видами пердставления и \emph{комплексные}.

\begin{newer}
X3J13 voted in March 1989 \issue{REAL-NUMBER-TYPE} to add the type \cd{real}.

X3J13 voted in March 1989 \issue{REAL-NUMBER-TYPE} добавляет тип \cd{real}.

The \cd{number} data type encompasses all kinds of
             numbers.  For convenience, there are names for some
             subclasses of numbers as well.  Integers and ratios are of
             type \cd{rational}.  Rational numbers and floating-point
             numbers are of type \cd{real}.  Real numbers and complex
             numbers are of type \cd{number}.

Числовой тип данных захватывает все числовые типы. Также для
удобства предоставлены имены для некоторых числовых
подтипов. Целые числа и дроби принадлежат к
\emph{рациональному}. Рациональные числа и с плавающей точкой к
\emph{действительному}. Действительные и комплексные к \emph{числовому} типу.

             Although the names of these types were chosen with the
             terminology of mathematics in mind, the correspondences
             are not always exact.  Integers and ratios model the
             corresponding mathematical concepts directly.  Numbers
             of type \cd{float} may be used to approximate real
             numbers, both rational and irrational.  The \cd{real} type
             includes all Common Lisp numbers that represent
             mathematical real numbers, though there are
             mathematical real numbers (irrational numbers)
             that do not have an exact Common Lisp representation.
             Only \cd{real} numbers may be ordered using the \cd{<}, \cd{>}, \cd{<=},
             and \cd{>=} functions.

Несмотря на то, что эти типы выбирались из математической
терминологии, соответствие не всегда полное. Модель целочисленных
(integers) и дробных (ratios) типов полностью совпадает с
математической. Числа с \emph{плавающей точкой (float)} могут
использоваться для аппроксимации действительных (real) чисел:
рациональных (rational) и иррациональных
(irrational). \emph{Действительный (real)} тип включает все Common Lisp
числе, что отображают действительные (real) математические числа,
однако для математических действительных (real) чисел
(иррациональных (irrational)) аналогии в Cоmmon Lisp'е нет. Только
\emph{действительные (real)} числа могут быть отсортированы с помощью
функций \cd{<}, \cd{>}, \cd{<=} и \cd{>=}. (Ох жеш FIXME). 

\beforenoterule
\begin{incompatibility}
The Fortran 77 standard defines the term
             \emph{real datum} to mean ``a processor approximation to the value
             of a real number.''  In practice the Fortran \emph{basic real} type
             is the floating-point data type that Common Lisp calls
             \cd{single-float}.  The Fortran \emph{double precision} type is
             Common Lisp's \cd{double-float}.  The Pascal \cd{real} data type is
             an ``implementation-defined subset of the real numbers.''  In
             practice this is usually a floating-point type, often what
             Common Lisp calls \cd{double-float}.

Примечание о совместимости: стандарт Fortran 77 определяет термин
\emph{действительное число}, как <<аппроксимация значения действительного
числа>>. На практике Fortran'овский \emph{базовый действительный} тип
соотносится с числом плавающей точкой, которое в Common Lisp
зовется \emph{single-float}. Fortran тип \emph{с двойной точностью} является
Common Lisp типом \emph{double-float}. Pascal'евский \emph{действительный} тип
данных является <<платформозависимым подмножеством действительных
чисел>>. На практике это обычно тип с плавающей точкой, то, что в
Common Lisp'е называется \emph{double-float}. 

             A translation of an algorithm written in Fortran or Pascal
             that uses \cd{real} data usually will use some appropriate
             precision of Common Lisp's \cd{float} type.  Some algorithms may
             gain accuracy or flexibility by using Common Lisp's
             \cd{rational} or \cd{real} type instead.

Трансляция алгоритмов написанные на Фортране или Паскале, что
используют \emph{действительные} числа, обычно соответствует необходимой
точности для Common Lisp'овых типов \emph{с плавающей точкой}
(float). Некоторые алгоритмы могут получить выигрыш или гибкость с
использованием Common Lisp'овых \emph{дробных (rational)} или
\emph{действительных (real)} типов. 
\end{incompatibility}
\afternoterule
\end{newer}

\subsection{Integers Целые числа}
\label{INTEGERS-SECTION}

\indexterm{integer}
The \cd{integer} data type is intended to represent mathematical integers.
Unlike most programming languages, Common Lisp in principle imposes no limit on
the magnitude of an integer; storage
is automatically allocated as necessary to represent large integers.

\emph{Целочисленный} тип даных предназначен для отображения
математических целых чисел. В отличие от большинства языков
программирования, Common Lisp принципиально не навязывает
ограничений на величину целого числа; место для хранения
выделяется автоматически по мере необходимости для отображения
больших чисел.

In every Common Lisp implementation there is a range of integers that are
represented more efficiently than others; each such integer is called a
\emph{fixnum}, and an integer that is not a fixnum is called a
\emph{bignum}.
Common Lisp is designed to hide this distinction as much as possible;
the distinction between fixnums and bignums is visible to
the user in only a few places where the efficiency of representation is
important.  Exactly which integers are
fixnums is implementation-dependent; typically they will be those
integers in the range $-2^{\hbox{\scriptsize\it n}}$ to $2^{\hbox{\scriptsize\it n}}-1$,
inclusive, for some \emph{n} not less than 15.
See \cd{most-positive-fixnum} and \cd{most-negative-fixnum}.

В каждой реализации Common Lisp'а есть интервал целых чисел,
которые хранятся более оптимально, чем другие; каждое такое число
называется \emph{fixnum}, и число не являющееся fixnum называется
\emph{bignum}. Common Lisp спроектирован так, чтобы скрыть различие так
сильно, как это возможно; различие между fixnums и bignums видимо
пользователю, только в тех местах, где важна эффективность работы
алгоритма. Какие числа являются fixnums зависит от реализации;
обычно это числа в интервале от $-2^{\hbox{\scriptsize\it n}}$ to
$2^{\hbox{\scriptsize\it n}}-1$, включительно, для некоторого n не меньше
15. См. \cd{most-positive-fixnum} и \cd{most-negative-fixnum}. 

\begin{new}
X3J13 voted in January 1989
\issue{FIXNUM-NON-PORTABLE}
to specify that \cd{fixnum} must be a supertype
of the type \cd{(signed-byte 16)}, and additionally that the value
of \cd{array-dimension-limit} must be a fixnum (implying that the implementor
should choose the range of fixnums to be large enough to accommodate the
largest size of array to be supported).

\cd{fixnum} должен быть супертипом для типа \cd{(signed-byte 16)},
и в дополнение к этому, значения \cd{array-dimension-limit} должно
принадлежать fixnum (реализаторы должны выбрать интервал fixnum,
чтобы в него можно было включить наибольший число поддерживаемых
измерений для массивов). 

\beforenoterule
\begin{rationale}
This specification allows programmers to declare variables in portable code
to be of type \cd{fixnum} for efficiency.  Fixnums are guaranteed to
encompass at least the set of 16-bit signed integers
(compare this to the data type \cd{short int} in the C programming language).
In addition, any valid array index must be a fixnum, and therefore variables
used to hold array indices (such as a \cd{dotimes} variable)
may be declared \cd{fixnum} in portable code.

Объяснение: Эта спецификация позволяет программистам объявлять
переменные в переносимом коде типа \cd{fixnum} для
эффективности. Fixnums гарантированно заключают в себе множество
знаковых 16-битных чисел чисел (это сравнимо с типом
данных \cd{short int} в языке программирования C). В дополнение к
всему, любой корректный индекс массива должен быть fixnum, и в
таком случае переменные, которые хранят индексы массива (например
переменная в \cd{dotimes}) могут быть объявлены как \cd{fixnum} в
переносимом коде. 
\end{rationale}
\afternoterule
\end{new}

Integers are ordinarily written in decimal notation, as a sequence
of decimal digits, optionally preceded by a sign and optionally followed
by a decimal point.
For example:

Целые числа обычно записываются в десятичном виде, как последовательность десятичных цифр, опционально с предшевствующим знаком и опционально с последующей точкой. Например:
\begin{lisp}
~~~~~~~~~~~~~~~~~~~~~~~~~~~~~~~~\=\kill
\>0~~~~~\';{\rm Zero} \\*
\>-0~~~~~\';{\rm This \emph{always} means the same as \cd{0}} \\*
\>+6~~~~~\';{\rm The first perfect number} \\
\>28~~~~~\';{\rm The second perfect number} \\
\>1024.~~~~~\';{\rm Two to the tenth power} \\*
\>-1~~~~~\';{\rm \(e^{\pi i}\)} \\*
\>15511210043330985984000000.~~~~~\';{\rm 25 factorial (25!), probably a bignum}
\end{lisp}

\begin{lisp}
~~~~~~~~~~~~~~~~~~~~~~~~~~~~~~~~\=\kill
\>0~~~~~\';{\rm Нуль} \\*
\>-0~~~~~\';{\rm Это \emph{всегда} значит то же, что и \cd{0}} \\*
\>+6~~~~~\';{\rm Первое совершенное число} \\
\>28~~~~~\';{\rm Второе совершенное число} \\
\>1024.~~~~~\';{\rm Два в десятой степени} \\*
\>-1~~~~~\';{\rm \(e^{\pi i}\)} \\*
\>15511210043330985984000000.~~~~~\';{\rm факториал от 25 (25!),
скорее все bignum}
\end{lisp}

\beforenoterule
\begin{incompatibility}
MacLisp and Lisp Machine Lisp normally assume that integers
are written in octal (radix-8) notation unless a decimal
point is present.
Interlisp assumes integers are written in decimal notation and uses a
trailing \cd{Q} to indicate octal radix; however, a decimal point,
even in trailing position, \emph{always} indicates a floating-point number.
This is of course consistent with Fortran.  Ada does not permit
trailing decimal points but instead requires them to be embedded.
In Common Lisp, integers written as described
above are always construed to be
in decimal notation, whether or not the decimal point is present;
allowing the decimal point to be present permits compatibility with
MacLisp.

Примечание о совместимости: точка в конце числа используется для
совместимости с MacLisp.
\end{incompatibility}
\afternoterule

\vskip 0pt plus 8pt%manual

Integers may be notated in radices other than ten.
The notation

Целые числа могут быть представлены с основаниями отличнымии от
десяти. Синтаксис: 
\begin{lisp}
\#\emph{nn}r\emph{ddddd}     {\rm or}     \#\emph{nn}R\emph{ddddd}
\end{lisp}
means the integer in radix-\emph{nn} notation denoted by the digits
\emph{ddddd}.  More precisely, one may write \cd{\#}, a non-empty sequence
of decimal digits representing an unsigned decimal integer \emph{n},
\cd{r} (or \cd{R}), an optional sign, and a sequence of radix-\emph{n}
digits, to indicate an integer written in radix \emph{n} (which must be
between 2 and 36, inclusive).  Only legal digits
for the specified radix may be used; for example, an octal number may
contain only the digits 0 through 7.  For digits above 9,
letters of the alphabet of either
case may be used in order.  Binary, octal, and
hexadecimal radices are useful enough to warrant the special
abbreviations \cd{\#b} for \cd{\#2r}, \cd{\#o} for \cd{\#8r}, and
\cd{\#x} for \cd{\#16r}.
For example:

означает, что целое число с основанием \emph{nn} определенное с
помощью цифр и букв \emph{ddddd}. Более точное описание:
символ \cd{\#}, непустая последовательность десятичных цифр
представляющих десятичное число \emph{n}, \cd{r} (или \cd{R}), опционально знак + или -, и последовательность цифр для заданной системы счисления (система счисления должна быть между 2 и 36, включительно). Только корректные символы для заданной системы счисления могут использоваться для задания числа; например для восьмеричного числа могут использоваться только цифры от 0 до 7 включительно. Для систем счисления больших десятичной, могут использоваться буквы алфавита в любом регистре в алфавитном порядке. Двоичные, восьмеричные и шестнадцатиричные основания можно использовать с помощью следующих аббревиатур: \cd{\#b} для \cd{\#2r}, \cd{\#o} для \cd{\#8r}, \cd{\#x} для \cd{\#16r}. Например:
\begin{lisp}
~~~~~~~~~~~~~~~~\=\kill
\>\#2r11010101~~~~~\';{\rm Another way of writing \cd{213} decimal} \\
\>\#b11010101~~~~~\';{\rm Ditto} \\
\>\#b+11010101~~~~~\';{\rm Ditto} \\
\>\#o325~~~~~\';{\rm Ditto, in octal radix} \\
\>\#xD5~~~~~\';{\rm Ditto, in hexadecimal radix} \\
\>\#16r+D5~~~~~\';{\rm Ditto} \\
\>\#o-300~~~~~\';{\rm Decimal \(-192\), written in base 8} \\
\>\#3r-21010~~~~~\';{\rm Same thing in base 3} \\
\>\#25R-7H~~~~~\';{\rm Same thing in base 25} \\
\>\#xACCEDED~~~~~\';{\rm 181202413, in hexadecimal radix}
\end{lisp}

\begin{lisp}
\>\#2r11010101~~~~~\';{\rm Другой способ определения
числа \cd{213}} \\
\>\#b11010101~~~~~\';{\rm то же самое} \\
\>\#b+11010101~~~~~\';{\rm то же самое} \\
\>\#o235~~~~~\';{\rm То же самое, в восьмеричной системе} \\
\>\#xD5~~~~~\';{\rm То же самое, в шестнадцатиричной системе} \\
\>\#16r+D5~~~~~\';{\rm То же самое} \\
\>\#o-300~~~~~\';{\rm Десятичное число -192, записанное восьмеричным числом} \\
\>\#3r-21010~~~~~\';{\rm То же самое, в троичное системе счисления} \\
\>\#25R-7H~~~~~\';{\rm То же самое с основанием 25} \\
\>\#xACCEDED~~~~~\';{\rm 181202413, в шестнадцатиричной системе}
\end{lisp}

\subsection{Ratios Дробные числа}

\indexterm{ratio}
\indexterm{rational}
A \emph{ratio} is a number representing the mathematical ratio
of two integers.  Integers and ratios collectively constitute
the type \cd{rational}.
The canonical representation of a rational number is as an
integer if its value is integral, and otherwise as the ratio of two
integers, the \emph{numerator} and \emph{denominator}, whose greatest
common divisor is 1, and of which the denominator is positive (and in
fact greater than 1, or else the value would be integral).
A ratio is notated with
\cd{/} as a separator, thus: \cd{3/5}.  It is possible to notate
ratios in non-canonical (unreduced) forms, such as \cd{4/6}, but the
Lisp function \cd{prin1} always prints the canonical form for a
ratio.

\emph{Дробное число} - это число отображающее математическое отношение
между двумя целыми числами. Целые и дробные числа вместе
составляют тип рациональных (rational) чисел. Образцовое
отображение дробных чисел - это целое число, если значение целое,
в противном случае это отношение двух целых чисел, \emph{числителя} и
\emph{знаменателя}, наибольший общий делитель которых единица, в котором
знаменатель положителен (и фактически больший чем единица, иначе
дробь является целым числом). Дробь записывается с помощью
разделителя \cd{/}, так: \cd{3/5}. Есть возможность
использовать нестандартную запись такую, как \cd{4/6}, но Lisp
функция \cd{prin1} всегда выводит дробь в стандартной форме. 

If any computation produces a result that is a ratio of
two integers such that the denominator evenly divides the
numerator, then the result is immediately converted to the equivalent
integer.  This is called the rule of \emph{rational canonicalization}.

Если какое-либо вычисление привело к результату, являющемуся
дробью двух целых чисел, где знаменатель делит числитель нацело,
тогда результат немедленно преобразуется в эквивалентное целое
число. Это называется правилом \emph{канонизации дробей}.

Rational numbers may be written as the possibly signed quotient of
decimal numerals: an optional sign followed by two non-empty sequences of
digits separated by a \cd{/}.  This syntax may be described as
follows:

Дробные числа могут быть записаны так: опционально знак + или -,
за ним следуют две непустые последовательности цифр разделенных с
помощью \cd{/} . Такой синтаксис может быть описан так:

\begin{tabbing}
\emph{ratio} ::= \Mopt{\emph{sign}} \Mplus{\emph{digit}} \cd{/} \Mplus{\emph{digit}}
\end{tabbing}

The second sequence may not consist
entirely of zeros.
For example:

Вторая последовательность не может состоять только из
нулей. Например:  
\begin{lisp}
2/3~~~~~~~~~~~~~~~~~~~~;{\rm This is in canonical form} \\
4/6~~~~~~~~~~~~~~~~~~~~;{\rm A non-canonical form for the same number} \\
-17/23~~~~~~~~~~~~~~~~~;{\rm A not very interesting ratio} \\
-30517578125/32768~~~~~;{\rm This is \((-5/2)^{15}\)} \\
10/5~~~~~~~~~~~~~~~~~~~;{\rm The canonical form for this is \cd{2}}
\end{lisp}

\begin{lisp}
2/3~~~~~~~~~~~~~~~~~;{\rm Это каноническая запись} \\
4/6~~~~~~~~~~~~~~~~~;{\rm Это неканоническая запись предыдущего числа} \\
-17/23~~~~~~~~~~~~~~;{\rm Не очень интересная дробь} \\
-30517578125/32768~~;{\rm Это \((-5/12)^{15}\)} \\
10/5~~~~~~~~~~~~~~~~;{\rm Это каноническая запись для \cd{2}}
\end{lisp}

To notate rational numbers in radices other than ten,
one uses the same radix specifiers
(one of \cd{\#\emph{nn}R}, \cd{\#O}, \cd{\#B}, or \cd{\#X}) as for integers.
For example:

Для задания дробей в системе счисления отличной от десятичной, необходимо
использовать спецификатор основания (один из \cd{\#\emph{nn}R}, \cd{\#O}, \cd{\#B} или \cd{\#X}) как и для
целых чисел. Например:

\begin{lisp}
\#o-101/75~~~~~~~~~~;{\rm Octal notation for \cd{-65/61}} \\
\#3r120/21~~~~~~~~~~;{\rm Ternary notation for \cd{15/7}} \\
\#Xbc/ad~~~~~~~~~~~~;{\rm Hexadecimal notation for \cd{188/173}} \\
\#xFADED/FACADE~~~~~;{\rm Hexadecimal notation for \cd{1027565/16435934}}
\end{lisp}

\begin{lisp}
\#o-101/75~~~~~~~~~;{\rm Восьмеричная запись для \cd{-65/61}} \\
\#3r120/21~~~~~~~~~;{\rm Третичная запись для \cd{15/7}} \\
\#Xbc/ad~~~~~~~~~~~;{\rm Шестнадцатиричная запись для \cd{188/173}} \\
\#xFADED/FACADE~~~~;{\rm Шестнадцатиричная запись для \cd{1027565/16435934}} 
\end{lisp}

\subsection{Floating-Point Numbers Числа с плавающей точкой}

Common Lisp allows an implementation to provide one or more kinds of
floating-point number, which collectively make up the type \cd{float}.
Now a floating-point number is a (mathematical)
rational number of the form
$\emph{s} \cdot \emph{f} \cdot \emph{b}^{\hbox{\scriptsize\it e}-\hbox{\scriptsize\it p}}$,
where \emph{s} is $+1$ or $-1$, the \emph{sign};
\emph{b} is an integer greater than 1,
the \emph{base} or \emph{radix} of the representation;
\emph{p} is a positive integer,
the \emph{precision} (in base-\emph{b} digits) of the floating-point number;
\emph{f} is a positive integer between
$\emph{b}^{\,\hbox{\scriptsize\it p}-1}$ and $\emph{b}^{\,\hbox{\scriptsize\it p}}-1$ (inclusive),
the \emph{significand};
and \emph{e} is an integer, the \emph{exponent}.
The value of \emph{p} and the range of \emph{e}
depends on the implementation and on the type of floating-point number
within that implementation.
In addition, there is a floating-point zero;
depending on the implementation, there may also be a ``minus zero.''
If there is no minus zero, then \cd{0.0} and \cd{-0.0} are
both interpreted as simply a floating-point zero.

Common Lisp позволяет реализации предоставлять один и более типов чисел с
плавающе точкой, которые все вместе составляют тип \cd{float}.
Число с плавающей точкой является (математически) рациональным числом формы
$\emph{s} \cdot \emph{f} \cdot \emph{b}^{\hbox{\scriptsize\it e}-\hbox{\scriptsize\it p}}$,
где \emph{s} $+1$ или $-1$, является \emph{знаком};
\emph{b} целое число большее 1,
является \emph{основанием} для представления;
\emph{p} положительное целое, является \emph{точностью} (количество цифр по
основанию \emph{b}) числа с плавающей точкой;
\emph{f} положительное целое между $\emph{b}^{\,\hbox{\scriptsize\it p}-1}$ и
$\emph{b}^{\,\hbox{\scriptsize\it p}}-1$ (включительно), является мантиссой;
и \emph{e} целое число, является экспонентой.
Значение \emph{p} и интервал \emph{e} зависит от реализации, также может быть
<<минус ноль>>. Если <<минус ноль>> отсутствует, тогда \cd{0.0} и \cd{-0.0} оба
интепретируются, как ноль с плавающей точкой.

\beforenoterule
\begin{implementation}
The form of the above description should not be construed
to require the internal representation to be in sign-magnitude form.
Two's-complement and other representations are also acceptable.  Note
that the radix of the internal representation may be other than 2, as on
the IBM 360 and 370, which use radix 16; see
\cd{float-radix}.
\end{implementation}
\afternoterule

Floating-point numbers may be provided in a variety of precisions and sizes,
depending on the implementation.  High-quality floating-point
software tends to depend critically on the precise nature of the
floating-point arithmetic and so may not always be completely portable.
As an aid in writing programs that are
moderately portable, however, certain definitions are made here:

Числа с плавающей точкой могут предоставляться с различными точностями и
размерами, в зависимости от реализации. Высококачественные программы с
вычислениями с плавающей точкой зависят от того, какая точность предоставляется,
и не всегда могут быть полностью перенесимы. Для содействия по умеренной
переносимости программ, сделаны следующие определения:
\begin{itemize}
\item
A \emph{short} floating-point number (type \cd{short-float})
is of the representation of smallest
fixed precision provided by an implementation.

\item
\emph{Короткий} тип числа с плавающей точкой (тип \cd{short-float}) является
представлением числа с наименьшей фиксированной точностью, предоставляемого реализацией.

\item
A \emph{long} floating-point number (type \cd{long-float})
is of the representation of the largest fixed 
precision provided by an implementation.

\item
\emph{Длинный} тип числа с плавающей точкой (тип \cd{long-float}) является
представлением числа с наибольшей фиксированной точностью, предоставляемого реализацией.

\item
Intermediate between short and long formats are two others, arbitrarily
called \emph{single} and \emph{double} (types \cd{single-float} and \cd{double-float}).

\item
Промежуточными форматами между коротким и длинным форматами является два других
формата, называемых \emph{одинарный} и \emph{двойной} (типы \cd{single-float} и \cd{double-float}).
\end{itemize}

The precise definition of these categories is implementation-dependent.
However, the rough intent is that short floating-point numbers be
precise to at least four decimal places (but also have
a space- efficient representation);
single floating-point numbers, to at least seven decimal places;
and double floating-point numbers, to at least fourteen decimal places.
It is suggested that
the precision (measured in bits, computed as $p \log_2 b$)
and the exponent size (also measured in bits, computed as the base-2
logarithm of 1 plus the maximum exponent value) be at least as great
as the values in table~\ref{Floating-Format-Requirements-Table}.

Определение точности для этих категорий зависит от реализации. Однако, примерная
цель, что короткий тип с плавающий точкой должен содержать точность, как минимум
4 позиции после запятой (и также должен иметь эффективное представление в
памяти);
одинарный тип с плавающей точкой -- как минимум 7 знаков после запятой;
двойной тип с плавающей точкой -- как минимум 14 знаков после запятой.
Предполагается, что размер точности (измеряется в битах и рассчитывается как $p \log_2 b$) и экспоненты (измеряется в битах и рассчитывается как логарифм
с основанием 2 от (1 плюс максимальное значение экспоненты) должен быть как
минимум таким же большим как значения из таблицы~\ref{Floating-Format-Requirements-Table-ru}.

\begin{table}[t]
\caption{Recommended Minimum Floating-Point Precision and Exponent Size}
\label{Floating-Format-Requirements-Table}
\begin{tabular}{@{}lll@{}}
{Format\quad\quad}&{Minimum Precision\quad\quad}&{Minimum Exponent Size} \\ \hlinesp
Short&13 bits&5 bits \\
Single&24 bits&8 bits \\
Double&50 bits&8 bits \\
Long&50 bits&8 bits
\end{tabular}
\end{table}

\begin{table}[t]
\caption{Рекомендуемый размер для точности и экспоненты для типа с плавающей точкой}
\label{Floating-Format-Requirements-Table-ru}
\begin{tabular}{@{}lll@{}}
{Формат\quad\quad}&{Минимальная точность\quad\quad}&{Минимальный размер экспоненты} \\ \hlinesp
Короткое&13 бит&5 бит \\
Одинарное&24 бит&8 бит \\
Двойное&50 бит&8 бит \\
Длинное&50 бит&8 бит
\end{tabular}
\end{table}

Floating-point numbers are written in either decimal fraction
or computerized scientific notation: an optional sign,
then a non-empty sequence of digits with an embedded decimal point,
then an optional decimal exponent specification.
If there is no exponent specifier, then
the decimal point is required, and there must be digits
after it.
The exponent specifier consists of an exponent marker,
an optional sign, and a non-empty sequence of digits.
For preciseness, here is a modified-BNF description of floating-point
notation.

Числа с плавающей точкой записываются в двух формах десятичной дробью и
компьютеризированной научной записью: необязательный знак, затем непустая
последовательность цифр с встроенной точкой, затем необязательная
часть определения экспоненты.
Если определения экспоненты нет, тогда требуется точка, и после нее должны быть
цифры.
Определение экспоненты составляется из маркера экспоненты, необязательного знака
и непустой последовательности цифр.
Для ясности приведена БНФ для записи чисел с плавающей точкой.
\begin{tabbing}
\emph{floating-point-number} ::= \=\Mopt{\emph{sign}} \Mstar{\emph{digit}} {\it
decimal-point} \Mplus{\emph{digit}} \Mopt{\emph{exponent}} \\*
\>\hbox to 0pt{\hss\Mor~}\Mopt{{\it
sign}} \Mplus{\emph{digit}} \Mopt{\emph{decimal-point} \Mstar{\emph{digit}}} {\it
exponent} \\
\emph{sign} ::= \cd{+} {\Mor} \cd{-} \\
\emph{decimal-point} ::= \cd{.} \\
\emph{digit} ::= \cd{0} {\Mor} \cd{1} {\Mor} \cd{2} {\Mor} \cd{3} {\Mor} \cd{4}
         {\Mor} \cd{5} {\Mor} \cd{6} {\Mor} \cd{7} {\Mor} \cd{8} {\Mor} \cd{9}\\
\emph{exponent} ::= \emph{exponent-marker} \Mopt{\emph{sign}} \Mplus{\emph{digit}}\\*
\emph{exponent-marker} ::= \cd{e} {\Mor} \cd{s} {\Mor} \cd{f}
{\Mor} \cd{d} {\Mor} \cd{l} {\Mor} \cd{E} {\Mor} \cd{S} {\Mor} \cd{F} {\Mor}
\cd{D} {\Mor} \cd{L}
\end{tabbing}

\begin{tabbing}
\emph{число-с-плавающей-точкой} ::= \=\Mopt{\emph{знак}} \Mstar{\emph{цифра}} {\it
точка} \Mplus{\emph{цифра}} \Mopt{\emph{экспонента}} \\*
\>\hbox to 0pt{\hss\Mor~}\Mopt{{\it
знак}} \Mplus{\emph{цифра}} \Mopt{\emph{точка} \Mstar{\emph{цифра}}} {\it
экспонента} \\
\emph{знак} ::= \cd{+} {\Mor} \cd{-} \\
\emph{точка} ::= \cd{.} \\
\emph{цифра} ::= \cd{0} {\Mor} \cd{1} {\Mor} \cd{2} {\Mor} \cd{3} {\Mor} \cd{4}
         {\Mor} \cd{5} {\Mor} \cd{6} {\Mor} \cd{7} {\Mor} \cd{8} {\Mor} \cd{9}\\
\emph{экспонента} ::= \emph{маркер-экспоненты} \Mopt{\emph{знак}} \Mplus{\emph{цифра}}\\*
\emph{маркер-экспоненты} ::= \cd{e} {\Mor} \cd{s} {\Mor} \cd{f}
{\Mor} \cd{d} {\Mor} \cd{l} {\Mor} \cd{E} {\Mor} \cd{S} {\Mor} \cd{F} {\Mor}
\cd{D} {\Mor} \cd{L}
\end{tabbing}
If no exponent specifier is present, or if the exponent marker \cd{e}
(or \cd{E}) is used, then the precise format to be used is not
specified.  When such a representation is read and
converted to an internal floating-point data object, the format specified
by the variable \cd{*read-default-float-format*} is used; the initial
value of this variable is \cd{single-float}.

Если определение экспоненты отсутствует или если используется маркер
экспоненты \cd{e} (или \cd{E}), тогда используемые формат точности не
задан. Когда такое представление считывается и конвертируется во внутренний
формат объекта числа с плавающей точкой, формат задается с помощью
переменной \cd{*read-default-float-format*}; первоначальное значение данной
переменной \cd{single-float}.

The letters \cd{s}, \cd{f}, \cd{d}, and \cd{l} (or their
respective uppercase equivalents) explicitly specify the
use of \emph{short}, \emph{single}, \emph{double}, and \emph{long} format, respectively.

Буквы  \cd{s}, \cd{f}, \cd{d} и \cd{l} (или их эквиваленты в верхнем регистре)
явно задают использование типа: \emph{короткий}, \emph{одинарный}, \emph{двойной} и
\emph{длинный}, соответственно.

Examples of floating-point numbers:
\begin{lisp}
0.0~~~~~~~~~~~~~~~~~~~~~~~~~;{\rm Floating-point zero in default format} \\
0E0~~~~~~~~~~~~~~~~~~~~~~~~~;{\rm Also floating-point zero in default format} \\
-.0~~~~~~~~~~~~~~~~~~~~~~~~~;{\rm This may be a zero or a minus zero,} \\
~~~~~~~~~~~~~~~~~~~~~~~~~~~~; {\rm depending on the implementation} \\
0.~~~~~~~~~~~~~~~~~~~~~~~~~~;{\rm The \emph{integer} zero, not a floating-point zero!} \\
0.0s0~~~~~~~~~~~~~~~~~~~~~~~;{\rm A floating-point zero in \emph{short} format} \\
0s0~~~~~~~~~~~~~~~~~~~~~~~~~;{\rm Also a floating-point zero in \emph{short} format} \\
3.1415926535897932384d0~~~~~;{\rm A \emph{double}-format approximation to \(\pi\)} \\
6.02E+23~~~~~~~~~~~~~~~~~~~~;{\rm Avogadro's number, in default format} \\
602E+21~~~~~~~~~~~~~~~~~~~~~;{\rm Also Avogadro's number, in default format} \\
3.010299957f-1~~~~~~~~~~~~~~;{\rm \(\log_{10} 2\), in \emph{single} format} \\
-0.000000001s9~~~~~~~~~~~~~~;{\rm \(e^{\pi i}\) in \emph{short} format, the hard way}
\end{lisp}

Examples of floating-point numbers:
\begin{lisp}
0.0~~~~~~~~~~~~~~~~~~~~~~~~~;{\rm Ноль с плавающей точкой в формате по умолчанию} \\
0E0~~~~~~~~~~~~~~~~~~~~~~~~~;{\rm Также ноль с плавающей точкой в формате по умолчанию} \\
-.0~~~~~~~~~~~~~~~~~~~~~~~~~;{\rm Это может быть нулем или минус нулем} \\
~~~~~~~~~~~~~~~~~~~~~~~~~~~~; {\rm в зависимости от реализации} \\
0.~~~~~~~~~~~~~~~~~~~~~~~~~~;{\rm \emph{Целый} ноль, не с плавающей точкой!} \\
0.0s0~~~~~~~~~~~~~~~~~~~~~~~;{\rm Ноль с плавающей точкой в \emph{коротком} формате} \\
0s0~~~~~~~~~~~~~~~~~~~~~~~~~;{\rm Также ноль с плавающей точкой в \emph{коротком} формате} \\
3.1415926535897932384d0~~~~~;{\rm Аппроксиммация числе пи в \emph{двойном} формате} \\
6.02E+23~~~~~~~~~~~~~~~~~~~~;{\rm Число Авогадро в формате по умолчанию} \\
602E+21~~~~~~~~~~~~~~~~~~~~~;{\rm Также число Авогадро в формате по умолчанию} \\
3.010299957f-1~~~~~~~~~~~~~~;{\rm \(\log_{10} 2\), в \emph{одинарном} формате} \\
-0.000000001s9~~~~~~~~~~~~~~;{\rm \(e^{\pi i}\) в коротком формате}
\end{lisp}

\begin{new}%CORR
\emph{Notice of correction.}
The first edition unfortunately listed an incorrect value (\cd{3.1010299957f-1})
for the base-10 logarithm of 2.
\end{new}

The internal format used for an external representation depends only
on the exponent marker and not on the number of decimal digits
in the external representation.

Внутренний формат использует для внешнего представления в только от
маркера экспоненты и не учитывает количество знаков после запятой во внешнем
представлении. 

While Common Lisp provides terminology and notation sufficient
to accommodate four distinct floating-point formats,
not all implementations will have the means to support
that many distinct formats.
An implementation is therefore permitted to provide
fewer than four distinct internal floating-point formats,
in which case at least one of them will be ``shared''
by more than one of the external format names \emph{short}, \emph{single},
\emph{double}, and \emph{long} according to the following rules:

Тогда как Common Lisp предоставляет терминологию и систему обозначений для
включения 4 различных типов чисел с плавающей точкой, не все реализации будет
иметь намерения для поддержки такого большого количества типов.
Реализация разрешается предоставлять меньшее, чем 4, количество внутренних
форматов чисел с плавающей точкой, в таком случае как минимум один из этих типов
будет <<общим>> для более одного внешнего имени \emph{короткого}, \emph{одинарного},
\emph{двойного}, and \emph{длинного} в соответствии со следующими правилами:
\begin{itemize}
\item
If one internal format is provided, then it is considered to be
\emph{single}, but serves also as \emph{short}, \emph{double}, and \emph{long}.
The data types \cd{short-float},
\cd{single-float}, \cd{double-float}, and \cd{long-float} are
considered to be identical.  An expression such as \cd{(eql 1.0s0 1.0d0)}
will be true in such an implementation
because the two numbers \cd{1.0s0} and \cd{1.0d0} will
be converted into the same internal format and therefore be considered
to have the same data type, despite the differing external syntax.
Similarly, \cd{(typep 1.0L0 'short-float)} will be true in such
an implementation.
For output purposes all floating-point numbers are assumed to be
of \emph{single} format and thus will print using the
exponent letter \cd{E} or \cd{F}.

\item
If two internal formats are provided, then either of two correspondences
may be used, depending on which is the more appropriate:
\begin{itemize}
\item
One format is \emph{short}; the other is \emph{single} and serves also
as \emph{double} and \emph{long}.
The data types
\cd{single-float}, \cd{double-float}, and \cd{long-float} are
considered to be identical, but \cd{short-float} is distinct.
An expression such as \cd{(eql 1.0s0 1.0d0)}
will be false, but \cd{(eql 1.0f0 1.0d0)} will be true.
Similarly, \cd{(typep 1.0L0 'short-float)} will be false,
but \cd{(typep 1.0L0 'single-float)} will be true.
For output purposes all floating-point numbers are assumed to be
of \emph{short} or \emph{single} format.

\item
One format is \emph{single} and serves also as \emph{short};
the other is \emph{double} and serves also as \emph{long}.
The data types \cd{short-float} and \cd{single-float} are considered to be
identical, and the data types \cd{double-float} and \cd{long-float} are
considered to be identical.
An expression such as \cd{(eql 1.0s0 1.0d0)}
will be false, as will \cd{(eql 1.0f0 1.0d0)};
but \cd{(eql 1.0d0 1.0L0)} will be true.
Similarly, \cd{(typep 1.0L0 'short-float)} will be false,
but \cd{(typep 1.0L0 'double-float)} will be true.
For output purposes all floating-point numbers are assumed to be
of \emph{single} or \emph{double} format.
\end{itemize}

\item
If three internal formats are provided, then either of two correspondences
may be used, depending on which is the more appropriate:
\begin{itemize}
\item
One format is \emph{short}; another format is \emph{single}; and the third format is
\emph{double} and serves also as \emph{long}.  Similar constraints apply.

\item
One format is \emph{single} and serves also as \emph{short};
another is \emph{double}; and the third format is \emph{long}.
\end{itemize}
\end{itemize}

\beforenoterule
\begin{implementation}
It is recommended that an implementation
provide as many distinct floating-point formats as feasible,
using table~\ref{Floating-Format-Requirements-Table} as a guideline.
Ideally, short-format floating-point numbers should have an
``immediate'' representation that does not require heap allocation;
single-format
floating-point numbers should approximate IEEE proposed standard
single-format floating-point numbers; and double-format floating-point
numbers should approximate IEEE proposed standard double-format
floating-point numbers
\cite{IEEE-PROPOSED-FLOATING-POINT-STANDARD,IEEE-FLOATING-POINT-IMPL-GUIDE,IEEE-FLOATING-POINT-IMPL-GUIDE-ERRATA}.
\end{implementation}
\afternoterule


\subsection{Complex Numbers Комплексные числа}

Complex numbers (type \cd{complex})
are represented in Cartesian form, with a real part and an imaginary
part, each of which is a non-complex number (integer, ratio, or floating-point
number).  It should be emphasized that the parts of a complex
number are not necessarily floating-point numbers; in this, Common Lisp
is like PL/I and differs from Fortran.  However, both parts must
be of the same type: either both are rational, or both are of the
same floating-point format. 

Комплексные числа (тип \cd{complex})
представляются в алгебраической форме, с действительной и мнимой частями, каждая
из которых является некомплексным числом (целым, дробным, или с плавающей
точкой). Следует отметить, что части комплексного числа не
обязательно числа с плавающей точкой; в это Common Lisp похож на PL/I и
отличается от Fortran'а. Однако обе части должны быть одного типа: обе
рациональные, или обе какого-либо формата с плавающей точкой.

Complex numbers may be notated by writing the characters \cd{\#C}
followed by a list of the real and imaginary parts.
If the two parts as notated are not of the same type, then
they are converted according to the rules of floating-point contagion
as described in chapter~\ref{NUMBER}.
(Indeed, \cd{\#C(\emph{a} \emph{b})} is equivalent to \cd{\#,(complex \emph{a} \emph{b})};
see the description of the function \cd{complex}.)
For example:

Комплексные числа могут быть обозначены с помощью записи символа \cd{\#C} с
последующим списком действительной и мнимой частей.
Если две части, как было отмечено, не принадлежат одному типу, тогда они будут
преобразованы в соотвествии с правилами преобразования чисел с плавающей точкой
описанными в главе~\ref{NUMBER}.
\begin{lisp}
\#C(3.0s1 2.0s-1)~~~~~;{\rm Real and imaginary parts are short format}\\
\#C(5 -3)~~~~~~~~~~~~~;{\rm A Gaussian integer} \\
\#C(5/3 7.0)~~~~~~~~~~;{\rm Will be converted internally to \cd{\#C(1.66666 7.0)}} \\
\#C(0 1)~~~~~~~~~~~~~~;{\rm The imaginary unit, that is, \emph{i}}
\end{lisp}

\begin{lisp}
\#C(3.0s1 2.0s-1)~~~~~;{\rm Действительная и мнимая части в коротком формате}\\
\#C(5 -3)~~~~~~~~~~~~~;{\rm Целое Гаусса} \\
\#C(5/3 7.0)~~~~~~~~~~;{\rm Будет преобразовано в \cd{\#C(1.66666 7.0)}} \\
\#C(0 1)~~~~~~~~~~~~~~;{\rm Мнимая единица, \emph{i}}
\end{lisp}

The type of a specific complex number is indicated by a list
of the word \cd{complex} and the type of the components; for example,
a specialized representation for complex numbers with short floating-point
parts would be of type \cd{(complex short-float)}.  The type \cd{complex}
encompasses all complex representations.

Тип заданного комплексного числа определяется с помощью списка: слова
\cd{complex} и типа компонентов; например, специализированное представление для
комплексных чисел с частями принадлежащими типу короткое с плавающей точкой,
будет выглядеть так \cd{(complex short-float)}. Тип \cd{complex} включает все
представления комплексных типов.

A complex number of type \cd{(complex rational)}, that is, one whose
components are rational, can never have a zero imaginary part.
If the result of a computation would be a complex rational
with a zero imaginary part, the result is immediately
converted to a non-complex rational number by taking the
real part.  This is called the rule of \emph{complex canonicalization}.
This rule does not apply to floating-point complex numbers;
\cd{\#C(5.0 0.0)} and \cd{5.0} are different.

Комплексное число типа \cd{(complex rational)}, в котором части принадлежат
дробному типу, никогда не может содержать нулевую мнимую часть. Если в
результате вычислений получится комплексное число с нулевой мнимой частью, то
данное число будет автоматически сконвертировано в некомплексное дробное число,
равное действительной часть исходного числа. Это называется правилом {\it
 канонизации комплексного числа}. Данное правило не применяется для комплексных
чисел с плавающими точками; \cd{\#C(5.0 0.0)} и \cd{5.0} различные числа.

\goodbreak

\section{Characters Строковые символы}

Characters are represented as data objects of type \cd{character}.

Строковые символы представляют собой объекты данных, принадлежащих типу
\cd{строковый символ (character)}.
\begin{obsolete}
There are two subtypes of interest,
called \cd{standard-char} and \cd{string-char}.
\end{obsolete}
\begin{newer}
X3J13 voted in March 1989 \issue{CHARACTER-PROPOSAL} to remove the type \cd{string-char}.
\end{newer}

A character object can be notated by writing \cd{\#{\Xbackslash}} followed
by the character itself.  For example, \cd{\#{\Xbackslash}g} means the character
object for a lowercase g.  This works well enough for printing
characters.  Non-printing characters have names, and can be notated
by writing \cd{\#{\Xbackslash}} and then the name; for example, \cd{\#{\Xbackslash}Space}
(or \cd{\#{\Xbackslash}SPACE} or \cd{\#{\Xbackslash}space} or \cd{\#{\Xbackslash}sPaCE})
means the space character.  The syntax for character names after \cd{\#{\Xbackslash}}
is the same as that for symbols.  However, only character names
that are known to the particular implementation may be used.

Объект строкового символа может быть записан, как знак \cd{\#{\Xbackslash}} и последующий строковый символ. Например:  \cd{\#{\Xbackslash}g}
обозначает строковый символ g в нижнем регистре. Это работает достаточно хорошо
для вывода символов. Невыводимые строковые символы имеют имена, и могут быть
записаны с помощью \cd{\#{\Xbackslash}} и последующего имени; например,
\cd{\#{\Xbackslash}Space} (или \cd{\#{\Xbackslash}SPACE} или
\cd{\#{\Xbackslash}space} или \cd{\#{\Xbackslash}sPaCE}) обозначает символ пробела.
Синтаксис для записи имени строкового символа после \cd{\#{\Xbackslash}}, такой
же как и для Lisp символов. Однако в работе могут использоваться только те
имена, которые известны данной реализации.

\subsection{Standard Characters}

Common Lisp defines a standard character set (subtype \cd{standard-char})
for two purposes.
Common Lisp programs that are \emph{written} in the standard character set
can be read by any Common Lisp implementation; and Common Lisp programs
that \emph{use} only standard characters as data objects are most likely
to be portable.  The Common Lisp character set consists of a space character
\cd{\#{\Xbackslash}Space}, a newline character \cd{\#{\Xbackslash}Newline}, and the
following ninety-four
non-blank printing characters or their equivalents:

Common Lisp определяет множество стандартных символов (подтип
\cd{standard-char}) для двух целей.
Common Lisp программы, которые \emph{записаны} используя множество стандартных
символов, могут быть прочитаны любой реализацией Common Lisp; и Common Lisp
программы, которые \emph{используют} только стандартные символы в качестве
объектов данных, скорее всего будут портируемыми. Множество строковых символов
Common Lisp состоит из символа пробела, \cd{\#{\Xbackslash}Space}, символа
новой строки \cd{\#{\Xbackslash}Newline}, и следующих сорока четырех печатаемых
символов и их эквивалентов:
\begin{lisp}
! " \# \$ \% \& ' ( ) * + , - . / 0 1 2 3 4 5 6 7 8 9 : ; < = > ? \\
{\Xatsign} A B C D E F G H I J K L M N O P Q R S T U V W X Y Z {\Xlbracket} {\Xbackslash} {\Xrbracket} {\Xcircumflex} {\Xunderscore} \\
{\Xbq} a b c d e f g h i j k l m n o p q r s t u v w x y z {\Xlbrace} | {\Xrbrace} {\Xtilde}
\end{lisp}
The Common Lisp standard character set is apparently equivalent to
the ninety-five standard ASCII printing characters plus a newline character.
Nevertheless, Common Lisp is designed to be relatively independent of
the ASCII character encoding.  For example, the collating sequence
is not specified except to say that digits must be properly ordered,
the uppercase letters must be properly ordered, and
the lowercase letters must be properly ordered
(see \cd{char<} for a precise specification).
Other character encodings, particularly EBCDIC, should be easily accommodated
(with a suitable mapping of printing characters).

Множество стандартных строковых символов Common Lisp'а явно соответствует
множеству из сорока пяти стандартных ASCII печатаемых символов и символа новой
строки. Как бы то ни было, Common Lisp спроектирован так, чтобы быть независимым
от ASCII кодировки символов. Например, сортировка последовательности не
определена, кроме того, что можно сказать, что цифры могут быть корректно
отсортированы, буквы в верхнем регистре могут быть корректно отсортрованы и
буквы в нижнем регистре могут быть корректно отсортированы (смотрите
спецификацию функции \cd{char<}). Другие реализация кодировка строковых
символов, в частности EBCDIC, должна быть легко приспособлена (с необходимым
соотвествием выводимых символов).

Of the ninety-four non-blank printing characters, the following are
used in only limited ways in the syntax of Common Lisp programs:

Из сорока четырех печатаемых символов, следующие испльзуются с ограничениями
связанными с синтаксисом Common Lisp програм:
\begin{lisp}
{\Xlbracket}~~{\Xrbracket}~~{\Xlbrace}~~{\Xrbrace}~~?~~!~~{\Xcircumflex}~~{\Xunderscore}~~{\Xtilde}~~\$~~\% 
\end{lisp}

\begin{obsolete}
\noindent
All of these characters except \cd{!} and \cd{{\Xunderscore}} are used within
\cd{format} strings as formatting directives.
Except for this,
\cd{{\Xlbracket}}, \cd{{\Xrbracket}}, \cd{{\Xlbrace}}, \cd{{\Xrbrace}},
\cd{?}, and \cd{!} are not used in Common Lisp and are reserved to the user
for syntactic extensions; \cd{{\Xcircumflex}} and \cd{{\Xunderscore}}
are not yet used in Common Lisp
but are part of the syntax of reserved tokens
and are reserved to implementors;
\cd{{\Xtilde}} is not yet used in Common Lisp and is reserved to implementors;
and \cd{\$} and \cd{\%} are normally regarded as alphabetic characters
but are not used in the names of any standard Common Lisp functions,
variables, or other entities.
\end{obsolete}

\begin{newer}
X3J13 voted in June 1989 \issue{PRETTY-PRINT-INTERFACE}
to add a \cd{format} directive \cd{{\Xtilde}{\Xunderscore}} (see chapter~\ref{PPRINT}).
\end{newer}

The following characters are called \emph{semi-standard}:

Следующие строковые символы называются \emph{слегка стандартизированными}:
\begin{lisp}
\#{\Xbackslash}Backspace~~\#{\Xbackslash}Tab~~\#{\Xbackslash}Linefeed~~\#{\Xbackslash}Page~~\#{\Xbackslash}Return~~\#{\Xbackslash}Rubout
\end{lisp}
Not all implementations of Common Lisp need to support them; but those
implementations that
use the standard ASCII character set should support them, treating them as
corresponding respectively to the ASCII characters BS (octal code 010),
HT (011), LF (012), FF (014), CR (015), and DEL
(177). These characters are not
members of the subtype \cd{standard-char} unless synonymous with
one of the standard characters specified above.
For example, in a given implementation it might
be sensible for the implementor to define
\cd{\#{\Xbackslash}Linefeed} or \cd{\#{\Xbackslash}Return} to be synonymous with \cd{\#{\Xbackslash}Newline},
or \cd{\#{\Xbackslash}Tab} to be synonymous with \cd{\#{\Xbackslash}Space}.

Не все реализации Common Lisp'а нуждаются в поддержке этих символов; но те
реализации, что используют ASCII кодировку должны их поддерживать,
соответственно BS (восьмеричный код 010), HT (011), LF (012), FF (014), CR
(015) и DEL (177). Эти строковые символы не являются членами подтипа
\cd{standard-char}, если не будут созданы синонимы для них.
Например, разработчик реализации может 
определить \cd{\#{\Xbackslash}Linefeed} или \cd{\#{\Xbackslash}Return} как
синоним для \cd{\#{\Xbackslash}Newline},
или \cd{\#{\Xbackslash}Tab} как синоним для \cd{\#{\Xbackslash}Space}.

\subsection{Line Divisions Разделители строк}

The treatment of line divisions is one of the most difficult issues
in designing portable software, simply because there is so little agreement
among operating systems.  Some use a single character to delimit lines;
the recommended ASCII character for this purpose is the line feed character
LF (also called the new line character, NL),
but some systems use the carriage
return character CR.  Much more common is the two-character sequence
CR followed by LF.  Frequently line divisions have no representation
as a character but are implicit in the structuring of a file into records,
each record containing a line of text.  A deck of punched cards has this
structure, for example.

Обработки разделителей строк является одним из самых сложных моментов в
проектировании переносимой программы, преимущественно потому, что между
операционными системами очень мало соглашений по этому поводу. Некоторые
используют только один символ; рекомендуемый для этого ASCII символ является
символом перевода строки LF (также называемый символом новой строки, NL),
но некоторые системы используют символ перевода каретки CR. Более
широко используется последовательность из двух символов CR и последующем
LF. Часто разделители строк не имеют выводимого представления, но неявно влияют
на структурирование файла в записи, каждая запись содержит строку
текста. Наприимер, дека перфокарт имеет такию структуру.

Common Lisp provides an abstract interface by requiring that there be a single
character, \cd{\#{\Xbackslash}Newline}, that within the language serves as a line
delimiter.  (The language C has a similar requirement.)
An implementation of Common Lisp must translate between this internal
single-character representation and whatever external representation(s)
may be used.

Common Lisp предоставляет абстрактный интерфейс, требуя наличия одного символа
\cd{\#{\Xbackslash}Newline}, который являет разделителем строк. (Язык C имеет
подобное требование.)
Реализация Common Lisp'а должна транслировать это односимвольное представление
разделители в то, что требуется во внешних системах в данной операционной системе.

\beforenoterule
\begin{implementation}
How the character called \cd{\#{\Xbackslash}Newline} is represented
internally is not specified here, but it is strongly suggested that
the ASCII LF character be used in Common Lisp implementations that use the
ASCII character encoding.  The ASCII CR character is a workable,
but in most cases inferior, alternative.
\end{implementation}
\afternoterule

\begin{newer}
When the first edition was written it was not yet clear that UNIX would
become so widely accepted.  The decision to represent
the line delimiter as a single character has proved to be a good one.
\end{newer}

The requirement that a line division be represented as a single character
has certain consequences.  A character string
written in the middle of a program in such a way as to span more than
one line must contain exactly one character to represent each line division.
Consider this code fragment:

Требование того, что разделитель строк должен быть представлен одним символом,
имеет следующие последствия. Строковый объект, записанный в середине программы и
содержащий несколько строк, должен содержать только один символ для каждого
разделителя. Рассмотрим фрагмент следующего кода:
\begin{lisp}
(setq a-string "This string \\
contains \\
forty-two characters.")
\end{lisp}
Between \cd{g} and \cd{c} there must be exactly one character,
\cd{\#{\Xbackslash}Newline}; a two-character sequence, such as \cd{\#{\Xbackslash}Return} and then
\cd{\#{\Xbackslash}Newline}, is not acceptable, nor is the absence of a character.
The same is true between \cd{s} and \cd{f}.

Между \cd{g} and \cd{c} должен быть только один символ,
\cd{\#{\Xbackslash}Newline}; последовательность из двух строковых символов
такая, как \cd{\#{\Xbackslash}Return} и\cd{\#{\Xbackslash}Newline},
некорректна.
Такя же ситуация и между \cd{s} и \cd{f}.

When the character \cd{\#{\Xbackslash}Newline} is written to an output file,
the Common Lisp implementation must take the appropriate action
to produce a line division.  This might involve writing out a
record or translating \cd{\#{\Xbackslash}Newline} to a CR/LF sequence.

Когда строковый символ \cd{\#{\Xbackslash}Newline} записывается в выходной файл,
реализация Common Lisp'а должна предпринять соотвествующие действия для
разделения строк. Это может быть реализовано, как трансляция
\cd{\#{\Xbackslash}Newline} в последовательность CR/LF.

\beforenoterule
\begin{implementation}
If an implementation uses the ASCII character encoding,
uses the CR/LF sequence externally to delimit lines,
uses LF to represent \cd{\#{\Xbackslash}Newline} internally, and supports \cd{\#{\Xbackslash}Return}
as a data object corresponding to the ASCII character CR, the
question arises as to what action to take when the program
writes out \cd{\#{\Xbackslash}Return} followed by \cd{\#{\Xbackslash}Newline}.
It should first be noted that \cd{\#{\Xbackslash}Return} is not a standard Common Lisp
character, and the action to be taken when \cd{\#{\Xbackslash}Return} is written out
is therefore not defined by the Common Lisp language.  A plausible approach
is to buffer the \cd{\#{\Xbackslash}Return} character and suppress it if and only if the
next character is \cd{\#{\Xbackslash}Newline} (the net effect is to generate a CR/LF
sequence).
Another plausible
approach is simply to ignore
the difficulty and declare that writing \cd{\#{\Xbackslash}Return} and then
\cd{\#{\Xbackslash}Newline} results in the sequence CR/CR/LF in the output.
\end{implementation}
\afternoterule

\subsection{Non-standard Characters Нестандартные символы}

Any implementation may provide additional characters, whether printing
characters or named characters.  Some plausible examples:

Любая реализация может предоставлять дополнительные строковые символы, и
печатаемые и именовынные. Некоторые вероятные примеры:

\newpage%manual

\begin{lisp}
\#{\Xbackslash}\(\pi\)~~\#{\Xbackslash}\(\alpha\)~~\#{\Xbackslash}Break~~\#{\Xbackslash}Home-Up~~\#{\Xbackslash}Escape
\end{lisp}
The use of such characters may render Common Lisp programs non-portable.

Использование таких символов, может создавать проблемы для портируемости Common
Lisp программы.

\begin{obsolete}
\subsection{Character Attributes Устарело}
Every object of type \cd{character}
has three attributes: \emph{code}, \emph{bits}, and \emph{font}.
The code attribute is intended to distinguish among the printed glyphs
and formatting functions for characters; it is a numerical encoding
of the character proper.
The bits attribute allows extra
flags to be associated with a character.  The font attribute permits
a specification of the style of the glyphs (such as italics).
Each of these attributes may be understood to be a non-negative integer.

The font attribute may be notated in unsigned decimal notation
between the \cd{\#} and the \cd{{\Xbackslash}}.  For example,
\cd{\#3{\Xbackslash}a} means the letter \cd{a} in font 3.
This might mean the same thing as \cd{\#{\Xbackslash}$\alpha$} if font 3
were used to represent Greek letters.
Note that not all Common Lisp implementations provide for non-zero
font attributes; see \cd{char-font-limit}.

The bits attribute may be notated
by preceding the name of the character by the names or initials
of the bits,
separated by hyphens.  The character itself may be written
instead of the name, preceded if necessary by \cd{{\Xbackslash}}.  For example:
\begin{lisp}
\hskip 0.5\textwidth\=\kill
\#{\Xbackslash}Control-Meta-Return\>\#{\Xbackslash}Meta-Control-Q \\
\#{\Xbackslash}Hyper-Space\>\#{\Xbackslash}Meta-{\Xbackslash}a \\
\#{\Xbackslash}Control-A\>\#{\Xbackslash}Meta-Hyper-{\Xbackslash}: \\
\#{\Xbackslash}C-M-Return\>\#{\Xbackslash}Hyper-{\Xbackslash}\(\pi\)
\end{lisp}
Note that not all Common Lisp implementations provide for non-zero
bits attributes; see \cd{char-bits-limit}.
\end{obsolete}

\begin{newer}
X3J13 voted in March 1989 \issue{CHARACTER-PROPOSAL}
to replace the notion of bits and font attributes with
that of implementation-defined attributes.
\end{newer}

\begin{obsolete}
\subsection{String Characters Устарело}

Any character whose bits and font attributes are zero may be contained
in strings.  All such characters together constitute a subtype of
the characters; this subtype is called \cd{string-char}.
\end{obsolete}


\begin{newer}
X3J13 voted in March 1989 \issue{CHARACTER-PROPOSAL}
to eliminate the type \cd{string-char}.
Two new subtypes of \cd{character} are \cd{base-character},
defined to be equivalent to the result of the function call
\begin{lisp}
(upgraded-array-element-type 'standard-char)
\end{lisp}
and \cd{extended-character}, defined to be equivalent to the type specifier
\begin{lisp}
(and character (not base-character))
\end{lisp}
An implementation may support additional subtypes of \cd{character}
that may or may not be supertypes of \cd{base-character}.
In addition, an implementation may define \cd{base-character}
to be equivalent to \cd{character}.  The choice of any base characters
that are not standard characters is implementation-defined.
Only base characters can be elements of a base string.
No upper bound is specified for the number of distinct characters
of type \cd{base-character}---that is implementation-dependent---but the lower
bound is 96, the number of standard Common Lisp characters.
\end{newer}

\section{Symbols Символы}

Symbols are Lisp data objects that serve several purposes
and have several interesting characteristics.  Every object of
type \cd{symbol} has a name,
called its \emph{print name}.  Given a symbol, one can
obtain its name in the form of a string.  Conversely,
given the name of a symbol as a string, one can obtain the
symbol itself.  (More precisely, symbols are organized into
\emph{packages}, and all the symbols in a package are uniquely
identified by name.  See chapter~\ref{XPACK}.)

Символы (прим. переводчика: не строковые) являются Lisp'овыми объектами данных,
созданы для нескольких целей и имеют несколько интересных характеристик. Каждый
объект типа \cd{symbol} имеет имя, называемое его \emph{выводимым именем (print
  name)}. Существует возможность получить имя символа в виде строки. Также
возможно обратное действие, получение имени символа из строки. (Более подробно:
символы могут быть организованы в \emph{пакеты}, и все символы в пакете имеют
уникальные имена. Смотрите главу~\ref{XPACK}.)

Symbols have a component called the \emph{property list}, or \emph{plist}.
By convention this is always a list whose even-numbered
components (calling the first component zero) are symbols,
here functioning as property names, and whose odd-numbered components
are associated property values.  Functions are provided for manipulating
this property list; in effect, these allow a symbol to be treated as an
extensible record structure.

У символов есть компонент, называемый \emph{список свойств}, или \emph{иplist}.
Список свойств всегда является списком, у которого четные элементы (начиная с
нулевого) являеются символами, они выступают в качестве имен свойств, и нечетные
элементы являются связанными со свойствами значениями. Для манипуляций с этим
списком свойств предоставляются функции, это позволяет символу выступать в роли расширяемой структуры.

Symbols are also used to represent certain kinds of variables in Lisp
programs, and there are functions for dealing with the values associated
with symbols in this role. 

Символы также используются для представления определенных видов переменных в
Lisp программах, и для манипуляции значениями связанными с символами в такой
роли также предоставляются функции.

A symbol can be notated simply by writing its name.
If its name is not empty, and if the name consists only of
uppercase alphabetic, numeric, or certain pseudo-alphabetic
special characters (but not
delimiter characters such as parentheses or space), and if
the name of the symbol cannot be mistaken for a number, then
the symbol can be notated by the sequence of characters in its name.
Any uppercase letters that appear in the (internal) name may
be written in either case in the external notation (more on this below).
For example:

Символ может быть обозначен просто записью его имени.
Если его имя непустое, и если его имя содержит только алфавитные буквы в верхнем
регистре, цифры или некоторые псевдо-алфавитные строковые символы (но не
разделители, как круглые скобки и пробелы), и если имя символа не может быть
интерпретировано как число, тогда имя символа задается последовательностью букв
его имени.
Все буквы записанные в имени символа переводятся в верхний регистр во внутреннем
представлении.
Например:
\begin{lisp}
~~~~~~~~~~~~~~~~~~~~\=\kill
FROBBOZ\>;{\rm The symbol whose name is \cd{FROBBOZ}} \\
frobboz\>;{\rm Another way to notate the same symbol} \\
fRObBoz\>;{\rm Yet another way to notate it} \\
unwind-protect\>;{\rm A symbol with a \cd{-} in its name} \\
+\$\>;{\rm The symbol named \cd{+\$}} \\
1+\>;{\rm The symbol named \cd{1+}} \\
+1\>;{\rm This is the integer 1, not a symbol} \\
pascal{\Xunderscore}style\>;{\rm This symbol has an underscore in its name} \\
b{\Xcircumflex}2-4*a*c\>;{\rm This is a single symbol!} \\
\>;~{\rm It has several special characters in its name} \\
file.rel.43\>;{\rm This symbol has periods in its name} \\
/usr/games/zork\>;{\rm This symbol has slashes in its name}
\end{lisp}

\begin{lisp}
~~~~~~~~~~~~~~~~~~~~\=\kill
FROBBOZ\>;{\rm Символ, имя которого \cd{FROBBOZ}} \\
frobboz\>;{\rm Другой путь записи того же символа} \\
fRObBoz\>;{\rm Еще один путь записи полюбившегося символа} \\
unwind-protect\>;{\rm Символ с дефисом в имени} \\
+\$\>;{\rm Символ с именем \cd{+\$}} \\
1+\>;{\rm Символ с именем \cd{1+}} \\
+1\>;{\rm Это число 1, а не символ} \\
pascal{\Xunderscore}style\>;{\rm Этот символ содержит знак подчеркивания в своем
  имени} \\
b{\Xcircumflex}2-4*a*c\>;{\rm Это один символ} \\
\>;~{\rm Этот символ содержит некоторые специальные знаки в своем имени} \\
file.rel.43\>;{\rm Символ содержит точки в своем имени} \\
/usr/games/zork\>;{\rm Символ содержит наклонные черты в своем имени}
\end{lisp}

In addition to letters and numbers, the following characters are normally
considered to be alphabetic for the purposes of notating
symbols:

В дополнение к буквам и числам, следующие строковые символы допускаются в
использовании в написании имени символа:
\begin{lisp}
+~~-~~*~~/~~{\Xatsign}~~\$~~\%~~{\Xcircumflex}~~\&~~{\Xunderscore}~~=~~<~~>~~{\Xtilde}~~.
\end{lisp}
Some of these characters have conventional purposes for naming things;
for example, symbols that name special variables
generally have names beginning and ending with
\cd{*}.  The last character listed above, the period, is considered alphabetic
\emph{provided} that a token does not consist entirely of periods.
A single period standing by itself is used in the notation
of conses and dotted lists; a token consisting of two or more periods
is syntactically illegal.  (The period also serves as the decimal point
in the notation of numbers.)

Некоторые из этих строковых символов имеют специальные общеприянтые значения
для имен;
например, символы, которые задают специальные переменные, обычно имеют имена
начинающиеся и заканчивающиеся зведочкой \d{*}.
Одиночная точка используется для задания cons-ячеек или списков с точкой. Точка
также является разделителем дробной части.

The following characters are also alphabetic by default but are explicitly
reserved to the user for definition as reader macro characters
(see section~\ref{MACRO-CHARACTERS-SECTION}) or any other desired purpose
and therefore should not be used routinely in names of symbols:

Следующие строковые символы предназначены для использования в качестве
макросимволов для изменения и расширения синтаксиса языка:
\begin{lisp}
?~~!~~{\Xlbracket}~~{\Xrbracket}~~{\Xlbrace}~~{\Xrbrace}
\end{lisp}

A symbol may have uppercase letters, lowercase letters, or both
in its print name.
However, the Lisp reader normally converts lowercase letters to
the corresponding uppercase letters when reading symbols.
The net effect is that most of the time case makes no
difference when \emph{notating} symbols.  Case \emph{does} make
a difference internally and when printing a symbol.
Internally the symbols that name all standard Common Lisp functions,
variables, and keywords have uppercase names; their names appear
in lowercase in this book for readability.  Typing such names
with lowercase letters works because the function \cd{read} will convert
lowercase letters to the equivalent uppercase letters.

Выводимое имя символ может содержать буквы в верхнем и нижнем регистрах.
Однако, при чтении Lisp reader обычно ковертирует буквы нижнего регистра в
верхний.
В реализации все символы, которые именуют все стандартные Common Lisp переменные
и функции хранятся в верхнем регистре. Однако в книге все эти символы для
удобства приводятся в нижнем регистре. Использование имен символов в нижнем
регистре при написании программы возможно потому, что \cd{read} конвертирует все
читываемые символы в верхний регистр.

\begin{newer}
X3J13 voted in June 1989 \issue{READ-CASE-SENSITIVITY} to introduce
\cd{readtable-case}, which controls whether \cd{read} will alter the case
of letters read as part of the name of a symbol.

Переменная \cd{readtable-case} контролирует поведение функции \cd{read}
касаемо преобразования регистров букв в именах символов.
\end{newer}

If a symbol cannot be simply notated by the characters of its name
because the (internal) name contains special characters or lowercase letters,
then there are two ``escape'' conventions for notating them.
Writing a \cd{{\Xbackslash}} character before any character causes the character
to be treated itself as an ordinary character for use in a symbol name;
in particular, it suppresses internal conversion of lowercase letters
to their uppercase equivalents.
If any character in a notation is preceded by \cd{{\Xbackslash}}, then that
notation can never be interpreted as a number.
For example:

Если символ не может быть задан, потому что в его имени используются
недопустимые буквы и знаки, их можно <<экранировать>> двумя способами. Один из
них заключается в использовании обратной наклонной черты перед каждым
экранируемым знаком. В таком случае имя символа никогда не будет ошибочно
интепретировано, как число.
Например:
\begin{lisp}
~~~~~~~~~~~~~~~~~~~~~~~~\=\kill
{\Xbackslash}(\>;{\rm Символ с именем \cd{(}} \\
{\Xbackslash}+1\>;{\rm Символ с именем \cd{+1}} \\
+{\Xbackslash}1\>;{\rm Также симол с именем \cd{+1}} \\
{\Xbackslash}frobboz\>;{\rm Символ с именем \cd{fROBBOZ}} \\
3.14159265{\Xbackslash}s0\>;{\rm Символ с именем \cd{3.14159265s0}} \\
3.14159265{\Xbackslash}S0\>;{\rm Другой символ с именем \cd{3.14159265S0}} \\
3.14159265s0\>;{\rm short-format с плавающей точкой для аппроксимации числа \(\pi\)} \\
APL{\Xbackslash}{\Xbackslash}360\>;{\rm Символ с именем \cd{APL{\Xbackslash}360}} \\
apl{\Xbackslash}{\Xbackslash}360\>;{\rm Также символ с именем \cd{APL{\Xbackslash}360}} \\
{\Xbackslash}(b{\Xcircumflex}2{\Xbackslash}){\Xbackslash} -{\Xbackslash} 4*a*c\>;{\rm Имя \cd{(B{\Xcircumflex}2) - 4*A*C};} \\
\>;~{\rm содержит скобки и два пробела} \\
{\Xbackslash}({\Xbackslash}b{\Xcircumflex}2{\Xbackslash}){\Xbackslash} -{\Xbackslash} 4*{\Xbackslash}a*{\Xbackslash}c\>;{\rm Имя \cd{(b{\Xcircumflex}2) - 4*a*c};} \\
\>;~{\rm буквы явно указаны в нижнем регистре}
\end{lisp}
It may be tedious to insert a \cd{{\Xbackslash}} before \emph{every} delimiter
character in the name of a symbol if there are many of them.
An alternative convention is to surround the name of a symbol
with vertical bars; these cause every character between them to
be taken as part of the symbol's name, as if \cd{{\Xbackslash}} had been written
before each one, excepting only
\cd{|} itself and \cd{{\Xbackslash}}, which must nevertheless be preceded by \cd{{\Xbackslash}}.
For example:

Использование \cd{{\Xbackslash}} перед \emph{каждой} буквой утомительно, если
таких <<запрещенных>> букв в имени много. Алтернативным методом экранирования
знаков в имени символа является заключение всего имени или только его части в
скобки из вертикальных черт. Это эквивалентно тому, что каждая буква была бы
экранирована обратной косой чертой.
\begin{lisp}
~~~~~~~~~~~~~~~~~~~~\=\kill
|"|\>;{\rm То же что и \cd{{\Xbackslash}"}} \\
|(b{\Xcircumflex}2) - 4*a*c|\>;{\rm Имя \cd{(b{\Xcircumflex}2) - 4*a*c}} \\
|frobboz|\>;{\rm Имя \cd{frobboz}, а не \cd{FROBBOZ}} \\
|APL{\Xbackslash}360|\>;{\rm Имя \cd{APL360}, потому что \cd{{\Xbackslash}} экранирует the \cd{3}} \\
|APL{\Xbackslash}{\Xbackslash}360|\>;{\rm Имя \cd{APL{\Xbackslash}360}} \\
|apl{\Xbackslash}{\Xbackslash}360|\>;{\rm Имя \cd{apl{\Xbackslash}360}} \\
|{\Xbackslash}|{\Xbackslash}||\>;{\rm То же, что и \cd{{\Xbackslash}|{\Xbackslash}|}: имя \cd{||}} \\
|(B{\Xcircumflex}2) - 4*A*C|\>;{\rm Имя \cd{(B{\Xcircumflex}2) - 4*A*C};} \\
\>;~{\rm содержит скобки и два пробела} \\
|(b{\Xcircumflex}2) - 4*a*c|\>;{\rm Имя \cd{(b{\Xcircumflex}2) - 4*a*c}}
\end{lisp}

%???
%\begin{newer}
%X3J13 voted in March 1989 \issue{CHARACTER-PROPOSAL}
%to clarify that the print name of a symbol may contain any character,
%although it may be necessary to use an escape character in the notation.
%However, some or all of the implementation-defined attributes may be removed
%from the character, at the discretion of the implementation,
%before inclusion in the print name.
%\end{newer}

\section{Lists and Conses Списки и Cons-ячейки}

\indexterm{cons}
A \cd{cons} is a record structure containing two components
called the \emph{car} and the \emph{cdr}.  Conses are used primarily
to represent lists.

\cd{cons-ячейка} является записью структуры, содержащей два элемента, называемых
\emph{car} и \emph{cdr}. Cons-ячейки используются преимущественно для отображения
списков.

A \emph{list} is recursively defined to be either the empty list
or a cons whose \emph{cdr} component is a list.
A list is therefore a chain of conses linked by their \emph{cdr} components
and terminated by {\nil}, the empty list.  The \emph{car} components of the conses
are called the \emph{elements} of the list.  For each element of the list
there is a cons.  The empty list has no elements at all.

\emph{Список} рекурсивно определяется пустым списком или cons-ячейкой, у
которой \emph{cdr} элемент является списком.
Таким образом, список является цепочкой cons-ячеек связанных с помощью их {\it
  cdr} элементов, заканчивающейся пустым списком с помощью {\nil}. \emph{car}
элементы cons-ячеек называются \emph{элементами} списка. Для каждого элемента
списка существует cons-ячейка. Пустой список не имеет элементов вообще.

A list is notated by writing the elements of the list in order,
separated by blank space (space, tab, or return characters)
and surrounded by parentheses.

Список записывается с помощью элементов в необходимом порядке, разделяемых
пробелом (пробел, таб, возврат каретки) и окруженных круглыми скобками.
\begin{lisp}
(a b c)~~~~~~~~~~~~~~~;{\rm A list of three symbols} \\
(2.0s0 (a 1) \#{\Xbackslash}*)~~~~~;{\rm A list of three things: a short floating-point} \\
~~~~~~~~~~~~~~~~~~~~~~;~{\rm number, another list, and a character object}
\end{lisp}

\begin{lisp}
(a b c)~~~~~~~~~~~~~~~;{\rm Список трех элементов} \\
(2.0s0 (a 1) \#{\Xbackslash}*)~~~~~;{\rm Список трех элементов: короткого с
  плавающей точкой} \\
~~~~~~~~~~~~~~~~~~~~~~;~{\rm числа, другого списка, и строкового символа}
\end{lisp}
The empty list {\nil} therefore can be written as {\emptylist}, because it is a list
with no elements.

Таким образом, пустой список {\nil} может быть записан, как {\emptylist}, потому что
является списком без элементов.

A \emph{dotted list} is one whose last cons does not have {\nil} for
its \emph{cdr}, rather some other data object (which is also not a cons,
or the first-mentioned cons would not be the last cons of the list).
Such a list is called ``dotted'' because of the special notation
used for it: the elements of the list are written between
parentheses as before, but after the last element and before
the right parenthesis are written a dot (surrounded by blank space)
and then the \emph{cdr} of the last cons.  As a special case,
a single cons is notated by writing the \emph{car} and the \emph{cdr} between
parentheses and separated by a space-surrounded dot.
For example:

\emph{Список с точкой} является списком, последняя cons-ячейка которого в {\it
  cdr} элементе содержит
объект данных, а не {\nil} (который не является
cons-ячейкой, иначе исходная cons-ячейка не была бы последней).
Такой список называется <<список с точкой>> по причине используемой для него
специальной записи: элементы списка записанные в двух последних поизициях списка
перед закрывающей круглое скобкой разделяются точкой (обрамленной с двух сторон
пробелами). Тогда последнее значение будет содержаться в \emph{cdr} элементе
последней cons-ячейки. В особых случаях, одиночная cons-ячейка может быть
записана с помощью \emph{car} и \emph{cdr} элементов, обрамленных в круглые скобки
и разделенных с помощью точки, окруженной пробелами. 
Например:
\begin{lisp}
(a . 4)~~~~~~~~~;{\rm A cons whose \emph{car} is a symbol} \\
~~~~~~~~~~~~~~~~;~{\rm and whose \emph{cdr} is an integer} \\
(a b c . d)~~~~~;{\rm A dotted list with three elements whose last cons} \\
~~~~~~~~~~~~~~~~;~{\rm has the symbol \cd{d} in its \emph{cdr}}
\end{lisp}

\begin{lisp}
(a . 4)~~~~~~~~~;{\rm const-ячейка, \emph{car} которой является символом} \\
~~~~~~~~~~~~~~~~;~{\rm и \emph{cdr} которой равен целому числу} \\
(a b c . d)~~~~~;{\rm Список с точкой с тремя элементами, у последней} \\
~~~~~~~~~~~~~~~~;~{\rm cons-ячейки \emph{cdr} равен символу \cd{d}}
\end{lisp}

\beforenoterule
\begin{incompatibility}
In MacLisp, the dot in dotted-list notation
need not be surrounded by white space or other delimiters.
The dot is required to be delimited in Common Lisp, as in Lisp Machine Lisp.
\end{incompatibility}
\afternoterule

It is legitimate to write something like \cd{(a b . (c d))};
this means the same as \cd{(a b c d)}.  The standard Lisp
output routines will never print a list in the first form, however;
they will avoid dot notation wherever possible.

Правильной записью также является что-то наподобие \cd{(a b . (c d))};
она означает то же, что и \cd{(a b c d)}. Стандартный Lisp вывод никогда не
распечатает список в первом виде, таким образом, он старается избавиться от
записи с точкой, когда это возможно.

Often the term \emph{list} is used to refer either to true lists or to
dotted lists.  When the distinction is important,
the term ``true list'' will be used to refer to a list
terminated by {\nil}.  Most functions
advertised to operate on lists expect to be given true lists. Throughout
this book, unless otherwise specified, it is an error to pass a dotted
list to a function that is specified to require a list as an argument.

Часто термин \emph{список} употребляется и для обычных списков и для списков с
точкой. Когда разница важна, для списка, заканчивающегося с помощью {\nil},
будет употребляться термин <<Ъ список>>. Большинство функций указывают, что
оперируют списками, ожидая что они Ъ. Везде в этой книге, если не указано иное,
передача списка с точкой в такие функции является ошибкой.

\beforenoterule
\begin{implementation}
Implementors are encouraged to use the equivalent
of the predicate \cd{endp} wherever it is necessary to test
for the end of a list.  Whenever feasible, this test should explicitly
signal an error if a list is found to be terminated by a non-{\nil} atom.
However, such an explicit error signal is not required, because
some such tests occur in important loops where efficiency is important.
In such cases, the predicate \cd{atom} may be used to test
for the end of the list, quietly treating any non-{\nil} list-terminating
atom as if it were {\nil}.
\end{implementation}
\afternoterule

Sometimes the term \emph{tree} is used to refer to some cons
and all the other conses transitively accessible to it
through \emph{car} and \emph{cdr} links until non-conses are reached;
these non-conses are called the \emph{leaves} of the tree.

Иногда используется термин \emph{дерево} для ссылки на некоторую cons-ячейку,
которая содержит другие cons-ячейки в своих \emph{car} и \emph{cdr} элементах,
которые также содержат cons-ячейки в своих элементах и так далее, пока не будут
достингуты элементы, не являющиеся cons-ячейками.
Такие элементы, не являющиеся cons-ячейками называются \emph{листьями} дерева.

Lists, dotted lists, and trees are not mutually exclusive data types;
they are simply useful points of view about structures of conses.
There are yet other terms, such as \emph{association list}.
None of these are true Lisp data types.  Conses are a data type,
and {\nil} is the sole object of type \cd{null}.
The Lisp data type \cd{list} is taken to mean the union of the
\cd{cons} and \cd{null} data types, and therefore encompasses both
true lists and dotted lists.

Списки, списки с точкой, и деревья вместе не завершают список типов данных;
они просто являются удобной точкой для рассмотрения таки структур, как
cons-ячейки.
Существуют также другие термины, такие как, например, \emph{ассоциативный
  список}. Ни один из этих типов данных не является Lisp'овым типом
данных. Cons-ячейки являются таким типом данных, и {\nil} является объектом типа
\cd{null}. Lisp'овый тип данных \cd{список} подразумаевает объединение типов
\cd{cons-ячеек} и \cd{null}, и по этой причине содержит в себе оба типа Ъ
список и список с точкой.

\section{Arrays Массивы}
\label{ARRAY-TYPE-SECTION}

\indexterm{array}
An \cd{array} is an object with components arranged according
to a Cartesian coordinate system.
In general, these components may be any Lisp data objects.

\cd{Массив} является объектом с элементами расположенными в соответсвии с
Декартовой системой координат.

The number of dimensions of an array is called its \emph{rank}
(this terminology is borrowed from APL);
the rank is a non-negative integer.
Likewise, each dimension is itself a non-negative integer.
The total number of elements in the array is the product of all the
dimensions.

Количество измерений массива называется \emph{ранг} (это терминология взята из
APL); ранг является неотрицательных целым.
Также каждое измерение само по себе является неотрицательным целым.
Общее количество элементов в массиве является произведением всех измерений.

An implementation of Common Lisp may impose a limit on the rank of an array,
but this limit may not be smaller than 7.  Therefore, any Common Lisp
program may assume the use of arrays of rank 7 or less.
(A program may determine the actual limit on array ranks for
a given implementation by examining the constant \cd{array-rank-limit}.)

Реализация Common Lisp'а может налагать ограничение на ранг массива, но данное
ограничение не может быть менее 7. Таким образом, любая Common Lisp программа
может использователь массивы с семью и менее измеренеиями.
(Программа может получить текущее ограничение для ранга для заданной реализации
с помощью константы \cd{array-rank-limit}.)

It is permissible for a dimension to be zero.  In this case,
the array has no elements, and any attempt to access an element
is in error.  However, other properties of the array, such as the
dimensions themselves, may be used.
If the rank is zero, then there are no dimensions, and the
product of the dimensions is then by definition 1.
A zero-rank array therefore has a single element.

Существование нулевого ранга допускается. В этом случае, массив не содержит
элементов, и любой доступ к элементам является ошибкой. Тогда как другие
свойства массива могут использоваться. Если ранг равен нулю, тогда массив не
имеет измерений, и их произведение приравнивается к 1 (FIXME).
Таким образом массив с нулевым рангом содержит один элемент.

An array element is specified by a sequence of indices.
The length of the sequence must equal the rank of the array.
Each index must be a non-negative integer strictly less than
the corresponding array dimension.  Array indexing is
therefore zero-origin, not one-origin as in (the default case of)
Fortran.

Элемент массива задается последовательностью индексов.
Длина данной последовательности должна равнятся рангу массива.
Каждый индекс должен быть неотрицательным целым строго меньшим размеру
соотвествующего измерения. Также индексация массива начинается с нуля, а не с
единицы, как в по умолчанию Fortran'е.

As an example, suppose that the variable \cd{foo} names a 3-by-5 array.
Then the first index may be 0, 1, or 2, and the second index
may be 0, 1, 2, 3, or 4.  One may refer to array elements using
the function \cd{aref}; for example, \cd{(aref foo 2 1)}
refers to element (2, 1) of the array.  Note that \cd{aref} takes
a variable number of arguments: an array, and as many indices
as the array has dimensions.
A zero-rank array has no dimensions, and therefore
\cd{aref} would take such an array and no indices, and return the sole
element of the array.

В качестве примера, предположим, что переменная \cd{foo} обозначает двумерный
массив с размерами измерений 3 и 5. Первый индекс может быть 0, 1 или 2, и второй
индекс может быть 0, 1, 2, 3 или 4. Обращение к элементам массива может быть
осуществлено с помощью функции \cd{aref}; например, \cd{(aref foo 2 1)}
ссылается на элемент массива (2, 1). Следует отметить, что \cd{aref} принимает
переменное число аргументов: массив, и столько индексов, сколько измерений у
массива.
Массив с нулевым рангом не имеет измерений, и в таком случае \cd{aref} принимает
только один параметр -- массив, и не принимает индексы, и возвращает одиночный
элемент массива.

In general, arrays can be multidimensional,
can share their contents with other array objects, and can have their
size altered dynamically (either enlarging or shrinking) after creation.
A one-dimensional array may also have a \emph{fill pointer}.

В общем случае, Массивы могут быть многомерными, могут иметь общее содержимое с
другими массивами. и могут динамически менять свой размер после создания (и
увеличивать, и уменьшать).
Одномерный массив может также иметь \emph{указатель заполнения}.

Multidimensional arrays store their components in row-major order;
that is, internally a multidimensional array is stored as a one-dimensional
array, with the multidimensional index sets ordered lexicographically,
last index varying fastest.  This is important in two situations:
(1) when arrays with different dimensions share their contents, and
(2) when accessing very large arrays in a virtual-memory implementation.
(The first situation is a matter of semantics; the second, a matter
of efficiency.)

Многомерные массивы сохраняют элементы построчно;
это значит, что внутренне многомерный массив хранится как одномерный массив с
порядком элементов, соответствующим лексикографическому порядку их индексов. Это
важно в двух ситуациях:
(1) когда массивы с разными измерениями имеют общее содержимое, и 
(2) когда осуществляется доступ к очень большому массиву в виртуальной памяти.
(Первая ситуация касается семантики; вторая -- эффективности)

An array that is not displaced to another array, has no fill pointer, and
is not to have its size adjusted dynamically after creation is called a
\emph{simple} array.  The user may provide declarations that certain arrays
will be simple.  Some implementations can handle simple arrays in an
especially efficient manner; for example, simple arrays may have a more
compact representation than non-simple arrays.

Массив, что не указывает на другой массив, не имеет указателя заполнения и не
имеет динамически расширяемого размера после создания называется \emph{простым}
массивом. Пользователи могут декларировать то, что конкретный массив будет
простым. Некоторые реализации могут обрабатывать простые массивы более
эффективным способом; например, простые массивы могут храниться более компактно,
что непростые массивы. 

\begin{newer}
X3J13 voted in June 1989
\issue{ADJUST-ARRAY-NOT-ADJUSTABLE}
to clarify that if one or more of the \cd{:adjustable}, \cd{:fill-pointer},
and \cd{:displaced-to} arguments is true when \cd{make-array}
is called, then whether the resulting
array is simple is unspecified; but if all three arguments are false,
then the resulting array is guaranteed to be simple.

Если один или более из \cd{:adjustable}, \cd{:fill-pointer} и
\cd{:displaced-to} аргументов равен true, когда вызывается \cd{make-array},
тогда является ли результат простым массивом не определено; однако если все три
аргумента равны false, тогда результат гарантированно будет простым массивом.
\end{newer}

\subsection{Vectors Векторы}

One-dimensional arrays are called \emph{vectors} in Common Lisp
and constitute the type \cd{vector} (which is therefore a subtype of \cd{array}).
Vectors and lists are collectively considered to be
\emph{sequences}.  They differ in that any component of a one-dimensional array
can be accessed in constant time,
whereas the average component access time for a
list is linear in the length of the list; on the other hand, adding a new
element to the front of a list takes constant time, whereas the same
operation on an array takes time linear in the length of the array.

В Common Lisp'е одномерные массивы называется \emph{векторами}, и составляют тип
\cd{вектор} (который в свою очередь является подтипом \cd{массива}).
Вектора и списки вместе являются \emph{последовательностями}. Они отличаются тем,
что любой элемент одномерного массива может быть получен за константное время,
тогда как среднее время доступа к компоненту для списка линейно зависит от длины
списка, с другой стороны, добавление нового элемента в начала списка занимает
константное время, тогда как эта же операция для массива занимает время линейно
зависящее от длины массива.

A general vector (a one-dimensional array
that can have any data object as an element but that has
no additional paraphernalia) can be notated by notating the
components in order, separated by whitespace and surrounded by \cd{\#(}
and \cd{)}.
For example:

Обычный вектор (одномерный массив, который может содержать любой тип объектов,
но не имеющий дополнительных атрибутов) может быть записан с помощью
перечисления элементов разделенных пробелом и окруженных \cd{\#(} и
\cd{)}.
Например:
\begin{lisp}
\#(a b c)~~~~~~~~~~~~~~~~~~~~;{\rm A vector of length 3} \\*
\#()~~~~~~~~~~~~~~~~~~~~~~~~~;{\rm An empty vector} \\
\#(2 3 5 7 11 13 17 19 23 29 31 37 41 43 47) \\*
~~~~~~~~~~~~~~~~~~~~~~~~~~~~;{\rm A vector containing the primes below 50}
\end{lisp}

\begin{lisp}
\#(a b c)~~~~~~~~~~~~~~~~~~~~;{\rm Вектор из трех элементов} \\*
\#()~~~~~~~~~~~~~~~~~~~~~~~~~;{\rm Путой вектор} \\
\#(2 3 5 7 11 13 17 19 23 29 31 37 41 43 47) \\*
~~~~~~~~~~~~~~~~~~~~~~~~~~~~;{\rm Вектор содержит простые числа меньшие пятидесяти}
\end{lisp}
Note that when the function \cd{read} parses this syntax, it always constructs
a \emph{simple} general vector.

Следует отметить, что когда функция \cd{read} парсит данный синтаксис, она
всегда создает \emph{простой} массив.

\beforenoterule
\begin{rationale}
Many people have suggested that brackets be used
to notate vectors, as \cd{{\Xlbracket}a b c{\Xrbracket}}
instead of \cd{\#(a b c)}.  This notation
would be shorter, perhaps more readable, and certainly in accord with
cultural conventions in other parts of computer science and mathematics.
However, to preserve the usefulness of the user-definable macro-character
feature of the function \cd{read}, it is necessary to leave some
characters to the user for this purpose.  Experience in MacLisp has
shown that users, especially implementors of languages for use
in artificial intelligence research, often want
to define special kinds of brackets.  Therefore Common Lisp avoids using
brackets and braces for any syntactic purpose.

Многие люди рекомендовали использовать квадратные скобки для задания векторов
так: \cd{{\Xlbracket}a b c{\Xrbracket}} вместо \cd{\#(a b c)}. Данная запись
короче, возможно более читаема, и безусловно совпадает с культурными традициями
в других частях компьютерных наук и математики. Однако, для достижения
предельной полезности от пользовательских макросимволов, что расширяют
возможности функции \cd{read}, необходимо было оставить некоторые строковые
символы для этих пользовательских целей. Опыт использования MacLisp'а
показывает, что пользователи, особенно разработчики языков для использования в
исследованиях искусственного интеллекта, часто хотят определять специальные
значения для квадратных скобос. Таким образом Common Lisp не использует
квадратных и фигурных скобок в своем синтаксисе.
\end{rationale}
\afternoterule

Implementations may provide certain specialized representations of
arrays for efficiency in the case where all the components are of
the same specialized (typically numeric) type.  All implementations
provide specialized arrays for the cases when the components
are characters (or rather, a special subset of the characters);
the one-dimensional instances of
this specialization are called \emph{strings}.
All implementations are also required to provide specialized arrays
of bits, that is, arrays of type \cd{(array bit)};
the one-dimensional instances of
this specialization are called \emph{bit-vectors}.

Реализации могут предоставлять специализированные представления массивов для
достижения эффективности в случаях, когда все элементы принадлежат одному
определенному типу (например, числовому). Все реализации предоставляют
специальные массивы в случаях, когда все элементы являются строковыми символами
(или специализированное подмножество строковых символов);
такие одномерные массивы называются \emph{строки}.
Все реализации также должны предоставлять специализированные битовые массивы,
которые принадлежат типу \cd{(array bit)};
такие одномерные массивы назваются \emph{битовые векторы}.

\subsection{Strings Строки}
\label{STRING-TYPE-SECTION}

\begin{obsolete}
A string is simply a vector of characters.
More precisely, a string is a specialized vector whose elements
are of type \cd{string-char}.
\end{obsolete}
\begin{newer}
X3J13 voted in March 1989 \issue{CHARACTER-PROPOSAL}
to eliminate the type \cd{string-char} and to redefine the type
\cd{string} to be the union of one or more specialized vector
types, the types of whose elements are subtypes of the type \cd{character}.
Subtypes of \cd{string} include \cd{simple-string}, \cd{base-string},
and \cd{simple-base-string}.

\vskip 3pt
\begin{lisp}
base-string \EQ\ (vector base-character) \\*
simple-base-string \EQ\ (simple-array base-character (*))
\end{lisp}
An implementation may support
other string subtypes as well.  All Common Lisp functions that operate
on strings treat all strings uniformly; note, however,
that it is an error to attempt to insert
an extended character into a base string.
\end{newer}

\newpage%manual

The type \cd{string} is therefore a subtype of the type \cd{vector}.

\cd{Строковый} тип является подтипом \cd{векторного} типа.

A string can be written as the sequence of characters contained in the
string, preceded and followed by a \cd{{\Xdquote}} (double quote) character.
Any \cd{{\Xdquote}} or \cd{{\Xbackslash}} character in the sequence must additionally
have a \cd{{\Xbackslash}} character before it.

Строка может быть записана как последовательность символов содержащихся в
строке, с предшествующим и последующим символом двойной кавычки \cd{{\Xdquote}}.
Любой символ \cd{{\Xdquote}} или \cd{{\Xbackslash}} в данной последовательности должен
иметь предшествующий символ \cd{{\Xbackslash}}.

For example:

Например:
\begin{lisp}
{\Xdquote}Foo{\Xdquote}~~~~~~~~~~~~~~~~~~~~~~~~~;{\rm A string with three characters in it} \\*
{\Xdquote}{\Xdquote}~~~~~~~~~~~~~~~~~~~~~~~~~~~~;{\rm An empty string} \\
{\Xdquote}{\Xbackslash}{\Xdquote}APL{\Xbackslash}{\Xbackslash}360?{\Xbackslash}{\Xdquote} he cried.{\Xdquote}~~~~~;{\rm A string with twenty characters} \\*
{\Xdquote}|x| = |-x|{\Xdquote}~~~~~~~~~~~~~~~~~~;{\rm A ten-character string}
\end{lisp}

\begin{lisp}
{\Xdquote}Foo{\Xdquote}~~~~~~~~~~~~~~~~~~~~~~~~~;{\rm Строка из трех символов} \\*
{\Xdquote}{\Xdquote}~~~~~~~~~~~~~~~~~~~~~~~~~~~~;{\rm Пустая строка} \\
{\Xdquote}{\Xbackslash}{\Xdquote}APL{\Xbackslash}{\Xbackslash}360?{\Xbackslash}{\Xdquote} he
cried.{\Xdquote}~~~~~;{\rm Строка из двенадцати символов} \\*
{\Xdquote}|x| = |-x|{\Xdquote}~~~~~~~~~~~~~~~~~~;{\rm Строка из десяти символов}
\end{lisp}
Notice that any vertical bar \cd{|} in a string need not be
preceded by a \cd{{\Xbackslash}}.  Similarly, any double quote in the name
of a symbol written using vertical-bar notation need not be
preceded by a \cd{{\Xbackslash}}.  The double-quote and vertical-bar notations
are similar but distinct: double quotes indicate a character string
containing the sequence of characters,
whereas vertical bars indicate a symbol whose name is the contained
sequence of characters.

Необходимо отметить, что символ вертикальной черты \cd{|} в строке не должен быть
экранирован с помощью \cd{{\Xbackslash}}. Также как и любая двойная кавычка в имени
символа, записанного с использованием вертикальных черт, не нуждается в
экранировании. Записи с помощью двойной кавычки и вертикальной черты похожи, но
используются для разных целей: двойная кавычка указывает на строку, содержающую
строковые символы, тогда как вертикальная черта указывает на символ, имя
которого содержит последовательность строковых символов.

The characters contained by the double quotes, taken from left to right,
occupy locations within the string with increasing indices.
The leftmost character is string element number 0, the next one
is element number 1, the next one is element number 2, and so on.

Строковые символы обрамленные двойными кавычками, считываются слева
направо. Индекс символа больше индекса предыдущего символа на 1. Самый левый
символ строки имеет индекс 0, следующий 1, следующий 2, и т.д.

Note that the function \cd{prin1} will print any character vector
(not just a simple one)
using this syntax, but the function \cd{read} will always construct
a simple string when it reads this syntax.

Следует отметить, что функция \cd{prin1} будет выводить на печать любой вектор
строковых символов (не только простой), используя данный синтаксис, но функция
\cd{read} будет всегда создавать простую строку, при разборе данного синтаксиса.

\subsection{Bit-Vectors}

A bit-vector can be written as the sequence of bits contained in the
string, preceded by \cd{\#*}; any delimiter character, such as whitespace,
will terminate the bit-vector syntax.
For example:

Битовый вектор может быть записан в виде последовательности битов заключенных в
строку с предшествующей \cd{\#*}; любой разделитится, например, как пробел
завершает синаксис битового вектора.
Например:
\begin{lisp}
\#*10110~~~~~;{\rm A five-bit bit-vector; bit 0 is a 1} \\
\#*~~~~~~~~~~;{\rm An empty bit-vector}
\end{lisp}

\begin{lisp}
\#*10110~~~~~;{\rm Пятибитный битовый вектор; нулевой бит 1} \\
\#*~~~~~~~~~~;{\rm Пустой битовый вектор}
\end{lisp}

The bits notated following the \cd{\#*}, taken from left to right,
occupy locations within the bit-vector with increasing indices.
The leftmost notated bit is bit-vector element number 0, the next one
is element number 1, and so on.

Биты записанные после \cd{\#*}, читаются слева направо. Индекс каждого бита
больше индекса предыдущего бита на 1. Индекс самого левого бита 0, следующего 1
и т.д.

The function \cd{prin1} will print any bit-vector (not just a simple one)
using this syntax, but the function \cd{read} will always construct
a simple bit-vector when it reads this syntax.

Функция \cd{prin1} распечатывают любой битовый вектор (не только простой)
используя этот синтаксис, однако функция \cd{read} будет всегда создавать простой
битовый веткор, когда разбирает данный синтаксис.

\section{Hash Tables Хеш-таблицы}
Hash tables provide an efficient way of mapping any
Lisp object (a \emph{key}) to an associated object.
They are provided as primitives of Common Lisp because
some implementations may need to use internal storage
management strategies that would make it very difficult
for the user to implement hash tables in a portable fashion.
Hash tables are described in chapter~\ref{HASH}.

Хеш-таблицы предоставляют эффективный путь для связи любого Lisp объекта ({\it
 ключа}) с другим объектом. Они предоставляются как примитивы Common Lisp'а,
потому что некоторые реализации могут нуждаться в использовании стратегий
управления внутренними хранилищами, что создало бы сложности для пользователя в
реализации портируемых хеш-таблиц.
Хеш-таблицы описаны в главе~\ref{HASH}.

\section{Readtables Readtables}

A readtable is a data structure that maps characters into syntax
types for the Lisp expression parser.
In particular, a readtable indicates for
each character with syntax \emph{macro character} what its macro
definition is.  This is a mechanism by which the user may reprogram
the parser to a limited but useful extent.
See section~\ref{READTABLE-SECTION}.

Readtable является структурой данных, которая отображает символы в
синтаксические типы для парсера Lisp выражений.
В частности, readtable указывает для каждого строкового символа с синтаксисом
\emph{макросимвола}, какой макрос ему соотвествует. Это механизм, с помощью
которого пользователь может запрограммировать парсер для выполнения
ограниченных, но полезных расширений.
Смотрите раздел~\ref{READTABLE-SECTION}.

\section{Packages Пакеты}

Packages are collections of symbols that serve as name spaces.
The parser recognizes symbols by looking up character sequences
in the current package.  Packages can be used to hide
names internal to a module from other code.  Mechanisms are provided
for exporting symbols from a given package to the primary ``user'' package.
See chapter~\ref{XPACK}.

Пакеты являются коллекциями символов, которые предоставлены в качестве
пространства имен. Парсер распознает символы с помощью поиска строки в текущем
пакете. Пакеты могут использоваться для скрытия имен внутрь модуля от другого
кода. Также предоставляются механизмы для экспортирования символов из заданного
пакета в главный <<user>> пакет.
Смотрите главу~\ref{XPACK}.

\section{Pathnames Имена файлов}
Pathnames are the means by which a Common Lisp program can
interface to an external file system in a reasonably implementation-independent
manner.  See section~\ref{PATHNAME}.

Имена файлов являеются сущностями, с помощью которых Common Lisp программа может
взаимодействовать с внешней файловой системой в приемлемой платформонезависимой
форме. Смотрите раздел~\ref{PATHNAME}.

\section{Streams Потоки}

A stream is a source or sink of data, typically characters or bytes.
Nearly all functions that perform I/O do so with respect to a specified
stream.  The function \cd{open} takes a pathname and returns a stream
connected to the file specified by the pathname.
There are a number of standard streams that are used by default for
various purposes.  See chapter~\ref{STREAM}.

Поток является источником или набором данных, обычно строковых символов или
байтов. Почти все функции, что выполняют ввод/вывод, делают это в отношении
заданного потока. Функция \cd{open} принимает путь к файлу и возвращается поток
подключенный к файлу заданному в параметре.
Существует несколько стандартных потоков, которые используются по умолчанию для
различных целей. Смотрите главу~\ref{STREAM}.

\begin{newer}
X3J13 voted in January 1989
\issue{STREAM-ACCESS}
to introduce subtypes of type \cd{stream}:
\cd{broadcast-stream}, \cd{concatenated-stream},
\cd{echo-stream}, \cd{synonym-stream}, \cd{string-stream}, \cd{file-stream},
and \cd{two-way-stream} are disjoint subtypes of \cd{stream}.
Note particularly that a synonym stream is always and only of type
\cd{synonym-stream}, regardless of the type of the stream for which it is a synonym.
\end{newer}

\section{Random-States Random-States}

An object of type \cd{random-state} is used to encapsulate
state information used by the pseudo-random number generator.
For more information about \cd{random-state} objects,
see section~\ref{RANDOM}.

Объект типа \cd{random-state} используется для инкапсулирования информации о
состоянии, используемом генератором случайных чисел (ГСЧ). Для боле подробной
информации об объектах \cd{random-state} смотрите главу~\ref{RANDOM}.

\section{Structures Структуры}

Structures are instances of user-defined data types that have
a fixed number of named components.  They are analogous to
records in Pascal.
Structures are declared using the \cd{defstruct} construct;
\cd{defstruct} automatically defines access and constructor functions for
the new data type.

Структуры являются экземплярами определенных пользователем типов данных, которые
имеют ограниченное количество именованных полей (свойств). Они являются
аналогами записей в Pascal'е.
Структуры декларируются используя конструкцию \cd{defsctruct};
\cd{defstruct} автоматически определяет конструктор и функции доступа для нового
типа данных.

Different structures may print out in different ways;
the definition of a structure type may specify a print procedure
to use for objects of that type (see the
\cd{:print-function} option to \cd{defstruct}).
The default notation for structures is

Различные структуры могут выводится на печать различными способами;
определение типа структуры может задавать процедуру вывода на печать для
объектов данного типа (смотрите опцию \cd{:print-function} для \cd{defstruct}).
Записью по умолчанию для структур является:
\begin{lisp}
\#S(\emph{structure-name} \\
~~~~~~~~\emph{slot-name-1} \emph{slot-value-1} \\
~~~~~~~~\emph{slot-name-2} \emph{slot-value-2} \\
~~~~~~~~~~~~~~~~~~~~~~...)
\end{lisp}
where \cd{\#S} indicates structure syntax, \emph{structure-name} is
the name (a symbol) of the structure type, each \emph{slot-name} is the name
(also a symbol) of a component, and each corresponding \emph{slot-value}
is the representation of the Lisp object in that slot.

\begin{lisp}
\#S(\emph{имя-структуры} \\
~~~~~~~~\emph{имя-слота-1} \emph{значение-слота-1} \\
~~~~~~~~\emph{имя-слота-2} \emph{значение-слота-2} \\
~~~~~~~~~~~~~~~~~~~~~~...)
\end{lisp}
где \cd{\#S} указывает на синтаксис структуры, \emph{имя-структуры} является
именем (символ) типа данной структуры, каждый \emph{имя-слота} является именем
слота (также символ), и каждое соответствующее \emph{значение-слота} --
отображением Lisp объекта в данном слоте.

\section{Functions Функции}
\label{FUNCTION-TYPE-SECTION}

\begin{obsolete}
A \emph{function} is anything that may be correctly given to the \cd{funcall}
or \cd{apply} function, and is
to be executed as code when arguments are supplied.

A \emph{compiled-function} is a compiled code object.

A lambda-expression
(a list whose \emph{car} is the symbol \cd{lambda}) may serve as a function.
Depending on the implementation, it may be possible for other lists to
serve as functions.  For example, an implementation might choose to
represent a ``lexical closure'' as a list whose \emph{car} contains some
special marker.

A symbol may serve as a function; an attempt to invoke a symbol as a function
causes the contents of the symbol's function cell to be used.
See \cd{symbol-function} and \cd{defun}.

The result of evaluating a \cd{function} special form
will always be a function.
\end{obsolete}

\begin{newer}
X3J13 voted in June 1988 \issue{FUNCTION-TYPE}
to revise these specifications.  The type \cd{function} is to be disjoint
from \cd{cons} and \cd{symbol}, and so a list whose \emph{car} is \cd{lambda}
is not, properly speaking, of type \cd{function}, nor is any symbol.
However,
standard Common Lisp functions that accept functional arguments
will accept a symbol or a list whose \emph{car} is \cd{lambda}
and automatically coerce it to be a function; such standard
functions include \cd{funcall}, \cd{apply}, and \cd{mapcar}.
Such functions do not, however, accept a lambda-expression as a functional
argument; therefore one may not write

\vskip 3pt
\begin{lisp}
(mapcar '(lambda (x y) (sqrt (* x y))) p q)
\end{lisp}
but instead one must write something like
\begin{lisp}
(mapcar \#'(lambda (x y) (sqrt (* x y))) p q)
\end{lisp}

This change makes it impermissible to represent a lexical closure
as a list whose \emph{car} is some special marker.

The value of a \cd{function} special form
will always be of type \cd{function}.
\end{newer}

\section{Unreadable Data Objects Нечитаемые объекты данных}

Some objects may print in implementation-dependent ways.
Such objects cannot necessarily be reliably reconstructed from
a printed representation, and so they are usually printed in
a format informative to the user but not acceptable to the \cd{read} function:
\cd{\#<\emph{useful information}>}.
The Lisp reader will signal an error on encountering \cd{\#<}.

Некоторые объекты могут быть выведены на печать в виде, который зависит от
реализации.
Такие объекты не могут быть полностью реконструированы из распечатанной формы,
так как они обычно распечатываются в форме информативной для пользователя, но не
подходящей для функции \cd{read}:
\cd{\#<\emph{полезная информация}}.

As a hypothetical example, an implementation might print
\begin{lisp}
\#<stack-pointer si:rename-within-new-definition-maybe \#o311037552>
\end{lisp}
for an implementation-specific ``internal stack pointer'' data type
whose printed representation includes the name of the type,
some information about the stack slot pointed to, and the machine address
(in octal) of the stack slot.

\begin{newer}
See \cd{print-unreadable-object}, a macro that prints an object using \cd{\#<}
syntax.
\end{newer}

\section{Overlap, Inclusion, and Disjointness of Types}
\label{DATA-TYPE-RELATIONSHIPS}

The Common Lisp data type hierarchy is tangled and purposely left somewhat
open-ended so that implementors may experiment with new data types
as extensions to the language.  This section explicitly states all
the defined relationships between types, including subtype/supertype
relationships,
disjointness, and exhaustive partitioning.  The user of Common Lisp
should not depend on any relationships not explicitly stated here.
For example, it is not valid to assume that because a number
is not complex and not rational that it must be a \cd{float}, because
implementations are permitted to provide yet other kinds of numbers.

First we need some terminology.
If \emph{x} is a supertype of \emph{y}, then any object of type \emph{y} is also
of type \emph{x}, and \emph{y} is said to be a subtype of \emph{x}.  If types
\emph{x} and \emph{y} are disjoint, then no object (in any implementation) may
be both of type \emph{x} and of type \emph{y}.  Types $\emph{a}_1$ through
$\emph{a}_{\hbox{\scriptsize\it n}}$ are an \emph{exhaustive union}
of type \emph{x} if each $\emph{a}_j$
is a subtype of \emph{x}, and any object of type \emph{x} is
necessarily of at least one of the types $\emph{a}_{\hbox{\scriptsize\it j}}$;
$\emph{a}_1$ through $\emph{a}_{\hbox{\scriptsize\it n}}$ are furthermore an \emph{exhaustive partition}
if they are also pairwise disjoint.

\begin{itemize}
\item
The type \cd{t} is a supertype of every type whatsoever.
Every object is of type \cd{t}.

\item
The type {\nil} is a subtype of every type whatsoever.
No object is of type {\nil}.
\end{itemize}

\begin{obsolete}
\begin{itemize}
\item
The types \cd{cons}, \cd{symbol}, \cd{array}, \cd{number}, and \cd{character}
are pairwise disjoint.
\end{itemize}
\end{obsolete}

\begin{new}
X3J13 voted in June 1988
\issue{DATA-TYPES-HIERARCHY-UNDERSPECIFIED}
to extend the preceding paragraph as follows.

\begin{itemize}
\item
The types \cd{cons}, \cd{symbol}, \cd{array}, \cd{number}, \cd{character},
\cd{hash-table}, \cd{readtable}, \cd{package}, \cd{pathname},
\cd{stream}, \cd{random-state}, and any single other type created by
\cd{defstruct} or \cd{defclass}
are pairwise disjoint.
\end{itemize}

The wording of the first edition was intended to allow implementors to use
the \cd{defstruct} facility to define the built-in types \cd{hash-table},
\cd{readtable}, \cd{package}, \cd{pathname}, \cd{stream}, \cd{random-state}.
The change still permits this implementation strategy but
forbids these built-in types from including, or being included in,
other types (in the sense of the \cd{defstruct} \cd{:include} option).
\end{new}

\begin{new}
X3J13 voted in June 1988 \issue{FUNCTION-TYPE}
to specify that the type \cd{function}
is disjoint from the types \cd{cons}, \cd{symbol}, \cd{array}, \cd{number}, and \cd{character}.
The type \cd{compiled-function} is a subtype of \cd{function};
implementations are free to define other subtypes of \cd{function}.
\end{new}

\begin{obsolete}
\begin{itemize}
\item
The types \cd{rational}, \cd{float}, and \cd{complex} are pairwise disjoint
subtypes of \cd{number}.
\end{itemize}
\end{obsolete}

\begin{newer}
X3J13 voted in March 1989 \issue{REAL-NUMBER-TYPE} to rewrite the preceding item
as follows.
\begin{itemize}
\item
The types \cd{real} and \cd{complex} are pairwise disjoint
subtypes of \cd{number}.
\end{itemize}

\beforenoterule
\begin{rationale}
It might be thought that \cd{real} and \cd{complex} should
form an exhaustive partition of the type \cd{number}.  This is purposely
avoided here in order to permit compatible experimentation with extensions
to the Common Lisp number system.
\end{rationale}
\afternoterule

\begin{itemize}
\item
The types \cd{rational} and \cd{float} are pairwise disjoint
subtypes of \cd{real}.
\end{itemize}

\beforenoterule
\begin{rationale}
It might be thought that \cd{rational} and \cd{float} should
form an exhaustive partition of the type \cd{real}.  This is purposely
avoided here in order to permit compatible experimentation with extensions
to the Common Lisp number system.
\end{rationale}
\afternoterule
\end{newer}

\begin{itemize}
\item
The types \cd{integer} and \cd{ratio} are disjoint subtypes of \cd{rational}.
\end{itemize}

\beforenoterule
\begin{rationale}
It might be thought that \cd{integer} and \cd{ratio} should
form an exhaustive partition of the type \cd{rational}.  This is purposely
avoided here in order to permit compatible experimentation with extensions
to the Common Lisp rational number system.
\end{rationale}
\afternoterule

\begin{obsolete}
\begin{itemize}
\item
The types \cd{fixnum} and \cd{bignum} are disjoint subtypes of \cd{integer}.
\end{itemize}

\beforenoterule
\begin{rationale}
It might be thought that \cd{fixnum} and \cd{bignum} should
form an exhaustive partition of the type \cd{integer}.  This is purposely
avoided here in order to permit compatible experimentation with
extensions to the Common Lisp integer number system, such as the idea of
adding explicit representations of infinity or of positive and negative
infinity.
\end{rationale}
\afternoterule
\end{obsolete}

\begin{new}
X3J13 voted in January 1989
\issue{FIXNUM-NON-PORTABLE}
to specify that the types \cd{fixnum} and \cd{bignum}
do in fact form an exhaustive partition of the type \cd{integer}; more precisely,
they voted to specify that the type \cd{bignum} is by definition equivalent
to \cd{(and~integer (not~fixnum))}.  This is consistent with the
first edition text in section~\ref{INTEGERS-SECTION}.

I interpret this to mean that implementators could still experiment with
such extensions as adding explicit representations of infinity, but such infinities
would necessarily be of type \cd{bignum}.
\end{new}

\begin{itemize}
\item
The types \cd{short-float}, \cd{single-float}, \cd{double-float}, and
\cd{long-float} are subtypes of \cd{float}.  Any two of them must be
either disjoint or identical; if identical, then any other types between
them in the above ordering must also be identical to them
(for example, if \cd{single-float} and \cd{long-float} are identical types,
then \cd{double-float} must be identical to them also).

\item
The type \cd{null} is a subtype of \cd{symbol}; the only object of type
\cd{null} is {\nil}.

\item
The types \cd{cons} and \cd{null} form an exhaustive partition of the type
\cd{list}.
\end{itemize}

\begin{obsolete}
\begin{itemize}
\item
The type \cd{standard-char} is a subtype of \cd{string-char};
\cd{string-char} is a subtype of \cd{character}.
\end{itemize}
\end{obsolete}


\begin{newer}
X3J13 voted in March 1989 \issue{CHARACTER-PROPOSAL} to remove the type \cd{string-char}.
The preceding item is replaced by the following.
\begin{itemize}
\item
The type \cd{standard-char} is a subtype of \cd{base-character}.
The types \cd{base-character} and \cd{extended-character}
form an exhaustive partition of \cd{character}.
\end{itemize}
\end{newer}

\begin{obsolete}
\begin{itemize}
\item
The type \cd{string} is a subtype of \cd{vector}, for \cd{string}
means \cd{(vector string-char)}.
\end{itemize}
\end{obsolete}

\newpage%manual
\begin{newer}
X3J13 voted in March 1989 \issue{CHARACTER-PROPOSAL} to remove the type \cd{string-char}.
The preceding item is replaced by the following.
\begin{itemize}
\item
The type \cd{string} is a subtype of \cd{vector}; it is the union of
all types \cd{(vector~\emph{c})} such that \emph{c} is a subtype of \cd{character}.
\end{itemize}
\end{newer}

\begin{itemize}
\item
The type \cd{bit-vector} is a subtype of \cd{vector}, for \cd{bit-vector}
means \cd{(vector bit)}.

\item
The types \cd{(vector t)}, \cd{string}, and \cd{bit-vector} are disjoint.

\item
The type \cd{vector} is a subtype of \cd{array}; for all types \emph{x},
the type \cd{(vector \emph{x})} is the same as the type \cd{(array \emph{x} (*))}.

\item
The type \cd{simple-array} is a subtype of \cd{array}.
\end{itemize}

\begin{obsolete}
\begin{itemize}
\item
The types \cd{simple-vector}, \cd{simple-string}, and
\cd{simple-bit-vector} are disjoint subtypes of \cd{simple-array}, for they
respectively mean \cd{(simple-array t (*))}, \cd{(simple-array string-char (*))},
and \cd{(simple-array bit (*))}.
\end{itemize}
\end{obsolete}

\begin{newer}
X3J13 voted in March 1989 \issue{CHARACTER-PROPOSAL} to remove the type \cd{string-char}.
The preceding item is replaced by the following.
\begin{itemize}
\item
The types \cd{simple-vector}, \cd{simple-string}, and
\cd{simple-bit-vector} are disjoint subtypes of \cd{simple-array}, for they
mean \cd{(simple-array t (*))}, the union of all types
\cd{(simple-array \emph{c} (*))} such that \emph{c} is a subtype of \cd{character},
and \cd{(simple-array bit (*))}, respectively.
\end{itemize}
\end{newer}

\begin{itemize}
\item
The type \cd{simple-vector} is a subtype of \cd{vector} and indeed
is a subtype of \cd{(vector t)}.

\item
The type \cd{simple-string} is a subtype of \cd{string}.
(Note that although \cd{string} is a subtype of \cd{vector},
\cd{simple-string} is not a subtype of \cd{simple-vector}.)
\end{itemize}

\beforenoterule
\begin{rationale}
The hypothetical name \cd{simple-general-vector} would have been more accurate than
\cd{simple-vector}, but in this instance euphony and
user convenience were deemed more important to the design
of Common Lisp than a rigid symmetry.
\end{rationale}
\afternoterule

\begin{itemize}
\item
The type \cd{simple-bit-vector} is a subtype of \cd{bit-vector}.
(Note that although \cd{bit-vector} is a subtype of \cd{vector},
\cd{simple-bit-vector} is not a subtype of \cd{simple-vector}.)

\item
The types \cd{vector} and \cd{list} are disjoint subtypes of \cd{sequence}.

\item
The types \cd{random-state}, \cd{readtable}, \cd{package}, \cd{pathname},
\cd{stream}, and \cd{hash-table} are pairwise disjoint.
\end{itemize}

\begin{new}
X3J13 voted in June 1988
\issue{DATA-TYPES-HIERARCHY-UNDERSPECIFIED}
to make \cd{random-state}, \cd{readtable}, \cd{package}, \cd{pathname},
\cd{stream}, and \cd{hash-table}
pairwise disjoint from a number of other types as well;
see note above.
\end{new}

\begin{new}
X3J13 voted in January 1989
\issue{STREAM-ACCESS}
to introduce subtypes of type \cd{stream}.

\begin{itemize}
\item
The types \cd{two-way-stream}, \cd{echo-stream},
\cd{broadcast-stream}, \cd{file-stream}, \cd{synonym-stream}, \cd{string-stream}, and
\cd{concatenated-stream} are disjoint subtypes of \cd{stream}.
\end{itemize}
\end{new}

\begin{itemize}
\item
Any two types created by \cd{defstruct} are disjoint unless
one is a supertype of the other by virtue of
the \cd{:include} option.
\end{itemize}

\begin{obsolete}
\begin{itemize}
\item
An exhaustive union for the type \cd{common} is formed by the types
\cd{cons}, \cd{symbol}, \cd{(array \emph{x})} where \emph{x} is either {\true} or 
a subtype
of \cd{common}, \cd{string}, \cd{fixnum}, \cd{bignum}, \cd{ratio},
\cd{short-float}, \cd{single-float}, \cd{double-float}, \cd{long-float},
\cd{(complex \emph{x})} where \emph{x} is a
subtype of \cd{common},
\cd{standard-char}, \cd{hash-table}, \cd{readtable}, \cd{package}, \cd{pathname},
\cd{stream}, \cd{random-state}, and all types created by the user
via \cd{defstruct}.
An implementation may not unilaterally add subtypes to
\cd{common}; however, future revisions to the Common Lisp standard may
extend the definition of the \cd{common} data type.
Note that a type such as \cd{number} or \cd{array} may or may
not be a subtype of \cd{common}, depending on whether or not the given
implementation has extended the set of objects of that type.
\end{itemize}
\end{obsolete}

\begin{newer}
X3J13 voted in March 1989
\issue{COMMON-TYPE}
to remove the type \cd{common} from the language.
\end{newer}
      % Enumeration of data types
%Part{Scope, Root = "CLM.MSS"}
%Chapter of Common Lisp Manual.  Copyright 1984, 1988, 1989 Guy L. Steele Jr.

\clearpage\def\pagestatus{FINAL PROOF}

\chapter{Scope and Extent Область и продолжительность видимости}
\label{SCOPE}

In describing various features of the Common Lisp language, the notions of
{\it scope} and {\it extent} are frequently useful.  These notion arise when
some object or construct must be referred to from some distant part of a
program.  {\it Scope} refers to the spatial or textual region of the
program within which references may occur.  {\it Extent} refers to the
interval of time during which references may occur.

При описании различных возможностей Common Lisp'а очень важными понятиями
являются {\it области и продолжительности видимости}. Эти понятия возникают,
когда к некоторому объекту или конструкции необходимо обратиться где-то далеко в
коде. {\it Область видимости} отмечает пространственный или текстовый регион, в
котором находящаяся внутри программа может ссылатся на эти 
объекты. {\it Продолжительность видимости} обозначает временной интервал, в течении
которого программа может ссылаться к данным.

As a simple example, consider this program:

Вот простой пример такой программы:
\begin{lisp}
(defun copy-cell (x) (cons (car x) (cdr x)))
\end{lisp}
The scope of the parameter named \cd{x} is the body of the \cd{defun} form.
There is no way to refer to this parameter from any other place but within
the body of the \cd{defun}.  Similarly, the extent of the parameter \cd{x}
(for any particular call to \cd{copy-cell}) is the interval from the time
the function is invoked to the time it is exited.  (In the general case,
the extent of a parameter may last beyond the time of function exit,
but that cannot occur in this simple case.)

Областью видимости параметра с именем \cd{x} является тело формы \cd{defun}.
Способа сослатся на этот параметр из какого-либо другого места программы
нет. Продолжительностью видимости параметра \cd{x} (для какого-нибудь вызова
\cd{copy-cell}) является интервал времени, начиная с вызова функции и заканчивая
выходом из нее. (В общем случае продолжительность видимости параметра может
продлиться и после завершения функции, но в данном простом случае такого не
может быть.) 

Within Common Lisp, a referenceable entity is {\it established} by the execution
of some language construct, and the scope and extent of the entity are
described relative to the construct and the time (during execution of the
construct) at which the entity is established.
For the purposes of this discussion, the term ``entity'' refers not only
to Common Lisp data objects, such as symbols and conses, but also to
variable bindings (both ordinary and special), catchers,
and \cd{go} targets.  It is important to distinguish between
an entity and a name for the entity.  In a function definition
such as

В Common Lisp сущность, на которую можно сослаться из кода, {\it создается} с
помощью специальных языковых конструкций, и область и продолжительность
видимости описывются в зависимости от этой конструкции и времени (выполнения
конструкции) в которое эта сущносность была создана.
Для предмета данного описания, термин <<сущность>> указывает не только на
объекты Common Lisp'а, такие как символы и cons-ячейки, но и также на связывания
переменных (обычных и специальных), ловушки, и \cd{метки переходов}. Важно
отметить различие между сущностью и именем для этой сущности. В определение
функции, такой как:

\begin{lisp}
(defun foo (x y) (* x (+ y 1)))
\end{lisp}

there is a single name, \cd{x}, used to refer to the first parameter
of the procedure whenever it is invoked; however, a new binding
is established on every invocation.  A {\it binding} is a particular
parameter instance.  The value of a reference to the name \cd{x}
depends not only on the scope within which it occurs (the one in
the body of \cd{foo} in the example occurs in the scope of the
function definition's parameters) but also on the particular
binding or instance involved.  (In this case, it depends on the
invocation during which
the reference is made).  More complicated examples
appear at the end of this chapter.

существует только одно имя, \cd{x}, используемое для ссылки на первый параметр
процедуры, когда бы они не была вызвана. {\it Связывание} --- это, в частности,
экземпляр параметра. Значение связанное с именем \cd{x} зависит не только от
области видимости, в которой данная связь возникла (в данном примере в теле
функции \cd{foo} связь возникла в области видимости определения параметров
функции), но также, в частности, от механизма связывания. (В данном случае,
значение зависит от вызова функции, в течение которого создается ссылка). Более
сложный пример приводится в конце данной главы. 

There are a few kinds of scope and extent that are particularly useful
in describing Common Lisp:

Вот некоторые виды областей и продолжительностей видимости, которые, в частности,
полезны при описании Common Lisp'а:

\begin{itemize}
\item
{\it Lexical scope}.  Here references to the established
entity can occur only within certain program portions that are
lexically (that is, textually) contained within the establishing construct.
Typically the construct will have a part designated the {\it body},
and the scope of all entities established will be (or include) the body.

Example: the names of parameters to a function normally are lexically scoped.

\item 
{\it Лексическая область видимости}. В ней связи к установленным сущностям могут
использоваться только в той части программы, которая лексически (т.е. текст
программы) находится в конструкции устанавливающей данные связи. Обычно эта
конструкция будет содержать часть, определенную как {\it тело (body)}, и область
видимости всех сущностей будет установлена только в этом теле.

Например: имена параметров в функции обычно ограничиваются лексической областью
видимости.

\item
{\it Indefinite scope}.  References may occur anywhere, in any program.

\item
{\it Неограниченная область видимости}. Ссылка на сущность может производится в
любом месте программы.

\item
{\it Dynamic extent}.  References may occur at any time in the interval
between establishment of the entity and the explicit disestablishment
of the entity.  As a rule, the entity is disestablished when execution
of the establishing construct completes or is otherwise terminated.
Therefore entities with dynamic extent obey a stack-like discipline,
paralleling the nested executions of their establishing constructs.

Example: the \cd{with-open-file} construct opens a connection to a file
and creates a stream object to represent the connection.  The stream object
has indefinite extent, but the connection to the open file has dynamic extent:
when control exits the \cd{with-open-file} construct, either normally
or abnormally, the stream is automatically closed.

Example: the binding of a ``special'' variable has dynamic extent.

\item
{\it Динамическая продолжительность видимости}. Ссылки на сущности могут
производится в любое время на интервале между установкой сущности и явного
упразднения сущности. Как правило, сущность упраздняется, когда выполнение конструкции
завершается или как-либо прерывается. Таким образом, сущности с динамической
продолжительностью видимости подчиняются механизму типа стек, они распараллеливают
выполнение кода, вложенного в их конструкции.

Например: \cd{with-open-file} открывает соединение с файлом и создает объект
потока для отображения этого соединения. Объект потока имеет неограниченную область
видимости, но соединение с открытым файлом имеет динамическую продолжительность
видимости: когда выполнение в любом, нормальном или аварийном случае, выйдет за
рамки конструкции \cd{with-open-file}, поток будет автоматически закрыт.

Например: связывание <<специальной (special)>> переменной имеет динамическую
продолжительность видимости.

\item
{\it Indefinite extent}.  The entity continues to exist as long as the
possibility of reference remains.  (An implementation is free to
destroy the entity if it can prove that reference to it is no longer possible.
Garbage collection strategies implicitly employ such proofs.)

Example: most Common Lisp data objects have indefinite extent.

Example: the bindings of lexically scoped parameters of a function have
indefinite extent.  (By contrast, in Algol the bindings of lexically scoped
parameters of a procedure have dynamic extent.)
The function definition
\begin{lisp}
(defun compose (f g) \\*
~~\#'(lambda (x) \\*
~~~~~~(funcall f (funcall g x))))
\end{lisp}
when given two arguments, immediately returns a function as its value.
The parameter bindings for \cd{f} and \cd{g} do not disappear because the
returned function, when called, could still refer to those bindings.
Therefore
\begin{lisp}
(funcall (compose \#'sqrt \#'abs) -9.0)
\end{lisp}
produces the value \cd{3.0}.  (An analogous procedure would not necessarily work
correctly in typical Algol implementations or, for that matter,
in most Lisp dialects.)

\item
{\it Неограниченная продолжительности видимости}. Сущность продолжает существовать
пока существует возможность ссылаться на нее. (Реализации разрешается удалить
сущность, если она может доказать, что ссылка на нее более невозможна. Стратегии
сборщика мусора неявно используют такие доказательства.)

Например: большинство Common Lisp объектов имеют неограниченную продолжительность
видимости.

Например: связывание лексически замкнутых параметров функции имеет неограниченную
продолжительность видимости. (В отличие от Algol'а, где связывание лексически
замкнутых параметров процедуры имеют динамическую продолжительность видимости.)
Определение функции
\begin{lisp}
(defun compose (f g) \\*
~~\#'(lambda (x) \\*
~~~~~~(funcall f (funcall g x))))
\end{lisp}
при получении двух параметров, немедленно возвращает функции в качестве
результата.
Связи параметров для \cd{f} и \cd{g} не теряются, потому что возвращенная
функция, когда будет вызвана, будет продолжать ссылаться на эти связи.
Таким образом
\begin{lisp}
(funcall (compose \#'sqrt \#'abs) -9.0)
\end{lisp}
вернет значение \cd{3.0}. (Аналогичная процедура не захочет корректно работать в
типичной реализации Algol'а или, даже, в большинстве диалектов Lisp'а.)
\end{itemize}

In addition to the above terms, it is convenient to define {\it dynamic scope}
to mean {\it indefinite scope and dynamic extent}.  Thus we speak of
``special'' variables as having dynamic scope, or being dynamically scoped,
because they have indefinite scope and dynamic extent: a special variable
can be referred to anywhere as long as its binding is currently
in effect.

В дополение к вышеназванным терминам, удобно определить {\it динамическую
  область видимости}, которая означает {\it неограниченную область видимости и
  динамическую продолжительность видимости}. Следовательно мы говорим о
<<специальных (special)>> переменных, как об имеющих динамическую область
видимости или будучи динамически замкнутых FIXME, потому что они имеют
неограниченную область видимости и динамическую продолжительность видимости:
к специальным переменным можно сослаться из любой точки программы на протяжении
существования их связываний.

\begin{newer}
The term ``dynamic scope'' is a misnomer.  Nevertheless
it is both traditional and useful.

Термин <<динамическая область видимости>> некорректен. Как бы то ни было это и
устоялось, и удобно.
\end{newer}

The above definitions do not take into account the possibility of
{\it shadowing}.  Remote reference of entities is accomplished by using
{\it names} of one kind or another.  If two entities have the same name,
then the second may shadow the first, in which case an occurrence
of the name will refer to the second and cannot refer to the first.

Сказанное выше не рассматривает возможность {\it скрытия
  (shadowing)}. Далекие (FIXME) ссылки на сущности осуществляются с
использованием {\it имен} того или иного типа. Если две сущности имеют
одинаковое имя, тогда второе имя может скрыть первое, в таком случае ссылка с
помощью этого имени будет осуществлена на вторую сущность и не может быть
осуществлена на первую.

In the case of lexical scope,
if two constructs that establish entities
with the same name are textually nested, then references within the inner
construct refer to the entity established by the inner one; the inner one
shadows the outer one.  Outside the inner construct but inside the outer one,
references refer to the entity established by the outer construct.
For example:

В случае лексической области видимости,
если две конструкции, что устанавливают сущности с одинаковыми именами,
расположены в тексте одна внутри другой, тогда ссылки внутри внутренней
конструкции указывают на сущности внутренней конструкции; внутренние сущности
скрывают внешние. Вне внутренней конструкции, но внутри внешней конструкции
ссылки указывают на сущности, установленные внешней конструкцией.

\begin{lisp}
(defun test (x z) \\
~~(let ((z (* x 2))) (print z)) \\
~~z)
\end{lisp}

The binding of the variable \cd{z} by the \cd{let} construct shadows
the parameter binding for the function \cd{test}.  The reference to the
variable \cd{z} in the \cd{print} form refers to the \cd{let} binding.
The reference to \cd{z} at the end of the function refers to the parameter
named \cd{z}.

Связывание переменной \cd{z} с помощью конструкции \cd{let} скрывает связывание
одноименного параметра функции \cd{test}. Ссылка на переменную \cd{z} в форме
\cd{print} указывает на \cd{let} связывание. Ссылка на \cd{z} в конце функции
указывает на параметр с именем \cd{z}.

In the case of dynamic extent, if the time intervals of two entities
overlap, then one interval will necessarily be nested within the
other one.  This is a property of the design of Common Lisp.

В случае динамической продолжительности видимости, если временные интервалы двух
сущностей перекрываются, тогда они будут обязательно вложенными один в другого. Это
свойство Common Lisp дизайна.

\beforenoterule
\begin{implementation}
Behind the assertion that dynamic extents nest properly
is the assumption that there is only a single program or process.
Common Lisp does not address the problems of multiprogramming
(timesharing) or
multiprocessing (more than one active processor)
within a single Lisp environment.  The documentation for
implementations that extend Common Lisp for multiprogramming or
multiprocessing should
be very clear on what modifications are induced by such extensions
to the rules of extent and scope.
Implementors should note that Common Lisp has been carefully designed
to allow special variables to be implemented using either
the ``deep binding'' technique or the ``shallow binding'' technique,
but the two techniques have different semantic
and performance implications for multiprogramming and multiprocessing.
\end{implementation}
\afternoterule

A reference by name to an entity with dynamic extent
will always refer to the entity of that name
that has been most recently established
that has not yet been disestablished.
For example:

Ссылка по имени на сущность с динамической продолжительностью жизни всегда
указывает на сущность с этим именем, что была установлена наипозднейшей и еще не
была упразднена.
Например:
\begin{lisp}
(defun fun1 (x) \\
~~(catch 'trap (+ 3 (fun2 x)))) \\
 \\
(defun fun2 (y) \\
~~(catch 'trap (* 5 (fun3 y)))) \\
 \\
(defun fun3 (z) \\
~~(throw 'trap z))
\end{lisp}
Consider the call \cd{(fun1 7)}.  The result will be \cd{10}.  At the time
the \cd{throw} is executed, there are two outstanding catchers with the
name \cd{trap}: one established within procedure \cd{fun1}, and the other
within procedure \cd{fun2}.  The latter is the more recent, and so the
value \cd{7} is returned from the \cd{catch} form in \cd{fun2}.
Viewed from within \cd{fun3}, the \cd{catch} in \cd{fun2} shadows the one in \cd{fun1}.
Had \cd{fun2}
been defined as

Рассмотрим вызов \cd{(fun1 7)}. Результатом будет \cd{10}. Во время выполнения
\cd{throw}, существует две ловушки с именем \cd{trap}: одна установлена в
процедуре \cd{fun1}, и другая --- в \cd{fun2}. Более поздняя в \cd{fun2}, и
тогда, из формы \cd{catch}, что в \cd{fun2}, возвращается значение \cd{7}.
Рассматриваемая из \cd{fun3}, \cd{catch} в \cd{fun2} скрывает одноименную в
\cd{fun1}.
Если бы \cd{fun2} была определена как
\begin{lisp}
(defun fun2 (y) \\
~~(catch 'snare (* 5 (fun3 y))))
\end{lisp}
then the two catchers would have different names, and therefore the one
in \cd{fun1} would not be shadowed.  The result would then have been \cd{7}.

тогда бы две ловушки имели разные имена, и в таком случае одна из них из
\cd{fun1} не была бы скрыта. Результатом бы стало \cd{7}.

As a rule, this book simply speaks of the scope or extent of an entity;
the possibility of shadowing is left implicit.

Как правило, данная книга по простому рассказывает об областях видимости и
продолжительности сущности, возможность скрытия оставляется без рассмотрения.

The important scope and extent rules in Common Lisp follow:
Далее важные правила области и продолжительности видимости в Common Lisp'е:
\begin{itemize}
\item
Variable bindings normally have lexical scope and indefinite extent.

\item
Связывания переменных обычно имеют лексическую область видимости и неограниченную
продолжительность видимости.
\end{itemize}

\begin{newer}
\begin{itemize}
\item Variable bindings for which there is a \cd{dynamic-extent}
declaration also have lexical scope and indefinite extent,
but objects that are the values of such bindings may have
dynamic extent.
(The declaration is the programmer's guarantee that
the program will behave correctly even if certain of the data objects have only
dynamic extent rather than the usual indefinite extent.)

\item Связывания переменных, для которых декларировано \cd{dynamic-extent} также
 имеют лексическую область видимости и неограниченную продолжительность, но
 объекты, которые являются значениями этих связываний могут иметь динамическую
 продолжительность видимости.
(Декларация является программистской гарантией того, что программа будет вести
себя корректно, даже если уверенность, что объекты данных имеют только
динамическую продолжительность, предпочтительнее, чем обычная неограниченную
продолжительность видимости. FIXME)

\item Bindings of variable names to symbol macros by
\cd{symbol-macrolet} have lexical scope and indefinite extent.

\item Связывания имен переменных с символом макроса с помощью
  \cd{symbol-macrolet} имеют лексическую область видимости и неограниченную
  продолжительность видимости.
\end{itemize}
\end{newer}

\begin{itemize}
\item
Variable bindings that are declared to be \cd{special} have dynamic scope
(indefinite scope and dynamic extent).

\item 
Связывания переменных, которые задекларированы быть \cd{специальными (special)},
имеют динамическую область видимости (неограниченную область видимости и
динамическую продолжительность).
\end{itemize}

\begin{newer}
\begin{itemize}
\item Bindings of function names established, for example, by \cd{flet} and
\cd{labels} have lexical scope and indefinite extent.

\item Связывания имен функций устанавливаются, например, формами \cd{flet} и
  \cd{labels} и имеют лексическую область видимости и неограниченную продолжительность.

\item Bindings of function names for which there is a \cd{dynamic-extent}
declaration also have lexical scope and indefinite extent,
but function objects that are the values of such bindings may have
dynamic extent.

\item Связывания имен функций, для которых задекларировано \cd{dynamic-extent},
  также имеют лексическую область видимости и неограниченную продолжительность,
  но объекты функции, которые являются значениями для данных связываний могут
  иметь динамическую продолжительность видимости.

\item Bindings of function names to macros as established by
\cd{macrolet} have lexical scope and indefinite extent.

\item Связывания имен функций с макросами, установленными с помощью
  \cd{macrolet} имеют лексическую область видимости и неограниченную
  продолжительность.

\item Condition handlers and restarts have dynamic scope
(see chapter~\ref{CONDITION}).

\item Обработчики условий и перезапусков (condition and restarts) имеют
  динамическую область видимости (смотрите главу~\ref{CONDITION}).
\end{itemize}
\end{newer}

\begin{itemize}
\item
A catcher established by a \cd{catch}
or \cd{unwind-protect} special form has dynamic
scope.

\item Ловушка установленная с помощью специальных форм \cd{catch} или
  \cd{unwind-protect} имеет динамическую область видимости.

\item
An exit point established by a \cd{block} construct has lexical
scope and dynamic extent.  (Such exit points are also established
by \cd{do}, \cd{prog}, and other iteration constructs.)

\item
Точка выхода установленная с помощью конструкции \cd{block} имеет лексическую
область видимости и динамическую продолжительность. (Такие точки выхода, также
устанавливаются с помощью \cd{do}, \cd{prog} и другими конструкциями для итераций.)

\item
The \cd{go} targets
established by a \cd{tagbody}, named by the tags in the \cd{tagbody},
and referred to by \cd{go}
have lexical scope and dynamic extent.  (Such \cd{go} targets
may also appear as tags in the bodies of
\cd{do}, \cd{prog}, and other iteration constructs.)

\item
Цели для \cd{go}, устанавливающиеся с помощью \cd{tagbody}, именующияся с
помощью тегов в \cd{tagbody}, на которые указывает \cd{go}, имеют лексическую
область видимости и динамическую продолжительность. (Такие \cd{go} цели могут
также появлятся как теги в телах \cd{do}, \cd{prog} и других конструкций для итераций.)

\item
Named constants such as \cd{nil} and \cd{pi} have indefinite
scope and indefinite extent.

\item
Именованные константы, такие как \cd{nil} и \cd{pi} имеют неограниченную область
видимости и неограниченную продолжительность.
\end{itemize}

The rules of lexical scoping imply that lambda-expressions
appearing in the \cd{function} construct will,
in general, result in ``closures''
over those non-special variables visible to the lambda-expression.
That is, the function represented by a lambda-expression
may refer to any lexically apparent non-special variable and get the
correct value, even if the construct that established the binding
has been exited in the course of execution.
The \cd{compose} example shown earlier in this chapter
provides one illustration of this.
The rules also imply that special variable bindings are not
``closed over'' as they may be in certain other dialects of Lisp.

Правила для лексической области видимости подразумевают, что лямбда-выражения
(анонимные функции), появляющиеся в \cd{function}, будут, в общем случае,
являться <<замыканиями>> над этими неспециальными (non-special) переменными,
которые видимы для лямбда-выражения.
Это значит, что функция предоставленная лямбда-выражением может ссылаться на
любую лексически доступную неспециальную (non-special) переменную и получать
корректное значение, даже если выполнение уже вышло из конструкции, которая
устанавливала связи.
Пример \cd{compose}, рассмотренный в данной главе ранее, предоставлял
изображение такого механизма.
Правила также пдразумевают, что связывания специальных переменных не
<<замыкаются>>, как может быть в некоторых других диалектах Lisp'а.

Constructs that use lexical scope effectively
generate a new name for each established entity on each execution.
Therefore dynamic shadowing cannot occur (though lexical shadowing may).
This is of particular importance when dynamic extent is involved.
For example:

Конструкции, которые используют лексическую область видимости генерируют новое имя
для каждой устанавливаемой сущности при каждом исполнении.
Таким образом, динамическое скрытие не может произойти (тогда как лексическое
может).
Это, в частности, важно, когда используется динамическая продолжительность
видимости.
Например:

\begin{lisp}
(defun contorted-example (f g x) \\
~~(if (= x 0) \\
~~~~~~(funcall f) \\
~~~~~~(block here \\
~~~~~~~~~(+ 5 (contorted-example g \\
~~~~~~~~~~~~~~~~~~~~~~~~~~~~~~~~~\#'(lambda () \\
~~~~~~~~~~~~~~~~~~~~~~~~~~~~~~~~~~~~~(return-from here 4)) \\
~~~~~~~~~~~~~~~~~~~~~~~~~~~~~~~~~(- x 1))))))
\end{lisp}
Consider the call \cd{(contorted-example nil nil 2)}.  This produces
the result \cd{4}.  During the course of execution, there are three
calls on \cd{contorted-example}, interleaved with two establishments
of blocks:

Рассмотрим вызов \cd{(contorted-example nil nil 2)}. Он вернет результат
\cd{4}. Во время исполнения, \cd{contorted-example} будет вызывана три раза,
чередуюясь с двумя блоками:

\begin{lisp}
(contorted-example nil nil 2) \\
 \\
~~(block here\({}_1\) ...) \\
 \\
~~~~(contorted-example nil \#'(lambda () (return-from here\({}_1\) 4)) 1) \\
 \\
~~~~~~(block here\({}_2\) ...) \\
 \\
~~~~~~~~(contorted-example \#'(lambda () (return-from here\({}_1\) 4)) \\
~~~~~~~~~~~~~~~~~~~~~~~~~~~\#'(lambda () (return-from here\({}_2\) 4)) \\
~~~~~~~~~~~~~~~~~~~~~~~~~~~0) \\
~~~~~~~~~~(funcall f) \\
~~~~~~~~~~~~~~~~{\rm where} f \EV\ \#'(lambda () (return-from here\({}_1\) 4)) \\
 \\
~~~~~~~~~~~~(return-from here\({}_1\) 4)
\end{lisp}

At the time the \cd{funcall} is executed
there are two \cd{block} exit points outstanding, each apparently
named \cd{here}.  In the trace above, these exit points are distinguished
for expository purposes by subscripts.
The \cd{return-from} form executed as a result of the \cd{funcall}
operation
refers to the {\it outer} outstanding exit point
(\cd{here\({}_1\)}), not the
inner one (\cd{here\({}_2\)}).
This is a consequence of the rules of lexical scoping: it
refers to that exit point textually visible at the point of
execution of the \cd{function}
construct (here abbreviated by the \cd{\#'} syntax) that resulted
in creation of the function object actually invoked by the \cd{funcall}.

В время выполнения \cd{funcall} существует две невыполненные точки выхода
\cd{block}, каждая с именем \cd{here}. В стеке вызовов выше, эти две точки
для наглядности проиндексированы.
Форма \cd{return-from}, выполненная как результат операции \cd{funcall},
ссылается на {\it внешнюю} невыполненную точку выхода (\cd{here\({}_1\)}), но не
на (\cd{here\({}_2\)}).
Это следствие правил лексических областей видимости: форма ссылается на ту точку
выхода, что видима по тексту в точке вызова создания функции
(здесь отмеченной с помощью синтаксиса \cd{\#'}). (FIXME)

If, in this example, one were to change the form \cd{(funcall f)} to
\cd{(funcall g)}, then the value of the call \cd{(contorted-example nil nil 2)}
would be \cd{9}.  The value would change because the \cd{funcall} would cause the
execution of \cd{(return-from here\({}_2\) 4)}, thereby causing
a return from the inner exit point (\cd{here\({}_2\)}).
When that occurs, the value \cd{4} is returned from the
middle invocation of \cd{contorted-example}, \cd{5} is added to that
to get \cd{9}, and that value is returned from the outer block
and the outermost call to \cd{contorted-example}.  The point
is that the choice of exit point returned from has nothing to do with its
being innermost or outermost; rather,
it depends on the lexical scoping information
that is effectively packaged up with a lambda-expression when the
\cd{function} construct is executed.

Если в данном примере, изменить форму \cd{(funcall f)} на \cd{(funcall g)},
тогда значение вызова \cd{(contorted-example nil nil 2)} будет \cd{9}. Значение
измениться по сравнению с предыдущим разом, потому что \cd{funcall} вызовет
выполнение \cd{(return-from here\({}_2\) 4)}, и это в свою очередь вызовет выход
из внутренней точки выхода (\cd{here\({}_2\)}).
Когда это случиться, значение \cd{4} будет возвращено из середины вызова
\cd{contorted-example}, к нему добавится \cd{5} и резульата окажется \cd{9}, и это
значение вернется из внешнего блока и вообще из вызова
\cd{contorted-example}. Цель данного примера, показать что выбор точки выхода
зависит от лексической области, которая была 
захвачена лямбда-выражением, когда вызывался код создания этой анонимной функции.

This function \cd{contorted-example} works only because the
function named by \cd{f} is invoked during the extent of the exit point.
Block exit points are like non-special variable bindings in having
lexical scope, but they differ in having dynamic extent rather than indefinite
extent.  Once the flow of execution has left the block construct,
the exit point is disestablished.  For example:

Эта функция \cd{contorted-example} работает только потому, что функция с именем
\cd{f} вызывается в процессе продолжительности действия точки выхода.
Точки выхода из блока ведут себя, как связывания неспециальных (non-special)
переменных в имеющимся лексическом окружении, но отличаются тем, что имеют
динамическую продолжительность видимости, а не неограниченную. Как только
выполнение покинет блок с этой точкой выхода, она перестанет
существовать. Например: 

\begin{lisp}
(defun illegal-example () \\
~~(let ((y (block here \#'(lambda (z) (return-from here z))))) \\
~~~~(if (numberp y) y (funcall y 5))))
\end{lisp}

One might expect the call \cd{(illegal-example)} to produce \cd{5}
by the following incorrect reasoning:
the \cd{let} statement binds the variable \cd{y} to the
value of the \cd{block} construct; this value is a function resulting
from the lambda-expression.  Because \cd{y} is not a number, it is
invoked on the value \cd{5}.  The \cd{return-from} should then
return this value from the exit point named \cd{here}, thereby
exiting from the block {\it again} and giving \cd{y} the value \cd{5}
which, being a number, is then returned as the value of the call
to \cd{illegal-example}.

Можно предположить, что вызов \cd{(illegal-example)} вернет \cd{5}:
Форма \cd{let} связывает переменную \cd{y} со значением выполнения конструкции
\cd{block}; ее значение получится равным анонимной функции. Так как \cd{y} не
является числом, она вызывается с параметром \cd{5}. \cd{return-from} тогда
должны вернуть данное значение с помощью точки выхода \cd{here}, тогда
осуществляется выход из блока {\it еще раз} и \cd{y} получает значение \cd{5},
которое будучи числом, возвращается в качестве значения для \cd{illegal-example}.

The argument fails only because exit points are defined in Common Lisp
to have dynamic extent.  The argument is correct up to the execution
of the \cd{return-from}.  The execution of the \cd{return-from} is
an error, however, {\it not} because it cannot refer to the exit point,
but because it does correctly refer to an exit point {\it and}
that exit point has been disestablished.

Рассуждения выше неверны, потому что точки выхода определяемые в
Common Lisp'е имеют динамическую продолжительность видимости. Аргументация верна
только до вызова \cd{return-from}. Вызов формы \cd{return-from} является
ошибкой, {\it не потому что} она не может сослаться на точку выхода, а потому
что она корректно ссылается на точку выхода {\it и} эта точка выхода уже была
упразднена.       % Discussion of scoping rules
%Part{Dtspec, Root = "CLM.MSS"}
%Chapter of Common Lisp Manual.  Copyright 1984, 1988, 1989 Guy L. Steele Jr.

\clearpage\def\pagestatus{FINAL PROOF}

\chapter{Type Specif{\kern0pt}iers}    % Avoid ligature
\label{DTSPEC}

In Common Lisp, types are named by Lisp objects, specifically symbols and lists,
called {\it type specifiers}.  Symbols name predefined classes of objects,
whereas lists usually indicate combinations or
specializations of simpler types.
Symbols or lists may also be abbreviations for types that could
be specified in other ways.

\section{Type Specifier Symbols}

The type symbols defined by the system include those shown in
table~\ref{TYPE-SYMBOLS-TABLE}.
In addition, when a structure type is defined using \cd{defstruct},
the name of the structure type becomes a valid type symbol.

\begin{new}%CORR
{\it Notice of correction.}
In the first edition, the type specifiers \cd{signed-byte} and
\cd{unsigned-byte} were inadvertently omitted from
table~\ref{TYPE-SYMBOLS-TABLE}.
\end{new}

\begin{newer}
X3J13 voted in March 1989 \issue{COMMON-TYPE}
to eliminate the type \cd{common}; this fact is indicated by the brackets around
the \cd{common} type specifier in the table.

X3J13 voted in March 1989 \issue{CHARACTER-PROPOSAL}
to eliminate the type \cd{string-char};
this fact is indicated by the brackets around
the \cd{string-char} type specifier in the table.

X3J13 voted in March 1989 \issue{CHARACTER-PROPOSAL}
to add the type \cd{extended-character} and the type \cd{base-character}.

X3J13 voted in March 1989 \issue{REAL-NUMBER-TYPE}
to add the type specifier \cd{real}.

X3J13 votes have also implicitly added
many other type specifiers as names of classes (see chapter~\ref{CLOS})
or of conditions (see chapter~\ref{CONDITION}).
\end{newer}

\section{Type Specifier Lists}

If a type specifier is a list, the {\it car}
of the list is a symbol, and the rest of the list is subsidiary
type information.  In many cases a subsidiary item may be
{\it unspecified}.  The unspecified subsidiary item is indicated
by writing \cd{*}.  For example, to completely specify
a vector type, one must mention the type of the elements
and the length of the vector, as for example
\begin{lisp}
(vector double-float 100)
\end{lisp}
To leave the length unspecified, one would write
\begin{lisp}
(vector double-float *)
\end{lisp}
To leave the element type unspecified, one would write
\begin{lisp}
(vector * 100)
\end{lisp}
\begin{newer}
\noindent
One may also leave both length and element type unspecified:
\begin{lisp}
(vector * *)
\end{lisp}
\end{newer}
Suppose that two type specifiers are the same except that the first
has a \cd{*} where the second has a more explicit specification.
Then the second denotes a subtype of the type denoted by the first.

\begin{table}[t]
\caption{Standard Type Specifier Symbols}
\label{TYPE-SYMBOLS-TABLE}
\divide\tabcolsep by 2\relax
\begin{flushleft}
\cf
\begin{tabular*}{\textwidth}{@{}l@{\extracolsep{\fill}}l@{\extracolsep{\fill}}l@{\extracolsep{\fill}}l@{}}
array&fixnum&package&simple-string \\
atom&float&pathname&simple-vector \\
bignum&function&random-state&single-float \\
bit&hash-table&ratio&standard-char \\
bit-vector&integer&rational&stream \\
character&keyword&readtable&string \\
{\rm [}common{\rm ]}&list&sequence&{\rm [}string-char{\rm ]} \\
compiled-function&long-float&short-float&symbol \\
complex&nil&signed-byte&t \\
cons&null&simple-array&unsigned-byte \\
double-float&number&simple-bit-vector&vector
\end{tabular*}
\end{flushleft}

\begin{newer}
X3J13 voted in March 1989 \issue{COMMON-TYPE} to remove the type \cd{common}.

X3J13 voted in March 1989 \issue{CHARACTER-PROPOSAL} to remove the type \cd{string-char}.

X3J13 voted in March 1989 \issue{CHARACTER-PROPOSAL}
to add \cd{base-character} and \cd{extended-character}.

X3J13 voted in March 1989 \issue{REAL-NUMBER-TYPE} to add the type \cd{real}.
\end{newer}
\end{table}

As a convenience, if a list
has one or more unspecified items at the end, such items
may simply be dropped rather than writing an explicit \cd{*} for each one.
If dropping all occurrences of \cd{*} results in a singleton list,
then the parentheses may be dropped as well (the list may be replaced
by the symbol in its {\it car}).  For example,
\cd{(vector double-float *)} may be abbreviated to \cd{(vector double-float)},
and \cd{(vector * *)} may be abbreviated to \cd{(vector)} and then to
simply \cd{vector}.

\section{Predicating Type Specifiers}
\label{PREDICATING-TYPE-SPECIFIERS-SECTION}

A type specifier list \cd{(satisfies {\it predicate-name})} denotes
the set of all objects that satisfy the predicate named by {\it predicate-name},
which must be a symbol whose global function definition is a one-argument
predicate.
(A name is required; lambda-expressions are disallowed in order to avoid
scoping problems.)  For example, the type \cd{(satisfies numberp)} is the
same as the type \cd{number}.
The call \cd{(typep x '(satisfies p))} results in applying \cd{p} to \cd{x}
and returning \cd{t} if the result is true and {\nil} if the result is false.

\begin{obsolete}
As an example, the type \cd{string-char} could be defined as
\begin{lisp}
(deftype string-char () \\
~~'(and character (satisfies string-char-p)))
\end{lisp}
See \cd{deftype}.
\end{obsolete}

\begin{newer}
X3J13 voted in March 1989 \issue{COMMON-TYPE} to remove the type \cd{string-char}
and the function \cd{string-char-p} from the language.
\end{newer}

It is not a good idea for
a predicate appearing in a \cd{satisfies} type specifier to
cause any side effects when invoked.

\section{Type Specifiers That Combine}

The following type specifier lists define a type in terms of
other types or objects.

\begin{flushdesc}
\item[\cd{(member {\it object1} {\it object2} ...)}]
This denotes the set
containing precisely those objects named.  An object is of
this type if and only if it is \cd{eql} to one of the specified objects.

\beforenoterule
\begin{incompatibility}
This is roughly equivalent to
the Interlisp DECL package's \cd{memq}.
\end{incompatibility}
\afternoterule
\end{flushdesc}

\begin{newer}
\begin{flushdesc}
\item[\cd{(eql {\it object})}]
X3J13 voted in June 1988 \issue{CLOS} to add the \cd{eql} type specifier.
It may be used as a parameter specializer for CLOS methods
(see section~\ref{Introduction-to-Methods-SECTION}
and \cd{find-method}).
It denotes the set of the one object named;  an object is of
this type if and only if it is \cd{eql} to {\it object}.  While
\cd{(eql {\it object})} denotes the same type as \cd{(member {\it object})},
only \cd{(eql {\it object})} may be used as a CLOS parameter specializer.
\end{flushdesc}
\end{newer}

\begin{flushdesc}
\item[\cd{(not {\it type})}]
This denotes the set of all those objects that
are {\it not} of the specified type.

\item[\cd{(and {\it type1} {\it type2} ...)}]
This denotes the intersection of
the specified types.

\beforenoterule
\begin{incompatibility}
This is roughly equivalent to
the Interlisp DECL package's \cd{allof}.
\end{incompatibility}
\afternoterule

When \cd{typep} processes an \cd{and} type specifier, it always
tests each of the component types in order from left to right
and stops processing as soon as one component of the intersection has
been found to which the object in question does not belong.
In this respect an \cd{and} type specifier is similar to an
executable \cd{and} form.  The purpose of this similarity is to allow
a \cd{satisfies} type specifier to depend on filtering by previous
type specifiers.  For example, suppose there were a function \cd{primep}
that takes an integer and says whether it is prime.  Suppose also that
it is an error to give any object other than an integer to \cd{primep}.
Then the type specifier
\begin{lisp}
(and integer (satisfies primep))
\end{lisp}
is guaranteed never to result in an error because the function \cd{primep}
will not be invoked unless the object in question has already been
determined to be an integer.

\item[\cd{(or {\it type1} {\it type2} ...)}]
This denotes the union of the
specified types.  For example, the type \cd{list} by definition is the same as
\cd{(or null cons)}.  Also, the value returned by the function
\cd{position} is always of type \cd{(or null (integer 0 *))}
(either {\nil} or a non-negative integer).

\beforenoterule
\begin{incompatibility}
This is roughly equivalent to
the Interlisp DECL package's \cd{oneof}.
\end{incompatibility}
\afternoterule

As for \cd{and},
when \cd{typep} processes an \cd{or} type specifier, it always
tests each of the component types in order from left to right
and stops processing as soon as one component of the union has
been found to which the object in question belongs.
\end{flushdesc}

\section{Type Specifiers That Specialize}
\label{SPECIALIZED-TYPE-SPECIFIER-SECTION}

Some type specifier lists denote {\it specializations} of
data types named by symbols.  These specializations may be
reflected by more efficient representations in the underlying
implementation.  As an example, consider the type \cd{(array short-float)}.
Implementation A may choose to provide a specialized representation
for arrays of short floating-point numbers, and implementation B
may choose not to.

If you should want to create an array for the
express purpose of holding only short-float objects, you may
optionally specify to \cd{make-array} the element type
\cd{short-float}.  This does not {\it require} \cd{make-array} to create
an object of type \cd{(array short-float)}; it merely {\it permits} it.  The
request is construed to mean ``Produce the most specialized array
representation capable of holding short-floats that the implementation
can provide.''  Implementation A will then produce a specialized
array of type \cd{(array short-float)}, and implementation B
will produce an ordinary array of type \cd{(array t)}.

If one were then to ask whether the array were actually of type
\cd{(array short-float)}, implementation A would say ``yes,'' but
implementation B would say ``no.''  This is a property of \cd{make-array}
and similar functions: what you ask for is not necessarily what you get.

\begin{obsolete}
Types can therefore be used for two different purposes:
{\it declaration} and {\it discrimination}.  Declaring to \cd{make-array}
that elements will always be of type \cd{short-float} permits
optimization.  Similarly, declaring that a variable takes on
values of type \cd{(array short-float)} amounts to saying that
the variable will take on values that might be produced by specifying
element type \cd{short-float} to \cd{make-array}.
On the other hand, if the predicate \cd{typep} is used to test
whether an object is of type \cd{(array short-float)},
only objects actually of that specialized type can satisfy the test;
in implementation B no object can pass that test.
\end{obsolete}

\begin{new}
X3J13 voted in January 1989
\issue{ARRAY-TYPE-ELEMENT-TYPE-SEMANTICS}
to eliminate the differing treatment of types
when used ``for discrimination'' rather than ``for declaration'' on the grounds
that implementors have not treated the distinction consistently
and (which is more important) users have found the distinction confusing.

As a consequence of this change, the behavior of \cd{typep} and \cd{subtypep}
on \cd{array} and \cd{complex} type specifiers must be modified.
See the descriptions of those functions.  In particular, under their new
behavior, implementation B would say ``yes,'' agreeing with implementation A,
in the discussion above.

Note that the distinction between declaration and discrimination remains
useful, if only so that we may remark that the specialized (list)
form of the
\cd{function} type specifier may still be used only for declaration and
not for discrimination.
\end{new}

\begin{new}
X3J13 voted in June 1988 \issue{FUNCTION-TYPE} to clarify that
while the specialized form of the \cd{function} type specifier
(a list of the symbol \cd{function} possibly followed by
argument and value type specifiers)
may be used only for declaration, the symbol form (simply the name
\cd{function}) may be used for discrimination.
\end{new}

The valid list-format names for data types are as follows:
\begin{flushdesc}
\item[\cd{(array {\it element-type} {\it dimensions})}]
This denotes the set
of specialized arrays
whose elements are all members of the type {\it element-type}
and whose dimensions match {\it dimensions}.
For declaration purposes, this type encompasses those arrays
that can result by specifying {\it element-type} as the element type
to the function \cd{make-array}; this may be different
from what the type means for discrimination purposes.
{\it element-type} must be a valid type specifier or unspecified.
{\it dimensions} may be a non-negative integer, which is the number
of dimensions, or it may be a list of non-negative integers
representing the length of each dimension (any dimension
may be unspecified instead), or it may be unspecified.
For example:
\begin{lisp}
(array integer 3)~~~~~~~~~~~;{\rm Three-dimensional arrays of integers} \\
(array integer (* * *))~~~~~;{\rm Three-dimensional arrays of integers} \\
(array * (4 5 6))~~~~~~~~~~~;{\rm 4-by-5-by-6 arrays} \\
(array character (3 *))~~~~~;{\rm Two-dimensional arrays of characters} \\
~~~~~~~~~~~~~~~~~~~~~~~~~~~~;~{\rm that have exactly three rows} \\
(array short-float {\emptylist})~~~~~~;{\rm Zero-rank arrays of short-format} \\
~~~~~~~~~~~~~~~~~~~~~~~~~~~~;~{\rm floating-point numbers}
\end{lisp}
Note that \cd{(array~t)} is a proper subset of \cd{(array~*)}.
The reason is that \cd{(array~t)} is the set of arrays that can
hold any Common Lisp object (the elements are of type \cd{t},
which includes all objects).  On the other hand, \cd{(array~*)}
is the set of all arrays whatsoever, including, for example,
arrays that can hold only characters.  Now
\cd{(array character)} is not a subset of \cd{(array~t)}; the two sets
are in fact disjoint because \cd{(array character)} is not the
set of all arrays that can hold characters but rather the set of
arrays that are specialized to hold precisely characters and no
other objects.  To test whether an array \cd{foo} can hold a character,
one should not use
\begin{lisp}
(typep foo '(array character))
\end{lisp}
but rather
\begin{lisp}
(subtypep 'character (array-element-type foo))
\end{lisp}
See \cd{array-element-type}.
\begin{new}
X3J13 voted in January 1989
\issue{ARRAY-TYPE-ELEMENT-TYPE-SEMANTICS}
to change \cd{typep} and \cd{subtypep}
so that the specialized \cd{array} type specifier
means the same thing for discrimination
as for declaration: it encompasses those arrays
that can result by specifying {\it element-type} as the element type
to the function \cd{make-array}.
Under this interpretation \cd{(array character)} might be
the same type as \cd{(array t)}
(although it also might not be the same).
See \cd{upgraded-array-element-type}.
However,
\begin{lisp}
(typep foo '(array character))
\end{lisp}
is still not a legitimate test of whether the array
\cd{foo} can hold a character; one must still say
\begin{lisp}
(subtypep 'character (array-element-type foo))
\end{lisp}
to determine that question.

X3J13 also voted in January 1989
\issue{DECLARE-ARRAY-TYPE-ELEMENT-REFERENCES}
to specify that within the lexical scope of an array type declaration,
it is an error for an array element, when referenced, not to be
of the exact declared element type.  A compiler may, for example,
treat every reference to an element of a declared array as if
the reference were surrounded by a \cd{the} form mentioning the
declared array element type ({\it not} the upgraded array element type).  Thus
\begin{lisp}
(defun snarf-hex-digits (the-array) \\*
~~(declare (type (array (unsigned-byte 4) 1) the-array)) \\*
~~(do ((j (- (length array) 1) (- j 1)) \\*
~~~~~~~(val 0 (logior (ash val 4) \\*
~~~~~~~~~~~~~~~~~~~~~~(aref the-array j)))) \\*
~~~~~~((< j 0) val)))
\end{lisp}
may be treated as
\begin{lisp}
(defun snarf-hex-digits (the-array) \\*
~~(declare (type (array (unsigned-byte 4) 1) the-array)) \\*
~~(do ((j (- (length array) 1) (- j 1)) \\*
~~~~~~~(val 0 (logior (ash val 4) \\*
~~~~~~~~~~~~~~~~~~~~~~(the (unsigned-byte 4) \\*
~~~~~~~~~~~~~~~~~~~~~~~~~~~(aref the-array j))))) \\*
~~~~~~((< j 0) val)))
\end{lisp}
The declaration amounts to a promise by the user that the \cd{aref}
will never produce a value outside the interval 0 to 15, even if
in that particular implementation the array element type
\cd{(unsigned-byte 4)} is upgraded to, say, \cd{(unsigned-byte 8)}.
If~such upgrading does occur, then values outside that range may in fact
be stored in \cd{the-array}, as long as the code in \cd{snarf-hex-digits}
never sees them.


As a general rule, a compiler would be justified in transforming
\begin{lisp}
(aref (the (array {\it elt-type} ...) {\it a}) ...)
\end{lisp}
into
\begin{lisp}
(the {\it elt-type} (aref (the (array {\it elt-type} ...) {\it a}) ...)
\end{lisp}
It may also make inferences involving more complex functions,
such as \cd{position} or \cd{find}.
For example, \cd{find} applied to an array always returns either \cd{nil}
or an object whose type is the element type of the array.
\end{new}

\item[\cd{(simple-array {\it element-type} {\it dimensions})}]
This is equivalent
to \cd{(array {\it element-type} {\it dimensions})} except that it additionally
specifies that objects of the type are {\it simple} arrays
(see section~\ref{ARRAY-TYPE-SECTION}).

\item[\cd{(vector {\it element-type} {\it size})}]
This denotes the set of
specialized one-dimensional arrays whose elements are all of type {\it
element-type} and whose lengths match {\it size}.  This is entirely equivalent to
\cd{(array {\it element-type} ({\it size}))}.
For example:
\begin{lisp}
(vector double-float)~~~~~;{\rm Vectors of double-format} \\
~~~~~~~~~~~~~~~~~~~~~~~~~~;~{\rm floating-point numbers} \\
(vector * 5)~~~~~~~~~~~~~~;{\rm Vectors of length 5} \\
(vector t 5)~~~~~~~~~~~~~~;{\rm General vectors of length 5} \\
(vector (mod 32) *)~~~~~~~;{\rm Vectors of integers between 0 and 31}
\end{lisp}
\begin{obsolete}
The specialized types \cd{(vector string-char)} and \cd{(vector bit)} are so
useful that they have the special names \cd{string} and \cd{bit-vector}.
Every implementation of Common Lisp must provide distinct representations for
these as distinct specialized data types.
\end{obsolete}

\begin{newer}
X3J13 voted in March 1989 \issue{CHARACTER-PROPOSAL}
to eliminate the type \cd{string-char} and to redefine the type
\cd{string} to be the union of one or more specialized vector
types, the types of whose elements are subtypes of the type \cd{character}.
\end{newer}

\item[\cd{(simple-vector {\it size})}]
This is the same
as \cd{(vector t {\it size})} except that it additionally specifies
that its elements are {\it simple} general vectors.

\item[\cd{(complex {\it type})}]
Every element of this type is a
complex number whose real part
and imaginary part are each of type {\it type}.
For declaration purposes, this type encompasses those complex numbers
that can result by giving numbers of the specified type
to the function \cd{complex}; this may be different
from what the type means for discrimination purposes.
As an example, Gaussian integers might be
described as \cd{(complex integer)}, even in implementations
where giving two integers to the function \cd{complex} results
in an object of type \cd{(complex rational)}.

\begin{new}
X3J13 voted in January 1989
\issue{ARRAY-TYPE-ELEMENT-TYPE-SEMANTICS}
to change \cd{typep} and \cd{subtypep}
so that the specialized \cd{complex}
type specifier means the same thing for discrimination purposes
as for declaration purposes.
See \cd{upgraded-complex-part-type}.
\end{new}

\vskip 0pt plus 6pt

\item[\cd{(function ({\it arg1-type} {\it arg2-type} ...) {\it value-type})}]
\relax This type may be used only for declaration and not for
discrimination; \cd{typep} will signal an error if it encounters a specifier of
this form. Every element of this type is
a function that accepts arguments at {\it least} of the
types specified by the {\it argj-type} forms and returns a value that is a
member of the types specified by the {\it value-type} form.  The
\cd{\&optional}, \cd{\&rest}, and \cd{\&key} markers
may appear in the list of argument types.
The {\it value-type} may be a \cd{values} type specifier
in order to indicate the types of multiple values.

\begin{new}
X3J13 voted in January 1989
\issue{FUNCTION-TYPE-REST-LIST-ELEMENT}
to specify that the {\it arg-type} that
follows a \cd{\&rest} marker indicates the type of each actual argument
that would be gathered into the list for a \cd{\&rest} parameter,
and not the type of the \cd{\&rest} parameter itself (which is always
\cd{list}).  Thus one might declare the function \cd{gcd} to
be of type \cd{(function (\&rest~integer) integer)}, or
the function \cd{aref} to be of type
\cd{(function (array \&rest fixnum) t)}.
\end{new}

\begin{newer}
X3J13 voted in March 1988 \issue{FUNCTION-TYPE-KEY-NAME}
to specify that, in a \cd{function} type specifier,
an argument type specifier following \cd{\&key}
must be a list of two items, a keyword and a type specifier.
The keyword must be a valid keyword-name symbol that may be
supplied in the actual arguments of a call to the function,
and the type specifier indicates the permitted type of
the corresponding argument value.  (The keyword-name symbol
is typically a keyword,
but another X3J13 vote \issue{KEYWORD-ARGUMENT-NAME-PACKAGE}
allows it to be any symbol.)
Furthermore, if \cd{\&allow-other-keys} is not present,
the set of keyword-names mentioned in the \cd{function}
type specifier may be assumed to be exhaustive;
for example, a compiler would be justified in issuing
a warning for a function call using a keyword argument name
not mentioned in the type declaration for the function being called.
If \cd{\&allow-other-keys}
is present in the \cd{function}
type specifier, other keyword arguments may be supplied
when calling a function of the indicated type, and if supplied such
arguments may possibly be used.
\end{newer}

\begin{obsolete}
As an example, the function \cd{cons} is of type \cd{(function (t t) cons)},
because it can accept any two arguments and always returns a cons.
The function \cd{cons} is
also of type \cd{(function (float string) list)}, because it can certainly
accept a floating-point number and a string (among other things), and its
result is always of type \cd{list} (in fact a \cd{cons} is never \cd{null},
but that does not matter for this type declaration).
The function \cd{truncate} is of type
\cd{(function (number number) (values number number))}, as well as of type
\cd{(function (integer (mod 8)) integer)}.
\end{obsolete}

\begin{new}
X3J13 voted in January 1989
\issue{FUNCTION-TYPE-ARGUMENT-TYPE-SEMANTICS}
to alter the meaning of the
\cd{function} type specifier when used in \cd{type} and \cd{ftype}
declarations.  While the preceding formulation may be theoretically
elegant, they have found that it is not useful to compiler implementors
and that it is not the interpretation that users expect.  X3J13 prescribed instead the
following interpretation of declarations.

A declaration specifier of the form
\begin{lisp}
(ftype (function ({\it arg1-type} {\it arg2-type} ... {\it argn-type}) {\it value-type}) {\it fname})
\end{lisp}
implies that any function call of the form
\begin{lisp}
({\it fname} {\it arg1} {\it arg2} ...)
\end{lisp}
within the scope of the declaration can be treated as if it were
rewritten to use \cd{the}-forms in the following manner:
\begin{lisp}
(the {\it value-type} \\*
~~~~~({\it fname} \=(the {\it arg1-type} {\it arg1}) \\*
                  \>(the {\it arg2-type} {\it arg2}) \\*
                  \>... \\*
                  \>(the {\it argn-type} {\it argn})))
\end{lisp}
That is, it is an error for any of the actual arguments not to be of
its specified type {\it arg-type} or for the result not to be of the specified
type {\it value-type}.  (In particular, if any argument is not of
its specified type, then the result is not guaranteed to be of the
specified type---if indeed a result is returned at all.)

Similarly, a declaration specifier of the form
\begin{lisp}
(type (function ({\it arg1-type} {\it arg2-type} ... {\it argn-type}) {\it value-type}) {\it var})
\end{lisp}
is interpreted to mean that any reference to the variable {\it var}
will find that its value is a function, and that
it is an error to call this function with any actual argument not of
its specified type {\it arg-type}.
Also, it is an error for the result not to be of the specified
type {\it value-type}.
For example, a function call of the form
\begin{lisp}
(funcall {\it var} {\it arg1} {\it arg2} ...)
\end{lisp}
could be rewritten to use \cd{the}-forms as well.
If any argument is not of
its specified type, then the result is not guaranteed to be of the
specified type---if indeed a result is returned at all.


Thus, a \cd{type} or \cd{ftype} declaration specifier describes type
requirements imposed on calls to a function
as opposed to requirements imposed on the definition of the function.
This is analogous to the treatment of type declarations of variables
as imposing type requirements on references to variables, rather than
on the contents of variables.  See the vote of X3J13 on \cd{type}
declaration specifiers in general, discussed
in section~\ref{DECLARATION-SPECIFIERS-SECTION}.

In the same manner as for variable type declarations in general,
if two or more
of these declarations apply to the same function call (which can
occur if declaration scopes are suitably nested), then they all apply;
in effect, the types for each argument or result are intersected.
For example, the code fragment
\begin{lisp}
(locally (declare (ftype (function (biped) digit) \\*
~~~~~~~~~~~~~~~~~~~~~~~~~butcher-fudge)) \\*
~~(locally (declare (ftype (function (featherless) opposable) \\*
~~~~~~~~~~~~~~~~~~~~~~~~~~~butcher-fudge)) \\*
~~~~(butcher-fudge sam)))
\end{lisp}
may be regarded as equivalent to
\begin{lisp}
(the opposable \\*
~~~~~(the digit (butcher-fudge (the featherless \\*
~~~~~~~~~~~~~~~~~~~~~~~~~~~~~~~~~~~~(the biped sam)))))
\end{lisp}
or to
\begin{lisp}
(the (and opposable digit) \\*
~~~~~(butcher-fudge (the (and featherless biped) sam)))
\end{lisp}
That is, \cd{sam} had better be both \cd{featherless} and a \cd{biped},
and the result of \cd{butcher-fudge} had better be both
\cd{opposable} and a \cd{digit}; otherwise the code is in error.
Therefore a compiler may generate code that relies on these type assumptions,
for example.
\end{new}


\item[\cd{(values {\it value1-type} {\it value2-type} ...)}]
This type specifier is extremely restricted: it may be used {\it only}
as the {\it value-type} in a \cd{function} type specifier or in
a \cd{the} special form.  It is used to specify individual types when
multiple values are involved.
The
\cd{\&optional}, \cd{\&rest}, and \cd{\&key} markers may appear in the {\it value-type} list;
they thereby indicate the parameter list of a
function that, when given to \cd{multiple-value-call} along with
the values, would be suitable for receiving those values.
\end{flushdesc}


\section{Type Specifiers That Abbreviate}

The following type specifiers are, for the most part,
abbreviations for other type specifiers that would be far too
verbose to write out explicitly (using, for example, \cd{member}).

\begin{flushdesc}
\item[\cd{(integer {\it low} {\it high})}]
Denotes the integers between
{\it low} and {\it high}.  The limits {\it low} and {\it high}
must each be an integer, a list of an integer, or unspecified.
An integer is an inclusive limit,
a list of an integer is an exclusive limit, and
\cd{*} means that a limit does not exist
and so effectively denotes minus or plus infinity, respectively.
The type \cd{fixnum} is simply a name
for \cd{(integer {\it smallest} {\it largest})} for implementation-dependent
values of {\it smallest} and {\it largest}
(see \cd{most-negative-fixnum} and \cd{most-positive-fixnum}).
The type \cd{(integer 0 1)}
is so useful that it has the special name \cd{bit}.

\item[\cd{(mod {\it n})}]
Denotes the set of non-negative integers less than {\it n}.
This is equivalent to \cd{(integer 0 ${\it n}-1$)}
or to \cd{(integer 0 ({\it n}))}.

\item[\cd{(signed-byte {\it s})}]
Denotes the set of integers that can be represented
in two's-complement form in a byte of {\it s} bits.  This is
equivalent to
\cd{(integer $-2^{\hbox{\scriptsize\it s}-1}$ $2^{\hbox{\scriptsize\it s}-1}-1$)}.
Simply \cd{signed-byte} or \cd{(signed-byte *)} is the same as \cd{integer}.

\item[\cd{(unsigned-byte {\it s})}]
Denotes the set of non-negative integers that can be
represented in a byte of {\it s} bits.  This is equivalent to \cd{(mod
$2^{\hbox{\scriptsize\it s}}$)}, that is, \cd{(integer 0 $2^{\hbox{\scriptsize\it s}}-1$)}.
Simply \cd{unsigned-byte} or \cd{(unsigned-byte *)} is the same as
\cd{(integer 0 *)}, the set of non-negative integers.

\item[\cd{(rational {\it low} {\it high})}]
Denotes the rationals between
{\it low} and {\it high}.  The limits {\it low} and {\it high}
must each be a rational, a list of a rational, or unspecified.
A rational is an inclusive limit,
a list of a rational is an exclusive limit, and
\cd{*} means that a limit does not exist
and so effectively denotes minus or plus infinity, respectively.

\item[\cd{(float {\it low} {\it high})}]
Denotes the set of floating-point numbers between
{\it low} and {\it high}.  The limits {\it low} and {\it high}
must each be a floating-point number, a list of a floating-point number,
or unspecified; a floating-point number is an inclusive limit, a list of a
floating-point number is an exclusive limit, and
\cd{*} means that a limit does not exist
and so effectively denotes minus or plus infinity, respectively.

In a similar manner, one may use:
\begin{lisp}
(short-float {\it low} {\it high}) \\
(single-float {\it low} {\it high}) \\
(double-float {\it low} {\it high}) \\
(long-float {\it low} {\it high})
\end{lisp}
In this case, if a limit is a floating-point
number (or a list of one), it must be one of the appropriate format.
\end{flushdesc}

\begin{newer}
X3J13 voted in March 1989 \issue{REAL-NUMBER-TYPE} to add a list form of the \cd{real}
type specifier to denote an interval of \cd{real} numbers.

\begin{flushdesc}
\item[\cd{(real {\it low} {\it high})}]
Denotes the real numbers between
{\it low} and {\it high}.  The limits {\it low} and {\it high}
must each be a real, a list of a real, or unspecified.
A real is an inclusive limit,
a list of a real is an exclusive limit, and
\cd{*} means that a limit does not exist
and so effectively denotes minus or plus infinity, respectively.
\end{flushdesc}
\end{newer}

\begin{obsolete}
\begin{flushdesc}
\item[\cd{(string {\it size})}]
Means the same as
\cd{(array string-char ({\it size}))}: the set of strings of the indicated size.

\item[\cd{(simple-string {\it size})}]
Means the same
as \cd{(simple-array string-char ({\it size}))}: the set of simple
strings of the indicated size.
\end{flushdesc}
\end{obsolete}

\begin{newer}
X3J13 voted in March 1989 \issue{CHARACTER-PROPOSAL}
to eliminate the type \cd{string-char} and to redefine the type
\cd{string} to be the union of one or more specialized vector
types, the types of whose elements are subtypes of the type \cd{character}.
Similarly, the type
\cd{simple-string} is redefined to be the union of one or more specialized
simple vector
types, the types of whose elements are subtypes of the type \cd{character}.

\begin{flushdesc}
\item[\cd{(base-string {\it size})}]
Means the same as
\cd{(vector base-character {\it size})}: the set of base
strings of the indicated size.

\penalty-3000%manual

\item[\cd{(simple-base-string {\it size})}]
Means the same
as \cd{(simple-array base-character ({\it size}))}: the set of simple base
strings of the indicated size.
\end{flushdesc}
\end{newer}

\begin{flushdesc}
\item[\cd{(bit-vector {\it size})}]
Means the same as \cd{(array bit ({\it size}))}:
the set of bit-vectors of the indicated size.

\item[\cd{(simple-bit-vector {\it size})}]
This means the same as
\cd{(simple-array bit ({\it size}))}: the set of bit-vectors of
the indicated size.
\end{flushdesc}

\section{Defining New Type Specifiers}

New type specifiers can come into existence in two ways.
First, defining a new structure type with \cd{defstruct} automatically
causes the name of the structure to be a new type specifier symbol.
Second, the \cd{deftype} special form can be used to define new type-specifier
abbreviations.


\begin{defmac}
deftype name lambda-list <{declaration}* | doc-string> {\,form}*

This is very similar to a \cd{defmacro} form: {\it name} is the
symbol that identifies the type specifier being defined, {\it lambda-list} is
a lambda-list (and may contain \cd{\&optional} and \cd{\&rest}
markers), and
the {\it forms} constitute the body of the expander function.  If we view a
type specifier list as a list containing the type specifier name and some argument forms,
the argument forms (unevaluated) are bound to the corresponding
parameters in {\it lambda-list}.  Then the body forms are evaluated
as an implicit \cd{progn}, and the value of the last form
is interpreted as a new type specifier for which the original specifier
was an abbreviation.  The {\it name} is returned as the value of the
\cd{deftype} form.

\cd{deftype} differs from \cd{defmacro} in that if no {\it initform}
is specified for an \cd{\&optional} parameter, the default value
is \cd{*}, not {\nil}.

If the optional documentation string {\it doc-string} is present,
then it is attached to the {\it name}
as a documentation string of type \cd{type}; see \cd{documentation}.

Here are some examples of the use of \cd{deftype}:
\begin{lisp}
(deftype mod (n) {\Xbq}(integer 0 (,n))) \\
 \\
(deftype list () '(or null cons))
\end{lisp}

%manual

\begin{lisp}
(deftype square-matrix (\cd{\&optional} type size) \\*
~~"SQUARE-MATRIX includes all square two-dimensional arrays." \\*
~~{\Xbq}(array ,type (,size ,size))) \\
 \\
(square-matrix short-float 7)  {\rm means}  (array short-float (7 7)) \\
 \\
(square-matrix bit)  {\rm means}  (array bit (* *))
\end{lisp}
If the type name defined by \cd{deftype} is used simply as a type
specifier symbol, it is interpreted as a type specifier list with
no argument forms.  Thus, in the example above, \cd{square-matrix}
would mean \cd{(array * (* *))}, the set of two-dimensional arrays.
This would unfortunately fail to convey the constraint that the two
dimensions be the same; \cd{(square-matrix bit)} has the same problem.
A better definition is
\begin{lisp}
(defun equidimensional (a) \\
~~(or (< (array-rank a) 2) \\
~~~~~~(apply \#'= (array-dimensions a)))) \\
 \\
(deftype square-matrix (\cd{\&optional} type size) \\
~~{\Xbq}(and (array ,type (,size ,size)) \\
~~~~~~~~(satisfies equidimensional)))
\end{lisp}

\begin{newer}
X3J13 voted in March 1988 \issue{FLET-IMPLICIT-BLOCK}
to specify that the body of the expander function defined
by \cd{deftype} is implicitly enclosed in a \cd{block} construct
whose name is the same as the {\it name} of the defined type.
Therefore \cd{return-from} may be used to exit from the function.
\end{newer}

\begin{newer}
X3J13 voted in March 1989 \issue{DEFINING-MACROS-NON-TOP-LEVEL}
to clarify that, while defining forms normally appear at top level,
it is meaningful to place them in non-top-level contexts;
\cd{deftype} must define the expander function
within the enclosing lexical environment, not within the global
environment.
\end{newer}

\end{defmac}

\section{Type Conversion Function}

The following function may be used to convert an object to an
equivalent object of another type.

\begin{defun}[Function]
coerce object result-type

The {\it result-type} must be a type specifier; the {\it object} is converted
to an ``equivalent'' object of the specified type.
If the coercion cannot be performed, then an error is signaled.
In particular, \cd{(coerce x 'nil)} always signals an error.
If {\it object} is already of the specified type, as determined
by \cd{typep}, then it is simply returned.
It is not generally
possible to convert any object to be of any type whatsoever; only certain
conversions are permitted:
\begin{itemize}
\item
Any sequence type may be converted to any other sequence type, provided
the new sequence can contain all actual elements of the old sequence
(it is an error if it cannot).  If the {\it result-type} is specified as
simply \cd{array}, for example, then \cd{(array t)} is assumed.  A
specialized type such as \cd{string} or \cd{(vector (complex short-float))}
may be specified; of course, the result may be of either that type or
some more general type, as determined by the implementation.
Elements of the new sequence will be \cd{eql} to corresponding elements
of the old sequence.
If the
{\it sequence} is already of the specified type, it may be returned without
copying it; in this, \cd{(coerce {\it sequence} {\it type})} differs from
\cd{(concatenate {\it type} {\it sequence})}, for the latter is required to
copy the argument {\it sequence}.  In particular, if one specifies
\cd{sequence}, then the argument may simply be returned if it already is
a \cd{sequence}.
\begin{lisp}
(coerce '(a b c) 'vector) \EV\ \#(a b c)
\end{lisp}
\end{itemize}

\begin{newer}
X3J13 voted in June 1989 \issue{SEQUENCE-TYPE-LENGTH} to specify that
\cd{coerce} should signal an error if the new sequence type specifies the
number of elements and the old sequence has a different length.
\end{newer}

\begin{newer}
X3J13 voted in March 1989 \issue{CHARACTER-PROPOSAL}
to specify that if the {\it result-type} is \cd{string}
then it is understood to mean \cd{(vector character)},
and \cd{simple-string} is understood to mean \cd{(simple-array character (*))}.
\end{newer}

\begin{obsolete}
\begin{itemize}
\item
Some strings, symbols, and integers may be converted to characters.
If {\it object} is a string of length 1, then the
sole element of the string is returned.  If {\it object} is a symbol
whose print name is of length 1, then the sole element of the print name
is returned.  If {\it object} is an integer {\it n}, then \cd{(int-char {\it n})}
is returned.  See \cd{character}.
\begin{lisp}
(coerce "a" 'character) \EV\ \#{\Xbackslash}a
\end{lisp}
\end{itemize}
\end{obsolete}

\begin{newer}
X3J13 voted in March 1989 \issue{CHARACTER-PROPOSAL}
to eliminate \cd{int-char} from Common Lisp.
Presumably this eliminates the possibility of coercing an
integer to a character, although the vote did not address
this question directly.
\end{newer}

\begin{itemize}
\item
Any non-complex number can be converted to a \cd{short-float},
\cd{single-float}, \cd{double-float}, or \cd{long-float}.  If simply \cd{float}
is specified, and {\it object} is not already a \cd{float} of some kind, then
the object is converted to a \cd{single-float}.
\begin{lisp}
(coerce 0 'short-float) \EV\ 0.0S0 \\
(coerce 3.5L0 'float) \EV\ 3.5L0 \\
(coerce 7/2 'float) \EV\ 3.5
\end{lisp}

\item
Any number can be converted to a complex number.
If the number is not already complex, then a zero imaginary part
is provided by coercing the integer zero to the type of the given real part.
(If the given real part is rational, however, then the rule of
canonical representation for complex rationals will result
in the immediate re-conversion of the result from type \cd{complex}
back to type \cd{rational}.)
\begin{lisp}
(coerce 4.5s0 'complex) \EV\ \#C(4.5S0 0.0S0) \\
(coerce 7/2 'complex) \EV\ 7/2 \\
(coerce \#C(7/2 0) '(complex double-float)) \\
~~~\EV\ \#C(3.5D0 0.0D0)
\end{lisp}

\item
Any object may be coerced to type \cd{t}.
\begin{lisp}
(coerce x 't) \EQ\ (identity x) \EQ\ x
\end{lisp}
\end{itemize}

\begin{newer}
X3J13 voted in June 1988 \issue{FUNCTION-TYPE}
to allow coercion of certain objects to the type \cd{function}:
\begin{itemize}
\item
A symbol or lambda-expression can be converted to a function.
A symbol is coerced to type \cd{function} as if by applying
\cd{symbol-function} to the symbol; an error is signaled if the predicate
\cd{fboundp} is not true of
the symbol or if the symbol names a macro or special form.
A list {\it x} whose {\it car} is the symbol \cd{lambda}
is coerced to a function as if by execution of \cd{(eval {\Xbq}\#',{\it x})},
that is, of \cd{(eval (list 'function~{\it x}))}.
\end{itemize}
\end{newer}

Coercions from floating-point numbers to rationals and from ratios
to integers are purposely {\it not} provided because of rounding
problems.  The functions \cd{rational}, \cd{rationalize},
\cd{floor}, \cd{ceiling}, \cd{truncate}, and \cd{round} may be used for
such purposes.  Similarly, coercions from characters to integers
are purposely not provided; \cd{char-code} or \cd{char-int} may be
used explicitly to perform such conversions.
\end{defun}

\section{Determining the Type of an Object}

The following function may be used to obtain a type specifier
describing the type of a given object.

\begin{defun}[Function]
type-of object

\begin{obsolete}
\noindent
\cd{(type-of {\it object})} returns an implementation-dependent result:
some {\it type} of which the {\it object} is a member.  Implementors
are encouraged to arrange for
\cd{type-of} to return the most specific type that can be
conveniently computed and is likely to be useful to the user.
If the argument is a user-defined named
structure created by \cd{defstruct}, then \cd{type-of} will return the type name
of that structure.
Because the result is implementation-dependent, it is usually better
to use \cd{type-of} primarily for debugging purposes;
however, in a few situations portable code requires the use of
\cd{type-of}, such as when the result is to be given to the
\cd{coerce} or \cd{map} function.
On the other hand, often the \cd{typep} function
or the \cd{typecase} construct
is more appropriate than \cd{type-of}.
\end{obsolete}

\beforenoterule
\begin{incompatibility}
In MacLisp the function \cd{type-of} is called \cd{typep},
and anomalously so, for it is not a predicate.
\end{incompatibility}
\afternoterule

\begin{new}
Many have observed (and rightly so) that this specification is totally wimpy
and therefore nearly useless.  X3J13 voted in June 1989
\issue{TYPE-OF-UNDERCONSTRAINED}
to place the following constraints on \cd{type-of}:
\begin{itemize}
\item
Let {\it x} be an object such that \cd{(typep~{\it x}~{\it type})}
is true and {\it type} is one of the following:

\begin{flushleft}
\cf
\begin{tabular}{@{}llll@{}}
array          & float        & package        & sequence \\
bit-vector     & function     & pathname       & short-float \\
character      & hash-table   & random-state~~ & single-float \\
complex        & integer      & ratio          & stream \\
condition      & long-float~~ & rational       & string \\
cons           & null         & readtable      & symbol \\
double-float~~ & number       & restart        & vector
\end{tabular}
\end{flushleft}

Then
\cd{(subtypep (type-of {\it x}) {\it type}))}
must return the values \cd{t} and \cd{t}; that is, \cd{type-of} applied
to {\it x} must return either {\it type} itself or a subtype of {\it type}
that \cd{subtypep} can recognize in that implementation.

\item
For any object {\it x}, \cd{(subtypep (type-of {\it x}) (class-of {\it x}))}
must produce the values \cd{t} and \cd{t}.

\item
For every object {\it x}, \cd{(typep {\it x} (type-of {\it x}))}
must be true.  (This implies that \cd{type-of} can never return \cd{nil},
for no object is of type \cd{nil}.)

\item
\cd{type-of} never returns \cd{t} and never uses
a \cd{satisfies}, \cd{and}, \cd{or}, \cd{not},
or \cd{values} type specifier in its result.

\item
For objects of CLOS metaclass \cd{structure-class} or of \cd{standard-class},
\cd{type-of} returns the proper name of the class returned by \cd{class-of}
if it has a proper name, and otherwise returns the class itself.
In particular,
for any object created by a \cd{defstruct} constructor function,
where the \cd{defstruct} had the name {\it name} and no \cd{:type} option,
\cd{type-of} will return {\it name}.
\end{itemize}

As an example, \cd{(type-of "acetylcholinesterase")}
may return \cd{string} or \cd{simple-string} or \cd{(simple-string ~20)},
but not \cd{array} or \cd{simple-vector}.
As another example, it is permitted for
\cd{(type-of 1729)} to return
\cd{integer} or \cd{fixnum} (if it is indeed a fixnum) or
\cd{(signed-byte 16)} or \cd{(integer 1729 1729)} or \cd{(integer 1685 1750)}
% Bach's "St. Matthew Passion"
or even \cd{(mod 1730)}, but not \cd{rational} or \cd{number}, because
\begin{lisp}
(typep (+ (expt 9 3) (expt 10 3)) 'integer)
\end{lisp}
is true, \cd{integer} is in the list of types mentioned above, and
\begin{lisp}
(subtypep (type-of (+ (expt 1 3) (expt 12 3))) 'integer)
\end{lisp}
would be false if \cd{type-of} were to return \cd{rational} or \cd{number}.
% Ramanujan and Hardy?
\end{new}
\end{defun}



\begin{new}

\section{Type Upgrading}

X3J13 voted in January 1989
\issue{ARRAY-TYPE-ELEMENT-TYPE-SEMANTICS}
to add new functions by which a program
can determine, in a given Common Lisp implementation, how that
implementation will {\it upgrade} a type when constructing an array
specialized to contain elements of that type,
or a complex number specialized to contain parts of that type.


\begin{defun}[Function]
upgraded-array-element-type type

A type specifier is returned, indicating the element type
of the most specialized array representation capable of holding
items of the specified argument {\it type}.
The result is necessarily a supertype of the given {\it type}.
Furthermore, if a type {\it A} is a subtype of type {\it B}, then
\cd{(upgraded-array-element-type {\it A})} is a subtype of
\cd{(upgraded-array-element-type {\it B})}.

The manner in which an array element type is upgraded depends
only on the element type as such and not on any other property of
the array such as size, rank, adjustability,
presence or absence of a fill pointer, or displacement.

\beforenoterule
\begin{rationale}
If upgrading were allowed to depend on any of these properties,
all of which can be referred to, directly or indirectly, in the
language of type specifiers, then it would not be possible
to displace an array in a consistent and dependable manner
to another array created with the same \cd{:element-type} argument
but differing in one of these properties.
\end{rationale}
\afternoterule

Note that \cd{upgraded-array-element-type} could be defined as
\begin{lisp}
(defun upgraded-array-element-type (type) \\
~~(array-element-type (make-array 0 :element-type type)))
\end{lisp}
but this definition has the disadvantage of allocating an array and
then immediately discarding it.  The clever implementor surely can
conjure up a more practical approach.
\end{defun}


\begin{defun}[Function]
upgraded-complex-part-type type

A type specifier is returned, indicating the element type
of the most specialized complex number representation capable of having
parts of the specified argument {\it type}.
The result is necessarily a supertype of the given {\it type}.
Furthermore, if a type {\it A} is a subtype of type {\it B}, then
\cd{(upgraded-complex-part-type {\it A})} is a subtype of
\cd{(upgraded-complex-part-type {\it B})}.
\end{defun}

\end{new}
      % Data type specifiers
%Part{Progs, Root = "CLM.MSS"}
%Chapter of Spice Lisp Manual.  Copyright 1984, 1988, 1989 Guy L. Steele Jr.

\clearpage\def\pagestatus{FINAL PROOF}

\ifx \rulang\Undef
\chapter{Program Structure Структура программы}
\label{PROGS}

In chapter~\ref{DTYPES} the syntax was sketched for notating data objects
in Common Lisp.  The same syntax is used for notating programs because all
Common Lisp programs have a representation as Common Lisp data objects.

Lisp programs are organized as forms and functions.  Forms are
\emph{evaluated} (relative to some context) to produce values and side
effects.  Functions are invoked by \emph{applying} them to arguments.
The most important kind of form performs a function call;
conversely, a function performs computation by evaluating forms.

In this chapter, forms are discussed first and then functions.
Finally, certain ``top level'' special forms are discussed; the most
important of these is \cdf{defun}, whose purpose is to define a
named function.

\section{Forms}

The standard unit of interaction with a Common Lisp implementation is the \emph{form},
which is simply a data object meant to be \emph{evaluated} as a program
to produce one or more \emph{values} (which are also data objects).
One may request evaluation of \emph{any} data object, but only certain ones
are meaningful.  For instance,
symbols and lists are meaningful forms, while arrays
normally are not.  Examples of meaningful forms are \cd{3},
whose value is \cd{3}, and \cd{(+ 3 4)}, whose value is \cd{7}.
We write \cd{3} \EV\ \cd{3} and \cd{(+ 3 4)} \EV\ \cd{7}
to indicate these facts.  (\EV\ means ``evaluates to.'')

Meaningful forms may be divided into three categories:
self-evaluating forms, such as numbers; symbols, which stand
for variables; and lists.  The lists in turn may be divided
into three categories: special forms, macro calls, and function calls.

\begin{newer}
X3J13 voted in October 1988 \issue{EVAL-OTHER} to specify that
\emph{all} standard Common Lisp data objects other than symbols
and lists (including \cdf{defstruct} structures defined
without the \cd{:type} option) are self-evaluating.
\end{newer}

\subsection{Self-Evaluating Forms}

All numbers, characters, strings, and bit-vectors
are \emph{self-evaluating} forms.
When such an object is evaluated, that object
(or possibly a copy in the case of numbers or characters)
is returned as the value
of the form.  The empty list {\emptylist}, which is also the false value {\false},
is also a self-evaluating form: the value of {\false} is {\false}.
Keywords (symbols written with a leading colon) also evaluate
to themselves: the value of \cd{:start} is \cd{:start}.

\begin{newer}
X3J13 voted in January 1989 \issue{CONSTANT-MODIFICATION} to clarify that
it is an error to destructively modify any object that appears as a constant
in executable code, whether as
a self-evaluating form or within a \cdf{quote} special form.
\end{newer}

\subsection{Variables}

Symbols are used as names of variables in Common Lisp programs.
When a symbol is evaluated as a form, the value of the variable it names
is produced.  For example, after doing \cd{(setq items 3)}, which assigns
the value \cd{3} to the variable named \cdf{items}, then \cdf{items} \EV\ \cd{3}.
Variables can be \emph{assigned} to, as by \cdf{setq}, or \emph{bound},
as by \cdf{let}.
Any program construct that binds a variable effectively saves the old
value of the variable and causes it to have a new value, and on exit from
the construct the old value is reinstated.

There are actually two kinds of variables in Common Lisp, called \emph{lexical} (or
\emph{static}) variables and \emph{special} (or \emph{dynamic}) variables.
At any given time either or both kinds of variable with the same name may
have a current value.  Which of the two kinds of variable is referred to
when a symbol is evaluated depends on the context of the evaluation.
The general rule is that if the symbol occurs textually within a program
construct that creates a \emph{binding} for a variable of the same name,
then the reference is to the variable specified by the binding;
if no such program construct textually contains the reference, then
it is taken to refer to the special variable of that name.

The distinction between the two kinds of variable is one of scope
and extent.  A lexically bound variable can be referred to \emph{only}
by forms occurring at any \emph{place} textually within the program construct that
binds the variable.  A dynamically bound (special) variable can
be referred to at any \emph{time} from the time the binding is made
until the time evaluation of the construct that binds the variable
terminates.  Therefore lexical binding of variables
imposes a spatial limitation
on occurrences of references (but no temporal limitation, for the
binding continues to exist as long as the possibility of reference
remains).  Conversely, dynamic binding of variables imposes a temporal
limitation on occurrences of references (but no spatial limitation).
For more information on scope and extent, see chapter~\ref{SCOPE}.

The value a special variable has when there are currently
no bindings of that variable is called the \emph{global} value of the
(special) variable.
A global value can be given to a variable only by assignment,
because a value given by binding is by definition not global.

It is possible for a special variable to have no value at all,
in which case it is said to be \emph{unbound}.
By default, every global variable is unbound unless and until
explicitly assigned a value, except for those global variables
defined in this book or by the implementation already to have values
when the Lisp system is first started.
It is also possible to establish a binding of a special variable
and then cause that binding to be valueless by using the
function \cdf{makunbound}.  In this situation the variable
is also said to be ``unbound,'' although this is a misnomer;
precisely speaking, it is bound but valueless.
It is an error to refer to a variable that is unbound.

\begin{newer}
X3J13 voted in June 1989 \issue{UNDEFINED-VARIABLES-AND-FUNCTIONS}
to specify more precisely the effects of referring to an unbound variable.

  Reading an unbound variable or an undefined function
  must be detected in the highest safety setting (see the
  \cdf{safety} quality of the \cdf{optimize} declaration specifier)
  but the effect is undefined in any other safety setting. That is,
   reading an unbound variable should signal an error and
   reading an undefined function should signal an error.
  (``Reading a function'' includes
  both references to the function using the \cdf{function}
  special form, such as \cdf{f} in \cd{(function~f)}, and references to the
  function in a call, such as \cdf{f} in \cd{(f~x~y)}.)

  For the case of \cdf{inline} functions (in implementations where they are
  supported), a permitted point of view is that performing the inlining
  constitutes the read of the function, so that an \cdf{fboundp}
  check need not be done at
  execution time. Put another way, the effect of the application of
  \cdf{fmakunbound} to a function name
  on potentially inlined references to that function is undefined.

  When an unbound variable 
  is detected an error of type \cdf{unbound-variable} is signaled,
  and the \cdf{name} slot of the
  \cdf{unbound-variable} condition is initialized to the name of the
  offending variable.

  When an undefined function
  is detected an error of type \cdf{undefined-function} is signaled,
  and the \cdf{name} slot of the
  \cdf{undefined-function} condition is initialized to the name of the
  offending function.

  The condition type \cdf{unbound-slot}, which inherits from
  \cdf{cell-error}, has an additional slot \cdf{instance}, which
  can be initialized using the \cd{:instance} keyword to \cdf{make-condition}.
  The function \cdf{unbound-slot-instance} accesses this slot.

  The type of error signaled by the default primary
  method for the CLOS \cdf{slot-unbound} generic function is \cdf{unbound-slot}.
  The \cdf{instance} slot
  of the \cdf{unbound-slot} condition is initialized to the offending instance
  and the \cdf{name} slot is initialized
  to the name of the offending variable.
\end{newer}

Certain global variables are reserved as ``named constants.''
They have a global value and may not be bound or assigned to.
For example,
the symbols {\true} and {\false} are reserved.
One may not assign a value to {\true} or {\false},
and one may not bind {\true} or {\false}.  The global value of
{\true} is always {\true}, and the global value of
{\false} is always {\false}.  Constant symbols defined by
\cdf{defconstant} also become reserved and may not be further
assigned to or bound (although they may be redefined, if necessary, by
using \cdf{defconstant} again).  Keyword symbols,
which are notated with a leading colon, are reserved and
may never be assigned to or bound; a keyword always evaluates
to itself.

\subsection{Special Forms}

If a list is to be evaluated as a form, the first step is to examine
the first element of the list.  If the first element is one of
the symbols appearing in table~\ref{SPECIAL-FORM-TABLE},
then the list is called a \emph{special form}.  (This use of the word
``special'' is unrelated to its use in the phrase ``special variable.'')

Special forms are generally environment and control constructs.
Every special form has its own idiosyncratic syntax.  An example
is the \cdf{if} special form:
\cd{(if p (+ x 4) 5)} in Common Lisp means what
``\textbf{if} \emph{p} \textbf{then} \emph{x}+4 \textbf{else} 5'' means in
Algol.

The evaluation of a special form normally produces a value or values,
but the evaluation may instead call for a non-local exit; see \cdf{return-from},
\cdf{go}, and \cdf{throw}.

The set of special forms is fixed in Common Lisp; no way is provided
for the user to define more.  The user can create new syntactic
constructs, however, by defining macros.

The set of special forms in Common Lisp is purposely kept very small
because any program-analyzing program must have special knowledge
about every type of special form.  Such a program needs no special
knowledge about macros because it is simple to expand the macro
and operate on the resulting expansion.  (This is not to say that
many such programs, particularly compilers, will not have such
special knowledge.  A compiler may be able
to produce much better code if it recognizes such constructs
as \cdf{typecase} and \cdf{multiple-value-bind} and gives them customized
treatment.)

\begin{table}[t]
\caption{Names of All Common Lisp Special Forms}
\label{SPECIAL-FORM-TABLE}
\begin{tabular*}{\textwidth}{@{\extracolsep{\fill}}lll@{}}
\cdf{block}&\cdf{if}&\cdf{progv} \\
\cdf{catch}&\cdf{labels}&\cdf{quote} \\
&\cdf{let}&\cdf{return-from} \\
\cdf{declare}&\cdf{let*}&\cdf{setq} \\
\cdf{eval-when}&\cdf{macrolet}&\cdf{tagbody} \\
\cdf{flet}&\cdf{multiple-value-call}&\cdf{the} \\
\cdf{function}&\cdf{multiple-value-prog1}&\cdf{throw} \\
\cdf{go}&\cdf{progn}&\cdf{unwind-protect} \\
\cdf{generic-flet}&\cdf{generic-labels}&\cdf{symbol-macrolet} \\
\cdf{with-added-methods}&\cdf{locally}&\cdf{load-time-value}
\end{tabular*}
\vskip 4pt
\end{table}

An implementation is free to implement as a macro any construct described
herein as a special form.  Conversely, an implementation is free
to implement as a special form any construct described herein as a macro
if an equivalent macro definition is also provided.
The practical consequence is that the predicates \cdf{macro-function} and
\cdf{special-operator-p} may both be true of the same symbol.
It is recommended that a program-analyzing program process
a form that is a list whose \emph{car} is a symbol as follows:

\begin{enumerate}
\item
If the program has particular knowledge about the symbol,
process the form using special-purpose code.
All of the symbols listed in table~\ref{SPECIAL-FORM-TABLE}
should fall into this category. 

\item
Otherwise, if \cdf{macro-function} is true of the symbol, apply either
\cdf{macroexpand} or \cdf{macroexpand-1}, as appropriate,
to the entire form and then start over.

\item
Otherwise, assume it is a function call.
\end{enumerate}

\subsection{Macros}

If a form is a list and the first element is not the name of a special
form, it may be the name of a \emph{macro}; if so, the form is said
to be a \emph{macro call}.  A macro is essentially a function from
forms to forms that will, given a call to that macro, compute
a new form to be evaluated in place of the macro call.
(This computation is sometimes referred to as \emph{macro expansion}.)
For example, the macro named \cdf{return} will take a form such as
\cd{(return x)} and from that form compute a new form
\cd{(return-from {\nil} x)}.  We say that the old
form \emph{expands} into the new form.  The new form is then evaluated in
place of the original form; the value of the new form is returned as the
value of the original form.

\begin{new}
X3J13 voted in January 1989
\issue{DOTTED-MACRO-FORMS}
to clarify that macro calls, and subforms
of macro calls, need not be proper lists, but that use of dotted forms
requires the macro definition to use ``\cd{.~\emph{var}}'' or
``\cd{\&rest~\emph{var}}'' in order to match them properly.
It is then the responsibility of the macro definition to recognize
and appropriately handle such dotted forms or subforms.
\end{new}

There are a number of standard macros in Common Lisp, and the user can define more
by using \cdf{defmacro}.

Macros provided by a Common Lisp implementation as described herein may expand
into code that is not portable among differing implementations.
That is, a macro call may be implementation-independent because
the macro is defined in this book, but the expansion need not be.

\beforenoterule
\begin{implementation}
Implementors are encouraged to implement the macros
defined in this book, as far as is possible, in such a way that
the expansion will not contain any implementation-dependent
special forms, nor contain as forms data objects that
are not considered to be forms in Common Lisp.
The purpose of this restriction is to ensure that the expansion
can be processed by a program-analyzing program in an
implementation-independent manner.
There is no problem with a macro expansion containing
calls to implementation-dependent functions.
This restriction is not a requirement of Common Lisp; it is recognized
that certain complex macros may be able to expand into significantly
more efficient code in certain implementations
by using implementation-dependent special forms in the macro expansion.
\end{implementation}
\afternoterule

\subsection{Function Calls}

If a list is to be evaluated as a form and the first element is
not a symbol that names a special form or macro, then the list
is assumed to be a \emph{function call}.  The first element of the
list is taken to name a function.  Any and all remaining elements
of the list are forms to be evaluated; one value is obtained
from each form,
and these values become the \emph{arguments} to the function.
The function is then \emph{applied} to the arguments.
The functional computation normally produces a value,
but it may instead call for a non-local exit; see \cdf{throw}.
A function that does return may produce no value or several values;
see \cdf{values}.
If and when the function returns, whatever values it returns
become the values of the function-call form.

For example, consider the evaluation of the form \cd{(+ 3 (* 4 5))}.
The symbol \cdf{+} names the addition function, not a special form or macro.
Therefore the two forms \cd{3} and \cd{(* 4 5)} are evaluated to produce
arguments.  The form \cd{3} evaluates to \cd{3}, and the form
\cd{(* 4 5)} is a function call (to the multiplication function).
Therefore the forms \cd{4} and \cd{5} are evaluated, producing arguments
\cd{4} and \cd{5} for the multiplication.  The multiplication function
calculates the number \cd{20} and returns it.  The values \cd{3} and \cd{20}
are then given as arguments to the addition function, which calculates
and returns the number \cd{23}.  Therefore we say \cd{(+ 3 (* 4 5)) \EV\ 23}.

\begin{newer}
X3J13 voted in October 1988 \issue{FUNCTION-CALL-EVALUATION-ORDER}
to clarify that while the arguments in a function call are always
evaluated in strict left-to-right order, whether the function to
be called is determined before or after argument evaluation
is unspecified.  Programs are in error
that rely on a particular order of evaluation
of the first element of a function call relative to the
argument forms.
\end{newer}

\section{Functions}

There are two ways to indicate a function to be used in a function-call
form.  One is to use a symbol that names the function.  This use of
symbols to name functions is completely independent of their use in
naming special and lexical variables.  The other way is to use a
\emph{lambda-expression}, which is a list whose first element is the symbol
\cdf{lambda}.  A lambda-expression is \emph{not} a form; it cannot be
meaningfully evaluated.  Lambda-expressions and symbols, when used in
programs as names of functions, can appear only as the first element of a
function-call form, or as the second element of the \cdf{function}
special form.  Note that symbols and lambda-expressions are treated as
\emph{names} of functions in these two contexts.  This should be
distinguished from the treatment of symbols and lambda-expressions as
\emph{function objects}, that is,
objects that satisfy the predicate \cdf{functionp},
as when giving such an object to \cdf{apply} or \cdf{funcall} to be
invoked.

\subsection{Named Functions}

A name can be given to a function in one of two ways.
A \emph{global name} can be given to a function by using the
\cdf{defun} construct.
A \emph{local name} can be given to a function by using the
\cdf{flet} or
\cdf{labels} special form.
When a function is named, a lambda-expression is effectively
associated with that name
along with information about the entities that are lexically apparent
at that point.
If a symbol appears as the first element of a function-call form, then it
refers to the definition established by the innermost \cdf{flet} or \cdf{labels}
construct that textually contains the reference, or to the global
definition (if any) if there is no such containing construct.

\subsection{Lambda-Expressions}
\label{LAMBDA-EXPRESSIONS-SECTION}
\indexterm{lambda-expression}

A \emph{lambda-expression} is a list with the following syntax:
\begin{lisp}
(lambda \emph{lambda-list} . \emph{body})
\end{lisp}
The first element must be the symbol \cdf{lambda}.  The second element
must be a list.  It is called the \emph{lambda-list}, and specifies
names for the \emph{parameters} of the function.  When the function
denoted by the lambda-expression is applied to arguments,
the arguments are matched with the parameters specified by the
lambda-list.  The \emph{body} may then refer to the arguments by using
the parameter names.  The \emph{body} consists of any number of
forms (possibly zero).  These forms are evaluated in sequence,
and the results of the \emph{last} form only are returned as the results
of the application (the value {\false} is returned if there are zero
forms in the body).
The complete syntax of a lambda-expression is:

\begingroup
\def\GrossOptVars{\Mstar{\emph{var} {\Mor} \cd{(}\emph{var} \Mopt{\emph{initform} \Mopt{\emph{svar}}}\cd{)}}}
\def\GrossKeyOptVars{\Mstar{\emph{var} {\Mor} \cd{(}\Mgroup{\emph{var} {\Mor} \cd{(}\emph{keyword} \emph{var}\cd{)}} \Mopt{\emph{initform} \Mopt{\emph{svar}}}\cd{)}}}
\def\GrossAuxVars{\Mstar{\emph{var} {\Mor} \cd{(}\emph{var} \Mopt{\emph{initform}}\cd{)}}}
\begin{lisp}
(lambda (\Mstar{\emph{var}} \\
~~~~~~~~~\Mopt{\cd{\&optional} \GrossOptVars} \\
~~~~~~~~~\Mopt{\cd{\&rest} \emph{var}} \\
~~~~~~~~~\Mopt{\cd{\&key} \GrossKeyOptVars 
~~~~~~~~~~~~~~\Mopt{\cd{\&allow-other-keys}}} \\
~~~~~~~~~\Mopt{\cd{\&aux} \GrossAuxVars}) \\
~~~\Mchoice{\Mstar\emph{declaration} {\Mor} \emph{documentation-string}} \\
~~~\Mstar{\emph{\,form}})
\end{lisp}
\endgroup

Each element of a lambda-list is either a \emph{parameter specifier}
or a \emph{lambda-list keyword}; lambda-list keywords begin with \cd{\&}.
(Note that lambda-list keywords are not keywords in the usual sense;
they do not belong to the keyword package.  They are ordinary symbols
each of whose names begins with an ampersand.  This terminology
is unfortunately confusing but is retained for historical reasons.)

\begin{newer}
X3J13 voted in March 1988 \issue{KEYWORD-ARGUMENT-NAME-PACKAGE}
to allow a \emph{keyword} in the preceding specification of a lambda-list
to be any symbol whatsoever, not just a keyword symbol
in the \cdf{keyword} package.  See below.
\end{newer}

A lambda-list has five parts, any or all of which may be empty:

\begin{itemize}
\item
Specifiers for the \emph{required} parameters.  These are all the parameter
specifiers up to the first lambda-list keyword; if there is no such
lambda-list keyword, then all the specifiers are for required parameters.

\item
Specifiers for \emph{optional} parameters.
If the lambda-list keyword \cd{\&optional} is present,
the \emph{optional} parameter specifiers are those following the
lambda-list keyword \cd{\&optional} up to the next lambda-list keyword or the
end of the list.

\item
A specifier for a \emph{rest} parameter.  The lambda-list keyword \cd{\&rest}, if present, must
be followed by a single \emph{rest} parameter specifier,
which in turn must be followed by another lambda-list keyword or the end
of the lambda-list.

\item
Specifiers for \emph{keyword} parameters.
If the lambda-list keyword \cd{\&key} is present, all specifiers up to the next lambda-list keyword
or the end of the list are \emph{keyword} parameter specifiers.
The keyword parameter specifiers may optionally be followed by the
lambda-list keyword \cd{\&allow-other-keys}.

\item
Specifiers for \emph{aux} variables.  These are not really parameters.
If the lambda-list keyword \cd{\&aux} is present, all specifiers after it are
\emph{auxiliary variable} specifiers.
\end{itemize}

When the function represented by the lambda-expression is applied
to arguments, the arguments and parameters are processed in order
from left to right.
In the simplest case, only required parameters are present
in the lambda-list; each is specified simply by a name \emph{var} for
the parameter variable.
When the function is applied,
there must be exactly as many arguments as there are parameters,
and each parameter is bound to one argument.  Here, and in general,
the parameter is bound as a lexical variable unless a
declaration has been made that it should be a special binding;
see \cdf{defvar}, \cdf{proclaim}, and \cdf{declare}.

In the more general case, if there are \emph{n} required parameters
(\emph{n} may be zero), there must be at least \emph{n} arguments,
and the required parameters are bound to the first \emph{n} arguments.
The other parameters are then processed using any remaining arguments.

If \emph{optional} parameters are specified, then each one is processed as
follows.  If any unprocessed arguments remain, then the parameter variable
\emph{var} is bound to the next remaining argument, just as for a required
parameter.  If no arguments remain, however, then the \emph{initform} part
of the parameter specifier is evaluated, and the parameter variable
is bound to the resulting value (or to {\false} if no \emph{initform} appears
in the parameter specifier).
If another variable name \emph{svar} appears in the specifier, it is bound
to \emph{true} if an argument was available, and to \emph{false} if no
argument remained (and therefore \emph{initform} had to be evaluated).
The variable \emph{svar} is called a \emph{supplied-p} parameter;
it is bound not to an argument but to a value indicating whether or not
an argument had been supplied for another parameter.

After all \emph{optional} parameter specifiers have been processed,
then there may or may not be a \emph{rest} parameter.
If there is a \emph{rest} parameter, it is bound to a list of all
as-yet-unprocessed arguments.  (If no unprocessed arguments remain,
the \emph{rest} parameter is bound to the empty list.)
If there is no \emph{rest} parameter and there are no \emph{keyword}
parameters,
then there should be no unprocessed arguments (it is an error if there are).

\begin{new}
X3J13 voted in January 1989
\issue{REST-LIST-ALLOCATION}
to clarify that if a function has a \emph{rest} parameter
and is called using \cdf{apply}, then the list to which the
\emph{rest} parameter is bound is permitted, but not required,
to share top-level list structure with the list that was the last
argument to \cdf{apply}.  Programmers should be careful about performing
side effects on the top-level list structure of a \emph{rest} parameter.

This was the result of a rather long discussion within X3J13 and the
wider Lisp community.  To set it in its historical context, I must remark
that in Lisp Machine Lisp the list to which a \emph{rest} parameter was
bound had only dynamic extent; this in conjunction with the
technique of ``cdr-coding'' permitted a clever stack-allocation technique
with very low overhead.  However, the early designers of
Common Lisp, after a great deal of debate, concluded that it was dangerous
for cons cells to have dynamic extent; as an example, the ``obvious''
definition of the function \cdf{list}
\begin{lisp}
(defun list (\&rest x) x)
\end{lisp}
could fail catastrophically.  Therefore the first edition simply implied
that the list for a \emph{rest} parameter, like all other lists, would
have indefinite extent.  This still left open the flip side of the
question, namely, Is the list for a \emph{rest} parameter guaranteed fresh?
This is the question addressed by the X3J13 vote.
If it is always freshly consed, then it is permissible to destroy it,
for example by giving it to \cdf{nconc}.  However, the requirement always
to cons fresh lists could impose an unacceptable overhead in many implementations.
The clarification approved by X3J13 specifies that the programmer may
not rely on the list being fresh; if the function was called using \cdf{apply},
there is no way to know where the list came from.
\end{new}

Next, any \emph{keyword} parameters are processed.
For this purpose the same arguments are processed that
would be made into a list for a \emph{rest} parameter.
(Indeed, it is permitted to specify both \cd{\&rest} and \cd{\&key}.
In this case the remaining arguments are used for both purposes;
that is, all remaining arguments are made into a list for the
\cd{\&rest} parameter and are also processed for the \cd{\&key} parameters.
This is the only situation in which an argument is used
in the processing of more than one parameter specifier.)
If \cd{\&key} is specified, there must remain
an even number of arguments; these are considered as pairs,
the first argument in each pair being interpreted as a keyword name
and the second as the corresponding value.

\begin{newer}
X3J13 voted in March 1988 \issue{KEYWORD-ARGUMENT-NAME-PACKAGE}
to allow a \emph{keyword} in a lambda-list
to be any symbol whatsoever, not just a keyword symbol
in the \cdf{keyword} package.  If, after \cd{\&key},
a variable appears alone or within only one set of parentheses
(possibly with an \emph{initform} and a \emph{svar}), then
the behavior is as before: a keyword symbol with the same name as
the variable is used as the keyword-name when matching arguments
to parameter specifiers.  Only a parameter specifier of the form
\cd{((\emph{keyword} \emph{var})~...)} can cause the keyword-name
not to be a keyword symbol, by specifying a symbol not in the \cdf{keyword}
package as the \emph{keyword}.
For example:
\begin{lisp}
(defun wager (\&key ((secret password) nil) amount) \\*
~~(format nil "You {\Xtilde}A \${\Xtilde}D" \\*
~~~~~~~~~~(if (eq password 'joe-sent-me) "win" "lose") \\*
~~~~~~~~~~amount)) \\
\\
(wager :amount 100) \EV\ "You lose \$100" \\*
(wager :amount 100 'secret 'joe-sent-me) \EV\ "You win \$100"
\end{lisp}
The \cdf{secret} word could be made even more secret in this example
by placing it in some other \cdf{obscure} package, so that one would
have to write
\begin{lisp}
(wager :amount 100 'obscure:secret 'joe-sent-me) \EV\ "You win \$100"
\end{lisp}
to win anything.
\end{newer}

In each keyword parameter specifier must be a name \emph{var} for the
parameter variable.  If an explicit \emph{keyword} is
specified, then that is the keyword name for the parameter.  Otherwise
the name \emph{var} serves to indicate the keyword name,
in that a keyword with the same name (in the \cdf{keyword} package) is used
as the keyword.  Thus
\begin{lisp}
(defun foo (\cd{\&key} radix (type 'integer)) ...)
\end{lisp}
means exactly the same as
\begin{lisp}
(defun foo (\cd{\&key} ((:radix radix)) ((:type type) 'integer)) ...)
\end{lisp}

The keyword parameter specifiers are, like all parameter specifiers,
effectively processed from left to right.
For each keyword parameter specifier, if there is an argument
pair whose keyword name matches that specifier's keyword name
(that is, the names are \cdf{eq}),
then the parameter variable for that specifier is bound to the
second item (the value) of that argument pair.
If more than one such argument pair matches, it is not an error;
the leftmost argument pair is used.
If no such argument pair exists, then
the \emph{initform} for that specifier is evaluated
and the parameter variable is bound to that value (or to {\false} if
no \emph{initform} was specified).  The variable \emph{svar} is treated
as for ordinary \emph{optional} parameters: it is bound to \emph{true}
if there was a matching argument pair, and to \emph{false} otherwise.

It is an error if an argument pair has a keyword name not matched
by any parameter specifier, unless at least one of the following
two conditions is met:

\begin{itemize}
\item
\cd{\&allow-other-keys} was specified in the lambda-list.

\item
Somewhere among the keyword argument pairs is a pair whose keyword
is \cd{:allow-other-keys} and whose value is not {\false}.
\end{itemize}
If either condition obtains, then it is not an error
for an argument pair to match no parameter specified,
and the argument pair is simply ignored (but such an
argument pair is accessible through the \cd{\&rest} parameter if
one was specified). The purpose of these mechanisms is to
allow sharing of argument lists among several functions
and to allow either the caller or the called function
to specify that such sharing may be taking place.

After all parameter specifiers have been processed, the auxiliary
variable specifiers (those following the lambda-list keyword \cd{\&aux}) are processed from
left to right.  For each one, the \emph{initform} is evaluated and the
variable \emph{var} bound to that value (or to {\false} if no \emph{initform} was
specified).  Nothing can be done with \cd{\&aux} variables that cannot be
done with the special form \cdf{let*}:
\begin{lisp}
(lambda (x y \&aux (a (car x)) (b 2) c) ...) \\
~~~\EQ\ (lambda (x y) (let* ((a (car x)) (b 2) c) ...))
\end{lisp}

Which to use is purely a matter of style.

Whenever any \emph{initform} is evaluated for any parameter
specifier, that form may refer to any parameter variable to the left of
the specifier in which the \emph{initform} appears, including any supplied-p
variables, and may rely on the fact that no other parameter variable
has yet been bound (including its own parameter variable).

Once the lambda-list has been processed, the forms in the body of the
lambda-expression are executed.  These forms may refer to the arguments
to the function by using the names of the parameters.  On exit from the
function, either by a normal return of the function's value(s) or by a
non-local exit, the parameter bindings, whether lexical or special, are
no longer in effect.  (The bindings are not necessarily permanently discarded,
for a lexical binding can later be reinstated if a
``closure'' over that binding was created,
perhaps by using \cdf{function}, and saved before the exit occurred.)

\noindent
Examples of \cd{\&optional} and \cd{\&rest} parameters:
\begin{lisp}
((lambda (a b) (+ a (* b 3))) 4 5) \EV\ 19 \\
((lambda (a \cd{\&optional} (b 2)) (+ a (* b 3))) 4 5) \EV\ 19 \\
((lambda (a \cd{\&optional} (b 2)) (+ a (* b 3))) 4) \EV\ 10 \\
((lambda (\cd{\&optional} (a 2 b) (c 3 d) \cd{\&rest} x) (list a b c d x))) \\*
~~~\EV\ (2 {\false} 3 {\false} {\false}) \\
((lambda (\cd{\&optional} (a 2 b) (c 3 d) \cd{\&rest} x) (list a b c d x)) \\*
~6) \\*
~~~\EV\ (6 t 3 {\false} {\false}) \\
((lambda (\cd{\&optional} (a 2 b) (c 3 d) \cd{\&rest} x) (list a b c d x)) \\*
~6 3) \\*
~~~\EV\ (6 t 3 t {\false}) \\
((lambda (\cd{\&optional} (a 2 b) (c 3 d) \cd{\&rest} x) (list a b c d x)) \\*
~6 3 8) \\*
~~~\EV\ (6 t 3 t (8)) \\
((lambda (\cd{\&optional} (a 2 b) (c 3 d) \cd{\&rest} x) (list a b c d x)) \\*
~6 3 8 9 10 11) \\*
~~~\EV\ (6 t 3 t (8 9 10 11))
\end{lisp}
Examples of \cd{\&key} parameters:
\begin{lisp}
((lambda (a b \cd{\&key} c d) (list a b c d)) 1 2) \\*
~~~\EV\ (1 2 {\nil} {\nil}) \\
((lambda (a b \cd{\&key} c d) (list a b c d)) 1 2 :c 6) \\*
~~~\EV\ (1 2 6 {\nil}) \\
((lambda (a b \cd{\&key} c d) (list a b c d)) 1 2 :d 8) \\*
~~~\EV\ (1 2 {\nil} 8) \\
((lambda (a b \cd{\&key} c d) (list a b c d)) 1 2 :c 6 :d 8) \\*
~~~\EV\ (1 2 6 8) \\
((lambda (a b \cd{\&key} c d) (list a b c d)) 1 2 :d 8 :c 6) \\*
~~~\EV\ (1 2 6 8) \\
((lambda (a b \cd{\&key} c d) (list a b c d)) :a 1 :d 8 :c 6) \\*
~~~\EV\ (:a 1 6 8) \\
((lambda (a b \cd{\&key} c d) (list a b c d)) :a :b :c :d) \\
~~~\EV\ (:a :b :d {\nil})
\end{lisp}
Examples of mixtures:
\begin{lisp}
((lambda (a \cd{\&optional} (b 3) \cd{\&rest} x \cd{\&key} c (d a)) \\*
~~~(list a b c d x)) \\*
~1)   \EV\ (1 3 {\nil} 1 ())
\end{lisp}

\newpage%manual

\begin{lisp}
((lambda (a \cd{\&optional} (b 3) \cd{\&rest} x \cd{\&key} c (d a)) \\*
~~~(list a b c d x)) \\*
~1 2)   \EV\ (1 2 {\nil} 1 ()) \\
 \\
((lambda (a \cd{\&optional} (b 3) \cd{\&rest} x \cd{\&key} c (d a)) \\*
~~~(list a b c d x)) \\*
~:c 7)   \EV\ (:c 7 {\nil} :c ()) \\
 \\
((lambda (a \cd{\&optional} (b 3) \cd{\&rest} x \cd{\&key} c (d a)) \\*
~~~(list a b c d x)) \\*
~1 6 :c 7)   \EV\ (1 6 7 1 (:c 7)) \\
 \\
((lambda (a \cd{\&optional} (b 3) \cd{\&rest} x \cd{\&key} c (d a)) \\*
~~~(list a b c d x)) \\*
~1 6 :d 8)   \EV\ (1 6 {\nil} 8 (:d 8)) \\
 \\
((lambda (a \cd{\&optional} (b 3) \cd{\&rest} x \cd{\&key} c (d a)) \\*
~~~(list a b c d x)) \\*
~1 6 :d 8 :c 9 :d 10)   \EV\ (1 6 9 8 (:d 8 :c 9 :d 10))
\end{lisp}

All lambda-list keywords are permitted, but not terribly useful, in
lambda-expressions appearing explicitly as the first element of a
function-call form.  They are extremely
useful, however, in functions given global names by \cdf{defun}.

All symbols whose names begin with \cd{\&} are conventionally reserved
for use as lambda-list keywords and should not be used as variable names.
Implementations of Common Lisp are free to provide additional lambda-list
keywords.

\begin{defun}[Constant]
lambda-list-keywords

The value of \cdf{lambda-list-keywords} is a list of all the lambda-list
keywords used in the implementation, including the additional ones
used only by \cdf{defmacro}.  This list must contain at least the symbols
\cd{\&optional}, \cd{\&rest}, \cd{\&key}, \cd{\&allow-other-keys}, \cd{\&aux}, \cd{\&body}, \cd{\&whole},
and \cd{\&environment}.
\end{defun}

As an example of the use of \cd{\&allow-other-keys} and \cd{:allow-other-keys},
consider a function that takes two keyword arguments of its own and also
accepts additional keyword arguments to be passed to \cdf{make-array}:
\begin{lisp}
(defun array-of-strings (str dims \cd{\&rest} keyword-pairs \\*
~~~~~~~~~~~~~~~~~~~~~~~~~\cd{\&key} (start 0) end \cd{\&allow-other-keys}) \\*
~~(apply \#'make-array dims \\*
~~~~~~~~~:initial-element (subseq str start end) \\
~~~~~~~~~:allow-other-keys t \\*
~~~~~~~~~keyword-pairs))
\end{lisp}

This function takes a string and dimensioning information and returns
an array of the specified dimensions, each of whose elements is the
specified string.  However, \cd{:start} and \cd{:end} keyword arguments
may be used in the usual manner (see chapter~\ref{KSEQUE}) to specify
that a substring of the given string should be used.  In addition,
the presence of \cd{\&allow-other-keys} in the lambda-list indicates that the caller
may specify additional keyword arguments; the \cd{\&rest} argument provides
access to them.  These additional keyword arguments are fed to \cdf{make-array}.
Now, \cdf{make-array} normally does not allow the keywords \cd{:start}
and \cd{:end} to be used, and it would be an error to specify such
keyword arguments to \cdf{make-array}.  However, the presence in the
call to \cdf{make-array} of the keyword argument \cd{:allow-other-keys}
with a non-{\false} value causes any extraneous keyword arguments,
including \cd{:start} and \cd{:end}, to be acceptable and ignored.

\begin{defun}[Constant]
lambda-parameters-limit

The value of \cdf{lambda-parameters-limit} is a positive integer that is
the upper exclusive bound on the number of distinct parameter names
that may appear in a single lambda-list.
This bound depends on the implementation
but will not be smaller than 50.
Implementors are encouraged to make this limit as large as practicable
without sacrificing performance.
See \cdf{call-arguments-limit}.
\end{defun}

\section{Top-Level Forms}

The standard way for the user to interact with a Common Lisp implementation is
via a \emph{read-eval-print loop}: the system repeatedly
reads a form from some input source (such as a keyboard or a disk file),
evaluates it, and then prints the value(s) to some output sink (such as a
display screen or another disk file).  Any form (evaluable
data object) is acceptable; however, certain special forms are specifically
designed to be convenient for use as \emph{top-level} forms,
rather than as forms embedded within other forms in the way
that \cd{(+ 3 4)}
is embedded within \cd{(if p (+ 3 4) 6)}.
These top-level special forms may be used to define globally named
functions, to define macros, to make declarations,
and to define global values for special variables.

\newpage%manual

\begin{newer}
X3J13 voted in March 1989 \issue{DEFINING-MACROS-NON-TOP-LEVEL}
to clarify that, while defining forms normally appear at top level,
it is meaningful to place them in non-top-level contexts.
All defining forms that create functional objects from code appearing
as argument forms must ensure that
such argument forms refer to the enclosing lexical environment.
Compilers must handle defining forms properly in all situations,
not just top-level contexts.  However, certain
compile-time side effects of these defining forms are performed only
when the defining forms occur at top level (see section~\ref{COMPILER-SECTION}).
\end{newer}

\beforenoterule
\begin{incompatibility}
In MacLisp, a top-level \cdf{progn} is considered to
contain top-level forms only if the first form is \cd{(quote compile)}.
This odd marker is unnecessary in Common Lisp.
\end{incompatibility}
\afternoterule

Macros are usually defined by using the special form \cdf{defmacro}.
This facility is fairly complicated; it is described in chapter~\ref{MACROS}.

\subsection{Defining Named Functions}

The \cdf{defun} special form is the usual means of defining named functions.

\begin{defmac}
defun name lambda-list <{declaration}* | doc-string> {\,form}*

Evaluating a \cdf{defun} form causes the symbol \emph{name} to be a global name
for the function specified by the lambda-expression
\begin{lisp}
(lambda \emph{lambda-list} \Mstar{\emph{declaration} {\Mor} \emph{doc-string}} \Mstar{\emph{\,form}})
\end{lisp}
defined in the lexical environment in which the \cdf{defun} form was
executed.  Because \cdf{defun} forms normally appear at top level, this is
normally the null lexical environment.

\begin{newer}
X3J13 voted in March 1989 \issue{DEFINING-MACROS-NON-TOP-LEVEL}
to clarify that, while defining forms normally appear at top level,
it is meaningful to place them in non-top-level contexts;
\cdf{defun} must define the function
within the enclosing lexical environment, not within the null lexical
environment.
\end{newer}

\begin{newer}
X3J13 voted in March 1989 \issue{FUNCTION-NAME} to extend \cdf{defun}
to accept any function-name (a symbol or a list
whose \emph{car} is \cdf{setf}---see section~\ref{FUNCTION-NAME-SECTION})
as a \emph{name}.
Thus one may write
\begin{lisp}
(defun (setf cadr) ...)
\end{lisp}
to define a \cdf{setf}
expansion function for \cdf{cadr} 
(although it may be much more convenient to
use \cdf{defsetf} or \cdf{define-modify-macro}).
\end{newer}

\newpage%manual

If the optional documentation string \emph{doc-string} is present,
then it is attached to the \emph{name}
as a documentation string of type \cdf{function}; see \cdf{documentation}.
If \emph{doc-string} is not
followed by a declaration, it may be
present only if at least one \emph{form} is also specified, as it is
otherwise taken to be a \emph{form}.
It is an error if more than one \emph{doc-string} is present.

The \emph{forms} constitute the body of the defined function; they are
executed as an implicit \cdf{progn}.

The body of the defined function is implicitly enclosed
in a \cdf{block} construct whose name is the same as the \emph{name}
of the function.  Therefore \cdf{return-from}
may be used to exit from the function.

Other implementation-dependent bookkeeping actions may be taken as well
by \cdf{defun}.  The \emph{name} is returned as the value of the \cdf{defun}
form.
For example:
\begin{lisp}
(defun discriminant (a b c) \\
~~(declare (number a b c)) \\
~~"Compute the discriminant for a quadratic equation. \\
~~~Given a, b, and c, the value b{\Xcircumflex}2-4*a*c is calculated. \\
~~~The quadratic equation a*x{\Xcircumflex}2+b*x+c=0 has real, multiple, \\
~~~or complex roots depending on whether this calculated \\
~~~value is positive, zero, or negative, respectively." \\
~~(- (* b b) (* 4 a c))) \\
~~~\EV\ discriminant \\
~~~\textrm{and now} (discriminant 1 2/3 -2) \EV\ 76/9
\end{lisp}
\begin{new}%CORR
The documentation string in this example neglects to mention that the
coefficients \cdf{a}, \cdf{b}, and \cdf{c}
must be real for the discrimination criterion to hold.
Here is an improved version:
\begin{lisp}
~~"Compute the discriminant for a quadratic equation. \\
~~~Given a, b, and c, the value b{\Xcircumflex}2-4*a*c is calculated. \\
~~~If the coefficients a, b, and c are all real (that is, \\
~~~not complex), then the quadratic equation a*x{\Xcircumflex}2+b*x+c=0 \\
~~~has real, multiple, or complex roots depending on \\
~~~whether this calculated value is positive, zero, or \\
~~~negative, respectively."
\end{lisp}
\end{new}

It is permissible to use \cdf{defun} to redefine a function,
to install a corrected version of an incorrect definition, for example.
It is permissible to redefine a macro as a function.
It is an error to attempt to redefine the name of a special
form (see table~\ref{SPECIAL-FORM-TABLE}) as a function.
\end{defmac}

\subsection{Declaring Global Variables and Named Constants}

The \cdf{defvar} and \cdf{defparameter} special forms are
the usual means of specifying globally defined variables.
The \cdf{defconstant} special form is used for defining named constants.

\begin{defmac}
defvar name [initial-value [documentation]] \\
defparameter name initial-value [documentation] \\
defconstant name initial-value [documentation]

\cdf{defvar} is the recommended way to declare the use
of a special variable in a program.
\begin{lisp}
(defvar \emph{variable})
\end{lisp}
proclaims \emph{variable} to be \cdf{special} (see \cdf{proclaim}),
and may perform other system-dependent bookkeeping actions.

\begin{newer}
X3J13 voted in June 1987 \issue{DEFVAR-INITIALIZATION} to clarify
that if no \emph{initial-value} form is provided, \cdf{defvar}
does not change the value of the \emph{variable};
if no \emph{initial-value} form is provided and the variable
has no value, \cdf{defvar} does not give it a value.
\end{newer}
If a second argument form is supplied,
\begin{lisp}
(defvar \emph{variable} \emph{initial-value})
\end{lisp}
then \emph{variable} is initialized to the result of evaluating the form
\emph{initial-value} unless it already has a value.  The \emph{initial-value} form
is not evaluated unless it is used; this fact is useful if
evaluation of the \emph{initial-value} form does something
expensive like creating a large data structure.

\begin{newer}
X3J13 voted in June 1987 \issue{DEFVAR-INIT-TIME} to clarify that
evaluation of the \emph{initial-value} and the initialization of the
variable occur, if at all, at the time the \cdf{defvar} form is executed,
and that the \emph{initial-value} form is evaluated
if and only if the \emph{variable} does not already have a value.
\end{newer}
The initialization is
performed by assignment and thus assigns a global value to the variable
unless there are currently special bindings of that variable.
Normally there should not be any such special bindings.

\cdf{defvar} also provides a good place to put a comment describing the
meaning of the variable, whereas an ordinary \cdf{special} proclamation
offers the
temptation to declare several variables at once and not have room to
describe them all.
\begin{lisp}
(defvar *visible-windows* 0 \\
~~"Number of windows at least partially visible on the screen")
\end{lisp}

\cdf{defparameter} is similar to \cdf{defvar}, but \cdf{defparameter} requires
an \emph{initial-value} form, always evaluates the form, and assigns the
result to the variable.  The semantic distinction is that \cdf{defvar}
is intended to declare a variable changed by the program, whereas
\cdf{defparameter} is intended to declare a variable that is normally
constant but can be changed (possibly at run time), where such a change
is considered a
change \emph{to} the program.  \cdf{defparameter} therefore does not indicate
that the quantity \emph{never} changes; in particular, it does not license
the compiler to build assumptions about the value into programs being
compiled.

\cdf{defconstant} is like \cdf{defparameter} but \emph{does} assert that
the value of the variable \emph{name} is fixed and does license
the compiler to build assumptions about the value into programs being
compiled.  (However, if the compiler chooses to replace references
to the name of the constant by the value of the constant in code
to be compiled, perhaps in order to allow further optimization,
the compiler must take care that such ``copies'' appear to be \cdf{eql}
to the object that is the actual value of the constant.  For example,
the compiler may freely make copies of numbers but must exercise
care when the value is a list.)

It is an error if there are any special bindings
of the variable at the time the \cdf{defconstant} form
is executed (but implementations may or may not check for this).

Once a name has been declared by \cdf{defconstant} to be constant,
any further assignment to or binding of that special variable is an error.
This is the case for such system-supplied constants as \cdf{t} and
\cdf{most-positive-fixnum}.
A compiler may also choose to issue warnings about bindings of
the lexical variable of the same name.

\begin{new}
X3J13 voted in January 1989
\issue{DEFCONSTANT-SPECIAL}
to clarify the preceding paragraph by specifying
that it is an error to rebind constant symbols
as either lexical or special variables.
Consequently, a valid reference to a symbol declared with \cdf{defconstant}
always refers to its global value.
(Unfortunately, this violates the principle of referential transparency,
for one cannot always choose names for lexical variables without regard
to surrounding context.)
\end{new}

For any of these constructs,
the documentation should be a string.  The string is attached
to the name of the variable, parameter, or constant
under the \cdf{variable} documentation type; see the \cdf{documentation}
function.

\begin{new}
X3J13 voted in March 1988
\issue{DEFVAR-DOCUMENTATION}
to clarify that the \emph{documentation-string}
is not evaluated but must appear as a literal string when the \cdf{defvar},
\cdf{defparameter}, or \cdf{defconstant} form is evaluated.

For example,
the form
\begin{lisp}
(defvar *avoid-registers* nil "Compilation control switch \#43")
\end{lisp}
is legitimate, but
\begin{lisp}
(defvar *avoid-registers* nil \\*
~~(format nil "Compilation control switch \#{\Xtilde}D" \\*
~~~~~~~~~~(incf *compiler-switch-number*)))
\end{lisp}
is erroneous because the call to \cdf{format} is not a literal string.

(On the other hand, the form
\begin{lisp}
(defvar *avoid-registers* nil \\*
~~\#.(format nil "Compilation control switch \#{\Xtilde}D" \\*
~~~~~~~~~~~~(incf *compiler-switch-number*)))
\end{lisp}
might be used to accomplish the same purpose, because the call to
\cdf{format} is evaluated at \cdf{read} time; when the \cdf{defvar} form
is evaluated, only the result of the call to \cdf{format}, a string,
appears in the \cdf{defvar} form.)
\end{new}

These constructs are normally used only as top-level forms.  The
value returned by each of these constructs is the \emph{name} declared.
\end{defmac}

\subsection{Control of Time of Evaluation}

\begin{newer}
X3J13 voted in March 1989 \issue{EVAL-WHEN-NON-TOP-LEVEL} to
completely redesign the \cdf{eval-when} construct to solve some problems
concerning its treatment in other than top-level contexts.
The new definition is upward compatible with the old definition,
but the old keywords are deprecated.

\begin{defspec}
eval-when ({situation}*) {\,form}*

  The body of an \cdf{eval-when} form is processed as an implicit \cdf{progn}, but
  only in the situations listed.  Each \emph{situation} must be a symbol,
  either \cd{:compile-toplevel},
  \cd{:load-toplevel}, or \cd{:execute}.

  The use of \cd{:compile-toplevel} and \cd{:load-toplevel}
  controls whether and when processing
  occurs for top-level forms. The use of \cd{:execute} controls whether
  processing occurs for non-top-level forms.

  The \cdf{eval-when} construct may be more precisely understood in terms of
  a model of how the file compiler, \cdf{compile-file}, processes forms in a
  file to be compiled.

  Successive forms are read from the file by the file compiler using 
  \cdf{read}. These top-level forms are normally processed in what we call
  ``not-compile-time'' mode. There is one other mode, called 
  ``compile-time-too'' mode, which can come into play for top-level
  forms. The \cdf{eval-when} special form is used to annotate a program
  in a way that allows the program doing the processing to select
  the appropriate mode.

  Processing of top-level forms in the file compiler works as follows:

\begin{itemize}
   \item If the form is a macro call, it is expanded and the result is
     processed as a top-level form in the same processing mode
     (compile-time-too or not-compile-time).

   \item If the form is a \cdf{progn} (or \cdf{locally} \issue{LOCALLY-TOP-LEVEL})
     form, each of its body forms is
     sequentially processed as top-level forms in the same processing
     mode.

   \item If the form is a \cdf{compiler-let}, \cdf{macrolet},
     or \cdf{symbol-macrolet},
     the file compiler makes the appropriate bindings and recursively
     processes the body forms as an implicit top-level \cdf{progn} with those 
     bindings in effect, in the same processing mode.

   \item If the form is an \cdf{eval-when} form, it is handled according to
     the following table:
     \begin{flushleft}
     \begin{tabular*}{\linewidth}{@{\extracolsep{\fill}}c@{}cccl@{}}
     LT&CT&EX&CTTM&Action \\ \hlinesp
       yes & yes &--   & --  &    process body in compile-time-too mode \\
       yes & no  &yes  & yes &    process body in compile-time-too mode \\
       yes & no  &--   & no  &    process body in not-compile-time mode \\
       yes & no  &no   & --  &    process body in not-compile-time mode \\
       no  & yes &--   & --  &    evaluate body \\
       no  & no  &yes  & yes &    evaluate body \\
       no  & no  &--   & no  &    do nothing \\
       no  & no  &no   & --  &    do nothing \\
       \hline
     \end{tabular*}
     \end{flushleft}
     In the preceding table the column LT asks whether \cd{:load-toplevel}
     is one of the situations specified in the \cdf{eval-when} form;
     CT similarly refers to \cd{:compile-toplevel} and EX to \cd{:execute}.
     The column CTTM asks whether the \cdf{eval-when} form was encountered
     while in compile-time-too mode.  The phrase
     ``process body'' means to process the body as an implicit top-level
     \cdf{progn} in the indicated mode, and  ``evaluate body'' means to
     evaluate the body forms sequentially as an
     implicit \cdf{progn} in the dynamic execution context of the compiler and
     in the lexical environment in which the \cdf{eval-when} appears.

   \item Otherwise, the form is a top-level form that is not one of the
     special cases.  If in compile-time-too mode, the compiler first
     evaluates the form and then performs normal compiler processing
     on it.  If in not-compile-time mode, only normal compiler
     processing is performed (see section~\ref{COMPILER-SECTION}).
     Any subforms are treated as non-top-level forms.
\end{itemize}

  Note that top-level forms are guaranteed to be processed in the order
  in which they textually appear in the file, and that each top-level
  form read by the compiler is processed before the next is read.
  However, the order of processing (including, in particular, macro
  expansion) of subforms that are not top-level forms is unspecified.

  For an \cdf{eval-when} form that is not a top-level form in the file compiler
  (that is, either in the interpreter, in \cdf{compile}, or in the file
  compiler but not at top level), if the \cd{:execute} situation is specified,
  its body is treated as an implicit \cdf{progn}.  Otherwise, the body
  is ignored and the \cdf{eval-when} form has the value \cdf{nil}.

  For the sake of backward compatibility,
  a \emph{situation} may also be \cdf{compile}, \cdf{load}, or \cdf{eval}.
  Within a top-level \cdf{eval-when} form
  these have the same meaning as \cd{:compile-toplevel}, \cd{:load-toplevel},
  and \cd{:execute}, respectively; but their effect is undefined when used
  in an \cdf{eval-when} form that is not at top level.

  The following effects are logical consequences of the preceding specification:

  \begin{itemize}
   \item It is never the case that the execution of a single \cdf{eval-when}
     expression will execute the body code more than once.

   \item The old keyword \cdf{eval} was a misnomer because execution of
     the body need not be done by \cdf{eval}.  For example, when the
     function definition
     \begin{lisp}
     (defun foo () (eval-when (:execute) (print 'foo)))
     \end{lisp}
     is compiled
     the call to \cdf{print} should be compiled, not evaluated at compile time.

   \item Macros intended for use in top-level forms should arrange for all
     side-effects to be done by the forms in the macro expansion.
     The macro-expander itself should not perform the side-effects.

\begin{lisp}
(defmacro foo () \\*
~~(really-foo)~~~~~~~~~~~~~~~~~~~~~~~~~~~~~~;{\rm Wrong}\\*
~~{\Xbq}(really-foo)) \\
\\
(defmacro foo () \\*
~~{\Xbq}(eval-when (:compile-toplevel \\*
~~~~~~~~~~~~~~~:load-toplevel :execute)~~~~~;{\rm Right} \\*
~~~~(really-foo)))
\end{lisp}

     Adherence to this convention will mean that such macros will behave
     intuitively when called in non-top-level positions.   

   \item Placing a variable binding around an \cdf{eval-when}
     reliably captures the
     binding because the ``compile-time-too'' mode cannot occur (because 
     the \cdf{eval-when} could not be a top-level form).
     For example,
\begin{lisp}
(let ((x 3)) \\*
~~(eval-when (:compile-toplevel :load-toplevel :execute) \\*
~~~~(print x)))
\end{lisp}
will print 3 at execution (that is, load) time
     and will not print anything at
     compile time.  This is important so that expansions of \cdf{defun} and 
     \cdf{defmacro} can be done in terms of \cdf{eval-when}
     and can correctly capture the lexical environment.
     For example, an implementation might expand a \cdf{defun} form such as
\begin{lisp}
(defun bar (x) (defun foo () (+ x 3)))
\end{lisp}
into
\begin{lisp}
(progn (eval-when (:compile-toplevel) \\*
~~~~~~~~~(compiler::notice-function 'bar '(x))) \\*
~~~~~~~(eval-when (:load-toplevel :execute) \\*
~~~~~~~~~(setf (symbol-function 'bar) \\*
~~~~~~~~~~~~~~~\#'(lambda (x) \\*
~~~~~~~~~~~~~~~~~~~(progn (eval-when (:compile-toplevel)  \\*
~~~~~~~~~~~~~~~~~~~~~~~~~~~~(compiler::notice-function 'foo \\*
~~~~~~~~~~~~~~~~~~~~~~~~~~~~~~~~~~~~~~~~~~~~~~~~~~~~~~~'())) \\*
~~~~~~~~~~~~~~~~~~~~~~~~~~(eval-when (:load-toplevel :execute) \\*
~~~~~~~~~~~~~~~~~~~~~~~~~~~~(setf (symbol-function 'foo) \\*
~~~~~~~~~~~~~~~~~~~~~~~~~~~~~~~~~~\#'(lambda () (+ x 3)))))))))
\end{lisp}
     which by the preceding rules would be treated the same as
\begin{lisp}
(progn (eval-when (:compile-toplevel) \\*
~~~~~~~~~(compiler::notice-function 'bar '(x))) \\*
~~~~~~~(eval-when (:load-toplevel :execute) \\*
~~~~~~~~~(setf (symbol-function 'bar) \\*
~~~~~~~~~~~~~~~\#'(lambda (x) \\*
~~~~~~~~~~~~~~~~~~~(progn (eval-when (:load-toplevel :execute) \\*
~~~~~~~~~~~~~~~~~~~~~~~~~~~~(setf (symbol-function 'foo) \\*
~~~~~~~~~~~~~~~~~~~~~~~~~~~~~~~~~~\#'(lambda () (+ x 3)))))))))
\end{lisp}

\end{itemize}

Here are some additional examples.
\begin{lisp} 
(let ((x 1)) \\*
~~(eval-when (:execute :load-toplevel :compile-toplevel) \\*
~~~~(setf (symbol-function 'foo1) \#'(lambda () x))))
\end{lisp}
 The \cdf{eval-when} in the preceding expression is not at top level,
       so only the \cd{:execute}
       keyword is considered.  At compile time, this has no effect.
       At load time (if the \cdf{let} is at top level), or at execution time
       (if the \cdf{let} is embedded in some other form which does not execute
       until later), this sets \cd{(symbol-function 'foo1)} to a function that
       returns \cd{1}.
\begin{lisp}
(eval-when (:execute :load-toplevel :compile-toplevel) \\*
~~(let ((x 2)) \\*
~~~~(eval-when (:execute :load-toplevel :compile-toplevel) \\*
~~~~~~(setf (symbol-function 'foo2) \#'(lambda () x)))))
\end{lisp}

 If the preceding expression occurs at the top level of a file to be compiled,
       it has \emph{both} a compile time \emph{and} a load-time effect of setting
       \cd{(symbol-function 'foo2)} to a function that returns \cd{2}.
\begin{lisp}
(eval-when (:execute :load-toplevel :compile-toplevel) \\*
~~(setf (symbol-function 'foo3) \#'(lambda () 3)))
\end{lisp}
 If the preceding expression occurs at the top level of a file to be compiled,
       it has \emph{both} a compile time \emph{and}
       a load-time effect of setting the
       function cell of \cd{foo3} to a function that returns \cd{3}.
\begin{lisp}
(eval-when (:compile-toplevel) \\*
~~(eval-when (:compile-toplevel)  \\*
~~~~(print 'foo4)))
\end{lisp}
  The preceding expression always does nothing; it simply returns \cdf{nil}.

\begin{lisp}
(eval-when (:compile-toplevel)  \\*
~~(eval-when (:execute) \\*
~~~~(print 'foo5)))
\end{lisp}
  If the preceding form occurs at the top level of a file to be compiled,
       \cd{foo5} is
       printed at compile time. If this form occurs in a non-top-level
       position, nothing is printed at compile time. Regardless of context,
       nothing is ever printed at load time or execution time.

\begin{lisp}
(eval-when (:execute :load-toplevel) \\*
~~(eval-when (:compile-toplevel) \\*
~~~~(print 'foo6)))
\end{lisp}

    If the preceding form occurs at the top level of a file to be compiled,
       \cd{foo6} is
       printed at compile time.  If this form occurs in a non-top-level
       position, nothing is printed at compile time. Regardless of context,
       nothing is ever printed at load time or execution time.
\end{defspec}
\end{newer}

%RUSSIAN
\else

\chapter{Структура программы}
\label{PROGS}

В главе~\ref{DTYPES} был рассказано о синтаксисе записи Common Lisp'овых объектов.
А так как все Common Lisp'овые программы также являются и объектами данных, то и
синтаксис у них одинаковый.

Lisp'овые программы составляются из форм и функций. Формы \emph{выполняются}
(относительно некоторого контекста) для получения значений и побочных
эффектов. Функции в свою очередь вызываются с некоторыми аргументами. Это
называется помощью \emph{применени} функции к аргументам. 
Наиболее важный вид форм выполняет вызов функции, и наоборот, функция выполняет
вычисление с помощью выполнения форм.

В данной главе, сначала обсуждаются формы и затем функции. В конце, обсуждаются
специальные формы <<верхнего уровня (top level)>>. Наиболее важной из этих форм
является \cdf{defun}, цель которой --- создание именных функций (будут
ещё и безымянные).

\section{Формы}

Стандартной единицей взаимодействия с реализацией Common Lisp'а является
\emph{форма}, которая является объектом данных, который выполняется как
программа для вычисления одного или более \emph{значений} (которые также
являются объектами данных). Запросить выполнение можно для \emph{любого}
объекта данных, но не для всех это имеет смысл. Например, символы и списки имеет
смысл выполнять, тогда как массивы обычно нет. Примеры содержательных форм:
\cd{3}, значение которой \cd{3}, и \cd{(+ 3 4)}, значение которой \cd{7}.
Для обозначения этих фактов мы пишем \cd{3} \EV\ \cd{3} и \cd{(+ 3 4)} \EV\
\cd{7}. (\EV\ означает <<вычисляется в>>)

Содержательные формы могут быть разделены на три категории:
самовычисляемые формы, такие как числа,
символы, которые используются для переменных,
и списки. Списки в свою очередь могут быть разделены на три категории:
специальные формы, 
вызовы макросов,
вызовы функций.

Все стандартные объекты данных Common Lisp, не являющиеся символами и списками
(включая \cdf{defstruct} структуры, определённые без опции \cd{:type}) являются
самовычисляемыми. 

\subsection{Самовычисляемые формы}

Все числа, строковые символы, строки и битовые векторы являются
\emph{самовычисляемыми} формами.
Когда данный объект вычисляется, тогда объект (или возможно копия в случае с
числами и строковыми символами) возвращается в качестве значения данной
формы. Пустой список {\emptylist}, который также является значением ложь
({\false}), также является самовычисляемой формой: значение {\false} является
{\false}.
Ключевые символы (примечание переводчика: не путать с ключевыми словами в других
языках, в Common Lisp'е это вид символов) также вычисляются сами в себя:
значение \cd{:start} является \cd{:start}.

Деструктивная модификация любого объекта, представленного как
константа с помощью самовычисляемой формы или специальной формы \cdf{quote},
является ошибкой.

\subsection{Переменные}

В Common Lisp программах символы используются в качестве имён переменных.
Когда символ вычисляется как форма, то в качестве результата возвращается
значение переменной, которую данный символ именовал. Например, после выполнения
\cd{(setq items 3)}, которая присвоила значение \cd{3} переменой именованной
символом \cdf{items}, форма \cdf{items} выполнится в \cd{3} (\cdf{items} \EV\
\cd{3}).
Переменные могут быть \emph{назначены} с помощью \cdf{setq} или \emph{связаны} с
помощью \cdf{let}.
Любая программная конструкция, которая связывает переменную, сохраняет старое
значение переменной, и назначает новое, и при выходе из конструкции
восстанавливается старое значение.

В Common Lisp'е есть два вида переменных. Они называются \cd{лексические} (или
\emph{статические}) и \emph{специальные} (или \emph{динамические}).
В одно время каждая из них или обе переменные с одинаковым именем могут иметь
некоторое значение. На какую переменную ссылается символ при его вычислении,
зависит от контекста выполнения. Главное правило заключается в том, что если
символ вычисляется по тексту в конструкции, которая создала \emph{связывание} для
переменной с одинаковым именем, то символ ссылается на переменную, обозначенную
в этом связывании, если же по тексту такой конструкции нет, то символ ссылается
на специальную переменную.

Различие между двумя видами переменных заключается в области видимости и
продолжительности видимости. Лексически связанная переменная может быть использована
\emph{только} по тексту в форме, которая установила связывание. Динамически
связанная (специальная) переменная может быть использована в любое
\emph{время} между установкой связи и до выполнения конструкции, которая
упраздняет связывание. Таким образом лексическое связывание переменных
накладывает ограничение на использование переменной только в некоторой текстовой
области (но не на временные ограничения, так связывание продолжает существовать,
пока возможно существование ссылки на переменную). И наоборот, динамическое
связывание переменных накладывает ограничение на временные рамки использования
переменной (но не на текстовую область).
Для более подробной информации смотрите главу~\ref{SCOPE}.

Когда нет связываний, значение, которое имеет специальная
переменная, называется \emph{глобальным} значением (специальной) переменной.
Глобальное значение может быть задано переменной только с помощью назначения,
потому что значение заданное связыванием по определению не глобально.

Специальная переменная может вообще не иметь значения, в таком случае,
говориться, что она \emph{несвязанная}. 
По умолчанию, каждая глобальная переменная является несвязанной, пока значение
не будет назначено явно, за исключением переменных определённых в этой книге или
реализацией, которые уже имеют значения сразу после первого запуска Lisp машины.
Кроме того, существует возможность установки связывания специальной переменной и
затем упразднения этого связывания с помощью функции \cdf{makunbound}. В такой
ситуации переменная также называется <<несвязанной>>, хотя это и неправильно,
если быть точнее, переменная связана, но без значения FIXME. Ссылка на несвязанную
переменную является ошибкой.

Некоторые глобальные переменные зарезервированы в качестве <<именованных
констант>>.
Они имеют глобальное значение и не могут быть связаны или переназначены.
Например символы {\true} и {\false} зарезервированы.
Этим символам невозможно назначить значение. Также и невозможно связать эти
символы с другими значениями. Символы констант определённых с помощью
\cdf{defconstant} также становятся зарезервированными и не могут быть
переназначены или связаны (но они могут быть переопределены с помощью вызова
\cdf{defconstant}). Ключевые символы также не могут быть переназначены или
связаны, ключевые символы всегда вычисляются сами в себя.

\subsection{Специальные формы}

Если список выполняется в качестве формы, первым шагом является определение
первого элемента списка. Если первый элемент списка является одним из символов,
перечисленных в таблице~\ref{SPECIAL-FORM-TABLE}, тогда список называется
\emph{специальной формой}. (Использование слова <<специальный>> никак не
связано с использованием этого слова в фразе <<специальная переменная>>.)

Специальные формы обычно являются окружениями и управляющими конструкциями.
Каждая специальная форма имеет свой идиосинкразический синтаксис. Например
специальная форма \cdf{if}:
\cd{(if p (+ x 4) 5)} в Common Lisp'е означает то же, что и
<<\textbf{if} \emph{p} \textbf{then} \emph{x}+4 \textbf{else} 5>> означает в
Algol'е.

Выполнение специальной формы обычно возвращает значение или значения, но
выполнение может и вызвать нелокальный выход; смотрите \cdf{return-from},
\cdf{go} и \cdf{throw}.

Множество специальных форм в Common Lisp'е фиксировано. Создание
пользовательских специальных форм невозможно. Однако пользователь может
создавать новые синтаксические конструкции с помощью определения макросов.

Множество специальных форм в Common Lisp'е специально держится малым, потому что
любая программа, анализирующая программы, должна содержать специальные знания о
каждом типе специальной формы. Такие программы не нуждаются в специальных
знаниях о макросах, так как раскрытие макроса просто, и далее остаётся только
оперирование с результатом раскрытия. (Это не значит, что программы, в
частности, компиляторы, не будут иметь специальных знаний о макросах. Компилятор
может генерировать боле эффективный код, если он распознает такие конструкции, как
\cdf{typecase} и \cdf{multiple-value-bind} и будет по-особому с ними
обращаться.)

\begin{table}[t]
\caption{Имена всех специальных форм}
\label{SPECIAL-FORM-TABLE}
\begin{tabular*}{\textwidth}{@{\extracolsep{\fill}}lll@{}}
\cdf{block}&\cdf{if}&\cdf{progv} \\
\cdf{catch}&\cdf{labels}&\cdf{quote} \\
&\cdf{let}&\cdf{return-from} \\
\cdf{declare}&\cdf{let*}&\cdf{setq} \\
\cdf{eval-when}&\cdf{macrolet}&\cdf{tagbody} \\
\cdf{flet}&\cdf{multiple-value-call}&\cdf{the} \\
\cdf{function}&\cdf{multiple-value-prog1}&\cdf{throw} \\
\cdf{go}&\cdf{progn}&\cdf{unwind-protect} \\
& &\cdf{symbol-macrolet} \\
&\cdf{locally}&\cdf{load-time-value}
\end{tabular*}
\vskip 4pt
\end{table}

Реализация может исполнять в виде макроса любую конструкцию описанную здесь как
специальную форму. И наоборот, реализация может выполнять в виде специальной
формы любую конструкцию описанную здесь как макрос, при условии, что также 
предоставляется эквивалентное определение макроса.
Практическое значение заключается в том, что предикаты \cdf{macro-function} и
\cdf{special-operator-p} могут оба возвращать true принимая один и тот же символ.
Рекомендуется, чтобы программа для анализа других программ обрабатывала форму
являющуюся списком с символом в первой позиции следующим образом:

\begin{enumerate}
\item
  Если программа имеет подробные знания о символе, обрабатывать форму необходимо с
  помощью специализированного кода. Все символы, перечисленные в
  таблице~\ref{SPECIAL-FORM-TABLE} должны попадать под данную категорию.

\item
  В противном случае, если для этого символа \cdf{macro-function} вычисляется в
  истину, необходимо применить \cdf{macroexpand} или \cdf{macroexpand-1} для раскрытия
  формы, и результат вновь анализировать.

\item
  В противном случае, необходимо расценивать форму как вызов функции.
\end{enumerate}

\subsection{Макросы}

Если форма является списком и первый элемент не обозначает специальную
форму, возможно он является именем \emph{макроса}. Если так, то форма
называется \emph{макровызовом или вызовом макроса (macrocall)}. Макрос
это функция, которая принимает формы и возвращает формы. Возвращённые формы
подставляются в то место, где происходил макровызов, и затем выполняются. (Этот
процесс иногда называется \emph{раскрытием макроса}.)
Например, макрос с именем \cdf{return} принимает форму, вот так: \cd{(return x)},
и полученная в результате раскрытия форма такая: \cd{(return-from {\nil} x)}. Мы
говорим: старая форма раскрылась в новую. Новая форма будет вычислена на месте
оригинальной формы. Значение новой формы будет возвращено, как значение
оригинальной формы.

В Common Lisp'е существует некоторое количество стандартных макросов, и
пользователь может определять свои макросы используя \cdf{defmacro}.

Макросы, предоставляемые реализацией Common Lisp'а и описанные здесь, могут
раскрываться в код, который не будет являться переносимым между реализациями.
Вызов макроса является портабельным, в то время как результат раскрытия нет.

\subsection{Вызовы функций}

Если список выполняется как форма, и первый элемент не является символом,
обозначающим специальную форму или макрос, тогда предполагается, что список
является \emph{вызовом функции}. Первый элемент списка является именем
функции. Все следующие элементы списка будут вычислены. Одно значение каждого
вычисленного элемента будет является \emph{аргументом} для вызываемой
функции. 
Затем функция будет \emph{применена} к аргументам. Вычисление функции обычно
возвращает значение, однако вместо этого может быть выполнен нелокальный выход,
смотрите \cdf{throw}. Функция может возвращать 0 и более значений, смотрите
\cdf{values}. 
Если и когда функция возвращает значения, они становятся значениями вычисления
формы вызова функции.

Например, рассмотрим вычисление формы: \cd{(+ (* 4 5))}.
Символ \cdf{+} обозначает функцию сложения, а не специальную форму или макрос.
Таким образом две формы \cd{3} и \cd{(* 4 5)} вычисляются для аргументов. Форма
\cd{3} вычисляется в \cd{3}, а форма \cd{(* 4 5)} является вызовом функции
(умножения). Таким образом формы \cd{4} и \cd{5} вычисляются сами в себя, тем
самым предоставляя аргументы для функции умножения. Функция умножения вычисляет
результат \cd{20} и возвращает его. Значения \cd{3} и \cd{20} становятся
аргументами функции сложения, которая вычисляет и возвращает результат
\cd{23}. Таким образом мы говорим \cd{(+3 (* 4 5)) \EV\ 23}.

\section{Функции}

Существуют два метода указать функцию для использования в форме вызова
функции. Один из них заключается в указании символа имени функции. Это
использование символов для обозначения функций полностью независимо от их
использования для обозначения специальных и лексических переменных. Другой путь
заключается в использовании \emph{лямбда-выражения}, которое является списком
с первым элементом равным \cdf{lambda}. Лямбда-выражение \emph{не} является
формой, оно не может быть полноценно вычислено. Лямбда выражения и символы,
когда они используются в программах для обозначения функций, могут быть
указаны в качестве первого элемента формы вызова функции, или только в качестве
второго параметры в специальной форме \cdf{function}. Следует отметить, что
в этих двух контекстах символы и лямбда-выражения обрабатываются, как
\emph{имена} функций. Необходимо отличать это от обработки символов и лямбда
выражений, как \emph{функциональных объектов, или объектов функций (function
  objects)}, которые удовлетворяют предикату \cdf{functionp}, как при
представлении таких объектов в вызовы функций \cdf{apply} или \cdf{funcall}. 

\subsection{Именованные функции}

Имя может задано функции двумя способами.
\emph{Глобальное имя} может быть дано функции с помощью конструкции
\emph{defun}.
\emph{Локальное имя} может быть дано функции с помощью специальных форм
\cdf{flet} или \cdf{labels}.
Когда функция именуется, лямбда-выражение связывается с этим именем вместе с
информацией о сущностях, которые были лексически доступны на момент связи.
Если символ используется в качестве первого элементы формы вызова функции, тогда
он ссылается на определение функции из наиболее ближней формы \cdf{flet} или
\cdf{labels}, которые по тексту содержат эту форму, иначе символ ссылается на
глобальное определение функции, при отсутствии вышеназванных форм.

\subsection{Лямбда-выражения}
\label{LAMBDA-EXPRESSIONS-SECTION}
\indexterm{lambda-expression}
\indexterm{\&optional}
\indexterm{\&rest}
\indexterm{\&key}
\indexterm{\&allow-other-keys}
\indexterm{\&aux}

\emph{Лямбда выражение} является списком со следующим синтаксисом:
\begin{lisp}
(lambda \emph{lambda-list} . \emph{body})
\end{lisp}
Первый элемент должен быть символом \cdf{lambda}. Второй элемент должен быть
списком. Он называется \emph{лямбда-списком}, и задаёт имена для
\emph{параметров} функции. Когда функция, обозначенная лямбда-выражением,
применяется к аргументам, аргументы подставляются в соответствии с
лямбда-списком. \emph{body} может впоследствии ссылаться на аргументы используя
имена параметров. \emph{body} состоит из любого количества форм (возможно
нулевого количества). Эти формы выполняются последовательно, и в качестве
значения возвращается результат только \emph{последней} формы (в случае отсутствия
форм, возвращается {\false}).
Полный синтаксис лямбда-выражения:

\begingroup
\def\GrossOptVars{\Mstar{\emph{var} {\Mor} \cd{(}\emph{var} \Mopt{\emph{initform} \Mopt{\emph{svar}}}\cd{)}}}
\def\GrossKeyOptVars{\Mstar{\emph{var} {\Mor} \cd{(}\Mgroup{\emph{var} {\Mor} \cd{(}\emph{keyword} \emph{var}\cd{)}} \Mopt{\emph{initform} \Mopt{\emph{svar}}}\cd{)}}}
\def\GrossAuxVars{\Mstar{\emph{var} {\Mor} \cd{(}\emph{var} \Mopt{\emph{initform}}\cd{)}}}
\begin{lisp}
(lambda (\Mstar{\emph{var}} \\
~~~~~~~~~\Mopt{\cd{\&optional} \GrossOptVars} \\
~~~~~~~~~\Mopt{\cd{\&rest} \emph{var}} \\
~~~~~~~~~\Mopt{\cd{\&key} \GrossKeyOptVars 
~~~~~~~~~~~~~~\Mopt{\cd{\&allow-other-keys}}} \\
~~~~~~~~~\Mopt{\cd{\&aux} \GrossAuxVars}) \\
~~~\Mchoice{\Mstar\emph{declaration} {\Mor} \emph{documentation-string}} \\
~~~\Mstar{\emph{\,form}})
\end{lisp}
\endgroup

Каждый элемент лямбда-списка является или спецификатором параметра или
\emph{ключевым символом лямбда-списка}. Ключевые символы лямбда списка
начинаются с символа \cd{\&}.
Следует отметить, что ключевые символы лямбда списка не является ключевыми
символами в обычном понимании. Они не принадлежат пакету keyword. Они являются
обычными символами, имена которых начинается амперсандом. Такая терминология
запутывает, но так сложилась история.

Лямбда-список имеет пять частей, любая или все могут быть пустыми:

\begin{itemize}

\item
Спецификаторы для \emph{обязательных параметров}. К ним относятся все
спецификаторы параметров до первого ключевого символа лямбда-списка. Если такой
ключевой символ отсутствует, все спецификаторы считаются обязательными.

\item
Спецификаторы для \emph{необязательных} параметров.
Если указан ключевой символ \cd{\&optional}, после него будут следовать спецификаторы
\emph{необязательных} параметров вплоть до следующего ключевого слова, или до
конца списка.

\item
Спецификатор для \emph{неопределённого количества или оставшегося (rest)}
параметра. Если указан ключевой символ \cd{\&rest}, за ним должен следовать
только один спецификатор \emph{оставшегося (rest)} параметра, за которым может
следовать другой ключевой символ или лямбда-список может заканчиваться.

\item
Спецификатор для \emph{именованных (keyword)} параметров. Если указан ключевой
символ \cd{\&key}, все спецификаторы после данного символа до
следующего ключевого символа или конца списка являются спецификаторами
\emph{именованных} параметров. За спецификаторами именованных параметров
опционально может использовать ключевой символ
\cd{\&allow-others-keys}.

\item
Спецификатор для \emph{вспомогательных (aux)} переменных. Они не являются
параметрами. Если указан ключевой символ \cd{\&aux}, все
спецификаторы после него являются спецификаторами вспомогательных переменных.
\end{itemize}

Когда функция, заданная лямбда-выражением, применяется к аргументам, то эти
аргументы и параметры вычисляются слева направо.
В простейшем случае, в лямбда-списке присутствуют только обязательные
параметры. Каждый из них задаётся просто именем переменной \emph{var}
параметра.
Когда функция применяется, аргументов должно быть столько же, сколько и
параметров, и каждый параметр связывается с одним аргументом. В общем случае,
каждый параметр связывается как лексическая переменная, если только с помощью
декларации не указано, что связь должна осуществляться, как для специальной
переменной. Смотрите \cdf{defvar}, \cdf{proclaim}, \cdf{declare}.

В более общем случае, если указано \emph{n} обязательных параметров
(\emph{n} может равняться нулю), тогда должно быть как минимум \emph{n}
аргументов, и обязательные параметры будут связаны с \emph{n} первыми
аргументами.

Если указаны необязательные параметры, тогда каждый из них будет обработан так,
как описано ниже. Если осталось некоторое количество аргументов, тогда
переменная параметра \cd{var} будет связана с оставшимся аргументом. Принцип
такой же, как и для обязательных параметров. Если не осталось аргументов, тогда
выполняется часть \emph{initform}, и переменная
параметра связывается с её результатом (или с {\false}, если форма
\emph{initform} не была задана).
Если в спецификаторе указано имя ещё одной переменной \emph{svar}, то она
связывается с \emph{true}, если аргумент был задан, и с \emph{false}
аргумент не был задан (и в таком случае выполнилась \emph{initform}).
Переменная \emph{svar} называется \emph{supplied-p} параметр. Она
связывается не с аргументом, а со значением, которое показывает был ли задан
аргумент для данного параметра или нет.

После того, как все \emph{необязательные} параметры были обработаны, может
быть указан \emph{оставшийся (rest)} параметр.
Если \emph{оставшийся (rest)} параметр указан, он будет связан со списком все
оставшихся необработанных аргументов. Если таких аргументов не осталось,
\emph{оставшийся (rest)} параметр будет связан с пустым списком. Если в лямбда
списке отсутствуют \emph{оставшийся (rest)} параметр и \emph{именованные
  (keyword)} параметры, то необработанных аргументов оставаться не должно (иначе
будет ошибка).

Далее обрабатываются все \emph{именованные (keyword)} параметры.
Для этих параметров обрабатываются те же аргументы, что и для
\emph{оставшегося (rest)} параметра.
Безусловно, возможно указывать и \cd{\&rest} и \cd{\&key}. В таком случае
оставшиеся аргументы используются для обеих целей:
все оставшиеся аргументы составляются в список для \cd{\&rest} параметра и они
также обрабатываются, как \cd{\&key} параметры. Только в этой ситуации один
аргумент может обрабатываться более чем для одного параметра.
Если указан \cd{\&key}, должно остаться чётное количество аргументов. Они будут
обработаны попарно. Первый аргумент в паре должен быть ключевым символом,
который задаёт имя параметра, второй аргумент должен быть соответствующим
значением.

В каждом именованном параметре спецификатор должен быть назван \emph{var} для
переменной параметра. FIXME
Если явно указан ключевой символ, тогда он будет использоваться для имени
параметра. В противном случае используется имя переменной \cdf{var} для поиска
ключевого символа в аргументах. Таким образом:
\begin{lisp}
(defun foo (\cd{\&key} radix (type 'integer)) ...)
\end{lisp}
означает то же, что и
\begin{lisp}
(defun foo (\cd{\&key} ((:radix radix)) ((:type type) 'integer)) ...)
\end{lisp}

Спецификатор именованного (keyword) параметра, как и все спецификаторы
параметров, обрабатывается слева направо.
Для каждого спецификатора именованного параметра, если в паре аргумента, в
которой ключевой символ совпадает с именем параметра (сравнение производится с
помощью \cdf{eq}), тогда переменная параметра связывается значением из этой
пары.
Если имеется более одной пар аргументов с одинаковым именем, то это не ошибка. В
таком случае используется наиболее левая пара.
Если пары аргументов не нашлось, тогда выполняется \emph{initform} и
переменная параметра связывается с этим значением (или с {\false}, если
\emph{initform} не задана). Переменная \emph{svar} используется в тех же
целях, что и для \emph{необязательных} параметров. Она будет связана с
\emph{истиной}, если была необходимая пара аргументов, и иначе --- с \emph{ложью}.

Если пара аргументов содержит ключевой символ, который не
присутствует в спецификаторах параметров в лямбда списке, то или возникнет
ошибка или возможны следующие условия:
\begin{itemize}
\item
В лямбда-списке был указан \cd{\&allow-other-keys}.

\item
Где-то среди именованных аргументов есть пара, в которой есть ключевой символ
\cd{:allow-other-keys} и значение не равно {\false}.
\end{itemize}

В случае возникновения одного из этих условий, можно использовать именованные
аргументы, которые не имеют соответствующих параметров (эти аргументы будут
доступны, как оставшийся \cd{\&rest} параметр). 
Целью этого механизма является возможность объединять лямбда-списки разных
функции без необходимости копировать все спецификаторы именованных (keyword)
параметров. Например функция обёртка может передать часть именованных аргументов
в обернутую функцию без необходимости явного ручного указания их всех. 

После того как все спецификаторы были обработаны, слева направо обрабатываются
спецификаторы вспомогательных параметров. Для каждого из них выполняется
\emph{initform} и переменная \emph{var} связывается с этим результатом (или
с {\false}, если \emph{initform} не определена). С \cd{\&aux} переменными
можно делать то же, что и со специальной формой \cdf{let*}:
\begin{lisp}
(lambda (x y \&aux (a (car x)) (b 2) c) ...) \\
~~~\EQ\ (lambda (x y) (let* ((a (car x)) (b 2) c) ...))
\end{lisp}

Что использовать зависит только от стиля.

Когда какая-либо форма \emph{initform} выполняется в каком-либо спецификаторе
параметра, данная форма может ссылаться на любую переменную параметра, стоящую
слева от данной формы, включая supplied-p переменные, и может рассчитывать на
то, что другие переменные параметров ещё не связаны (включая переменную данного
параметра).

После того как был обработан лямбда-список, выполняются формы из тела
лямбда-выражения. Эти формы могут ссылаться на аргументы функции, используя
имена параметров. При выходе из функции, как с помощью нормального возврата, так
и с помощью нелокального выхода, связывания параметров, и лексические, и
специальные, упраздняются. В случае создания <<замыкания>> над данными
связываниями, связи упраздняются не сразу, а сначала сохраняются, чтобы потом
быть вновь восстановленными. 

\noindent
Примеры использования \cd{\&optional} и \cd{\&rest} параметров:
\begin{lisp}
((lambda (a b) (+ a (* b 3))) 4 5) \EV\ 19 \\
((lambda (a \cd{\&optional} (b 2)) (+ a (* b 3))) 4 5) \EV\ 19 \\
((lambda (a \cd{\&optional} (b 2)) (+ a (* b 3))) 4) \EV\ 10 \\
((lambda (\cd{\&optional} (a 2 b) (c 3 d) \cd{\&rest} x) (list a b c d x))) \\*
~~~\EV\ (2 {\false} 3 {\false} {\false}) \\
((lambda (\cd{\&optional} (a 2 b) (c 3 d) \cd{\&rest} x) (list a b c d x)) \\*
~6) \\*
~~~\EV\ (6 t 3 {\false} {\false}) \\
((lambda (\cd{\&optional} (a 2 b) (c 3 d) \cd{\&rest} x) (list a b c d x)) \\*
~6 3) \\*
~~~\EV\ (6 t 3 t {\false}) \\
((lambda (\cd{\&optional} (a 2 b) (c 3 d) \cd{\&rest} x) (list a b c d x)) \\*
~6 3 8) \\*
~~~\EV\ (6 t 3 t (8)) \\
((lambda (\cd{\&optional} (a 2 b) (c 3 d) \cd{\&rest} x) (list a b c d x)) \\*
~6 3 8 9 10 11) \\*
~~~\EV\ (6 t 3 t (8 9 10 11))
\end{lisp}
Примеры \cd{\&key} параметров:
\begin{lisp}
((lambda (a b \cd{\&key} c d) (list a b c d)) 1 2) \\*
~~~\EV\ (1 2 {\nil} {\nil}) \\
((lambda (a b \cd{\&key} c d) (list a b c d)) 1 2 :c 6) \\*
~~~\EV\ (1 2 6 {\nil}) \\
((lambda (a b \cd{\&key} c d) (list a b c d)) 1 2 :d 8) \\*
~~~\EV\ (1 2 {\nil} 8) \\
((lambda (a b \cd{\&key} c d) (list a b c d)) 1 2 :c 6 :d 8) \\*
~~~\EV\ (1 2 6 8) \\
((lambda (a b \cd{\&key} c d) (list a b c d)) 1 2 :d 8 :c 6) \\*
~~~\EV\ (1 2 6 8) \\
((lambda (a b \cd{\&key} c d) (list a b c d)) :a 1 :d 8 :c 6) \\*
~~~\EV\ (:a 1 6 8) \\
((lambda (a b \cd{\&key} c d) (list a b c d)) :a :b :c :d) \\
~~~\EV\ (:a :b :d {\nil})
\end{lisp}
Пример смешения всех:
\begin{lisp}
((lambda (a \cd{\&optional} (b 3) \cd{\&rest} x \cd{\&key} c (d a)) \\*
~~~(list a b c d x)) \\*
~1)   \EV\ (1 3 {\nil} 1 ())
\end{lisp}

\begin{lisp}
((lambda (a \cd{\&optional} (b 3) \cd{\&rest} x \cd{\&key} c (d a)) \\*
~~~(list a b c d x)) \\*
~1 2)   \EV\ (1 2 {\nil} 1 ()) \\
 \\
((lambda (a \cd{\&optional} (b 3) \cd{\&rest} x \cd{\&key} c (d a)) \\*
~~~(list a b c d x)) \\*
~:c 7)   \EV\ (:c 7 {\nil} :c ()) \\
 \\
((lambda (a \cd{\&optional} (b 3) \cd{\&rest} x \cd{\&key} c (d a)) \\*
~~~(list a b c d x)) \\*
~1 6 :c 7)   \EV\ (1 6 7 1 (:c 7)) \\
 \\
((lambda (a \cd{\&optional} (b 3) \cd{\&rest} x \cd{\&key} c (d a)) \\*
~~~(list a b c d x)) \\*
~1 6 :d 8)   \EV\ (1 6 {\nil} 8 (:d 8)) \\
 \\
((lambda (a \cd{\&optional} (b 3) \cd{\&rest} x \cd{\&key} c (d a)) \\*
~~~(list a b c d x)) \\*
~1 6 :d 8 :c 9 :d 10)   \EV\ (1 6 9 8 (:d 8 :c 9 :d 10))
\end{lisp}

В лямбда-выражении, если оно стоит на первом месте в списке формы вызова функции,
допускаются все ключевые символы лямбда-списка, хотя они и не очень-то полезны в
таком контексте.
Гораздо полезнее их использовать в глобальных функциях, определённых с помощью
\cdf{defun}.

Все символы, что начинаются на \cd{\&} обычно зарезервированы для использования в
качестве ключевых символов лямбда-списка, и не должны использоваться для имён
переменных.
Реализации Common Lisp'а могут также предоставлять свои дополнительные ключевые
символы лямбда-списка.

\begin{defun}[Константа]
lambda-list-keywords

Значение \cdf{lambda-list-keywords} является списком всех ключевых символов
лямбда-списка, используемых в данной реализации, включая те, которые
используются только в \cdf{defmacro}. Этот список должен содержать как минимум
символы \cd{\&optional}, \cd{\&rest}, \cd{\&key}, \cd{\&allow-other-keys},
\cd{\&aux}, \cd{\&body}, \cd{\&whole} и \cd{\&environment}.
\end{defun}

Вот пример использования \cd{\&allow-other-keys} и \cd{:allow-other-keys}, 
рассматривающий функцию, которая принимает два своих именованных аргумента и
также дополнительные именованные аргументы, которые затем передаются
\cdf{make-array}:
\begin{lisp}
(defun array-of-strings (str dims \cd{\&rest} keyword-pairs \\*
~~~~~~~~~~~~~~~~~~~~~~~~~\cd{\&key} (start 0) end \cd{\&allow-other-keys}) \\*
~~(apply \#'make-array dims \\*
~~~~~~~~~:initial-element (subseq str start end) \\
~~~~~~~~~:allow-other-keys t \\*
~~~~~~~~~keyword-pairs))
\end{lisp}

Такая функция принимает строку и информацию о размерности и возвращает массив с
заданной размерностью, каждый из элементов которого равен заданной
строке. Именованные аргументы \cd{:start} и \cd{:end}, как обычно (смотрите
главу~\ref{KSEQUE}), можно использовать для указания того, что должна
использоваться подстрока. Кроме того, использование \cd{\&allow-other-keys} в
лямбда списке указывает на то, что вызов этой функции может содержать
дополнительные именованные аргументы. Для доступа к ним используется \cd{\&rest}
аргумент. Эти дополнительные именованные аргументы передаются в
\cdf{make-array}. \cdf{make-array} не принимает именованные аргументы
\cd{:start} и \cd{:end}, и было бы ошибкой допустить их использование. Однако
указание \cd{:allow-other-keys} равное не-{\false} значению позволяет передавать
любые другие именованные аргументы, включая \cd{:start} и \cd{:end}, и они были
бы приняты и проигнорированы.

\begin{defun}[Константа]
lambda-parameters-limit

Значение \cdf{lambda-parameters-limit} является положительным целым, которое
невключительно является верхней границей допустимого количества имён параметров,
которые могут использоваться в лямбда-списке.
Значение зависит от реализации, но не может быть менее 50.
Разработчики поощряются за создание данной границы как можно большей без потери
производительности.
Смотрите \cdf{call-arguments-list}.
\end{defun}

\section{Формы верхнего уровня}

Стандартный путь взаимодействия с реализацией Common Lisp через
\emph{цикл чтение-выполнение-печать} (\emph{read-eval-print loop}): система
циклично считывает форму из некоторого источника ввода (клавиатура, файл),
выполняет её, затем выводит результат(ы) в некоторое устройство вывода (дисплей,
другой файл). Допускается любая форма (выполняемый объект данных), однако
существуют некоторые формы разработанные для удобного применения в качестве форм 
\emph{верхнего уровня}.
Эти специальные формы верхнего уровня могут использоваться для определения
глобальных функции (globally named functions), макросов, создания деклараций и
определения глобальных значений для специальных переменных.

Макросы обычно определяются с помощью специальной формы \cdf{defmacro}.
Этот механизм достаточно сложен. Для подробностей смотрите главу~\ref{MACROS}.

\subsection{Определение именованных функций}

Специальная форма \cdf{defun} обычно обозначает определение именованных функций.

\begin{defmac}
defun name lambda-list <{declaration}* | doc-string> {\,form}*

Выполнение формы \cdf{defun} приводит к тому, что символ \emph{name}
становиться глобальным именем для функции определённой лямбда-выражением.
\begin{lisp}
(lambda \emph{lambda-list} \Mstar{\emph{declaration} {\Mor} \emph{doc-string}} \Mstar{\emph{\,form}})
\end{lisp}
определяется в лексическом окружении, в котором выполнялась форма
\cdf{defun}. А так как формы \cdf{defun} обычно выполняются на самом верхнем
уровне, лямбда-выражение обычно выполняется в нулевом лексическом окружении.


\cdf{defun} в качестве параметра \emph{name} принимает любое имя функции (символ
или список, у которого \emph{car} элемент равен \cdf{setf}---смотрите
раздел~\ref{FUNCTION-NAME-SECTION}).

Так теперь можно записать
\begin{lisp}
(defun (setf cadr) ...)
\end{lisp}
для определения \cdf{setf}-оператора для функции \cdf{cadr}. Это удобнее,
чем использование \cdf{defsetf} или \cdf{define-modify-macro}.

Если указана необязательная строка документации \emph{doc-string}, тогда она
присоединяется к символу \emph{name} в качестве строки документации типа
\cdf{function}. Смотрите \cdf{documentation}. Если после \emph{doc-string} нет
деклараций, строка документации может быть использована только при условии
существования хотя бы одной формы после неё, иначе она будет использована в
качестве форм \emph{form} функции. Указывать более чем одну строку
\emph{doc-string} является ошибкой.

Формы \emph{forms} составляют тело определяемой функции. Они выполняются как
неявный \cdf{progn}.

Тело определяемой функции неявно заключается в конструкцию \cdf{block}, имя
которой совпадает с \emph{именем (name)} функции. Таким образом для выхода из функции
может быть использовано выражение\cdf{return-from}.

В некоторых реализациях в \cdf{defun} могут также выполняться другие специальные
учётные действия. \emph{name} возвращается в качестве значения формы \cdf{defun}.
Например:
\begin{lisp}
(defun discriminant (a b c) \\
~~(declare (number a b c)) \\
~~"Вычисляет дискриминант квадратного уравнения. \\
~~~Получает a, b и c, вычисляет значение b{\Xcircumflex}2-4*a*c. \\
~~~Квадратное уравнение a*x{\Xcircumflex}2+b*x+c=0 имеет действительные, multiple, \\
~~~или комплексные корни в зависимости от того, какое соответственно значение было получено \\
~~~положительное, ноль или отрицательное." \\
~~(- (* b b) (* 4 a c))) \\
~~~\EV\ discriminant \\
~~~\textrm{теперь} (discriminant 1 2/3 -2) \EV\ 76/9
\end{lisp}

Пользователю разрешено использовать \cdf{defun} для переопределения функции, например, для
установки корректной версии некорректного определения.
Пользователю также разрешено переопределять макрос на функцию.
Однако является ошибкой, попытка переопределить имя специальной формы (смотрите
таблицу~\ref{SPECIAL-FORM-TABLE}) на функцию.
\end{defmac}

\subsection{Определение глобальных переменных и констант}

Для определения глобальных переменных используются  специальные формы
\cdf{defvar} и \cdf{defparameter}. 
Для определения констант используется специальная форма \cdf{defconstant}. 

\begin{defmac}
defvar name [initial-value [documentation]] \\
defparameter name initial-value [documentation] \\
defconstant name initial-value [documentation]

\cdf{defvar} рекомендуется для декларации использования в программе специальных
переменных.
\begin{lisp}
(defvar \emph{variable})
\end{lisp}
указывает на то, что переменная \emph{variable} будет специальной
(\cdf{special}) (смотрите \cdf{proclaim}), и может выполнять некоторые учётные
действия, зависимые от реализации.

Если для формы указан второй аргумент,
\begin{lisp}
(defvar \emph{variable} \emph{initial-value})
\end{lisp}
тогда переменная \emph{variable}, если она ещё не была проинициализирована,
инициализируется результатом выполнения формы \emph{initial-value}. Форма
\emph{initial-value} не выполняется, если в этом нет необходимости. Это
полезно, если форма \emph{initial-value} выполняет что-то
трудоёмкое, как, например, создание большой структуры данных.

Если не существует специального связывания этой переменной, инициализация
производится присвоением глобального значения переменной. 
Обычно, такого связывания быть не должно. FIXME.

\cdf{defvar} также предоставляют хорошее место для комментария, описывающего
значение переменной, тогда как обычное \cd{special} указание соблазняет
задекларировать несколько переменных за один раз и не предоставляет возможности
прокомментировать их.
\begin{lisp}
(defvar *visible-windows* 0 \\
~~"Количество видимых окон на экране")
\end{lisp}

\cdf{defparameter} подобна \cdf{defvar}, но \cdf{defparameter} требует
обязательной формы \emph{initial-value}, и, выполняя эту форму присваивает
результат переменной. Семантическое различие заключается в том, что 
\cdf{defvar} предназначена декларировать переменную, изменяемую программой,
тогда как \cdf{defparameter} предназначена для декларации переменной, как
константы, которая может быть изменена (и во время выполнения программы), для
изменения поведения программы. Таким образом \cdf{defparameter} не указывает,
что количество никогда не изменяется, в частности, она не разрешает 
компилятору предположить то, что значение может быть вкомпилировано в
программу.

\cdf{defconstant} похожа на \cdf{defparameter}, но в отличие от последней,
указывает, что значение переменной \emph{name} фиксировано и позволяет
компилятору предположить, что значение может быть вкомпилировано в
программу. Однако, если компилятор  для оптимизации выбирает путь замены ссылок
на имя константы на значения этой константы в компилируемом коде, он должен
позаботиться о том, чтобы такие <<копии>> были эквивалентны \cdf{eql}
объектам-значениям констант. Например, компилятор может спокойно копировать
числа, но должен позаботиться об этом правиле, если значение константы
является списком.

Если для переменной на момент вызова формы \cdf{defconstant} существует
специальные связывания, то возникает ошибка (но реализации могут проверять, а
могут и не проверять этот факт).

Если имя задекларировано с помощью \cdf{defconstant},
последующие присваивания и связывания данной специальной переменной будут
являться ошибкой. Это справедливо для системозависимых констант, например,
\cdf{t} и \cdf{most-positive-fixnum}.
Компилятор может также сигнализировать о связывании лексической переменной с
одинаковым именем.

Для любой из этих конструкций, документация должна быть строкой. Строка
присоединяется к имени переменной, параметра или константы как тип документации
\cdf{variable}, смотрите функцию \cdf{documentation}.

\emph{documentation-string}
не выполняется и должна представлять строку, когда выполняется \cdf{defvar},
\cdf{defparameter} или \cdf{defconstant}.

Например, форма
\begin{lisp}
(defvar *avoid-registers* nil "Compilation control switch \#43")
\end{lisp}
законна, но
\begin{lisp}
(defvar *avoid-registers* nil \\*
~~(format nil "Compilation control switch \#{\Xtilde}D" \\*
~~~~~~~~~~(incf *compiler-switch-number*)))
\end{lisp}
ошибочна, так как вызов \cdf{format} не является дословно строкой.

С другой стороны, форма
\begin{lisp}
(defvar *avoid-registers* nil \\*
~~\#.(format nil "Compilation control switch \#{\Xtilde}D" \\*
~~~~~~~~~~~~(incf *compiler-switch-number*)))
\end{lisp}
Может использоваться для вышеназванной цели, потому что вызов \cdf{format}
выполняется на во время чтения кода \cdf{read}, когда форма \cdf{defvar}
выполняется, в ней указана строка, которая являлась результатом вызова
\cdf{format}.

Эти конструкции обычно используются только как формы верхнего уровня. Значения
возвращаемые каждой из этих конструкций это задекларированные имена \emph{name}.
\end{defmac}

\subsection{Контроль времени выполнения}

\begin{defspec}
eval-when ({situation}*) {\,form}*

Тело формы \cdf{eval-when} выполняется как неявный \cdf{progn}, но только в
перечисленных ниже ситуациях. Каждая ситуация \emph{situation} должна быть
одним символов, \cd{:compile-toplevel}, \cd{:load-toplevel} или \cd{:execute}.

Использование \cd{:compile-toplevel} и \cd{:load-toplevel} контролирует, что и
когда выполняется для форм верхнего уровня. Использование \cd{:execute}
контролирует будет ли производится выполнения форм не верхнего уровня.

Конструкция \cdf{eval-when} может быть более понятна в терминах модели того,
как компилятор файлов, \cdf{compile-file}, выполняет формы в файле для
компиляции.

Формы следующие друг за другом читаются из файла с помощью компилятора файла
используя \cdf{read}. Эти формы верхнего уровня обычно обрабатываются в том,
что мы называем режим <<времени некомпиляции (not-compile-time
mode)>>. Существует и другой режим, называемый режим <<времени-компиляции
(compile-time-too mode)>>, которые вступает в игру для форм верхнего
уровня. Специальная форма \cdf{eval-when} используется для аннотации программы
таким образом, чтобы предоставить программе, осуществлять выбор режима.

Обработка форм верхнего уровня в компиляторе файла работает так, как
рассказано ниже:

\begin{itemize}

\item Если форма является макровызовом, она разворачивается и результат
  обрабатывается, как форма верхнего уровня в том же режиме обработки
  (времени-компиляции или времени-некомпиляции, (compile-time-too или not-compile-time).

\item Если форма \cdf{progn} (или \cdf{locally}), каждая из форм из их тел
  обрабатываются, как формы верхнего уровня в том же режиме обработки.

\item Если форма \cdf{compiler-let}, \cdf{macrolet} или
  \cdf{symbol-macrolet}, компилятор файла создаёт соответствующие связывания и
  рекурсивно обрабатывает тела форм, как неявный \cdf{progn} верхнего уровня
  в контексте установленных связей в том же режиме обработки.

\item Если форма \cdf{eval-when}, она обрабатывается в соответствии со
  следующей таблицей:
  \begin{flushleft}
    \begin{tabular*}{\linewidth}{@{\extracolsep{\fill}}c@{}cccl@{}}
      LT  &CT    &EX  &CTTM &Действие \\ \hlinesp
      да  & да   &--  & --  &    обработать тело в режиме время-компиляции \\
      да  & нет  &да  & да  &    обработать тело в режиме время-компиляции \\
      да  & нет  &--  & нет &    обработать тело в режиме время-некомпиляции \\
      да  & нет  &нет & --  &    обработать тело в режиме время-некомпиляции \\
      нет & да   &--  & --  &    выполнить тело \\
      нет & нет  &да  & да  &    выполнить тело \\
      нет & нет  &--  & нет &    ничего не делать \\
      нет & нет  &нет  & -- &    ничего не делать \\
      \hline
    \end{tabular*}
  \end{flushleft}
  В этой таблице столбец LT спрашивает присутствует ли \cd{:load-toplevel} в
  ситуациях указанных в форме \cdf{eval-when}.
  CT соответственно указывает на \cd{:compile-toplevel} и EX на
  \cd{:execute}. Столбец CTTM спрашивает встречается ли форма \cdf{eval-when}
  в режиме времени-компиляции. Фраза <<обработка тела>> означает обработку
  последовательно форм тела, как неявного \cdf{progn} верхнего уровня в
  указанном режиме, и <<выполнение тела>> означает выполнение форм тела
  последовательно, как неявный \cdf{progn} в динамическом контексте
  выполнения компилятора и в лексическом окружении, в котором встретилась \cdf{eval-when}.

\item В противном случае, форма верхнего уровня, которая не представлена в
  специальных случаях. Если в режиме времени-компиляции, компилятор сначала
  выполняет форму и затем выполняет обычную обработку компилятором. Если
  установлен режим времени-некомпиляции, выполняется только обычная обработка
  компилятором (смотрите раздел~\ref{COMPILER-SECTION}).
  Любые подформы обрабатываются как формы не верхнего уровня.
\end{itemize}

Следует отметить, что формы верхнего уровня обрабатываются гарантированно в
порядке, в котором они были перечислены в тексте в файле, и каждая форма
верхнего уровня прочтённая компилятором обрабатывается перед тем, как будет
прочтена следующая.
Однако, порядок обработки (включая, в частности, раскрытие макросов) подформ,
которые не являются формами верхнего уровня, не определён.

Для формы \cdf{eval-when}, которая не является формой верхнего уровня в
компиляторе файлов (то есть либо в интерпретаторе, либо \cdf{compile}, либо в
компиляторе файлов, но не на верхнем уровне), если указана ситуация
\cd{:execute}, тело формы обрабатывается как неявный \cdf{progn}. В противном
случае, тело игнорируется и форма \cdf{eval-when} имеет значение \cdf{nil}.

Для сохранения обратной совместимости, \emph{situation} может также быть
\cdf{compile}, \cdf{load} или \cdf{eval}.
Внутри формы верхнего уровня \cdf{eval-when}, они имеют значения
\cd{:compile-toplevel}, \cd{:load-toplevel} и \cd{:execute} соответственно.
Однако их поведение не определено при использовании в \cdf{eval-when} не
верхнего уровня.

Следующие правила являются логическим продолжением предыдущих определений:

\begin{itemize}

\item Никогда не случится так, чтобы выполнение одного \cdf{eval-when}
  выражения приведёт к выполнению тела более чем один раз.

\item Старый ключевой символ \cd{eval} был неправильно использован, потому
  что выполнение тела не нуждается в \cd{eval}. Например, когда определение
  функции
  \begin{lisp}
    (defun foo () (eval-when (:execute) (print 'foo)))
  \end{lisp}
  скомпилируется,
  вызов \cdf{print} должен быть скомпилирован, а не выполнен во время
  компиляции.

\item Макросы, предназначенные для использования в качестве форм верхнего
  уровня, должны контролировать все побочные эффекты, которые будут сделаны
  формами в процессе развёртывания.
  Разворачиватель макроса сам по себе не должен порождать никаких побочных
  эффектов.
  
  \begin{lisp}
    (defmacro foo () \\*
    ~~(really-foo)~~~~~~~~~~~~~~~~~~~~~~~~~~~~~~;\textrm{Неправильно}\\*
    ~~{\Xbq}(really-foo)) \\
    \\
    (defmacro foo () \\*
    ~~{\Xbq}(eval-when (:compile-toplevel \\*
    ~~~~~~~~~~~~~~~:load-toplevel :execute)~~~~~;\textrm{Правильно} \\*
    ~~~~(really-foo)))
  \end{lisp}
  Соблюдение этого правила будет значит, что такие макросы будут вести себя
  интуитивно понятно при вызовах в формах не верхнего уровня.

\item Расположение связывания переменной окружённой \cdf{eval-when}
  захватывает связывание, потому что режим <<время-компиляции>> не может
  случиться (потому что \cdf{eval-when} не может быть формой верхнего уровня)
  \begin{lisp}
    (let ((x 3)) \\*
    ~~(eval-when (:compile-toplevel :load-toplevel :execute) \\*
    ~~~~(print x)))
  \end{lisp}
  выведет 3 во время выполнения (в данном случае загрузки) и не будет ничего
  выводить во время компиляции. Разворачивание \cdf{defun} и \cdf{defmacro} может
  быть выполнено в контексте \cdf{eval-when} и могут корректно захватывать
  лексическое окружение.
  Например, реализация может разворачивать форму \cdf{defun}, такую как:
  \begin{lisp}
    (defun bar (x) (defun foo () (+ x 3)))
  \end{lisp}
  в
  \begin{lisp}
    (progn (eval-when (:compile-toplevel) \\*
    ~~~~~~~~~(compiler::notice-function 'bar '(x))) \\*
    ~~~~~~~(eval-when (:load-toplevel :execute) \\*
    ~~~~~~~~~(setf (symbol-function 'bar) \\*
    ~~~~~~~~~~~~~~~\#'(lambda (x) \\*
    ~~~~~~~~~~~~~~~~~~~(progn (eval-when (:compile-toplevel)  \\*
    ~~~~~~~~~~~~~~~~~~~~~~~~~~~~(compiler::notice-function 'foo \\*
    ~~~~~~~~~~~~~~~~~~~~~~~~~~~~~~~~~~~~~~~~~~~~~~~~~~~~~~~'())) \\*
    ~~~~~~~~~~~~~~~~~~~~~~~~~~(eval-when (:load-toplevel :execute) \\*
    ~~~~~~~~~~~~~~~~~~~~~~~~~~~~(setf (symbol-function 'foo) \\*
    ~~~~~~~~~~~~~~~~~~~~~~~~~~~~~~~~~~\#'(lambda () (+ x 3)))))))))
  \end{lisp}
  которая по предыдущим правилам будет обработана также, как и 
  \begin{lisp}
    (progn (eval-when (:compile-toplevel) \\*
    ~~~~~~~~~(compiler::notice-function 'bar '(x))) \\*
    ~~~~~~~(eval-when (:load-toplevel :execute) \\*
    ~~~~~~~~~(setf (symbol-function 'bar) \\*
    ~~~~~~~~~~~~~~~\#'(lambda (x) \\*
    ~~~~~~~~~~~~~~~~~~~(progn (eval-when (:load-toplevel :execute) \\*
    ~~~~~~~~~~~~~~~~~~~~~~~~~~~~(setf (symbol-function 'foo) \\*
    ~~~~~~~~~~~~~~~~~~~~~~~~~~~~~~~~~~\#'(lambda () (+ x 3)))))))))
  \end{lisp}

\end{itemize}

Вот несколько дополнительных примеров.
\begin{lisp} 
  (let ((x 1)) \\*
  ~~(eval-when (:execute :load-toplevel :compile-toplevel) \\*
  ~~~~(setf (symbol-function 'foo1) \#'(lambda () x))))
\end{lisp}
\cdf{eval-when} в предыдущем выражении не является формой верхнего уровня, таким
образом во внимание берётся только ключевой символ \cd{:execute}. это не будет
иметь эффекта во время компиляции. Однако этот код установит
в \cd{(symbol-function 'foo1)} функцию которая возвращает \cd{1} во время
загрузки (если \cdf{let} форма верхнего уровня) или во время выполнения (если
форма \cdf{let} вложена в какую-либо другую форму, которая ещё не была
выполнена). 
\begin{lisp}
  (eval-when (:execute :load-toplevel :compile-toplevel) \\*
  ~~(let ((x 2)) \\*
  ~~~~(eval-when (:execute :load-toplevel :compile-toplevel) \\*
  ~~~~~~(setf (symbol-function 'foo2) \#'(lambda () x)))))
\end{lisp}
Если предыдущее выражение находилось на верхнем уровне в компилируемом файле, оно
будет выполнятся в обоих случаях, и во время компиляции и во время загрузки.

\begin{lisp}
  (eval-when (:execute :load-toplevel :compile-toplevel) \\*
  ~~(setf (symbol-function 'foo3) \#'(lambda () 3)))
\end{lisp}
Если предыдущее выражение находилось на верхнем уровне в компилируемом файле, оно
будет выполняться в обоих случаях, и во время компиляции и во время загрузки.

\begin{lisp}
  (eval-when (:compile-toplevel) \\*
  ~~(eval-when (:compile-toplevel)  \\*
  ~~~~(print 'foo4)))
\end{lisp}
Предыдущее выражение ничего не делает, оно просто возвращает \cdf{nil}.

\begin{lisp}
  (eval-when (:compile-toplevel)  \\*
  ~~(eval-when (:execute) \\*
  ~~~~(print 'foo5)))
\end{lisp}
Если предыдущее выражение находилось на верхнем уровне в компилируемом файле,
\cd{foo5} будет выведено во время компиляции. Если эта форма была не на верхнем
уровне, ничего не будет выведено во время компиляции. Вне зависимости от 
контекста, ничего не будет выведено во время загрузки или выполнения.

\begin{lisp}
  (eval-when (:execute :load-toplevel) \\*
  ~~(eval-when (:compile-toplevel) \\*
  ~~~~(print 'foo6)))
\end{lisp}
Если предыдущая форма находилась на верхнем уровне в компилируемом файле,
\cd{foo6} будет выведено во время компиляции. Если форма была не на верхнем
уровне, ничего не будет выведено во время компиляции. Вне зависимости от
контекста, ничего не будет выведение во время загрузки или выполнения кода.
\end{defspec}

\fi       % Representation of programs, simple DEFUN
%Part{Preds, Root = "CLM.MSS"}
%Chapter of Common Lisp Manual.  Copyright 1984, 1988, 1989 Guy L. Steele Jr.

\clearpage\def\pagestatus{FINAL PROOF}

\ifx \rulang\Undef

\chapter{Predicates}
\label{PREDS}

A \emph{predicate} is a function that tests for some condition involving
its arguments and returns {\false} if the condition is false, or some
non-{\false} value if the condition is true.  One may think of a predicate as
producing a Boolean value, where {\false} stands for \emph{false} and anything
else stands for \emph{true}.  Conditional control structures such as
\cdf{cond},
\cdf{if}, \cdf{when}, and \cdf{unless} test such Boolean values.
We say that a predicate \emph{is true} when it returns a non-{\false} value,
and \emph{is false} when it returns {\false}; that is, it is true or false
according to whether the condition being tested is true or false.

By convention, the names of predicates usually end in the letter
\cd{p} (which stands for ``predicate'').
Common Lisp uses a uniform convention in hyphenating names of predicates.
If the name of the predicate is formed by adding a \cd{p} to
an existing name, such as the name of a data type,
a hyphen is placed before the final \cd{p} if and only if there is
a hyphen in the existing name.  For example, \cd{number} begets \cd{numberp}
but \cd{standard-char} begets \cd{standard-char-p}.
On the other hand, if the name of a predicate is formed by adding
a prefixing qualifier to the front of an existing predicate name,
the two names are joined with a hyphen and the presence or absence
of a hyphen before the final \cd{p} is not changed.  For example,
the predicate \cd{string-lessp} has no hyphen before the \cd{p}
because it is the string version of \cdf{lessp} (a MacLisp function
that has been renamed \cdf{<} in Common Lisp).  The name \cd{string-less-p}
would incorrectly imply that it is a predicate that tests for a kind
of object called a \cdf{string-less}, and the name \cd{stringlessp}
would connote a predicate that tests whether something has no strings
(is ``stringless'')!

The control structures that test Boolean values only test for
whether or not the value is {\false}, which is considered to be false.  Any
other value is considered to be true.  Often a predicate will return {\false} if
it ``fails'' and some \emph{useful} value if it ``succeeds'';
such a function can be used not only as a test but
also for the useful value provided in case of success.  An example
is \cdf{member}.

If no better non-{\nil} value is available for the purpose of indicating
success, by convention the symbol \cdf{t} is used as the ``standard''
true value.

\section{Logical Values}

The names \cdf{nil} and \cdf{t} are constants in Common Lisp.  Although they
are symbols like any other symbols, and appear to be treated
as variables when evaluated, it is not permitted to modify their
values.  See \cdf{defconstant}.

\begin{defun}[Constant]
nil

The value of {\nil} is always {\nil}.  This object represents the logical
\emph{false} value and also the empty list.  It can also be written \cd{()}.
\end{defun}

\begin{defun}[Constant]
t

The value of \cdf{t} is always \cdf{t}.
\end{defun}

\section{Data Type Predicates}

Perhaps the most important predicates in Lisp are those that deal
with data types;  that is, given a data object one can determine whether
or not it belongs to a given type, or one can compare two type specifiers.

\subsection{General Type Predicates}

If a data type is viewed as the set of all objects belonging to the type,
then the \cdf{typep} function is a set membership test, while \cdf{subtypep}
is a subset test.

\begin{defun}[Function]
typep object type

\cdf{typep} is a predicate that
is true if \emph{object} is of type \emph{type}, and is false otherwise.
Note that an object can be ``of'' more than one type, since one type can
include another.  The \emph{type} may be any of the type specifiers
mentioned in chapter~\ref{DTSPEC} \emph{except} that it may not
be or contain a type specifier list whose first element is \cdf{function}
or \cdf{values}.
A specifier of the form \cd{(satisfies \emph{fn})} is handled simply
by applying the function \emph{fn} to \emph{object}
(see \cdf{funcall}); the \emph{object} is considered
to be of the specified type if the result is not {\false}.

\begin{new}
X3J13 voted in January 1989
\issue{ARRAY-TYPE-ELEMENT-TYPE-SEMANTICS}
to change \cdf{typep} to give specialized
\cdf{array} and \cdf{complex} type specifiers the same meaning for
purposes of type discrimination as they have for declaration purposes.
Of course, this also applies to such type specifiers as \cdf{vector}
and \cdf{simple-array}
(see section~\ref{SPECIALIZED-TYPE-SPECIFIER-SECTION}).
Thus
\begin{lisp}
(typep foo '(array bignum))
\end{lisp}
in the first edition asked the question, Is \cdf{foo} an array
specialized to hold bignums? but under the new interpretation
asks the question, Could the array \cdf{foo} have resulted from
giving \cdf{bignum} as the \cd{:element-type} argument
to \cdf{make-array}?
\end{new}
\end{defun}

\begin{defun}[Function]
subtypep type1 type2

The arguments must be type specifiers that are acceptable to \cdf{typep}.
The two type specifiers are compared; this predicate is true
if \emph{type1} is definitely a (not necessarily proper) subtype of \emph{type2}.
If the result is {\false}, however, then \emph{type1} may or may not be a subtype of
\emph{type2} (sometimes it is impossible to tell, especially when
\cdf{satisfies} type specifiers are involved).
A second returned value indicates the certainty of the result;
if it is true, then the first value is an accurate indication
of the subtype relationship.  Thus there are three possible
result combinations:
\begin{tabbing}
~~~~~~~~\=~~~~~~~~\=\kill
{\true}\>{\true}\>\emph{type1} is definitely a subtype of \emph{type2} \\
{\false}\>{\true}\>\emph{type1} is definitely not a subtype of \emph{type2} \\
{\false}\>{\false}\>\cdf{subtypep} could not determine the relationship
\end{tabbing}

\begin{new}
X3J13 voted in January 1989
\issue{SUBTYPEP-TOO-VAGUE}
to place certain requirements upon the implementation of \cdf{subtypep},
for it noted that implementations in many cases simply ``give up''
and return the two values \cdf{nil} and \cdf{nil} when in fact it would have been
possible to determine the relationship between the given types.
The requirements are as follows, where it is understood that a type specifier \emph{s}
\emph{involves} a type specifier \emph{u} if either \emph{s} contains an occurrence of \emph{u}
directly or \emph{s} contains a type specifier \emph{w} defined by \cdf{deftype} whose
expansion involves \emph{u}.
\begin{itemize}
\item \cdf{subtypep} is not permitted to return a second value of \cdf{nil}
unless one or both of its arguments involves \cdf{satisfies},
\cdf{and}, \cdf{or}, \cdf{not}, or \cdf{member}.
\item \cdf{subtypep} should signal an error when one or both of its arguments
involves \cdf{values} or the list form of the \cdf{function} type specifier.
\item \cdf{subtypep} must always return the two values \cdf{t} and \cdf{t}
in the case where its arguments, after expansion of specifiers
defined by \cdf{deftype}, are \cdf{equal}.
\end{itemize}
In addition, X3J13 voted to clarify that in some cases
the relationships between types
as reflected by \cdf{subtypep} may be implementation-specific.
For example, in an implementation supporting only one type of
floating-point number, \cd{(subtypep 'float 'long-float)} would return
\cdf{t} and \cdf{t}, since the two types would be identical.

Note that \cdf{satisfies} is an exception because relationships between
types involving \cdf{satisfies} are undecidable in general, but (as X3J13 noted)
\cdf{and}, \cdf{or}, \cdf{not}, and \cdf{member} are merely very messy to deal
with.  In all likelihood these will not be addressed unless and
until someone is willing to write a careful specification that covers
all the cases for the processing of these type
specifiers by \cdf{subtypep}.  The requirements stated above were easy
to state and probably suffice for most cases of interest.
\end{new}

\begin{new}
X3J13 voted in January 1989
\issue{ARRAY-TYPE-ELEMENT-TYPE-SEMANTICS}
to change \cdf{subtypep} to give specialized
\cdf{array} and \cdf{complex} type specifiers the same meaning for
purposes of type discrimination as they have for declaration purposes.
Of course, this also applies to such type specifiers as \cdf{vector}
and \cdf{simple-array}
(see section~\ref{SPECIALIZED-TYPE-SPECIFIER-SECTION}).

If \emph{A} and \emph{B} are type specifiers (other than \cdf{*}, which technically
is not a type specifier anyway), then \cd{(array~\emph{A})}
and \cd{(array~\emph{B})} represent the same type in a given implementation
if and only if they denote arrays
of the same specialized representation in that implementation;
otherwise they are disjoint.
To put it another way, they represent the same type
%(and otherwise are disjoint)
if and only if
\cd{(upgraded-array-element-type~'\emph{A})} and
\cd{(upgraded-array-element-type~'\emph{B})} are the same type.
Therefore
\begin{lisp}
(subtypep '(array \emph{A}) '(array \emph{B}))
\end{lisp}
is true if and only if
\cd{(upgraded-array-element-type~'\emph{A})}
is the same type as
\cd{(upgraded-array-element-type~'\emph{B})}.

The \cdf{complex} type specifier is treated in a similar but subtly different
manner.
If \emph{A} and \emph{B} are two type specifiers (but not \cdf{*}, which technically
is not a type specifier anyway), then \cd{(complex~\emph{A})}
and \cd{(complex~\emph{B})} represent the same type in a given implementation
if and only if they refer to complex numbers
of the same specialized representation in that implementation;
otherwise they are disjoint.
Note, however, that there is no function called \cdf{make-complex} that
allows one to specify a particular element type (then to be upgraded);
instead, one must describe specialized complex numbers in terms of
the actual types of the parts from which they were constructed.
There is no number of type (or rather, \emph{representation\/})
\cdf{float} as such; there are only numbers of type \cdf{single-float},
numbers of type \cdf{double-float},
and so on.  Therefore we want \cd{(complex single-float)} to
be a subtype of \cd{(complex float)}.

The rule, then, is that \cd{(complex~\emph{A})}
and \cd{(complex~\emph{B})} represent the same type (and otherwise are disjoint)
in a given implementation
if and only if \emph{either} the type \emph{A} is a subtype of \emph{B}, \emph{or}
\cd{(upgraded-complex-part-type~'\emph{A})} and
\cd{(upgraded-complex-part-type~'\emph{B})} are the same type.
In the latter case \cd{(complex~\emph{A})}
and \cd{(complex~\emph{B})} in fact refer to the same specialized representation.
Therefore
\begin{lisp}
(subtypep '(complex \emph{A}) '(complex \emph{B}))
\end{lisp}
is true if and only if the results of
\cd{(upgraded-complex-part-type~'\emph{A})} and
\cd{(upgraded-complex-part-type~'\emph{B})} are the same type.

Under this interpretation
\begin{lisp}
(subtypep '(complex single-float) '(complex float))
\end{lisp}
must be true in all implementations; but
\begin{lisp}
(subtypep '(array single-float) '(array float))
\end{lisp}
is true only in implementations that do not have a specialized array representation
for \cdf{single-float} elements distinct from that for \cdf{float} elements in
general.
\end{new}
\end{defun}

\subsection{Specific Data Type Predicates Специальные предикаты типов}

The following predicates test for individual data types.

\begin{defun}[Function]
null object

\cdf{null} is true if its argument is {\emptylist},
and otherwise is false.
This is the same operation performed by the function \cdf{not};
however, \cdf{not} is normally used to invert a Boolean value,
whereas \cdf{null} is normally used to test for an empty list.  The programmer
can therefore express \emph{intent} by the choice of function name.
\begin{lisp}
(null x) \EQ\ (typep x 'null) \EQ\ (eq x '{\emptylist})
\end{lisp}
\end{defun}

\begin{defun}[Function]
symbolp object

\cdf{symbolp} is true if its argument is a symbol,
and otherwise is false.
\begin{lisp}
(symbolp x) \EQ\ (typep x 'symbol)
\end{lisp}

\beforenoterule
\begin{incompatibility}
The Interlisp equivalent of \cdf{symbolp} is
called \cdf{litatom}.
\end{incompatibility}
\afternoterule
\end{defun}

\begin{defun}[Function]
atom object

The predicate \cdf{atom} is true if its argument is not a cons,
and otherwise is false.
Note that \cd{(atom '{\emptylist})} is true, because {\emptylist}$\;\equiv\;${\nil}.
\begin{lisp}
(atom x) \EQ\ (typep x 'atom) \EQ\ (not (typep x 'cons))
\end{lisp}

\beforenoterule
\begin{incompatibility}
In some Lisp dialects, notably Interlisp,
only symbols and numbers are considered to be atoms; arrays
and strings are considered to be neither atoms nor lists (conses).
\end{incompatibility}
\afternoterule
\end{defun}

\begin{defun}[Function]
consp object

The predicate \cdf{consp} is true if its argument is a cons,
and otherwise is false.
Note that the empty list is not a cons, so
\cd{(consp '{\emptylist})} \EQ\ \cd{(consp '{\nil})} \EV\ {\nil}.
\begin{lisp}
(consp x) \EQ\ (typep x 'cons) \EQ\ (not (typep x 'atom))
\end{lisp}

\beforenoterule
\begin{incompatibility}
Some Lisp implementations call this function
\cdf{pairp} or \cdf{listp}.  The name \cdf{pairp} was rejected for Common Lisp
because it emphasizes too strongly the dotted-pair notion rather than the
usual usage of conses in lists.  On the other hand, \cdf{listp} too strongly
implies that the cons is in fact part of a list, which after all it might
not be; moreover, {\emptylist} is a list, though not a cons.
The name \cdf{consp} seems to be the appropriate compromise.
\end{incompatibility}
\afternoterule
\end{defun}

\begin{defun}[Function]
listp object

\cdf{listp} is true if its argument is a cons or the empty list {\emptylist},
and otherwise is false.  It does not check for whether the list
is a ``true list'' (one terminated by {\nil}) or a ``dotted list''
(one terminated by a non-null atom).
\begin{lisp}
(listp x) \EQ\ (typep x 'list) \EQ\ (typep x '(or cons null))
\end{lisp}
\end{defun}

\begin{defun}[Function]
numberp object

\cdf{numberp} is true if its argument is any kind of number,
and otherwise is false.
\begin{lisp}
(numberp x) \EQ\ (typep x 'number)
\end{lisp}
\end{defun}

\begin{defun}[Function]
integerp object

\cdf{integerp} is true if its argument is an integer, and otherwise
is false.
\begin{lisp}
(integerp x) \EQ\ (typep x 'integer)
\end{lisp}

\beforenoterule
\begin{incompatibility}
In MacLisp this is called \cdf{fixp}.
Users have been confused as to whether this meant \cdf{integerp}
or \cdf{fixnump}, and so the name \cdf{integerp} has been adopted here.
\end{incompatibility}
\afternoterule
\end{defun}

\begin{defun}[Function]
rationalp object

\cdf{rationalp} is true if its argument is a rational number (a ratio or
an integer), and otherwise is false.
\begin{lisp}
(rationalp x) \EQ\ (typep x 'rational)
\end{lisp}
\end{defun}

\begin{defun}[Function]
floatp object

\cdf{floatp} is true if its argument is a floating-point number,
and otherwise is false.
\begin{lisp}
(floatp x) \EQ\ (typep x 'float)
\end{lisp}
\end{defun}


\begin{newer}
\begin{defun}[Function]
realp object

X3J13 voted in March 1989 \issue{REAL-NUMBER-TYPE} to add the function \cdf{realp}.
\cdf{realp} is true if its argument is a real number,
and otherwise is false.
\begin{lisp}
(realp x) \EQ\ (typep x 'real)
\end{lisp}
\end{defun}
\end{newer}

\begin{defun}[Function]
complexp object

\cdf{complexp} is true if its argument is a complex number,
and otherwise is false.
\begin{lisp}
(complexp x) \EQ\ (typep x 'complex)
\end{lisp}
\end{defun}

\begin{defun}[Function]
characterp object

\cdf{characterp} is true if its argument is a character,
and otherwise is false.
\begin{lisp}
(characterp x) \EQ\ (typep x 'character)
\end{lisp}
\end{defun}

\begin{defun}[Function]
stringp object

\cdf{stringp} is true if its argument is a string,
and otherwise is false.
\begin{lisp}
(stringp x) \EQ\ (typep x 'string)
\end{lisp}
\end{defun}

\begin{defun}[Function]
bit-vector-p object

\cdf{bit-vector-p} is true if its argument is a bit-vector,
and otherwise is false.
\begin{lisp}
(bit-vector-p x) \EQ\ (typep x 'bit-vector)
\end{lisp}
\end{defun}

\begin{defun}[Function]
vectorp object

\cdf{vectorp} is true if its argument is a vector,
and otherwise is false.
\begin{lisp}
(vectorp x) \EQ\ (typep x 'vector)
\end{lisp}
\end{defun}

\begin{defun}[Function]
simple-vector-p object

\cdf{vectorp} is true if its argument is a simple general vector,
and otherwise is false.
\begin{lisp}
(simple-vector-p x) \EQ\ (typep x 'simple-vector)
\end{lisp}
\end{defun}

\begin{defun}[Function]
simple-string-p object

\cdf{simple-string-p} is true if its argument is a simple string,
and otherwise is false.
\begin{lisp}
(simple-string-p x) \EQ\ (typep x 'simple-string)
\end{lisp}
\end{defun}

\begin{defun}[Function]
simple-bit-vector-p object

\cdf{simple-bit-vector-p} is true if its argument is a simple bit-vector,
and otherwise is false.
\begin{lisp}
(simple-bit-vector-p x) \EQ\ (typep x 'simple-bit-vector)
\end{lisp}
\end{defun}

\begin{defun}[Function]
arrayp object

\cdf{arrayp} is true if its argument is an array,
and otherwise is false.
\begin{lisp}
(arrayp x) \EQ\ (typep x 'array)
\end{lisp}
\end{defun}

\begin{defun}[Function]
packagep object

\cdf{packagep} is true if its argument is a package,
and otherwise is false.
\begin{lisp}
(packagep x) \EQ\ (typep x 'package)
\end{lisp}
\end{defun}

\begin{defun}[Function]
functionp object

\begin{obsolete}
\cdf{functionp} is true if its argument is suitable for applying
to arguments, using for example the \cdf{funcall} or \cdf{apply} function.
Otherwise \cdf{functionp} is false.

\cdf{functionp} is always true of symbols, lists whose \emph{car}
is the symbol \cdf{lambda}, any value returned by the \cdf{function}
special form, and any values returned by the function \cdf{compile}
when the first argument is {\nil}.
\end{obsolete}
\begin{newer}
X3J13 voted in June 1988 \issue{FUNCTION-TYPE}
to define
\begin{lisp}
(functionp x) \EQ\ (typep x 'function)
\end{lisp}
Because the vote also specifies that types \cdf{cons} and \cdf{symbol} are disjoint
from the type \cdf{function}, this is an incompatible change;
now \cdf{functionp} is in fact always false of symbols and lists.
\end{newer}
\end{defun}

\begin{defun}[Function]
compiled-function-p object

\cdf{compiled-function-p} is true if its argument is any compiled code object,
and otherwise is false.
\begin{lisp}
(compiled-function-p x) \EQ\ (typep x 'compiled-function)
\end{lisp}
\end{defun}

\begin{obsolete}
\begin{defun}[Function]
commonp object

\cdf{commonp} is true if its argument is any standard Common Lisp data type,
and otherwise is false.
\begin{lisp}
(commonp x) \EQ\ (typep x 'common)
\end{lisp}

\cdf{commonp} является истиной, если аргумент какой-либо стандартный тип данных
Common Lisp'а, иначе является ложью.
\begin{lisp}
(commonp x) \EQ\ (typep x 'common)
\end{lisp}
\end{defun}
\end{obsolete}

\begin{newer}
X3J13 voted in March 1989
\issue{COMMON-TYPE}
to remove the predicate \cdf{commonp} (and the type \cdf{common}) from the
language.
\end{newer}

\medskip

See also \cdf{standard-char-p}, \cdf{string-char-p},
\cdf{streamp}, \cdf{random-state-p},
\cdf{readtablep},
\cdf{hash-table-p}, and \cdf{pathnamep}.

\section{Equality Predicates}

Common Lisp provides a spectrum of predicates for testing for equality of
two objects: \cdf{eq} (the most specific), \cdf{eql}, \cdf{equal}, and \cdf{equalp}
(the most general).  \cdf{eq} and \cdf{equal} have the meanings traditional
in Lisp.  \cdf{eql} was added because it is frequently needed, and
\cdf{equalp} was added primarily in order to have a version of \cdf{equal}
that would ignore type differences when comparing numbers
and case differences when comparing characters.
If two objects satisfy any one of these equality predicates,
then they also satisfy all those that are more general.

\begin{defun}[Function]
eq x y

\cd{(eq \emph{x} \emph{y})} is true
if and only if \emph{x} and \emph{y} are the same identical object.
(Implementationally, \emph{x} and \emph{y} are usually
\cdf{eq} if and only if they address the same identical memory location.)

It should be noted that things that print the same are not necessarily \cdf{eq}
to each other.  Symbols with the same print name usually are \cdf{eq} to
each other because of the use of the \cdf{intern} function.
However, numbers with the same value
need not be \cdf{eq}, and two similar lists are usually not \cdf{eq}.
For example:
\begin{lisp}
(eq 'a 'b) \textrm{is false.} \\
(eq 'a 'a) \textrm{is true.} \\
(eq 3 3) \textrm{might be true or false, depending on the implementation.} \\
(eq 3 3.0) \textrm{is false.} \\
(eq 3.0 3.0) \textrm{might be true or false, depending on the implementation.} \\
(eq \#c(3 -4) \#c(3 -4)) \\
~~\textrm{might be true or false, depending on the implementation.} \\
(eq \#c(3 -4.0) \#c(3 -4)) \textrm{is false.} \\
(eq (cons 'a 'b) (cons 'a 'c)) \textrm{is false.} \\
(eq (cons 'a 'b) (cons 'a 'b)) \textrm{is false.} \\
(eq '(a . b) '(a . b)) \textrm{might be true or false.} \\
(progn (setq x (cons 'a 'b)) (eq x x)) \textrm{is true.} \\
(progn (setq x '(a . b)) (eq x x)) \textrm{is true.} \\
(eq \#{\Xbackslash}A \#{\Xbackslash}A) \textrm{might be true or false, depending on the implementation.} \\
(eq "Foo" "Foo") \textrm{might be true or false.} \\
(eq "Foo" (copy-seq "Foo")) \textrm{is false.} \\
(eq "FOO" "foo") \textrm{is false.}
\end{lisp}

In Common Lisp, unlike some other Lisp dialects, the implementation
is permitted to make ``copies'' of
characters and numbers at any time.  (This permission is granted
because it allows tremendous performance improvements in many
common situations.)  The net effect is that
Common Lisp makes no guarantee that \cdf{eq} will be true even when both
its arguments are ``the same thing'' if that thing is a character or number.
For example:
\begin{lisp}
(let ((x 5)) (eq x x)) \textrm{might be true or false.}
\end{lisp}

The predicate \cdf{eql} is the same as \cdf{eq}, except that if the
arguments are characters or numbers of the same type then their
values are compared.  Thus \cdf{eql} tells whether two objects
are \emph{conceptually} the same, whereas \cdf{eq} tells whether two
objects are \emph{implementationally} identical.  It is for this reason
that \cdf{eql}, not \cdf{eq}, is the default comparison predicate
for the sequence functions defined in chapter~\ref{KSEQUE}.

\beforenoterule
\begin{implementation}
\cdf{eq} simply compares the two given pointers,
so any kind of object that is represented in an ``immediate'' fashion
will indeed have like-valued instances satisfy \cdf{eq}.
In some implementations, for example,
fixnums and characters happen to ``work.''
However, no program should depend on this, as other implementations
of Common Lisp might not use an immediate representation for these data types.
\end{implementation}
\afternoterule

\begin{obsolete}
An additional problem with \cdf{eq} is that the implementation is permitted
to ``collapse'' constants (or portions thereof)
appearing in code to be compiled if they are
\cdf{equal}.  An object is considered to be a constant in code to be compiled
if it is a self-evaluating form or is contained in a \cdf{quote} form.
This is why \cd{(eq "Foo" "Foo")} might be true or false; in interpreted
code it would normally be false, because reading in the
form \cd{(eq "Foo" "Foo")} would construct distinct strings for the two
arguments to \cdf{eq}, but the compiler might choose to use the same
identical string or two distinct copies as the two arguments in the
call to \cdf{eq}.  Similarly, \cd{(eq '(a . b) '(a . b))} might be true
or false, depending on whether the constant conses appearing in the
\cdf{quote} forms were collapsed by the compiler.  However,
\cd{(eq (cons 'a 'b) (cons 'a 'b))} is always false, because every distinct
call to the \cdf{cons} function necessarily produces a new and distinct cons.
\end{obsolete}

\begin{newer}
X3J13 voted in March 1989 \issue{QUOTE-SEMANTICS} to clarify that
\cdf{eval} and \cdf{compile} are not permitted either to copy or
to coalesce (``collapse'') constants (see \cdf{eq})
appearing in the code they process; the resulting
program behavior must refer to objects that are \cdf{eql} to the
corresponding objects in the source code.
Only the \cdf{compile-file}/\cdf{load} process is permitted
to copy or coalesce constants (see section~\ref{COMPILER-SECTION}).
\end{newer}
\end{defun}

\begin{defun}[Function]
eql x y

The \cdf{eql} predicate is true if its arguments are \cdf{eq},
or if they are numbers of the same type with the same value,
or if they are character objects
that represent the same character.
For example:
\begin{lisp}
(eql 'a 'b) \textrm{is false.} \\
(eql 'a 'a) \textrm{is true.} \\
(eql 3 3) \textrm{is true.} \\
(eql 3 3.0) \textrm{is false.} \\
(eql 3.0 3.0) \textrm{is true.} \\
(eql \#c(3 -4) \#c(3 -4)) \textrm{is true.} \\
(eql \#c(3 -4.0) \#c(3 -4)) \textrm{is false.} \\
(eql (cons 'a 'b) (cons 'a 'c)) \textrm{is false.} \\
(eql (cons 'a 'b) (cons 'a 'b)) \textrm{is false.} \\
(eql '(a . b) '(a . b)) \textrm{might be true or false.} \\
(progn (setq x (cons 'a 'b)) (eql x x)) \textrm{is true.} \\
(progn (setq x '(a . b)) (eql x x)) \textrm{is true.} \\
(eql \#{\Xbackslash}A \#{\Xbackslash}A) \textrm{is true.} \\
(eql "Foo" "Foo") \textrm{might be true or false.} \\
(eql "Foo" (copy-seq "Foo")) \textrm{is false.} \\
(eql "FOO" "foo") \textrm{is false.}
\end{lisp}

Normally \cd{(eql 1.0s0 1.0d0)} would be false, under the assumption
that \cd{1.0s0} and \cd{1.0d0} are of distinct data types.
However, implementations that do not provide four distinct floating-point
formats are permitted to ``collapse'' the four formats into some
smaller number of them; in such an implementation \cd{(eql 1.0s0 1.0d0)}
might be true.  The predicate \cdf{=} will compare
the values of two numbers even if the numbers are of different types.

If an implementation supports positive and negative zeros as distinct
values (as in the IEEE proposed standard floating-point format),
then \cd{(eql 0.0 -0.0)} will be false.  Otherwise, when the syntax
\cd{-0.0} is read it will be interpreted as the value \cd{0.0},
and so \cd{(eql 0.0 -0.0)} will be true.  The predicate \cdf{=}
differs from \cdf{eql} in that \cd{(= 0.0 -0.0)} will always be true,
because \cdf{=} compares the mathematical values of its operands,
whereas \cdf{eql} compares the representational values, so to speak.

Two complex numbers are considered to be \cdf{eql}
if their real parts are \cdf{eql} and their imaginary parts are \cdf{eql}.
For example, \cd{(eql \#C(4 5) \#C(4 5))} is true and
\cd{(eql \#C(4 5) \#C(4.0 5.0))} is false.
Note that while \cd{(eql \#C(5.0 0.0) 5.0)} is false,
\cd{(eql \#C(5 0) 5)} is true.
In the case of \cd{(eql \#C(5.0 0.0) 5.0)} the
two arguments are of different types
and so cannot satisfy \cdf{eql}; that's all there is to it.
In the case of \cd{(eql \#C(5 0) 5)}, however,
\cd{\#C(5 0)} is not a complex number but
is always automatically reduced by the rule of complex
canonicalization to the integer \cd{5},
just as the apparent ratio \cd{20/4} is always simplified to \cd{5}.

The case of \cd{(eql "Foo" "Foo")} is discussed above in the description
of \cdf{eq}.  While \cdf{eql} compares the values of numbers and
characters, it does not compare the contents of strings.  To compare
the characters of two strings, one should use \cdf{equal}, \cdf{equalp},
\cdf{string=}, or \cdf{string-equal}.

\beforenoterule
\begin{incompatibility}
The Common Lisp function \cdf{eql} is similar to the
Interlisp function \cdf{eqp}.  However, \cdf{eql} considers \cd{3} and
\cd{3.0} to be different, whereas \cdf{eqp} considers them to be the same;
\cdf{eqp} behaves like the Common Lisp \cdf{=} function, not like \cdf{eql},
when both arguments are numbers.
\end{incompatibility}
\afternoterule
\end{defun}

\begin{defun}[Function]
equal x y

The \cdf{equal} predicate is true if its arguments are structurally similar
(isomorphic) objects.  A rough rule of thumb is that two objects
are \cdf{equal} if and only if their printed representations are the same.

Numbers and characters are compared as for \cdf{eql}.
Symbols are compared as for \cdf{eq}.  This method
of comparing symbols can violate the rule
of thumb for \cdf{equal} and printed representations,
but only in the infrequently occurring case of two distinct
symbols with the same print name.

Certain objects that have components are \cdf{equal} if they are of the same
type and corresponding components are \cdf{equal}.
This test is implemented in a recursive manner and may fail to
terminate for circular structures.

For conses, \cdf{equal} is defined recursively as
the two \emph{car}'s being \cdf{equal} and the two \emph{cdr}'s being
\cdf{equal}.

Two arrays are \cdf{equal} only if they are \cdf{eq},
with one exception:
strings and bit-vectors are compared element-by-element.
If either argument has a fill pointer, the fill pointer limits
the number of elements examined by \cdf{equal}.
Uppercase and lowercase letters in strings are considered by
\cdf{equal} to be distinct.  (In contrast, \cdf{equalp} ignores
case distinctions in strings.)

Two pathname objects are \cdf{equal} if and only if
all the corresponding components
(host, device, and so on) are equivalent.  (Whether or not
uppercase and lowercase letters are considered equivalent
in strings appearing in components depends on the file
name conventions of the file system.)  Pathnames
that are \cdf{equal} should be functionally equivalent.

\begin{new}
X3J13 voted in June 1989
\issue{EQUAL-STRUCTURE}
to clarify that \cdf{equal} never recursively
descends any structure or data type other than the ones explicitly
described above: conses, bit-vectors, strings, and pathnames.
Numbers and characters are compared as if by \cdf{eql}, and all other
data objects are compared as if by \cdf{eq}.
\end{new}

\begin{lisp}
(equal 'a 'b) \textrm{is false.} \\
(equal 'a 'a) \textrm{is true.} \\
(equal 3 3) \textrm{is true.} \\
(equal 3 3.0) \textrm{is false.} \\
(equal 3.0 3.0) \textrm{is true.} \\
(equal \#c(3 -4) \#c(3 -4)) \textrm{is true.} \\
(equal \#c(3 -4.0) \#c(3 -4)) \textrm{is false.} \\
(equal (cons 'a 'b) (cons 'a 'c)) \textrm{is false.} \\
(equal (cons 'a 'b) (cons 'a 'b)) \textrm{is true.} \\
(equal '(a . b) '(a . b)) \textrm{is true.} \\
(progn (setq x (cons 'a 'b)) (equal x x)) \textrm{is true.} \\
(progn (setq x '(a . b)) (equal x x)) \textrm{is true.} \\
(equal \#{\Xbackslash}A \#{\Xbackslash}A) \textrm{is true.} \\
(equal "Foo" "Foo") \textrm{is true.} \\
(equal "Foo" (copy-seq "Foo")) \textrm{is true.} \\
(equal "FOO" "foo") \textrm{is false.}
\end{lisp}
To compare a tree of conses using \cdf{eql}
(or any other desired predicate) on the leaves, use \cdf{tree-equal}.

\end{defun}

\begin{defun}[Function]
equalp x y

Two objects are \cdf{equalp} if they are \cdf{equal};
if they are characters and satisfy \cdf{char-equal},
which ignores alphabetic case and certain other attributes of characters;
if they are numbers and have the same numerical value,
even if they are of different types;
or if they have components that are all \cdf{equalp}.

Objects that have components are \cdf{equalp} if they are of the same
type and corresponding components are \cdf{equalp}.
This test is implemented in a recursive manner and may fail to
terminate for circular structures.
For conses, \cdf{equalp} is defined recursively as
the two \emph{car}'s being \cdf{equalp} and the two \emph{cdr}'s being
\cdf{equalp}.

Two arrays are \cdf{equalp} if and only if they have the same
number of dimensions, the dimensions match,
and the corresponding components are \cdf{equalp}.
The specializations need not match; for example,
a string and a general array that happens to contain the same characters
will be \cdf{equalp} (though definitely not \cdf{equal}).
If either argument has a fill pointer, the fill pointer limits
the number of elements examined by \cdf{equalp}.
Because \cdf{equalp} performs element-by-element comparisons
of strings and ignores the alphabetic case of characters,
case distinctions are therefore also ignored when \cdf{equalp} compares
strings.

Two symbols can be \cdf{equalp} only if they are \cdf{eq}, that is, the same
identical object.

\begin{new}
X3J13 voted in June 1989
\issue{EQUAL-STRUCTURE}
to specify that \cdf{equalp} compares components
of hash tables (see below), and to
clarify that otherwise \cdf{equalp} never recursively
descends any structure or data type other than the ones explicitly
described above: conses, arrays (including bit-vectors and strings), and pathnames.
Numbers are compared for numerical equality (see \cdf{=}),
characters are compared as if by \cdf{char-equal}, and all other
data objects are compared as if by \cdf{eq}.

Two hash tables are considered the same by \cdf{equalp} if and only if
they satisfy a four-part test:
\begin{itemize}
\item They must be
of the same kind; that is, equivalent \cd{:test} arguments were given to
\cdf{make-hash-table} when the two hash tables were created.

\item They must have the same number of entries (see \cdf{hash-table-count}).

\item For every entry (\emph{key1}, \emph{value1\/}) in one hash table
there must be a corresponding entry (\emph{key2}, \emph{value2\/}) in the
other, such that \emph{key1} and \emph{key2} are considered to be the same
by the \cd{:test} function associated with the hash tables.

\item For every entry (\emph{key1}, \emph{value1\/}) in one hash table
and its corresponding entry (\emph{key2}, \emph{value2\/}) in the
other, such that \emph{key1} and \emph{key2} are the same,
\cdf{equalp} must be true of \emph{value1} and \emph{value2}.
\end{itemize}
The four parts of this test are carried out in the order shown, and
if some part of the test fails, \cdf{equalp} returns \cdf{nil} and
the other parts of the test are not attempted.

If \cdf{equalp} must compare two structures and the \cdf{defstruct}
definition for one used the \cd{:type} option and the other did not,
then \cdf{equalp} returns \cdf{nil}.

If \cdf{equalp} must compare two structures and neither \cdf{defstruct}
definition used the \cd{:type} option,
then \cdf{equalp} returns \cdf{t} if and only if the structures have the
same type (that is, the same \cdf{defstruct} name) and the values
of all corresponding slots (slots having the same name) are \cdf{equalp}.

As part of the X3J13 discussion of this issue
the following observations were made.
    Object equality is not a concept for which there is a uniquely
    determined correct algorithm. The appropriateness of an equality
    predicate can be judged only in the context of the needs of some
    particular program. Although these functions take any type of
    argument and their names sound very generic, \cdf{equal} and \cdf{equalp} are
    not appropriate for every application. Any decision to use or not
    use them should be determined by what they are documented to do
    rather than by any abstract characterization of their function. If
    neither \cdf{equal} nor \cdf{equalp} is found to be appropriate in a particular
    situation, programmers are encouraged to create another operator
    that is appropriate rather than blame \cdf{equal} or \cdf{equalp} for ``doing
    the wrong thing.''
\end{new}

\begin{new}
Note that one consequence
of the vote to change the rules of
floating-point contagion
\issue{CONTAGION-ON-NUMERICAL-COMPARISONS}
(described in section~\ref{PRECISION-CONTAGION-COERCION-SECTION})
is to make \cdf{equalp}
a true equivalence relation on numbers.
\end{new}

\begin{lisp}
(equalp 'a 'b) \textrm{is false.} \\
(equalp 'a 'a) \textrm{is true.} \\
(equalp 3 3) \textrm{is true.} \\
(equalp 3 3.0) \textrm{is true.} \\
(equalp 3.0 3.0) \textrm{is true.} \\
(equalp \#c(3 -4) \#c(3 -4)) \textrm{is true.} \\
(equalp \#c(3 -4.0) \#c(3 -4)) \textrm{is true.} \\
(equalp (cons 'a 'b) (cons 'a 'c)) \textrm{is false.} \\
(equalp (cons 'a 'b) (cons 'a 'b)) \textrm{is true.} \\
(equalp '(a . b) '(a . b)) \textrm{is true.} \\
(progn (setq x (cons 'a 'b)) (equalp x x)) \textrm{is true.} \\
(progn (setq x '(a . b)) (equalp x x)) \textrm{is true.} \\
(equalp \#{\Xbackslash}A \#{\Xbackslash}A) \textrm{is true.} \\
(equalp "Foo" "Foo") \textrm{is true.} \\
(equalp "Foo" (copy-seq "Foo")) \textrm{is true.} \\
(equalp "FOO" "foo") \textrm{is true.}
\end{lisp}

\end{defun}

\section{Logical operators}

Common Lisp provides three operators on Boolean values: \cdf{and}, \cdf{or},
and \cdf{not}.  Of these, \cdf{and} and \cdf{or}
are also control structures because their arguments are evaluated
conditionally.
The function \cdf{not} necessarily examines its single argument, and so
is a simple function.

\begin{defun}[Function]
not x

\cdf{not} returns {\true} if \emph{x} is {\false}, and otherwise returns {\false}.
It therefore inverts its argument considered as a Boolean value.

\cdf{null} is the same as \cdf{not}; both functions are included for the sake
of clarity.  As a matter of style,
it is customary to use \cdf{null} to check whether something is the empty list
and to use \cdf{not} to invert the sense of a logical value.
\end{defun}

\begin{defmac}
and {\,form}*

\cd{(and \emph{form1} \emph{form2} ... )} evaluates each \emph{form}, one at a time,
from left to right.  If any \emph{form} evaluates to {\false}, the value {\nil}
is immediately returned without evaluating the remaining
\emph{form\/}s.  If every \emph{form} but the last evaluates to a non-{\false} value,
\cdf{and} returns whatever the last \emph{form} returns.
Therefore in general \cdf{and} can be used both for logical operations,
where {\false} stands for \emph{false} and non-{\false} values stand for \emph{true},
and as a conditional expression.
An example follows.
\begin{lisp}
(if (and (>= n 0) \\
~~~~~~~~~(< n (length a-simple-vector)) \\
~~~~~~~~~(eq (elt a-simple-vector n) 'foo)) \\
~~~~(princ "Foo!"))
\end{lisp}
The above expression prints \cd{Foo!} if element \cd{n} of \cd{a-simple-vector}
is the symbol \cd{foo}, provided also that \cdf{n} is indeed a valid index
for \cdf{a-simple-vector}.  Because \cdf{and} guarantees left-to-right testing
of its parts, \cdf{elt} is not called if \cd{n} is out of range.

To put it another way,
the \cdf{and} special form does \emph{short-circuit} Boolean evaluation,
like the \textbf{and then} operator in Ada
and what in some Pascal-like languages is called \textbf{cand} (for ``conditional
and''); the Lisp \cdf{and} special form is
unlike the Pascal or Ada \textbf{and} operator,
which always evaluates both arguments.

In the previous example writing
\begin{lisp}
(and (>= n 0) \\
~~~~~(< n (length a-simple-vector)) \\
~~~~~(eq (elt a-simple-vector n) 'foo) \\
~~~~~(princ "Foo!"))
\end{lisp}
would accomplish the same thing.  The difference is purely stylistic.
Some programmers never use expressions containing side effects
within \cdf{and}, preferring to use \cdf{if} or \cdf{when} for that purpose.

From the general definition, one can deduce that
\cd{(and \emph{x})} \EQ\ \emph{x}.  Also,
\cd{(and)} evaluates to {\true}, which is an identity for this operation.

One can define \cdf{and} in terms of \cdf{cond} in this way:
\begin{lisp}
(and \emph{x} \emph{y} \emph{z} ... \emph{w}) \EQ\ (cond \=((not \emph{x}) {\false}) \\
\>((not \emph{y}) {\false}) \\
\>((not \emph{z}) {\false}) \\
\>$\ldots$ \\
\>({\true} \emph{w}))
\end{lisp}

See \cdf{if} and \cdf{when}, which are sometimes stylistically
more appropriate than \cdf{and} for conditional purposes.
If it is necessary to test whether a predicate is true
of all elements of a list or vector (element 0 \emph{and} element 1 \emph{and}
element 2 \emph{and} $\ldots$), then the function \cdf{every} may be useful.
\end{defmac}

\begin{defmac}
or {\,form}*

\cd{(or \emph{form1} \emph{form2} ... )} evaluates each \emph{form}, one at a time,
from left to right.  If any \emph{form} other than the last
evaluates to something other than {\false},
\cdf{or}
immediately returns that non-{\false} value without evaluating the remaining
\emph{form\/}s.  If every \emph{form} but the last evaluates to {\false},
\cdf{or} returns whatever evaluation of the last of the \emph{form\/}s returns.
Therefore in general \cdf{or} can be used both for logical operations,
where {\false} stands for \emph{false} and non-{\false} values stand for \emph{true},
and as a conditional expression.

To put it another way,
the \cdf{or} special form does \emph{short-circuit} Boolean evaluation,
like the \textbf{or else} operator in Ada
and what in some Pascal-like languages is called \textbf{cor} (for ``conditional
or''); the Lisp \cdf{or} special form is
unlike the Pascal or Ada \textbf{or} operator,
which always evaluates both arguments.

From the general definition, one can deduce that
\cd{(or \emph{x})} \EQ\ \emph{x}.  Also,
\cd{(or)} evaluates to {\nil}, which is the identity for this operation.

One can define \cdf{or} in terms of \cdf{cond} in this way:
\begin{lisp}
(or \emph{x} \emph{y} \emph{z} ... \emph{w}) \EQ\ (cond (\emph{x}) (\emph{y}) (\emph{z}) ... ({\true} \emph{w}))
\end{lisp}

See \cdf{if} and \cdf{unless}, which are sometimes
stylistically more appropriate than \cdf{or} for conditional purposes.
If it is necessary to test whether a predicate is true of
one or more elements of a list or vector (element 0 \emph{or} element 1 \emph{or}
element 2 \emph{or} $\ldots$), then the function \cdf{some} may be useful.
\end{defmac}

%RUSSIAN
\else

\chapter{Предикаты}
\label{PREDS}

\emph{Предикат} --- это функция, которая проверяет некоторое условие
переданное в аргументах и возвращает {\false}, если условие ложное, или
не-{\false}, если условие истинное. Можно рассматривать, что предикат
производит булево значение, где \cd{\false} обозначает \emph{ложь} и все
остальное --- \emph{истину}. Условные управляющие структуры, такие как
\cdf{cond}, \cdf{if}, \cdf{when} и \cdf{unless} осуществляют проверку таких
булевых значений. 
Мы говорим, что предикат \emph{истинен}, когда он возвращает не-{\false}
значение, и \emph{ложен}, когда он возвращает {\false}, то есть он истинен
или ложен в зависимости от того, истинно или ложно проверяемое условие.

По соглашению, имена предикатов обычно заканчиваются на букву \cd{p} (которая
обозначает <<предикат (predicate)>>).
Common Lisp использует единое соглашение для использования дефисов в именах
предикатов. Если имя предиката создано с помощью добавления \cd{p} к уже
существующему имени, такому как имя типа данных, тогда дефис помещается перед
последним \cd{p} тогда и только тогда, когда в исходном имени были
дефисы. Например, \cd{number} становится \cd{numberp}, но \cd{standard-char}
становится \cd{standard-char-p}.
С другой стороны, если имя предиката сформировано добавлением префиксного
спецификатора в начало существующего имени предиката, то два имени соединяются с
помощью дефиса, и наличие или отсутствие перед завершающим \cd{p} не
изменяется. Например, предикат \cd{string-lessp} не содержит дефиса перед
\cd{p}, потому что это строковая версия \cd{lessp}. Имя \cd{string-less-p} было бы
некорректно указывающим на то, что это предикат проверяющий тип объекта
называемого \cd{string-less}, а имя \cd{stringlessp} имело бы смысл того, что
проверяет отсутствие строк в чем-либо. 

Управляющие структуры, которые проверяют булевы значения, проверяют только
является или нет значение ложью ({\false}). Любое
другое значение рассматривается как истинное. Часто предикат будет возвращать
{\false}, в случае <<неудачи>> и некоторое \emph{полезное} значение в случае
<<успеха>>. Такие функции могут использоваться не только для проверки, но и
также для использования полезного значения, получаемого в случае
успеха. Например \cdf{member}.

Если лучшего, чем не-{\nil} значения, в целях указания успеха не оказалось, по
соглашению в качестве <<стандартного>> значения истины используется символ \cdf{t}.

\section{Логические значения}

Имена \cdf{nil} и \cdf{t} в Common Lisp'е являются константами. Несмотря на то,
что они являются обычными символами, и могут использоваться в качестве переменных
при вычислениях, их значения не могут быть изменены. Смотрите \cd{defconstant}.

\begin{defun}[Константа]
nil

Значение {\nil} всегда {\nil}. Этот объект обозначает логическую ложь, а также
пустой список. Он также может быть записан, как \cd{()}.
\end{defun}

\begin{defun}[Константа]
t

Значение \cdf{t} всегда\cdf{t}.
\end{defun}

\section{Предикаты типов данных}

Возможно наиболее важными предикатами в Lisp'е это предикаты, которые различают
типы данных. То есть позволяют узнать принадлежит ли заданный объект данному 
типу. Также предикаты могут сравнивать два спецификатора типов.

\subsection{Базовые предикаты типов}

Если тип данных рассматривать, как множество все объектов, принадлежащих этому
типу, тогда функция \cdf{typep} проверяет принадлежность множеству, тогда как
\cd{subtypep} --- принадлежность подмножеству.

\begin{defun}[Функция]
typep object type

\cdf{typep} является предикатом, который истинен, если объект \emph{object}
принадлежит типу \emph{type}, и ложен в противном случае.
Следует отметить, что объект может принадлежать нескольким типам, так как один
тип может включать другой. \emph{type} может быть любым спецификатором типа,
описанным в главе~\ref{DTSPEC}, за исключением того, что он не может быть или
включать список спецификатор типа, у которого первый элемент равен
\cdf{function} или \cdf{values}.
Спецификатор формы \cd{(satisfies \emph{fn})} обрабатывается просто как применение
функции \emph{fn} к объекту \emph{object} (смотрите \cdf{funcall}). Объект
\emph{object} принадлежит заданному типу, если результат не равен {\false}.
\end{defun}

\begin{defun}[Функция]
subtypep type1 type2

Аргументы должны быть спецификаторами типов, но только теми, которые могут
использоваться и для \cdf{typep}.
Два спецификатора типа сравниваются. Данный предикат истинен, если
тип \emph{type1} точно является подтипом типа \emph{type2}, иначе предикат ложен.
Если результат {\false}, тогда тип \emph{type1} может быть, а может и не быть
подтипом типа \emph{type2} (иногда это невозможно определить, особенно когда
используется тип \cdf{satisfies}).
Второе возвращаемое значение указывает на точность результата. Если оно является
истиной, значит первое значение указывает на точную принадлежность типов. Таким
образом возможны следующие комбинации результатов:
\begin{tabbing}
~~~~~~~~\=~~~~~~~~\=\kill
{\true}\>{\true}\>\emph{type1} точно является подтипом \emph{type2} \\
{\false}\>{\true}\>\emph{type1} точно не является подтипом \emph{type2} \\
{\false}\>{\false}\>\cdf{subtypep} не может определить отношение
\end{tabbing}
\end{defun}

\subsection{Специальные предикаты типов}

Следующие предикаты осуществляют проверку определённых типов данных.

\begin{defun}[Функция]
null object

\cdf{null} истинен, если аргумент является {\emptylist}, иначе является
ложью. Похожая операция производится \cdf{not}, однако \cdf{not} используется для
отрицания булевых значение, тогда как \cdf{null} используется для проверки
того, пустой ли список. Таким образом программист может выразить свои намерения, 
выбрав нужное имя функции.
\begin{lisp}
(null x) \EQ\ (typep x 'null) \EQ\ (eq x '{\emptylist})
\end{lisp}
\end{defun}

\begin{defun}[Функция]
symbolp object

\cdf{symbolp} истинен, если её аргумент является символом, в противном
случае ложен.
\begin{lisp}
(symbolp x) \EQ\ (typep x 'symbol)
\end{lisp}
\end{defun}

\begin{defun}[Функция]
atom object

Предикат \cdf{atom} истинен, если аргумент не является cons-ячейкой, в
противном случае ложен.
Следует отметить \cd{(atom '{\emptylist})} являет истиной, потому что
{\emptylist}$\;\equiv\;${\nil}.
\begin{lisp}
(atom x) \EQ\ (typep x 'atom) \EQ\ (not (typep x 'cons))
\end{lisp}
\end{defun}

\begin{defun}[Функция]
consp object

Предикат \cdf{consp} истинен, если его аргумент является cons-ячейкой,
в противном случае ложен.
Следует отметить, пустой список не является cons-ячейкой, так 
\cd{(consp '{\emptylist})} \EQ\ \cd{(consp '{\nil})} \EV\ {\nil}. 
\begin{lisp}
(consp x) \EQ\ (typep x 'cons) \EQ\ (not (typep x 'atom))
\end{lisp}
\end{defun}

\begin{defun}[Функция]
listp object

\cdf{listp} истинен, если его аргумент является cons-ячейкой или пустым
списком {\emptylist}, в противном случае ложен. Она не проверяет
является ли <<список Ъ (true list)>> (завершающийся {\nil}) или <<с точкой (dotted)>>
(завершающийся не-null атомом).
\begin{lisp}
(listp x) \EQ\ (typep x 'list) \EQ\ (typep x '(or cons null))
\end{lisp}
\end{defun}

\begin{defun}[Функция]
numberp object

\cdf{numberp} истинен, если аргумент это любой вид числа, в
противном случае ложен.
\begin{lisp}
(numberp x) \EQ\ (typep x 'number)
\end{lisp}
\end{defun}

\begin{defun}[Функция]
integerp object

\emph{integerp} истинен, если аргумент целое число, в противном
случае ложен.
\begin{lisp}
(integerp x) \EQ\ (typep x 'integer)
\end{lisp}
\end{defun}

\begin{defun}[Функция]
rationalp object

\cdf{rationalp} истинен, если аргумент рациональное число (дробь или
целое), в противном случае ложен.
\begin{lisp}
(rationalp x) \EQ\ (typep x 'rational)
\end{lisp}
\end{defun}

\begin{defun}[Функция]
floatp object

\cdf{floatp} истинен, если аргумент число с плавающей точкой, в
противном случае ложен.
\begin{lisp}
(floatp x) \EQ\ (typep x 'float)
\end{lisp}
\end{defun}

\begin{defun}[Функция]
realp object

\cdf{realp} истинна, если аргумент является действительным числом,
иначе ложна.
\begin{lisp}
(realp x) \EQ\ (typep x 'real)
\end{lisp}
\end{defun}

\begin{defun}[Функция]
complexp object

\emph{complexp} истинен, если аргумент комплексное число, в противном
случае ложен.
\begin{lisp}
(complexp x) \EQ\ (typep x 'complex)
\end{lisp}
\end{defun}

\begin{defun}[Функция]
characterp object

\cdf{characterp} истинен, если аргумент строковый символ, иначе
ложен.
\begin{lisp}
(characterp x) \EQ\ (typep x 'character)
\end{lisp}
\end{defun}

\begin{defun}[Функция]
stringp object

\cdf{stringp} истинен, если аргумент строка, иначе ложен.
\begin{lisp}
(stringp x) \EQ\ (typep x 'string)
\end{lisp}
\end{defun}

\begin{defun}[Функция]
bit-vector-p object

\cdf{bit-vector-p} истинен, если аргумент битовый вектор, иначе ложен.
\begin{lisp}
(bit-vector-p x) \EQ\ (typep x 'bit-vector)
\end{lisp}
\end{defun}

\begin{defun}[Функция]
vectorp object

\cdf{vectorp} истинен, если аргумент вектор, иначе ложен.
\begin{lisp}
(vectorp x) \EQ\ (typep x 'vector)
\end{lisp}
\end{defun}

\begin{defun}[Функция]
simple-vector-p object

\cdf{vectorp} истинен, если аргумент простой общий вектор, иначе
ложен.
\begin{lisp}
(simple-vector-p x) \EQ\ (typep x 'simple-vector)
\end{lisp}
\end{defun}

\begin{defun}[Функция]
simple-string-p object

\cdf{simple-string-p} истинен, если аргумент простая строка, иначе
ложен.
\begin{lisp}
(simple-string-p x) \EQ\ (typep x 'simple-string)
\end{lisp}
\end{defun}

\begin{defun}[Функция]
simple-bit-vector-p object

\cdf{simple-bit-vector-p} истинен, если аргумент простой битовый
вектор, иначе ложен.
\begin{lisp}
(simple-bit-vector-p x) \EQ\ (typep x 'simple-bit-vector)
\end{lisp}
\end{defun}

\begin{defun}[Функция]
arrayp object

\cdf{arrayp} истинен, если аргумент массив, иначе ложен.
\begin{lisp}
(arrayp x) \EQ\ (typep x 'array)
\end{lisp}
\end{defun}

\begin{defun}[Функция]
packagep object

\cdf{packagep} истинен, если аргумент является пакетом, иначе является
ложью.
\begin{lisp}
(packagep x) \EQ\ (typep x 'package)
\end{lisp}
\end{defun}

\begin{defun}[Функция]
functionp object

\begin{newer}
X3J13 проголосовал в июне 1988 \issue{FUNCTION-TYPE}
определить
\begin{lisp}
(functionp x) \EQ\ (typep x 'function)
\end{lisp}
Так как голосование также определило, что типы \cdf{cons} и \cdf{symbol}
непересекаются с типом \cdf{function}, это было несовместимым изменением.
Теперь \cdf{functionp} является ложной для символов и списков.
\end{newer}
\end{defun}


\begin{defun}[Функция]
compiled-function-p object

\cdf{compiled-function-p} истинен, если аргумент скомпилированный
объект кода, иначе ложен.
\begin{lisp}
(compiled-function-p x) \EQ\ (typep x 'compiled-function)
\end{lisp}
\end{defun}

Смотрите также \cdf{standard-char-p}, \cdf{string-char-p},
\cdf{streamp}, \cdf{random-state-p},
\cdf{readtablep},
\cdf{hash-table-p} и \cdf{pathnamep}.

\section{Предикаты равенства}

Common Lisp предоставляет ряд предикатов для проверки равенства двух
объектов:  \cdf{eq} (наиболее частный), \cdf{eql}, \cdf{equal} и \cdf{equalp}
(наиболее общий). \cdf{eq} и \cdf{equal} имеют значения традиционные в
Lisp'е. \cdf{eql} был добавлен, потому что он часто бывает необходим, и
\cdf{equalp} был добавлен преимущественно, как версия \cdf{equal}, которая
игнорирует различия типов при сравнении двух чисел и различия регистров при
сравнении строковых символов.
Если два объекта удовлетворяют любому из этих предикатов, то они
также удовлетворяют всем тем, которые носят более общий характер.

\begin{defun}[Функция]
eq x y

\cd{(eq \emph{x} \emph{y})} является истиной тогда и только тогда, когда,
\emph{x} и \emph{y} являются идентичными объектами.
(В реализациях, \emph{x} и \emph{y} обычно равны \cdf{eq} тогда и только
тогда, когда обращаются к одной ячейке памяти.)

Необходимо отметить, что вещи, которые выводят одно и то же, необязательно равны
\cdf{eql} друг другу. Символы с одинаковым именем обычно равны \cdf{eq} друг
другу, потому что используется функция \cdf{intern}.
Однако, одинаковые значения чисел могут быть не равны \cdf{eq}, и два похожих
списка обычно не равны \cdf{eq}.
Например:
\begin{lisp}
(eq 'a 'b) \textrm{ложь} \\
(eq 'a 'a) \textrm{истина} \\
(eq 3 3) \textrm{может быть истина или ложь, в зависимости от реализации} \\
(eq 3 3.0) \textrm{ложь} \\
(eq 3.0 3.0) \textrm{может быть истина или ложь, в зависимости от реализации} \\
(eq \#c(3 -4) \#c(3 -4)) \\
~~\textrm{может быть истина или ложь, в зависимости от реализации} \\
(eq \#c(3 -4.0) \#c(3 -4)) \textrm{ложь} \\
(eq (cons 'a 'b) (cons 'a 'c)) \textrm{ложь} \\
(eq (cons 'a 'b) (cons 'a 'b)) \textrm{ложь} \\
(eq '(a . b) '(a . b)) \textrm{может быть истина или ложь} \\
(progn (setq x (cons 'a 'b)) (eq x x)) \textrm{истина} \\
(progn (setq x '(a . b)) (eq x x)) \textrm{истина} \\
(eq \#{\Xbackslash}A \#{\Xbackslash}A) \textrm{может быть истина или ложь, в зависимости от реализации} \\
(eq "Foo" "Foo") \textrm{может быть истина или ложь} \\
(eq "Foo" (copy-seq "Foo")) \textrm{ложь} \\
(eq "FOO" "foo") \textrm{ложь}
\end{lisp}

В Common Lisp'е, в отличие от других диалектов, реализация в любое время может
создавать <<копии>> строковых символов и чисел. (Это сделано для возможности в
повышении производительности.) Из этого следует правило,
что Common Lisp не гарантирует для строковых символов и чисел то, что \cdf{eq}
будет истинен, когда оба аргумента являются <<одним и тем же>>.
Например:
\begin{lisp}
(let ((x 5)) (eq x x)) \textrm{может быть истиной или ложью}
\end{lisp}

Предикат \cdf{eql} означает то же, что и \cdf{eq}, за исключением того, что если
аргументы являются строковыми символами или числами одинакового типа, тогда
сравниваются их значения. Таким образом \cdf{eql} говорит, являются ли два объекта
<<концептуально (conceptually)>> одинаковыми, тогда как \cdf{eq} указывает, являются ли два
объекта <<реализационно (implementationally)>> одинаковыми. По этой причине
сравнительным предикатом для функций работы с последовательностями, описанными в
главе~\ref{KSEQUE}, является \cdf{eql}, а не \cdf{eq}.
\end{defun}

\begin{defun}[Функция]
eql x y

Предикат \cdf{eql} истинен, если его аргументы равны \cdf{eq}, или 
если это числа одинакового типа и с одинаковыми значениями, или если это
одинаковые строковые символы.
Например:
\begin{lisp}
(eql 'a 'b) \textrm{ложь} \\
(eql 'a 'a) \textrm{истина} \\
(eql 3 3) \textrm{истина} \\
(eql 3 3.0) \textrm{ложь} \\
(eql 3.0 3.0) \textrm{истина} \\
(eql \#c(3 -4) \#c(3 -4)) \textrm{истина} \\
(eql \#c(3 -4.0) \#c(3 -4)) \textrm{ложь} \\
(eql (cons 'a 'b) (cons 'a 'c)) \textrm{ложь} \\
(eql (cons 'a 'b) (cons 'a 'b)) \textrm{ложь} \\
(eql '(a . b) '(a . b)) \textrm{может быть истиной или ложью} \\
(progn (setq x (cons 'a 'b)) (eql x x)) \textrm{истина} \\
(progn (setq x '(a . b)) (eql x x)) \textrm{истина} \\
(eql \#{\Xbackslash}A \#{\Xbackslash}A) \textrm{истина} \\
(eql "Foo" "Foo") \textrm{может быть истиной или ложью} \\
(eql "Foo" (copy-seq "Foo")) \textrm{ложь} \\
(eql "FOO" "foo") \textrm{ложь}
\end{lisp}

Обычно \cd{(eql 1.0s0 1.0d0)} будет ложью, так как \cd{1.0s0} и \cd{1.0d0} не
принадлежат одному типу данных. Однако в реализация может отсутствовать полный
набор чисел с плавающей точкой, поэтому в такой ситуации \cd{(eql 1.0s0 1.0d0)}
может быть истиной. Предикат \cdf{=} будет сравнивать значения двух чисел, даже
если числа принадлежат разным типам.

Если реализация поддерживает положительный и отрицательный нули, как различные
значения (так IEEE стандарт предлагает реализовывать формат числа с плавающей
точкой), тогда \cd{(eql 0.0 -0.0)} будет ложью. В противном случае, когда
синтаксис \cd{-0.0} интерпретируется, как значение \cd{0.0}, тогда \cd{(eql 0.0
  -0.0)} будет истиной. Предикат \cdf{=} отличается от \cdf{eql} в том, что
\cd{(= 0.0 -0.0)} будет всегда истинно, потому что \cdf{=} сравнивает
математические значения операндов, тогда как \cdf{eql} сравнивает, так сказать,
репрезентативные (representational) значения. FIXME.

Два комплексных числа будут равны \cdf{eql}, если их действительные части равны
\cd{eql} и мнимые части равны \cdf{eql}.
Например, \cd{(eql \#C(4 5) \#C(4 5))} является истиной и
\cd{(eql \#C(4 5) \#C(4.0 5.0))} является ложью.
Следует отметить, что \cd{(eql \#C(5.0 0.0) 5.0)} ложь,
а \cd{(eql \#C(5 0) 5)} истина.
В случае с \cd{(eql \#C(5.0 0.0) 5.0)}
два аргумента принадлежат разным типам и не равны \cdf{eql},
Однако, в случае \cd{(eql \#C(5 0) 5)},
\cd{\#C(5 0)} не является комплексным числом, и автоматически преобразуется, по
правилу канонизации комплексных чисел, в целое \cd{5}, так как дробное число
\cd{20/4} всегда упрощается до \cd{5}.

Случай \cd{(eql "Foo" "Foo")} обсуждался выше в описании \cdf{eq}. Тогда как
\cdf{eql} сравнивает значения чисел и строковых символов, он не сравнивает
содержимое строк. Сравнение символов двух строк может быть выполнено с помощью
\cdf{equal}, \cdf{equalp}, \cdf{string=} или \cdf{string-equal}.

\begin{defun}[Функция]
equal x y

Предикат \cdf{equal} истинен, если его аргументы это структурно похожие
(изоморфные) объекты. Грубое правило такое, что два объекта равны \cdf{equal}
тогда и только тогда, когда одинаково их выводимое представление.

Числа и строковые символы сравниваются также как и в \cdf{eql}.
Символы сравниваются как в \cdf{eq}. Этот метод сравнения символов может
нарушать правило и сравнении выводимого представления, в случае если различия
двух символов с одинаковым выводимым представлением.

Объекты, которые содержат другие элементы, будут равны \cdf{equal}, если они
принадлежат одному типу и содержащиеся элементы равны \cdf{equal}.
Эта проверка реализована в рекурсивном стиле и может быть зациклиться на
закольцованных структурах.

Для cons-ячеек, \cdf{equal} определён рекурсивно, как сравнение \cd{equal}
сначала \emph{car} элементов, а затем \emph{cdr}.

Два массива равны \cdf{equal} только, если они равны \cdf{eq}, с одним
исключением:
строки и битовые вектора сравниваются поэлементно.
Если какой-либо аргумент или оба содержат указатель заполнения (fill pointer),
данный указатель ограничит количество проверяемых с помощью \cd{equal}
элементов.
Буквы верхнего и нижнего регистров в строках расцениваются предикатом \cdf{equal}
как разные. (А \cdf{equalp} игнорирует различие в регистрах в строках.) 

Два объекта имени файла (pathname objects) равны \cdf{equal} тогда и только
тогда, когда все элементы (хост, устройство, и т.д.) равны. (Будут ли равны
буквы разных регистров зависит от файловой системы.) Имена файлов, которые равны
\cdf{equal}, должны быть функционально эквивалентны.

\begin{lisp}
(equal 'a 'b) \textrm{ложь} \\
(equal 'a 'a) \textrm{истина} \\
(equal 3 3) \textrm{истина} \\
(equal 3 3.0) \textrm{ложь} \\
(equal 3.0 3.0) \textrm{истина} \\
(equal \#c(3 -4) \#c(3 -4)) \textrm{истина} \\
(equal \#c(3 -4.0) \#c(3 -4)) \textrm{ложь} \\
(equal (cons 'a 'b) (cons 'a 'c)) \textrm{ложь} \\
(equal (cons 'a 'b) (cons 'a 'b)) \textrm{истина} \\
(equal '(a . b) '(a . b)) \textrm{истина} \\
(progn (setq x (cons 'a 'b)) (equal x x)) \textrm{истина} \\
(progn (setq x '(a . b)) (equal x x)) \textrm{истина} \\
(equal \#{\Xbackslash}A \#{\Xbackslash}A) \textrm{истина} \\
(equal "Foo" "Foo") \textrm{истина} \\
(equal "Foo" (copy-seq "Foo")) \textrm{истина} \\
(equal "FOO" "foo") \textrm{ложь}
\end{lisp}
Для сравнения дерева cons-ячеек применяя \cdf{eql} (или любой другой желаемый
предикат) для листьев, используйте \cdf{tree-equal}.

\end{defun}

\begin{defun}[Функция]
equalp x y

Два объекта равны \cdf{equalp}, если они равны \cdf{equal},
если они строковые символы и удовлетворяют предикату \cdf{char-equal}, который
игнорирует регистр и другие атрибуты символов,
если они числа и имеют одинаковое значение, даже если числа разных типов,
если они включает в себя элементы, которые также равны \cdf{equalp}.

Объекты, которые включают в себя элементы, равны \cdf{equalp}, если они
принадлежат одному типу и содержащиеся элементы равны \cdf{equalp}.
Проверка осуществляется в рекурсивном стиле и может не завершится на
закольцованных структурах.
Для cons-ячеек, предикат \cdf{equalp} определён рекурсивно и сравнивает сначала
\emph{car} элементы, а затем \emph{cdr}.

Два массива равны \cdf{equalp} тогда и только тогда, когда они имеют одинаковое
количество измерений, и размеры измерений совпадают, и все элементы равны
\cdf{equalp}. Специализация массива не сравнивается. Например,
строка и общий массив, случилось так, имеют одинаковые строковые символы,
тогда они будут равны \cdf{equalp} (но определённо не равны \cdf{equal}).
Если какой-либо аргумент содержит указатель заполнения, этот указатель
ограничивает число сравниваемых элементов. Так как \cdf{equalp} сравнивает
строки побуквенно, и не различает разных регистров букв, то сравнение строк
регистронезависимо.

Два символа могут быть равны \cdf{equalp} только тогда, когда они \cdf{eq},
т.е. являются идентичными объектами.

\begin{lisp}
(equalp 'a 'b) \textrm{ложь} \\
(equalp 'a 'a) \textrm{истина} \\
(equalp 3 3) \textrm{истина} \\
(equalp 3 3.0) \textrm{истина} \\
(equalp 3.0 3.0) \textrm{истина} \\
(equalp \#c(3 -4) \#c(3 -4)) \textrm{истина} \\
(equalp \#c(3 -4.0) \#c(3 -4)) \textrm{истина} \\
(equalp (cons 'a 'b) (cons 'a 'c)) \textrm{ложь} \\
(equalp (cons 'a 'b) (cons 'a 'b)) \textrm{истина} \\
(equalp '(a . b) '(a . b)) \textrm{истина} \\
(progn (setq x (cons 'a 'b)) (equalp x x)) \textrm{истина} \\
(progn (setq x '(a . b)) (equalp x x)) \textrm{истина} \\
(equalp \#{\Xbackslash}A \#{\Xbackslash}A) \textrm{истина} \\
(equalp "Foo" "Foo") \textrm{истина} \\
(equalp "Foo" (copy-seq "Foo")) \textrm{истина} \\
(equalp "FOO" "foo") \textrm{истина}
\end{lisp}
\end{defun}

\section{Логические операторы}

Common Lisp содержит три логических оператора для булевых значений:
\cdf{and}, \cdf{or} и \cdf{not} (и, или, не, соответственно). \cdf{and} и
\cdf{or} являются управляющими структурами, потому что их аргументы
вычисляются в зависимости от условия.
Функции \cdf{not} необходимо инвертировать её один аргумент, поэтому она может
быть простой функцией.

\begin{defun}[Функция]
not x

\cdf{not} возвращает {\true}, если \emph{x} является {\false}, иначе
возвращает {\false}.
Таким образом она инвертирует аргумент как булево значение.

\cdf{null} то же, что и \cdf{not}, обе функции включены для ясности. По
соглашению принято использовать \cdf{null}, когда надо проверить пустой ли
список, и \cdf{not}, когда надо инвертировать булево значение.
\end{defun}

\begin{defmac}
and {\,form}*

\cd{(and \emph{form1} \emph{form2} ... )} последовательно слева направо
вычисляет формы. Если какая-либо форма \emph{formN} вычислилась в {\false},
тогда немедленно возвращается значение {\nil} без выполнения оставшихся форм. Если все
формы кроме последней вычисляются в не-{\false} значение, \cdf{and} возвращает
то, что вернула последняя форма.
Таким образом, \cdf{and} может использоваться, как для логических операций, где
{\false} обозначает ложь и не-{\false} значения истину, так и для условных
выражений.
Например:
\begin{lisp}
(if (and (>= n 0) \\
~~~~~~~~~(< n (length a-simple-vector)) \\
~~~~~~~~~(eq (elt a-simple-vector n) 'foo)) \\
~~~~(princ "Foo!"))
\end{lisp}
Выражение выше выводит \cd{Foo!}, если \cd{n}-ый элемент вектора
\cd{a-simple-vector} является символом \cd{foo}, проверяя при этом вхождения
\cd{n} в границы вектора \cd{a-simple-vector}. \cdf{elt} не будет вызвано с
аргументом \cd{n} выходящим за границы вектора, так как \cdf{and} гарантирует
ленивую проверку аргументов слева направо.

Специальная форма Lisp'а \cdf{and} отличается тем, что в определённых случаях
вычисляет не все аргументы.

Запись предыдущего примера
\begin{lisp}
(and (>= n 0) \\
~~~~~(< n (length a-simple-vector)) \\
~~~~~(eq (elt a-simple-vector n) 'foo) \\
~~~~~(princ "Foo!"))
\end{lisp}
будет выполнять ту же функцию. Разница в них только стилистическая. Некоторые
программисты никогда не используют в форме \cdf{and} выражения с побочными
эффектами, предпочитая для этих целей использовать \cdf{if} или \cdf{when}. 

Из общего определения можно сделать дедуктивный вывод о том, что 
\cd{(and \emph{x})} \EQ\ \emph{x}. Также \cd{(and)} выполняется в {\true},
который тождественен этой операции.

Можно определить \cdf{and} в терминах \cdf{cond} таким образом:
\begin{lisp}
(and \emph{x} \emph{y} \emph{z} ... \emph{w}) \EQ\ (cond \=((not \emph{x}) {\false}) \\
\>((not \emph{y}) {\false}) \\
\>((not \emph{z}) {\false}) \\
\>$\ldots$ \\
\>({\true} \emph{w}))
\end{lisp}

Смотрите \cdf{id} и \cdf{when}, которые иногда являются стилистически более
удобными, чем \cdf{and} в целях ветвления.
Если необходимо проверить истинность предиката для всех элементов списка или
вектора (element 0 \emph{and} element 1 \emph{and}
element 2 \emph{and} $\ldots$), можно использовать функцию \cdf{every}.
\end{defmac}

\begin{defmac}
or {\,form}*

\cd{(or \emph{form1} \emph{form2} ... )} последовательно выполняет каждую
форму слева направо. Если какая-либо непоследняя форма выполняется в что-либо
отличное от {\false}, \cdf{or} немедленно возвращает это не-{\false} значение
без выполнения оставшихся форм. Если все формы кроме последней, вычисляются в
{\false}, \cdf{or} возвращает то, что вернула последняя форма.
Таким образом \cdf{or} может быть использована как для логических операций, 
в который {\false} обозначает ложь, и не-{\false} истину,
так и для условного выполнения форм.

Специальная форма Lisp'а \cdf{or} отличается тем, что в определённых случаях
вычисляет не все аргументы.

Из общего определения, можно сделать дедуктивный вывод о том, что \cd{(or
  \emph{x})} \EQ\ \emph{x}. Также, \cd{(or)} выполняется в {\nil}, который
тождественен этой операции.

Можно определить \cdf{or} в терминах \cdf{cond} таким образом:
\begin{lisp}
(or \emph{x} \emph{y} \emph{z} ... \emph{w}) \EQ\ (cond (\emph{x}) (\emph{y}) (\emph{z}) ... ({\true} \emph{w}))
\end{lisp}

Смотрите \cdf{id} и \cdf{unless}, которые иногда являются стилистически более
удобными, чем \cdf{or} в целях ветвления.
Если необходимо проверить истинность предиката для всех элементов списка или
вектора (element 0 \emph{or} element 1 \emph{or}
element 2 \emph{or} $\ldots$), можно использовать функцию \cdf{some}.
\end{defmac}

\fi       % Primarily type discrimination predicates
%Part{Contrl, Root = "CLM.MSS"}
%Chapter of Common Lisp Manual.  Copyright 1984, 1988, 1989 Guy L. Steele Jr.

\clearpage\def\pagestatus{ULTIMATE}

\ifx \rulang\Undef
\chapter{Control Structure}
\label{CONTRL}

Common Lisp provides a variety of special structures for organizing
programs.  Some have to do with flow of control (control structures),
while others control access to variables (environment structures).
Some of these features are implemented as special forms;
others are implemented as macros, which typically expand into
complex program fragments expressed in terms of special forms
or other macros.

Function application is the primary method for construction of Lisp
programs.  Operations are written as the application of a function
to its arguments.  Usually, Lisp programs are written as a large collection
of small functions, each of which implements a simple operation.
These functions operate by calling one another, and so larger
operations are defined in terms of smaller ones.
Lisp functions may call upon themselves recursively,
either directly or indirectly.

\begin{new}
Locally defined functions (\cdf{flet}, \cdf{labels}) and macros (\cdf{macrolet})
are quite versatile.
The new symbol macro facility allows even more syntactic flexibility.
\end{new}

While the Lisp language
is more applicative in style than statement-oriented, it
nevertheless provides many operations that produce side effects and
consequently requires constructs for controlling the sequencing of
side effects.  The construct
\cdf{progn}, which is roughly equivalent to an Algol \textbf{begin}-\textbf{end}
block with all its semicolons, executes a number of forms sequentially,
discarding the values of all but the last.
Many Lisp control constructs
include sequencing implicitly, in which case they are said to
provide an ``implicit \cdf{progn}.''
\indexterm{implicit \cdf{progn}}
Other sequencing constructs include \cdf{prog1} and \cdf{prog2}.

For looping, Common Lisp provides the general iteration facility
\cdf{do} as well as a variety
of special-purpose iteration facilities for iterating or mapping
over various data structures.

Common Lisp provides the simple one-way conditionals \cdf{when} and \cdf{unless},
the simple two-way conditional \cdf{if}, and the more general multi-way
conditionals such as \cdf{cond} and \cdf{case}.  The choice of which form
to use in any particular situation is a matter of taste and
style.

Constructs for performing non-local exits with various scoping
disciplines are provided: \cdf{block}, \cdf{return},
\cdf{return-from},
\cdf{catch}, and \cdf{throw}.

The multiple-value constructs provide an efficient way for a function
to return more than one value; see \cdf{values}.

\section{Constants and Variables}
\label{FUNCTION-NAME-SECTION}

Because some Lisp data objects are used to represent programs,
one cannot always notate a constant data object in a program simply
by writing the notation for the object unadorned; it would be ambiguous
whether a constant object or a program fragment was intended.
The \cdf{quote} special form resolves this ambiguity.

There are two kinds of variables in Common Lisp, in effect: ordinary
variables and function names.  There are some similarities between
the two kinds, and in a few cases there are similar functions for
dealing with them, for example \cdf{boundp} and \cdf{fboundp}.
However, for the most part the two kinds of variables are
used for very different purposes: one to name defined functions,
macros, and special forms, and the other to name data objects.

\begin{newer}
X3J13 voted in March 1989 \issue{FUNCTION-NAME} to introduce the concept
of a \emph{function-name}, which may be either a symbol or a two-element list whose
first element is the symbol \cdf{setf} and whose second element is a symbol.
The primary purpose of this is to allow \cdf{setf} expander functions to be
CLOS generic functions with user-defined methods.
Many places in Common Lisp that used to require a symbol for a function
name are changed to allow 2-lists as well; for example, \cdf{defun}
is changed so that one may write \cd{(defun (setf~foo) ...)},
and the \cdf{function} special form is changed to accept any function-name.
See also \cdf{fdefinition}.

By convention, any function named \cd{(setf \emph{f\/})} should return its first
argument as its only value, in order to preserve the specification that
\cdf{setf} returns its \emph{newvalue}.  See \cdf{setf}.

Implementations are free to extend the syntax of function-names to
include lists beginning with additional symbols other than \cdf{setf}
or \cdf{lambda}.
\end{newer}

\subsection{Reference}

The value of an ordinary variable
may be obtained simply by writing the name of the variable
as a form to be executed.  Whether this is treated as the name
of a special variable or a lexical variable is determined
by the presence or absence of an applicable \cdf{special} declaration;
see chapter~\ref{DECLAR}.

The following functions and special forms allow reference to the
values of constants and variables in other ways.

\begin{defspec}
quote object

\cd{(quote \emph{x})} simply returns \emph{x}.
The \emph{object} is not evaluated and may be any Lisp object whatsoever.
This construct allows any Lisp object to be written as a constant
value in a program.
For example:
\begin{lisp}
(setq a 43) \\
(list a (cons a 3)) \EV\ (43 (43 . 3)) \\
(list (quote a) (quote (cons a 3)) \EV\ (a (cons a 3))
\end{lisp}
Since \cdf{quote} forms are so frequently useful
but somewhat cumbersome to type, a standard abbreviation is defined for them:
any form \emph{f} preceded by a single quote (\cd{~'~}) character
is assumed to have \cd{(quote~~)} wrapped around it to
make \cd{(quote \emph{f})}.
For example:
\begin{lisp}
(setq x '(the magic quote hack))
\end{lisp}
is normally interpreted by \cdf{read} to mean
\begin{lisp}
(setq x (quote (the magic quote hack)))
\end{lisp}
See section~\ref{MACRO-CHARACTERS-SECTION}.

\begin{newer}
X3J13 voted in January 1989 \issue{CONSTANT-MODIFICATION} to clarify that
it is an error to destructively modify any object that appears as a constant
in executable code, whether within a \cdf{quote} special form or as
a self-evaluating form.

See section~\ref{COMPILER-SECTION} for a discussion of how quoted constants
are treated by the compiler.
\end{newer}

\begin{newer}
X3J13 voted in March 1989 \issue{QUOTE-SEMANTICS} to clarify that
\cdf{eval} and \cdf{compile} are not permitted either to copy or
to coalesce (``collapse'') constants (see \cdf{eq})
appearing in the code they process; the resulting
program behavior must refer to objects that are \cdf{eql} to the
corresponding objects in the source code.
Moreover, the constraints introduced by the votes on
issues \issue{CONSTANT-COMPILABLE-TYPES}
and \issue{CONSTANT-CIRCULAR-COMPILATION}
on what kinds of objects may appear
as constants apply only to \cdf{compile-file} (see section~\ref{COMPILER-SECTION}).
\end{newer}
\end{defspec}

\begin{defspec}
function fn

The value of \cdf{function} is always the functional interpretation
of \emph{fn}; \emph{fn} is interpreted as if it had appeared
in the functional position of a function invocation.
In particular,
if \emph{fn} is a symbol, the functional definition associated with
that symbol is returned; see \cdf{symbol-function}.
If \emph{fn} is a lambda-expression, then a
``lexical closure'' is returned, that is, a function that when invoked
will execute the body of the lambda-expression in such a way as to
observe the rules of lexical scoping properly.

\begin{newer}
X3J13 voted in June 1988 \issue{FUNCTION-TYPE}
to specify that the result of a \cdf{function} special form is always
of type \cdf{function}.  This implies that a form \cd{(function~\emph{fn\/})}
may be interpreted as \cd{(the (function~\emph{fn\/}))}.

It is an error to use the \cdf{function} special form on a
    symbol that does not denote a function in the lexical or global environment in
    which the special form appears.  Specifically, it is an error to use the
    \cdf{function} special form on a symbol that denotes a macro or special form.
    Some implementations may choose not to signal this error for
        performance reasons, but implementations are forbidden
        to extend the semantics of \cdf{function} in this respect; that is, an
        implementation is not allowed to
        define the failure to signal an error to be a ``useful'' behavior.
\end{newer}

\begin{newer}
X3J13 voted in March 1989 \issue{FUNCTION-NAME} to extend \cdf{function}
to accept any function-name (a symbol or a list
whose \emph{car} is \cdf{setf}---see section~\ref{FUNCTION-NAME-SECTION})
as well as lambda-expressions.
Thus one may write \cd{(function (setf cadr))} to refer to the \cdf{setf}
expansion function for \cdf{cadr}.
\end{newer}

\indexterm{closure}
For example:
\begin{lisp}
(defun adder (x) (function (lambda (y) (+ x y))))
\end{lisp}
The result of \cd{(adder 3)} is a function that will add \cd{3} to its
argument:
\begin{lisp}
(setq add3 (adder 3)) \\
(funcall add3 5) \EV\ 8
\end{lisp}
This works because \cdf{function} creates a closure of
the inner lambda-expression that is able to refer to the value \cd{3}
of the variable \cd{x} even after control has returned from the
function \cdf{adder}.

More generally, a lexical closure in effect retains the ability to
refer to lexically visible \emph{bindings}, not just values.
Consider this code:
\begin{lisp}
(defun two-funs (x) \\
~~(list (function (lambda () x)) \\
~~~~~~~~(function (lambda (y) (setq x y))))) \\
(setq funs (two-funs 6)) \\
(funcall (car funs)) \EV\ 6 \\
(funcall (cadr funs) 43) \EV\ 43 \\
(funcall (car funs)) \EV\ 43
\end{lisp}
The function \cdf{two-funs} returns a list of two functions, each of which
refers to the \emph{binding} of the variable \cdf{x} created on entry to
the function \cdf{two-funs} when it was called with argument \cd{6}.
This binding has the value \cd{6} initially, but \cdf{setq} can alter
a binding.  The lexical closure created for the first lambda-expression
does not ``snapshot'' the value \cd{6} for \cdf{x} when the closure is created.
The second function can be used to alter the binding (to \cd{43}, in the
example), and this altered
value then becomes accessible to the first function.

In situations where a closure of a lambda-expression over the same set
of bindings may be produced more than once, the various resulting closures
may or may not be \cdf{eq}, at the discretion of the implementation.
For example:
\begin{lisp}
(let ((x 5) (funs '())) \\*
~~(dotimes (j 10) \\*
~~~~(push \#'(lambda (z) \\*
~~~~~~~~~~~~~~(if (null z) (setq x 0) (+ x z))) \\*
~~~~~~~~~~funs)) \\*
~~funs)
\end{lisp}
The result of the above expression is a list of ten closures.
Each logically requires only the binding of \cdf{x}.
It is the same binding in each case,
so the ten closures may or may not be the same identical (\cdf{eq}) object.
On the other hand, the result of the expression
\begin{lisp}
(let ((funs '())) \\*
~~(dotimes (j 10) \\*
~~~~(let ((x 5)) \\*
~~~~~~(push (function (lambda (z) \\*
~~~~~~~~~~~~~~~~~~~~~~~~(if (null z) (setq x 0) (+ x z)))) \\*
~~~~~~~~~~~~funs))) \\*
~~funs)
\end{lisp}
is also a list of ten closures.
However, in this case no two of the closures may be \cdf{eq}, because each
closure is over a distinct binding of \cdf{x}, and these bindings can
be behaviorally distinguished because of the use of \cdf{setq}.

The question of distinguishable behavior is important; the result of
the simpler expression
\begin{lisp}
(let ((funs '())) \\*
~~(dotimes (j 10) \\*
~~~~(let ((x 5)) \\*
~~~~~~(push (function (lambda (z) (+ x z))) \\*
~~~~~~~~~~~~funs))) \\*
~~funs)
\end{lisp}
is a list of ten closures that \emph{may} be pairwise \cdf{eq}.  Although
one might think that a different binding of \cdf{x} is involved for
each closure (which is indeed the case), the bindings cannot be distinguished
because their values are identical and immutable, there being no occurrence
of \cdf{setq} on \cdf{x}.  A compiler would therefore be justified in
transforming the expression to
\begin{lisp}
(let ((funs '())) \\*
~~(dotimes (j 10) \\*
~~~~(push (function (lambda (z) (+ 5 z))) \\*
~~~~~~~~~~funs)) \\*
~~funs)
\end{lisp}
where clearly the closures may be the same after all.
The general rule, then, is that the implementation is free to
have two distinct evaluations of the same \cdf{function} form
produce identical (\cdf{eq}) closures if it can prove that the
two conceptually distinct resulting closures must in fact be
behaviorally identical with respect to invocation.
This is merely a permitted optimization; a perfectly valid
implementation might simply cause every distinct evaluation of a \cdf{function}
form to produce a new closure object not \cdf{eq} to any other.

Frequently a compiler can deduce that a closure in fact does not
need to close over any variable bindings.  For example,
in the code fragment
\begin{lisp}
(mapcar (function (lambda (x) (+ x 2))) y)
\end{lisp}
the function \cd{(lambda (x) (+ x 2))} contains no references to any outside
entity.  In this important special case, the same ``closure'' may be used
as the value for all evaluations of the \cdf{function} special form.
Indeed, this value need not be a closure object at all; it may
be a simple compiled function containing no environment information.
This example is simply a special case of the foregoing discussion and
is included as a hint to implementors familiar with previous methods
of implementing Lisp.  The distinction between closures and other
kinds of functions is somewhat pointless, actually, as Common Lisp defines
no particular representation for closures and no way to distinguish
between closures and non-closure functions.  All that matters is that
the rules of lexical scoping be obeyed.

Since \cdf{function} forms are so frequently useful
but somewhat cumbersome to type, a standard abbreviation is defined for them:
any form \emph{f} preceded by \cd{\#'} (\cd{\#} followed by an apostrophe)
is assumed to have \cd{(function  )} wrapped around it to make
\cd{(function \emph{f})}.  For example,
\begin{lisp}
(remove-if \#'numberp '(1 a b 3))
\end{lisp}
is normally interpreted by \cdf{read} to mean
\begin{lisp}
(remove-if (function numberp) '(1 a b 3))
\end{lisp}
See section~\ref{SHARP-SIGN-MACRO-CHARACTER-SECTION}.
\end{defspec}

\begin{defun}[Function]
symbol-value symbol

\cdf{symbol-value} returns the current value of the dynamic (special) variable
named by \emph{symbol}.
An error occurs if the symbol has no value; see \cdf{boundp}
and \cdf{makunbound}.
Note that constant symbols are really variables that cannot be changed,
and so \cdf{symbol-value} may be used to get the value of
a named constant.  In particular, \cdf{symbol-value} of a keyword
will return that keyword.

\cdf{symbol-value} cannot access the value of a lexical variable.

This function is particularly useful for implementing interpreters
for languages embedded in Lisp.
The corresponding assignment primitive is \cdf{set};
alternatively, \cdf{symbol-value} may be used with \cdf{setf}.
\end{defun}

\begin{defun}[Function]
symbol-function symbol

\cdf{symbol-function} returns the current global function definition
named by \emph{symbol}.  An error is signalled if the symbol has no function
definition; see \cdf{fboundp}.  Note that the definition may be a
function or may be an object representing a special form or macro.
In the latter case, however, it is an error
to attempt to invoke the object as a function.
If it is desired to process macros, special forms, and functions
equally well, as when writing an interpreter,
it is best first to test the symbol with \cdf{macro-function}
and \cdf{special-operator-p}
and then to invoke the functional value only if these
two tests both yield false.

This function is particularly useful for implementing interpreters
for languages embedded in Lisp.

\cdf{symbol-function} cannot access the value of a lexical function name
produced by \cdf{flet} or \cdf{labels}; it can access only
the global function value.

The global function definition of a symbol may be altered
by using \cdf{setf} with \cdf{symbol-function}.
Performing this operation causes the symbol to have \emph{only} the
specified definition as its global function definition; any previous
definition, whether as a macro or as a function, is lost.
It is an error to attempt to redefine the name of a special
form (see table~\ref{SPECIAL-FORM-TABLE}).

\begin{newer}
X3J13 voted in June 1988 \issue{FUNCTION-TYPE} to clarify the behavior
of \cdf{symbol-function} in the light of the redefinition of the type \cdf{function}.
\begin{itemize}
\item It is permissible to call \cdf{symbol-function}
    on any symbol for which \cdf{fboundp} returns true.
 Note that \cdf{fboundp} must return true for a symbol naming a macro or
    a special form.

\item If \cdf{fboundp} returns true for a symbol
        but the symbol denotes a macro or special form,
        then the value returned by \cdf{symbol-function} is not well-defined
        but \cdf{symbol-function} will not signal an error. 

\item When \cdf{symbol-function} is used with \cdf{setf}
 the new value must be of type \cdf{function}.
	It is an error to set the \cdf{symbol-function} of a symbol to a
	symbol, a list, or the value returned by \cdf{symbol-function} on
	the name of a macro or a special form.
\end{itemize}
\end{newer}
\end{defun}

\begin{newer}
\begin{defun}[Function]
fdefinition function-name

X3J13 voted in March 1989 \issue{FUNCTION-NAME} to add the function
\cdf{fdefinition} to the language.
It is exactly like \cdf{symbol-function}
except that its argument may be any function-name (a symbol or a list
whose \emph{car} is \cdf{setf}---see section~\ref{FUNCTION-NAME-SECTION});
it returns the current global function
definition named by the argument \emph{function-name}.
One may use \cdf{fdefinition} with \cdf{setf}
to change the current global function definition associated with
a function-name.
\end{defun}
\end{newer}

\begin{defun}[Function]
boundp symbol

\cdf{boundp} is true if the dynamic (special) variable named by \emph{symbol}
has a value; otherwise, it returns {\false}.

See also \cdf{set} and \cdf{makunbound}.
\end{defun}

\begin{defun}[Function]
fboundp symbol

\cdf{fboundp} is true if the symbol has a global function definition.
Note that \cdf{fboundp} is true when the symbol names a special form or
macro.  \cdf{macro-function} and \cdf{special-operator-p} may be used to test
for these cases.

\begin{newer}
X3J13 voted in June 1988 \issue{FUNCTION-TYPE} to emphasize that,
despite the tightening of the definition of the type \cdf{function},
\cdf{fboundp} must return true when the argument names a special form or
macro.
\end{newer}

See also \cdf{symbol-function} and \cdf{fmakunbound}.

\begin{newer}
X3J13 voted in March 1989 \issue{FUNCTION-NAME} to extend \cdf{fboundp}
to accept any function-name (a symbol or a list
whose \emph{car} is \cdf{setf}---see section~\ref{FUNCTION-NAME-SECTION}).
Thus one may write \cd{(fboundp '(setf cadr))} to determine whether a \cdf{setf}
expansion function has been globally defined for \cdf{cadr}.
\end{newer}
\end{defun}

\begin{defun}[Function]
special-operator-p symbol

The function \cdf{special-operator-p} takes a symbol.  If the symbol
globally names a special form,
then a non-{\false} value is returned; otherwise {\false} is returned.
A returned non-{\nil} value is typically a function
of implementation-dependent nature that can be used to
interpret (evaluate) the special form.

It is possible for \emph{both} \cdf{special-operator-p} and \cdf{macro-function}
to be true of a symbol.  This is possible because an implementation is
permitted to implement any macro also as a special form for speed.
On the other hand, the macro definition must be available
for use by programs that understand only the standard special forms
listed in table~\ref{SPECIAL-FORM-TABLE}.
\end{defun}

\subsection{Assignment}

The following facilities allow the value of a variable (more specifically,
the value associated with the current binding of the variable) to be altered.
Such alteration is different from establishing a new binding.
Constructs for establishing new bindings of variables are described
in section~\ref{VAR-BINDING-SECTION}.

\begin{defspec}
setq {var form}*

The special form \cd{(setq \emph{var1} \emph{form1} \emph{var2} \emph{form2} ...)} is the
``simple variable assignment statement'' of Lisp.
First \emph{form1} is evaluated
and the result is stored in the variable \emph{var1}, then \emph{form2}
is evaluated and the result stored in \emph{var2}, and so forth.
The variables are represented as symbols, of course, and are interpreted
as referring to static or dynamic instances according to the usual rules.
Therefore \cdf{setq} may be used for assignment of both lexical
and special variables.

\cdf{setq} returns the last value assigned, that is, the result of the
evaluation of its last argument.
As a boundary case, the form \cd{(setq)} is legal and returns {\false}.
There must be an even number of argument forms.
For example, in
\begin{lisp}
(setq x (+ 3 2 1) y (cons x nil))
\end{lisp}
\cdf{x} is set to \cd{6}, \cdf{y} is set to \cd{(6)}, and the \cdf{setq}
returns \cd{(6)}.  Note that the first assignment is performed before
the second form is evaluated, allowing that form to
use the new value of \cdf{x}.

See also the description of \cdf{setf},
the Common Lisp ``general assignment statement'' that is capable of assigning
to variables, array elements, and other locations.

\begin{newer}
Some programmers choose to avoid
\cdf{setq} as a matter of style, always using \cdf{setf} for any kind of
structure modification.  Others use \cdf{setq} with simple variable names and
\cdf{setf} with all other generalized variables.
\end{newer}

\begin{new}
X3J13 voted in March 1989
\issue{SYMBOL-MACROLET-SEMANTICS} to specify that if any \emph{var}
refers not to an ordinary variable but to a binding made by
\cdf{symbol-macrolet}, then that \emph{var} is handled as
if \cdf{setf} had been used instead of \cdf{setq}.
\end{new}
\end{defspec}

\begin{defmac}
psetq {var form}*

A \cdf{psetq} form is just like a \cdf{setq} form, except
that the assignments happen in parallel.  First all of the forms
are evaluated, and then the variables are set to the resulting
values.  The value of the \cdf{psetq} form is {\false}.
For example:
\begin{lisp}
(setq a 1) \\
(setq b 2) \\
(psetq a b  b a) \\
a \EV\ 2 \\
b \EV\ 1
\end{lisp}
In this example, the values of \cdf{a} and \cdf{b} are exchanged by
using parallel assignment.
(If several variables are to be assigned in parallel in
the context of a loop, the \cdf{do} construct may be appropriate.)

See also the description of \cdf{psetf},
the Common Lisp ``general parallel assignment statement'' that
is capable of assigning
to variables, array elements, and other locations.

\begin{newer}
X3J13 voted in March 1989
\issue{SYMBOL-MACROLET-SEMANTICS} to specify that if any \emph{var}
refers not to an ordinary variable but to a binding made by
\cdf{symbol-macrolet}, then that \emph{var} is handled as
if \cdf{psetf} had been used instead of \cdf{psetq}.
\end{newer}
\end{defmac}

\begin{defun}[Function]
set symbol value

\cdf{set} allows alteration of the value of a dynamic (special) variable.
\cdf{set} causes the dynamic variable named by \emph{symbol} to take on
\emph{value} as its value.

\begin{new}
X3J13 voted in January 1989
\issue{ARGUMENTS-UNDERSPECIFIED}
to clarify that the \emph{value}
may be any Lisp datum whatsoever.
\end{new}

Only the value of the current dynamic binding is altered;
if there are no bindings in effect, the most global value is altered.
For example,
\begin{lisp}
(set (if (eq a b) 'c 'd) 'foo)
\end{lisp}
will either set \cdf{c} to \cdf{foo} or set \cdf{d} to \cdf{foo}, depending
on the outcome of the test \cd{(eq~a~b)}.

\cdf{set} returns \emph{value} as its result.

\cdf{set} cannot alter
the value of a local (lexically bound) variable.
The special form \cdf{setq}
is usually used for altering the values of variables
(lexical or dynamic) in programs.
\cdf{set} is particularly useful for implementing interpreters for
languages embedded in Lisp.
See also \cdf{progv}, a construct that performs binding rather
than assignment of dynamic variables.
\end{defun}

\begin{defun}[Function]
makunbound symbol \\
fmakunbound symbol

\cdf{makunbound} causes the dynamic (special) variable named
by \emph{symbol} to become unbound (have no value).  \cdf{fmakunbound}
does the analogous thing for the global function definition named
by \emph{symbol}.
For example:
\begin{lisp}
(setq a 1) \\
a \EV\ 1 \\
(makunbound 'a) \\
a \EV\ \textrm{causes an error} \\
\\
(defun foo (x) (+ x 1)) \\
(foo 4) \EV\ 5 \\
(fmakunbound 'foo) \\
(foo 4) \EV\ \textrm{causes an error}
\end{lisp}
Both functions return \emph{symbol} as the result value.

\begin{newer}
X3J13 voted in March 1989 \issue{FUNCTION-NAME} to extend \cdf{fmakunbound}
to accept any function-name (a symbol or a list
whose \emph{car} is \cdf{setf}---see section~\ref{FUNCTION-NAME-SECTION}).
Thus one may write \cd{(fmakunbound '(setf cadr))} to remove any
global definition of a \cdf{setf} expansion function for \cdf{cadr}.
\end{newer}
\end{defun}

\section{Generalized Variables}
\label{SETF-SECTION}
 
In Lisp, a variable can remember one piece of data,
that is, one Lisp object.
The main operations on a variable are to recover that object and
to alter the variable to remember a new object; these operations are
often called \emph{access} and \emph{update} operations.  The concept of
variables named by symbols can be generalized to any storage location
that can remember one piece of data, no matter how that location is
named.  Examples of such storage locations are the \emph{car} and \emph{cdr} of
a cons, elements of an array, and components of a structure.

For each kind of generalized variable, typically there are two functions
that implement the conceptual \emph{access} and \emph{update} operations.
For a variable, merely mentioning the name of the variable accesses it,
while the \cdf{setq} special form can be used to update it.
The function \cdf{car} accesses the \emph{car} of a cons,
and the function \cdf{rplaca} updates it.
The function \cdf{symbol-value} accesses the dynamic value of a variable
named by a given symbol, and the function \cdf{set} updates it.

Rather than thinking about two distinct functions that respectively
access and update a storage location somehow deduced from their
arguments, we can instead simply think of a call to the access function
with given arguments as a \emph{name} for the storage location.  Thus, just
as \cdf{x} may be considered a name for a storage location (a variable), so
\cd{(car x)} is a name for the \emph{car} of some cons (which is in turn
named by \cdf{x}).  Now, rather than having to remember two functions for
each kind of generalized variable (having to remember, for example, that
\cdf{rplaca} corresponds to \cdf{car}), we adopt a uniform syntax for updating
storage locations named in this way, using the \cdf{setf} macro.
This is analogous to the way we use the \cdf{setq} special form to convert
the name of a variable (which is also a form that accesses it) into a
form that updates it.  The uniformity of this approach is illustrated in
the following table.

\begin{flushleft}
\begin{tabular*}{\textwidth}{@{}l@{\extracolsep{\fill}}ll@{}}
\textrm{Access Function}&\textrm{Update Function}&\textrm{Update Using \cdf{setf}} \\
\hlinesp
\cd{x}&\cd{(setq x datum)}&\cd{(setf x datum)} \\
\cd{(car x)}&\cd{(rplaca x datum)}&\cd{(setf (car x) datum)} \\
\cd{(symbol-value x)}&\cd{(set x datum)}&\cd{(setf (symbol-value x) datum)} \\
\hline
\end{tabular*}
\end{flushleft}
\cdf{setf} is actually a macro that examines an access form and
produces a call to the corresponding update function.

Given the existence of \cdf{setf} in Common Lisp, it is not necessary to have
\cdf{setq}, \cdf{rplaca}, and \cdf{set}; they are redundant.  They
are retained in Common Lisp because of their historical importance in Lisp.
However, most other update functions (such as \cdf{putprop}, the update
function for \cdf{get}) have been eliminated from Common Lisp
in the expectation that \cdf{setf} will be uniformly used in their place.

\begin{defmac}
setf {place newvalue}*

\cd{(setf \emph{place} \emph{newvalue})} takes a form \emph{place} that when evaluated
\emph{accesses} a data object in some location and ``inverts''
it to produce a corresponding form to \emph{update} the location.
A call to the \cdf{setf} macro therefore
expands into an update form that stores the result of evaluating
the form \emph{newvalue} into the place referred to by the access form.

If more than one \emph{place}-\emph{newvalue} pair is specified,
the pairs are processed sequentially; that is,
\begin{lisp}
(setf \emph{place1} \emph{newvalue1} \\
~~~~~~\emph{place2} \emph{newvalue2}) \\
~~~~~~... \\
~~~~~~\emph{placen} \emph{newvaluen})
\end{lisp}
is precisely equivalent to
\begin{lisp}
(progn (setf \emph{place1} \emph{newvalue1}) \\
~~~~~~~(setf \emph{place2} \emph{newvalue2}) \\
~~~~~~~... \\
~~~~~~~(setf \emph{placen} \emph{newvaluen}))
\end{lisp}
For consistency, it is legal to write \cd{(setf)}, which simply returns {\nil}.

The form \emph{place} may be any one of the following:
\begin{itemize}
\item
The name of a variable (either lexical or dynamic).

\item
A function call form whose first element is the name of
any one of the following functions:

\begin{flushleft}
\begin{tabular}{@{}llll@{}}
\cdf{aref}&\cdf{car}&\cdf{svref}& \\
\cdf{nth}&\cdf{cdr}&\cdf{get}& \\
\cdf{elt}&\cdf{caar}&\cdf{getf}&\cdf{symbol-value} \\
\cdf{rest}&\cdf{cadr}&\cdf{gethash}&\cdf{symbol-function} \\
\cdf{first}&\cdf{cdar}&\cdf{documentation}&\cdf{symbol-plist} \\
\cdf{second}&\cdf{cddr}&\cdf{fill-pointer}&\cdf{macro-function} \\
\cdf{third}&\cdf{caaar}&\cdf{caaaar}&\cdf{cdaaar} \\
\cdf{fourth}&\cdf{caadr}&\cdf{caaadr}&\cdf{cdaadr} \\
\cdf{fifth}&\cdf{cadar}&\cdf{caadar}&\cdf{cdadar} \\
\cdf{sixth}&\cdf{caddr}&\cdf{caaddr}&\cdf{cdaddr} \\
\cdf{seventh}&\cdf{cdaar}&\cdf{cadaar}&\cdf{cddaar} \\
\cdf{eighth}&\cdf{cdadr}&\cdf{cadadr}&\cdf{cddadr} \\
\cdf{ninth}&\cdf{cddar}&\cdf{caddar}&\cdf{cdddar} \\
\cdf{tenth}&\cdf{cdddr}&\cdf{cadddr}&\cdf{cddddr}
\end{tabular}
\end{flushleft}

\begin{new}
X3J13 voted in March 1988 \issue{AREF-1D}
to add \cdf{row-major-aref} to this list.
\end{new}

\begin{newer}
X3J13 voted in June 1989 \issue{DEFINE-COMPILER-MACRO}
to add \cdf{compiler-macro-function} to this list.
\end{newer}

\begin{newer}
X3J13 voted in March 1989 \issue{FUNCTION-NAME} to clarify that this
rule applies only when the function name refers to a global function
definition and not to a locally defined function or macro.
\end{newer}

\item
A function call form whose first element is the name of
a selector function constructed by \cdf{defstruct}.

\begin{newer}
X3J13 voted in March 1989 \issue{FUNCTION-NAME} to clarify that this
rule applies only when the function name refers to a global function
definition and not to a locally defined function or macro.
\end{newer}

\item
A function call form whose first element is the name of
any one of the following functions, provided that the new value
\vadjust{\penalty-10000}%manual
is of the specified type so that it can be used to
replace the specified ``location'' (which is in each of these cases
not truly a generalized variable):

\begin{obsolete}
\begin{flushleft}
\leavevmode
\begin{tabular}{@{}ll@{}}
Function Name&Required Type \\
\hlinesp
\cdf{char}&\cdf{string-char} \\
\cdf{schar}&\cdf{string-char} \\
\cdf{bit}&\cdf{bit} \\
\cdf{sbit}&\cdf{bit} \\
\cdf{subseq}&\cdf{sequence} \\
\hline
\end{tabular}
\end{flushleft}
\end{obsolete}

\begin{newer}
X3J13 voted in March 1989 \issue{CHARACTER-PROPOSAL}
to eliminate the type \cdf{string-char} and to redefine
\cdf{string} to be the union of one or more specialized vector
types, the types of whose elements are subtypes of the type \cdf{character}.
In the preceding table, the type \cdf{string-char} should be replaced
by some such phrase as ``the element-type of the argument vector.''
\end{newer}

\begin{newer}
X3J13 voted in March 1989 \issue{FUNCTION-NAME} to clarify that this
rule applies only when the function name refers to a global function
definition and not to a locally defined function or macro.
\end{newer}

In the case of \cdf{subseq}, the replacement value must be a sequence
whose elements may be contained by the sequence argument to \cdf{subseq}.
(Note that this is not so stringent as to require that the
replacement value be a sequence of the same type as the sequence
of which the subsequence is specified.)
If the length of the replacement value does not equal the length of
the subsequence to be replaced, then the shorter length determines
the number of elements to be stored, as for the function \cdf{replace}.

\item
A function call form whose first element is the name of
any one of the following functions, provided that the specified argument
to that function is in turn a \emph{place} form;
in this case the new \emph{place} has stored back into it the
result of applying the specified ``update'' function
(which is in each of these cases not a true update function):

\begin{flushleft}
\begin{tabular}{@{}lll@{}}
Function Name&Argument That Is a \emph{place}&Update Function Used \\
\hlinesp
\cdf{char-bit}&first&\cdf{set-char-bit} \\
\cdf{ldb}&second&\cdf{dpb} \\
\cdf{mask-field}&second&\cdf{deposit-field} \\
\hline
\end{tabular}
\end{flushleft}

\begin{newer}
X3J13 voted in March 1989 \issue{CHARACTER-PROPOSAL}
to eliminate \cdf{char-bit} and \cdf{set-char-bit}.
\end{newer}

\begin{newer}
X3J13 voted in March 1989 \issue{FUNCTION-NAME} to clarify that this
rule applies only when the function name refers to a global function
definition and not to a locally defined function or macro.
\end{newer}

\item
A \cdf{the} type declaration form, in which case the declaration is
transferred to the \emph{newvalue} form, and the resulting \cdf{setf} form is
analyzed.  For example,
\begin{lisp}
(setf (the integer (cadr x)) (+ y 3))
\end{lisp}
is processed as if it were
\begin{lisp}
(setf (cadr x) (the integer (+ y 3)))
\end{lisp}

\item
A call to \cdf{apply} where the first argument form is of the form
\cd{\#'\emph{name}}, that is, \cd{(function \emph{name})}, where \emph{name}
is the name of a function, calls to which are recognized as places by \cdf{setf}.
Suppose that the use of \cdf{setf} with \cdf{apply} looks like this:
\begin{lisp}
(setf (apply \#'\emph{name} \emph{x1} \emph{x2} ... \emph{xn} \emph{rest}) \emph{x0})
\end{lisp}
The \cdf{setf} method for the function \emph{name} must be such that
\begin{lisp}
(setf (\emph{name} \emph{z1} \emph{z2} ... \emph{zm}) \emph{z0})
\end{lisp}
expands into a store form
\begin{lisp}
(\emph{storefn} \emph{zi${}_1$} \emph{zi${}_2$} ... \emph{zi${}_k$} \emph{zm})
\end{lisp}
That is, it must expand into a function call such that all arguments but
the last may be any permutation or subset of the new value \emph{z0} and
the arguments of the access form, but the \emph{last} argument of the storing
call must be the same as the last argument of the access call.
See \cdf{define-setf-method} for more details on accessing
and storing forms.

Given this, the \cdf{setf}-of-\cdf{apply} form shown above expands into
\begin{lisp}
(apply \#'\emph{storefn} \emph{xi${}_1$} \emph{xi${}_2$} ... \emph{xi${}_k$} \emph{rest})
\end{lisp}
As an example, suppose that the variable \cdf{indexes} contains a list
of subscripts for a multidimensional array \emph{foo} whose rank is not
known until run time.  One may access the indicated
element of the array by writing
\begin{lisp}
(apply \#'aref foo indexes)
\end{lisp}
and one may alter the value of the indicated element to that
of \cdf{newvalue} by writing
\begin{lisp}
(setf (apply \#'aref foo indexes) newvalue)
\end{lisp}

\begin{newer}
X3J13 voted in March 1989 \issue{FUNCTION-NAME} to clarify that this
rule applies only when the function name \cdf{apply} refers to the global function
definition and not to a locally defined function or macro named \cdf{apply}.
\end{newer}

\item
A macro call, in which case \cdf{setf} expands the macro call and
then analyzes the resulting form.

\begin{newer}
X3J13 voted in March 1989 \issue{FUNCTION-NAME} to clarify that this
step uses \cdf{macroexpand-1}, not \cdf{macroexpand}.  This allows the chance
to apply any of the rules preceding this one to any of the intermediate expansions.
\end{newer}

\item
Any form for which a \cdf{defsetf}
or \cdf{define-setf-method} declaration has been made.

\begin{newer}
X3J13 voted in March 1989 \issue{FUNCTION-NAME} to clarify that this
rule applies only when the function name in the form refers to a global function
definition and not to a locally defined function or macro.
\end{newer}

\end{itemize}

\begin{newer}
X3J13 voted in March 1989 \issue{FUNCTION-NAME} to add one more rule to
the preceding list, coming after all those listed above:
\begin{itemize}
\item  Any other list whose first element is a symbol (call it \emph{f\/}).
     In this case, the call to \cdf{setf} expands into a call to the function
     named by the
     list \cd{(setf~\emph{f\/})} (see section~\ref{FUNCTION-NAME-SECTION}).
     The first argument is the new value and the
     remaining arguments are the values of the remaining elements of
     \emph{place}.  This expansion occurs regardless of whether either \emph{f\/} or
     \cd{(setf \emph{f\/})} is defined as a function locally, globally, or not at
     all.  For example,
\begin{lisp}
(setf (\emph{f\/} \emph{arg1} \emph{arg2} ...) \emph{newvalue})
\end{lisp}
     expands into a form with the same effect and value as
\begin{lisp}
(let ((\#:temp1 \emph{arg1})~~~~~;\textrm{Force correct order of evaluation} \\*
~~~~~~(\#:temp2 \emph{arg2}) \\*
~~~~~~... \\*
~~~~~~(\#:temp0 newvalue)) \\*
~~(funcall (function (setf \emph{f\/})) \\*
~~~~~~~~~~~\#:temp0 \\*
~~~~~~~~~~~\#:temp1 \\*
~~~~~~~~~~~\#:temp2 ...))
\end{lisp}
By convention, any function named \cd{(setf \emph{f\/})} should return its first
argument as its only value, in order to preserve the specification that
\cdf{setf} returns its \emph{newvalue}.
\end{itemize}
\end{newer}

\begin{new}
X3J13 voted in March 1989
\issue{SYMBOL-MACROLET-SEMANTICS} to add this case as well:
\begin{itemize}
\item A variable reference that refers to a symbol macro definition made by
\cd{symbol-\discretionary{}{}{}macrolet}, in which case \cdf{setf} expands the reference and
then analyzes the resulting form.
\end{itemize}
\end{new}

\newpage%manual

\cdf{setf} carefully arranges to preserve the usual left-to-right
order in which the various subforms are evaluated.
On the other hand,
the exact expansion for any particular form is not guaranteed and
may even be implementation-dependent; all that is guaranteed is that
the expansion of a \cdf{setf} form will be an update form that works
for that particular implementation, and that the left-to-right evaluation
of subforms is preserved.

The ultimate result of evaluating a \cdf{setf} form is the value
of \emph{newvalue}.  Therefore \cd{(setf (car x) y)} does not expand
into precisely \cd{(rplaca x y)}, but into something more like
\begin{lisp}
(let ((G1 x) (G2 y)) (rplaca G1 G2) G2)
\end{lisp}
the precise expansion being implementation-dependent.

The user can define new \cdf{setf} expansions by using \cdf{defsetf}.

\begin{newer}
X3J13 voted in June 1989 \issue{SETF-MULTIPLE-STORE-VARIABLES}
to extend the specification of \cdf{setf} to allow a \emph{place}
whose \cdf{setf} method has more than one store variable (see \cdf{define-setf-method}).
In such a case as many values are accepted from the \emph{newvalue} form
as there are store variables; extra values are ignored
and missing values default to \cdf{nil},
as is usual in situations involving multiple values.

A proposal was submitted to X3J13 in September 1989
to add a \cdf{setf} method for \cdf{values} so that one could
in fact write, for example,
\begin{lisp}
(setf (values quotient remainder) \\
~~~~~~(truncate linewidth tabstop))
\end{lisp}
but unless this proposal is accepted users will have to
define a \cdf{setf} method for \cdf{values} themselves (not a difficult task).
\end{newer}
\end{defmac}

\begin{defmac}
psetf {place newvalue}*

\cdf{psetf} is like \cdf{setf} except that if more than one \emph{place}-\emph{newvalue}
pair is specified, then the assignments of new values to places are
done in parallel.  More precisely, all subforms that are to be evaluated
are evaluated from left to right; after all evaluations have been performed,
all of the assignments are performed in an unpredictable order.
(The unpredictability matters only if more than one \emph{place} form
refers to the same place.)
\cdf{psetf} always returns {\false}.

\begin{newer}
X3J13 voted in June 1989 \issue{SETF-MULTIPLE-STORE-VARIABLES}
to extend the specification of \cdf{psetf} to allow a \emph{place}
whose \cdf{setf} method has more than one store variable (see \cdf{define-setf-method}).
In such a case as many values are accepted from the \emph{newvalue} form
as there are store variables; extra values are ignored
and missing values default to \cdf{nil},
as is usual in situations involving multiple values.
\end{newer}
\end{defmac}

\begin{defmac}
shiftf {place}+ newvalue

Each \emph{place} form may be any form acceptable
as a generalized variable to \cdf{setf}.
In the form \cd{(shiftf \emph{place1} \emph{place2} ... \emph{placen} \emph{newvalue})},
the values in \emph{place1} through \emph{placen} are accessed and saved,
and \emph{newvalue} is evaluated, for a total of $\emph{n}+1$ values in all.
Values 2 through $\emph{n}+1$ are then stored into \emph{place1} through \emph{placen},
and value 1 (the original value of \emph{place1}) is returned.
It is as if all the places form a shift register; the \emph{newvalue}
is shifted in from the right, all values shift over to the left one place,
and the value shifted out of \emph{place1} is returned.  For example:
\begin{lisp}
(setq x (list 'a 'b 'c)) \EV\ (a b c) \\
 \\
(shiftf (cadr x) 'z) \EV\ b \\
~~~\textrm{and now} x \EV\ (a z c) \\
 \\
(shiftf (cadr x) (cddr x) 'q) \EV\ z \\
~~~\textrm{and now} x \EV\ (a (c) . q)
\end{lisp}
The effect of \cd{(shiftf \emph{place1} \emph{place2} ... \emph{placen} \emph{newvalue})}
is equivalent to
\begin{lisp}
(let ((\emph{var1} \emph{place1}) \\
~~~~~~(\emph{var2} \emph{place2}) \\
~~~~~~... \\
~~~~~~(\emph{varn} \emph{placen})) \\
~~(setf \emph{place1} \emph{var2}) \\
~~(setf \emph{place2} \emph{var3}) \\
~~... \\
~~(setf \emph{placen} \emph{newvalue}) \\
~~\emph{var1})
\end{lisp}
except that the latter would evaluate any subforms of each \emph{place} twice,
whereas \cdf{shiftf} takes care to evaluate them only once.
For example:
\begin{lisp}
(setq n 0) \\
(setq x '(a b c d)) \\
(shiftf (nth (setq n (+ n 1)) x) 'z) \EV\ b \\
~~~\textrm{and now} x \EV\ (a z c d) \\[4pt]
\emph{but} \\[4pt]
(setq n 0) \\
(setq x '(a b c d)) \\
(prog1 (nth (setq n (+ n 1)) x) \\*
~~~~~~~(setf (nth (setq n (+ n 1)) x) 'z)) \EV\ b \\
~~~\textrm{and now} x \EV\ (a b z d)
\end{lisp}
Moreover, for certain \emph{place} forms \cdf{shiftf} may be
significantly more efficient than the \cdf{prog1} version.

\begin{newer}
X3J13 voted in June 1989 \issue{SETF-MULTIPLE-STORE-VARIABLES}
to extend the specification of \cdf{shiftf} to allow a \emph{place}
whose \cdf{setf} method has more than one store variable (see \cdf{define-setf-method}).
In such a case as many values are accepted from the \emph{newvalue} form
as there are store variables; extra values are ignored
and missing values default to \cdf{nil},
as is usual in situations involving multiple values.
\end{newer}

\beforenoterule
\begin{rationale}
\cdf{shiftf} and \cdf{rotatef} have been included in Common Lisp
as generalizations of two-argument versions formerly called \cdf{swapf}
and \cdf{exchf}.  The two-argument versions have been found to be
very useful, but the names were easily confused.  The generalization
to many argument forms and the change of names were both inspired
by the work of Suzuki \cite{SUZUKI-POINTER-ROTATION},
which indicates that use of these primitives can make certain complex
pointer-manipulation programs clearer and easier to prove correct.
\end{rationale}
\afternoterule
\end{defmac}

\begin{defmac}
rotatef {place}*

Each \emph{place} form may be any form acceptable
as a generalized variable to \cdf{setf}.
In the form \cd{(rotatef \emph{place1} \emph{place2} ... \emph{placen})},
the values in \emph{place1} through \emph{placen} are accessed and saved.
Values 2 through \emph{n} and value 1 are then stored into \emph{place1} through \emph{placen}.
It is as if all the places form an end-around shift register
that is rotated one place to the left, with the value of \emph{place1}
being shifted around the end to \emph{placen}.
Note that \cd{(rotatef \emph{place1} \emph{place2})} exchanges the contents
of \emph{place1} and \emph{place2}.

The effect of \cd{(rotatef \emph{place1} \emph{place2} ... \emph{placen})}
is roughly equivalent to
\begin{lisp}
(psetf \emph{place1} \emph{place2} \\
~~~~~~~\emph{place2} \emph{place3} \\
~~~~~~~... \\
~~~~~~~\emph{placen} \emph{place1})
\end{lisp}
except that the latter would evaluate any subforms of each \emph{place} twice,
whereas \cdf{rotatef} takes care to evaluate them only once.
Moreover, for certain \emph{place} forms \cdf{rotatef} may be
significantly more efficient.

\cdf{rotatef} always returns {\false}.

\begin{newer}
X3J13 voted in June 1989 \issue{SETF-MULTIPLE-STORE-VARIABLES}
to extend the specification of \cdf{rotatef} to allow a \emph{place}
whose \cdf{setf} method has more than one store variable (see \cdf{define-setf-method}).
In such a case as many values are accepted from the \emph{newvalue} form
as there are store variables; extra values are ignored
and missing values default to \cdf{nil},
as is usual in situations involving multiple values.
\end{newer}
\end{defmac}

Other macros that manipulate generalized variables include
\cdf{getf}, \cdf{remf},
\cdf{incf}, \cdf{decf}, \cdf{push}, \cdf{pop},
\cdf{assert}, \cdf{ctypecase}, and \cdf{ccase}.

Macros that manipulate generalized variables must guarantee the ``obvious''
semantics:  subforms of generalized-variable references are
evaluated exactly as many times as they appear in the source program, and
they are evaluated in exactly the same order as they appear in the source
program.

In generalized-variable references such as \cdf{shiftf}, \cdf{incf}, \cdf{push},
and \cdf{setf} of \cdf{ldb}, the generalized variables are both read and
written in the same reference.   Preserving the source program order of
evaluation and the number of evaluations is particularly important.

As an example of these semantic rules, in the generalized-variable
reference \cd{(setf \emph{reference} \emph{value})} the \emph{value} form
must be evaluated \emph{after} all the subforms of the reference because
the \emph{value} form appears to the right of them.

The expansion of these macros must consist of code that follows these
rules or has the same effect as such code.  This is accomplished by
introducing temporary variables bound to the subforms of the reference.
As an optimization in the implementation,
temporary variables may be eliminated whenever it
can be proved that removing them has no effect on the semantics of the program.
For example, a constant need never be saved in a temporary variable.
A variable, or for that matter any form that does not have side effects, need not be
saved in a temporary variable if it can be proved that its value will
not change within the scope of the generalized-variable reference.

Common Lisp provides built-in facilities to take care of
these semantic complications and optimizations.  Since the required
semantics can be guaranteed by these facilities, the user does not
have to worry about writing correct code for them, especially in
complex cases.  Even experts can become confused and make mistakes
while writing this sort of code.

\begin{newer}
X3J13 voted in March 1988 \issue{PUSH-EVALUATION-ORDER}
to clarify the preceding discussion about the order of evaluation of
subforms in calls to \cdf{setf} and related macros.
The general intent is clear: evaluation
proceeds from left to right whenever possible. However, the left-to-right rule does not
remove the obligation on writers of macros and \cdf{define-setf-method} to work
to ensure left-to-right order of evaluation.

Let it be emphasized that, in the following discussion,
a \emph{form} is something whose syntactic use is such that it will
be evaluated.  A \emph{subform} means a form that is nested inside another form,
not merely any Lisp object nested inside a form regardless of syntactic context. 

The evaluation ordering of subforms within a generalized variable
reference is determined by the order specified by the second value returned by
\cdf{get-setf-method}.  For all predefined generalized variable references
(\cdf{getf}, \cdf{ldb}), this order of evaluation is exactly left-to-right.
When a generalized
variable reference is derived from a macro expansion, this rule is applied
\emph{after} the macro is expanded to find the appropriate generalized variable
reference. 

This is intended to make it clear that if the user writes a \cdf{defmacro} or
\cdf{define-setf-method} macro that doesn't preserve left-to-right
evaluation order, the order specified in the
user's code holds.  For example, given
\begin{lisp}
(defmacro wrong-order (x y) {\Xbq}(getf ,y ,x))
\end{lisp}
then
\begin{lisp}
(push \emph{value} (wrong-order \emph{place1} \emph{place2}))
\end{lisp}
will evaluate \emph{place2} first and then \emph{place1} because that is the order they
are evaluated in the macro expansion.
 
For the macros that manipulate generalized variables (\cdf{push}, \cdf{pushnew}, \cdf{getf},
\cdf{remf}, \cdf{incf}, \cdf{decf}, \cdf{shiftf}, \cdf{rotatef},
\cdf{psetf}, \cdf{setf}, \cdf{pop}, and those defined with
\cdf{define-modify-macro}) the subforms of the macro call are evaluated exactly once
in left-to-right order, with the subforms of the generalized variable
references evaluated in the order specified above.

Each of
\cdf{push}, \cdf{pushnew}, \cdf{getf}, \cdf{remf}, \cdf{incf}, \cdf{decf},
\cdf{shiftf}, \cdf{rotatef}, \cdf{psetf}, and \cdf{pop} evaluates
all subforms before modifying any of the generalized variable locations.  Moreover,
\cdf{setf} itself,
in the case when a call on it has more than two arguments, performs its
operation on each pair in sequence.  That is, in
\begin{lisp}
(setf \emph{place1} \emph{value1} \emph{place2} \emph{value2} ...)
\end{lisp}
the subforms of \emph{place1} and \emph{value1} are evaluated, the
location specified by \emph{place1} is modified to contain the value returned by
\emph{value1}, and then the rest of the \cdf{setf} form is processed in a like manner.

For the macros \cdf{check-type}, \cdf{ctypecase}, and \cdf{ccase}, subforms of the
generalized variable reference are evaluated once per test of a generalized
variable, but they may be
evaluated again if the type check fails (in the case of \cdf{check-type}) or if none of
the cases holds (in \cdf{ctypecase} or \cdf{ccase}).

For the macro \cdf{assert}, the order of evaluation of the generalized variable
references is not specified.
\end{newer}

Another reason for building in these functions is that the
appropriate optimizations will differ from implementation to
implementation.  In some implementations most of the optimization is
performed by the compiler, while in others a simpler compiler is used and
most of the optimization is performed in the macros.  The cost of
binding a temporary variable relative to the cost of other Lisp
operations may differ greatly between one implementation
and another, and some implementations may find it
best never to remove temporary variables except in the simplest cases.

A good example of the issues involved can be seen in the following
generalized-variable reference:
\begin{lisp}
(incf (ldb byte-field variable))
\end{lisp}
This ought to expand into something like
\begin{lisp}
(setq variable \\
~~~~~~(dpb (1+ (ldb byte-field variable)) \\
~~~~~~~~~~~byte-field \\
~~~~~~~~~~~variable))
\end{lisp}
In this expansion example we have
ignored the further complexity of returning the correct
value, which is the incremented byte, not the new value of \cdf{variable}.
Note that the variable \cdf{byte-field} is evaluated twice, and the
variable \cdf{variable} is referred to three times:
once as the location in which to store a value,
and twice during the computation of that value.

Now consider this expression:
\begin{lisp}
(incf (ldb (aref byte-fields (incf i)) \\
~~~~~~~~~~~(aref (determine-words-array) i)))
\end{lisp}
It ought to expand into something like this:
\begin{lisp}
(let ((temp1 (aref byte-fields (incf i))) \\
~~~~~~(temp2 (determine-words-array))) \\
~~(setf (aref temp2 i) \\
~~~~~~~~(dpb (1+ (ldb temp1 (aref temp2 i))) \\
~~~~~~~~~~~~~temp1 \\
~~~~~~~~~~~~~(aref temp2 i))))
\end{lisp}
Again we have ignored the complexity of returning the correct value.
What is important here is that the expressions \cd{(incf i)}
and \cd{(determine-words-array)}
must not be duplicated because each may have a side effect or
be affected by side effects.

\begin{newer}
X3J13 voted in January 1989 \issue{SETF-SUB-METHODS}
to specify more precisely the order of evaluation of subforms
when \cdf{setf} is used with an access function that itself
takes a \emph{place} as an argument, for example, \cdf{ldb}, \cdf{mask-field}, and
\cdf{getf}.  (The vote also discussed the function \cdf{char-bit},
but another vote \issue{CHARACTER-PROPOSAL} removed that function
from the language.)  The \cdf{setf} methods for such accessors produce expansions
that effectively require explicit calls to \cdf{get-setf-method}.

The code produced as the macro expansion of a \cdf{setf} form that
itself admits a generalized variable as an argument must essentially
do the following major steps:
\begin{itemize}
\item It evaluates the value-producing subforms, in left-to-right order, and 
     binds the temporary variables to them; this is called \emph{binding the temporaries}.

\item It reads the value from the generalized variable, using the supplied 
     accessing form, to get the old value;  this is called \emph{doing the
     access}.  Note that this is done after all the evaluations of the 
     preceding step, including any side effects they may have.

\item It binds the store variable to a new value, and then installs this
     new value into the generalized variable using the supplied storing 
     form; this is called \emph{doing the store}.
\end{itemize}
Doing the access for a generalized variable reference is not part of
the series of evaluations that must be done in left-to-right order. 

The place-specifier forms \cdf{ldb}, \cdf{mask-field}, and \cdf{getf} admit (other)
\emph{place} specifiers as arguments. During the \cdf{setf} expansion of these forms, it 
is necessary to call \cdf{get-setf-method} to determine how the inner, 
nested generalized variable must be treated.

    In a form such as
\begin{lisp}
(setf (ldb \emph{byte-spec} \emph{place-form}) \emph{newvalue-form})
\end{lisp}
    the place referred to by the \emph{place-form} must always be both accessed 
    and updated;  note that the update is to the generalized variable 
    specified by \emph{place-form}, not to any object of type \cdf{integer}.

    Thus this call to \cdf{setf} should generate code to do the following:
\begin{itemize}
    \item Evaluate \emph{byte-spec} and bind into a temporary
    \item Bind the temporaries for \emph{place-form}
    \item Evaluate \emph{newvalue-form} and bind into the store variable
    \item Do the access to \emph{place-form}
    \item Do the store into \emph{place-form} with the given bit-field of the
          accessed integer replaced with the value in the store variable
\end{itemize}
    If the evaluation of \emph{newvalue-form} alters what is found in the 
    given \emph{place}---such as setting a different bit-field of the
    integer---then the change of the bit-field denoted by
    \emph{byte-spec} will be to that 
    altered integer, because the access step must be done after the \emph{newvalue-form}
    evaluation.  Nevertheless, the 
    evaluations required for binding the temporaries are done before the
    evaluation of the \emph{newvalue-form}, thereby preserving
    the required left-to-right evaluation order.

The treatment of \cdf{mask-field} is similar to that of \cdf{ldb}.

    In a form such as:
\begin{lisp}
(setf (getf \emph{place-form} \emph{ind-form}) \emph{newvalue-form})
\end{lisp}
    the place referred to by the \emph{place-form} must always be both accessed 
    and updated;  note that the update is to the generalized variable 
    specified by \emph{place-form}, not necessarily to the particular list
    which is the property list in question.

    Thus this call to \cdf{setf} should generate code to do the following:
\begin{itemize}
    \item Bind the temporaries for \emph{place-form} 
    \item Evaluate \emph{ind-form} and bind into a temporary
    \item Evaluate the \emph{newvalue-form} and bind into the store variable
    \item Do the access to \emph{place-form}
    \item Do the store into \emph{place-form} with a possibly new property list
       obtained by combining the results of the evaluations and the access
\end{itemize}

    If the evaluation of \emph{newvalue-form} alters what is found in the 
    given \emph{place}---such as setting a different named property in the
    list---then the change of the property denoted by \emph{ind-form} will be to that 
    altered list, because the access step is done after the \emph{newvalue-form}
    evaluation.   Nevertheless, the 
    evaluations required for binding the temporaries are done before the
    evaluation of the \emph{newvalue-form}, thereby preserving
    the required left-to-right evaluation order.

    Note that the phrase ``possibly new property list'' treats the 
    implementation of property lists as a ``black box''; it can mean that 
    the former property list is somehow destructively re-used, or it can 
    mean partial or full copying of it.  A side effect may or may not occur;
    therefore \cdf{setf} must proceed as if the resultant property list
    were a different copy
    needing to be stored back into the generalized variable.
\end{newer}

The Common Lisp facilities provided to deal with these semantic issues include:
\begin{itemize}
\item
Built-in macros such as \cdf{setf} and \cdf{push} that follow the semantic rules.

\item
The \cdf{define-modify-macro} macro, which allows new generalized-variable
manipulating macros (of a certain restricted kind) to be defined easily.
It takes care of the semantic rules automatically.

\item
The \cdf{defsetf} macro, which allows new types of generalized-variable references
to be defined easily.  It takes care of the semantic rules automatically.

\item
The \cdf{define-setf-method} macro and the \cdf{get-setf-method} function, which
provide access to the internal mechanisms when it is necessary
to define a complicated new type of generalized-variable reference
or generalized-variable-manipulating macro.
\end{itemize}

\begin{newer}
Also important are the changes that allow lexical environments to be
used in appropriate ways in \cdf{setf} methods.
\end{newer}

\begin{defmac}
define-modify-macro name lambda-list function [doc-string]

This macro defines a read-modify-write macro
named \emph{name}.  An example of such a macro is \cdf{incf}.  The first
subform of the macro will be a generalized-variable reference.
The \emph{function} is literally the function to apply to the old contents of the
generalized-variable to get the new contents; it is not evaluated.
\emph{lambda-list} describes
the remaining arguments for the \emph{function}; these arguments come from
the remaining subforms of the macro after the generalized-variable reference.
\emph{lambda-list} may contain \cd{\&optional} and \cd{\&rest} markers.
(The \cd{\&key} marker is not permitted here; \cd{\&rest} suffices for the purposes
of \cdf{define-modify-macro}.)
\emph{doc-string} is documentation for the macro \emph{name} being defined.

The expansion of a \cdf{define-modify-macro} is equivalent to the following, except
that it generates code that follows the semantic rules outlined above.
\begin{lisp}
(defmacro \emph{name} (\emph{reference} . \emph{lambda-list}) \\
~~\emph{doc-string} \\
~~{\Xbq}(setf ,\emph{reference} \\
~~~~~~~~~(\emph{function} ,\emph{reference} ,\emph{arg1} ,\emph{arg2} ...)))
\end{lisp}
where \emph{arg1}, \emph{arg2}, ..., are the parameters appearing in \emph{lambda-list};
appropriate provision is made for a \cd{\&rest} parameter.

As an example, \cdf{incf} could have been defined by:
\begin{lisp}
(define-modify-macro incf (\&optional (delta 1)) +)
\end{lisp}

An example of a possibly useful macro not predefined in Common Lisp is
\begin{lisp}
(define-modify-macro unionf (other-set \&rest keywords) union)
\end{lisp}

\begin{newer}
X3J13 voted in March 1988 \issue{GET-SETF-METHOD-ENVIRONMENT}
to specify that \cdf{define-modify-macro} creates macros that
take \cd{\&environment} arguments and perform the
equivalent of correctly passing such lexical
environments to \cdf{get-setf-method} in order to correctly maintain 
lexical references.
\end{newer}
\end{defmac}

\begin{defmac}
defsetf access-fn {update-fn [doc-string] | lambda-list (store-variable) <{declaration}* | doc-string> {\,form}*}

This defines how to \cdf{setf} a generalized-variable reference
of the form \cd{(\emph{access-fn} ...)}.  The value of a generalized-variable
reference can always be obtained simply by evaluating it, so \emph{access-fn}
should be the name of a function or a macro.

The user of \cdf{defsetf} provides a description of how to store into the
generalized-variable reference and return the value that was stored (because
\cdf{setf} is defined to return this value).  The implementation
of \cdf{defsetf} takes care of
ensuring that subforms of the reference are evaluated exactly once and
in the proper left-to-right order.  In order to do this,
\cdf{defsetf} requires that \emph{access-fn} be a function or a macro
that evaluates its arguments, behaving like a function.
Furthermore, a \cdf{setf} of a call on \emph{access-fn} will also evaluate
all of \emph{access-fn}'s arguments; it cannot treat any of them specially.
This means that \cdf{defsetf} cannot be used to describe how to store into
a generalized variable that is a byte, such as \cd{(ldb field reference)}.
To handle situations that do not fit the restrictions imposed by \cdf{defsetf},
use \cdf{define-setf-method}, which gives the user additional control
at the cost of increased complexity.

A \cdf{defsetf} declaration may take one of two forms.
The simple form is
\begin{lisp}
(defsetf \emph{access-fn} \emph{update-fn} \Mopt{\emph{doc-string}})
\end{lisp}
The \emph{update-fn} must name a function (or macro) that takes one more argument
than \emph{access-fn} takes.  When \cdf{setf} is given a \emph{place}
that is a call on \emph{access-fn}, it expands into
a call on \emph{update-fn} that is given all the arguments to
\emph{access-fn} and also, as its last argument, the new value
(which must be returned by \emph{update-fn} as its value).
For example, the effect of
\begin{lisp}
(defsetf symbol-value set)
\end{lisp}
is built into the Common Lisp system.
This causes the expansion
\begin{lisp}
(setf (symbol-value foo) fu) \EX\ (set foo fu)
\end{lisp}
for example.  Note that
\begin{lisp}
(defsetf car rplaca)
\end{lisp}
would be incorrect because \cdf{rplaca} does not return its last argument.

The complex form of \cdf{defsetf} looks like
\begin{lisp}
(defsetf \emph{access-fn} \emph{lambda-list} (\emph{store-variable}) . \emph{body})
\end{lisp}
and resembles \cdf{defmacro}.  The \emph{body} must
compute the expansion of a \cdf{setf} of a call on \emph{access-fn}.

The \emph{lambda-list} describes the arguments of \emph{access-fn}.  \cd{\&optional},
\cd{\&rest}, and \cd{\&key} markers are permitted in \emph{lambda-list}.
Optional arguments may
have defaults and ``supplied-p'' flags.  The \emph{store-variable} describes the
value to be stored into the generalized-variable reference.

\beforenoterule
\begin{rationale}
The \emph{store-variable} is enclosed
in parentheses to provide for an extension
to multiple store variables that would
receive multiple values from the second subform of \cdf{setf}.
The rules given below for coding \cdf{setf} methods discuss
the proper handling of multiple store variables to allow for
the possibility that this extension may be incorporated into Common Lisp
in the future.
\end{rationale}
\afternoterule

The \emph{body} forms can be written as if the variables in the \emph{lambda-list}
were bound to subforms of the call on \emph{access-fn} and the
\emph{store-variable} were bound to the second subform of \cdf{setf}.
However, this is not actually the case.  During the evaluation of the
\emph{body} forms, these variables are bound to names of temporary variables,
generated as if by \cdf{gensym} or \cdf{gentemp},
that will be bound by the
expansion of \cdf{setf} to the values of those subforms.  This binding
permits the
\emph{body} forms to be written without regard for order-of-evaluation
issues.  \cdf{defsetf} arranges for the temporary variables to be
optimized out of the final result in cases where that is possible.  In
other words, an attempt is made by \cdf{defsetf} to generate
the best code possible in a particular implementation.

Note that the code generated by the \emph{body} forms must include provision
for returning the correct value (the value of \emph{store-variable}).  This is
handled by the \emph{body} forms rather than by \cdf{defsetf} because
in many cases this value can be returned at no extra cost, by calling a
function that simultaneously stores into the generalized variable and
returns the correct value.

An example of the use of the complex form of \cdf{defsetf}:
\begin{lisp}
(defsetf subseq (sequence start \&optional end) (new-sequence) \\
~~{\Xbq}(progn (replace ,sequence ,new-sequence \\
~~~~~~~~~~~~~~~~~~~:start1 ,start :end1 ,end) \\
~~~~~~~~~~,new-sequence))
\end{lisp}

\begin{newer}
X3J13 voted in March 1988 \issue{FLET-IMPLICIT-BLOCK}
to specify that the body of the expander function defined
by the complex form of \cdf{defsetf} is implicitly enclosed in a \cdf{block} construct
whose name is the same as the \emph{name} of the \emph{access-fn}.
Therefore \cdf{return-from} may be used to exit from the function.
\end{newer}

\begin{newer}
X3J13 voted in March 1989 \issue{DEFINING-MACROS-NON-TOP-LEVEL}
to clarify that, while defining forms normally appear at top level,
it is meaningful to place them in non-top-level contexts; the complex form of
\cdf{defsetf} must define the expander function
within the enclosing lexical environment, not within the global
environment.
\end{newer}
\end{defmac}

The underlying theory by which \cdf{setf} and related macros arrange to
conform to the semantic rules given above is that from any
generalized-variable reference one may derive its ``\cdf{setf} method,''
which describes how to store into that reference and which subforms of
it are evaluated.

\beforenoterule
\begin{incompatibility}
To avoid confusion,
it should be noted that the use of the word ``method'' here in connection
with \cdf{setf} has nothing to do with its use in Lisp Machine Lisp in connection
with message-passing and the Lisp Machine Lisp ``flavor system.''
\begin{new}
And of course it also has nothing to do with the methods in
the Common Lisp Object System
\issue{CLOS}.
\end{new}
\end{incompatibility}
\afternoterule

Given knowledge of the subforms of the reference,
it is possible to avoid evaluating them multiple times or in the wrong
order.  A \cdf{setf} method for a given access form can be expressed as
five values:
\begin{itemize}
\item
A list of \emph{temporary variables}

\item
A list of \emph{value forms} (subforms of the given form)
to whose values the temporary variables are to be bound

\item
A second list of temporary variables, called \emph{store variables}

\item
A \emph{storing form}

\item
An \emph{accessing form}
\end{itemize}

The temporary variables will be bound to the values of
the value forms as if by \cdf{let*}; that is, the
value forms will be evaluated in the order given
and may refer to the values of earlier value forms
by using the corresponding variables.

The store variables are to be bound to the values of the \emph{newvalue} form,
that is, the values to be
stored into the generalized variable.  In almost all cases only a
single value is to be stored, and there is only one store variable.

The storing form and the accessing form may contain references to
the temporary variables (and also, in the case of the storing form,
to the store variables).  The accessing form returns the value of the
generalized variable.  The storing form modifies the value of the
generalized variable and guarantees to return the values of the
store variables as
its values; these are the correct values for \cdf{setf} to
return.  (Again, in most cases there is a single store variable
and thus a single value to be returned.)
The value returned by the accessing form is, of course,
affected by execution of the storing form, but either of these
forms may be evaluated any number of times and therefore should be
free of side effects (other than the storing action of the storing form).

The temporary variables and the store variables are generated names,
as if by \cdf{gensym} or \cdf{gentemp},
so that there is never any problem of name clashes among them, or
between them and other variables in the program.  This is necessary to
make the special forms that do more than one \cdf{setf} in parallel work
properly; these are \cdf{psetf}, \cdf{shiftf}, and \cdf{rotatef}.  Computation
of the \cdf{setf} method must always create new variable names; it may not return
the same ones every time.

Some examples of \cdf{setf} methods for particular forms:
\begin{itemize}
\item
For a variable \cdf{x}:
\begin{lisp}
() \\*
() \\*
(g0001) \\
(setq x g0001) \\*
x
\end{lisp}

\item
For \cd{(car \emph{exp})}:
\begin{lisp}
(g0002) \\*
(\emph{exp}) \\*
(g0003)  \\
(progn (rplaca g0002 g0003) g0003) \\*
(car g0002)
\end{lisp}

\item
For \cd{(subseq \emph{seq} \emph{s} \emph{e})}:
\begin{lisp}
(g0004 g0005 g0006) \\*
(\emph{seq} \emph{s} \emph{e}) \\*
(g0007) \\
(progn (replace g0004 g0007 :start1 g0005 :end1 g0006) \\*
~~~~~~~g0007) \\*
(subseq g0004 g0005 g0006)
\end{lisp}
\end{itemize}

\begin{defmac}
define-setf-method access-fn lambda-list <{declaration}* | doc-string> {\,form}*

This defines how
to \cdf{setf} a generalized-variable reference that is of the form
\cd{(\emph{access-fn}...)}.  The value of a generalized-variable reference can
always be obtained simply by evaluating it, so \emph{access-fn} should be the
name of a function or a macro.

The \emph{lambda-list} describes the subforms of the generalized-variable
reference, as with \cdf{defmacro}.  The result of evaluating the
\emph{forms} in the body must be five values representing
the \cdf{setf} method, as described
above.  Note that \cdf{define-setf-method} differs from the complex form of
\cdf{defsetf} in that while the body is being executed the variables in
\emph{lambda-list} are bound to parts of the generalized-variable reference,
not to temporary variables that will be bound to the values of such parts.
In addition, \cdf{define-setf-method} does not have \cdf{defsetf}'s
restriction that \emph{access-fn} must be a function or a function-like
macro; an arbitrary \cdf{defmacro} destructuring pattern is permitted in
\emph{lambda-list}.

By definition there are no good small examples of \cdf{define-setf-method}
because the easy cases can all be handled by \cdf{defsetf}.
A typical use is to define the \cdf{setf} method for \cdf{ldb}:
\begin{obsolete}
\begin{lisp}
;;; SETF method for the form (LDB bytespec int). \\*
;;; Recall that the int form must itself be suitable for SETF. \\
(define-setf-method ldb (bytespec int) \\*
~~(multiple-value-bind (temps vals stores \\*
~~~~~~~~~~~~~~~~~~~~~~~~store-form access-form) \\*
~~~~~~(get-setf-method int)~~~~~~~~~;Get SETF method for int \\
~~~~(let ((btemp (gensym))~~~~~~~~~~;Temp var for byte specifier \\*
~~~~~~~~~~(store (gensym))~~~~~~~~~~;Temp var for byte to store \\*
~~~~~~~~~~(stemp (first stores)))~~~;Temp var for int to store \\
~~~~~~;; Return the SETF method for LDB as five values. \\*
~~~~~~(values (cons btemp temps)~~~~;Temporary variables \\*
~~~~~~~~~~~~~~(cons bytespec vals)~~;Value forms \\*
~~~~~~~~~~~~~~(list store)~~~~~~~~~~;Store variables \\
~~~~~~~~~~~~~~{\Xbq}(let ((,stemp (dpb ,store ,btemp ,access-form))) \\*
~~~~~~~~~~~~~~~~~,store-form \\*
~~~~~~~~~~~~~~~~~,store)~~~~~~~~~~~~~~~~~~~~~;Storing form \\*
~~~~~~~~~~~~~~{\Xbq}(ldb ,btemp ,access-form)~~~~~;Accessing form \\*
~~~~~~~~~~~~~~))))
\end{lisp}
\end{obsolete}

\begin{newer}
X3J13 voted in March 1988 \issue{GET-SETF-METHOD-ENVIRONMENT}
to specify that the \cd{\&environment} lambda-list keyword may appear in the
\emph{lambda-list} in the same manner as for \cdf{defmacro} in order
to obtain the lexical environment of the call to the \cdf{setf} macro.
The preceding example should be modified to take advantage of
this new feature.  The \cdf{setf} method must accept an \cd{\&environment}
parameter, which will receive the lexical environment of the call to \cdf{setf};
this environment must then be given to \cdf{get-setf-method} in order
that it may correctly use any locally bound \cdf{setf} method that
might be applicable to the \emph{place} form that appears as the second
argument to \cdf{ldb} in the call to \cdf{setf}.

\begin{lisp}
;;; SETF method for the form (LDB bytespec int). \\*
;;; Recall that the int form must itself be suitable for SETF. \\*
;;; Note the use of an \&environment parameter to receive the \\*
;;; lexical environment of the call for use with GET-SETF-METHOD. \\*
(define-setf-method ldb (bytespec int \&environment env) \\*
~~(multiple-value-bind (temps vals stores \\*
~~~~~~~~~~~~~~~~~~~~~~~~store-form access-form) \\*
~~~~~~(get-setf-method int env)~~~~~;Get SETF method for int \\
~~~~(let ((btemp (gensym))~~~~~~~~~~;Temp var for byte specifier \\*
~~~~~~~~~~(store (gensym))~~~~~~~~~~;Temp var for byte to store \\*
~~~~~~~~~~(stemp (first stores)))~~~;Temp var for int to store \\
~~~~~~;; Return the SETF method for LDB as five values. \\*
~~~~~~(values (cons btemp temps)~~~~;Temporary variables \\*
~~~~~~~~~~~~~~(cons bytespec vals)~~;Value forms \\*
~~~~~~~~~~~~~~(list store)~~~~~~~~~~;Store variables \\
~~~~~~~~~~~~~~{\Xbq}(let ((,stemp (dpb ,store ,btemp ,access-form))) \\*
~~~~~~~~~~~~~~~~~,store-form \\*
~~~~~~~~~~~~~~~~~,store)~~~~~~~~~~~~~~~~~~~~~;Storing form \\*
~~~~~~~~~~~~~~{\Xbq}(ldb ,btemp ,access-form)~~~~~;Accessing form \\*
~~~~~~~~~~~~~~))))
\end{lisp}
\end{newer}

\begin{newer}
X3J13 voted in March 1988 \issue{FLET-IMPLICIT-BLOCK}
to specify that the body of the expander function defined
by \cdf{define-setf-method} is implicitly enclosed in a \cdf{block} construct
whose name is the same as the \emph{name} of the \emph{access-fn}.
Therefore \cdf{return-from} may be used to exit from the function.
\end{newer}

\begin{newer}
X3J13 voted in March 1989 \issue{DEFINING-MACROS-NON-TOP-LEVEL}
to clarify that, while defining forms normally appear at top level,
it is meaningful to place them in non-top-level contexts;
\cdf{define-setf-method} must define the expander function
within the enclosing lexical environment, not within the global
environment.
\end{newer}
\end{defmac}

\begin{obsolete}
\begin{defun}[Function]
get-setf-method form

\cdf{get-setf-method} returns
five values constituting the \cdf{setf} method for \emph{form}.
The \emph{form} must be a
generalized-variable reference.  \cdf{get-setf-method} takes care of
error-checking and macro expansion and guarantees to return exactly one
store variable.

As an example, an extremely simplified version of \cdf{setf},
allowing no more and no fewer than two
subforms, containing no optimization to remove unnecessary variables, and
not allowing storing of multiple values, could be defined by:
\begin{lisp}
(defmacro setf (reference value) \\*
~~(multiple-value-bind (vars vals stores store-form access-form) \\*
~~~~~~(get-setf-method reference) \\
~~~~(declare (ignore access-form)) \\
~~~~{\Xbq}(let* ,(mapcar \#'list \\*
~~~~~~~~~~~~~~~~~~~~(append vars stores) \\*
~~~~~~~~~~~~~~~~~~~~(append vals (list value))) \\*
~~~~~~~,store-form)))
\end{lisp}
\end{defun}
\end{obsolete}

\begin{newer}
X3J13 voted in March 1988 \issue{GET-SETF-METHOD-ENVIRONMENT}
to add an optional environment argument to \cdf{get-setf-method}.
The revised definition and example are as follows.

\begin{defun}[Function]
get-setf-method form &optional env

\cdf{get-setf-method} returns
five values constituting the \cdf{setf} method for \emph{form}.
The \emph{form} must be a
generalized-variable reference.
The \emph{env} must be an environment of the sort obtained through
the \cd{\&environment} lambda-list keyword; if \emph{env} is \cdf{nil} or omitted,
the null lexical environment is assumed.
\cdf{get-setf-method} takes care of
error checking and macro expansion and guarantees to return exactly one
store variable.

As an example, an extremely simplified version of \cdf{setf},
allowing no more and no fewer than two
subforms, containing no optimization to remove unnecessary variables, and
not allowing storing of multiple values, could be defined by:
\begin{lisp}
(defmacro setf (reference value \&environment env) \\*
~~(multiple-value-bind (vars vals stores store-form access-form) \\*
~~~~~~(get-setf-method reference env)~~~~~;\textrm{Note use of environment}\\
~~~~(declare (ignore access-form)) \\
~~~~{\Xbq}(let* ,(mapcar \#'list \\*
~~~~~~~~~~~~~~~~~~~~(append vars stores) \\*
~~~~~~~~~~~~~~~~~~~~(append vals (list value))) \\*
~~~~~~~,store-form)))
\end{lisp}
\end{defun}
\end{newer}

\begin{obsolete}
\begin{defun}[Function]
get-setf-method-multiple-value form

\cdf{get-setf-method-multiple-value}
returns five values constituting the \cdf{setf} method for \emph{form}.
The \emph{form} must be a
generalized-variable reference.  This is the same as \cdf{get-setf-method}
except that it does not check the number of store variables; use this
in cases that allow storing multiple values into a generalized variable.
There are no such cases in standard Common Lisp, but this function is provided
to allow for possible extensions.
\end{defun}
\end{obsolete}

\begin{newer}
X3J13 voted in March 1988 \issue{GET-SETF-METHOD-ENVIRONMENT}
to add an optional environment argument to \cdf{get-setf-method}.
The revised definition is as follows.

\begin{defun}[Function]
get-setf-method-multiple-value form &optional env

\cdf{get-setf-method-multiple-value}
returns five values constituting the \cdf{setf} method for \emph{form}.
The \emph{form} must be a
generalized-variable reference.
The \emph{env} must be an environment of the sort obtained through
the \cd{\&environment} lambda-list keyword; if \emph{env} is \cdf{nil} or omitted,
the null lexical environment is assumed.

This is the same as \cdf{get-setf-method}
except that it does not check the number of store variables; use this
in cases that allow storing multiple values into a generalized variable.
There are no such cases in standard Common Lisp, but this function is provided
to allow for possible extensions.
\end{defun}

\end{newer}

\begin{newer}
X3J13 voted in March 1988 \issue{GET-SETF-METHOD-ENVIRONMENT}
to clarify that a \cdf{setf} method for a functional name is applicable
only when the global binding of that name is lexically visible.
If such a name has a local binding introduced by \cdf{flet}, \cdf{labels},
or \cdf{macrolet}, then global definitions of \cdf{setf} methods for
that name do not apply and are not visible.  All of the standard Common Lisp
macros that modify a \cdf{setf} \emph{place} (for example,
\cdf{incf}, \cdf{decf}, \cdf{pop}, and \cdf{rotatef}) obey this convention.
\end{newer}

\section{Function Invocation}

The most primitive form for function invocation in Lisp of course
has no name; any list that has no other interpretation
as a macro call or special form is taken to be a function call.
Other constructs are provided for less common but
nevertheless frequently useful situations.

\begin{defun}[Function]
apply function arg &rest more-args

This applies \emph{function} to a list of arguments.

\begin{obsolete}
The \emph{function} may be a
compiled-code object, or a lambda-expression, or a symbol; in the latter
case the global functional value of that symbol is used (but it is
illegal for the symbol to be the name of a macro or special form).
\end{obsolete}
\begin{newer}
X3J13 voted in June 1988 \issue{FUNCTION-TYPE} to allow the \emph{function}
to be only of type \cdf{symbol} or \cdf{function}; a lambda-expression
is no longer acceptable as a functional argument.  One must use the
\cdf{function} special form or the abbreviation \cd{\#'} before
a lambda-expression that appears as an  explicit argument form.
\end{newer}
The arguments for the \emph{function} consist of the last argument
to \cdf{apply} appended to the end of a list of all the other
arguments to \cdf{apply} but the \emph{function} itself;
it is as if all the arguments to \cdf{apply} except the \emph{function}
were given to \cd{list*} to create the argument list.
For example:
\begin{lisp}
(setq f '+) (apply f '(1 2)) \EV\ 3 \\
(setq f \#'-) (apply f '(1 2)) \EV\ -1 \\
(apply \#'max 3 5 '(2 7 3)) \EV\ 7 \\
(apply 'cons '((+ 2 3) 4)) {\EV} \\
~~~~~~~~((+ 2 3) . 4)	\emph{not} (5 . 4) \\
(apply \#'+ '()) \EV\ 0
\end{lisp}
Note that if the function takes keyword arguments, the
keywords as well as the corresponding values must appear in the argument
list:
\begin{lisp}
(apply \#'(lambda (\cd{\&key} a b) (list a b)) '(:b 3)) \EV\ ({\nil} 3)
\end{lisp}
This can be very useful in conjunction with the \cd{\&allow-other-keys} feature:
\begin{lisp}
(defun foo (size \cd{\&rest} keys \cd{\&key} double \cd{\&allow-other-keys}) \\
~~(let ((v (apply \#'make-array size :allow-other-keys t keys))) \\
~~~~(if double (concatenate (type-of v) v v) v))) \\
 \\
(foo 4 :initial-contents '(a b c d) :double t) \\
~~~\EV\ \#(a b c d a b c d)
\end{lisp}
\end{defun}

\begin{defun}[Function]
funcall fn &rest arguments

\cd{(funcall \emph{fn} \emph{a1} \emph{a2} ... \emph{an})}
applies the function \emph{fn} to the arguments
\emph{a1}, \emph{a2}, ..., \emph{an}.
The \emph{fn} may not
be a special form or a macro; this would not be meaningful.

\begin{newer}
X3J13 voted in June 1988 \issue{FUNCTION-TYPE} to allow the \emph{fn}
to be only of type \cdf{symbol} or \cdf{function}; a lambda-expression
is no longer acceptable as a functional argument.  One must use the
\cdf{function} special form or the abbreviation \cd{\#'} before
a lambda-expression that appears as an  explicit argument form.
\end{newer}

For example:
\begin{lisp}
(cons 1 2) \EV\ (1 . 2) \\
(setq cons (symbol-function '+)) \\
(funcall cons 1 2) \EV\ 3
\end{lisp}
The difference between \cdf{funcall} and an ordinary function call is that
the function is obtained by ordinary Lisp evaluation rather than
by the special interpretation of the function position that normally
occurs.

\beforenoterule
\begin{incompatibility}
The Common Lisp function \cdf{funcall} corresponds roughly to
the Interlisp primitive \cd{apply*}.
\end{incompatibility}
\afternoterule
\end{defun}

\begin{defun}[Constant]
call-arguments-limit

The value of \cdf{call-arguments-limit} is a positive integer that is
the upper exclusive bound on the number of arguments that may
be passed to a function.  This bound depends on the implementation
but will not be smaller than 50.
(Implementors are encouraged to make this limit as large as practicable
without sacrificing performance.)
The value of \cdf{call-arguments-limit} must be at
least as great as that of \cdf{lambda-parameters-limit}.
See also \cdf{multiple-values-limit}.
\end{defun}

\section{Simple Sequencing}

Each of the constructs in this section simply evaluates all the
argument forms in order.  They differ only in what results
are returned.

\begin{defspec}
progn {\,form}*

The \cdf{progn} construct takes a number of forms and evaluates
them sequentially, in order, from left to right.  The values of all
the forms but the last are discarded; whatever the last form returns
is returned by the \cdf{progn} form.
One says that all the forms but the last are evaluated for \emph{effect},
because their execution is useful only for the side effects caused,
but the last form is executed for \emph{value}.

\cdf{progn} is the primitive control structure construct for ``compound
statements,'' such as \textbf{begin}-\textbf{end} blocks in
Algol-like languages.
Many Lisp constructs are ``implicit \cdf{progn}'' forms:
as part of their syntax each allows many forms to be written
that are to be evaluated sequentially, discarding the results
of all forms but the last and returning the results of the last form.

If the last form of the \cdf{progn} returns multiple values, then those
multiple values are returned by the \cdf{progn} form.  If there are no forms
for the \cdf{progn}, then the result is {\false}.  These rules generally hold for
implicit \cdf{progn} forms as well.
\end{defspec}

\begin{defmac}
prog1 first {\,form}*

\cdf{prog1} is similar to \cdf{progn}, but it returns the value of
its \emph{first} form.  All the argument forms are executed sequentially;
the value of the first form is saved while all the others are executed
and is then returned.

\cdf{prog1} is most commonly used to evaluate an expression with side
effects and to return a value that must be computed \emph{before} the side
effects happen.
For example:
\begin{lisp}
(prog1 (car x) (rplaca x 'foo))
\end{lisp}
alters the \emph{car} of \cdf{x} to be \cdf{foo} and returns the old \emph{car}
of \cdf{x}.

\cdf{prog1} always returns a single value, even if the first form
tries to return multiple values.
As a consequence,
\cd{(prog1 \emph{x})} and \cd{(progn \emph{x})} may behave differently
if \emph{x} can produce multiple values.  See \cdf{multiple-value-prog1}.
A point of style:
although \cdf{prog1} can be used to force exactly a single value to
be returned, it is conventional to use the
function \cdf{values} for this purpose.
\end{defmac}

\begin{defmac}
prog2 first second {\,form}*

\cdf{prog2} is similar to \cdf{prog1}, but it returns the value of
its \emph{second} form.  All the argument forms are executed sequentially;
the value of the second form
is saved while all the other forms are executed and is then returned.
\cdf{prog2} is provided mostly for historical compatibility.
\begin{lisp}
(prog2 \emph{a} \emph{b} \emph{c} ... \emph{z}) \EQ\ (progn \emph{a} (prog1 \emph{b} \emph{c} ... \emph{z}))
\end{lisp}
Occasionally it is desirable to perform one side effect, then a value-producing
operation, then another side effect.  In such a peculiar case, \cdf{prog2}
is fairly perspicuous.
For example:
\begin{lisp}
(prog2 (open-a-file) (process-the-file) (close-the-file)) \\
\`;\textrm{value is that of \cdf{process-the-file}}
\end{lisp}

\cdf{prog2}, like \cdf{prog1},
always returns a single value, even if the second form
tries to return multiple values.  As a consequence of this,
\cd{(prog2 \emph{x} \emph{y})} and \cd{(progn \emph{x} \emph{y})} may behave differently
if \emph{y} can produce multiple values.
\end{defmac}


\section{Establishing New Variable Bindings}
\label{VAR-BINDING-SECTION}

During the invocation of
a function represented by a lambda-expression (or a closure of
a lambda-expression, as produced by \cdf{function}),
new bindings are established for the variables that are the
parameters of the lambda-expression.  These bindings initially
have values determined by the parameter-binding protocol discussed
in section~\ref{LAMBDA-EXPRESSIONS-SECTION}.

The following constructs may also be used to establish bindings of variables,
both ordinary and functional.

\begin{defspec}
let ({var | (var value)}*) {declaration}* {\,form}*

A \cdf{let} form can be used to execute a series of forms
with specified variables bound to specified values.

More precisely, the form
\begin{lisp}
(let ((\emph{var1} \emph{value1}) \\
~~~~~~(\emph{var2} \emph{value2}) \\
~~~~~~... \\
~~~~~~(\emph{varm} \emph{valuem})) \\
~~\emph{declaration1} \\
~~\emph{declaration2} \\
~~... \\
~~\emph{declarationp} \\
~~\emph{body1} \\
~~\emph{body2} \\
~~... \\
~~\emph{bodyn})
\end{lisp}
first evaluates the expressions \emph{value1}, \emph{value2}, and so on,
in that order, saving the resulting values.
Then all of the variables \emph{varj} are bound to the corresponding
values in parallel; each binding will be a lexical binding unless
there is a \cdf{special} declaration to the contrary.
The expressions \emph{bodyk} are then evaluated
in order; the values of all but the last are discarded
(that is, the body of a \cdf{let} form is an implicit \cdf{progn}).
The \cdf{let} form returns what evaluating \emph{bodyn} produces (if the
body is empty, which is fairly useless, \cdf{let} returns {\false} as its value).
The bindings of the variables have lexical scope and indefinite extent.

Instead of a list \cd{(\emph{varj} \emph{valuej})}, one may write simply
\emph{varj}.  In this case \emph{varj} is initialized to {\false}.  As a matter
of style, it is recommended that \emph{varj} be written only when that
variable will be stored into (such as by \cdf{setq}) before its first
use.  If it is important that the initial value be {\false} rather than
some undefined value, then it is clearer to write out
\cd{(\emph{varj} {\false})} if the initial value is intended to mean ``false,'' or
\cd{(\emph{varj} '{\emptylist})} if the initial value is intended to be an empty
list.  Note that the code
\begin{lisp}
(let (x) \\
~~(declare (integer x)) \\
~~(setq x (gcd y z)) \\
~~...)
\end{lisp}
is incorrect; although \cdf{x} is indeed set before it is used,
and is set to a value of the declared type \cdf{integer}, nevertheless
\cdf{x} momentarily takes on the value {\nil} in violation of the type
declaration.

Declarations may appear at the beginning of the body of a \cdf{let}.
See \cdf{declare}.

\begin{newer}
See also \cdf{destructuring-bind}.
Смотрите также \cdf{destructuring-bind}.
\end{newer}

\begin{new}
X3J13 voted in January 1989
\issue{VARIABLE-LIST-ASYMMETRY}
to regularize the binding formats for \cdf{do}, \cdf{do*}, \cdf{let},
\cdf{let*}, \cdf{prog}, \cdf{prog*}, and \cdf{compiler-let}.
The new syntactic definition for \cdf{let} makes the \emph{value} optional:

\begin{defmac}
let ({var | (var [value])}*) {declaration}* {\,form}*

This changes \cdf{let} to allow a list \cd{(\emph{var})} to appear,
meaning the same as simply \emph{var}.
\end{defmac}
\end{new}
\end{defspec}

\begin{defspec}
let* ({var | (var value)}*) {declaration}* {\,form}*

\cdf{let*} is similar to \cdf{let}, but the bindings of variables
are performed sequentially rather than in parallel.  This allows
the expression for the value of a variable to refer to variables
previously bound in the \cdf{let*} form.

More precisely, the form
\begin{lisp}
(let* ((\emph{var1} \emph{value1}) \\
~~~~~~~(\emph{var2} \emph{value2}) \\
~~~~~~~... \\
~~~~~~~(\emph{varm} \emph{valuem})) \\
~~\emph{declaration1} \\
~~\emph{declaration2} \\
~~... \\
~~\emph{declarationp} \\
~~\emph{body1} \\
~~\emph{body2} \\
~~... \\
~~\emph{bodyn})
\end{lisp}
first evaluates the expression \emph{value1}, then binds the variable
\emph{var1} to that value; then it evaluates \emph{value2} and binds \emph{var2};
and so on.
The expressions \emph{bodyj} are then evaluated
in order; the values of all but the last are discarded
(that is, the body of a \cdf{let*} form is an implicit \cdf{progn}).
The \cdf{let*} form returns the results of evaluating \emph{bodyn} (if the
body is empty, which is fairly useless, \cdf{let*} returns {\false} as its value).
The bindings of the variables have lexical scope and indefinite extent.

Instead of a list \cd{(\emph{varj} \emph{valuej})}, one may write simply \emph{varj}.
In this case \emph{varj} is initialized to {\false}.  As a matter of style,
it is recommended that \emph{varj} be written only when that variable
will be stored into (such as by \cdf{setq}) before its first use.
If it is important that the initial value be {\nil} rather than
some undefined value, then it is clearer to write out
\cd{(\emph{varj} {\false})} if the initial value is intended to mean ``false,'' or
\cd{(\emph{varj} '{\emptylist})} if the initial value is intended to be an empty
list.

Declarations may appear at the beginning of the body of a \cdf{let*}.
See \cdf{declare}.

\begin{new}
X3J13 voted in January 1989
\issue{VARIABLE-LIST-ASYMMETRY}
to regularize the binding formats for \cdf{do}, \cdf{do*}, \cdf{let},
\cdf{let*}, \cdf{prog}, \cdf{prog*}, and \cdf{compiler-let}.
The new syntactic definition for \cdf{let*} makes the \emph{value} optional:

\begin{defmac}
let* ({var | (var [value])}*) {declaration}* {\,form}*

This changes \cdf{let*} to allow a list \cd{(\emph{var})} to appear,
meaning the same as simply \emph{var}.
\end{defmac}
\end{new}
\end{defspec}

\begin{obsolete}
\begin{defspec}*
compiler-let ({var | (var value)}*) {\,form}*

When executed by the Lisp interpreter, \cdf{compiler-let} behaves
exactly like \cdf{let} with all the variable bindings implicitly
declared \cdf{special}.  When the compiler processes this form,
however, no code is compiled for the bindings;
instead, the processing of the body by the compiler
(including, in particular, the expansion of any macro calls
within the body) is done with
the special variables bound to the indicated values \emph{in the
execution context of the compiler}.  This is primarily useful for
communication among complicated macros.

Declarations may \emph{not} appear at the beginning of the body
of a \cdf{compiler-let}.

\beforenoterule
\begin{rationale}
Because of the unorthodox
handling by \cdf{compiler-let} of its variable bindings,
it would be complicated and confusing to permit declarations
that apparently referred to the variables bound by \cdf{compiler-let}.
Disallowing declarations eliminates the problem.
\end{rationale}
\afternoterule

X3J13 voted in January 1989
\issue{VARIABLE-LIST-ASYMMETRY}
to regularize the binding formats for \cdf{do}, \cdf{do*}, \cdf{let},
\cdf{let*}, \cdf{prog}, \cdf{prog*}, and \cdf{compiler-let}.
The new syntactic definition for \cdf{compiler-let} makes the \emph{value} optional:

\begin{defmac}
compiler-let ({var | (var [value])}*) {\,form}*

This changes \cdf{compiler-let} to allow a list \cd{(\emph{var})} to appear,
meaning the same as simply \emph{var}.
\end{defmac}
\end{defspec}
\end{obsolete}

\begin{newer}
X3J13 voted in June 1989 \issue{COMPILER-LET-CONFUSION} to remove
\cdf{compiler-let} from the language.  Many uses of \cdf{compiler-let}
can be replaced with more portable code that uses \cdf{macrolet}
or \cdf{symbol-macrolet}.
\end{newer}

\goodbreak

\begin{defspec}
progv symbols values {\,form}*

\cdf{progv} is a special form that allows binding one or more dynamic
variables whose names may be determined at run time.  The sequence of
forms (an implicit \cdf{progn})
is evaluated with the dynamic variables whose names are in the list
\emph{symbols} bound to corresponding values from the list \emph{values}.
(If too few values are supplied, the remaining symbols are bound and then
made to have no value; see \cdf{makunbound}.  If too many values are
supplied, the excess values are ignored.)  The results of the \cdf{progv}
form are those of the last
\emph{form}.  The bindings of the dynamic variables are undone on
exit from the \cdf{progv} form.  The lists of symbols and values are
computed quantities; this is what makes \cdf{progv} different from, for
example, \cdf{let}, where the variable names are stated explicitly in
the program text.

\cdf{progv} is particularly useful for writing interpreters for languages
embedded in Lisp; it provides a handle on the mechanism for binding
dynamic variables.
\end{defspec}

\begin{defspec}
flet ({(name lambda-list
        <{declaration}* | doc-string> {\,form}*)}*)
     {\,form}* \\
labels ({(name lambda-list
          <{declaration}* | doc-string> {\,form}*)}*)
       {\,form}* \\
macrolet ({(name varlist
            <{declaration}* | doc-string> {\,form}*)}*)
         {\,form}*

\cdf{flet} may be used to define locally named functions.  Within the
body of the \cdf{flet} form, function names matching those defined
by the \cdf{flet} refer to the locally defined functions rather than to
the global function definitions of the same name.

Any number of functions may be simultaneously defined.  Each definition
is similar in format to a \cdf{defun} form: first a name,
then a parameter list (which may contain \cd{\&optional}, \cd{\&rest}, or \cd{\&key}
parameters), then optional declarations and documentation string,
and finally a body.
\begin{lisp}
(flet ((safesqrt (x) (sqrt (abs x)))) \\*
~~;; The safesqrt function is used in two places. \\*
~~(safesqrt (apply \#'+ (map 'list \#'safesqrt longlist))))
\end{lisp}

The \cdf{labels} construct is identical in form to the \cdf{flet} construct.
These constructs differ
in that the scope of the defined function names for \cdf{flet}
encompasses only the body, whereas for \cdf{labels} it encompasses the
function definitions themselves.  That is, \cdf{labels} can be used to
define mutually recursive functions, but \cdf{flet} cannot.  This
distinction is useful.  Using \cdf{flet} one can locally redefine a global
function name, and the new definition can refer to the global definition;
the same construction using \cdf{labels} would not have that effect.
\begin{lisp}
(defun integer-power (n k)~~~~~~~;A highly "bummed" integer \\*
~~(declare (integer n))~~~~~~~~~~; exponentiation routine \\*
~~(declare (type (integer 0 *) k)) \\
~~(labels ((expt0 (x k a) \\*
~~~~~~~~~~~~~(declare (integer x a) (type (integer 0 *) k)) \\*
~~~~~~~~~~~~~(cond ((zerop k) a) \\*
~~~~~~~~~~~~~~~~~~~((evenp k) (expt1 (* x x) (floor k 2) a)) \\*
~~~~~~~~~~~~~~~~~~~(t (expt0 (* x x) (floor k 2) (* x a))))) \\
~~~~~~~~~~~(expt1 (x k a) \\*
~~~~~~~~~~~~~(declare (integer x a) (type (integer 1 *) k)) \\*
~~~~~~~~~~~~~(cond ((evenp k) (expt1 (* x x) (floor k 2) a)) \\*
~~~~~~~~~~~~~~~~~~~(t (expt0 (* x x) (floor k 2) (* x a)))))) \\*
~~~~(expt0 n k 1)))
\end{lisp}

\cdf{macrolet} is similar in form to \cdf{flet} but defines local macros,
using the same format used by \cdf{defmacro}.
The names established by \cdf{macrolet} as names for macros are
lexically scoped.

\begin{new}
I have observed that, while most Common Lisp users pronounce \cdf{macrolet}
to rhyme with ``silhouette,'' a small but vocal minority pronounce it
to rhyme with ``Chevrolet.''  A very few extremists furthermore
adjust their pronunciation
of \cdf{flet} similarly: they say ``flay.''
Hey, hey!  \emph{Tr\`es outr\'e.}
\end{new}

Macros often must be expanded at ``compile time'' (more generally,
at a time before the program itself is executed), and so
the run-time values of variables are not available to macros
defined by \cdf{macrolet}.

\begin{obsolete}
The precise rule is that the macro-expansion
functions defined by \cdf{macrolet} are defined in the \emph{global} environment;
lexically scoped entities that would ordinarily be lexically apparent
are not visible within the expansion functions.
\end{obsolete}

\begin{newer}
X3J13 voted in March 1989 \issue{DEFINING-MACROS-NON-TOP-LEVEL}
to retract the previous sentence and specify that the macro-expansion
functions created by \cdf{macrolet} are defined in the lexical environment in which
the \cdf{macrolet} form appears, not in the null lexical environment.
Declarations, \cdf{macrolet} definitions, and \cdf{symbol-macrolet} definitions
affect code within the expansion functions in a \cdf{macrolet}, but the
consequences are undefined if such code attempts to refer to
any local variable or function bindings that are visible in that
lexical environment.
\end{newer}

However,
lexically scoped entities \emph{are} visible
within the body of the \cdf{macrolet} form and \emph{are} visible
to the code that is the expansion of a macro call.  The following example
should make this clear:
\begin{lisp}
;;; Example of scoping in macrolet. \\*
\\*
(defun foo (x flag) \\*
~~(macrolet ((fudge (z) \\*
~~~~~~~~~~~~~~~~;;\textrm{The parameters \cdf{x} and \cdf{flag} are not accessible} \\*
~~~~~~~~~~~~~~~~;; \textrm{at this point; a reference to \cdf{flag} would be to} \\*
~~~~~~~~~~~~~~~~;; \textrm{the global variable of that name.} \\*
~~~~~~~~~~~~~~~~{\Xbq}(if flag \\
~~~~~~~~~~~~~~~~~~~~~(* ,z ,z) \\
~~~~~~~~~~~~~~~~~~~~~,z))) \\
~~~~;;\textrm{The parameters \cdf{x} and \cdf{flag} are accessible here.} \\*
~~~~(+ x \\*
~~~~~~~(fudge x) \\*
~~~~~~~(fudge (+ x 1)))))
\end{lisp}
The body of the \cdf{macrolet} becomes
\begin{lisp}
(+ x \\*
~~~(if flag \\*
~~~~~~~(* x x) \\*
~~~~~~~x)) \\*
~~~(if flag \\*
~~~~~~~(* (+ x 1) (+ x 1)) \\*
~~~~~~~(+ x 1)))
\end{lisp}
after macro expansion.  The occurrences of \cdf{x} and \cdf{flag} legitimately
refer to the parameters of the function \cdf{foo} because those parameters are
visible at the site of the macro call which produced the expansion.

\begin{newer}
X3J13 voted in March 1988 \issue{FLET-IMPLICIT-BLOCK}
to specify that the body of each function or expander function defined
by \cdf{flet}, \cdf{labels}, or \cdf{macrolet}
is implicitly enclosed in a \cdf{block} construct
whose name is the same as the \emph{name} of the function.
Therefore \cdf{return-from} may be used to exit from the function.
\end{newer}

\begin{newer}
X3J13 voted in March 1989 \issue{FUNCTION-NAME} to extend \cdf{flet} and \cdf{labels}
to accept any function-name (a symbol or a list
whose \emph{car} is \cdf{setf}---see section~\ref{FUNCTION-NAME-SECTION}) as a \emph{name}
for a function to be locally defined.  In this way one can create local definitions
for \cdf{setf} expansion functions.  (X3J13 explicitly declined to extend
\cdf{macrolet} in the same manner.)
\end{newer}

\begin{new}
X3J13 voted in March 1988
\issue{FLET-DECLARATIONS}
to change \cdf{flet}, \cdf{labels}, and \cdf{macrolet}
to allow declarations to appear before the body.
The new descriptions are therefore as follows:

\begin{defmac}
flet ({(name lambda-list
        <{declaration}* | doc-string> {\,form}*)}*)
     {declaration}* {\,form}* \\
labels ({(name lambda-list
          <{declaration}* | doc-string> {\,form}*)}*)
       {declaration}* {\,form}* \\
macrolet ({(name varlist
            <{declaration}* | doc-string> {\,form}*)}*)
         {declaration}* {\,form}*

These are now syntactically more similar to such
other binding forms as \cdf{let}.

For \cdf{flet} and \cdf{labels}, the bodies of
the locally defined functions are part of
the scope of pervasive declarations appearing before the main body.
(This is consistent with the treatment of initialization forms in \cdf{let}.)
For \cdf{macrolet}, however, the bodies of
the locally defined macro expander functions are \emph{not} included in
the scope of pervasive declarations appearing before the main body.
(This is consistent with the rule, stated below, that the bodies of
macro expander functions are in the global environment, not the local
lexical environment.)
Here is an example:
\begin{lisp}
(flet ((stretch (x) (* x *stretch-factor*)) \\*
~~~~~~~(chop (x) (- x *chop-margin*))) \\*
~~(declare (inline stretch chop))~~~;\textrm{Illegal in original Common Lisp} \\
~~(if (> x *chop-margin*) (stretch (chop x)) (chop (stretch x))))
\end{lisp}
X3J13 voted to permit declarations of the sort noted above.
\end{defmac}
\end{new}
\end{defspec}


\begin{new}
\begin{defspec}
symbol-macrolet ({(var expansion)}*)
                {declaration}* {\,form}*

X3J13 voted in June 1988
\issue{CLOS}
to adopt the Common Lisp Object System.  Part of this proposal
is a general mechanism, \cdf{symbol-macrolet},
for treating certain variable names as if they were
parameterless macro calls.  This facility may be useful independent of CLOS.
X3J13 voted in March 1989
\issue{SYMBOL-MACROLET-SEMANTICS}
to modify the definition of \cdf{symbol-macrolet} substantially
and also voted
\issue{SYMBOL-MACROLET-DECLARE} to allow declarations before the body
of \cdf{symbol-macrolet} but with peculiar treatment of \cdf{special}
and type declarations.

The \emph{forms} are executed as an implicit \cdf{progn} in a lexical
environment that causes every reference to any defined \emph{var}
to be replaced by the corresponding \emph{expansion}.  It is as if
the reference to the \emph{var} were a parameterless macro call;
the \emph{expansion} is evaluated or otherwise processed
in place of the reference
\vadjust{\penalty-10000}%manual
(in particular, the expansion form is itself subject
to further expansion---this is one of the changes
\issue{SYMBOL-MACROLET-SEMANTICS}
from the
original definition in the CLOS proposal).  Note, however, that the names of
such symbol macros occupy the name space of variables, not the
name space of functions; just as one may have a function
(or macro, or special form) and a variable
with the same name without interference, so one may have an ordinary
macro (or function, or special form)
and a symbol macro with the same name.
The use of \cdf{symbol-macrolet} can therefore be shadowed by \cdf{let}
or other constructs that bind variables; \cdf{symbol-macrolet} does
not substitute for all occurrences of a \emph{var} as a variable
but only for those occurrences that would be construed as
references in the scope of a lexical binding of \emph{var} as
a variable.  For example:
\begin{lisp}
(symbol-macrolet ((pollyanna 'goody)) \\*
~~(list pollyanna (let ((pollyanna 'two-shoes)) pollyanna))) \\*
~{\EV} (goody two-shoes)\textrm{, \emph{not}} (goody goody)
\end{lisp}
% novel "Pollyanna" by Eleanor Porter, 1913
One might think that \cd{'goody} simply replaces all occurrences of
\cdf{pollyanna}, and so the value of the \cdf{let} would be
\cdf{goody}; but this is not so.  A little reflection shows that under
this incorrect interpretation the body in expanded form would be
\begin{lisp}
(list 'goody (let (('goody 'two-shoes)) 'goody))
\end{lisp}
which is syntactically malformed.  The correct expanded form is
\begin{lisp}
(list 'goody (let ((pollyanna 'two-shoes)) pollyanna))
\end{lisp}
because the rebinding of \cdf{pollyanna} by the \cdf{let} form
shadows the symbol macro definition.

The \emph{expansion} for each \emph{var} is not evaluated at binding time
but only after it has replaced a reference to the \emph{var}.
The \cdf{setf} macro allows a symbol macro to be used as a \emph{place},
in which case its expansion is used; moreover, \cdf{setq} of a variable
that is really a symbol macro will be treated as if \cdf{setf} had
been used.
The values of the last form are returned, or \cdf{nil} if there is no value.

See \cdf{macroexpand} and \cdf{macroexpand-1}; they will expand symbol
macros as well as ordinary macros.

Certain \emph{declarations} before the body are handled in a peculiar manner;
see section~\ref{DECLARE-SYNTAX-SECTION}.

%??? See the related CLOS features \cdf{with-accessors} and \cdf{with-slots}.
\end{defspec}
\end{new}

\section{Conditionals}

The traditional conditional construct in Lisp is \cdf{cond}.
However, \cdf{if} is much simpler and is directly comparable
to conditional constructs in other programming languages,
so it is considered to be primitive in Common Lisp and is described first.
Common Lisp also provides the dispatching constructs \cdf{case} and \cdf{typecase},
which are often more convenient than \cdf{cond}.

\begin{defspec}
if test then [else]

The \cdf{if} special form corresponds to the \textbf{if}-\textbf{then}-\textbf{else} construct
found in most algebraic programming languages.
First the form \emph{test} is evaluated.  If the result is not {\false},
then the form \emph{then} is selected; otherwise the form \emph{else} is selected.
Whichever form is selected is then evaluated, and \cdf{if} returns
whatever is returned by evaluation of the selected form.
\begin{lisp}
(if \emph{test} \emph{then} \emph{else}) \EQ\ (cond (\emph{test} \emph{then}) ({\true} \emph{else}))
\end{lisp}
but \cdf{if} is considered more readable in some situations.

The \emph{else} form may be omitted, in which case if the value of \emph{test}
is {\false} then nothing is done and the value of the \cdf{if} form is {\false}.
If the value of
the \cdf{if} form is important in this situation, then the \cdf{and}
construct may be stylistically preferable,
depending on the context.
If the value is not important, but only the effect, then the \cdf{when}
construct may be stylistically preferable.
\end{defspec}

\begin{defmac}
when test {\,form}*

\cd{(when \emph{test} \emph{form1} \emph{form2} ... )}
first evaluates \emph{test}.  If the result is {\false},
then no \emph{form} is evaluated, and {\false} is returned.
Otherwise the \emph{form}s constitute an implicit \cdf{progn}
and are evaluated sequentially from left to right,
and the value of the last one is returned.
\begin{lisp}
(when \emph{p} \emph{a} \emph{b} \emph{c}) \EQ\ (and \emph{p} (progn \emph{a} \emph{b} \emph{c})) \\
(when \emph{p} \emph{a} \emph{b} \emph{c}) \EQ\ (cond (\emph{p} \emph{a} \emph{b} \emph{c})) \\
(when \emph{p} \emph{a} \emph{b} \emph{c}) \EQ\ (if \emph{p} (progn \emph{a} \emph{b} \emph{c}) {\false}) \\
(when \emph{p} \emph{a} \emph{b} \emph{c}) \EQ\ (unless (not \emph{p}) \emph{a} \emph{b} \emph{c})
\end{lisp}
As a matter of style,
\cdf{when} is normally used to conditionally produce some side effects,
and the value of the \cdf{when} form is normally not used.
If the value is relevant, then it may be
stylistically more appropriate to use \cdf{and} or \cdf{if}.
\end{defmac}

\begin{defmac}
unless test {\,form}*

\cd{(unless \emph{test} \emph{form1} \emph{form2} ... )}
first evaluates \emph{test}.  If the result is \emph{not} {\false},
then the \emph{form}s are not evaluated, and {\false} is returned.
Otherwise the \emph{form}s constitute an implicit \cdf{progn}
and are evaluated sequentially from left to right,
and the value of the last one is returned.
\begin{lisp}
(unless \emph{p} \emph{a} \emph{b} \emph{c}) \EQ\ (cond ((not \emph{p}) \emph{a} \emph{b} \emph{c})) \\
(unless \emph{p} \emph{a} \emph{b} \emph{c}) \EQ\ (if \emph{p} {\false} (progn \emph{a} \emph{b} \emph{c})) \\
(unless \emph{p} \emph{a} \emph{b} \emph{c}) \EQ\ (when (not \emph{p}) \emph{a} \emph{b} \emph{c})
\end{lisp}
As a matter of style,
\cdf{unless} is normally used to conditionally produce some side effects,
and the value of the \cdf{unless} form is normally not used.
If the value is relevant, then it may be
stylistically more appropriate to use \cdf{if}.
\end{defmac}

\begin{defmac}
cond {(test {\,form}*)}*

A \cdf{cond} form has a number (possibly zero) of
\emph{clauses}, which are lists of forms.
Each clause consists of a \emph{test} followed
by zero or more \emph{consequents}.
For example:
\begin{lisp}
(cond (\emph{test-1} \emph{consequent-1-1} \emph{consequent-1-2} ...) \\
~~~~~~(\emph{test-2}) \\
~~~~~~(\emph{test-3} \emph{consequent-3-1} ...) \\
~~~~~~... )
\end{lisp}

The first clause whose \emph{test} evaluates to non-{\false}
is selected; all other clauses are ignored, and the consequents
of the selected clause are evaluated in order (as an implicit \cdf{progn}).

More specifically, \cdf{cond} processes its clauses in order from left to
right.  For each clause, the \emph{test} is evaluated.  If the result is
{\false}, \cdf{cond} advances to the next clause.  Otherwise, the \emph{cdr} of
the clause is treated as a list of forms, or consequents; these forms are
evaluated in order from left to right, as an implicit \cdf{progn}.
After evaluating the consequents,
\cdf{cond} returns without inspecting any remaining clauses.
The \cdf{cond} special form returns the results
of evaluating the last of the selected consequents;
if there were no consequents in
the selected clause,
then the single (and necessarily non-null) value of the \emph{test} is returned.
If \cdf{cond} runs out of clauses (every test produced {\false},
and therefore no clause was selected), the value of the \cdf{cond} form is
{\false}.

If it is desired to select the last clause unconditionally if all others
fail, the standard convention is to use {\true} for the \emph{test}.
As a matter of style, it is desirable to write a last clause
\cd{({\true} {\false})} if the value of the \cdf{cond} form is to be used
for something.  Similarly, it is in questionable
taste to let the last clause of
a \cdf{cond} be a ``singleton clause''; an explicit {\true} should be provided.
(Note moreover that \cd{(cond ... (\emph{x}))} may behave differently from
\cd{(cond ... ({\true} \emph{x}))} if \emph{x} might produce multiple values;
the former always returns a single value, whereas the latter returns whatever
values \emph{x} returns.  However, as a matter of style it is preferable
to obtain this behavior by writing \cd{(cond ... (t (values \emph{x})))},
using the \cdf{values} function explicitly to indicate the discarding
of any excess values.)
For example:
\begin{lisp}
~~~~~~~~~~~~~~~~~~~~~~~~~~~~~~~~~~~~~~~~~~~~~~\=\kill
(setq z (cond (a 'foo) (b 'bar)))\>;\textrm{Possibly confusing} \\*
(setq z (cond (a 'foo) (b 'bar) ({\true} {\false})))\>;\textrm{Better} \\
(cond (a b) (c d) (e))\>;\textrm{Possibly confusing} \\*
(cond (a b) (c d) ({\true} e))\>;\textrm{Better} \\*
(cond (a b) (c d) ({\true} (values e)))\>;\textrm{Better (if one value} \\*
                                       \>;~\textrm{needed)} \\
(cond (a b) (c))\>;\textrm{Possibly confusing} \\
(cond (a b) (t c))\>;\textrm{Better} \\*
(if a b c)\>;\textrm{Also better}
\end{lisp}
A Lisp \cdf{cond} form may be compared to a continued \textbf{if}-\textbf{then}-\textbf{else}
as found in many algebraic programming languages:
\begin{lisp}
~~~~~~~~~~~~~~~~~~~~\=~~~~~~~~~~~~~~~~~~~~~\=\kill
(cond (\emph{p} ...)\>\>\textbf{if} \emph{p} \textbf{then} ... \\
~~~~~~(\emph{q} ...)\>\textrm{roughly}\>\textbf{else} \textbf{if} \emph{q} \textbf{then} ... \\
~~~~~~(\emph{r} ...)\>\textrm{corresponds}\>\textbf{else} \textbf{if} \emph{r} \textbf{then} ... \\
~~~~~~...\>\textrm{to}\>... \\
~~~~~~({\true} ...))\>\>\textbf{else} ...
\end{lisp}
\end{defmac}

\begin{defmac}
case keyform {({({key}*) | key} {\,form}*)}*

\cdf{case} is a conditional that chooses one of its clauses to execute
by comparing a value to various constants, which are
typically keyword symbols, integers, or characters
(but may be any objects).  Its form is as follows:
\begin{lisp}
(case \emph{keyform} \\
~~(\emph{keylist-1} \emph{consequent-1-1} \emph{consequent-1-2} ...) \\
~~(\emph{keylist-2} \emph{consequent-2-1} ...) \\
~~(\emph{keylist-3} \emph{consequent-3-1} ...) \\
~~...)
\end{lisp}
Structurally \cdf{case} is much like \cdf{cond},
and it behaves like \cdf{cond}
in selecting one clause and then executing all consequents of that clause.
However, \cdf{case} differs in the mechanism of clause selection.

The first thing \cdf{case} does is to evaluate the form \emph{keyform}
to produce an object called the \emph{key object}.
Then \cdf{case} considers
each of the clauses in turn.  If \emph{key} is in the \emph{keylist}
(that is, is \cdf{eql} to any item in the \emph{keylist}) of a clause,
the consequents of that
clause are evaluated as an implicit \cdf{progn};
\cdf{case} returns what was returned by the last
consequent (or {\false} if there are no consequents in that clause).
If no clause is satisfied, \cdf{case} returns {\false}.

The keys in the keylists are \emph{not} evaluated; literal key values
must appear in the keylists.
It is an error for the same key to appear in more than one clause;
a consequence is that the order of the clauses does not affect
the behavior of the \cdf{case} construct.

Instead of a \emph{keylist}, one may write one of the symbols
{\true} and \cdf{otherwise}.  A clause with such a symbol
always succeeds and must be the last clause (this is an exception
to the order-independence of clauses).
See also \cdf{ecase} and \cdf{ccase}, each of which provides
an implicit \cdf{otherwise} clause to signal an error if no clause
is satisfied.

If there is only one key for a clause, then that key may be written
in place of a list of that key, provided that no ambiguity results.
Such a ``singleton key'' may not be {\nil} (which is confusable
with {\emptylist}, a list of no keys), {\true}, \cdf{otherwise}, or a cons.

\beforenoterule
\begin{incompatibility}
The Lisp Machine Lisp \cdf{caseq} construct
uses \cdf{eq} for the comparison.
In Lisp Machine Lisp \cdf{caseq} therefore works for
fixnums but not bignums.
The MacLisp \cdf{caseq} construct simply prohibits the use of bignums;
indeed, it permits only fixnums and symbols as clause keys.
In the interest of hiding the fixnum-bignum distinction,
and for general language consistency,
\cdf{case} uses \cdf{eql} in Common Lisp.

The Interlisp \cdf{selectq} construct is similar to \cdf{case}.
\end{incompatibility}
\afternoterule
\end{defmac}

\begin{defmac}
typecase keyform {(type {\,form}*)}*

\cdf{typecase} is a conditional that chooses one of its clauses by
examining the type of an object.
Its form is as follows:
\begin{lisp}
(typecase \emph{keyform} \\*
~~(\emph{type-1} \emph{consequent-1-1} \emph{consequent-1-2} ...) \\*
~~(\emph{type-2} \emph{consequent-2-1} ...) \\*
~~(\emph{type-3} \emph{consequent-3-1} ...) \\
~~...)
\end{lisp}
Structurally \cdf{typecase} is much like \cdf{cond} or \cdf{case},
and it behaves like them
in selecting one clause and then executing all consequents of that clause.
It differs in the mechanism of clause selection.

The first thing \cdf{typecase} does is to evaluate the form \emph{keyform}
to produce an object called the key object.
Then \cdf{typecase} considers
each of the clauses in turn.  The \emph{type} that appears
in each clause is a type specifier; it is not evaluated
but is a literal type specifier.
The first clause for which the key
is of that clause's specified \emph{type}
is selected, the consequents of this
clause are evaluated as an implicit \cdf{progn},
and \cdf{typecase} returns what was returned by the last
consequent (or {\false} if there are no consequents in that clause).
If no clause is satisfied, \cdf{typecase} returns {\false}.

As for \cdf{case}, the symbol {\true} or \cdf{otherwise} may be written
for \emph{type} to indicate that the clause should always be selected.
See also \cdf{etypecase} and \cdf{ctypecase}, each of which provides
an implicit \cdf{otherwise} clause to signal an error if no clause
is satisfied.

It is permissible for more than one clause to specify a given type,
particularly if one is a subtype of another; the earliest applicable
clause is chosen.  Thus for \cdf{typecase}, unlike \cdf{case}, the order
of the clauses may affect the behavior of the construct.
For example:
\begin{lisp}
~~~~~~~~~~~~~~~~~~~~~~~~~~~~~~\=\kill
(typecase an-object \\
~~~(string ...)\>;\textrm{This clause handles strings} \\
~~~((array t) ...)\>;\textrm{This clause handles general arrays} \\
~~~((array bit) ...)\>;\textrm{This clause handles bit arrays} \\
~~~(array ...)\>;\textrm{This handles all other arrays} \\
~~~((or list number) ...)\>;\textrm{This handles lists and numbers} \\
~~~(t ...))\>;\textrm{This handles all other objects}
\end{lisp}
A Common Lisp compiler may choose to issue a warning if
a clause cannot be selected because it is completely shadowed by
earlier clauses.
\end{defmac}

\section{Blocks and Exits}
\label{BLOCK-RETURN-SECTION}

The \cdf{block} and \cdf{return-from} constructs provide a structured lexical
non-local exit facility.  At any point lexically within a \cdf{block}
construct, a \cdf{return-from} with the same name may be used to
perform an immediate transfer of control that
exits from the \cdf{block}.  In the most common cases this mechanism is
more efficient than the dynamic non-local exit facility
provided by \cdf{catch} and \cdf{throw}, described in
section~\ref{CATCH-THROW-SECTION}.

\begin{defspec}
block name {\,form}*

The \cdf{block} construct executes each \emph{form} from left to right,
returning whatever is returned by the last \emph{form}.
If, however, a \cdf{return} or \cdf{return-from} form that specifies the
same \emph{name} is executed
during the execution of some \emph{form}, then the results
specified by the \cdf{return} or \cdf{return-from} are immediately
returned as the value of the \cdf{block} construct, and execution
proceeds as if the \cdf{block} had terminated normally.
In this, \cdf{block} differs from \cdf{progn}; the \cdf{progn} construct
has nothing to do with \cdf{return}.

The \emph{name} is not evaluated; it must be a symbol.
The scope of the \emph{name} is lexical; only a \cdf{return} or \cdf{return-from}
textually contained in some \emph{form} can exit from the block.
The extent of the name is dynamic.
Therefore it is only possible to exit from a given run-time incarnation of a
block once, either normally or by explicit return.

The \cdf{defun} form implicitly puts a \cdf{block} around the
body of the function defined; the \cdf{block} has the same name as the function.
Therefore one may use \cdf{return-from} to return
prematurely from a function defined by \cdf{defun}.

The lexical scoping of the block name
is fully general and has consequences that may be surprising
to users and implementors of other Lisp systems.
For example, the \cdf{return-from} in the following example actually does
work in Common Lisp as one might expect:
\begin{lisp}
(block loser \\
~~~(catch 'stuff \\
~~~~~~(mapcar \#'(lambda (x) (if (numberp x) \\
~~~~~~~~~~~~~~~~~~~~~~~~~~~~~~~~(hairyfun x) \\
~~~~~~~~~~~~~~~~~~~~~~~~~~~~~~~~(return-from loser {\nil}))) \\
~~~~~~~~~~~~~~items)))
\end{lisp}
Depending on the situation, a \cdf{return} in Common Lisp
may not be simple.
A \cdf{return} can break up catchers if necessary to get
to the block in question.
It is possible for a ``closure'' created by \cdf{function} for
a lambda-expression to refer to a block name as long as the name
is lexically apparent.
\end{defspec}

\begin{defspec}
return-from name [result]

\cdf{return-from}
is used to return from a \cdf{block} or from such constructs
as \cdf{do} and \cdf{prog} that implicitly establish a \cdf{block}.
The \emph{name} is not evaluated and must be a symbol.
A \cdf{block} construct with the same name must lexically
enclose the occurrence of \cdf{return-from};
whatever the evaluation of \emph{result} produces
is immediately returned from the block.
(If the \emph{result} form is omitted, it defaults to {\nil}.
As a matter of style, this form ought to be used to indicate that
the particular value returned doesn't matter.)

The \cdf{return-from} form itself never returns and cannot have a value;
it causes results to be returned from a \cdf{block} construct.
If the evaluation of \emph{result} produces multiple values,
those multiple values are returned by the construct exited.
\end{defspec}

\begin{defmac}
return [result]

\cd{(return \emph{form})} is identical in meaning
to \cd{(return-from {\nil} \emph{form})}; it returns from a block named {\nil}.
Blocks established implicitly by iteration constructs such
as \cdf{do} are named {\nil}, so that \cdf{return} will exit properly from
such a construct.
\end{defmac}

\section{Цикл}
\indexterm{iteration}

Common Lisp provides a number of iteration constructs.  The \cdf{loop}
construct provides a trivial iteration facility; it is little more
than a \cdf{progn} with a branch from the bottom back to the top.
The \cdf{do}
and \cdf{do*} constructs provide a general iteration facility
for controlling the variation of several variables on each cycle.
For specialized iterations
over the elements of a list or \emph{n} consecutive integers, \cdf{dolist} and
\cdf{dotimes} are provided.  The \cdf{tagbody} construct is the most
general, permitting arbitrary \cdf{go} statements within it.  (The
traditional \cdf{prog} construct is a synthesis of \cdf{tagbody},
\cdf{block}, and \cdf{let}.)
Most of the iteration constructs permit statically defined non-local exits
(see \cdf{return-from} and \cdf{return}).

\subsection{Indefinite Iteration}

The \cdf{loop} construct is the simplest iteration facility.
It controls no variables, and simply executes its body repeatedly.

\begin{defmac}
loop {\,form}*

Each \emph{form} is evaluated in turn from left to right.
When the last \emph{form} has been evaluated, then the first \emph{form}
is evaluated again, and so on, in a never-ending cycle.
The \cdf{loop} construct never returns a value.  Its execution must be terminated
explicitly, using \cdf{return} or \cdf{throw}, for example.

\cdf{loop}, like most iteration constructs,
establishes an implicit block named {\nil}.
Thus \cdf{return} may be used to exit from a \cdf{loop} with specified results.

\cdf{loop}
\begin{obsolete}
A \cdf{loop} construct has this meaning only if every \emph{form} is
non-atomic (a list).  The case where some \emph{form} is
atomic is reserved for future extensions.

\beforenoterule
\begin{implementation}
There have been several proposals for a powerful iteration
mechanism to be called \cdf{loop}.  One version is provided in Lisp Machine Lisp.
Implementors are encouraged to experiment with extensions to the \cdf{loop}
syntax, but users should be advised that in all likelihood some specific
set of extensions to \cdf{loop} will be adopted in a future revision of Common Lisp.
\end{implementation}
\afternoterule
\end{obsolete}

\begin{new}
X3J13 voted in January 1989
\issue{LOOP-FACILITY}
to include just such an extension of \cdf{loop}.  See chapter~\ref{LOOP}.
\end{new}
\end{defmac}


\subsection{General Iteration}

In contrast to \cdf{loop}, \cdf{do} and \cdf{do*} provide a powerful
and general mechanism for repetitively recalculating many variables.

\begin{defmac}
do ({(var [init [step]])}*)
   (end-test {result}*)
   {declaration}* {tag | statement}* \\
do* ({(var [init [step]])}*)
    (end-test {result}*)
    {declaration}* {tag | statement}*

The \cdf{do} special form provides a generalized iteration facility,
with an arbitrary number of ``index variables.''
These variables are bound within the iteration and stepped in parallel
in specified ways.  They may be used both to generate successive
values of interest (such as successive integers) or to accumulate results.
When an end condition is met, the iteration terminates with a specified value.

In general, a \cdf{do} loop looks like this:
\begin{lisp}
(do ((\emph{var1} \emph{init1} \emph{step1}) \\
~~~~~(\emph{var2} \emph{init2} \emph{step2}) \\
~~~~~... \\
~~~~~(\emph{varn} \emph{initn} \emph{stepn})) \\
~~~~(\emph{end-test} . \emph{result}) \\
~~\Mstar{\emph{declaration}} \\
~~. \emph{tagbody})
\end{lisp}
A \cdf{do*} loop looks exactly the same except that the name \cdf{do} is
replaced by \cdf{do*}.

The first item in the form is a list of zero or more index-variable
specifiers.  Each index-variable specifier is a list of the name of a
variable \emph{var}, an initial value \emph{init},
and a stepping form \emph{step}.
If \emph{init} is omitted, it defaults to {\false}.
If \emph{step} is omitted, the \emph{var} is not changed by the \cdf{do} construct
between repetitions (though code within the \cdf{do} is free to alter
the value of the variable by using \cdf{setq}).

An index-variable specifier can also be just the name of a variable.
In this case, the variable has an initial value of {\false} and is
not changed between repetitions.
As a matter
of style, it is recommended that an unadorned variable name
be written only when that
variable will be stored into (such as by \cdf{setq}) before its first
use.  If it is important that the initial value be {\false} rather than
some undefined value, then it is clearer to write out
\cd{(\emph{varj} {\false})} if the initial value is intended to mean ``false,'' or
\cd{(\emph{varj} '{\emptylist})} if the initial value is intended to be an empty
list.

\begin{new}
X3J13 voted in January 1989
\issue{VARIABLE-LIST-ASYMMETRY}
to regularize the binding formats for \cdf{do}, \cdf{do*}, \cdf{let},
\cdf{let*}, \cdf{prog}, \cdf{prog*}, and \cdf{compiler-let}.
In the case of \cdf{do} and \cdf{do*} the first edition was inconsistent;
the formal syntax fails to reflect the fact that a simple variable
name may appear, as described in the preceding paragraph.  The
definitions should read

\begin{defmac}
do ({var | (var [init [step]])}*)
   (end-test {result}*)
   {declaration}* {tag | statement}* \\
do* ({var | (var [init [step]])}*)
    (end-test {result}*)
    {declaration}* {tag | statement}*

for consistency with the reading of the first edition and the X3J13 vote.
\end{defmac}
\end{new}

Before the first iteration, all the \emph{init} forms are evaluated, and
each \emph{var} is bound to the value of its respective \emph{init}.
This is a binding, not an assignment; when the loop terminates,
the old values of those variables will be restored.
For \cdf{do}, \emph{all} of the \emph{init} forms are evaluated \emph{before} any \emph{var}
is bound; hence all the
\emph{init} forms may refer to the old bindings of all the variables
(that is, to the values visible before beginning execution of
the \cdf{do} construct).
For \cdf{do*}, the first \emph{init} form is evaluated, then the first
\emph{var} is bound to that value, then the second \emph{init} form
is evaluated, then the second \emph{var} is bound, and so on;
in general, the \emph{initj} form can refer to the \emph{new} binding \emph{vark}
if $k<j$, and otherwise to the \emph{old} binding of
\emph{vark}.

The second element of the loop is a list of an end-testing
predicate form \emph{end-test} and zero or more \emph{result} forms.
This resembles a \cdf{cond} clause.
At the beginning of each iteration, after processing the variables,
the \emph{end-test} is evaluated.  If the result is
{\false}, execution proceeds with the body of the \cdf{do} (or \cdf{do*}) form.
If the
result is not {\false}, the \emph{result} forms are evaluated in order
as an implicit \cdf{progn},
\indexterm{implicit \cdf{progn}}
and then \cdf{do} returns.  \cdf{do} returns the results of evaluating
the last \emph{result} form.
If there are no \emph{result} forms, the value of \cdf{do} is {\false}.
Note that this is not quite analogous to the treatment of
clauses in a \cdf{cond} form, because a \cdf{cond} clause
with no \emph{result} forms returns the (non-{\nil}) result of the test.

At the beginning of each iteration other than the first, the
index variables are updated as follows.  All the \emph{step} forms
are evaluated, from left to right, and the resulting values are
assigned to the respective index variables.
Any variable that has no associated \emph{step} form is not assigned to.
For \cdf{do}, all the \emph{step} forms are evaluated before any variable
is updated; the assignment of values to variables is done in parallel,
as if by \cdf{psetq}.
Because \emph{all} of the \emph{step} forms are evaluated before \emph{any}
of the variables are altered, a \emph{step} form when evaluated always has
access to the \emph{old} values of \emph{all} the index variables,
even if other \emph{step} forms precede it.
For \cdf{do*}, the first \emph{step} form is evaluated, then the
value is assigned to the first \emph{var}, then the second \emph{step} form
is evaluated, then the value is assigned to the second \emph{var}, and so on;
the assignment of values to variables is done sequentially,
as if by \cdf{setq}.
For either \cdf{do} or \cdf{do*},
after the variables have been updated,
the \emph{end-test} is evaluated as described above, and the iteration
continues.

If the \emph{end-test} of a \cdf{do} form is \cd{{\false}},
the test will never succeed.
Therefore this provides an idiom for ``do forever'':
the \emph{body} of the \cdf{do} is executed repeatedly, stepping variables
as usual.  (The \cdf{loop} construct performs
a ``do forever'' that steps no variables.)
The infinite loop can be terminated by the use of \cdf{return},
\cdf{return-from}, \cdf{go} to an outer level, or \cdf{throw}.
For example:
\begin{lisp}
(do ((j 0 (+ j 1))) \\
~~~~({\false})~~~~~~~~~~~~~~~~~~~~~~~~;\textrm{Do forever} \\
~~(format t "{\Xtilde}\%Input {\Xtilde}D:" j) \\
~~(let ((item (read))) \\
~~~~(if (null item) (return)~~~~~;\textrm{Process items until {\false} seen} \\
~~~~~~~~(format t "{\Xtilde}\&Output {\Xtilde}D: {\Xtilde}S" j (process item)))))
\end{lisp}

The remainder of the \cdf{do} form constitutes an implicit \cdf{tagbody}.
Tags may appear within the body of a \cdf{do} loop
for use by \cdf{go} statements appearing in the body (but such \cdf{go}
statements may not appear in the variable specifiers, the \emph{end-test},
or the \emph{result} forms).
When the end of a \cdf{do} body is reached, the next iteration cycle
(beginning with the evaluation of \emph{step} forms) occurs.

An implicit \cdf{block} named {\nil} surrounds the entire \cdf{do} form.
A \cdf{return} statement may be used at any point to exit the loop
immediately.

\cdf{declare} forms may appear at the beginning of a \cdf{do} body.
They apply to code in the \cdf{do} body, to the bindings of the \cdf{do}
variables, to the \emph{init} forms, to the \emph{step} forms,
to the \emph{end-test}, and to the \emph{result} forms.

\beforenoterule
\begin{incompatibility}
``Old-style'' MacLisp \cdf{do} loops, that is, those
of the form \cd{(do \emph{var} \emph{init} \emph{step} \emph{end-test} . \emph{body})},
are not supported in Common Lisp.
Such old-style loops are considered obsolete
and in any case are easily converted to a new-style
\cdf{do} with the insertion of three pairs of parentheses.
In practice the compiler can catch nearly all instances of old-style
\cdf{do} loops because they will not have a legal format anyway.
\end{incompatibility}
\afternoterule

Here are some examples of the use of \cdf{do}:
\begin{lisp}
(do ((i 0 (+ i 1))~~~~~;\textrm{Sets every null element of \cdf{a-vector} to zero} \\*
~~~~~(n (length a-vector))) \\*
~~~~((= i n)) \\
~~(when (null (aref a-vector i)) \\*
~~~~(setf (aref a-vector i) 0)))
\end{lisp}
The construction
\begin{lisp}
(do ((x e (cdr x)) \\*
~~~~~(oldx x x)) \\*
~~~~((null x)) \\*
~~\emph{body})
\end{lisp}
exploits parallel assignment to index variables.  On the first
iteration, the value of \cdf{oldx} is whatever value \cdf{x} had before
the \cdf{do} was entered.  On succeeding iterations, \cdf{oldx} contains
the value that \cdf{x} had on the previous iteration. 

Very often an iterative algorithm can be most clearly expressed entirely
in the \emph{step} forms of a \cdf{do}, and the \emph{body} is empty.
For example,
\begin{lisp}
(do ((x foo (cdr x)) \\*
~~~~~(y bar (cdr y)) \\*
~~~~~(z '{\emptylist} (cons (f (car x) (car y)) z))) \\
~~~~((or (null x) (null y)) \\*
~~~~~(nreverse z)))
\end{lisp}
does the same thing as \cd{(mapcar \#'f foo bar)}.  Note that the \emph{step}
computation for \cdf{z} exploits the fact that variables are stepped in parallel.
Also, the body of the loop is empty.  Finally, the use of \cdf{nreverse}
to put an accumulated \cdf{do} loop result into the correct order
is a standard idiom.  Another example:
\begin{lisp}
(defun list-reverse (list) \\*
~~~~~~~(do ((x list (cdr x)) \\*
~~~~~~~~~~~~(y '{\emptylist} (cons (car x) y))) \\*
~~~~~~~~~~~((endp x) y)))
\end{lisp}
Note the use of \cdf{endp} rather than \cdf{null} or \cdf{atom}
to test for the end of a list; this may result in more robust code.

As an example of nested loops, suppose that \cdf{env} holds a list
of conses.  The \emph{car} of each cons is a list of symbols,
and the \emph{cdr} of each cons is a list of equal length containing
corresponding values.  Such a data structure is similar to an association
list
but is divided into ``frames''; the overall structure resembles a rib cage.
A lookup function on such a data structure might be
\begin{lisp}
(defun ribcage-lookup (sym ribcage) \\*
~~~~~~~(do ((r ribcage (cdr r))) \\*
~~~~~~~~~~~((null r) {\false}) \\
~~~~~~~~~(do ((s (caar r) (cdr s)) \\*
~~~~~~~~~~~~~~(v (cdar r) (cdr v))) \\*
~~~~~~~~~~~~~((null s)) \\
~~~~~~~~~~~(when (eq (car s) sym) \\*
~~~~~~~~~~~~~(return-from ribcage-lookup (car v))))))
\end{lisp}
(Notice the use of indentation in the above example
to set off the bodies of the \cdf{do} loops.)

A \cdf{do} loop may be explained in terms of the more primitive constructs
\cdf{block}, \cdf{return}, \cdf{let}, \cdf{loop}, \cdf{tagbody},
and \cdf{psetq} as follows:
\begin{lisp}
(block nil \\*
~~(let ((\emph{var1} \emph{init1}) \\*
~~~~~~~~(\emph{var2} \emph{init2}) \\
~~~~~~~~... \\*
~~~~~~~~(\emph{varn} \emph{initn})) \\*
~~~~\Mstar{\emph{declaration}} \\
~~~~(loop (when \emph{end-test} (return (progn . \emph{result}))) \\*
~~~~~~~~~~(tagbody . \emph{tagbody}) \\*
~~~~~~~~~~(psetq \emph{var1} \emph{step1} \\*
~~~~~~~~~~~~~~~~~\emph{var2} \emph{step2} \\*
~~~~~~~~~~~~~~~~~... \\*
~~~~~~~~~~~~~~~~~\emph{varn} \emph{stepn}))))
\end{lisp}
\cdf{do*} is exactly like \cdf{do} except that the bindings and steppings
of the variables are performed sequentially rather than in parallel.
It is as if, in the above explanation,
\cdf{let} were replaced by \cdf{let*} and \cdf{psetq} were replaced
by \cdf{setq}.
\end{defmac}

\subsection{Simple Iteration Constructs}

The constructs \cdf{dolist} and \cdf{dotimes} execute a body of code
once for each value taken by a single variable.  They are expressible
in terms of \cdf{do}, but capture very common patterns of use.

Both \cdf{dolist} and \cdf{dotimes} perform
a body of statements repeatedly.  On each iteration a specified
variable is bound to an element of interest that the body may
examine.  \cdf{dolist} examines successive elements of a list,
and \cdf{dotimes} examines integers from 0 to $n-1$
for some specified positive integer \emph{n}.

The value of any of these constructs may be specified by an optional result
form, which if omitted defaults to the value {\false}.

The \cdf{return} statement may be used to return
immediately from a \cdf{dolist} or \cdf{dotimes} form,
discarding any following iterations
that might have been performed; in effect, a \cdf{block} named {\nil}
surrounds the construct.
The body of the loop is implicitly a \cdf{tagbody} construct;
it may contain tags to serve as the targets of \cdf{go} statements.
Declarations may appear before the body of the loop.

\begin{defmac}
dolist (var listform [resultform])
       {declaration}* {tag | statement}*

\cdf{dolist} provides straightforward iteration over the elements of a list.
First \cdf{dolist}
evaluates the form \emph{listform},
which should produce a list.  It then executes the body
once for each element in the list, in order, with
the variable \emph{var} bound to the element.
Then \emph{resultform} (a single form, \emph{not} an implicit \cdf{progn})
is evaluated, and the result is the value of the \cdf{dolist}
form.  (When the \emph{resultform} is evaluated, the control variable \emph{var}
is still bound and has the value {\nil}.)
If \emph{resultform} is omitted, the result is {\false}.

\begin{lisp}
(dolist (x '(a b c d)) (prin1 x) (princ " ")) \EV\ {\false} \\
~~~\textrm{after printing ``\cd{a b c d }'' (note the trailing space)}
\end{lisp}
\end{lisp}

An explicit \cdf{return} statement may be used to terminate the loop
and return a specified value.

\begin{new}
X3J13 voted in January 1989
\issue{MAPPING-DESTRUCTIVE-INTERACTION}
to restrict user side effects; see section~\ref{STRUCTURE-TRAVERSAL-SECTION}.
\end{new}
\end{defmac}

\begin{defmac}
dotimes (var countform [resultform])
        {declaration}* {tag | statement}*

\cdf{dotimes} provides straightforward iteration over a sequence of integers.
The expression
\cd{(dotimes (\emph{var} \emph{countform} \emph{resultform}) . \emph{progbody})}
evaluates the form \emph{countform}, which should produce an integer.  It then
performs \emph{progbody} once for each integer from zero (inclusive) to
\emph{count} (exclusive), in order, with the variable \emph{var} bound to the
integer; if the value of \emph{countform} is zero or negative,
then the \emph{progbody} is
performed zero times.  Finally, \emph{resultform} (a single form, \emph{not} an
implicit \cdf{progn}) is evaluated, and the result is the value of the
\cdf{dotimes} form.  (When the \emph{resultform} is evaluated, the control
variable \emph{var} is still bound and has as its value the number of times
the body was executed.)
If \emph{resultform} is omitted, the result is {\false}.

An explicit \cdf{return} statement may be used to terminate the loop
and return a specified value.

Here is an example of the use of \cdf{dotimes} in processing strings:
\begin{lisp}
;;; True if the specified subsequence of the string is a \\*
;;; palindrome (reads the same forwards and backwards). \\*
\\*
(defun palindromep (string \cd{\&optional} \\*
~~~~~~~~~~~~~~~~~~~~~~~~~~~(start 0) \\*
~~~~~~~~~~~~~~~~~~~~~~~~~~~(end (length string))) \\
~~(dotimes (k (floor (- end start) 2) {\true}) \\*
~~~~(unless (char-equal (char string (+ start k)) \\*
~~~~~~~~~~~~~~~~~~~~~~~~(char string (- end k 1))) \\*
~~~~~~(return {\false})))) \\
\\
(palindromep "Able was I ere I saw Elba") \EV\ {\true} \\
 \\
(palindromep "A man, a plan, a canal--Panama!") \EV\ {\false} \\
 \\
(remove-if-not \#'alpha-char-p~~~~~;\textrm{Remove punctuation} \\*
~~~~~~~~~~~~~~~"A man, a plan, a canal--Panama!") \\*
~~~\EV\ "AmanaplanacanalPanama" \\
 \\
(palindromep \\*
~(remove-if-not \#'alpha-char-p \\*
~~~~~~~~~~~~~~~~"A man, a plan, a canal--Panama!")) \EV\ {\true} \\
 \\
(palindromep \\*
~(remove-if-not \\*
~~~\#'alpha-char-p \\*
~~~"Unremarkable was I ere I saw Elba Kramer, nu?")) \EV\ {\true} \\
 \\
(palindromep \\*
~(remove-if-not \\*
~~~\#'alpha-char-p \\*
~~~"A man, a plan, a cat, a ham, a yak, \\*
~~~~~~~~~~~~~~~~~~~a yam, a hat, a canal--Panama!")) \EV\ {\true}
\\
(palindromep \\*
~(remove-if-not \\*
~~~\#'alpha-char-p \\*
~~~"Ja-da, ja-da, ja-da ja-da jing jing jing")) \EV\ {\false}
\end{lisp}

Altering the value of \emph{var} in the body of the loop (by using \cdf{setq},
for example) will have unpredictable, possibly implementation-dependent
results.  A Common Lisp compiler may choose to issue a warning if such a variable
appears in a \cdf{setq}.

\beforenoterule
\begin{incompatibility}
The \cdf{dotimes} construct is the closest thing
in Common Lisp to the Interlisp \cdf{rptq} construct.
\end{incompatibility}
\afternoterule
\end{defmac}

See also \cdf{do-symbols}, \cdf{do-external-symbols},
and \cdf{do-all-symbols}.

\subsection{Mapping}
\indexterm{mapping}

Mapping is a type of iteration in which a function is 
successively applied to pieces of one or more sequences.
The result of the iteration is a sequence containing the respective
results of the function applications.
There are several options for the way in which the pieces of the list are
chosen and for what is done with the results returned by the applications of
the function.

The function \cdf{map} may be used to map over any kind of sequence.
The following functions operate only on lists.

\begin{defun}[Function]
mapcar function list &rest more-lists \\
maplist function list &rest more-lists \\
mapc function list &rest more-lists \\
mapl function list &rest more-lists \\
mapcan function list &rest more-lists \\
mapcon function list &rest more-lists

For each of these mapping functions,
the first argument is a function and the rest must be lists.
The function must take as many arguments as there are lists.

\cdf{mapcar} operates on successive elements of the lists.
First the function is applied to the \emph{car} of each list,
then to the \emph{cadr} of each list, and so on.
(Ideally all the lists are the same length; if not,
the iteration terminates when the shortest list runs out,
and excess elements in other lists are ignored.)
The value returned by \cdf{mapcar} is a list of the
results of the successive calls to the function.
For example:
\begin{lisp}
(mapcar \#'abs '(3 -4 2 -5 -6)) \EV\ (3 4 2 5 6) \\
(mapcar \#'cons '(a b c) '(1 2 3)) \EV\ ((a . 1) (b . 2) (c . 3))
\end{lisp}

\cdf{maplist} is like \cdf{mapcar} except that the function is applied to
the lists and successive \emph{cdr}'s of those lists rather than to successive
elements of the lists.
For example:
\begin{lisp}
(maplist \#'(lambda (x) (cons 'foo x)) \\*
~~~~~~~~~'(a b c d)) \\*
~~~\EV\ ((foo a b c d) (foo b c d) (foo c d) (foo d))
\end{lisp}

\begin{lisp}
(maplist \#'(lambda (x) (if (member (car x) (cdr x)) 0 1))) \\*
~~~~~~~~~'(a b a c d b c)) \\*
~~~\EV\ (0 0 1 0 1 1 1) \\*
~~~;\textrm{An entry is \cd{1} if the corresponding element of the input} \\*
~~~;~\textrm{list was the last instance of that element in the input list.}
\end{lisp}

\cdf{mapl} and \cdf{mapc} are like \cdf{maplist} and \cdf{mapcar},
respectively, except that they do not accumulate the results
of calling the function.

\beforenoterule
\begin{incompatibility}
In all Lisp systems since Lisp 1.5,
\cdf{mapl} has been called \cdf{map}.  In the chapter on sequences
it is explained why this was a bad choice.  Here the name \cdf{map}
is used for the far more useful generic sequence mapper,
in closer accordance with the computer science literature,
especially the growing body of papers on functional programming.
\begin{new}
Note that this remark, predating the design of the Common Lisp Object System,
uses the term ``generic'' in a generic sense and not necessarily
in the technical sense used by CLOS
(see chapter~\ref{DTYPES}).
\end{new}
\end{incompatibility}
\afternoterule

These functions are used when the function is being called merely for its
side effects rather than for its returned values.
The value returned by \cdf{mapl} or \cdf{mapc} is the second argument,
that is, the first sequence argument.

\cdf{mapcan} and \cdf{mapcon} are like \cdf{mapcar} and \cdf{maplist}, respectively,
except that they combine the results of
the function using \cdf{nconc} instead of \cdf{list}.  That is,
\begin{lisp}
(mapcon \emph{f} \emph{x1} ... \emph{xn}) \\
~~~\EQ\ (apply \#'nconc (maplist \emph{f} \emph{x1} ... \emph{xn}))
\end{lisp}
and similarly for the relationship between \cdf{mapcan} and \cdf{mapcar}.
Conceptually, these functions allow the mapped function to return
a variable number of items to be put into the output list.
This is particularly useful for effectively returning zero or one item:
\begin{lisp}
(mapcan \#'(lambda (x) (and (numberp x) (list x))) \\
~~~~~~~~'(a 1 b c 3 4 d 5)) \\
~~~\EV\ (1 3 4 5)
\end{lisp}
In this case the function serves as a filter; this is a standard Lisp
idiom using \cdf{mapcan}.
(The function \cdf{remove-if-not} might have been useful in this
particular context, however.)
Remember that \cdf{nconc} is a destructive operation, and therefore
so are \cdf{mapcan} and \cdf{mapcon}; the lists returned by the \emph{function}
are altered in order to concatenate them.

Sometimes a \cdf{do} or a straightforward recursion is preferable to a
mapping operation;  however, the mapping functions should be used wherever they
naturally apply because this increases the clarity of the code.

The functional argument to a mapping function must be acceptable
to \cdf{apply}; it cannot be a macro or the name of a special form.
Of course, there is nothing wrong with using a function that has \cd{\&optional}
and \cd{\&rest} parameters as the functional argument.

\begin{newer}
X3J13 voted in June 1988 \issue{FUNCTION-TYPE} to allow the \emph{function}
to be only of type \cdf{symbol} or \cdf{function}; a lambda-expression
is no longer acceptable as a functional argument.  One must use the
\cdf{function} special form or the abbreviation \cd{\#'} before
a lambda-expression that appears as an  explicit argument form.
\end{newer}

\begin{new}
X3J13 voted in January 1989
\issue{MAPPING-DESTRUCTIVE-INTERACTION}
to restrict user side effects; see section~\ref{STRUCTURE-TRAVERSAL-SECTION}.
\end{new}
\end{defun}

\subsection{The ``Program Feature''}

Lisp implementations since Lisp 1.5 have had what was originally
called ``the program feature,'' as if it were impossible to write
programs without it!  The \cdf{prog} construct allows one to
write in an Algol-like or Fortran-like statement-oriented
style, using \cdf{go} statements that can refer to tags in the
body of the \cdf{prog}.  Modern Lisp programming style tends to use
\cdf{prog} rather infrequently.  The various iteration constructs,
such as \cdf{do}, have bodies with the characteristics of a \cdf{prog}.
(However, the ability to use \cdf{go} statements within iteration
constructs is very seldom called upon in practice.)

Three distinct operations are performed by \cdf{prog}: it binds local variables,
it permits use of the \cdf{return} statement, and it permits use of the \cdf{go}
statement.
In Common Lisp, these three operations have been separated into three
distinct constructs: \cdf{let}, \cdf{block}, and \cdf{tagbody}.
These three constructs may be used independently as building blocks
for other types of constructs.

\begin{defspec}
tagbody {tag | statement}*

The part of a \cdf{tagbody} after the variable list is called the \emph{body}.
An item in the body may be a symbol or an integer, in which case it is called a
\emph{tag}, or an item in the body may be a list, in which case it is called a 
\emph{statement}.

Each element of the body is processed from left to right.
A \emph{tag} is ignored; a \emph{statement}
is evaluated, and its results are discarded.  If the end of the body
is reached, the \cdf{tagbody} returns {\false}.

If \cd{(go \emph{tag})} is evaluated, control jumps to the part of the body
labelled with the \emph{tag}.

\beforenoterule
\begin{incompatibility}
The ``computed \cdf{go}'' feature of MacLisp is not
supported.  The syntax of a computed \cdf{go} is idiosyncratic,
and the feature is not supported by Lisp Machine Lisp, NIL (New Implementation of Lisp), or Interlisp.
The computed \cdf{go} has been infrequently used in MacLisp anyway
and is easily simulated with no loss of
efficiency by using a \cdf{case} statement each of whose
clauses performs a (non-computed) \cdf{go}.
\end{incompatibility}
\afternoterule

The scope of the tags established by a \cdf{tagbody} is lexical,
and the extent is dynamic.  Once a \cdf{tagbody} construct has
been exited, it is no longer legal to \cdf{go} to a \emph{tag} in its body.
It is permissible for a \cdf{go} to jump to a \cdf{tagbody} that is not
the innermost \cdf{tagbody} construct containing that \cdf{go};
the tags established by a \cdf{tagbody} will only shadow other tags
of like name.

The lexical scoping of the \cdf{go} targets named by tags is
fully general and has consequences that may be surprising
to users and implementors of other Lisp systems.
For example, the \cdf{go} in the following example actually does
work in Common Lisp as one might expect:
\begin{lisp}
(tagbody \\*
~~~(catch 'stuff \\*
~~~~~~(mapcar \#'(lambda (x) (if (numberp x) \\*
~~~~~~~~~~~~~~~~~~~~~~~~~~~~~~~~(hairyfun x) \\*
~~~~~~~~~~~~~~~~~~~~~~~~~~~~~~~~(go lose))) \\*
~~~~~~~~~~~~~~items)) \\
~~~(return) \\*
~lose \\*
~~~(error "I lost big!"))
\end{lisp}
Depending on the situation, a \cdf{go} in Common Lisp does not necessarily
correspond to a simple machine ``jump'' instruction.
A \cdf{go} can break up catchers if necessary to get
to the target.  It is possible for a ``closure'' created by \cdf{function} for
a lambda-expression to refer to a \cdf{go} target as long as the tag
is lexically apparent.  See chapter~\ref{SCOPE} for an elaborate
example of this.

\begin{new}
There are some holes in this specification (and that of \cdf{go}) that
leave some room for interpretation.  For example, there is no explicit
prohibition against the same tag appearing more than once in the same
\cdf{tagbody} body.  Every implementation I know of will complain
in the compiler, if not in the interpreter, if there is a \cdf{go} to
such a duplicated tag; but some implementors take the position
that duplicate tags are permitted provided there is no \cdf{go} to
such a tag.  (``If a tree falls in the forest, and there is no one
there to hear it, then no one needs to yell `Timber!'\thinspace'')
Also, some implementations allow objects other than symbols, integers,
and lists in the body and typically ignore them.
Consequently,
some programmers use redundant tags such as \cdf{---} for formatting purposes,
and strings as comments:
\begin{lisp}
(defun dining-philosopher (j) \\*
~~(tagbody --- \\*
~~~think~~~(unless (hungry) (go think)) \\*
~~~~~~~~~~~--- \\
~~~~~~~~~~~"Can't eat without chopsticks." \\*
~~~~~~~~~~~(snatch (chopstick j)) \\*
~~~~~~~~~~~(snatch (chopstick (mod (+ j 1) 5))) \\*
~~~~~~~~~~~--- \\
~~~eat~~~~~(when (hungry) \\*
~~~~~~~~~~~~~(mapc \#'gobble-down \\*
~~~~~~~~~~~~~~~~~~~'(twice-cooked-pork kung-pao-chi-ding \\*
~~~~~~~~~~~~~~~~~~~~~wu-dip-har orange-flavor-beef \\*
~~~~~~~~~~~~~~~~~~~~~two-side-yellow-noodles twinkies)) \\*
~~~~~~~~~~~~~(go eat)) \\*
~~~~~~~~~~~--- \\
~~~~~~~~~~~"Can't think with my neighbors' stomachs rumbling." \\*
~~~~~~~~~~~(relinquish (chopstick j)) \\*
~~~~~~~~~~~(relinquish (chopstick (mod (+ j 1) 5))) \\*
~~~~~~~~~~~--- \\*
~~~~~~~~~~~(if (happy) (go think) \\*
~~~~~~~~~~~~~~~(become insurance-salesman))))
\end{lisp}
In certain implementations of Common Lisp they get away with it.
Others abhor what they view as an abuse of unintended ambiguity
in the language specification.  For maximum portability, I advise
users to steer clear of these issues.  Similarly, it is best
to avoid using \cdf{nil} as a tag, even though it is a symbol, because
a few implementations treat \cdf{nil} as a list to be executed.
To be extra careful, avoid calling from within a \cdf{tagbody}
a macro whose expansion might not be a non-\cdf{nil} list; wrap such a 
call in \cd{(progn~...)}, or rewrite the macro to return \cd{(progn~...)}
if possible.
\end{new}
\end{defspec}

\begin{defmac}
prog ({var | (var [init])}*) {declaration}* {tag | statement}* \\
prog* ({var | (var [init])}*) {declaration}* {tag | statement}*

The \cdf{prog} construct is a synthesis of \cdf{let}, \cdf{block},
and \cdf{tagbody}, allowing bound variables and the use of \cdf{return} and \cdf{go}
within a single construct.  A typical \cdf{prog} construct looks like this:
\begin{lisp}
(prog (\emph{var1} \emph{var2} (\emph{var3} \emph{init3}) \emph{var4} (\emph{var5} \emph{init5})) \\*
~~~~~~\Mstar{\emph{declaration}} \\*
~~~~~~\emph{statement1} \\
~\emph{tag1} \\*
~~~~~~\emph{statement2} \\*
~~~~~~\emph{statement3} \\*
~~~~~~\emph{statement4} \\
~\emph{tag2} \\*
~~~~~~\emph{statement5} \\*
~~~~~~... \\*
~~~~~~)
\end{lisp}
The list after the keyword \cdf{prog} is a set of specifications for binding
\emph{var1}, \emph{var2}, etc.,
which are temporary variables bound locally to the \cdf{prog}.
This list is processed exactly as the list in a \cdf{let} statement:
first all the \emph{init} forms are evaluated from left to right
(where {\false} is used for
any omitted \emph{init} form), and then the variables are all bound in
parallel to the respective results.
Any \emph{declaration} appearing in the \cdf{prog} is used as if appearing
at the top of the \cdf{let} body.

The body of the \cdf{prog} is executed as if it were a \cdf{tagbody}
construct; the \cdf{go} statement may be used to transfer control
to a \emph{tag}.

A \cdf{prog} implicitly establishes a \cdf{block} named {\nil} around
the entire \cdf{prog} construct, so that \cdf{return} may be used
at any time to exit from the \cdf{prog} construct.

Here is a fine example of what can be done with \cdf{prog}:
\begin{lisp}
(defun king-of-confusion (w) \\
~~"Take a cons of two lists and make a list of conses. \\
~~~Think of this function as being like a zipper." \\
~~(prog (x y z)~~~~~;\textrm{Initialize \cdf{x}, \cdf{y}, \cdf{z} to {\false}} \\
~~~~~~~~(setq y (car w) z (cdr w)) \\
~~~loop \\
~~~~~~~~(cond ((null y) (return x)) \\
~~~~~~~~~~~~~~((null z) (go err))) \\
~~~rejoin \\
~~~~~~~~(setq x (cons (cons (car y) (car z)) x)) \\
~~~~~~~~(setq y (cdr y) z (cdr z)) \\
~~~~~~~~(go loop) \\
~~~err \\
~~~~~~~~(cerror "Will self-pair extraneous items" \\
~~~~~~~~~~~~~~~~"Mismatch - gleep!  ~S" y) \\
~~~~~~~~(setq z y) \\
~~~~~~~~(go rejoin)))
\end{lisp}
which is accomplished somewhat more perspicuously by
\begin{lisp}
(defun prince-of-clarity (w) \\
~~"Take a cons of two lists and make a list of conses. \\
~~~Think of this function as being like a zipper." \\
~~(do ((y (car w) (cdr y)) \\
~~~~~~~(z (cdr w) (cdr z)) \\
~~~~~~~(x '{\emptylist} (cons (cons (car y) (car z)) x))) \\
~~~~~~((null y) x) \\
~~~~(when (null z) \\
~~~~~~(cerror "Will self-pair extraneous items" \\
~~~~~~~~~~~~~~"Mismatch - gleep!  ~S" y) \\
~~~~~~(setq z y))))
\end{lisp}

The \cdf{prog} construct may be explained in terms of the simpler
constructs \cdf{block}, \cdf{let}, and \cdf{tagbody} as
follows:
\begin{lisp}
(prog \emph{variable-list} \Mstar{\emph{declaration}} . \emph{body}) \\
~~~\EQ\ (block nil (let \emph{variable-list} \Mstar{\emph{declaration}} (tagbody . \emph{body})))
\end{lisp}

The \cdf{prog*} special form is almost the same as \cdf{prog}.  The only
difference is that the binding and initialization of the temporary
variables is done \emph{sequentially}, so that the \emph{init} form for each
one can use the values of previous ones.
Therefore \cdf{prog*} is to \cdf{prog} as \cdf{let*} is to \cdf{let}.
For example,
\begin{lisp}
(prog* ((y z) (x (car y))) \\
~~~~~~~(return x))
\end{lisp}
returns the \emph{car} of the value of \cdf{z}.

\begin{new}
I haven't seen \cdf{prog} used very much in the last several years.
Apparently splitting it into functional constituents
(\cdf{let}, \cdf{block}, \cdf{tagbody}) has been a success.  Common Lisp
programmers now tend to use whichever specific construct is appropriate.
\end{new}
\end{defmac}

\begin{defspec}
go tag

The \cd{(go \emph{tag})} special form is used to do a ``go to'' within
a \cdf{tagbody} construct.  The \emph{tag} must be a symbol or an integer;
the \emph{tag} is not evaluated.
\cdf{go} transfers control to the point in the body labelled by a
tag \cdf{eql} to the one given.  If there is no such tag in the body, the
bodies of lexically containing \cdf{tagbody} constructs
(if any) are examined as well.
It is an error if there is no matching tag lexically visible
to the point of the \cdf{go}.

The \cdf{go} form does not ever return a value.

As a matter of style, it is recommended that the user think twice before
using a \cdf{go}.  Most purposes of \cdf{go} can be accomplished with one of
the iteration primitives, nested conditional forms, or
\cdf{return-from}.  If the use of \cdf{go} seems to be unavoidable,
perhaps the control structure implemented by \cdf{go} should be packaged
as a macro definition.
\end{defspec}


\begin{new}
\section{Structure Traversal and Side Effects}
\label{STRUCTURE-TRAVERSAL-SECTION}

X3J13 voted in January 1989 \issue{MAPPING-DESTRUCTIVE-INTERACTION}
to restrict side effects during the course
of a built-in operation that can execute user-supplied code while
traversing a data structure.

Consider the following example:
\begin{lisp}
(let ((x '(apples peaches pumpkin pie))) \\*
~~(dolist (z x) \\*
~~~~(when (eq z 'peaches) \\*
~~~~~~(setf (cddr x) '(mango kumquat))) \\*
~~~~(format t "~S " (car z))))
\end{lisp}
Depending on the details of the implementation of \cdf{dolist},
this bit of code could easily print
\begin{lisp}
apples peaches mango kumquat
\end{lisp}
(which is perhaps what was intended), but it might as easily print
\begin{lisp}
apples peaches pumpkin pie
\end{lisp}
Here is a plausible implementation of \cdf{dolist} that
produces the first result:
\begin{lisp}
(defmacro dolist ((var listform \&optional (resultform ''nil)) \\*
~~~~~~~~~~~~~~~~~~\&body body) \\*
~~(let ((tailvar (gensym "DOLIST"))) \\*
~~~~{\Xbq}(do ((,tailvar ,listform (cdr ,tailvar))) \\*
~~~~~~~~~((null ,tailvar) ,resultform) \\*
~~~~~~~(let ((,var (car ,tailvar))) ,@body))
\end{lisp}
But here is a plausible implementation of \cdf{dolist} that
produces the second result:
\begin{lisp}
(defmacro dolist ((var listform \&optional (resultform ''nil)) \\*
~~~~~~~~~~~~~~~~~~\&body body) \\*
~~(let ((tailvar (gensym "DOLIST"))) \\*
~~~~{\Xbq}(do ((,tailvar ,listform)) \\*
~~~~~~~~~((null ,tailvar) ,resultform) \\*
~~~~~~~(let ((,var (pop ,tailvar))) ,@body))
\end{lisp}

The X3J13 recognizes and legitimizes varying implementation practices:
in general it is an error for code executed during a ``structure-traversing''
operation to destructively modify the structure in a way that might
affect the ongoing traversal operation.  The committee identified in particular
the following special cases.

For list traversal operations, the \emph{cdr} chain
may not be destructively modified.

For array traversal operations, the array may not be adjusted
(see \cdf{adjust-array}) and its fill pointer, if any, may not be modified.

For hash table operations (such as \cdf{with-hash-table-iterator}
and \cdf{maphash}), new entries may not be added or deleted,
\emph{except} that the very entry being processed by user code
may be changed or deleted.

For package symbol operations (for example, \cdf{with-package-iterator}
and \cdf{do-symbols}), new symbols may not be interned in,
nor symbols uninterned from, the packages being traversed or
any packages they use, \emph{except} that the very symbol
being processed by user code may be uninterned.

X3J13 noted that this vote is intended to clarify restrictions
on the use of structure traversal operations that are not themselves
inherently destructive; for example, it applies to \cdf{map} and \cdf{dolist}.
Destructive operators such as \cdf{delete} require even more complicated
restrictions and are addressed by a separate proposal.

The X3J13 vote did not specify a complete list of the operations to which these
restrictions apply.  Table~\ref{TRAVERSAL-OPERATIONS-TABLE}
shows what I believe to be a complete list of operations
that traverse structures and take user code as a body (in the case of
macros) or as a functional argument (in the case of functions).

In addition, note that user code should not modify list
structure that might be undergoing interpretation by the evaluator,
whether explicitly invoked via \cdf{eval} or implicitly invoked,
for example as in the case of a hook function (a \cdf{defstruct}
print function, the value of \cdf{*evalhook*} or \cdf{*applyhook*}, etc.)
that happens to be a closure of interpreted code.  Similarly, \cdf{defstruct}
print functions and other hooks should not perform side effects
on data structures being printed or being processed by \cdf{format}, or on a
string given to \cdf{make-string-input-stream}.  You get the idea;
be sensible.

Note that an operation such as \cdf{mapcar} or \cdf{dolist} traverses
not only \emph{cdr} pointers (in order to chase down the list)
but also \emph{car} pointers (in order to obtain the elements themselves).
The restriction against modification appears to apply to all these pointers.
\end{new}

\begin{table}[t]
\begin{new}
\leavevmode
\vtop{
\caption{Structure Traversal Operations Subject to Side Effect Restrictions}
\label{TRAVERSAL-OPERATIONS-TABLE}
}

\begingroup\cf \tabcolsep0pt\relax
\begin{tabular*}{\textwidth}{@{}l@{\extracolsep{\fill}}ll@{}}
adjoin & maphash & reduce \\
assoc & mapl & remove \\
assoc-if & maplist & remove-duplicates \\
assoc-if-not & member & remove-if \\
count & member-if & remove-if-not \\
count-if & member-if-not & search \\
count-if-not & merge & set-difference \\
delete & mismatch & set-exclusive-or \\
delete-duplicates & nintersection & some \\
delete-if & notany & sort \\
delete-if-not & notevery & stable-sort \\
do-all-symbols & nset-difference & sublis \\
do-external-symbols & nset-exclusive-or & subsetp \\
do-symbols & nsublis & subst \\
dolist & nsubst & subst-if \\
eval & nsubst-if & subst-if-not \\
every & nsubst-if-not & substitute \\
find & nsubstitute & substitute-if \\
find-if & nsubstitute-if & substitute-if-not \\
find-if-not & nsubstitute-if-not & tree-equal \\
intersection & nunion & union \\
\textrm{certain} loop \textrm{clauses} & position & with-hash-table-iterator \\
map & position-if & with-input-from-string \\
mapc & position-if-not & with-output-to-string \\
mapcan & rassoc & with-package-iterator \\
mapcar & rassoc-if \\
mapcon & rassoc-if-not
\end{tabular*}
\endgroup
\end{new}
\end{table}


\section{Multiple Values}
\indexterm{multiple values}

Ordinarily the result of calling a Lisp function is a single Lisp object.
Sometimes, however, it is convenient for a function to compute several
objects and return them.
Common Lisp provides a mechanism for handling multiple values directly.
This mechanism is cleaner and more efficient than the usual tricks
involving returning a list of results or stashing results in global
variables.

\subsection{Constructs for Handling Multiple Values}

Normally multiple values are not used.  Special forms are
required both to \emph{produce} multiple values and to \emph{receive} them.
If the caller of a function does not request multiple values,
but the called function produces multiple values, then the first
value is given to the caller and all others are discarded;
if the called function produces zero values, then the caller gets {\false}
as a value.

The primary primitive for producing multiple values is \cdf{values},
which takes any number of arguments and returns that many values.  If the
last form in the body of a function is a \cdf{values} with three arguments,
then a call to that function will return three values.  Other special
forms also produce multiple values, but they can be described in terms of
\cdf{values}.  Some built-in Common Lisp functions, such as \cdf{floor}, return
multiple values; those that do are so documented.

The special forms and macros for receiving multiple values are as follows:
\begin{lisp}
multiple-value-list \\
multiple-value-call \\
multiple-value-prog1 \\
multiple-value-bind \\
multiple-value-setq
\end{lisp}
These specify a form to evaluate and an indication of where to put
the values returned by that form.

\begin{defun}[Function]
values &rest args

All of the arguments are returned, in order, as values.
For example:
\begin{lisp}
(defun polar (x y) \\
~~(values (sqrt (+ (* x x) (* y y))) (atan y x))) \\
 \\
(multiple-value-bind (r theta) (polar 3.0 4.0) \\
~~(vector r theta)) \\
~~~\EV\ \#(5.0 0.9272952)
\end{lisp}

The expression \cd{(values)} returns zero values.  This is the standard idiom
for returning no values from a function.

Sometimes it is desirable to indicate explicitly that a function will return
exactly one value.  For example, the function
\begin{lisp}
(defun foo (x y) \\
~~(floor (+ x y) y))
\end{lisp}
will return two values because \cdf{floor} returns
two values.  It may be that the second value makes no sense,
or that for efficiency reasons it is desired not to compute the
second value.  The \cdf{values} function is the standard idiom
for indicating that only one value is to be returned,
as shown in the following example.
\begin{lisp}
(defun foo (x y) \\
~~(values (floor (+ x y) y)))
\end{lisp}
This works because \cdf{values} returns exactly \emph{one} value for each of
its argument forms; as for any function call,
if any argument form to \cdf{values} produces more than one value, all but the
first are discarded.

There is absolutely no way in Common Lisp for a caller to distinguish between
returning a single value in the ordinary manner and returning
exactly one ``multiple value.''  For example, the values returned
by the expressions \cd{(+~1~2)} and \cd{(values (+~1~2))} are identical
in every respect: the single value \cd{3}.
\end{defun}

\begin{defun}[Constant]
multiple-values-limit

The value of \cdf{multiple-values-limit} is a positive integer that is
the upper exclusive bound on the number of values that may
be returned from a function.  This bound depends on the implementation
but will not be smaller than 20.
(Implementors are encouraged to make this limit as large as practicable
without sacrificing performance.)
See \cdf{lambda-parameters-limit} and \cdf{call-arguments-limit}.
\end{defun}

\begin{defun}[Function]
values-list list

All of the elements of \emph{list} are returned as multiple values.
For example:
\begin{lisp}
(values-list (list a b c)) \EQ\ (values a b c)
\end{lisp}
In general,
\begin{lisp}
(values-list \emph{list}) \EQ\ (apply \#'values \emph{list})
\end{lisp}
but \cdf{values-list} may be clearer or more efficient.
\end{defun}

\begin{defmac}
multiple-value-list form

\cdf{multiple-value-list} evaluates \emph{form} and returns a list of
the multiple values it returned.
For example:
\begin{lisp}
(multiple-value-list (floor -3 4)) \EV\ (-1 1)
\end{lisp}
\cdf{multiple-value-list} and \cdf{values-list} are therefore inverses
of each other.
\end{defmac}

\begin{defspec}
multiple-value-call function {\,form}*

\cdf{multiple-value-call} first evaluates \emph{function} to obtain a function
and then evaluates all of the \emph{form\/}s.  All the values
of the \emph{form\/}s are gathered together (not just one value from each)
and are all given as arguments to the function.  The result of \cdf{multiple-value-call}
is whatever is returned by the function.
For example:
\begin{lisp}
(+ (floor 5 3) (floor 19 4)) \\
~~~\EQ\ (+ 1 4) \EV\ 5 \\
(multiple-value-call \#'+ (floor 5 3) (floor 19 4)) \\
~~~\EQ\ (+ 1 2 4 3) \EV\ 10 \\
(multiple-value-list \emph{form}) \EQ\ (multiple-value-call \#'list \emph{form})
\end{lisp}
\end{defspec}

\begin{defspec}
multiple-value-prog1 form {\,form}*

\cdf{multiple-value-prog1} evaluates the first \emph{form} and saves all the values
produced by that form.  It then evaluates the other \emph{form}s
from left to right, discarding their values.  The values produced
by the first \emph{form} are returned by \cdf{multiple-value-prog1}.  See \cdf{prog1},
which always returns a single value.
\end{defspec}

\begin{defmac}
multiple-value-bind ({var}*) values-form
                    {declaration}* {\,form}*

The \emph{values-form} is evaluated, and each of the variables \emph{var} is
bound to the respective value returned by that form.  If there are more
variables than values returned, extra values of {\false} are given to the
remaining variables.  If there are more values than variables, the excess
values are simply discarded.  The variables are bound to the values over
the execution of the forms, which make up an implicit \cdf{progn}.
For example:
\begin{lisp}
(multiple-value-bind (x) (floor 5 3) (list x)) \EV\ (1) \\
(multiple-value-bind (x y) (floor 5 3) (list x y)) \EV\ (1 2) \\
(multiple-value-bind (x y z) (floor 5 3) (list x y z)) \\
~~~\EV\ (1 2 {\false})
\end{lisp}
\end{defmac}

\begin{defmac}
multiple-value-setq variables form

The \emph{variables} must be a list of variables.  The \emph{form} is
evaluated, and the variables are \emph{set} (not bound) to the values
returned by that form.  If there are more variables than values returned,
extra values of {\false} are assigned to the remaining variables.  If there
are more values than variables, the excess values are simply discarded.

\beforenoterule
\begin{incompatibility}
In Lisp Machine Lisp this is called \cdf{multiple-value}.
The added clarity of the name \cdf{multiple-value-setq} in Common Lisp was deemed
worth the incompatibility with Lisp Machine Lisp.
\end{incompatibility}
\afternoterule

\newpage%manual

\cdf{multiple-value-setq} always returns a single value, which is the first
value returned by \emph{form}, or {\false} if \emph{form} produces zero values.

\begin{newer}
X3J13 voted in March 1989
\issue{SYMBOL-MACROLET-SEMANTICS} to specify that if any \emph{var}
refers not to an ordinary variable but to a binding made by
\cdf{symbol-macrolet}, then that \emph{var} is handled as
if \cdf{setq} were used to assign the appropriate value to it.
\end{newer}
\end{defmac}


\begin{new}
\begin{defmac}*
nth-value n form

X3J13 voted in January 1989
\issue{NTH-VALUE}
to add a new macro named \cdf{nth-value}.
The argument forms \emph{n} and \emph{form} are both evaluated.
The value of \emph{n} must be a non-negative integer,
and the \emph{form} may produce any number of values.
The integer \emph{n} is used as a zero-based index into the
list of values.
Value \emph{n} of the \emph{form} is returned as the
single value of the \cdf{nth-value} form; \cdf{nil} is returned if the
\emph{form} produces no more than \emph{n} values.

As an example, \cdf{mod} could be defined as
\begin{lisp}
(defun mod (number divisor) \\*
~~(nth-value 1 (floor number divisor)))
\end{lisp}
Value number 1 is the \emph{second} value returned by \cdf{floor},
the first value being value number 0.

One could define \cdf{nth-value} simply as
\begin{lisp}
(defmacro nth-value (n form) \\*
~~{\Xbq}(nth ,n (multiple-value-list ,form)))
\end{lisp}
but the clever implementor will doubtless find an implementation
technique for \cdf{nth-value} that avoids constructing an intermediate
list of all the values of the \emph{form}.
\end{defmac}
\end{new}

\subsection{Rules Governing the Passing of Multiple Values}

It is often the case that the value
of a special form or macro call
is defined to be the value of one of its subforms.
For example, the
value of a \cdf{cond} is the value of the last form in the selected clause.

In most such cases, if the subform produces multiple values, then the original
form will also produce all of those values.
This \emph{passing back} of
multiple values of course has no effect unless eventually one of the
special forms for receiving multiple values is reached.

To be explicit, multiple values can result from a special form
under precisely these circumstances:
\begin{flushdesc}
\item[\emph{Evaluation and application}]\leavevmode
\begin{itemize}
\item
\cdf{eval} returns multiple values if the form given it to
evaluate produces multiple values.

\item
\cdf{apply}, \cdf{funcall}, and \cdf{multiple-value-call}
pass back multiple values from the function applied or called.
\end{itemize}

\item[\emph{Implicit \cdf{progn} contexts}]\leavevmode
\begin{itemize}
\item
The special form \cdf{progn}
passes back multiple values resulting from evaluation of the
last subform.  Other situations referred to as ``implicit \cdf{progn},''
where several forms are evaluated and the results of all but the last form
are discarded, also pass back multiple values from the last form.
These situations include the body of a lambda-expression,
in particular those constructed by \cdf{defun},
\cdf{defmacro}, and \cdf{deftype}.
Also included are bodies of the constructs
\cdf{eval-when},
\cdf{progv}, \cdf{let},
\cdf{let*}, \cdf{when}, \cdf{unless},
\cdf{block},
\cdf{multiple-value-bind}, and \cdf{catch},
as well as clauses in such conditional
constructs as
\cdf{case}, \cdf{typecase},
\cdf{ecase}, \cdf{etypecase}, \cdf{ccase}, and \cdf{ctypecase}.
\end{itemize}
\end{flushdesc}

\begin{new}
X3J13 has voted to add many new constructs to the language that contain
implicit \cdf{progn} contexts.  I won't attempt to list them all here.
Of particular interest, however, is \cdf{locally}, which may be regarded
as simply a version of \cdf{progn} that permits declarations before its
body.  This provides a useful building block for constructing macros
that permit declarations (but not documentation strings)
before their bodies and pass back any multiple values
produced by the last sub-form of a body.  (If a body can contain a documentation
string, most likely \cdf{lambda} is the correct building block to use.)
\end{new}

\begin{flushdesc}
\item[\emph{Conditional constructs}]\leavevmode
\begin{itemize}
\item
\cdf{if} passes back multiple values from whichever subform is selected
(the \emph{then} form or the \emph{else} form).

\item
\cdf{and} and \cdf{or} pass back multiple values from the last subform
but not from subforms other than the last.

\item
\cdf{cond} passes back multiple values from the last subform of
the implicit \cdf{progn} of the selected clause.
If, however, the clause selected is a singleton clause,
then only a single value (the non-{\false} predicate value)
is returned.  This is true even if the singleton clause is
the last clause of the \cdf{cond}.  It is \emph{not} permitted to
treat a final clause \cd{(x)} as being the same as \cd{(t x)}
for this reason; the latter passes back multiple values from the form \cdf{x}.
\end{itemize}

\item[\emph{Returning from a block}]\leavevmode
\begin{itemize}
\item
The \cdf{block} construct passes back multiple values from its last subform
when it exits normally.  If \cdf{return-from} (or \cdf{return}) is
used to terminate the \cdf{block} prematurely, then \cdf{return-from}
passes back multiple values from its subform as the values of the
terminated \cdf{block}.  Other constructs that create implicit blocks,
such as
\cdf{do}, \cdf{dolist}, \cdf{dotimes}, \cdf{prog}, and
\cdf{prog*}, also pass back multiple values specified by
\cdf{return-from} (or \cdf{return}).

\item
\cdf{do} passes back multiple values from
the last form of the exit clause, exactly as if the exit clause
were a \cdf{cond} clause.  Similarly, \cdf{dolist} and \cdf{dotimes}
pass back multiple values from the \emph{resultform} if that is executed.
These situations are all examples of implicit uses of \cdf{return-from}.
\end{itemize}

\item[\emph{Throwing out of a catch}]\leavevmode
\begin{itemize}
\item
The \cdf{catch} construct returns multiple values if
the result form in a \cdf{throw} exiting from
such a catch produces multiple values.
\end{itemize}

\item[\emph{Miscellaneous situations}]\leavevmode
\begin{itemize}
\item
\cdf{multiple-value-prog1} passes back multiple values from its first
subform.  However, \cdf{prog1} always returns a single value.

\item
\cdf{unwind-protect} returns multiple values if the form it protects
returns multiple values.

\item
\cdf{the} returns multiple values if the form it contains returns
multiple values.
\end{itemize}
\end{flushdesc}

Among special forms that \emph{never} pass back multiple values are
\cdf{prog1},
\cdf{prog2}, \cdf{setq}, and \cdf{multiple-value-setq}.
The conventional way to force only one value to be returned from a form \cdf{x}
is to write \cd{(values x)}.

The most important rule about multiple values is:
\textbf{No matter how many values a form produces,
if the form is an argument form in a function call,
then exactly one value (the first one) is used.}

For example, if you write \cd{(cons (floor x))}, then \cdf{cons} will always
receive \emph{exactly} one argument (which is of course an error),
even though \cdf{floor} returns two values.  To pass both values from \cdf{floor}
to \cdf{cons}, one must write something like
\cd{(multiple-value-call \#'cons (floor x))}.
In an ordinary function call,
each argument form produces exactly \emph{one} argument;  if such a form
returns zero values, {\false} is used for the argument, and if more than one
value, all but the first are discarded.
Similarly, conditional constructs such as \cdf{if} that test the value of a form
will use exactly one value, the first one, from that form and discard the rest;
such constructs will use {\false} as the test value if zero values are returned.

\section{Dynamic Non-Local Exits}
\label{CATCH-THROW-SECTION}
\indexterm{non-local exit}
\indexterm{dynamic exit}
\indexterm{catch}
\indexterm{throw}

Common Lisp provides a facility for exiting from a complex process
in a non-local, dynamically scoped manner.  There are two classes of
special forms for this purpose, called \emph{catch} forms and \emph{throw}
forms, or simply \emph{catches} and \emph{throws}.  A catch form evaluates some
subforms in such a way that, if a throw form is executed during such
evaluation, the evaluation is aborted at that point and the catch form
immediately returns a value specified by the throw.  Unlike \cdf{block}
and \cdf{return} (section~\ref{BLOCK-RETURN-SECTION}),
which allow for exiting a \cdf{block} form from any
point lexically within the body of the \cdf{block}, the catch/throw
mechanism works even if the throw form is not textually within the body
of the catch form.  The throw need only occur within the extent (time
span) of the evaluation of the body of the catch.  This is analogous to
the distinction between dynamically bound (special) variables and
lexically bound (local) variables.

\begin{defspec}
catch tag {\,form}*

The \cdf{catch} special form serves as a target for transfer
of control by \cdf{throw}.
The form \emph{tag} is evaluated first to produce an object
that names the catch; it may be any Lisp object.
A catcher is then established with the object as the tag.
The \emph{form\/}s are evaluated as an implicit \cdf{progn},
and the results of the last form are returned,
except that if during the evaluation of the \emph{form\/}s
a throw should be executed such that the tag
of the throw matches (is \cdf{eq} to) the tag of the \cdf{catch}
and the catcher is the most recent outstanding catcher with that tag,
then the evaluation of the \emph{form\/}s is aborted and the results
specified by the throw
are immediately returned from the \cdf{catch} expression.
The catcher established by the \cdf{catch} expression is disestablished
just before the results are returned.

The tag is used to match throws with catches.
\cd{(catch 'foo \emph{form})} will catch a \cd{(throw 'foo \emph{form})} but
not a \cd{(throw 'bar \emph{form})}.  It is an error if \cdf{throw} is done
when there is no suitable \cdf{catch} ready to catch it.

Catch tags are compared using \cdf{eq},
not \cdf{eql}; therefore numbers and characters
should not be used as catch tags.

\beforenoterule
\begin{incompatibility}
The name \cdf{catch} comes from MacLisp,
but the syntax of \cdf{catch} in Common Lisp is different.
The MacLisp syntax was \cd{(catch \emph{form} \emph{tag})},
where the \emph{tag} was not evaluated.
\end{incompatibility}
\afternoterule
\end{defspec}

\indexterm{unwind protection}
\indexterm{cleanup handler}
\begin{defspec}
unwind-protect protected-form {cleanup-form}*

Sometimes it is necessary to evaluate a form and make sure that
certain side effects take place after the form is evaluated;
a typical example is
\begin{lisp}
(progn (start-motor) \\*
~~~~~~~(drill-hole) \\*
~~~~~~~(stop-motor))
\end{lisp}
The non-local exit facility of Common Lisp creates a situation in which
the above code won't work, however: if \cdf{drill-hole} should
do a throw to a catch that is outside of the \cdf{progn}
form (perhaps because the drill bit broke),
then \cd{(stop-motor)} will never be evaluated
(and the motor will presumably be left running).
This is particularly likely if \cdf{drill-hole} causes a Lisp error
and the user tells the error-handler to give up and abort
the computation.
(A possibly more practical example might be
\begin{lisp}
(prog2 (open-a-file) \\*
~~~~~~~(process-file) \\*
~~~~~~~(close-the-file))
\end{lisp}
where it is desired always to close the file when the computation
is terminated for whatever reason.  This case is so important
that Common Lisp provides the special form \cdf{with-open-file} for
this purpose.)

In order to allow the example hole-drilling program to work, it can
be rewritten using \cdf{unwind-protect} as follows:
\begin{lisp}
;; Stop the motor no matter what (even if it failed to start). \\*
\\*
(unwind-protect \\*
~~(progn (start-motor) \\*
~~~~~~~~~(drill-hole)) \\*
~~(stop-motor))
\end{lisp}
If \cdf{drill-hole} does a throw that attempts to quit out of the
\cdf{unwind-protect}, then \cd{(stop-motor)} will be executed.

This example assumes that it is correct to call \cdf{stop-motor}
even if the motor has not yet been started.  Remember that
an error or interrupt may cause an exit even before any initialization
forms have been executed.  Any state restoration code
should operate correctly no matter where in the protected code an
exit occurred.  For example, the following code
is not correct:
\begin{lisp}
(unwind-protect \\*
~~(progn (incf *access-count*) \\*
~~~~~~~~~(perform-access)) \\*
~~(decf *access-count*))
\end{lisp}
If an exit occurs before completion of the \cdf{incf} operation
the \cdf{decf} operation will be executed anyway, resulting in an
incorrect value for \cd{*access-count*}.
The correct way to code this is as follows:
\begin{lisp}
(let ((old-count *access-count*)) \\
~~(unwind-protect \\
~~~~(progn (incf *access-count*) \\
~~~~~~~~~~~(perform-access)) \\
~~~~(setq *access-count* old-count)))
\end{lisp}

As a general rule, \cdf{unwind-protect} guarantees to execute
the \emph{cleanup-form\/}s before exiting, whether it terminates
normally or is aborted by a throw of some kind.
(If, however, an exit occurs during execution of the \emph{cleanup-form\/}s,
no special action is taken.  The \emph{cleanup-form\/}s of an \cdf{unwind-protect}
are not protected by that \cdf{unwind-protect}, though they may be
protected if that \cdf{unwind-protect} occurs within the protected
form of another \cdf{unwind-protect}.)
\cdf{unwind-protect} returns whatever results from evaluation of
the \emph{protected-form} and discards all the results
from the \emph{cleanup-form\/}s.

It should be emphasized that \cdf{unwind-protect} protects against
\emph{all} attempts to exit from the protected form,
including not only ``dynamic exit'' facilities such as \cdf{throw}
but also ``lexical exit'' facilities such as \cdf{go} and
\cdf{return-from}.  Consider this situation:
\begin{lisp}
(tagbody \\
~~(let ((x 3)) \\
~~~~(unwind-protect \\
~~~~~~(if (numberp x) (go out)) \\
~~~~~~(print x))) \\
~out \\
~~...)
\end{lisp}
When the \cdf{go} is executed, the call to \cdf{print} is executed first,
and then the transfer of control to the tag \cdf{out} is completed.

\begin{newer}
X3J13 voted in March 1989 \issue{EXIT-EXTENT} to clarify the interaction
of \cdf{unwind-protect} with constructs that perform exits.

Let an \emph{exit} be a point out of which control can be transferred.
For a \cdf{throw} the exit is the matching \cdf{catch}; for
a \cdf{return-from} the exit is the corresponding \cdf{block}.
For a \cdf{go} the exit is the statement within the \cdf{tagbody} (the one to which
the target tag belongs) which is being executed at the time the \cdf{go} is performed.

The extent of an exit is dynamic; it is not indefinite. The extent
of an exit begins when the corresponding form (\cdf{catch}, \cdf{block}, or \cdf{tagbody}
statement) is entered.  When the extent of an exit has ended, it is no
longer legal to return from it.

Note that the extent of an exit is not the same thing as the scope or extent of the
designator by which the exit is identified. For example, a \cdf{block}
name has lexical scope but the extent of its exit is dynamic.  The
extent of a \cdf{catch} tag could differ from the extent of the exit
associated with the \cdf{catch} (which is exactly what is at issue here).
The difference matters when there are transfers
of control from the cleanup clauses of an \cdf{unwind-protect}.

When a transfer of control out of an exit is initiated by \cdf{throw},
\cdf{return-from}, or \cdf{go},
a variety of events occur before the transfer of control is complete:
\begin{itemize}
\item The cleanup clauses of any intervening \cdf{unwind-protect} clauses
    are evaluated.
\item Intervening dynamic bindings of special variables and catch tags
    are undone.
\item Intervening exits are \emph{abandoned}, that is, their extent ends and it
    is no longer legal to attempt to transfer control from them.
\item The extent of the exit being invoked ends.
\item Control is finally passed to the target.
\end{itemize}
The first edition left the order of these events in some doubt.
The implementation note for \cdf{throw} hinted that the first two processes
are interwoven, but it was unclear whether it is permissible
for an implementation to abandon all 
intervening exits before processing any intervening \cdf{unwind-protect}
cleanup clauses.

The clarification adopted by X3J13 is as follows.
Intervening exits are abandoned as soon as the transfer of control is initiated;
in the case of a \cdf{throw}, this occurs at the beginning of the ``second pass''
mentioned in the implementation note.  It is an error to
attempt a transfer of control to an exit whose dynamic extent has
ended.

Next the evaluation of \cdf{unwind-protect} cleanup clauses and the
undoing of dynamic bindings and \cdf{catch} tags are performed together,
in the order corresponding to the reverse of the order
in which they were established.
The effect of this is that the cleanup clauses of an \cdf{unwind-protect}
will see the same dynamic bindings of variables and \cdf{catch} tags as were
visible when the \cdf{unwind-protect} was entered.  (However, some of those
\cdf{catch} tags may not be useable because they correspond to abandoned
exit points.)

Finally  control is transferred to
the originally invoked exit and simultaneously that exit is abandoned.

The effect of this specification is that once a program has attempted
to transfer control to a particular exit, an \cdf{unwind-protect} cleanup
form cannot
step in and decide to transfer control to a more recent (nested) exit,
blithely forgetting the original exit request.  However, a cleanup form
may restate the request to transfer to the same exit that started
the cleanup process.

Here is an example based on a nautical metaphor.  The function \cdf{gently}
moves an oar in the water with low force, but if an oar gets stuck, the caller
will catch a crab.  The function \cdf{row}
takes a boat, an oar-stroking function,
a stream, and a count; an oar is constructed for the boat and stream
and the oar-stroking function is called \cd{:count} times.
The function \cdf{life} rows a particular boat.
Merriment follows, except that if the oarsman is winded he must stop
to catch his breath.
\begin{lisp}
(defun gently (oar) \\*
~~(stroke oar :force 0.5) \\*
~~(when (stuck oar) \\*
~~~~(throw 'crab nil))) \\
\\
(defun row (boat stroke-fn stream \&key count) \\*
~~(let ((oar (make-oar boat stream))) \\*
~~~~(loop repeat count do (funcall stroke-fn oar)))) \\
\\
(defun life () \\*
~~(catch 'crab \\*
~~~~(catch 'breath \\*
~~~~~~(unwind-protect \\*
~~~~~~~~(row *your-boat* \#'gently *query-io* :count 3)) \\*
~~~~~~~~(when (winded) (throw 'breath nil))) \\*
~~~~~~(loop repeat 4 (set-mode :merry)) \\*
~~~~~~(dream))))
\end{lisp}
Suppose that the oar gets stuck, causing \cdf{gently} to call \cdf{throw}
with the tag \cdf{crab}.
The program is then committed to exiting from the outer \cdf{catch} (the one
with the tag \cdf{crab}).  As control breaks out of the \cdf{unwind-protect} form,
the \cdf{winded} test is executed.  Suppose it is true; then another call to \cdf{throw}
occurs, this time with the tag \cdf{breath}.  The inner \cdf{catch} (the one with
the tag \cdf{breath}) has been abandoned as a result of the first
\cdf{throw} operation (still in progress).  The clarification voted by X3J13
specifies that the program is in error for attempting to transfer control
to an abandoned exit point.  To put it in terms of the example: once you have
begun to catch
a crab, you cannot rely on being able to catch your breath.

Implementations may support longer extents for exits than is
required by this specification,
but portable programs may not rely on such extended extents.

(This specification is somewhat controversial.  An alternative proposal was
that the abandoning of exits should be lumped in with
the evaluation of \cdf{unwind-protect} cleanup clauses and the
undoing of dynamic bindings and \cdf{catch} tags, performing all
in reverse order of establishment.  X3J13 agreed that this approach is
theoretically cleaner and more elegant but also more stringent
and of little additional practical use.  There was some concern that
a more stringent specification might be a great added burden to some
implementors and would achieve only a small gain for users.)
\end{newer}
\end{defspec}

\begin{defspec}
throw tag result

The \cdf{throw} special form transfers control to a matching
\cdf{catch} construct.
The \emph{tag} is evaluated first to produce an object
called the throw tag; then the \emph{result} form is evaluated,
and its results are saved (if the \emph{result} form produces
multiple values, then \emph{all} the values are saved).
The most recent outstanding catch whose tag matches the throw tag
is exited; the saved results are returned as the value(s) of the catch.
A \cdf{catch} matches only if the catch tag is \cdf{eq} to the throw tag.

In the process, dynamic variable
bindings are undone back to the point of the catch, and any intervening
\cdf{unwind-protect} cleanup code is executed.
The \emph{result} form is evaluated before the unwinding process commences,
and whatever results it produces are returned from the catch.

If there is no outstanding catcher whose tag matches the throw tag,
no unwinding of the stack is performed, and an error is signalled.
When the error is signalled, the outstanding catchers and the dynamic
variable bindings are those in force at the point of the throw.

\beforenoterule
\begin{implementation}
These requirements imply that throwing should typically
make two passes over the control stack.  In the first pass it simply
searches for a matching catch.  In this search every \cdf{catch}
must be considered, but every
\cdf{unwind-protect} should be ignored.  On the second pass the stack
is actually unwound, one frame at a time, undoing dynamic bindings
and outstanding \cdf{unwind-protect} constructs in reverse order of creation
until the matching catch is reached.
\end{implementation}

\betweennoterule

\begin{incompatibility}
The name \cdf{throw} comes from MacLisp,
but the syntax of \cdf{throw} in Common Lisp is different.
The MacLisp syntax was \cd{(throw \emph{form} \emph{tag})},
where the \emph{tag} was not evaluated.
\end{incompatibility}

\afternoterule
\end{defspec}

%RUSSIAN
\else

\chapter{Управляющие конструкции}
\label{CONTRL}

Common Lisp предоставляет набор специальных структур для построения
программ. Некоторые из них связаны со порядком выполнения (управляющие
структуры), тогда как другие с управлением доступа к переменным (структурам
окружения).
Некоторые из этих возможностей реализованы как специальные формы;
другие как макросы, которые в свою очередь разворачиваются в совокупность
фрагментов программы, выраженных в терминах специальных форм или других
макросов.

Вызов функции (применение функции) является основным методом создания Lisp
программ. Операции записываются, как применение функции к её аргументам. Обычно
Lisp программы пишутся, как большая совокупность маленьких функции,
взаимодействующих с помощью вызовов одна другой, таким образом большие операции
определяются в терминах меньших.
Функции Lisp'а могут рекурсивно вызывать сами себя, как напрямую, так и косвенно.

\begin{new}
Локально определённые функция (\cdf{flet}, \cdf{labels}) и макросы
(\cdf{macrolet}) достаточно универсальны.
Новая функциональность макросимволов позволяет использовать ещё больше синтаксической гибкости.
\end{new}


Несмотря на то, что язык Lisp более функциональный (applicative), чем императивный
(statement-oriented), он предоставляет много операций, имеющих побочные эффекты,
и следовательно требует конструкции для управления последовательностью вызовов с
побочными эффектами. Конструкция \cdf{progn}, которая является приблизительным
эквивалентом Algol'ному блоку \textbf{begin}-\textbf{end} с всеми его точками с
запятыми, последовательно выполняет некоторое количество форм, игнорируя все их
значения, кроме последней.
Много Lisp'овых управляющих конструкции неявно включают последовательное
выполнение форм, в таком случае говорится, что это <<неявный progn>>.
\indexterm{implicit \cdf{progn}}
Существуют также другие управляющие конструкции такие, как \cdf{prog1} и
\cdf{prog2}.

Для циклов Common Lisp предоставляет, как общую функцию для итераций \cdf{do},
так и набор более специализированных функций для итераций или отображений
(mapping) различных структур данных.

Common Lisp предоставляет простые одноветочные условные операторы \cdf{when} и
\cdf{unless}, простой двуветочный условный оператор \cdf{if} и более общие
многоветочные \cdf{cond} и \cdf{case}. Выбор одного из них для использования в
какой-либо ситуации зависит от стиля и вкуса.

Предоставляются конструкции выполнения нелокальных выходов с различными 
правилами областей видимости: \cdf{block}, \cdf{return}, \cdf{return-from},
\cdf{catch} и \cdf{throw}.

Конструкции multiple-value предоставляют удобный способ для возврата более одного
значения из функции, смотрите \cdf{values}.

\section{Константы и переменные}
\label{FUNCTION-NAME-SECTION}


Так как некоторые Lisp'овые объекты данных используются для отображения
программ, можно всегда обозначить константный объект данных с помощью записи
без приукрашательств формы данного объекта. Однако порождается двусмысленность:
константный это объект или фрагмент кода. Эту двусмысленность разрешает
специальная форма \cdf{quote}.

В Common Lisp'е присутствуют два вида переменных, а именно: обычные переменные и
имена функций. Между этими типами есть несколько сходств, и в некоторых случаях
для взаимодействия с ними используются похожие функции, например \cdf{boundp} и
\cdf{fboundp}. 
Однако для в большинстве случаев два вида переменных используются для совсем
разных целей: один указывает на функции, макросы и специальные формы, и другие
на объекты данных.

\subsection{Использование переменных}

Значение обычной переменной может быть получено просто с помощью записи его
имени как формы, которая будет выполнена. Будет ли данное имя распознано как имя
специальной или лексической переменной зависит от наличия или отсутствия
соответствующей декларации \cdf{special}. Смотрите главу~\ref{DECLAR}.

Следующие функции и специальные формы позволяют ссылаться на значения констант и
переменных.

\begin{defspec}
quote object

\cd{(quote \emph{x})} возвращает \emph{x}.
\emph{object} не выполняется и может быть любым объектом Lisp'а.
Конструкция позволяет записать в программе любой объект, как константное
значение.
Например:
\begin{lisp}
(setq a 43) \\
(list a (cons a 3)) \EV\ (43 (43 . 3)) \\
(list (quote a) (quote (cons a 3)) \EV\ (a (cons a 3))
\end{lisp}
Так как \cdf{quote} форма так полезна, но записывать её трудоёмко, для неё
определена стандартная аббревиатура:
любая форма \emph{f} с предшествующей одинарной кавычкой (\cd{~'~})
оборачивается формой \cd{(quote~~)} для создания \cd{(quote \emph{f})}.
Например:
\begin{lisp}
(setq x '(the magic quote hack))
\end{lisp}
обычно интерпретируется функцией \cdf{read}, как
\begin{lisp}
(setq x (quote (the magic quote hack)))
\end{lisp}
Смотрите раздел~\ref{MACRO-CHARACTERS-SECTION}.
\end{defspec}

\begin{defspec}
function fn

Значением \cdf{function} всегда является функциональной интерпретацией
\emph{fn}. \emph{fn} интерпретируется как, если бы она была использована на
позиции функции в форме вызова функции.
В частности, если \emph{fn} является символом, возвращается определение функции,
связанное с этим символом, смотрите \cdf{symbol-function}.
Если \emph{fn} является лямбда-выражением, тогда возвращается <<лексическое
замыкание>>, это значит функция, которая при вызове выполняет тело
лямбда-выражения таким образом, чтобы правила лексического контекста выполнялись 
правильно.

\begin{newer}
X3J13 проголосовал в марте 1989 \issue{FUNCTION-NAME}
расширить функцию \cdf{function}
до принятия любого имени функции (символ или список,
\emph{car} элементы которого является \cdf{setf}---смотрите раздел~\ref{FUNCTION-NAME-SECTION})
Также принимает принимает лямбда-выражение.
Так можно записать \cd{(function (setf cadr))} для ссылки на функцию раскрытия
\cdf{setf} для \cdf{car}.
\end{newer}

\indexterm{closure}
Например:
\begin{lisp}
(defun adder (x) (function (lambda (y) (+ x y))))
\end{lisp}
Результат \cd{(adder 3)} является функцией, которая добавляет \cd{3} к её
аргументу:
\begin{lisp}
(setq add3 (adder 3)) \\
(funcall add3 5) \EV\ 8
\end{lisp}
Это работает, потому что \cdf{function} создаёт замыкание над внутренним
лямбда-выражением, которое может ссылаться на значение \cd{3} переменной \cd{x}
даже после того, как выполнение вышло из функции \cd{adder}.

Если посмотреть глубже, то лексическое замыкание обладает возможностью ссылаться
на лексически видимые \emph{связывание}, а не просто на значения.
Рассмотрим такой код:
\begin{lisp}
(defun two-funs (x) \\
~~(list (function (lambda () x)) \\
~~~~~~~~(function (lambda (y) (setq x y))))) \\
(setq funs (two-funs 6)) \\
(funcall (car funs)) \EV\ 6 \\
(funcall (cadr funs) 43) \EV\ 43 \\
(funcall (car funs)) \EV\ 43
\end{lisp}
Функция \cdf{two-funs} возвращает список двух функций, каждая из которых
ссылается на \emph{связывание} переменной \cdf{x}, созданной в момент входа в
функцию \cd{two-funs}, когда она была вызвана с аргументом \cd{6}.
Это связывание сначала имеет значение \cd{6}, но \cdf{setq} может изменить
связывание. Лексическое замыкание для первого лямбда-выражения не является
<<создаёт снимок>> значения \cd{6} для \cd{x} при создании замыкания. Вторая
функция может использоваться для изменения связывания (на \cd{43} например), и
это изменённое значение станет доступным в первой функции.

В ситуации, когда замыкание лямбда-выражения над одним и тем же множеством
связываний может создаваться несколько раз, эти полученные разные замыкания
могут быть равны или не равны \cdf{eq} в зависимости от реализации.
Например:
\begin{lisp}
(let ((x 5) (funs '())) \\*
~~(dotimes (j 10) \\*
~~~~(push \#'(lambda (z) \\*
~~~~~~~~~~~~~~(if (null z) (setq x 0) (+ x z))) \\*
~~~~~~~~~~funs)) \\*
~~funs)
\end{lisp}
Результат данного выражения является списком десяти замыканий.
Каждое логически требует только связывания \cd{x}.
В любом случае это одно и то же связывание, но десять замыканий могут быть равны
или не равны \cdf{eq} друг другу.
С другой стороны, результат выражения
\begin{lisp}
(let ((funs '())) \\*
~~(dotimes (j 10) \\*
~~~~(let ((x 5)) \\*
~~~~~~(push (function (lambda (z) \\*
~~~~~~~~~~~~~~~~~~~~~~~~(if (null z) (setq x 0) (+ x z)))) \\*
~~~~~~~~~~~~funs))) \\*
~~funs)
\end{lisp}
также является списком из десяти замыканий.
Однако в этом случае, но одна из пар замыканий не будет равна \cdf{eq}, потому
что каждое замыкание имеет своё связывание \cd{x} отличное от
другого. Связывания отличаются, так как в замыкании используется \cdf{setq}.

Вопрос различного поведения важен, поэтому рассмотрим следующее простое выражение:
\begin{lisp}
(let ((funs '())) \\*
~~(dotimes (j 10) \\*
~~~~(let ((x 5)) \\*
~~~~~~(push (function (lambda (z) (+ x z))) \\*
~~~~~~~~~~~~funs))) \\*
~~funs)
\end{lisp}
Результатом является десять замыканий, которые \emph{могут} быть равны \cdf{eq}
попарно. Однако, можно подумать что связывания \cd{x} для каждого замыкания разные, так
как создаются в цикле, но связывания не могут различаться, потому что их значения
идентичны и неизменяемы (иммутабельны), в замыканиях отсутствует \cdf{setq} для
\cd{x}.
Компилятор может в таких случаях оптимизировать выражение так:
\begin{lisp}
(let ((funs '())) \\*
~~(dotimes (j 10) \\*
~~~~(push (function (lambda (z) (+ 5 z))) \\*
~~~~~~~~~~funs)) \\*
~~funs)
\end{lisp}
после чего, в конце концов, замыкания точно могут быть равны.
Общее правило такое, что реализация может в двух различных случаях выполнения
формы \cdf{function} вернуть идентичные (\cdf{eq}) замыкания, если она может
доказать, что два концептуально различающихся замыкания по факту ведут себя
одинаково при одинаковых параметрах вызова.
Это просто разрешается для оптимизации. Полностью корректная реализация может
каждый раз при выполнении формы \cdf{function} возвращать новое замыкание не
равное \cdf{eq} другим.

Часто компилятор может сделать вывод, что замыкание по факту не нуждается в
замыкании над какими-либо связываниями переменных. Например,
в фрагменте кода
\begin{lisp}
(mapcar (function (lambda (x) (+ x 2))) y)
\end{lisp}
функция \cd{(lambda (x) (+ x 2))} не содержит ссылок на какие-либо внешние
сущности. В этом важном случае, одно и то же <<замыкание>> может
быть использовано в качестве результата всех выполнений специальной формы
\cdf{function}.
Несомненно, данное значение может и не быть объектом замыкания. Оно может быть
просто скомпилированной функцией, не содержащей информации об окружении.
Данный пример просто является частным случаем предыдущего разговора и включён в
качестве подсказки для разработчиков знакомых с предыдущими методами реализации
Lisp'а. Различие между замыканиями и другими видами функций слегка размыто,
Common Lisp не определяет отображения для замыканий и метода различия замыканий
и простых функций. Все что имеет значение, это соблюдение правил лексической
области видимости.

Так как форма \cdf{function} используются часто, но её запись длинная,
для неё определена стандартная аббревиатура: любая форма \emph{f} с
предшествующими \cf{\#'} разворачивается в форму \cd{(function \emph{f})}.
Например,
\begin{lisp}
(remove-if \#'numberp '(1 a b 3))
\end{lisp}
обычно интерпретируется функцией \cdf{read} как
\begin{lisp}
(remove-if (function numberp) '(1 a b 3))
\end{lisp}
Смотрите раздел~\ref{SHARP-SIGN-MACRO-CHARACTER-SECTION}.
\end{defspec}

\begin{defun}[Функция]
symbol-value symbol

\cdf{symbol-value} возвращает текущее значение динамической (специальной)
переменной с именем \emph{symbol}.
Если символ не имеет значения, возникает ошибка. Смотрите \cdf{boundp} и
\cdf{makunbound}. 
Следует отметить, что константные символы являются переменными, которые не могут
быть изменены, таким образом \cdf{symbol-value} может использоваться для
получения значения именованной константы. \cdf{symbol-value} от ключевого
символа будет возвращать этот ключевой символ.

\cdf{symbol-value} не может получить доступ к значению лексической переменной.

В частности, эта функция полезна для реализации интерпретаторов для встраиваемых
языков в Lisp'е.
Соответствующая функция присваивания \cdf{set}. Кроме того, можно пользоваться
конструкцией \cdf{setf} с \cdf{symbol-value}.
\end{defun}

\begin{defun}[Функция]
symbol-function symbol

\cdf{symbol-function} возвращает текущее глобальное определение функции с именем
\emph{symbol}. В случае если символ не имеет определения функции сигнализируется
ошибка, смотрите \cdf{fboundp}. Следует отметить что определение может быть
функцией или объектом отображающим специальную форму или макрос.
Однако, в последнем случае, попытка вызова объекта как функции будет является
ошибкой.
Лучше всего заранее проверить символ с помощью \cdf{macro-function} и
\cdf{special-operator-p} и только затем вызвать функциональное значение, если оба
предыдущих теста вернули ложь.

Эта функция полезна, в частности, для реализации интерпретаторов языков
встроенных в Lisp.

\cdf{symbol-function} не может получить доступ к значению имени лексической
функции, созданной с помощью \cdf{flet} или \cdf{labels}. Она может получать
только глобальное функциональное значение.

Глобальное определение функции для некоторого символа может быть изменено с
помощью \cdf{setf} и \cdf{symbol-function}.
При использовании этой операции символ будет иметь \emph{только} заданное определение в качестве своего
глобального функционального значения. Любое предыдущее определение, было ли оно
макросом или функцией, будет потеряно.
Попытка переопределения специальной формы (смотрите
таблицу~\ref{SPECIAL-FORM-TABLE}) будет является ошибкой.
\end{defun}

\begin{newer}
\begin{defun}[Функция]
fdefinition function-name

X3J13 проголосовал в марте 1989 \issue{FUNCTION-NAME} добавить в язык функцию
\cdf{fdefinition}.
Она похожа на  \cdf{symbol-function}
за исключением того, что её аргументы может быть любым именем функции (символ
или списка, у которого \emph{car} элемент равен \cdf{setf}---смотрите раздел~\ref{FUNCTION-NAME-SECTION}).
Функция возвращает текущее глобальное значение определения функции с именем \emph{function-name}.
Можно использовать \cdf{fdefinition} вместе с \cdf{setf}
для изменения глобального определения функции связанной с переданным в параметре
именем.
\end{defun}
\end{newer}

\begin{defun}[Функция]
boundp symbol

\cdf{boundp} является истиной, если динамическая (специальная) переменная с
именем \emph{symbol} имеет значение, иначе возвращает {\false}.

Смотрите также \cdf{set} и \cdf{makunbound}.
\end{defun}

\begin{defun}[Функция]
fboundp symbol

\cdf{fboundp} является истиной, если символ имеет глобальное определение
функции.
Следует отметить, что \cdf{fboundp} является истиной, если символ указывает на
специальную форму или макрос. \cdf{macro-function} и \cdf{special-operator-p} могут
использоваться для проверки таких случаев.

Смотрите также \cdf{symbol-function} и \cdf{fmakunbound}.
\end{defun}

\begin{defun}[Функция]
special-operator-p symbol

Функция \cdf{special-operator-p} принимает символ. Если символ указывает на
специальную форму, тогда возвращается значение не-{\false}, иначе возвращается {\false}.
Возвращённое не-{\nil} значение является функцией,
которая может быть использована для интерпретации (вычисления) специальной
формы. FIXME

Возможно также то, что \emph{обе} функции \cdf{special-operator-p} и
\cdf{macro-function} будут истинными для одного и того же символа. Это потому,
что реализация может иметь любой макрос как специальную форму для скорости.
С другой стороны, определение макроса должно быть доступно для использования
программами, которые понимают только стандартные специальные формы,
перечисленные в таблице~\ref{SPECIAL-FORM-TABLE}. FIXME
\end{defun}

\subsection{Присваивание}

Следующая функциональность позволяет изменять значение переменной (если быть
точнее, значению соединённому с текущим связыванием переменной).
Такое изменение отличается от создания нового связывания.
Конструкции для создания новых связываний переменных описаны в
разделе~\ref{VAR-BINDING-SECTION}.

\begin{defspec}
setq {var form}*

Специальная форма \cd{(setq \emph{var1} \emph{form1} \emph{var2} \emph{form2}
  ...)} является <<конструкцией присваивания простых переменных>> Lisp'а.
Вычисляется первая форма \emph{form1} и результат сохраняется в переменной
\emph{var1}, затем вычисляется \emph{form2} и результат сохраняется в переменной
\emph{var2}, и так далее.
Переменные, конечно же, представлены символами, и интерпретируются как ссылки к
динамическим или статическим переменным в соответствии с обычными правилами.
Таким образов \cdf{setq} может быть использована для присваивания как
лексических, так и специальных переменных.

\cdf{setq} возвращает последнее присваиваемое значение, другими словами,
результат вычисления последнего аргумента.
В другом случае, форма \cd{(setq)} является корректной и возвращает {\false}.
В форме должно быть чётное количество форм аргументов.
Например, в 
\begin{lisp}
(setq x (+ 3 2 1) y (cons x nil))
\end{lisp}
\cd{x} устанавливается в \cd{6}, \cd{y} в \cd{(6)}, и \cdf{setq} возвращает
\cd{(6)}. Следует отметить, что первое присваивание выполняется перед тем, как
будет выполнено второе, тем самым каждое следующее присваивание может
использовать значение предыдущих.

\begin{newer}
Смотрите также описание \cdf{setf}, <<общая конструкция
присваивания>> Common Lisp'а, которая позволяет присваивать значения переменным,
элементам массива, и другим местам.

Некоторые программисты выбирают путь отречения от \cdf{setq}, и всегда используют
\cdf{setf}. Другие используют \cdf{setq} для простых переменных и \cdf{setf} для
всех остальных.
\end{newer}
\end{defspec}

\begin{defmac}
psetq {var form}*

Форма \cdf{psetq} похожа на форму \cdf{setq} за исключением того, что выполняет
присваивание параллельно. Сначала выполняются все формы, а затем переменные
получают значения этих форм. Значение формы \cdf{psetq} {\false}.
Например:
\begin{lisp}
(setq a 1) \\
(setq b 2) \\
(psetq a b  b a) \\
a \EV\ 2 \\
b \EV\ 1
\end{lisp}
В этом примере, значения \cd{a} и \cd{b} меняются местами с помощью
параллельного присваивания.
(Если несколько переменных должны быть присвоены параллельно в рамках цикла,
целесообразнее использовать конструкцию \cdf{do}.)

Смотрите также описание \cdf{psetf}, <<общая конструкция параллельного
присваивания>> Common Lisp'а, которая позволяет присваивать переменным,
элементам массива, и другим местам.
\end{defmac}

\begin{defun}[Функция]
set symbol value

\cdf{set} позволяет изменить значение динамической (специальной) переменной.
\cdf{set} устанавливает динамической переменной с именем \emph{symbol} значение
\emph{value}.

Изменено будет только значение текущего динамического связывания. Если такого
связывания нет, будет изменено наиболее глобальное значение.
Например,
\begin{lisp}
(set (if (eq a b) 'c 'd) 'foo)
\end{lisp}
установит значение \cd{с} в \cd{foo} или \cdf{do*} в \cd{foo}, в зависимости от
результата проверки \cd{(eq~a~b)}.

\cdf{set} в качестве результата возвращает значение \emph{value}.

\cdf{set} не может изменить значение локальной (лексически связанной)
переменной.
Обычно для изменения переменных (лексических или динамических) используется
специальная форма \cdf{setq}.
\cdf{set} полезна в частности для реализации интерпретаторов языков встроенных в
Lisp.
Смотрите также \cdf{progv}, конструкция, которая создаёт связывания, а не
присваивания динамических переменных.
\end{defun}

\begin{defun}[Функция]
makunbound symbol \\
fmakunbound symbol

\cdf{makunbound} упраздняет связывание динамической (специальной) переменной
заданной символом \emph{symbol} (упраздняет значение). \cdf{fmakunbound}
аналогично упраздняет связь символа с глобальным определением функции.
Например:
\begin{lisp}
(setq a 1) \\
a \EV\ 1 \\
(makunbound 'a) \\
a \EV\ \textrm{ошибка} \\
\\
(defun foo (x) (+ x 1)) \\
(foo 4) \EV\ 5 \\
(fmakunbound 'foo) \\
(foo 4) \EV\ \textrm{ошибка}
\end{lisp}
Обе функции возвращают символ \emph{symbol} в качестве результата.
\end{defun}

\section{Обобщённые переменные}
\label{SETF-SECTION}

В Lisp'е, переменная может запомнить одну часть данных, а точнее, один Lisp
объект. 
Главные операции над переменной это получить её значение и задать ей другое
значение. Их часто называют операциями \emph{доступа} и
\emph{изменения}. Концепция переменных с именем в виде символа может быть
обобщена до того, что любое место может сохранять в себе части данных вне
зависимости от того, как данное место именуется. Примерами таких мест хранения
являются \emph{car} и \emph{cdr} элементы cons-ячейки, элементы массива, и
компоненты структуры.

Для каждого вида обобщённых переменных существуют две функции, которые реализуют
операции \emph{доступа} и \emph{изменения}.
Для переменных это имя переменной для доступа, а для изменения специальная форма \cdf{setq}.
Функция \cdf{car} получает доступ к \emph{car} элементу cons-ячейки, а функция
\cdf{rplaca} изменяет этот элемент ячейки.
Функция \cdf{symbol-value} получает динамическое значение переменной именованной
некоторым символом, а функция \cdf{set} изменяет эту переменную.

Вместо того, чтобы думать о двух разных функциях, которые соответственно
получают доступ и изменяют некоторое место хранения в зависимости от своих
аргументов, мы может думать просто о вызове функции доступа с некоторыми
аргументами, как о \emph{имени} данного места хранения. Таким образом, просто
\cd{x} является именем места хранения (переменной), \cd{(car x)} имя для
\emph{car} элементы для некоторой cons-ячейки (которая в свою очередь именуется
символом \cd{x}). Теперь вместо того, чтобы запоминать по две функции для
каждого вида обобщённых переменных (например \cdf{rplaca} для \cdf{car}), мы
адаптировали единый синтаксис для изменения некоторого места хранения с помощью
макроса \cdf{setf}. Это аналогично способу, где мы используем специальную
форму \cdf{setq} для преобразования имени переменной (которая является также
формой для доступа к ней) в форму, которая изменяет переменную
FIXME. Эта универсальной отображения в следующей таблице.

\begin{flushleft}
\begin{tabular*}{\textwidth}{@{}l@{\extracolsep{\fill}}ll@{}}
\textrm{Функция доступа}&\textrm{Функция изменения}&\textrm{Изменения с помощью \cdf{setf}} \\
\hlinesp
\cd{x}&\cd{(setq x datum)}&\cd{(setf x datum)} \\
\cd{(car x)}&\cd{(rplaca x datum)}&\cd{(setf (car x) datum)} \\
\cd{(symbol-value x)}&\cd{(set x datum)}&\cd{(setf (symbol-value x) datum)} \\
\hline
\end{tabular*}
\end{flushleft}
\cdf{setf} это макрос, который анализирует форму доступа, и производит вызов
соответствующей функции изменения.

С появление в Common Lisp'е \cdf{setf}, необходимость в \cdf{setq}, \cdf{rplaca}
и \cdf{set} отпала. Они оставлены в Common Lisp из-за их исторической важности в
Lisp'е.
Однако, большинство других функций изменения (например \cdf{putprop}, функция
изменения для \cdf{get}) были устранены из Common Lisp'а в расчёте на то, что
везде на их месте будет использоваться \cdf{setf}.

\begin{defmac}
setf {place newvalue}*

\cd{(setf \emph{place} \emph{newvalue})} принимает форму \emph{place}, которая
при своём вычислении получает доступ к объекту в некотором месте хранения и
<<инвертирует>> эту форму в соответствующую форму \emph{изменения}.
Таким образом вызов макроса \cdf{setf} разворачивается в форму изменения,
которая сохраняет результат вычисления формы \emph{newvalue} в место хранения,
на которое ссылалась форма доступа.

Если пар \emph{place}-\emph{newvalue} указано более одной, эти пары
обрабатываются последовательно. Таким образом:
\begin{lisp}
(setf \emph{place1} \emph{newvalue1} \\
~~~~~~\emph{place2} \emph{newvalue2}) \\
~~~~~~... \\
~~~~~~\emph{placen} \emph{newvaluen})
\end{lisp}
эквивалентно
\begin{lisp}
(setf \emph{place1} \emph{newvalue1} \\
~~~~~~\emph{place2} \emph{newvalue2}) \\
~~~~~~... \\
~~~~~~\emph{placen} \emph{newvaluen})
\end{lisp}
Следует отметить, что запись \cd{(setf)} является корректной и возвращает {\nil}.

Форма \emph{place} может быть одной из следующих:
\begin{itemize}

\item
Имя переменной (лексической и динамической).

\item
Формой вызова функции, у которой первый элемент принадлежит множеству указанному
в следующей таблице:

\begin{flushleft}
\begin{tabular}{@{}llll@{}}
\cdf{aref}&\cdf{car}&\cdf{svref}& \\
\cdf{nth}&\cdf{cdr}&\cdf{get}& \\
\cdf{elt}&\cdf{caar}&\cdf{getf}&\cdf{symbol-value} \\
\cdf{rest}&\cdf{cadr}&\cdf{gethash}&\cdf{symbol-function} \\
\cdf{first}&\cdf{cdar}&\cdf{documentation}&\cdf{symbol-plist} \\
\cdf{second}&\cdf{cddr}&\cdf{fill-pointer}&\cdf{macro-function} \\
\cdf{third}&\cdf{caaar}&\cdf{caaaar}&\cdf{cdaaar} \\
\cdf{fourth}&\cdf{caadr}&\cdf{caaadr}&\cdf{cdaadr} \\
\cdf{fifth}&\cdf{cadar}&\cdf{caadar}&\cdf{cdadar} \\
\cdf{sixth}&\cdf{caddr}&\cdf{caaddr}&\cdf{cdaddr} \\
\cdf{seventh}&\cdf{cdaar}&\cdf{cadaar}&\cdf{cddaar} \\
\cdf{eighth}&\cdf{cdadr}&\cdf{cadadr}&\cdf{cddadr} \\
\cdf{ninth}&\cdf{cddar}&\cdf{caddar}&\cdf{cdddar} \\
\cdf{tenth}&\cdf{cdddr}&\cdf{cadddr}&\cdf{cddddr}
\end{tabular}
\end{flushleft}

\item
Формой вызова функции, у которой первый элемент является именем
функции-селектора созданной с помощью \cdf{defstruct}.

\item
Форма вызова функции, первый элемент которой является именем одной из
следующих функций при условии, что указанный аргумент этой функции в свою
очередь является формой \emph{place}. 
in this case the new \emph{place} has stored back into it the
result of applying the specified ``update'' function
(which is in each of these cases not a true update function): FIXME

\begin{flushleft}
\begin{tabular}{@{}lll@{}}
Имя функции&Аргумент являющийся \emph{местом}&Функция изменения \\
\hlinesp
\cdf{char-bit}&first&\cdf{set-char-bit} \\
\cdf{ldb}&second&\cdf{dpb} \\
\cdf{mask-field}&second&\cdf{deposit-field} \\
\hline
\end{tabular}
\end{flushleft}

\item
Форма декларации типа \cdf{the}, в таком случае декларация переносится на форму
\emph{newvalue}, и анализируется результирующая \cdf{setf} форма. Например:
\begin{lisp}
(setf (the integer (cadr x)) (+ y 3))
\end{lisp}
будет обработана как
\begin{lisp}
(setf (cadr x) (the integer (+ y 3)))
\end{lisp}

\item 
Вызов функции \cdf{apply}, в которой первый аргумент является функцией, которая
может является \emph{местом} для \cdf{setf}.

\item 
Макровызов, в случае чего \cdf{setf} разворачивает макровызов и затем
анализирует полученную форму.

\item 
Любая форма, для которой было сделано определение с помощью \cdf{defsetf} или
\cdf{define-setf-method}.
\end{itemize}

\cdf{setf} тщательно сохраняет обычный порядок выполнения подформ слева
направо.
С другой стороны, точное раскрытие для какой-нибудь частной формы не
гарантируется и может зависеть от реализации. Все, что гарантируется, это 
раскрытие \cdf{setf} формы в некоторую функцию изменения, используемую данной
реализацией, и выполнение подформ слева направо.

Конечным результатом вычисления формы \cdf{setf} является значение
\emph{newvalue}. Таким образом \cd{(setf (car x) y)} раскрывается не прямо в
\cd{(rplaca x y)}, а в что-то вроде
\begin{lisp}
(let ((G1 x) (G2 y)) (rplaca G1 G2) G2)
\end{lisp}
точное раскрытие зависит от реализации.

Пользователь может определить новое раскрытие для \cdf{setf} используя
\cdf{defsetf}.
\end{defmac}

\begin{defmac}
psetf {place newvalue}*

\cdf{psetf} похожа на \cdf{setf} за исключением того, что если указано более одной 
пары
\emph{place}-\emph{newvalue} , то присваивание местам новых значений 
происходит параллельно. Если говорить точнее, то все подформы, которые должны 
быть вычислены, вычисляются слева направо. После выполнения всех вычислений,
выполняются все присваивания в неопределённом порядке.
(Неопределённый порядок влияет на поведение в случае, если более одной формы
\emph{place} ссылаются на одно и то же место.)
\cdf{psetf} всегда возвращает {\false}.
\end{defmac}

\begin{defmac}
shiftf {place}+ newvalue

Каждая форма \emph{place} может быть любой обобщённой переменной, как для
\cdf{setf}. 
В форме \cd{(shiftf \emph{place1} \emph{place2} ... \emph{placen}
  \emph{newvalue})}, вычисляются и сохраняются значения с \emph{place1} по
\emph{placen} и вычисляется \emph{newvalue}, как значение с номером $\emph{n}+1$.
Значения с 2-го по $\emph{n}+1$ сохраняются в интервале с \emph{place1} по
\emph{placen} и возвращается 1-ое значение (оригинальное значение \emph{place1}).
Механизм работает как сдвиг регистров. \emph{newvalue} сдвигается с правой
стороны, все значения сдвигаются влево на одну позицию, и возвращается
сдвигаемое самое левое значение \emph{place1}. Например:
\begin{lisp}
(setq x (list 'a 'b 'c)) \EV\ (a b c) \\
 \\
(shiftf (cadr x) 'z) \EV\ b \\
~~~\textrm{and now} x \EV\ (a z c) \\
 \\
(shiftf (cadr x) (cddr x) 'q) \EV\ z \\
~~~\textrm{and now} x \EV\ (a (c) . q)
\end{lisp}
Эффект от \cd{(shiftf \emph{place1} \emph{place2} ... \emph{placen}
  \emph{newvalue})} эквивалентен
\begin{lisp}
(let ((\emph{var1} \emph{place1}) \\
~~~~~~(\emph{var2} \emph{place2}) \\
~~~~~~... \\
~~~~~~(\emph{varn} \emph{placen})) \\
~~(setf \emph{place1} \emph{var2}) \\
~~(setf \emph{place2} \emph{var3}) \\
~~... \\
~~(setf \emph{placen} \emph{newvalue}) \\
~~\emph{var1})
\end{lisp}
за исключением того, что последний вариант выполняет все подформы для каждого
\emph{place} дважды, тогда как \cdf{shiftf} выполняет только единожды.
Например:
\begin{lisp}
(setq n 0) \\
(setq x '(a b c d)) \\
(shiftf (nth (setq n (+ n 1)) x) 'z) \EV\ b \\
~~~\textrm{теперь} x \EV\ (a z c d) \\[4pt]
\emph{but} \\[4pt]
(setq n 0) \\
(setq x '(a b c d)) \\
(prog1 (nth (setq n (+ n 1)) x) \\*
~~~~~~~(setf (nth (setq n (+ n 1)) x) 'z)) \EV\ b \\
~~~\textrm{и теперь} x \EV\ (a b z d)
\end{lisp}
Более того, для заданных форм \emph{place} \cdf{shiftf} может быть более
производительной, чем версия с \cdf{prog1}.
\end{defmac}

\begin{defmac}
rotatef {place}*

Каждая \emph{place} может быть обобщённой переменной, как для \cdf{setf}.
В форме \cd{(rotatef \emph{place1} \emph{place2} ... \emph{placen})},
вычисляются и сохраняются значения c \emph{place1} по \emph{placen}.
Механизм действует как круговой сдвиг регистров влево, и значение
\emph{place1} сдвигается в конец на \emph{placen}.
Следует отметить, что \cd{(rotatef \emph{place1} \emph{place2})} меняет значения
между \emph{place1} и \emph{place2}.

Эффект от использования \cd{(rotatef \emph{place1} \emph{place2}
  ... \emph{placen})} эквивалентен
\begin{lisp}
(psetf \emph{place1} \emph{place2} \\
~~~~~~~\emph{place2} \emph{place3} \\
~~~~~~~... \\
~~~~~~~\emph{placen} \emph{place1})
\end{lisp}
за исключением того, что в последнем вычисление форм происходит дважды, тогда как \cdf{rotatef} выполняет только единожды.
Более того, для заданных форм \emph{place} \cdf{rotatef} может быть более
производительной, чем версия с \cdf{prog1}.

\cdf{rotatef} всегда возвращает {\false}.
\end{defmac}

Другие макрос, которые управляют обобщёнными переменными, включают 
\cdf{getf}, \cdf{remf},
\cdf{incf}, \cdf{decf}, \cdf{push}, \cdf{pop},
\cdf{assert}, \cdf{ctypecase} и \cdf{ccase}.

Макросы, которые управляют обобщёнными переменными, должны гарантировать
<<явную>> семантику: подформы обобщённых переменных вычисляются точно столько
раз, сколько они встречаются в выражении, и в том же порядке, в котором
встречаются.

В ссылках на обобщённые переменные, как в \cdf{shiftf}, \cdf{incf}, \cdf{push} и
\cdf{setf} или \cdf{ldb}, обобщённые переменные считываются и записываются в
одну и ту же ссылку. Сохранение порядка выполнения исходной программы и
количества выполнений чрезвычайно важно.

\textbf{Остаток раздела далее не переведён}

\section{Вызов функции}

Наиболее примитивная форма вызова функции в Lisp'е, конечно, не имеет
имени. Любой список, который не интерпретируется как макровызов или вызов
специальной формы, рассматривается как вызов функции.
Другие конструкции предназначены для менее распространённых, но тем не менее
полезных ситуаций.

\begin{defun}[Функция]
apply function arg &rest more-args

Функция применяет функцию \emph{function} к списку аргументов.

Все аргументы, кроме \emph{function} передаются в применяемую функцию.
Если последний аргумент список, то он присоединяется в конец к списку первых
аргументов. Например:
\begin{lisp}
(setq f '+) (apply f '(1 2)) \EV\ 3 \\
(setq f \#'-) (apply f '(1 2)) \EV\ -1 \\
(apply \#'max 3 5 '(2 7 3)) \EV\ 7 \\
(apply 'cons '((+ 2 3) 4)) {\EV} \\
~~~~~~~~((+ 2 3) . 4)	\emph{not} (5 . 4) \\
(apply \#'+ '()) \EV\ 0
\end{lisp}
Следует отметить, что если функция принимает именованные аргументы, ключевые
символы должны быть перечислены в списке аргументов так же, как и обычные
значения:
\begin{lisp}
(apply \#'(lambda (\cd{\&key} a b) (list a b)) '(:b 3)) \EV\ ({\nil} 3)
\end{lisp}
Это бывает очень полезным в связке с возможностью \cd{\&allow-other-keys}:
\begin{lisp}
(defun foo (size \cd{\&rest} keys \cd{\&key} double \cd{\&allow-other-keys}) \\
~~(let ((v (apply \#'make-array size :allow-other-keys t keys))) \\
~~~~(if double (concatenate (type-of v) v v) v))) \\
 \\
(foo 4 :initial-contents '(a b c d) :double t) \\
~~~\EV\ \#(a b c d a b c d)
\end{lisp}
\end{defun}

\begin{defun}[Функция]
funcall fn &rest arguments

\cd{(funcall \emph{fn} \emph{a1} \emph{a2} ... \emph{an})}
применяет функция \emph{fn} к аргументам 
\emph{a1}, \emph{a2}, ..., \emph{an}.
\emph{fn} не может быть специальной формой или макросом, это не имело бы
смысла. 

Например:
\begin{lisp}
(cons 1 2) \EV\ (1 . 2) \\
(setq cons (symbol-function '+)) \\
(funcall cons 1 2) \EV\ 3
\end{lisp}
Различие \cdf{funcall} и обычным вызовом функции в том, что функция
получается с помощью обычных Lisp вычислений, а не с помощью специальной
интерпретации первой позиции в списке формы.
\end{defun}

\begin{defun}[Константа]
call-arguments-limit

Значение \cdf{call-arguments-limit} положительное целое, которое отображает
невключительно максимальное количество аргументов, которые могут быть переданы в
функцию. Эта граница зависит от реализации, но не может быть меньше 50.
(Разработчикам предлагается сделать этот предел большим на сколько это возможно
без ущерба для производительности.)
Значение \cdf{call-argument-limit} должно, как минимум, равняться
\cdf{lambda-parameters-limit}.
Смотрите также \cdf{multiple-values-limit}.
\end{defun}

\section{Последовательное выполнение}

Все конструкции в данном разделе выполняют все формы аргументов в прямой
последовательности. Различие заключается только в возвращаемых ими результатах.

\begin{defspec}
progn {\,form}*

Конструкция \cdf{progn} принимает некоторое количество форм и последовательно
вычисляет их слева направо. Значения всех кроме последней формы
игнорируется. Результатом формы \cdf{progn} становиться то, что вернула
последняя форма.
Можно сказать, что все формы кроме последней выполняются для побочных эффектов,
а последняя форма для значения.

\cdf{progn} является примитивной управляющей структурой для <<составных
выражений>>, как блоки \textbf{begin}-\textbf{end} в Algol'ных языках.
Много Lisp'овых конструкций является <<неявным \cdf{progn}>>:
так как часть их синтаксиса допускает запись нескольких форм,
которые будут выполнены последовательно и возвращён будет результат последней
формы.

Если последняя форма \cdf{progn} возвращает несколько значение, тогда все они
будут возвращены из формы \cdf{progn}. Если форм в \cdf{progn} нет вообще, то
результатом будет {\false}. Эти правила сохраняются также и для неявного
\cdf{progn}.
\end{defspec}

\begin{defmac}
prog1 first {\,form}*

\cdf{prog1} похожа на \cdf{progn} за исключением того, что он возвращает
результат \emph{первой} формы. Все формы-аргументы выполняются
последовательно. Значение первой формы сохраняется, затем выполняются все формы,
и, наконец, возвращается сохранённое значение.

\cdf{prog1} чаще всего используется, когда необходимо вычислить выражение с
побочными эффектами и возвращаемое значение должно быть вычислено \emph{до}
побочных эффектов.
Например:
\begin{lisp}
(prog1 (car x) (rplaca x 'foo))
\end{lisp}
изменяет \emph{car} от \cd{x} на \cd{foo} и возвращает старое значение.

\cdf{prog1} всегда возвращает одно значение, даже если первая форма возвращает
несколько значений.
В следствие, \cd{(prog1 \emph{x})} и \cd{(progn \emph{x})} могут вести себя
по-разному, \emph{x} возвращает несколько значений. Смотрите \cdf{multiple-value-prog1}.
И хотя \cdf{prog1} может использоваться для явного указания возврата только
одного значения, для этих целей лучше использовать функцию \cdf{values}.
\end{defmac}

\begin{defmac}
prog2 first second {\,form}*

\cdf{prog2} похожа на \cdf{prog1}, но она возвращает значение её \emph{второй}
формы. Все формы-аргументы выполняются последовательно. Значение второй формы
сохраняется и возвращается после выполнения остальных форм.
\cdf{prog2} представлена по большей части по историческим причинам.
\begin{lisp}
(prog2 \emph{a} \emph{b} \emph{c} ... \emph{z}) \EQ\ (progn \emph{a} (prog1 \emph{b} \emph{c} ... \emph{z}))
\end{lisp}
Иногда необходимо получить один побочный эффект, затем полезный результат, затем
другой побочный эффекта. В таком случае \cdf{prog2} полезна.
Например:
\begin{lisp}
(prog2 (open-a-file) (process-the-file) (close-the-file)) \\
\`;\textrm{возвращаемое значение \cdf{process-the-file}}
\end{lisp}
\end{defmac}

\section{Установка новых связываний переменных}
\label{VAR-BINDING-SECTION}

В течение вызова функции представленной лямбда-выражением (или замыканием
лямбда-выражения возвращённым функцией \cdf{function}),
для переменных параметров лямбда-выражения устанавливаются новые связывания. Эти
связывания первоначально имеют значения установленные с помощью протокола
связывания параметров, описанного в~\ref{LAMBDA-EXPRESSIONS-SECTION}.

Для установки связываний переменных, обычных и функциональных, также полезны
следующие конструкции.

\begin{defspec}
let ({var | (var [value])}*) {declaration}* {\,form}*

Форма \cdf{let} может быть использована для связи множества переменных со
значениями соответствующего множества форм.

Если быть точнее, форма
\begin{lisp}
(let ((\emph{var1} \emph{value1}) \\
~~~~~~(\emph{var2} \emph{value2}) \\
~~~~~~... \\
~~~~~~(\emph{varm} \emph{valuem})) \\
~~\emph{declaration1} \\
~~\emph{declaration2} \\
~~... \\
~~\emph{declarationp} \\
~~\emph{body1} \\
~~\emph{body2} \\
~~... \\
~~\emph{bodyn})
\end{lisp}
сначала последовательно выполняет выражения \emph{value1}, \emph{value2} и т.д.,
сохраняя результаты.
Затем все переменные \emph{varj} параллельно привязываются к сохранённым
значениям. Каждое связывание будет является лексическим, кроме тех, для которых
указана декларация \cdf{special}.
Затем последовательно выполняются выражения \emph{bodyk}. Все из значения, кроме
последнего, игнорируются (другими словами, тело \cdf{let} является неявным
\cdf{progn}).
Форма \cdf{let} возвращает значение \emph{bodyn} (если тело пустое, что в
принципе бесполезно, то \cdf{let} возвращает {\false}).
Связывания переменных имеют лексическую область видимости и неограниченную
продолжительность.

Вместо списка \cd{(\emph{varj} \emph{valuej})}, можно записать просто
\emph{varj}. В таком случае \emph{varj} инициализируется значением {\false}.
В целях хорошего стиля рекомендуется, записывать \emph{varj} только, если в неё
будет что-нибудь записано (с помощью \cdf{setq} например), перед первым
использованием. 
Если важно, чтобы первоначальное значение было {\false}, вместо некоторого
неопределённого значения,
тогда будет лучше записать \cd{(\emph{varj} {\false})} или \cd{(\emph{varj}
  '{\emptylist})}, если значение должно обозначать пустой список. Обратите
внимание, что код
\begin{lisp}
(let (x) \\
~~(declare (integer x)) \\
~~(setq x (gcd y z)) \\
~~...)
\end{lisp}
неправильный. Так как \emph{x} объявлен без первоначального значения и также
объявлено, что \emph{x} это целое число, то произойдёт исключение, так как
\emph{x} при связывании получает {\nil} значение, которое не принадлежит
целочисленному типу.

Декларации могут использоваться в начале тела \cdf{let}. Смотрите \cdf{declare}.
\end{defspec}

\begin{defspec}
let* ({var | (var [value])}*) {declaration}* {\,form}*

\cdf{let*} похожа на \cdf{let}, но связывания переменных осуществляются
последовательно, а не параллельно. Это позволяет выражениям для значений
переменных ссылаться на ранее связанные переменные.

Если точнее, форма
\begin{lisp}
(let* ((\emph{var1} \emph{value1}) \\
~~~~~~~(\emph{var2} \emph{value2}) \\
~~~~~~~... \\
~~~~~~~(\emph{varm} \emph{valuem})) \\
~~\emph{declaration1} \\
~~\emph{declaration2} \\
~~... \\
~~\emph{declarationp} \\
~~\emph{body1} \\
~~\emph{body2} \\
~~... \\
~~\emph{bodyn})
\end{lisp}
сначала вычисляет выражение \emph{value1}, затем с этим значением связывает
переменную \emph{var1}, затем вычисляет \emph{value2} и связывает с 
результатом переменную \emph{var2}, и так далее.
Затем последовательно вычисляются выражения \emph{bodyj}.
Значения всех выражений, кроме последнего, игнорируются. То есть тело формы
\cdf{let*} является неявным \cdf{progn}.
Форма \cdf{let*} возвращает результаты вычисления \emph{bodyn} (если тело
пустое, что, в принципе, бесполезно, \cdf{let*} возвращает {\false}).
Связывания переменных имеют лексическую область видимости и неограниченную продолжительность.

Вместо списка \cd{(\emph{varj} \emph{valuej})}, можно записать просто
\emph{varj}.
В таком случае \emph{varj} будет инициализирована в {\false}. В целях стиля,
рекомендуется записывать \emph{varj}, только ей будет что-нибудь присвоено с
помощью \cdf{setq} перед первым использованием.
Если необходимо инициализировать переменную значением {\nil}, а не
неопределённым, лучше писать \cd{(\emph{varj} {\false})} для инициализации
<<ложью>> или \cd{(\emph{varj} '{\emptylist})} для инициализации пустым
списком.

В начале тела \cd{let*} могут использоваться декларации. Смотрите \cdf{declare}.
\end{defspec}

\begin{defspec}
progv symbols values {\,form}*

\cdf{progv} является специальной формой, которая позволяет создавать связывания
одной и более динамических переменных, чьи имена устанавливаются во время
выполнения. Последовательность форм (неявный \cdf{progn})
выполняется с динамическими переменными, что имена в списке \emph{symbols}
связаны с соответствующими значениями в списке \emph{values}.
(Если значений меньше, чем переменных, то соответствующие переменные получают
соответствующие значения, а оставшиеся остаются без значений. Смотрите
\cdf{makunbound}. Если значений больше, чем переменных, они игнорируются.)
Результатом \cdf{progv} является результат последней формы. Связывания
динамических переменных упраздняются при выходе из формы \cdf{progv}. Списки
переменных и значений это вычисляемые значения. Это то, что отличает \cdf{progv}
от, например, \cdf{let}, в которой имена переменных указываются явно в тексте
программы.

\cdf{progv} полезна, в частности, для написания интерпретаторов языков
встраиваемых в Lisp. Она предоставляет управление механизмом связывания
динамических переменных.
\end{defspec}


\begin{defmac}
flet ({(name lambda-list
        <{declaration}* | doc-string> {\,form}*)}*)
     {declaration}* {\,form}* \\
labels ({(name lambda-list
          <{declaration}* | doc-string> {\,form}*)}*)
       {declaration}* {\,form}* \\
macrolet ({(name varlist
            <{declaration}* | doc-string> {\,form}*)}*)
         {declaration}* {\,form}*

\cdf{flet} может быть использована для определения локальных именованных
функций. Внутри тела формы \cdf{flet}, имена функций, совпадающие с именами
определёнными в \cdf{flet}, ссылаются на локально определённые функции, а не на
глобальные определения функции с теми же именами.

Может быть определено любое количество функций. Каждое определение
осуществляется формате, как в форме \cdf{defun}: сначала имя, затем список
параметров (который может содержать \cd{\&optional}, \cd{\&rest} или \cd{\&key}
параметры), затем необязательные декларации и строка документации, и, наконец,
тело.
\begin{lisp}
(flet ((safesqrt (x) (sqrt (abs x)))) \\*
~~;; Функция safesqrt используется в двух местах. \\*
~~(safesqrt (apply \#'+ (map 'list \#'safesqrt longlist))))
\end{lisp}

Конструкция \cdf{labels} идентична по форме конструкции \cdf{flet}.
Эти конструкции различаются в том, что область видимости определённых функций
для \cdf{flet} заключена только в теле, тогда как видимость в \cdf{labels}
охватывает даже определения этих функций. Это значит, что \cdf{labels} может
быть использована для определения взаимно рекурсивных функций, а \cdf{flet} не
может. Это различие бывает полезно. Использование \cdf{flet} может локально
переопределить глобальную функцию, и новое определение может ссылаться на
глобальное. Однако такая же конструкция \cdf{labels} не будет обладать этим
свойством.
\begin{lisp}
(defun integer-power (n k)~~~~~~~; Быстрое возведение \\*
~~(declare (integer n))~~~~~~~~~~; целого числа в степень \\*
~~(declare (type (integer 0 *) k)) \\
~~(labels ((expt0 (x k a) \\*
~~~~~~~~~~~~~(declare (integer x a) (type (integer 0 *) k)) \\*
~~~~~~~~~~~~~(cond ((zerop k) a) \\*
~~~~~~~~~~~~~~~~~~~((evenp k) (expt1 (* x x) (floor k 2) a)) \\*
~~~~~~~~~~~~~~~~~~~(t (expt0 (* x x) (floor k 2) (* x a))))) \\
~~~~~~~~~~~(expt1 (x k a) \\*
~~~~~~~~~~~~~(declare (integer x a) (type (integer 1 *) k)) \\*
~~~~~~~~~~~~~(cond ((evenp k) (expt1 (* x x) (floor k 2) a)) \\*
~~~~~~~~~~~~~~~~~~~(t (expt0 (* x x) (floor k 2) (* x a)))))) \\*
~~~~(expt0 n k 1)))
\end{lisp}

\cdf{macrolet} похожа на форму \cdf{flet}, но определяет локальные макросы,
используя тот же формат записи, что и \cdf{defmacro}.
Имена для макросов, установленные с помощью \cdf{macrolet}, имеют лексическую
область видимости.

Макросы часто должны быть раскрыты во <<время компиляции>> (общими словами,
во время перед тем, как сама программа будет выполнена), таким образом, значения
переменных во время выполнения не доступны для макросов, определённых с помощью
\cdf{macrolet}.

Однако, сущности, имеющие лексическую область видимости, \emph{видны} внутри
тела формы \cdf{macrolet} и \emph{видны} в коде, который является результатом
раскрытия макровызова. Следующий пример должен помочь в понимании:
\begin{lisp}
;;; Пример macrolet. \\*
\\*
(defun foo (x flag) \\*
~~(macrolet ((fudge (z) \\*
~~~~~~~~~~~~~~~~;;\textrm{Параметры \cd{x} и \cdf{flag} в данной точке} \\*
~~~~~~~~~~~~~~~~;; \textrm{недоступны; ссылка на \cd{flag} была бы} \\*
~~~~~~~~~~~~~~~~;; \textrm{одноимённую глобальную переменную.} \\*
~~~~~~~~~~~~~~~~{\Xbq}(if flag \\
~~~~~~~~~~~~~~~~~~~~~(* ,z ,z) \\
~~~~~~~~~~~~~~~~~~~~~,z))) \\
~~~~;;\textrm{Параметры \cd{x} и \cd{flag} доступны здесь.} \\*
~~~~(+ x \\*
~~~~~~~(fudge x) \\*
~~~~~~~(fudge (+ x 1)))))
\end{lisp}
Тело данного примера после разворачивания макросов превращается в
\begin{lisp}
(+ x \\*
~~~(if flag \\*
~~~~~~~(* x x) \\*
~~~~~~~x)) \\*
~~~(if flag \\*
~~~~~~~(* (+ x 1) (+ x 1)) \\*
~~~~~~~(+ x 1)))
\end{lisp}
\cd{x} и \cd{flag} легитимно ссылаются на параметры функции
\cd{foo}, потому что эти параметры видимы в месте макровызова.
\end{defmac}

\begin{defspec}
symbol-macrolet ({(var expansion)}*)
                {declaration}* {\,form}*

X3J13 проголосовал в июне 1988
\issue{CLOS}
адаптировать Common Lisp'овую систему объектов (CLOS). Часть этого является
общий механизм, \cdf{symbol-macrolet}, для обработки заданных имён переменным,
как если бы они были макровызовами без параметров. Эта функциональность
полезно независимо от CLOS.

Формы \emph{forms} выполняются как неявный \cdf{progn} в лексическом окружении,
в котором любая ссылка на обозначенную переменную \emph{var} будет заменена на
соответствующее выражение \emph{expansion}. Это происходит, как будто ссылка на
переменную \emph{var} является макровызовом без параметров.
Выражение \emph{expansion} вычисляется или обрабатывается в месте появления
ссылки. Однако, следует отметить, что имена таких макросимволов работает в
пространстве имен переменных, не в пространстве функций. 
Использование \cdf{symbol-macrolet} может быть в свою очередь перекрыто с
помощью \cdf{let} или другой конструкцией, связывающей переменные. Например:
\begin{lisp}
(symbol-macrolet ((pollyanna 'goody)) \\*
~~(list pollyanna (let ((pollyanna 'two-shoes)) pollyanna))) \\*
~{\EV} (goody two-shoes)\textrm{, \emph{not}} (goody goody)
\end{lisp}

Выражение \emph{expansion} для каждой переменной \emph{var} вычисляется не во
время связывания, а во время подстановки вместо ссылок на \emph{var}.
Конструкция возвращает значения последней вычисленной формы, или \cdf{nil}, если
таких значений не было.

Смотрите документация \cdf{macroexpand} и \cdf{macroexpand-1}. Они раскрывают
макросы символов, также как и обычные макросы.

Указанные декларации \emph{declarations} перед телом обрабатываются так как
описано в разделе~\ref{DECLARE-SYNTAX-SECTION}.
\end{defspec}

\section{Операторы условных переходов}

Традиционная условная конструкция в Lisp'е это \cdf{cond}.
Однако, \cdf{if} гораздо проще и очень похожа на условные конструкции в других
языках программирования. Она сделана примитивом в Common Lisp'е.
Common Lisp также предоставляет конструкции диспетчеризации (распределения)
\cdf{case} и \cdf{typecase}, которые часто более удобны, чем \cdf{cond}.

\begin{defspec}
if test then [else]

Специальная формы \cdf{if} обозначает то же, что и конструкция
\textbf{if}-\textbf{then}-\textbf{else} в большинстве других языках
программирования.
Сначала выполняется форма \emph{test}. Если результат не равен {\false}, тогда
выбирается форма \emph{then}. Иначе выбирается форма \emph{else}.
Выбранная ранее форма выполняется, и \cdf{if} возвращает то, что вернула это
форма.
\begin{lisp}
(if \emph{test} \emph{then} \emph{else}) \EQ\ (cond (\emph{test} \emph{then}) ({\true} \emph{else}))
\end{lisp}
Но в некоторых ситуациях \cdf{if} оказывается более читабельным.

Форма \emph{else} может быть опущена. В таком случае, если значение формы
\emph{test} является {\false}, тогда ничего не будет выполнено и возвращаемое
значение формы \cdf{if} будет {\false}.
Если в этой ситуации значение формы \cdf{if} важно, тогда в зависимости от
контекста стилистически удобнее использовать форму \cdf{and}.
Если значение не важно, тогда удобнее использовать конструкцию \cdf{when}.
\end{defspec}

\begin{defmac}
when test {\,form}*

\cd{(when \emph{test} \emph{form1} \emph{form2} ... )}
сначала выполняет \emph{test}. Если результат {\false}, тогда ничего не
выполняется и возвращается {\false}.
Иначе, последовательно выполняются формы \emph{form}
слева направо (как неявный \cdf{progn}), и возвращается значение последней
формы.
\begin{lisp}
(when \emph{p} \emph{a} \emph{b} \emph{c}) \EQ\ (and \emph{p} (progn \emph{a} \emph{b} \emph{c})) \\
(when \emph{p} \emph{a} \emph{b} \emph{c}) \EQ\ (cond (\emph{p} \emph{a} \emph{b} \emph{c})) \\
(when \emph{p} \emph{a} \emph{b} \emph{c}) \EQ\ (if \emph{p} (progn \emph{a} \emph{b} \emph{c}) {\false}) \\
(when \emph{p} \emph{a} \emph{b} \emph{c}) \EQ\ (unless (not \emph{p}) \emph{a} \emph{b} \emph{c})
\end{lisp}
В целях хорошего стиля, \cdf{when} обычно используется для выполнения побочных
эффектов при некоторых условиях, и значение \cdf{when} не используется.
Если значение все-таки важно, тогда, может быть, стилистически функции
\cdf{and} или \cdf{if} более подходят.
\end{defmac}

\begin{defmac}
unless test {\,form}*

\cd{(unless \emph{test} \emph{form1} \emph{form2} ... )}
сначала выполняет \emph{test}. Если результат \emph{не} {\false}, тогда ничего не
выполняется и возвращается {\false}.
Иначе, последовательно выполняются формы \emph{form}
слева направо (как неявный \cdf{progn}), и возвращается значение последней
формы.
\begin{lisp}
(unless \emph{p} \emph{a} \emph{b} \emph{c}) \EQ\ (cond ((not \emph{p}) \emph{a} \emph{b} \emph{c})) \\
(unless \emph{p} \emph{a} \emph{b} \emph{c}) \EQ\ (if \emph{p} {\false} (progn \emph{a} \emph{b} \emph{c})) \\
(unless \emph{p} \emph{a} \emph{b} \emph{c}) \EQ\ (when (not \emph{p}) \emph{a} \emph{b} \emph{c})
\end{lisp}
В целях хорошего стиля, \cdf{unless} обычно используется для выполнения побочных
эффектов при некоторых условиях, и значение \cdf{unless} не используется.
Если значение все-таки важно, тогда может быть стилистически более подходящая
функция \cdf{if}.
\end{defmac}

\begin{defmac}
cond {(test {\,form}*)}*

Форма \cdf{cond} содержит некоторое (возможно нулевое) количество подвыражений,
которые является списками форм.
Каждое подвыражение содержит форму условия и ноль и более форм для выполнения.
Например:
\begin{lisp}
(cond (\emph{test-1} \emph{consequent-1-1} \emph{consequent-1-2} ...) \\
~~~~~~(\emph{test-2}) \\
~~~~~~(\emph{test-3} \emph{consequent-3-1} ...) \\
~~~~~~... )
\end{lisp}

Отбирается первое подвыражение, чья форма условия вычисляется в не-{\false}. Все
остальные подвыражения игнорируются. Формы отобранного подвыражения
последовательно выполняются (как неявный \cdf{progn}).

Если быть точнее, \cdf{cond} обрабатывает свои подвыражения слева направо. Для
каждого подвыражения, вычисляется форма условия. Если результат {\false},
\cdf{cond} переходит к следующему подвыражению. Если результат {\true},
\emph{cdr} подвыражения обрабатывается, как список форм. Этот список выполняется
слева направо, как неявный \cdf{progn}.
После выполнения списка форм, \cdf{cond} возвращает управление без обработки
оставшихся подвыражений.
Специальная форма \cdf{cond} возвращает результат выполнения последней формы из
списка. Если этот список пустой, тогда возвращается значение формы условия.
Если \cdf{cond} вернула управление без вычисления какой-либо ветки (все условные
формы вычислялись в {\false}), возвращается значение {\false}.

Для того, чтобы выполнить последнее подвыражение, в случае если раньше ничего не
выполнилось, можно использовать {\true} для формы условия.
В целях стиля, если значение \cdf{cond} будет для чего-то использоваться,
желательно записывать последнее выражение так: \cd{({\true} {\false})}.
Также вопросом вкуса является запись последнего подвыражения \cdf{cond} как
<<синглтон>>, в таком случае, используется неявный {\true}.
(Следует отметить, если \emph{x} может возвращать несколько значений, то
\cd{(cond ... (\emph{x}))} может вести себя отлично от 
\cd{(cond ... ({\true} \emph{x}))}. Первое выражение всегда возвращает одно
выражение, тогда как второе возвращает все то же, что и \emph{x}. В зависимости
от стиля, можно указывать поведение явно \cd{(cond ... (t (values \emph{x})))},
используя функцию \cdf{values} для явного указания возврата одного значения.)
Например:
\begin{lisp}
~~~~~~~~~~~~~~~~~~~~~~~~~~~~~~~~~~~~~~~~~~~~~~\=\kill
(setq z (cond (a 'foo) (b 'bar)))\>;\textrm{Возможна неопределённость} \\*
(setq z (cond (a 'foo) (b 'bar) ({\true} {\false})))\>;\textrm{Уже лучше} \\
(cond (a b) (c d) (e))\>;\textrm{Возможна неопределённость} \\*
(cond (a b) (c d) ({\true} e))\>;\textrm{Уже лучше} \\*
(cond (a b) (c d) ({\true} (values e)))\>;\textrm{Неплохо (если необходимо} \\*
                                       \>;~\textrm{одно значение)} \\
(cond (a b) (c))\>;\textrm{Возможна неопределённость} \\
(cond (a b) (t c))\>;\textrm{Уже лучше} \\*
(if a b c)\>;\textrm{Тоже неплохо}
\end{lisp}
Lisp'овая форма \cdf{cond} сравнима с последовательностью
\textbf{if}-\textbf{then}-\textbf{else}, используемой в большинстве
алгебраических языках программирования:
\begin{lisp}
~~~~~~~~~~~~~~~~~~~~\=~~~~~~~~~~~~~~~~~~~~~\=\kill
(cond (\emph{p} ...)\>\>\textbf{if} \emph{p} \textbf{тогда} ... \\
~~~~~~(\emph{q} ...)\>\textrm{roughly}\>\textbf{иначе} \textbf{если} \emph{q} \textbf{тогда} ... \\
~~~~~~(\emph{r} ...)\>\textrm{corresponds}\>\textbf{иначе} \textbf{если} \emph{r} \textbf{тогда} ... \\
~~~~~~...\>\textrm{to}\>... \\
~~~~~~({\true} ...))\>\>\textbf{иначе} ...
\end{lisp}
\end{defmac}

\begin{defmac}
case keyform {({({key}*) | key} {\,form}*)}*

\cdf{case} условный оператор, который выбирает ветку для выполнения, в
зависимости от равенства некоторой переменной некоторой константе. Константа
обычно представляет собой ключевой символ, целое число или строковый символ (но
может быть и любой другой объект). Вот развёрнутая форма:
\begin{lisp}
(case \emph{keyform} \\
~~(\emph{keylist-1} \emph{consequent-1-1} \emph{consequent-1-2} ...) \\
~~(\emph{keylist-2} \emph{consequent-2-1} ...) \\
~~(\emph{keylist-3} \emph{consequent-3-1} ...) \\
~~...)
\end{lisp}
Структурно \cdf{case} очень похож на \cdf{cond}, и поведение такое же: выбрать
список форм и выполнить их.
Однако \cdf{case} отличается механизмом выбора подвыражения.

Сперва \cdf{case} вычисляет форму \emph{keyform} для получения объекта, который
называется \emph{ключевой объект}.
Затем \cdf{case} рассматривает все подвыражения. Если \emph{ключевой объект}
присутствует в списке \emph{keylist} (то есть, если \emph{ключевой объект} равен
\cdf{eql} хотя бы одному элементу из списка \emph{keylist}), то список форм
выбранного подвыражения вычисляется, как неявный \cdf{progn}.
\cdf{case} возвращает то же, что и последняя форма списка (или {\false} если
список форм был пустой).
Если ни одно подвыражение не удовлетворило условию, то \cdf{case} возвращает
{\false}.

Ключи в списке ключей \emph{keylist} \emph{не} выполняются. В данном списке
должны быть указаны литеральные ключи.
Если один ключ попадается более чем в одном подвыражении, это считается
ошибкой.
Следствием является то, что порядок этих подвыражений не влияет на поведение
конструкции \cdf{case}.

Вместо \emph{keylist} можно записать один из символов: {\true} или
\cdf{otherwise}. Подвыражение с одним из таких символов всегда удовлетворяет
условию выбора. Такое подвыражение должно быть последним (это исключение из
правила о произвольности положения подвыражений).
Смотрите также \cdf{ecase} и \cdf{ccase}, каждая из которых предоставляет
неявное \cdf{otherwise} подвыражение для сигнализирования об ошибке, если ни
одно подвыражение не удовлетворило условию.

Если в подвыражении только один ключ, тогда этот ключ может быть записан вместо
списка.
Такой <<синглтоновый ключ>> не может быть {\nil} (так как возникают конфликты с
{\emptylist}, который означает список без ключей), {\true}, \cdf{otherwise} или
cons-ячейкой.
\end{defmac}

\begin{defmac}
typecase keyform {(type {\,form}*)}*

\cdf{typecase} условный оператор, который выбирает подвыражение на основе типа
объекта.
Развёрнутая форма:
\begin{lisp}
(typecase \emph{keyform} \\*
~~(\emph{type-1} \emph{consequent-1-1} \emph{consequent-1-2} ...) \\*
~~(\emph{type-2} \emph{consequent-2-1} ...) \\*
~~(\emph{type-3} \emph{consequent-3-1} ...) \\
~~...)
\end{lisp}
Структура \cdf{typecase} похожа на \cdf{cond} или \cdf{case}. Поведение также
схоже в том, что выбирается подвыражение в зависимости от условия.
Различие заключается в механизме выбора подвыражения.

Сперва \cdf{typecase} вычисляет форму \emph{keyform} для создания объекта,
называемого ключевым объектом.
Далее \cdf{typecase} друг за другом рассматривает каждое подвыражение. Форма
\emph{type}, которая встречается в каждом подвыражении, является спецификатором
типа. Данный спецификатор не вычисляется, поэтому должен быть литеральным.
Когда ключевой объект принадлежит некоторый типу, то выделенный список форм
\emph{consequent} выполняется последовательно (как неявный
\cdf{progn}). \cdf{typecase} возвращает то, что вернула последняя форма из
списка (или {\false} если список был пуст).
Если не одно подвыражение не было выбрано, \cdf{typecase} возвращает {\false}.

Как и для \cdf{case} можно использовать {\true} или \cdf{otherwise} на позиции
\emph{типа} для задания подвыражений, которые будут выполняться, только если не
было выполнено других подвыражений.
Смотрите также \cdf{etypecase} и \cdf{ctypecase}, каждая из которых
предоставляет неявную ветку \cdf{otherwise} для сигнализирования об ошибке, что
ни одно подвыражение не удовлетворило условию.

Допустимо указывать более одного подвыражение, тип условия которого уже является
подтипом условия другого подвыражения. В таком случае будет выбрано первое
встретившееся подвыражение. Таким образом в \cdf{typecase}, в отличие от
\cdf{case}, порядок следования подвыражений влияет на поведение всей
конструкции.
\begin{lisp}
~~~~~~~~~~~~~~~~~~~~~~~~~~~~~~\=\kill
(typecase an-object \\
~~~(string ...)\>;\textrm{Подвыражение обрабатывает строки} \\
~~~((array t) ...)\>;\textrm{Подвыражение обрабатывает общие массивы} \\
~~~((array bit) ...)\>;\textrm{Подвыражение обрабатывает битовые массивы} \\
~~~(array ...)\>;\textrm{Обрабатывает все остальные массивы} \\
~~~((or list number) ...)\>;\textrm{Подвыражение обрабатывает списки и числа} \\
~~~(t ...))\>;\textrm{Подвыражение обрабатывает все остальные объекты}
\end{lisp}
\end{defmac}

\section{Блоки и выходы}
\label{BLOCK-RETURN-SECTION}

Конструкции \cdf{block} и \cdf{return-from} предоставляют функциональность для
структурированного лексического нелокального выхода. В любом месте лексически
внутри конструкции \cdf{block}, для мгновенного возврата управления из
\cdf{block} может быть использована \cdf{return-from} с тем же именем.
В большинстве случаев этот механизм более эффективный, чем функциональность
динамического нелокального выхода, предоставляемая формами \cdf{catch} и
\cdf{throw}, описанными в разделе~\ref{CATCH-THROW-SECTION}.

\begin{defspec}
block name {\,form}*

Конструкция \cdf{block} слева направо выполняет каждую форму \emph{form},
возвращая то, что возвращает последняя форма.
Если, однако, в процессе выполнения форм, будет выполнена \cdf{return} или
\cdf{return-from} с именем \emph{name}, тогда будет возвращён результат заданный
одной из этих форм, и поток выполнения немедленно выйдет из формы \cdf{block}.
Таким образом \cdf{block} отличается от \cdf{progn} тем, что последняя никак не
реагирует на \cdf{return}.

Имя блока не выполняется. Оно должно быть символом.
Область видимости имени блока лексическая. Из блока можно осуществить выход с
помощью \cdf{return} или \cdf{return-from}, только если они содержаться в тексте
в блоке. Продолжительность видимости имени динамическая.
Таким образом из блока во время выполнения можно выйти только один раз, обычно
или явно с помощью \cdf{return}.

Форма \cdf{defun} неявно помещает тело функции в одноимённый блок. 
Таким образом можно использовать \cdf{return-from} для преждевременного выхода
из функции в определении \cdf{defun}.

Лексическая область видимости имени блока полноценна и имеет последствия,
которые могут быть сюрпризом для пользователей и разработчиков других Lisp
систем.
Например, \cdf{return-from} в следующем примере в Common Lisp работает так как
и ожидается:
\begin{lisp}
(block loser \\
~~~(catch 'stuff \\
~~~~~~(mapcar \#'(lambda (x) (if (numberp x) \\
~~~~~~~~~~~~~~~~~~~~~~~~~~~~~~~~(hairyfun x) \\
~~~~~~~~~~~~~~~~~~~~~~~~~~~~~~~~(return-from loser {\nil}))) \\
~~~~~~~~~~~~~~items)))
\end{lisp}
В зависимости от ситуации, \cdf{return} в Common Lisp'е может быть не проста.
\cdf{return} может перескочить ловушки, если это необходимо, для
рассматриваемого блока.
Также возможно для <<замыкания>>, созданного с помощью \cdf{function} для
лямбда-выражения, ссылаться на имя блока на протяжении лексической доступности
этого блока.
\end{defspec}

\begin{defspec}
return-from name [result]

\cdf{return-from} используется для возврата из \cdf{block} или из таких
конструкций, как \cdf{do} и \cdf{prog}, которые неявно устанавливают
\cdf{block}.
Имя \emph{name} не выполняется и должно быть символом.
Конструкция \cdf{block} с этим именем должна лексически охватывать форму
\cdf{return-from}.
Каков бы ни был результат вычисления формы \emph{result}, управление немедленно
возвращается из блока.
(Если форма \emph{result} опущена, тогда используется значение {\nil}. В целях
стиля, эта форма обязана использоваться для указания того, что возвращаемое
значение не имеет ценности.)

Форма \cdf{return-from} сама по себе ничего и никогда не возвращает.
Она указывает на то, что результат выполнения будет возвращён из конструкции
\cdf{block}.
Если вычисление формы \emph{result} приводит к нескольким значение, эти
несколько значений и будут возвращены и конструкции.
\end{defspec}

\begin{defmac}
return [result]

\cd{(return \emph{form})} идентично по смыслу 
\cd{(return-from {\nil} \emph{form})}. Она возвращает управление из блока с
именем {\nil}.
Такие блоки с именем {/nil} устанавливаются автоматически в конструкциях циклов,
таких как \cdf{do}, таким образом \cdf{return} будет производить корректный
выход из таких конструкций.
\end{defmac}

\section{Формы циклов}
\indexterm{iteration}

Common Lisp предоставляет некоторые конструкции для циклов. Конструкция
\cdf{loop} предоставляет простую функциональность. Она слегка больше, чем
\cdf{progn}, и имеет ветку для переноса управления снизу вверх.
Конструкции \cdf{do} и \cdf{do*} предоставляют общую функциональность для
управления на каждом цикле изменением нескольких переменных.
Для специализированных циклов над элементами списка или \emph{n}
последовательных чисел предоставляются формы \cdf{dolist} и \cdf{dotimes}.
Конструкция \cdf{tagbody} наиболее общая конструкция, которая внутри себя
позволяет использование выражений \cdf{go}. (Традиционная конструкция \cdf{prog}
--- это синтез \cdf{tagbody}, \cdf{block} и \cdf{let}.)
Большинство конструкций циклов позволяют определённые статически нелокальные
выходы (смотрите \cdf{return-from} и \cdf{return}).

\subsection{Бесконечный цикл}

Конструкция \cdf{loop} является наипростейшей функциональностью для итераций.
Она не управляет переменными, и просто циклично выполняет своё тело.

\begin{defmac}
loop {\,form}*

Каждая форма \emph{form} выполняется последовательно слева направо.
Когда вычислена последняя форма, тогда вычисляется первая форма и так далее,
в безостановочном цикле.
Конструкция \cdf{loop} никогда не возвращает значение. Её выполнение может быть
остановлено явно, с помощью \cdf{return} или \cdf{throw}, например.

\cdf{loop}, как и многие конструкции циклов, устанавливает неявный блок с именем
{\nil}.
Таким образом, \cdf{return} с заданным результатом может использоваться для
выхода и \cdf{loop}.
\end{defmac}

\subsection{Базовые формы циклов}

В отличие от \cdf{loop}, \cdf{do} и \cdf{do*} предоставляют мощный механизм для
повторных вычислений большого количества переменных. 


\begin{defmac}
do ({var | (var [init [step]])}*)
   (end-test {result}*)
   {declaration}* {tag | statement}* \\
do* ({var | (var [init [step]])}*)
    (end-test {result}*)
    {declaration}* {tag | statement}*

Специальная форма \cdf{do} представляет общую функциональность цикла, с
произвольным количеством <<переменных-индексов>>.
Эти переменные связываются при входе в цикл и параллельно наращиваются, как это
было задано. Они могут быть использованы, как для генерации необходимых
последовательных чисел (как, например, последовательные целые числа), так и для
накопления результата.
Когда условие окончания цикла успешно выполнилось, тогда цикл завершается с
заданным значением.

В общем виде \cdf{do} выглядит так:
\begin{lisp}
(do ((\emph{var1} \emph{init1} \emph{step1}) \\
~~~~~(\emph{var2} \emph{init2} \emph{step2}) \\
~~~~~... \\
~~~~~(\emph{varn} \emph{initn} \emph{stepn})) \\
~~~~(\emph{end-test} . \emph{result}) \\
~~\Mstar{\emph{declaration}} \\
~~. \emph{tagbody})
\end{lisp}
Цикл \cdf{do*} выглядит также, кроме изменения имени с \cdf{do} на \cdf{do*}.

Первый элемент формы является списком нуля и более спецификаторов
переменных-индексов. Каждый спецификатор является списком из имени переменной
\emph{var}, первоначального значения \emph{init}, и форма приращения
\emph{step}.
Если \emph{init} опущен, используется первоначальное значение {\false}.
Если \emph{step} опущен, \emph{var} не изменяется на итерациях цикла (но может
изменяться в теле цикла с помощью формы \cdf{setq}).

Спецификатор переменной-индекса может также быть просто именем переменной.
В этом случае переменная будет иметь первоначальное значение {\false} и не будет
изменяться при итерациях цикла.
В целях стиля, использовать просто имя переменной рекомендуется, только если перед
первым использованием для неё устанавливается значение с помощью \cdf{setq}.
Если необходимо чтобы первоначальное значение было {\false}, а не
неопределённое, то лучше указывать это явно, если нужна ложь так:
\cd{(\emph{varj} {\false})}
или если нужен пустой список так:
\cd{(\emph{varj} '{\emptylist})}.

Перед первой итерацией вычисляются все формы \emph{init}, и каждая переменная
\emph{var} связывается с соответствующим результатом вычислений \emph{init}.
Используется именно связывание, а не присвоение. Когда цикл завершается, старые
значения этих переменных восстанавливаются.
Для \cdf{do}, \emph{все} формы \emph{init} вычисляется перед тем, как будут
связаны переменные \emph{var}. Таким образом все формы могут ссылаться на старые
связывания этих переменных
(то есть на значения, которые были видимы до начала выполнения конструкции
\cdf{do}).
Для \cdf{do*} вычисляется первая форма \emph{init}, затем первая переменная
связывается с результатом этих вычислений. Затем вычисляется вторая форма
\emph{init} и вторая переменная \emph{var} связывается с этим значением, и так
далее.
В целом, форма \emph{initj} может ссылаться на \emph{новые} связывания
\emph{vark}, если $k<j$, иначе ссылка происходит на \emph{старое} связывание.

Второй элемент конструкции цикла это список из формы предиката-выхода
\emph{end-test} и нуля и более форм результата \emph{result}.
Этот элемент напоминает подвыражение \cdf{cond}.
В начале каждой итерации, после обработки всех переменных, вычисляется форма
\emph{end-test}. Если результат {\false}, выполняется тело формы \cdf{do} (или
\cdf{do*}).
Если результат не {\false}, последовательно вычисляются формы \emph{result}, как
неявный \cdf{progn},
и затем \cdf{do} возвращает управление. \cdf{do} возвращает результаты
вычисления последней формы \emph{result}.
Если таких форм не быть, значением \cdf{do} становиться {\false}.
Следует отметить, что аналогия с подвыражениями \cdf{cond} не полная, так как
\cdf{cond} в этом случае возвращает результат формы условия.

Переменные-индексы изменяются в начале каждой непервой итерации так, как
написано далее. Слева направо вычисляются все формы \emph{step}, и затем
результаты присваиваются переменным-индексам.
Если такой формы \emph{step} для переменной указано не было, то переменная и не
изменяется.
Для \cdf{do}, все формы \emph{step} вычисляются перед там, как будут изменены
переменные. Присваивания переменным осуществляются параллельно, как в
\cdf{psetq}.
Так как \emph{все} формы \emph{step} вычисляются перед тем, как будет изменена
хоть одна переменных, форма \emph{step} при вычислении всегда ссылается на
старые значения \emph{всех} переменных-индексов, даже если другие формы
\emph{step} были выполнены.
Для \cdf{do*}, вычисляется первая форма \emph{step}, затем полученное значение
присваивается первой переменной-индексом, затем вычисляется вторая форма
\emph{step}, и полученное значение присваивается второй переменной, и так
далее. Присваивание происходит последовательно, как в \cdf{setq}.
И для \cdf{do}, и для \cdf{do*} после того как переменные были изменены,
вычисляется \emph{end-test} так, как уже было описано выше. Затем продолжаются
итерации.

Если \emph{end-test} формы \cdf{do} равен \cd{{\false}}, тогда предикат всегда
ложен.
Таким образом получается <<бесконечный цикл>>:
тело \emph{body} \cdf{do} выполняется циклично, переменные-индексы изменяются
как обычно. (Конструкция \cdf{loop} также является <<бесконечным циклом>>,
только без переменных-индексов.)
Бесконечный цикл может быть остановлен использованием \cdf{return},
\cdf{return-from}, \cdf{go} на более высокий уровень или \cdf{throw}.
Например:
\begin{lisp}
(do ((j 0 (+ j 1))) \\
~~~~({\false})~~~~~~~~~~~~~~~~~~~~~~~~;\textrm{Выполнять вечно} \\
~~(format t "{\Xtilde}\%Input {\Xtilde}D:" j) \\
~~(let ((item (read))) \\
~~~~(if (null item) (return)~~~~~;\textrm{Обрабатывать элементы пока не найден {\false}} \\
~~~~~~~~(format t "{\Xtilde}\&Output {\Xtilde}D: {\Xtilde}S" j (process item)))))
\end{lisp}

Оставшаяся часть \cdf{do} оборачивается в неявный \cdf{tagbody}.
Теги могут использоваться внутри тела цикла \cdf{do} для того, чтобы затем
использовать выражения \cdf{go}. На такие выражения \cdf{go} не могут
использоваться в спецификаторах переменных-индексов, в предикате
\emph{end-test} и в формах результата \emph{result}.
Когда управление достигает конца тела цикла \cdf{do}, наступает следующая
цикл итерации (начинающийся с вычисления форм \emph{step}).

Неявный \cdf{block} с именем {\nil} окружает всю форму \cdf{do}.
Выражение \cdf{return} может использоваться в любом месте для немедленного
выхода из цикла.

Формы \cdf{declare} могут использоваться в начала тела \cdf{do}.
Они применяются к коду внутри тела \cdf{do}, для связываний переменных-индексов,
для форм \emph{init}, для форм \emph{step}, для предиката \emph{end-test} и для
форм результата \emph{result}.

Вот парочка примеров использования \cdf{do}:
\begin{lisp}
(do ((i 0 (+ i 1))~~~~~;\textrm{Sets every null element of \cdf{a-vector} to zero} \\*
~~~~~(n (length a-vector))) \\*
~~~~((= i n)) \\
~~(when (null (aref a-vector i)) \\*
~~~~(setf (aref a-vector i) 0)))
\end{lisp}
Конструкция
\begin{lisp}
(do ((x e (cdr x)) \\*
~~~~~(oldx x x)) \\*
~~~~((null x)) \\*
~~\emph{body})
\end{lisp}
использует параллельное присваивание переменным-индексам. На первой итерации
значение \cdf{oldx} получает значение \cd{x}, которое было до входа в цикл.
При выходе из цикла \cd{oldx} будет содержать значение \cd{x}, которое было на
предыдущей итерации.

Очень часто алгоритм цикла может быть по большей части выражен в формах
\emph{step} и тело при этом останется пустым.
Например,
\begin{lisp}
(do ((x foo (cdr x)) \\*
~~~~~(y bar (cdr y)) \\*
~~~~~(z '{\emptylist} (cons (f (car x) (car y)) z))) \\
~~~~((or (null x) (null y)) \\*
~~~~~(nreverse z)))
\end{lisp}
делает то же, что и \cd{(mapcar \#'f foo bar)}. Следует отметить, что вычисление
\emph{step} для \cd{z} использует тот факт, что переменные переприсваиваются
параллельно.
Тело функции пустое. Наконец, использование \cdf{nreverse} в форме возврата
результата, переставляет элементы списка для правильного результата. Другой
пример:
\begin{lisp}
(defun list-reverse (list) \\*
~~~~~~~(do ((x list (cdr x)) \\*
~~~~~~~~~~~~(y '{\emptylist} (cons (car x) y))) \\*
~~~~~~~~~~~((endp x) y)))
\end{lisp}
Нужно заметить, что используется \cdf{endp} вместо \cdf{null} или \cdf{atom}
для проверки конца списка. Это даёт более надёжный алгоритм.

В качестве примера вложенных циклов, предположим что \cd{env} содержит список
cons-ячеек.
\emph{car} элемента каждой cons-ячейки является списком символов, и \emph{cdr}
каждой cons-ячейки является списком такой же длины с соответствующими
значениями.
Такая структура данных похожа на ассоциативный список, но она в отличие
разделена на <<кадры>>. Общая структура напоминает грудную клетку.
Функция поиска по такой структуре может быть такой:
\begin{lisp}
(defun ribcage-lookup (sym ribcage) \\*
~~~~~~~(do ((r ribcage (cdr r))) \\*
~~~~~~~~~~~((null r) {\false}) \\
~~~~~~~~~(do ((s (caar r) (cdr s)) \\*
~~~~~~~~~~~~~~(v (cdar r) (cdr v))) \\*
~~~~~~~~~~~~~((null s)) \\
~~~~~~~~~~~(when (eq (car s) sym) \\*
~~~~~~~~~~~~~(return-from ribcage-lookup (car v))))))
\end{lisp}
(Примечание, использование отступов в примере выше выделяет тела вложенных
циклов.) 

Цикл \cdf{do} может быть выражен в терминах более примитивных конструкций 
\cdf{block}, \cdf{return}, \cdf{let}, \cdf{loop}, \cdf{tagbody}
и \cdf{psetq}:
\begin{lisp}
(block nil \\*
~~(let ((\emph{var1} \emph{init1}) \\*
~~~~~~~~(\emph{var2} \emph{init2}) \\
~~~~~~~~... \\*
~~~~~~~~(\emph{varn} \emph{initn})) \\*
~~~~\Mstar{\emph{declaration}} \\
~~~~(loop (when \emph{end-test} (return (progn . \emph{result}))) \\*
~~~~~~~~~~(tagbody . \emph{tagbody}) \\*
~~~~~~~~~~(psetq \emph{var1} \emph{step1} \\*
~~~~~~~~~~~~~~~~~\emph{var2} \emph{step2} \\*
~~~~~~~~~~~~~~~~~... \\*
~~~~~~~~~~~~~~~~~\emph{varn} \emph{stepn}))))
\end{lisp}
\cdf{do*} почти то же, что и \cdf{do} за исключением того, что связывание и
наращение переменных происходит последовательно, а не параллельно.
Таким образом, в вышеприведённой конструкции, \cdf{let} будет заменена на
\cdf{let*} и \cdf{psetq} на \cdf{setq}.
\end{defmac}

\subsection{Простые формы циклов}

Конструкции \cdf{dolist} и \cdf{dotimes} для каждого значения взятого для одной
переменной выполняют тело один раз. Они хоть и выражаются в терминах \cdf{do},
но захватывают очень простые шаблоны использования.

И \cdf{dolist} и \cdf{dotimes} циклично выполняют тело. На каждой итерации
заданная переменная связывается с элементом, которая затем может использоваться
в теле. \cdf{dolist} использует элементы списка, и \cdf{dotimes} использует
целые числа от 0 по $n-1$, при некотором указанном положительном целом \emph{n}.

Результат двух этих конструкций может быть указан с помощью необязательной формы
результата. Если эта форма опущена результат равен {\false}.

Выражение \cdf{return} может быть использовано для немедленного возврата из форм
\cdf{dolist} или \cdf{dotimes}, игнорируя все оставшиеся итерации, которые
должны были быть выполнены. \cdf{block} с именем {\nil} окружает конструкцию.
Тело цикла неявно обернуто конструкцией \cdf{tagbody}.
Таким образом тело может содержать теги и \cdf{go} выражения.
Декларации могут быть указаны перед телом цикла.

\begin{defmac}
dolist (var listform [resultform])
       {declaration}* {tag | statement}*

\cdf{dolist} предоставляет прямой цикл по списку элементов.
Сначала \cdf{dolist}
вычисляет форму \emph{listform}, которая должна вернуть список.
Затем для каждого элемента вычисляется тело цикла. Данный элемент на каждой
итерации связывается с переменной \emph{var}.
Затем вычисляется \emph{resultform} (одна форма, \emph{не} неявный \cdf{progn}).
(Когда вычисляется \emph{resultform}, переменная \emph{var} все ещё связана, и
имеет значение {\nil}.)
Если \emph{resultform} опущена, то результат равен {\false}.
\begin{lisp}
(dolist (x '(a b c d)) (prin1 x) (princ " ")) \EV\ {\false} \\
~~~\textrm{вывод <<\cd{a b c d }>> (в том числе пробел в конце)}
\end{lisp}
Для завершения цикла и возврата заданного значения может использоваться явное
выражение \cdf{return}.
\end{defmac}

\begin{defmac}
dotimes (var countform [resultform])
        {declaration}* {tag | statement}*

\cdf{dotimes} предоставляет цикл над последовательностью целых чисел.
Выражение
\cd{(dotimes (\emph{var} \emph{countform} \emph{resultform}) . \emph{progbody})}
вычисляет форму \emph{countform}, которая должна вернуть целое число. Затем
тело цикла выполняется по порядку один раз для каждого число от нуля (включительно) до
\emph{count} (исключая). При этом переменная \emph{var} связывается с текущим
целым числом. Если значение \emph{countform} отрицательно или равно нулю, тогда
\emph{progbody} не выполняется ни разу. Наконец выполняется \emph{resultform} (одна форма,
\emph{не} неявный \cdf{progn}), и полученный результат возвращается из формы
цикла.
(Когда \emph{result} вычисляется, переменная-индекс \emph{var} все ещё связана и
содержит количество выполненных итераций.)
Если \emph{resultform} опущена, то результат равен {\false}.

Для завершения цикла и возврата заданного значения может использоваться
выражение \cdf{return}.

Пример использования \cdf{dotimes} для обработки строк:
\begin{lisp}
;;; True if the specified subsequence of the string is a \\*
;;; palindrome (reads the same forwards and backwards). \\*
\\*
(defun palindromep (string \cd{\&optional} \\*
~~~~~~~~~~~~~~~~~~~~~~~~~~~(start 0) \\*
~~~~~~~~~~~~~~~~~~~~~~~~~~~(end (length string))) \\
~~(dotimes (k (floor (- end start) 2) {\true}) \\*
~~~~(unless (char-equal (char string (+ start k)) \\*
~~~~~~~~~~~~~~~~~~~~~~~~(char string (- end k 1))) \\*
~~~~~~(return {\false})))) \\
\\
(palindromep "Able was I ere I saw Elba") \EV\ {\true} \\
 \\
(palindromep "A man, a plan, a canal--Panama!") \EV\ {\false} \\
 \\
(remove-if-not \#'alpha-char-p~~~~~;\textrm{Удалить знаки препинания} \\*
~~~~~~~~~~~~~~~"A man, a plan, a canal--Panama!") \\*
~~~\EV\ "AmanaplanacanalPanama" \\
 \\
(palindromep \\*
~(remove-if-not \#'alpha-char-p \\*
~~~~~~~~~~~~~~~~"A man, a plan, a canal--Panama!")) \EV\ {\true} \\
 \\
(palindromep \\*
~(remove-if-not \\*
~~~\#'alpha-char-p \\*
~~~"Unremarkable was I ere I saw Elba Kramer, nu?")) \EV\ {\true} \\
 \\
(palindromep \\*
~(remove-if-not \\*
~~~\#'alpha-char-p \\*
~~~"A man, a plan, a cat, a ham, a yak, \\*
~~~~~~~~~~~~~~~~~~~a yam, a hat, a canal--Panama!")) \EV\ {\true}
\\
(palindromep \\*
~(remove-if-not \\*
~~~\#'alpha-char-p \\*
~~~"Ja-da, ja-da, ja-da ja-da jing jing jing")) \EV\ {\false}
\end{lisp}

Изменение значения переменной \emph{var} в теле цикла (с помощью \cdf{setq}
например) будет иметь непредсказуемые последствия, возможно зависящие от
реализации. Компилятор Common Lisp'а может вывести предупреждение о том, что
переменная-индекс используется в \cdf{setq}.
\end{defmac}

Смотрите также \cdf{do-symbols}, \cdf{do-external-symbols} и
\cdf{do-all-symbols}.

\subsection{Отображение}
\indexterm{mapping}

Отображение --- это тип цикла, в котором заданная функция применяется к частям
одной или более последовательностей. Результатом цикла является
последовательность, полученная из результатов выполнения это функции.
Существует несколько опции для указания того, какие части списка будут
использоваться в цикле, и что будет происходить с результатом применения
функции.

Функция \cdf{map} может быть использована для отображения любого типа
последовательности.
Следующие же функции оперируют только списками.

\begin{defun}[Функция]
mapcar function list &rest more-lists \\
maplist function list &rest more-lists \\
mapc function list &rest more-lists \\
mapl function list &rest more-lists \\
mapcan function list &rest more-lists \\
mapcon function list &rest more-lists

Для каждой из этих функций отображения,
первый аргумент является функцией и оставшиеся аргументы должны быть списками.
Функция в первом аргументе должно принимать столько аргументов, сколько было
передано списков в функцию отображения.

\cdf{mapcar} последовательно обрабатывает элементы списков.
Сначала функция применяется к \emph{car} элементу каждого списка,
затем к \emph{cadr} элементу, и так далее.
(Лучше всего, чтобы все переданные списки имели одинаковую длину. Если это не
так, то цикл завершиться, как только закончится самый короткий список, и все
оставшиеся элементы в других списках будут проигнорированы.)
Значение, возвращаемое \cdf{mapcar}, является списком результатов
последовательных вызовов функции из первого параметра.
Например:
\begin{lisp}
(mapcar \#'abs '(3 -4 2 -5 -6)) \EV\ (3 4 2 5 6) \\
(mapcar \#'cons '(a b c) '(1 2 3)) \EV\ ((a . 1) (b . 2) (c . 3))
\end{lisp}

\cdf{maplist} похожа на \cdf{mapcar} за исключением того, что функция
применяется к спискам и последующим \emph{cdr} элементам этих списков, а не
последовательно к элементам спискам.
Например:
\begin{lisp}
(maplist \#'(lambda (x) (cons 'foo x)) \\*
~~~~~~~~~'(a b c d)) \\*
~~~\EV\ ((foo a b c d) (foo b c d) (foo c d) (foo d))
\end{lisp}

\begin{lisp}
(maplist \#'(lambda (x) (if (member (car x) (cdr x)) 0 1))) \\*
~~~~~~~~~'(a b a c d b c)) \\*
~~~\EV\ (0 0 1 0 1 1 1) \\*
~~~;\textrm{Возвращается \cd{1}, если соответствующий элемент входящего списка} \\*
~~~;~\textrm{ появлялся последний раз в данном списке.}
\end{lisp}

\cdf{mapl} и \cdf{mapc} похожи на \cdf{maplist} и \cdf{mapcar}, соответственно,
за исключением того, что они не накапливают результаты вызова функций.

Эти функции используются, когда функция в первом параметре предназначена для
побочных эффектов, а не для возвращаемого значения.
Значение возвращаемое \cdf{mapl} или \cdf{mapc} является вторым аргументом, то
есть первой последовательностью для отображения.

\cdf{mapcan} и \cdf{mapcon} похожи на \cdf{mapcar} и \cdf{maplist}
соответственно, за исключением того, что результат создаётся с помощью функции
\cdf{nconc}, а не \cdf{list}. То есть, 
\begin{lisp}
(mapcon \emph{f} \emph{x1} ... \emph{xn}) \\
~~~\EQ\ (apply \#'nconc (maplist \emph{f} \emph{x1} ... \emph{xn}))
\end{lisp}
Такое же различие между \cdf{mapcan} и \cdf{mapcar}.
Концептуально, эти функции позволяют функции отображения возвращать переменное
количество элементов. Таким образом длина результата может быть не равна длине
входного списка.
Это, в частности, полезно для возврата нуля или одного элемента:
\begin{lisp}
(mapcan \#'(lambda (x) (and (numberp x) (list x))) \\
~~~~~~~~'(a 1 b c 3 4 d 5)) \\
~~~\EV\ (1 3 4 5)
\end{lisp}
В этом случае функция действует, как фильтр. Это стандартная Lisp'овая идиома
использования \cdf{mapcan}.
(Однако, в этом контексте функция \cdf{remove-if-not} также может быть полезна.)
Помните, что \cdf{nconc} деструктивная операция, следовательно и \cdf{mapcan} и
\cdf{mapcon} также деструктивны. Список возвращаемый функцией \emph{function}
изменяется для соединения и возврата результата.

Иногда \cdf{do} или прямая последовательная рекурсия удобнее, чем функции
отображения. Однако, функции отображения должны быть использованы везде, где они
действительно необходимы, так как они увеличивают ясность кода.

Функциональный аргумент функции отображения должен быть подходящим для функции
\cdf{apply}. Он не может быть макросов или именем специальной формы.
Кроме того, в качестве функционального аргумента можно использовать функцию,
имеющую \cd{\&optional} и \cd{\&rest} параметры.
\end{defun}

\subsection{Использование <<GOTO>>}

Реализации Lisp'а начиная с Lisp'а 1.5 содержат то, что изначально называлось
<<the program feature>>, как будто без этого невозможно писать программы!
Конструкция \cdf{prog} позволяет писать в Algol- или Fortran- императивном
стиле, используя выражения \cdf{go}, которые могут ссылаться на теги в теле
\cdf{prog}. Современный стиль программирования на Lisp'е стремится снизить
использование \cdf{prog}. Различные конструкции циклов, как \cdf{do}, имеют тела
с характеристиками \cdf{prog}.
(Тем не менее, возможность использовать выражения \cdf{go} внутри конструкции
цикла очень редко используется на практике.)

\cdf{prog} предоставляет три различные операции:
связывание локальный переменных,
использование выражения \cdf{return}
и использование выражения \cdf{go}.
В Common Lisp'е эти три операции были разделены на три конструкции:
\cdf{let}, \cdf{block} и \cdf{tagbody}.
Эти три конструкции могут использоваться независимо, как строительные кирпичики
для других типов конструкций.

\begin{defspec}
tagbody {tag | statement}*

Часть \cdf{tagbody} после списка переменные называется \emph{телом}.
Элемент тела может быть символов или целым числом, и называться в этом случае
\emph{тег}. Также элемент тела может быть списком, и называться в этом случае
\emph{выражением}.

Каждый элемент тела обрабатывается слева направо.
\emph{Теги} игнорируются. \emph{Выражения} вычисляются и их результаты
игнорируются. Если управление достигает конца тела, \cdf{tagbody} возвращает
{\false}.

Если вычисляется форма \cd{(go \emph{tag})}, управление перемещается на часть
тела, обозначенную \emph{тегом}.

Область видимости тегов, устанавливаемых в \cdf{tagbody}, является лексической. 
Продолжительность видимости тегов динамическая. Когда управление вышло из
конструкции \cdf{tagbody}, ссылка \cdf{go} на теги в её теле невозможны.
Существует возможность для \cdf{go} прыжка в \cdf{tagbody}, которая не находится
внутри конструкции \cdf{tagbody}, которая и содержала этот \cdf{go}. То есть
возможны прыжки в родительский \cdf{tagbody}.
Теги устанавливаемые \cdf{tagbody} будут только скрывать другие одноимённые
теги.

Лексическая область видимости для тегов (целей \cdf{go}) полностью полноправно и
последствия могут быть сюрпризом для пользователей и разработчиков других Lisp
систем.
Например, \cdf{go} в следующем примере, работает в Common Lisp так, как это и
ожидается:
\begin{lisp}
(tagbody \\*
~~~(catch 'stuff \\*
~~~~~~(mapcar \#'(lambda (x) (if (numberp x) \\*
~~~~~~~~~~~~~~~~~~~~~~~~~~~~~~~~(hairyfun x) \\*
~~~~~~~~~~~~~~~~~~~~~~~~~~~~~~~~(go lose))) \\*
~~~~~~~~~~~~~~items)) \\
~~~(return) \\*
~lose \\*
~~~(error "I lost big!"))
\end{lisp}
В зависимости от ситуации, \cdf{go} в Common Lisp'е не обязательно похож на
простую машинную инструкцию <<jump>>. Если необходимо, \cdf{go} может
перепрыгивать ловушки исключений. Возможно так, что <<замыкание>>, созданное с
помощью \cdf{function}, для лямбда-выражения ссылается на тег (цель \cdf{go})
так долго, сколько лексически доступен данный тег. Смотрите~\ref{SCOPE} для
понимания этого примера.
\end{defspec}

\begin{defmac}
prog ({var | (var [init])}*) {declaration}* {tag | statement}* \\
prog* ({var | (var [init])}*) {declaration}* {tag | statement}*

Конструкция \cdf{prog} является синтезом \cdf{let}, \cdf{block} и \cdf{tagbody},
позволяющая связывать переменные, использовать \cdf{return} и \cdf{go} в одной
конструкции. Обычно конструкция \cdf{prog} выглядит так:
\begin{lisp}
(prog (\emph{var1} \emph{var2} (\emph{var3} \emph{init3}) \emph{var4} (\emph{var5} \emph{init5})) \\*
~~~~~~\Mstar{\emph{declaration}} \\*
~~~~~~\emph{statement1} \\
~\emph{tag1} \\*
~~~~~~\emph{statement2} \\*
~~~~~~\emph{statement3} \\*
~~~~~~\emph{statement4} \\
~\emph{tag2} \\*
~~~~~~\emph{statement5} \\*
~~~~~~... \\*
~~~~~~)
\end{lisp}
Список после ключевого символа \cdf{prog} является множеством спецификаторов для
связывания переменных \emph{var1}, \emph{var2}.
Этот список обрабатывается так же, как и в выражении \cdf{let}:
сначала слева направо выполняются все формы \emph{init} (если формы нет, берётся
значение {\false}), и затем переменные параллельно связываются с полученными
ранее значениями. 
Возможно использовать \emph{декларации} в начале дела \cdf{prog} так же, как и в
\cdf{let}.

Тело \cdf{prog} выполняется, как обернутое в \cdf{tagbody}. Таким образом, для
перемещения управления к \emph{тегу} могут использоваться выражения \cdf{go}.

\cdf{prog} неявно устанавливает вокруг тела \cdf{block} с именем {\nil}. Это
значит, можно в любое время использовать \cdf{return} для выхода из конструкции \cdf{prog}.

Вот небольшой пример того, что можно сделать с помощью \cdf{prog}:
\begin{lisp}
(defun king-of-confusion (w) \\
~~"Take a cons of two lists and make a list of conses. \\
~~~Think of this function as being like a zipper." \\
~~(prog (x y z)~~~~~;\textrm{Инициализировать \cd{x}, \cd{y}, \cd{z} в {\false}} \\
~~~~~~~~(setq y (car w) z (cdr w)) \\
~~~loop \\
~~~~~~~~(cond ((null y) (return x)) \\
~~~~~~~~~~~~~~((null z) (go err))) \\
~~~rejoin \\
~~~~~~~~(setq x (cons (cons (car y) (car z)) x)) \\
~~~~~~~~(setq y (cdr y) z (cdr z)) \\
~~~~~~~~(go loop) \\
~~~err \\
~~~~~~~~(cerror "Will self-pair extraneous items" \\
~~~~~~~~~~~~~~~~"Mismatch - gleep!  ~S" y) \\
~~~~~~~~(setq z y) \\
~~~~~~~~(go rejoin)))
\end{lisp}
которые делает то же, что и:
\begin{lisp}
(defun prince-of-clarity (w) \\
~~"Take a cons of two lists and make a list of conses. \\
~~~Think of this function as being like a zipper." \\
~~(do ((y (car w) (cdr y)) \\
~~~~~~~(z (cdr w) (cdr z)) \\
~~~~~~~(x '{\emptylist} (cons (cons (car y) (car z)) x))) \\
~~~~~~((null y) x) \\
~~~~(when (null z) \\
~~~~~~(cerror "Will self-pair extraneous items" \\
~~~~~~~~~~~~~~"Mismatch - gleep!  ~S" y) \\
~~~~~~(setq z y))))
\end{lisp}

Конструкция \cdf{prog} может быть выражена в терминах более простых конструкций
\cdf{block}, \cdf{let} и \cdf{tagbody}:
\begin{lisp}
(prog \emph{variable-list} \Mstar{\emph{declaration}} . \emph{body}) \\
~~~\EQ\ (block nil (let \emph{variable-list} \Mstar{\emph{declaration}} (tagbody . \emph{body})))
\end{lisp}

Специальная форма \cdf{prog*} очень похожа на \cdf{prog}. Одно отличие в том,
что связывание и инициализация переменных осуществляется \emph{последовательно},
тем самым форма \emph{init} использовать значения ранее связанных переменных.
Таким образом \cdf{prog*} относится к \cdf{prog}, как \cdf{let*} к \cdf{let}.
Например,
\begin{lisp}
(prog* ((y z) (x (car y))) \\
~~~~~~~(return x))
\end{lisp}
возвращает \emph{car} элемент значения \cd{z}.
\end{defmac}

\begin{defspec}
go tag

Специальная форма \cd{(go \emph{tag})} используется для применения <<goto>>
внутри конструкции \cdf{tagbody}. \emph{tag} должен быть символов или целым
числом. \emph{tag} не вычисляется.
\cdf{go} переносит управление на точку тела, которая была помечена тегом равным
\cdf{eql} заданному. Если такого тега в теле нет, поиск осуществляется в
лексически доступном теле другой конструкции \cdf{tagbody}.
Использоваться \cdf{go} с тегом, которого нет, является ошибкой.

Форма \cdf{go} никогда не возвращает значение.

В целях стиля, рекомендуется дважды подумать, прежде чем использовать
\cdf{go}. Большинство функций \cdf{go} могут быть заменены циклами, вложенными
условными формами или \cdf{return-from}. Если использование \cdf{go} неизбежно,
рекомендуется управляющую структуру реализованную с помощью \cdf{go} <<упаковать>>
в определении макроса. 
\end{defspec}

\section{Возврат и обработка нескольких значений}
\indexterm{multiple values}

Обычно результатом вызова Lisp'овой функции является один Lisp'овый объект.
Однако, иногда для функции удобно вычислить несколько объектов и вернуть их.
Common Lisp представляет механизм для прямой обработки нескольких значений.
Механизм удобнее и эффективнее, чем исполнение трюков со списками или
глобальными переменными.

\subsection{Конструкции для обработки нескольких значений}

Обычно не используется несколько значений. Для возврата и обработки нескольких
значений требуются специальные формы.
Если вызывающий функцию код не требует нескольких значений, однако вызываемая
функция их несколько, то для кода берётся только первое значение. Все оставшиеся
значения игнорируются.
Если вызываемая функция возвращает ноль значений, вызывающий код в качестве
значения получает {\false}.

\cdf{values} --- это главный примитив для возврата нескольких значений. Он
принимает любое количество аргументов и возвращает столько же значений. Если
последняя форма тела функции является \cdf{values} с тремя аргументами, то вызов
такой функции вернёт три значения.
Другие специальные формы также возвращают несколько значения, но они могут быть
описаны в терминах \cdf{values}. Некоторые встроенные Common Lisp функции, такая
как \cdf{floor}, возвращают несколько значений.

Специальные формы обрабатывающие несколько значений представлены ниже:
\begin{lisp}
multiple-value-list \\
multiple-value-call \\
multiple-value-prog1 \\
multiple-value-bind \\
multiple-value-setq
\end{lisp}
Они задают форму для вычисления и указывают куда поместить значения возвращаемые
данной формой.

\begin{defun}[Функция]
values &rest args

Все аргументы в таком же порядке возвращаются, как значения.
Например,
\begin{lisp}
(defun polar (x y) \\
~~(values (sqrt (+ (* x x) (* y y))) (atan y x))) \\
 \\
(multiple-value-bind (r theta) (polar 3.0 4.0) \\
~~(vector r theta)) \\
~~~\EV\ \#(5.0 0.9272952)
\end{lisp}

Выражение \cd{(values)} возвращает ноль значений. Это стандартная идиома для
возврата из функции нулевого количества значений.

Иногда необходимо указать явно, что функция будет возвращать только одно
значение. Например, функция
\begin{lisp}
(defun foo (x y) \\
~~(floor (+ x y) y))
\end{lisp}
будет возвращать два значения, потому что \cdf{floor} возвращает два
значения. Может быть второе значение не имеет смысла в данном контексте, или
есть причины не тратить время на вычисления второго значения. Функция
\cdf{values} является стандартной идиомой для указания того, что будет
возвращено только одно значение, как показано в следующем примере.
\begin{lisp}
(defun foo (x y) \\
~~(values (floor (+ x y) y)))
\end{lisp}
Это работает, потому что \cdf{values} возвращает только \emph{одно} значения для
каждой формы аргумента. Если форма аргумента возвращает несколько значений, то
используется только первое, а остальные игнорируются.

В Common Lisp'е для вызывающего кода нет возможности различить, было ли
возвращено просто одно значение или было возвращено только одно значение в с
помощью \cdf{values}. Например значения, возвращённые выражением \cd{(+~1~2)} и 
\cd{(values (+~1~2))}, идентичны во всех отношениях: они оба просто равны \cd{3}.
\end{defun}

\begin{defun}[Константа]
multiple-values-limit

Значение \cdf{multiple-values-limit} является положительным целым числом,
которое невключительно указывает наибольшее возможное количество возвращаемых
значений. Это значение зависит от реализации, но не может быть менее 20.
(Разработчики рекомендуется делать это ограничение как можно большим без потери
в производительности.)
Смотрите \cdf{lambda-parameters-limit} и \cdf{call-arguments-limit}.
\end{defun}

\begin{defun}[Функция]
values-list list

Все элементы списка \emph{list} будут возвращены как несколько значений.
Например:
\begin{lisp}
(values-list (list a b c)) \EQ\ (values a b c)
\end{lisp}
Можно обозначить так,
\begin{lisp}
(values-list \emph{list}) \EQ\ (apply \#'values \emph{list})
\end{lisp}
но \cdf{values-list} может быть более ясным или эффективным.
\end{defun}

\begin{defmac}
multiple-value-list form

\cdf{multiple-value-list} вычисляет форму \emph{form} и возвращает список из
того количества значений, которое было возвращено формой.
Например,
\begin{lisp}
(multiple-value-list (floor -3 4)) \EV\ (-1 1)
\end{lisp}
Таким образом, \cdf{multiple-value-list} и \cdf{values-list} являются
антиподами. FIXME
\end{defmac}

\begin{defspec}
multiple-value-call function {\,form}*

\cdf{multiple-value-call} сначала вычисляет \emph{function} для получения
функции и затем вычисляет все формы \emph{forms}. Все значения форм \emph{forms}
собираются вместе (все, а не только первые) и передаются как аргументы
функции. Результат \cdf{multiple-value-call} является тем, что вернула функция.
Например:
\begin{lisp}
(+ (floor 5 3) (floor 19 4)) \\
~~~\EQ\ (+ 1 4) \EV\ 5 \\
(multiple-value-call \#'+ (floor 5 3) (floor 19 4)) \\
~~~\EQ\ (+ 1 2 4 3) \EV\ 10 \\
(multiple-value-list \emph{form}) \EQ\ (multiple-value-call \#'list \emph{form})
\end{lisp}
\end{defspec}

\begin{defspec}
multiple-value-prog1 form {\,form}*

\cdf{multiple-value-prog1} вычисляет первую форму \emph{form} и сохраняет все
значения, возвращённые данной формой. Затем она слева направо вычисляет оставшиеся
\emph{формы}, игнорируя их значения. Значения, возвращённые первой формой,
становятся результатом всей формы \cdf{multiple-value-prog1}. Смотрите
\cdf{prog1}, которая всегда возвращает одно значение.
\end{defspec}

\begin{defmac}
multiple-value-bind ({var}*) values-form
                    {declaration}* {\,form}*

Вычисляется \emph{values-form} и каждое значение результата связывается
соответственно с переменными указанными в первом параметре.
Если переменных больше, чем возвращаемых значений, для оставшихся переменных
используется значение {\false}. Если значений больше, чем переменных, лишние
значения игнорируются. Переменные связываются со значениями только на время
выполнения форм тела, которое является неявным \cdf{progn}.
Например:
\begin{lisp}
(multiple-value-bind (x) (floor 5 3) (list x)) \EV\ (1) \\
(multiple-value-bind (x y) (floor 5 3) (list x y)) \EV\ (1 2) \\
(multiple-value-bind (x y z) (floor 5 3) (list x y z)) \\
~~~\EV\ (1 2 {\false})
\end{lisp}
\end{defmac}

\begin{defmac}
multiple-value-setq variables form

Аргумент \emph{variables} должен быть списком переменных. Вычисляется форма
\emph{form} и переменным \emph{присваиваются} (не связываются) значения,
возвращённые этой формой. Если переменных больше, чем значений, оставшимся
переменным присваивается {\false}. Если значений больше, чем переменных, лишние
значения игнорируются.

\cdf{multiple-value-setq} всегда возвращает одно значение, которое является
первым из возвращённых значений формы \emph{form}, или {\false}, если форма
\emph{form} вернула ноль значений.
\end{defmac}

\begin{defmac}*
nth-value n form

Формы аргументов \emph{n} и \emph{form} вычисляются.
Значение \emph{n} должно быть неотрицательным целым, и форма \emph{form} должна
возвращать любое количество значение.
Целое число \emph{n} используется, как индекс (начиная с нуля), для доступа к
списку значений.
Форма возвращает элемент в позиции \emph{n} из списка результатов вычисления
формы \emph{form}. Если позиции не существует, возвращается \cdf{nil}.

В качестве примера, \cdf{mod} мог бы быть определён так:
\begin{lisp}
(defun mod (number divisor) \\*
~~(nth-value 1 (floor number divisor)))
\end{lisp}
\end{defmac}

\subsection{Правила управления возвратом нескольких значений}

Часто случается так, что значение специальной формы или макроса определено как
значение одного из подвыражений.
Например, значение \cdf{cond} является значением последнего подвыражения в
исполняемой ветке.

В большинстве таких случаев, если подвыражение возвращает несколько значений,
тогда и оригинальная форма возвращает все эти значения. Эта \emph{передача
  значений}, конечно, не будет иметь место, если не была выполнена специальная
форма для обработки возврата нескольких значений.

Неявно несколько значений может быть возвращены из специальных форм в
соответствие со следующими правилами:
\goodbreak
\begin{flushdesc}
\item[\emph{Вычисление и применение}]\leavevmode
\begin{itemize}

\item
\cdf{eval} возвращает несколько значений, если переданная ему форму при
вычислении вернула несколько значений.

\item
\cdf{apply}, \cdf{funcall} и \cdf{multiple-value-call} передают обратно
несколько значений из применяемой или вызываемой функции.
\end{itemize}

\item[\emph{В контексте неявного \cdf{progn}}]\leavevmode
\begin{itemize}

\item
Специальная форма \cdf{progn}
передаёт обратно несколько значений полученных при вычислении последней
подформы. В другие ситуациях, называемых <<неявный \cdf{progn}>>, в которых
вычисляется несколько форм и результаты всех, кроме последней формы,
игнорируются, также передаётся обратно несколько значений от формы.
Такие ситуации включают тело лямбда-выражения, в частности в \cdf{defun},
\cdf{defmacro} и \cdf{deftype}.
Также включаются тела конструкций
\cdf{eval-when},
\cdf{progv}, \cdf{let},
\cdf{let*}, \cdf{when}, \cdf{unless},
\cdf{block},
\cdf{multiple-value-bind} и \cdf{catch}.
И также включаются подвыражения условных конструкций 
\cdf{case}, \cdf{typecase},
\cdf{ecase}, \cdf{etypecase}, \cdf{ccase} и \cdf{ctypecase}.
\end{itemize}
\end{flushdesc}

\begin{flushdesc}
\item[\emph{Условные конструкции}]\leavevmode
\begin{itemize}

\item
\cdf{if} передаёт обратно несколько значений из любой выбранной подформы (ветки
\emph{then} или \emph{else}).

\item
\cdf{and} и \cdf{or} передают обратно несколько значений из последней подформы,
но ни из какой другой непоследней подформы.

\item
\cdf{cond} передаёт обратно несколько значений из последней подформы неявного
\cdf{progn} выделенного подвыражения.
Однако, если выделенное подвыражение является <<синглтоном>>, будет возвращено
только одно значение (не-{\false} значение предиката).
Это верно, даже если выражение <<синглтон>> является последним подвыражением
\cdf{cond}.
Нельзя рассматривать последнее подвыражение \cdf{(x)}, как будто оно \cd{(t
  x)}. Последнее подвыражение передаёт обратно несколько значений из формы
\cd{x}.
\end{itemize}

\item[\emph{Возврат из блока}]\leavevmode
\begin{itemize}

\item
При нормальном завершении конструкция \cdf{block} передаёт обратно несколько значений из её последней
подформы. Если для завершения использовалась \cdf{return-from} (или
\cdf{return}), тогда \cdf{return-from} передаёт обратно несколько значений из
своей подформы, как значения всей конструкции  \cdf{block}. Другие конструкции,
создающие неявные блоки, такие как 
\cdf{do}, \cdf{dolist}, \cdf{dotimes}, \cdf{prog} и
\cdf{prog*}, также передают обратно несколько значений, заданных с помощью 
\cdf{return-from} (или \cdf{return}).

\item
\cdf{do} передаёт обратно несколько значений из последней формы подвыражения
выхода, точно также, как если бы подвыражение выхода было подвыражение
\cdf{cond}.
Подобным образом действуют \cdf{dolist} и \cdf{dotimes}. Они возвращают
несколько значений из формы \emph{resultform}, если она была выполнена.
Эти ситуации также являются примерами явного использования \cdf{return-from}.
\end{itemize}

\item[\emph{Выход из ловушки исключения}]\leavevmode
\begin{itemize}

\item
Конструкция \cdf{catch} возвращает несколько значений, если результат формы в
\cdf{throw}, которая осуществляет выход из этого catch, возвращает несколько
значений.
\end{itemize}

\item[\emph{Остальные ситуации}]\leavevmode
\begin{itemize}

\item
\cdf{multiple-value-prog1} передаёт обратно несколько значений из его первой
подформы. Однако, \cdf{prog1} всегда возвращает одно значение.

\item
\cdf{unwind-protect} возвращает несколько значений, если форма, которую он
защищает, вернула несколько значений.

\item
\cdf{the} возвращает несколько значений, если в нем содержащаяся форма
возвращает несколько значений.
\end{itemize}
\end{flushdesc}

\cdf{prog1},
\cdf{prog2}, \cdf{setq} и \cdf{multiple-value-setq} --- это формы, которые
\emph{никогда} не передают обратно несколько значений.
Стандартный метод для явного указания возврата одного значения из формы \cd{x} 
является запись \cd{(values x)}.

Наиболее важное правило насчёт нескольких значений:
\textbf{Не важно сколько значений возвращает форма, если форма является
  аргументом в вызове функции, то будет использовано только одно значение
  (первое).} 

Например, если вы записали \cd{(cons (floor x))}, тогда \cdf{cons} всегда
получить \emph{только} один аргумент (что, конечно, является ошибкой), хотя и
\cdf{floor} возвращает два значения. Для того, чтобы поместить оба значения
\cdf{floor} в \cdf{cons}, можно записать что-то вроде этого:
\cd{(multiple-value-call \#'cons (floor x))}.
В обычном вызове функции, каждый форма аргумента предоставляется только
\emph{один} аргумент. Если такая форма возвращает ноль значение, в качестве
аргумента используется {\false}. А если возвращает более одного значения, все
они, кроме первого, игнорируются.
Также и условные конструкции, например \cdf{if}, которые проверяют значения
формы, используют только одно первое значение, остальные игнорируются.
Такие конструкции будут использовать {\false} если форма вернула ноль значений.
\section{Динамические нелокальные выходы}
\label{CATCH-THROW-SECTION}
\indexterm{non-local exit}
\indexterm{dynamic exit}
\indexterm{catch}
\indexterm{throw}

Common Lisp предоставляет функциональность для выхода из сложного процесса в
нелокальном, динамических ограниченном стиле. Для этих целей есть два вида
специальных форм, называемых \emph{catch} и \emph{throw}.
Форма catch выполняет некоторые подформы так, что если форма throw выполняется в
ходе этих вычислений, в данной точке вычисления прерываются и форма catch
немедленно возвращает значение указанное в throw. В отличие от \cdf{block} и
\cdf{return} (раздел~\ref{BLOCK-RETURN-SECTION}), которые позволяют выйти из
тела \cdf{block} из любой точки лексически, находящейся внутри тела, catch/throw
механизм работает, даже если форма throw по тексту не находится внутри тела
формы catch.
Throw может использовать только в течение (продолжительности времени) выполнения
тела catch. Можно провести аналогию между ними, как между динамически
связываемыми переменными и лексически связываемыми.

\begin{defspec}
catch tag {\,form}*

Специальная форма \cdf{catch} служит мишенью для передачи управления с помощью
\cdf{throw}.
Первой выполняется форма \emph{tag} для создания объекта для имени catch.
Он может быть любым Lisp'овым объектом.
Затем устанавливается ловушка с тегом, в качестве которого используется этот
объект.
Формы \emph{form} выполняются как неявный \cdf{progn},
и возвращается результат последней формы.
Однако если во время вычислений будет выполнена форма \cdf{throw} с тегом,
который равен \cdf{eq} тегу catch, то управление немедленно прерывается и
возвращается результат указанный в форме throw.
Ловушка, устанавливаемая с помощью \cdf{catch}, упраздняется перед тем, как
будет возвращён результат.

Тег используется для соотнесения throws и catches.
\cd{(catch 'foo \emph{form})} будет перехватывать \cd{(throw 'foo \emph{form})},
но не будет \cd{(throw 'bar \emph{form})}. Если \cdf{throw} выполнен без
соответствующего \cdf{catch}, готового его обработать, то это является ошибкой.

Теги catch сравниваются с использованием \cdf{eq}, а не \cdf{eql}.
Таким образом числа и строковые символы не могут использоваться в качестве
тегов.
\end{defspec}

\indexterm{unwind protection}
\indexterm{cleanup handler}
\begin{defspec}
unwind-protect protected-form {cleanup-form}*

Иногда необходимо выполнить форму и быть уверенным, что некоторые побочные
эффекты выполняются после её завершения.
Например:
\begin{lisp}
(progn (start-motor) \\*
~~~~~~~(drill-hole) \\*
~~~~~~~(stop-motor))
\end{lisp}
Функциональность нелокальных выходов в Common Lisp создаёт ситуацию, в которой
однако данный код может не сработать правильно.
Если \cd{drill-hole} может бросить исключение в ловушку, находящуюся выше
данного \cdf{progn}, то \cd{(stop-motor)} никогда не выполниться.
Более удобный пример с открытием/закрытием файлов:
\begin{lisp}
(prog2 (open-a-file) \\*
~~~~~~~(process-file) \\*
~~~~~~~(close-the-file))
\end{lisp}
где закрытие файла может не произойти, по причине ошибки в функции
\cd{process-file}.

Для того, чтобы вышеприведённый код работал корректно, можно переписать его с
использованием \cdf{unwind-protect}:
\begin{lisp}
;; Stop the motor no matter what (even if it failed to start). \\*
\\*
(unwind-protect \\*
~~(progn (start-motor) \\*
~~~~~~~~~(drill-hole)) \\*
~~(stop-motor))
\end{lisp}
Если \cdf{drill-hole} бросит исключение, которое попытается выйти из блока
\cdf{unwind-protect}, то \cd{(stop-motor)} будет обязательно выполнена.

Этот пример допускает то, что вызов \cdf{stop-motor} корректен, даже если
мотор ещё не был запущен. Помните, что ошибка или прерывание может осуществить
выход перед тем, как будет проведена инициализация. Любой код по восстановлению
состояния должен правильно работать вне зависимости от того, где произошла
ошибка.
Например, следующий код неправильный:
\begin{lisp}
(unwind-protect \\*
~~(progn (incf *access-count*) \\*
~~~~~~~~~(perform-access)) \\*
~~(decf *access-count*))
\end{lisp}
Если выход случиться перед тем, как выполниться операция \cdf{incf}, то
выполнение \cdf{decf} приведён к некорректному значению в \cd{*access-count*}.
Правильно будет записать этот код так:
\begin{lisp}
(let ((old-count *access-count*)) \\
~~(unwind-protect \\
~~~~(progn (incf *access-count*) \\
~~~~~~~~~~~(perform-access)) \\
~~~~(setq *access-count* old-count)))
\end{lisp}

Как правило, \cdf{unwind-protect} гарантирует выполнение форм
\emph{cleanup-form} перед выходом, как в случае нормального выхода, так и в
случае генерации исключения.
(Если, однако, выход случился в ходе выполнения форм \emph{cleanup-form},
никакого специального действия не предпринимается. Формы \emph{cleanup-form} не
защищаются. Для этого необходимо использовать дополнительные
конструкции \cdf{unwind-protect}.)
\cdf{unwind-protect} возвращает результат выполнения защищённой формы
\emph{protected-form} и игнорирует все результаты выполнения форм чистки
\emph{cleanup-form}.

Следует подчеркнуть, что \cdf{unwind-protect} защищает против \emph{всех}
попыток выйти из защищённой формы, включая не только <<динамический выход>> с
помощью \cdf{throw}, но и также <<лексический выход>> с помощью \cdf{go} и
\cdf{return-from}. Рассмотрим следующую ситуацию:
\begin{lisp}
(tagbody \\
~~(let ((x 3)) \\
~~~~(unwind-protect \\
~~~~~~(if (numberp x) (go out)) \\
~~~~~~(print x))) \\
~out \\
~~...)
\end{lisp}
Когда выполнится \cdf{go}, то сначала выполнится \cdf{print}, а затем перенос управления
на тег \cd{out} будет завершён.
\end{defspec}

\begin{defspec}
throw tag result

Специальная форма \cdf{throw} переносит управление к соответствующей конструкции
\cdf{catch}.
Сначала выполняется \emph{tag} для вычисления объекта, называемого тег
throw. Затем вычисляется форма результата \emph{result}, и этот результат
сохраняется (если форма \emph{result} возвращает несколько значений, то
\emph{все} значения сохраняются).
Управление выходит и самого последнего установленного catch, тег которого
совпадает с тегом throw. Сохранённые результаты возвращаются, как значения
конструкции \cdf{catch}.
Теги catch и throw совпадают, только если они равны \cdf{eq}.

В процессе, связывания динамических переменных упраздняются до точки catch, и
выполнятся все формы очистки в конструкциях unwind-protect.
Форма \emph{result} вычисляется перед началом процесса раскручивания, и её
значение возвращается из блока catch.

Если внешний блок catch с совпадающим тегом не найден, раскрутка стека не
происходит и сигнализируется ошибка.
Когда ошибка сигнализируется, ловушки и динамические связывания переменных
являются теми, которые были в точке throw.
\end{defspec}

\fi      % Flow of control, environment handling
%Part{Macro, Root = "CLM.MSS"}
%Chapter of Common Lisp Manual.  Copyright 1984, 1988, 1989 Guy L. Steele Jr.

\clearpage\def\pagestatus{ULTIMATE}

\chapter{Macros}
\label{MACROS}


The Common Lisp macro facility allows the user to define arbitrary
functions that convert certain Lisp forms into different forms before
evaluating or compiling them.  This is done at the expression level,
not at the character-string level as in most other languages.  Macros
are important in the writing of good code: they make it possible to
write code that is clear and elegant at the user level but that is
converted to a more complex or more efficient internal form for
execution.

When \cdf{eval} is given a list whose {\it car} is a symbol, it looks
for local definitions of that symbol (by \cdf{flet}, \cdf{labels},
and \cdf{macrolet}); if that fails, it looks for a global definition.
If the definition is a macro definition, then the original
list is said to be a {\it macro call}.  Associated with the definition
will be a function of two arguments, called the {\it expansion function}.
This function is called with the entire macro call as its first argument
(the second argument is a lexical environment);
it must return some new Lisp form, called the {\it expansion} of the
macro call.  (Actually, a more general mechanism is involved;
see \cdf{macroexpand}.)
This expansion is then evaluated in place of the original
form.

When a function is being compiled, any macros it contains are expanded
at compilation time.  This means that a macro definition must be seen by the
compiler before the first use of the macro.

More generally, an implementation of Common Lisp has great latitude in deciding
exactly when to expand macro calls within a program.  For example,
it is acceptable for the \cdf{defun} special form to expand all macro
calls within its body at the time the \cdf{defun} form is executed
and record the fully expanded body as the body of the function
being defined.
(An implementation might even choose always to compile functions defined
by \cdf{defun}, even when operating in an ``interpretive'' mode.)

Macros should be written so as to depend as little as possible
on the execution environment to produce a correct expansion.  To ensure
consistent behavior, it is best to ensure that all macro definitions are
available, whether to the interpreter or compiler, before any code
containing calls to those macros is introduced.

In Common Lisp, macros are not functions.
In particular, macros cannot be used as
functional arguments to such functions as \cdf{apply}, \cdf{funcall},
or \cdf{map}; in such situations, the list representing the ``original macro
call'' does not exist, and cannot exist, because in some sense the arguments
have already been evaluated.


\section{Macro Definition}

The function \cdf{macro-function} determines whether a given symbol
is the name of a macro.  The \cdf{defmacro} construct provides
a convenient way to define new macros.

\begin{obsolete}
\begin{defun}[Function]
macro-function symbol

The argument must be a symbol.  If the symbol has a global function definition
that is a macro definition, then the expansion function
(a function of two arguments, the macro-call form and an environment)
is returned.
If the symbol has no global function definition, or has a definition
as an ordinary function or as a special form but not as a macro, then
{\false} is returned.  The function \cdf{macroexpand}
is the best way to invoke the expansion function.

It is possible for {\it both} \cdf{macro-function} and \cdf{special-form-p}
to be true of a symbol.  This is possible because an implementation is
permitted to implement any macro also as a special form for speed.
On the other hand, the macro definition must be available
for use by programs that understand only the standard special forms
listed in table~\ref{SPECIAL-FORM-TABLE}.

\cdf{macro-function} cannot be used to determine whether a symbol names
a locally defined macro established by \cdf{macrolet};
\cdf{macro-function} can
examine only global definitions.

\cdf{setf} may be used with \cdf{macro-function} to install
a macro as a symbol's global function definition:
\begin{lisp}
(setf (macro-function {\it symbol}) {\it fn})
\end{lisp}
The value installed must be a function that accepts two arguments,
an entire macro call and an environment, and computes the expansion for that call.
Performing this operation causes the symbol to have {\it only} that
macro definition as its global function definition; any previous
definition, whether as a macro or as a function, is lost.
It is an error to attempt to redefine the name of a special
form.
\end{defun}
\end{obsolete}

\begin{newer}
X3J13 voted in March 1988 \issue{MACRO-FUNCTION-ENVIRONMENT}
to add an optional environment argument to \cdf{macro-function}.

\begin{defun}[Function]
macro-function symbol &optional env

The first argument must be a symbol.  If the symbol has a function definition
that is a macro definition, whether a local one established in the
environment {\it env} by \cdf{macrolet} or a global one established as
if by \cdf{defmacro},
then the expansion function
(a function of two arguments, the macro-call form and an environment)
is returned.
If the symbol has no function definition, or has a definition
as an ordinary function or as a special form but not as a macro, then
{\false} is returned.  The function \cdf{macroexpand} or \cd{macroexpand-1}
is the best way to invoke the expansion function.

It is possible for {\it both} \cdf{macro-function} and \cdf{special-form-p}
to be true of a symbol.  This is possible because an implementation is
permitted to implement any macro also as a special form for speed.
On the other hand, the macro definition must be available
for use by programs that understand only the standard special forms
listed in table~\ref{SPECIAL-FORM-TABLE}.

\cdf{setf} may be used with \cdf{macro-function} to install
a macro as a symbol's global function definition:
\begin{lisp}
(setf (macro-function {\it symbol}) {\it fn})
\end{lisp}
The value installed must be a function that accepts two arguments,
an entire macro call and an environment, and computes the expansion for that call.
Performing this operation causes the symbol to have {\it only} that
macro definition as its global function definition; any previous
definition, whether as a macro or as a function, is lost.
One cannot use \cdf{setf} to establish a local macro definition;
it is an error to supply a second argument to \cdf{macro-function}
when using it with \cdf{setf}.
It is an error to attempt to redefine the name of a special form.

See also \cdf{compiler-macro-function}.
\end{defun}
\end{newer}

\begin{defmac}
defmacro name lambda-list <{declaration}* | doc-string> {\,form}*

\cdf{defmacro} is a macro-defining macro that
arranges to decompose the macro-call form in an elegant and useful way.
\cdf{defmacro} has essentially the same syntax as \cdf{defun}: {\it name} is the
symbol whose macro definition we are creating, {\it lambda-list} is similar in
form to a lambda-list, and
the {\it form\/}s constitute the body of the expander function.
The \cdf{defmacro} construct arranges to install this expander function,
as the global macro definition of {\it name}.

\begin{obsolete}
The expander function
is effectively defined in the {\it global} environment;
lexically scoped entities established
outside the \cdf{defmacro} form that would ordinarily be lexically apparent
are not visible within the body of the expansion function.
\end{obsolete}

\begin{newer}
X3J13 voted in March 1989 \issue{DEFINING-MACROS-NON-TOP-LEVEL}
to clarify that, while defining forms normally appear at top level,
it is meaningful to place them in non-top-level contexts.
Furthermore, \cdf{defmacro} should define the expander function
within the enclosing lexical environment, not within the global
environment.
\end{newer}

\begin{newer}
X3J13 voted in March 1988 \issue{FLET-IMPLICIT-BLOCK}
to specify that the body of the expander function defined
by \cdf{defmacro} is implicitly enclosed in a \cdf{block} construct
whose name is the same as the {\it name} of the defined macro.
Therefore \cdf{return-from} may be used to exit from the function.
\end{newer}

The {\it name} is returned
as the value of the \cdf{defmacro} form.

If we view the 
macro call as a list containing a function name and some argument forms,
in effect the expander function and the list of (unevaluated) argument
forms is given to \cdf{apply}.
The parameter specifiers are processed as for any lambda-expression,
using the macro-call argument forms as the arguments.
Then the body forms are evaluated
as an implicit \cdf{progn}, and the value of the last form
is returned as the expansion of the macro call.

If the optional documentation string {\it doc-string} is present (if not
followed by a declaration, it may be
present only if at least one {\it form} is also specified, as it is
otherwise taken to be a {\it form}), then it is attached to the {\it name}
as a documentation string of type \cdf{function}; see \cdf{documentation}.

\begin{obsolete}
Like the lambda-list in a \cdf{defun}, a \cdf{defmacro} {\it lambda-list} may contain
the lambda-list keywords \cd{\&optional}, \cd{\&rest}, \cd{\&key},
\cd{\&allow-other-keys}, and \cd{\&aux}.
For \cd{\&optional} and \cd{\&key} parameters, initialization forms and
supplied-p parameters may be specified, just as for \cdf{defun}.
Three additional markers
are allowed in \cdf{defmacro} variable lists only.
\end{obsolete}
\begin{new}
These three markers are now allowed in other constructs as well.
\end{new}
\begin{indentdesc}{6pc}
\item[\cd{\&body}]
This is identical in function to \cd{\&rest}, but it informs certain
output-formatting and editing functions that the remainder of the form is
treated as a body and should be indented accordingly.
(Only one of \cd{\&body} or \cd{\&rest} may be used.)

\item[\cd{\&whole}]
This is followed by a single variable that is bound to the
entire macro-call form; this is the value that the macro definition function
receives as its single argument.
\cd{\&whole} and the following variable should appear first in the lambda-list,
before any other parameter or lambda-list keyword.

\item[\cd{\&environment}]
This is followed by a single variable that is bound
to an environment representing the lexical environment in which the
macro call is to be interpreted.   This environment may not be the
complete lexical environment; it should be used only with
the function \cdf{macroexpand} for the sake of any local
macro definitions that the \cdf{macrolet} construct may have
established within that lexical environment.  This is useful primarily
in the rare cases where a macro definition must explicitly expand any macros
in a subform of the macro call before computing its own expansion.
\end{indentdesc}
See \cdf{lambda-list-keywords}.

\begin{new}%CORR
{\it Notice of correction.}
In the first edition, the symbol \cd{\&environment} at the
left margin above was inadvertently omitted.
\end{new}

\begin{newer}
X3J13 voted in March 1989 \issue{MACRO-ENVIRONMENT-EXTENT}
to specify that macro environment objects received with the \cd{\&environment}
argument of a macro function
have only dynamic extent.  The consequences are undefined if such objects are
referred to outside the dynamic extent of that particular
invocation of the macro function.
This allows implementations to use somewhat more efficient techniques
for representing environment objects. 
\end{newer}

\begin{newer}
X3J13 voted in March 1989 \issue{DEFMACRO-LAMBDA-LIST} to clarify the permitted
uses of \cd{\&body}, \cd{\&whole}, and \cd{\&environment}:
\begin{itemize}
\item \cd{\&body} may appear at any level of a \cdf{defmacro} lambda-list.
\item \cd{\&whole} may appear at any level of a \cdf{defmacro} lambda-list.
At inner levels a \cd{\&whole} variable is bound to that part of the argument
that matches the sub-lambda-list in which \cd{\&whole} appears.  No matter where
\cd{\&whole} is used, other parameters or lambda-list keywords may follow it.
\item \cd{\&environment} may occur only at the outermost level of a \cdf{defmacro}
lambda-list, and it may occur at most once, but it may occur anywhere within
that lambda-list, even before an occurrence of \cd{\&whole}.
\end{itemize}
\end{newer}

\cdf{defmacro}, unlike any other Common Lisp construct that has a lambda-list
as part of its syntax, provides an additional facility known as
{\it destructuring}.
\begin{newer}
See \cdf{destructuring-bind}, which provides the destructuring facility separately.
\end{newer}
Anywhere in the lambda-list where a parameter
name may appear, and where ordinary lambda-list syntax (as described
in section~\ref{LAMBDA-EXPRESSIONS-SECTION}) does not
otherwise allow a list, a lambda-list may appear in place
of the parameter name.  When this is done, then the argument form
that would match the parameter is treated as a (possibly dotted) list,
to be used as an argument forms list for satisfying the
parameters in the embedded lambda-list.
As an example, one could write the macro definition
for \cdf{dolist} in this manner:
\begin{lisp}
(defmacro dolist ((var listform \cd{\&optional} resultform) \\
~~~~~~~~~~~~~~~~~~\&rest body) \\
~~...)
\end{lisp}
More examples of embedded lambda-lists in \cdf{defmacro} are shown below.

Another destructuring rule is that \cdf{defmacro} allows any lambda-list
(whether top-level or embedded) to be dotted, ending
in a parameter name.  This situation is treated exactly as if the
parameter name that ends the list had appeared preceded by \cd{\&rest}.
For example, the definition skeleton for \cdf{dolist} shown above could
instead have been written
\begin{lisp}
(defmacro dolist ((var listform \&optional resultform) \\
~~~~~~~~~~~~~~~~~~. body) \\
~~...)
\end{lisp}

If the compiler encounters a \cdf{defmacro},
the new macro is added to the compilation
environment, and a compiled form of the expansion function is also added
to the output file so that the new macro will be operative at run time.
If this is not the desired effect, the \cdf{defmacro} form can be wrapped
in an \cdf{eval-when} construct.

It is permissible to use \cdf{defmacro} to redefine a macro
(for example, to install
a corrected version of an incorrect definition), or to redefine
a function as a macro.
It is an error to attempt to redefine the name of a special
form (see table~\ref{SPECIAL-FORM-TABLE}) as a macro.
See \cdf{macrolet}, which establishes macro
definitions over a restricted lexical scope.

\begin{newer}
See also \cdf{define-compiler-macro}.
\end{newer}

Suppose, for the sake of example, that it were desirable
to implement a conditional construct analogous to the
Fortran arithmetic IF statement.  (This of course requires a certain
stretching of the imagination and suspension of disbelief.)
The construct should accept four forms: a {\it test-value},
a {\it neg-form}, a {\it zero-form}, and a {\it pos-form}.
One of the last three forms is chosen to be executed according
to whether the value of the {\it test-form} is positive, negative,
or zero.
Using \cdf{defmacro}, a definition for such a construct
might look like this:
\begin{lisp}
(defmacro arithmetic-if (test neg-form zero-form pos-form) \\
~~(let ((var (gensym))) \\
~~~~{\Xbq}(let ((,var ,test)) \\
~~~~~~~(cond ((< ,var 0) ,neg-form) \\
~~~~~~~~~~~~~((= ,var 0) ,zero-form) \\
~~~~~~~~~~~~~(t ,pos-form)))))
\end{lisp}
Note the use of the backquote facility in this definition
(see section~\ref{MACRO-CHARACTERS-SECTION}).
Also note the use of \cdf{gensym} to generate a new variable name.
This is necessary to avoid conflict with any variables that might
be referred to in {\it neg-form}, {\it zero-form}, or {\it pos-form}.

If the form is executed by the interpreter, it will cause the
function definition of the symbol \cdf{arithmetic-if}
to be a macro associated with which is
a two-argument expansion function roughly equivalent to
\begin{lisp}
(lambda (calling-form environment) \\
~~(declare (ignore environment)) \\
~~(let ((var (gensym))) \\
~~~~(list 'let \\
~~~~~~~~~~(list (list 'var (cadr calling-form))) \\
~~~~~~~~~~(list 'cond \\
~~~~~~~~~~~~~~~~(list (list '< var '0) (caddr calling-form)) \\
~~~~~~~~~~~~~~~~(list (list '= var '0) (cadddr calling-form)) \\
~~~~~~~~~~~~~~~~(list 't (fifth calling-form))))))
\end{lisp}
The lambda-expression is produced by the \cdf{defmacro} declaration.
The calls to \cdf{list} are the (hypothetical) result of the backquote (\cd{{\Xbq}})
macro character and its associated commas.
The precise macro expansion function may depend on the implementation,
for example providing some degree of explicit error checking on the number
of argument forms in the macro call.

Now, if \cdf{eval} encounters
\begin{lisp}
(arithmetic-if (- x 4.0) \\
~~~~~~~~~~~~~~~(- x) \\
~~~~~~~~~~~~~~~(error "Strange zero") \\
~~~~~~~~~~~~~~~x)
\end{lisp}
this will be expanded into something like
\begin{lisp}
(let ((g407 (- x 4.0))) \\
~~(cond ((< g407 0) (- x)) \\
~~~~~~~~((= g407 0) (error "Strange zero")) \\
~~~~~~~~(t x)))
\end{lisp}
and \cdf{eval} tries again on this new form.
(It should be clear now that the backquote facility
is very useful in writing macros, since the form to be returned is
normally a complex list structure, typically consisting of a
mostly constant template with a few evaluated forms here and there.
The backquote template provides a ``picture'' of the resulting
code, with places to be filled in indicated by preceding commas.)

To expand on this example, stretching credibility to its limit,
we might allow the {\it pos-form}
and {\it zero-form} to be omitted, allowing their values to default to {\nil},
in much the same way that the {\it else} form of a Common Lisp \cdf{if} construct
may be omitted:
\begin{lisp}
(defmacro arithmetic-if (test neg-form \\*
~~~~~~~~~~~~~~~~~~~~~~~~~\cd{\&optional} zero-form pos-form) \\*
~~(let ((var (gensym))) \\*
~~~~{\Xbq}(let ((,var ,test)) \\*
~~~~~~~(cond ((< ,var 0) ,neg-form) \\*
~~~~~~~~~~~~~((= ,var 0) ,zero-form) \\*
~~~~~~~~~~~~~(t ,pos-form)))))
\end{lisp}
Then one could write
\begin{lisp}
(arithmetic-if (- x 4.0) (print x))
\end{lisp}
which would be expanded into something like
\begin{lisp}
(let ((g408 (- x 4.0))) \\*
~~(cond ((< g408 0) (print x)) \\*
~~~~~~~~((= g408 0) nil) \\*
~~~~~~~~(t nil)))
\end{lisp}
The resulting code is correct but rather silly-looking.
One might rewrite the macro definition to produce better code
when {\it pos-form} and possibly {\it zero-form} are omitted,
or one might simply rely on the Common Lisp implementation to provide
a compiler smart enough to improve the code itself.

Destructuring is a very powerful facility that allows
the \cdf{defmacro} lambda-list to express the structure of
a complicated macro-call syntax.  If no lambda-list keywords
appear, then the \cdf{defmacro} lambda-list is simply a list,
nested to some extent, containing parameter names at the leaves.
The macro-call form must have the same list structure.
For example, consider this macro definition:
\begin{lisp}
(defmacro halibut ((mouth eye1 eye2) \\*
~~~~~~~~~~~~~~~~~~~((fin1 length1) (fin2 length2)) \\*
~~~~~~~~~~~~~~~~~~~tail) \\*
~~...)
\end{lisp}
Now consider this macro call:
\begin{lisp}
(halibut (m (car eyes) (cdr eyes)) \\*
~~~~~~~~~((f1 (count-scales f1)) (f2 (count-scales f2))) \\*
~~~~~~~~~my-favorite-tail)
\end{lisp}
This would cause the expansion function to receive the following
values for its parameters:
\begin{flushleft}
\cf
\begin{tabular}{@{}ll@{}}
{\rm Parameter}&{\rm Value} \\
\hlinesp
mouth&m \\
eye1&(car eyes) \\
eye2&(cdr eyes) \\
fin1&f1 \\
length1&(count-scales f1) \\
fin2&f2 \\
length2&(count-scales f2) \\
tail&my-favorite-tail \\
\hline
\end{tabular}
\end{flushleft}
The following macro call would be in error because there would be no
argument form to match the parameter \cd{length1}:
\begin{lisp}
(halibut (m (car eyes) (cdr eyes)) \\
~~~~~~~~~((f1) (f2 (count-scales f2))) \\
~~~~~~~~~my-favorite-tail)
\end{lisp}
The following macro call would be in error because a symbol appears
in the call where the structure of the lambda-list requires a list.
\begin{lisp}
(halibut my-favorite-head \\
~~~~~~~~~((f1 (count-scales f1)) (f2 (count-scales f2))) \\
~~~~~~~~~my-favorite-tail)
\end{lisp}
The fact that the value of the variable \cdf{my-favorite-head}
might happen to be a list is irrelevant here.  It is the macro call
itself whose structure must match that of the \cdf{defmacro} lambda-list.

The use of lambda-list keywords adds even greater flexibility.
For example, suppose it is convenient within the expansion
function for \cdf{halibut} to be able to refer to the list
whose components are called \cdf{mouth}, \cd{eye1}, and \cd{eye2} as \cdf{head}.
One may write this:
\begin{lisp}
(defmacro halibut ((\cd{\&whole} head mouth eye1 eye2) \\
~~~~~~~~~~~~~~~~~~~((fin1 length1) (fin2 length2)) \\
~~~~~~~~~~~~~~~~~~~tail)
\end{lisp}
Now consider the same valid macro call as before:
\begin{lisp}
(halibut (m (car eyes) (cdr eyes)) \\
~~~~~~~~~((f1 (count-scales f1)) (f2 (count-scales f2))) \\
~~~~~~~~~my-favorite-tail)
\end{lisp}
This would cause the expansion function to receive the same
values for its parameters and also a value for the parameter \cdf{head}:
\begin{flushleft}
\cf
\begin{tabular}{@{}ll@{}}
{\rm Parameter}&{\rm Value} \\
\hlinesp
head&(m (car eyes) (cdr eyes)) \\
\hline
\end{tabular}
\end{flushleft}

The stipulation that
an embedded lambda-list is permitted only
where ordinary lambda-list syntax would permit a parameter name
but not a list is made to prevent ambiguity.  For example,
one may not write
\begin{lisp}
(defmacro loser (x \cd{\&optional} (a b \cd{\&rest} c) \cd{\&rest} z) \\
~~...)
\end{lisp}
because ordinary lambda-list syntax does permit a list following \cd{\&optional};
the list \cd{(a b \cd{\&rest} c)} would be interpreted as describing an
optional parameter named \cdf{a} whose default value is that of the
form \cdf{b}, with a supplied-p parameter named \cd{\&rest} (not legal),
and an extraneous symbol \cdf{c} in the list (also not legal).  An almost
correct way to express this is
\begin{lisp}
(defmacro loser (x \cd{\&optional} ((a b \cd{\&rest} c)) \cd{\&rest} z) \\
~~...)
\end{lisp}
The extra set of parentheses removes the ambiguity.  However, the
definition is now incorrect because a macro call such as \cd{(loser (car pool))}
would not provide any argument form for the lambda-list \cd{(a b \cd{\&rest} c)},
and so the default value against which to match the lambda-list would be
{\nil} because no explicit default value was specified.  This is in error
because {\nil} is an empty list; it does not have forms to satisfy the
parameters \cdf{a} and \cdf{b}.  The fully correct definition would be either
\begin{lisp}
(defmacro loser (x \cd{\&optional} ((a b \cd{\&rest} c) '(nil nil)) \cd{\&rest} z) \\
~~...)
\end{lisp}
or
\begin{lisp}
(defmacro loser (x \cd{\&optional} ((\cd{\&optional} a b \cd{\&rest} c)) \cd{\&rest} z) \\
~~...)
\end{lisp}
These differ slightly: the first requires that if the macro call
specifies \cdf{a} explicitly then it must also specify \cdf{b} explicitly,
whereas the second does not have this requirement.  For example,
\begin{lisp}
(loser (car pool) ((+ x 1)))
\end{lisp}
would be a valid call for the second definition but not for the first.
\end{defmac}

\section{Macro Expansion}

The \cdf{macroexpand} function is the conventional means for
expanding a macro call.  A hook is provided for a user function
to gain control during the expansion process.

\begin{defun}[Function]
macroexpand form &optional env \\
macroexpand-1 form &optional env

If {\it form} is a macro call, then \cd{macroexpand-1} will expand the macro
call {\it once} and return two values: the expansion and \cdf{t}.
If {\it form} is not a macro call, then the two values {\it form} and {\nil} are
returned.

A {\it form} is considered to be a macro call only if it is a cons whose
{\it car} is a symbol that names a macro.  The environment {\it env} is similar
to that used within the evaluator (see \cdf{evalhook});
it defaults to a null environment.
Any local macro definitions established within {\it env} by
\cdf{macrolet} will be considered.  If only {\it form} is given as an
argument, then the environment is effectively null,
and only global macro definitions
(as established by \cdf{defmacro}) will be considered.

Macro expansion is carried out as follows.  Once \cd{macroexpand-1} has
determined that a symbol names a macro, it obtains the expansion
function for that macro.  The value of the variable
\cd{*macroexpand-hook*} is then called as a function of three arguments:
the expansion function, the {\it form}, and the environment {\it env}.
The value returned from
this call is taken to be the expansion of the macro call.
The initial value of \cd{*macroexpand-hook*} is \cdf{funcall},
and the net effect is to invoke the expansion function, giving
it {\it form} and {\it env} as its two arguments.

\begin{newer}
X3J13 voted in June 1988 \issue{FUNCTION-TYPE} to specify
that the value of \cd{*macroexpand-hook*} is first coerced to a
function before being called as the expansion interface hook.
Therefore its value may be a symbol, a lambda-expression, or any
object of type \cdf{function}.
\end{newer}

\begin{newer}
X3J13 voted in March 1989 \issue{MACRO-ENVIRONMENT-EXTENT}
to specify that macro environment objects received
by a \cd{*macroexpand-hook*} function
have only dynamic extent.  The consequences are undefined if such objects are
referred to outside the dynamic extent of that particular invocation of the hook
function.  This allows implementations to use somewhat more efficient techniques
for representing environment objects. 
\end{newer}

\begin{obsolete}
(The purpose of
\cd{*macroexpand-hook*} is to facilitate various techniques
for improving interpretation speed by caching macro expansions.)
\end{obsolete}

\begin{newer}
X3J13 voted in June 1989 \issue{MACRO-CACHING} to clarify that, while
\cd{*macroexpand-hook*} may be useful for debugging purposes, despite
the original design intent there is
currently no correct portable way to use it for caching macro expansions.
\begin{itemize}
\item
 Caching by displacement (performing a side effect on the
 macro-call form) won't work because the same (\cdf{eq}) macro-call
 form may appear in distinct lexical contexts.  In addition, the macro-call
 form may be a read-only constant (see \cdf{quote} and also
 section~\ref{COMPILER-SECTION}).
\item
 Caching by table lookup won't work because such a table would have to
 be keyed by both the macro-call form and the environment,
 but X3J13 voted in March 1989 \issue{MACRO-ENVIRONMENT-EXTENT}
 to permit macro environments to have only dynamic extent.
\item
 Caching by storing macro-call forms and expansions within the
 environment object itself would work, but there are no portable
 primitives that would allow users to do this.
\end{itemize}
X3J13 also noted that, although there seems to be no correct portable way to use
\cd{*macroexpand-hook*} to cache macro expansions, there is no
requirement that an implementation call the macro expansion
function more than once for a given form and lexical environment.
\end{newer}

\begin{new}
X3J13 voted in March 1989
\issue{SYMBOL-MACROLET-SEMANTICS}
to specify that \cd{macroexpand-1} will also expand symbol macros
defined by \cdf{symbol-macrolet}; therefore a {\it form} may also be
a macro call if it is a symbol.  The vote did not address the interaction
of this feature with the \cd{*macroexpand-hook*} function.  An obvious
implementation choice is that the hook function is indeed called
and given a special expansion function that, when applied to the
{\it form} (a symbol) and {\it env}, will produce the expansion,
just as for an ordinary macro; but this is only my suggestion.
\end{new}

The evaluator expands macro calls as if through the use of \cd{macroexpand-1};
the point is that \cdf{eval} also uses \cd{*macroexpand-hook*}.

\cdf{macroexpand} is similar to \cd{macroexpand-1},
but repeatedly expands {\it form} until it is no longer a macro call.
(In effect, \cdf{macroexpand} simply calls \cd{macroexpand-1} repeatedly
until the second value returned is {\nil}.)
A second value of \cdf{t} or {\nil} is returned as for \cd{macroexpand-1},
indicating whether the original {\it form} was a macro call.
\end{defun}

\begin{defun}[Variable]
*macroexpand-hook*

The value of \cd{*macroexpand-hook*} is used as the expansion
interface hook by \cd{macroexpand-1}.
\end{defun}

\begin{newer}
\section{Destructuring}

X3J13 voted in March 1989 \issue{DESTRUCTURING-BIND}
to make the destructuring feature of \cdf{defmacro}
available as a separate facility.

\begin{defmac}
destructuring-bind lambda-list expression {declaration}* {\,form}*

   This macro binds the variables specified in {\it lambda-list} to the corresponding
   values in the tree structure resulting from evaluating the {\it expression},
   then executes the {\it form\/}s as an implicit \cdf{progn}.

A \cdf{destructuring-bind} {\it lambda-list} may contain
the lambda-list keywords \cd{\&optional}, \cd{\&rest}, \cd{\&key},
\cd{\&allow-other-keys}, and \cd{\&aux}; \cd{\&body} and \cd{\&whole}
may also be used as they are in \cdf{defmacro}, but \cd{\&environment} may
{\it not} be used.  Nested and dotted lambda-lists are also permitted
as for \cdf{defmacro}.
The idea is that a \cdf{destructuring-bind} {\it lambda-list}
has the same format as inner levels of a \cdf{defmacro} lambda-list.

   If the result of evaluating the {\it expression} does not match the 
   destructuring pattern, an error should be signaled.
\end{defmac}
\end{newer}


\begin{newer}
\section{Compiler Macros}

X3J13 voted in June 1989 \issue{DEFINE-COMPILER-MACRO}
to add a facility for defining {\it compiler macros} that
take effect only when compiling code, not when interpreting it.

The purpose of this facility is to permit selective source-code
transformations only when the compiler is processing the code.
When the compiler is about to compile a non-atomic form, it first calls
\cd{compiler-macroexpand-1} repeatedly until there is no more expansion
(there might not be any to begin with).  Then it continues its
remaining processing, which may include calling \cd{macroexpand-1} and so on.

The compiler is required to expand compiler macros.  It is unspecified
whether the interpreter does so.  The intention is that only the
compiler will do so, but the range of possible ``compiled-only''
implementation strategies precludes any firm specification.


\begin{defmac}
define-compiler-macro name lambda-list
                      {declaration | doc-string}* {\,form}*

  This is just like \cdf{defmacro} except the definition is not stored in the
  symbol function cell of {\it name} and is not seen by \cd{macroexpand-1}.
  It is, however, seen by \cd{compiler-macroexpand-1}.  As with \cdf{defmacro}, the
  {\it lambda-list} may include \cd{\&environment} and \cd{\&whole}
  and may include destructuring.  The definition is
  global.  (There is no provision for defining local compiler
  macros in the way that \cdf{macrolet} defines local macros.)

  A top-level call to \cdf{define-compiler-macro} in a file being compiled by
  \cdf{compile-file} has an effect on the compilation environment similar to
  that of a call to \cdf{defmacro}, except it is noticed as a
  compiler macro (see section~\ref{COMPILER-SECTION}).

Note that compiler macro definitions do not appear in information returned by
\cdf{function-information}; they are global, and their interaction
with other lexical and global definitions can be reconstructed by
\cdf{compiler-macro-function}.  It is up to code-walking programs to decide
whether to invoke compiler macro expansion.


\begin{newer}
X3J13 voted in March 1988 \issue{FLET-IMPLICIT-BLOCK}
to specify that the body of the expander function defined
by \cdf{defmacro} is implicitly enclosed in a \cdf{block} construct
whose name is the same as the {\it name} of the defined macro;
presumably this applies also to \cdf{define-compiler-macro}.
Therefore \cdf{return-from} may be used to exit from the function.
\end{newer}

\end{defmac}

\begin{defun}[Function]
compiler-macro-function name &optional env

  The {\it name} must be a symbol.
  If it has been defined as a compiler macro, then
  \cdf{compiler-macro-function} returns the macro expansion
  function; otherwise it returns \cdf{nil}.  The
  lexical environment {\it env} may override any global definition for {\it name}
  by defining a local function or local macro (such as by \cdf{flet}, \cdf{labels}, or
  \cdf{macrolet}) in which case \cdf{nil} is returned.

  \cdf{setf} may be used with \cdf{compiler-macro-function} to install a function as
  the expansion function for the compiler macro {\it name}, in the same manner as for
  \cdf{macro-function}.  Storing the value \cdf{nil} removes any existing
  compiler macro definition.  As with \cdf{macro-function}, a non-\cdf{nil} stored value
  must be a function of two arguments, the entire macro call and 
  the environment.  The second argument to \cdf{compiler-macro-function} must
  be omitted when it is used with \cdf{setf}.
\end{defun}

\begin{defun}[Function]
compiler-macroexpand form &optional env \\
compiler-macroexpand-1 form &optional env

  These are just like \cdf{macroexpand} and \cd{macroexpand-1}
  except that the expander function is obtained as if by a call to
  \cdf{compiler-macro-function} on the {\it car} of the {\it form} rather than by a call to
  \cdf{macro-function}.
  Note that \cdf{compiler-macroexpand} performs repeated expansion
  but \cd{compiler-macroexpand-1} performs at most one expansion.
  Two values are returned, the expansion (or the original {\it form})
  and a value that is true if any expansion occurred and \cdf{nil} otherwise.

  There are three cases where no expansion happens:
  \begin{itemize}
    \item There is no compiler macro definition for the {\it car} of {\it form}.
    \item There is such a definition but there is also a \cdf{notinline}
        declaration, either globally or in the lexical environment {\it env}.
    \item A global compiler macro definition is shadowed by a local
        function or macro definition (such as by \cdf{flet}, \cdf{labels}, or
        \cdf{macrolet}).
  \end{itemize}
  Note that if there is no expansion, the original {\it form} is returned as
  the first value, and \cdf{nil} as the second value.
  
  Any macro expansion performed by the function \cdf{compiler-macroexpand}
  or by the function \cd{compiler-macroexpand-1} is carried out
  by calling the function that is the value of \cd{*macroexpand-hook*}.

A compiler macro may decline to provide any expansion merely
by returning the original form. This is useful when using the facility
to put ``compiler optimizers'' on various function names.  For example,
here is a compiler macro that ``optimizes'' (one would hope)
the zero-argument and one-argument cases of
a function called \cdf{plus}:
\begin{lisp}
(define-compiler-macro plus (\&whole form \&rest args) \\*
~~(case (length args) \\*
~~~~(0 0) \\*
~~~~(1 (car args)) \\*
~~~~(t form)))
\end{lisp}
\end{defun}
\end{newer}


\begin{newer}
\section{Environments}

X3J13 voted in June 1989 \issue{SYNTACTIC-ENVIRONMENT-ACCESS} to add some facilities for obtaining information
from environment objects of the kind received as arguments
by macro expansion functions, \cd{*macroexpand-hook*} functions,
and \cd{*evalhook*} functions.
There is a minimal set of accessors (\cdf{variable-information},
\cdf{function-information}, and \cdf{declaration-information}) and a constructor
(\cdf{augment-environment}) for environments.

All of the standard declaration specifiers, with the exception of \cdf{special},
can be defined fairly easily using \cdf{define-declaration}.  It also
seems to be able to handle most extended declarations.

The function \cdf{parse-macro} is provided so that
users don't have to write their
  own code to destructure macro arguments.
This function is not entirely necessary since X3J13 voted
in March 1989 \issue{DESTRUCTURING-BIND}
to add \cdf{destructuring-bind} to the language.
  However, \cdf{parse-macro} is worth having anyway, since any program-analyzing
  program is going to need to define it, and the implementation isn't completely
  trivial even with \cdf{destructuring-bind} to build upon.

  The function \cdf{enclose} allows expander functions to be defined in a non-null
  lexical environment, as required by the vote of X3J13 in
  March 1989 \issue{DEFINING-MACROS-NON-TOP-LEVEL}.  It
  also provides a mechanism by which a program processing
  the body of an \cd{(eval-when (:compile-toplevel)~...)} form
  can execute it in the enclosing environment (see issue
  \issue{EVAL-WHEN-NON-TOP-LEVEL}).

In all of these functions the argument named {\it env} is an environment
object.  (It is not required that implementations
 provide a distinguished representation for such objects.)  Optional {\it env}
 arguments default to \cdf{nil}, which represents the local null lexical environment
 (containing only global definitions and proclamations that are present in the
 run-time environment).  All of these functions should signal an error of type
 \cdf{type-error} if the value of an environment argument is not a syntactic
 environment object.

 The accessor functions \cdf{variable-information}, \cdf{function-information}, and
 \cdf{declaration-information} retrieve information about
 declarations that are in
 effect in the environment.  Since implementations are permitted to ignore
 declarations (except for \cdf{special} declarations and \cd{optimize safety}
 declarations if they ever compile unsafe code), these accessors are required
 only to return information about declarations that were explicitly added to
 the environment using \cdf{augment-environment}.  They might also return
 information about declarations recognized and added to the environment by the
 interpreter or the compiler, but that is at the discretion of the
 implementor.  Implementations are also permitted to canonicalize
 declarations, so the information returned by the accessors might not be
 identical to the information that was passed to \cdf{augment-environment}.

\begin{defun}[Function]
variable-information variable &optional env

  This function returns information about the interpretation of the symbol
  {\it variable} when it appears as a variable within the lexical environment {\it env}.
  Three values are returned.

  The first value indicates the type of definition or binding for {\it variable}
  in {\it env\/}:
\begin{indentdesc}{7pc}
\item[\cdf{nil}]
There is no apparent definition or binding for {\it variable}.

\item[\cd{:special}]
The {\it variable} refers to a special variable, either declared or proclaimed. 

\item[\cd{:lexical}]
The {\it variable} refers to a lexical variable.

\item[\cd{:symbol-macro}]
The {\it variable} refers to a \cdf{symbol-macrolet} binding.

\item[\cd{:constant}]
Either the {\it variable} refers to a named constant defined by
\cdf{defconstant} or the {\it variable} is a keyword symbol.
\end{indentdesc}

  The second value indicates whether there is a local binding of the name.  If
  the name is locally bound, the second value is true; otherwise, the second value
  is \cdf{nil}.

  The third value is an a-list containing information about declarations
  that apply to the apparent binding of the {\it variable}.  The keys in the a-list
  are symbols that name declaration specifiers, and the format of the
  corresponding value in the {\it cdr} of each pair depends on the particular 
  declaration name involved.  The standard declaration names
  that might appear as keys in this a-list are:
\begin{indentdesc}{7pc}
\item[\cdf{dynamic-extent}]
A non-\cdf{nil} value indicates that the {\it variable} has been
                declared \cdf{dynamic-extent}. If the value is \cdf{nil}, the pair
                might be omitted.

\item[\cdf{ignore}]
A non-\cdf{nil} value indicates that the {\it variable} has been declared
                \cdf{ignore}. If the value is \cdf{nil}, the pair might be omitted.

\item[\cdf{type}]
The value is a type specifier associated with the {\it variable} by a \cdf{type}
                declaration or an abbreviated declaration such as
                \cd{(fixnum {\it variable})}.
                If no explicit association exists, either by \cdf{proclaim} or
                \cdf{declare}, then the type specifier is \cdf{t}.  It is permissible for
                implementations to use a type specifier that is equivalent
                to or a supertype of the one appearing in the original
                declaration.  If the value is \cdf{t}, the pair might be
                omitted.
\end{indentdesc}
  If an implementation supports additional declaration specifiers that
  apply to variable bindings, those declaration names might also
  appear in the a-list.  However, the corresponding key must not
  be a symbol that is external in any package defined in the standard
  or that is otherwise accessible in the \cdf{common-lisp-user} package.

  The a-list might contain multiple entries for a given key.
  The consequences of destructively modifying the list
  structure of this a-list or its elements (except for values that 
  appear in the a-list as a result of \cdf{define-declaration}) are undefined.

  Note that the global binding might differ from the
  local one and can be retrieved by calling \cdf{variable-information}
  with a null lexical environment.
\end{defun}

\begin{defun}[Function]
function-information function &optional env

  This function returns information about the interpretation of the function-name
  {\it function} when it appears in a functional position within lexical 
  environment {\it env}.  Three values are returned.

  The first value indicates the type of definition or binding of the function-name
  which is apparent in {\it env}:
\begin{indentdesc}{7pc}
\item[\cdf{nil}] There is no apparent definition for {\it function}.

\item[\cd{:function}] The {\it function} refers to a function.

\item[\cd{:macro}] The {\it function} refers to a macro.

\item[\cd{:special-form}] The {\it function} refers to a special form.
\end{indentdesc}
  Some function-names can refer to both a global macro and a global special
  form.  In such a case the macro takes precedence and \cd{:macro} is returned as
  the first value.

  The second value specifies whether the definition is local or global.  If
  local, the second value is true; it is \cdf{nil} when the definition is
  global.

  The third value is an a-list containing information about declarations
  that apply to the apparent binding of the function.  The keys in the a-list
  are symbols that name declaration specifiers, and the format of the
  corresponding values in the {\it cdr} of each pair depends on the particular 
  declaration name involved.  The standard declaration names
  that might appear as keys in this a-list are:
\begin{indentdesc}{7pc}
\item[\cdf{dynamic-extent}]
A non-\cdf{nil} value indicates that the function has been
                declared \cdf{dynamic-extent}.  If the value is \cdf{nil}, the pair
                might be omitted.

\item[\cdf{inline}]
The value is one of the symbols \cdf{inline}, \cdf{notinline}, or \cdf{nil} to indicate
                whether the function-name has been declared \cdf{inline},
                declared \cdf{notinline}, or neither, respectively.
                If the value is \cdf{nil}, the pair might be omitted.

\item[\cdf{ftype}]
The value is the type specifier associated with the function-name in the
                environment, or the symbol \cdf{function} if there is no functional
                type declaration or proclamation associated with the function-name.
                This value might not include all the apparent \cdf{ftype}
                declarations for the function-name.  It is permissible for
                implementations to use a type specifier that is equivalent
                to or a supertype of the one that appeared in the original
                declaration.  If the value is \cdf{function}, the pair might be
                omitted. 
\end{indentdesc}
  If an implementation supports additional declaration specifiers that
  apply to function bindings, those declaration names might also
  appear in the a-list.  However, the corresponding key must not be
  a symbol that is external in any package defined in the standard or
  that is otherwise accessible in the \cdf{common-lisp-user} package.

  The a-list might contain multiple entries for a given key.
  In this case the value associated with the first entry has
  precedence.  The consequences of destructively modifying the list
  structure of this a-list or its elements (except for values
  that appear in the a-list as a result of \cdf{define-declaration}) are 
  undefined.

  Note that the global binding might differ from the local
  one and can be retrieved by calling \cdf{function-information} with a null
  lexical environment.
\end{defun}

\begin{defun}[Function]
declaration-information decl-name &optional env

  This function returns information about declarations named by the
  symbol {\it decl-name} that are in force in the environment {\it env}.
  Only declarations that do not apply to function or variable bindings
  can be accessed with this function.  The format of the information
  that is returned depends on the {\it decl-name} involved.

  It is required that this function recognize \cdf{optimize} and \cdf{declaration} as
  {\it decl-name\/}s.  The values returned for these two cases are as follows:
\begin{indentdesc}{7pc}
\item[\cdf{optimize}]
A single value is returned,
a list whose entries are of the form \cd{({\it quality} {\it value})}, where
                {\it quality} is one of the standard optimization qualities
                (\cdf{speed}, \cdf{safety}, \cdf{compilation-speed}, \cdf{space}, \cdf{debug})
                or some implementation-specific optimization quality, and
                {\it value} is an integer in the range 0 to 3 (inclusive).
                The returned list
                always contains an entry for each of the standard qualities and
                for each of the implementation-specific qualities.  In the
                absence of any previous declarations, the associated values are
                implementation-dependent.  The list might contain multiple
                entries for a quality, in which case the first such entry
                specifies the current value.
                The consequences of destructively modifying this list or
		its elements are undefined.
                

\item[\cdf{declaration}]
A single value is returned,
a list of the declaration names that have been proclaimed as
                valid through the use of the \cdf{declaration} proclamation.
                The consequences of destructively modifying this list or
		its elements are undefined.
\end{indentdesc}
  If an implementation is extended to recognize additional
  declaration specifiers in \cdf{declare} or \cdf{proclaim}, it is required that
  either the \cdf{declaration-information} function should recognize those
  declarations also or the implementation should provide a similar accessor that is
  specialized for that declaration specifier.  If \cdf{declaration-information}
  is used to return the information, the corresponding {\it decl-name} must not
  be a symbol that is external in any package defined in the standard or
  that is otherwise accessible in the \cdf{common-lisp-user} package.
\end{defun}

\begin{defun}[Function]
augment-environment env &key :variable :symbol-macro :function :macro :declare

  This function returns a new environment containing the information present in
  {\it env} augmented with the information provided by the keyword arguments.  It is
  intended to be used by program analyzers that perform a code walk.

  The arguments are supplied as follows.
\begin{flushdesc}
\item[\cd{:variable}]
     The argument is a list of symbols that will be visible as bound variables in
                the new environment.  Whether each binding is to be interpreted
                as special or lexical depends on \cdf{special} declarations recorded
                in the environment or provided in the \cd{:declare} argument.

\item[\cd{:symbol-macro}]
 The argument is a list of symbol macro definitions, each of the form
                \cd{({\it name} {\it definition})}; that is, the argument is
                in the same format as the
                {\it cadr} of a \cdf{symbol-macrolet} special form.  The new environment
                will have local symbol-macro bindings of each symbol to the
                corresponding expansion, so that \cdf{macroexpand} will be able to
                expand them properly.  A type declaration in the \cd{:declare}
                argument that refers to a name in this list implicitly
                modifies the definition associated with the name.  The effect
                is to wrap a \cdf{the} form mentioning the type around the
                definition.

\item[\cd{:function}]
     The argument is a list of function-names that will be visible as local
                function bindings in the new environment.

\item[\cd{:macro}]
        The argument is a list of local macro definitions, 
        each of the form \cd{({\it name} {\it definition})}.
        Note that the argument is {\it not}
                in the same format as the
                {\it cadr} of a \cdf{macrolet} special form.
                Each {\it definition} must be a function of two
                arguments (a form and an environment).  The new environment
                will have local macro bindings of each name to the
                corresponding expander function, which will be returned by
                \cdf{macro-function} and used by \cdf{macroexpand}.

\item[\cd{:declare}]
      The argument is a list of declaration specifiers.
      Information about these declarations can
                be retrieved from the resulting environment using
                \cdf{variable-information}, \cdf{function-information}, and
                \cdf{declaration-information}.
\end{flushdesc}
  The consequences of subsequently
  destructively modifying the list
  structure of any of the arguments to this function are undefined.

  An error is signaled if any of the symbols naming a symbol macro in the
  \cd{:symbol-macro} argument is also included in the \cd{:variable} argument.
  An error is
  signaled if any symbol naming a symbol macro in the \cd{:symbol-macro} argument is
  also included in a \cdf{special} declaration specifier in the \cd{:declare} argument.
  An error is
  signaled if any symbol naming a macro in the \cd{:macro} argument is also included
  in the \cd{:function} argument.
  The condition type of each of these errors is \cdf{program-error}.

  The extent of the returned environment is the same as the extent of the
  argument environment {\it env}.  The result might share structure with {\it env}
  but {\it env} is not modified.

  While an environment argument received by an \cd{*evalhook*}
  function is permitted to be used as the
  environment argument to \cdf{augment-environment}, the consequences are undefined if an
  attempt is made to use the result of \cdf{augment-environment} as the environment
  argument for \cdf{evalhook}.  The environment
  returned by \cdf{augment-environment} can be used only for syntactic analysis, that is,
  as an argument to
  the functions defined in this section and functions such as \cdf{macroexpand}.
\end{defun}

\begin{defmac}
define-declaration decl-name lambda-list {\,form}*

  This macro defines a handler for the named declaration.  It is the mechanism by which
  \cdf{augment-environment} is extended to support additional declaration
  specifiers.  The function defined by this macro will be called with two
  arguments, a declaration specifier whose {\it car} is {\it decl-name}
  and the {\it env} argument to
  \cdf{augment-environment}.  This function must return two values.  The
  first value must be one of the following keywords:
\begin{indentdesc}{7pc}
\item[\cd{:variable}]     The declaration applies to variable bindings.
\item[\cd{:function}]     The declaration applies to function bindings.
\item[\cd{:declare}]      The declaration does not apply to bindings.
\end{indentdesc}
If the first value is \cd{:variable} or \cd{:function}
then the second value must be a list, the elements of which are lists of the
  form \cd{({\it binding-name} {\it key} {\it value})}.  If the corresponding information
  function (either \cdf{variable-information} or \cdf{function-information}) is applied to
  the {\it binding-name} and the augmented environment, the a-list returned
  by the information function as its third value will contain the {\it value}
  under the specified {\it key}.

  If the first value is \cd{:declare}, the second value must be a cons
  of the form \cd{({\it key}~.~{\it value})}.  The function
  \cdf{declaration-information} will return {\it value} when applied to the
  {\it key} and the augmented environment.

  \cdf{define-declaration} causes {\it decl-name} to be proclaimed to be a
  declaration; it is as if its expansion included a call \cd{(proclaim
  '(declaration {\it decl-name}))}.  As is the case with standard
  declaration specifiers, the evaluator and compiler are permitted,
  but not required, to add information about declaration specifiers
  defined with \cdf{define-declaration} to the macro expansion and \cd{*evalhook*}
  environments.

  The consequences are undefined if {\it decl-name} is a symbol that can
  appear as the {\it car} of any standard declaration specifier.

  The consequences are also undefined if the return value from a 
  declaration handler defined with \cdf{define-declaration} includes a {\it key} name
  that is used by the corresponding accessor to return information about
  any standard declaration specifier.  (For example, if
  the first return value from the handler is \cd{:variable}, the second return
  value may not use the symbols \cdf{dynamic-extent}, \cdf{ignore}, or \cdf{type}
  as {\it key} names.)

  The \cdf{define-declaration} macro does not have any special compile-time
  side effects (see section~\ref{COMPILER-SECTION}).
\end{defmac}

\begin{defun}[Function]
parse-macro name lambda-list body &optional env

  This function is used to process a macro definition in the same way
  as \cdf{defmacro} and \cdf{macrolet}.  It returns a lambda-expression that accepts
  two arguments, a form and an environment.  The {\it name}, {\it lambda-list},
  and {\it body} arguments correspond to the parts of a \cdf{defmacro} or \cdf{macrolet}
  definition.

  The {\it lambda-list} argument may include \cd{\&environment} and \cd{\&whole}
  and may include destructuring.
  The {\it name}
  argument is used to enclose the {\it body} in an implicit \cdf{block} and might also
  be used for implementation-dependent purposes (such as including the name of
  the macro in error messages if the form does not match the {\it lambda-list}).
\end{defun}

\begin{defun}[Function]
enclose lambda-expression &optional env

  This function returns an object of type \cdf{function} that is equivalent to what
  would be obtained by evaluating \cd{{\Xbq}(function ,{\it lambda-expression})}
  in a syntactic
  environment {\it env}.  The {\it lambda-expression} is permitted to reference only the
  parts of the environment argument {\it env} that are relevant only to syntactic
  processing, specifically declarations and the definitions of macros and
  symbol macros.  The consequences are undefined if the {\it lambda-expression}
  contains any references to variable or function bindings that are 
  lexically visible in {\it env}, any \cdf{go} to a tag that is lexically visible in 
  {\it env}, or any \cdf{return-from} mentioning a block name that is lexically 
  visible in {\it env}.
\end{defun}  
\end{newer}
       % Macros
%Part{Declar, Root = "CLM.MSS"}
%Chapter of Common Lisp Manual.  Copyright 1984, 1988, 1989 Guy L. Steele Jr.

\clearpage\def\pagestatus{ULTIMATE}

\chapter{Declarations}
\label{DECLAR}

Declarations allow you to specify extra information about your program
to the Lisp system.  With one exception,
declarations are completely optional
and correct declarations do not affect the meaning
of a correct program.  The exception is that
\cd{special} declarations {\it do} affect the interpretation of variable
bindings and references and so {\it must} be specified where appropriate.
All other declarations are of an advisory nature, and may be used
by the Lisp system to aid the programmer by performing extra error checking
or producing more efficient compiled code.  Declarations are also
a good way to add documentation to a program.

Note that it is considered an error for a program to violate a
declaration (such as a \cd{type} declaration), but an implementation is
not required to detect such errors (though such detection, where
feasible, is to be encouraged).

\section{Declaration Syntax}
\label{DECLARE-SYNTAX-SECTION}

The \cd{declare} construct is used for embedding declarations within
executable code.  Global declarations and declarations that are computed
by a program are established by the \cd{proclaim} construct.

\begin{newer}
X3J13 voted in June 1989 \issue{PROCLAIM-ETC-IN-COMPILE-FILE}
to introduce the new macro \cd{declaim}, which is guaranteed
to be recognized appropriately by the compiler and is often more convenient
than \cd{proclaim} for establishing global declarations.
\end{newer}

\begin{defspec}
declare {decl-spec}*

A \cd{declare} form is known as a {\it declaration}.
Declarations may occur only at the beginning of the bodies of
certain special forms;
that is, a declaration may occur only as a statement
of such a special form, and all statements preceding it (if any) must
also be \cd{declare} forms (or possibly documentation strings, in some cases).
Declarations may occur in lambda-expressions and in the forms listed here.
\begin{lisp}
\hskip 12pc\=\kill
\cd{define-setf-method}\>\cd{labels} \\*
\cd{defmacro}\>\cd{let} \\*
\cd{defsetf}\>\cd{let*} \\*
\cd{deftype}\>\cd{locally} \\
\cd{defun}\>\cd{macrolet} \\
\cd{do}\>\cd{multiple-value-bind} \\
\cd{do*}\>\cd{prog} \\
\cd{do-all-symbols}\>\cd{prog*} \\
\cd{do-external-symbols}\>\cd{with-input-from-string} \\
\cd{do-symbols}\>\cd{with-open-file} \\
\cd{dolist}\>\cd{with-open-stream} \\*
\cd{dotimes}\>\cd{with-output-to-string} \\*
\cd{flet} 
\end{lisp}
\begin{new}%CORR
{\it Notice of correction.}
In the first edition, the above list failed to mention the forms
\cd{define-setf-method}, \cd{with-input-from-string}, \cd{with-open-file},
\cd{with-open-stream}, and \cd{with-output-to-string}, even though
their individual descriptions in the first edition specified that declarations
may appear in those forms.
\end{new}

X3J13 voted in June 1989 \issue{CONDITION-RESTARTS} to add \cd{with-condition-restarts}
and also \issue{DATA-IO} to add \cd{print-unreadable-object}
and \cd{with-standard-io-syntax}.
The X3J13 vote left it unclear whether these macros
permit declarations to appear at the heads of their bodies.
I believe that was the intent,
but this is only my interpretation.

\begin{new}
X3J13 voted in June 1988
\issue{CLOS}
to adopt the Common Lisp Object System,
which includes the following additional forms in which declarations
may occur:
\begin{lisp}
\hskip 12pc\=\kill
\cd{defgeneric}\>\cd{generic-function} \\*
\cd{define-method-combination}\>\cd{generic-labels} \\*
\cd{defmethod}\>\cd{with-added-methods} \\*
\cd{generic-flet}
\end{lisp}
Furthermore X3J13 voted in January 1989
\issue{SYMBOL-MACROLET-DECLARE}
to allow declarations to
occur before the bodies of these forms:
\begin{lisp}
\hskip 12pc\=\kill
\cd{symbol-macrolet}\>\cd{with-slots} \\*
\cd{with-accessors}
\end{lisp}
There are certain aspects peculiar to \cd{symbol-macrolet}
(and therefore also to \cd{with-accessors} and \cd{with-slots},
which expand into uses of \cd{symbol-macrolet}).
An error is signaled if a name defined by \cd{symbol-macrolet}
is declared \cd{special}, and a type declaration of a name
defined by \cd{symbol-macrolet} is equivalent in effect
to wrapping a \cd{the} form mentioning that type around
the expansion of the defined symbol.
\end{new}

It is an error to attempt to evaluate a declaration.
Those special forms that permit declarations to appear
perform explicit checks for their presence.

\beforenoterule
\begin{incompatibility}
In MacLisp, \cd{declare} is a special form
that does nothing but return the symbol \cd{declare} as its
result.  The MacLisp interpreter knows nothing about declarations
but just blindly evaluates them, effectively ignoring them.
The MacLisp compiler recognizes declarations but processes
them simply by evaluating the subforms of the declaration in
the compilation context.  In Common Lisp it is
important that both the interpreter and compiler recognize
declarations (especially \cd{special} declarations) and treat them
consistently,
and so the rules about the structure and use of declarations
have been made considerably more stringent.
The odd tricks played in MacLisp by writing arbitrary forms
to be evaluated within a \cd{declare} form
are better done in both MacLisp and Common Lisp by using \cd{eval-when}.
\end{incompatibility}
\afternoterule

It is permissible for a macro call to expand into a declaration
and be recognized as such, provided that the macro call
appears where a declaration may legitimately appear.
(However, a macro call may not appear in place of a {\it decl-spec}.)

\begin{new}
X3J13 voted in March 1988
\issue{DECLARE-MACROS}
to eliminate the recognition of
a declaration resulting from the expansion of a macro call.
This feature proved to be seldom used and
awkward to implement in interpreters, compilers, and other code-analyzing programs.

Under this change, a declaration is recognized only as such if
it appears explicitly, as a list whose {\it car} is the symbol \cd{declare},
in the body of a relevant special form.  (Note, however, that it
is still possible for a macro to expand into a call to the \cd{proclaim}
function.)
\end{new}

Each {\it decl-spec} is a list whose {\it car} is a symbol
specifying the kind of declaration to be made.  Declarations may be
divided into two classes: those that concern the bindings of variables,
and those that do not.  (The \cd{special} declaration is the sole
exception: it effectively falls into both classes, as explained below.)
Those that concern variable bindings apply
only to the bindings made by the form at the head of whose body they
appear.  For example, in
\begin{lisp}
(defun foo (x) \\
~~(declare (type float x)) ... \\
~~(let ((x 'a)) ...) \\
~~...)
\end{lisp}
the \cd{type} declaration applies only to the outer binding of \cd{x},
and not to the binding made in the \cd{let}.

\beforenoterule
\begin{incompatibility}
This represents a difference from MacLisp, in which type
declarations are pervasive.
\end{incompatibility}
\afternoterule

Declarations that do not concern themselves with variable bindings are
pervasive, affecting all code in the body of the special form.
As an example of a pervasive declaration,
\begin{lisp}
(defun foo (x y) (declare (notinline floor)) ...)
\end{lisp}
advises that everywhere within the body of \cd{foo} the function
\cd{floor} should not be open-coded but called as an out-of-line subroutine.

Some special forms contain pieces of code that, properly speaking,
are not part of the body of the special form.  Examples of this
are initialization forms that provide values for bound variables,
and the result forms of iteration constructs.
In all cases such additional code is within the scope of any pervasive
declarations appearing before the body of the special form.
Non-pervasive declarations have no effect on such code, except (of course)
in those situations where the code is defined to be within the scope
of the variables affected by such non-pervasive declarations.
For example:
\begin{lisp}
(defun few (x \cd{\&optional} (y *print-circle*)) \\*
~~(declare (special *print-circle*)) \\*
~~...)
\end{lisp}
The reference to \cd{*print-circle*} in the first line of this example is special
because of the declaration in the second line.
\begin{lisp}
(defun nonsense (k x z) \\*
~~(foo z x)~~~~~~~~~~~~~~~;{\rm First call to \cd{foo}} \\*
~~(let ((j (foo k x))~~~~~;{\rm Second call to \cd{foo}} \\*
~~~~~~~~(x (* k k))) \\*
~~~~(declare (inline foo) (special x z)) \\*
~~~~(foo x j z)))~~~~~~~~~;{\rm Third call to \cd{foo}}
\end{lisp}
In this rather nonsensical example,
the \cd{inline} declaration applies to the
second and third calls to \cd{foo}, but not to the first one.
The \cd{special} declaration of \cd{x} causes the \cd{let} form
to make a special binding for \cd{x} and causes the reference to \cd{x}
in the body of the \cd{let} to be a special reference.
The reference to \cd{x} in the second call to \cd{foo} is also a special
reference.
The reference to \cd{x} in the first call to \cd{foo} is a local
reference, not a special one.  The \cd{special} declaration of \cd{z}
causes the reference to \cd{z} in the call
to \cd{foo} to be a special reference; it will not
refer to the parameter to \cd{nonsense} named \cd{z}, because that
parameter binding has not been declared to be \cd{special}.
(The \cd{special} declaration of \cd{z} does not appear in the body
of the \cd{defun}, but in an inner construct, and therefore does not
affect the binding of the parameter.)

\begin{new}
X3J13 voted in January 1989
\issue{DECLARATION-SCOPE}
to replace the rules concerning the scope of
declarations occurring at the head of a special form or lambda-expression:
\begin{itemize}
\item The scope of a declaration always includes the body forms, as well as any
``stepper'' or ``result'' forms (which are logically part of the body), of the
special form or lambda-expression.

\item If the declaration applies to a name binding, then the scope of the
declaration also includes the scope of the name binding.
\end{itemize}
Note that
the distinction between pervasive and non-pervasive
declarations is eliminated.  An important change
from the first edition is that ``initialization''
forms are specifically {\it not} included as part of the body under the first
rule; on the other hand, in many cases initialization forms may fall
within the scope of certain declarations under the second rule.
\end{new}

\begin{new}
X3J13 also voted in January 1989
\issue{DECLARE-TYPE-FREE}
to change the interpretation
of \cd{type} declarations (see section \ref{DECLARATION-SPECIFIERS-SECTION}).
\end{new}

\begin{new}
These changes affect the interpretation of some of the examples from the
first edition.
\begin{lisp}
(defun foo (x) \\*
~~(declare (type float x)) ... \\*
~~(let ((x 'a)) ...) \\*
~~...)
\end{lisp}
Under the interpretation approved by X3J13, the type
declaration applies to {\it both} bindings of \cd{x}.
More accurately, the type declaration is considered to apply to
variable references rather than bindings, and the type declaration refers
to every reference in the body of \cd{foo} to a variable named \cd{x},
no matter to what binding it may refer.
\begin{lisp}
(defun foo (x y) (declare (notinline floor)) ...)
\end{lisp}
This example of the use of \cd{notinline} stands unchanged, but the following
slight extension of it would change:
\begin{lisp}
(defun foo (x \&optional (y (floor x))) \\*
~~(declare (notinline floor)) ...)
\end{lisp}
Under first edition rules, the \cd{notinline} declaration would be
considered to apply to the call to \cd{floor} in the initialization
form for \cd{y}.  Under the interpretation approved by X3J13, the
\cd{notinline} would {\it not} apply to that particular call to \cd{floor}.
Instead the user must write something like
\begin{lisp}
(defun foo (x \&optional (y (locally (declare (notinline floor)) \\*
~~~~~~~~~~~~~~~~~~~~~~~~~~~~~~~~~~~~(floor x)))) \\*
~~(declare (notinline floor)) ...)
\end{lisp}
or perhaps
\begin{lisp}
(locally (declare (notinline floor)) \\*
~~(defun foo (x \&optional (y (floor x))) ...))
\end{lisp}
Similarly, the \cd{special} declaration in
\begin{lisp}
(defun few (x \cd{\&optional} (y *print-circle*)) \\*
~~(declare (special *print-circle*)) \\*
~~...)
\end{lisp}
is not considered to apply to the reference in the initialization form
for \cd{y} in \cd{few}.  As for the \cd{nonsense} example,
\begin{lisp}
(defun nonsense (k x z) \\*
~~(foo z x)~~~~~~~~~~~~~~~;{\rm First call to \cd{foo}} \\*
~~(let ((j (foo k x))~~~~~;{\rm Second call to \cd{foo}} \\*
~~~~~~~~(x (* k k))) \\*
~~~~(declare (inline foo) (special x z)) \\*
~~~~(foo x j z)))~~~~~~~~~;{\rm Third call to \cd{foo}}
\end{lisp}
under the interpretation approved by X3J13, the \cd{inline}
declaration is no longer considered to apply to the second
call to \cd{foo}, because it is in an initialization form, which is
no longer considered in the scope of the declaration.  Similarly,
the reference to \cd{x} in that second call to \cd{foo} is no longer
taken to be a special reference, but a local reference to the second
parameter of \cd{nonsense}.
\end{new}
\end{defspec}

\begin{obsolete}
\begin{defmac}
locally {declaration}* {\,form}*

This macro may be used to make local pervasive declarations
where desired.   It does not bind any variables and therefore cannot
be used meaningfully for declarations of variable bindings.
(Note that the \cd{special} declaration may be used with \cd{locally}
to pervasively affect references to, rather than bindings of, variables.)
For example:
\begin{lisp}
(locally (declare (inline floor) (notinline car cdr)) \\
~~~~~~~~~(declare (optimize space)) \\
~~(floor (car x) (cdr y)))
\end{lisp}
\end{defmac}
\end{obsolete}

\begin{new}
X3J13 voted in January 1989
\issue{RETURN-VALUES-UNSPECIFIED}
to specify that \cd{locally} executes the {\it form\/}s as an implicit
\cd{progn} and returns the value(s) of the last {\it form}.
\end{new}

\begin{newer}
X3J13 voted in March 1989 \issue{LOCALLY-TOP-LEVEL} to make \cd{locally}
be a special form rather than a macro.  It still has the same syntax.

\begin{defspec}
locally {declaration}* {\,form}*

This change was made to accommodate the new compilation model for top-level forms
in a file (see section~\ref{COMPILER-SECTION}).
When a \cd{locally} form appears at top level, the forms in its body are
processed as top-level forms.  This means that one may, for example, meaningfully use
\cd{locally} to wrap declarations around a \cd{defun} or \cd{defmacro} form:
\begin{lisp}
(locally \\*
~~(declare (optimize (safety 3) (space 3) (debug 3) (speed 1))) \\*
~~(defun foo (x \&optional (y (abs x)) (z (sqrt y))) \\*
~~~~(bar x y z)))
\end{lisp}
Without assurance that this works
one must write something cumbersome such as
\begin{lisp}
 \\*
(defun foo (x \&optional (y (locally \\*
~~~~~~~~~~~~~~~~~~~~~~~~~~~~~~(declare (optimize (safety 3) \\*
~~~~~~~~~~~~~~~~~~~~~~~~~~~~~~~~~~~~~~~~~~~~~~~~~(space 3) \\
~~~~~~~~~~~~~~~~~~~~~~~~~~~~~~~~~~~~~~~~~~~~~~~~~(debug 3) \\*
~~~~~~~~~~~~~~~~~~~~~~~~~~~~~~~~~~~~~~~~~~~~~~~~~(speed 1))) \\*
~~~~~~~~~~~~~~~~~~~~~~~~~~~~~~(abs x))) \\
~~~~~~~~~~~~~~~~~~~~~~~~~(z (locally \\*
~~~~~~~~~~~~~~~~~~~~~~~~~~~~~~(declare (optimize (safety 3) \\*
~~~~~~~~~~~~~~~~~~~~~~~~~~~~~~~~~~~~~~~~~~~~~~~~~(space 3) \\
~~~~~~~~~~~~~~~~~~~~~~~~~~~~~~~~~~~~~~~~~~~~~~~~~(debug 3) \\*
~~~~~~~~~~~~~~~~~~~~~~~~~~~~~~~~~~~~~~~~~~~~~~~~~(speed 1))) \\*
~~~~~~~~~~~~~~~~~~~~~~~~~~~~~~(sqrt y)))) \\
~~(locally \\*
~~~~(declare (optimize (safety 3) (space 3) (debug 3) (speed 1))) \\*
~~~~(bar x y z)))
\end{lisp}
\end{defspec}
\end{newer}

\newpage%manual

\begin{defun}[Function]
proclaim decl-spec

The function \cd{proclaim} takes a {\it decl-spec} as its
argument and puts it into effect globally.  (Such a global
declaration is called a {\it proclamation}.)
Because \cd{proclaim} is a function, its argument is always evaluated.
This allows a program to compute a declaration and then put
it into effect by calling \cd{proclaim}.

Any variable names
mentioned are assumed to refer to the dynamic values of the
variable.  For example, the proclamation
\begin{lisp}
(proclaim '(type float tolerance))
\end{lisp}
once executed,
specifies that the dynamic value of \cd{tolerance} should always
be a floating-point number.
Similarly, any function-names mentioned are assumed to refer to
the global function definition.

A proclamation constitutes a universal declaration, always in force
unless locally shadowed.  For example,
\begin{lisp}
(proclaim '(inline floor))
\end{lisp}
advises that \cd{floor} should normally be open-coded in-line by the
compiler (but in the situation
\begin{lisp}
(defun foo (x y) (declare (notinline floor)) ...)
\end{lisp}
it will be compiled out-of-line anyway in the body of \cd{foo},
because of the shadowing local declaration to that effect).

\begin{newer}
X3J13 voted in January 1989 \issue{SPECIAL-TYPE-SHADOWING}
to clarify that such shadowing does not occur in the case of type declarations.
If there is a local type declaration for a special variable and there is also a global
proclamation for that same variable, then the value of the variable within the scope
of the local declaration must be a member of the intersection of the two
declared types.
This is consistent with the treatment of nested local type declarations
on which X3J13 also voted in January 1989 \issue{DECLARE-TYPE-FREE}.
\end{newer}

As a special case (so to speak), \cd{proclaim} treats a \cd{special}
{\it decl-spec} as applying to all bindings as well as to
all references of the mentioned variables.
\begin{new}%CORR
{\it Notice of correction.}
In the first edition, this sentence referred to a ``\cd{special}
{\it declaration-form}.''  That was incorrect; \cd{proclaim} accepts
only a {\it decl-spec}, not a {\it declaration-form}.
\end{new}

For example, after
\begin{lisp}
(proclaim '(special x))
\end{lisp}
in a function definition such as
\begin{lisp}
(defun example (x) ...)
\end{lisp}
the parameter \cd{x} will be bound as a special (dynamic) variable
rather than as a lexical (static) variable.  This facility should
be used with caution.  The usual way to define a globally special
variable is with \cd{defvar} or \cd{defparameter}.
\end{defun}

\begin{newer}
X3J13 voted in June 1989 \issue{PROCLAIM-ETC-IN-COMPILE-FILE}
to clarify that the compiler is not required to treat
calls to \cd{proclaim} any differently from the way it treats
any other function call.  If a top-level call to \cd{proclaim}
is to take effect at compile time, it should be surrounded
by an appropriate \cd{eval-when} form.  Better yet,
the new macro \cd{declaim} may be used instead.

\begin{defmac}
declaim {decl-spec}*

This macro is syntactically like \cd{declare} and semantically
like \cd{proclaim}.  It is an executable form and may be used
anywhere \cd{proclaim} may be called.  However, each {\it decl-spec}
is not evaluated.

If a call to this macro appears at top level in a file
  being processed by the file compiler, the proclamations are also
  made at compile time.  As with other defining macros, it is 
  unspecified whether or not the compile-time side effects of a 
  \cd{declaim} persist after the file has been compiled
  (see section~\ref{COMPILER-SECTION}).
\end{defmac}
\end{newer}

\section{Declaration Specifiers}
\label{DECLARATION-SPECIFIERS-SECTION}

Here is a list of valid declaration specifiers for use in
\cd{declare}.  A construct is said to be ``affected'' by a declaration
if it occurs within the scope of a declaration.

\begin{flushdesc}
\item[\cd{special}]
\cd{(special {\it var1} {\it var2} ...)} specifies that all of
the variables named are to be considered {\it special}.
This specifier affects variable bindings but also pervasively
affects references.
All variable bindings affected are made to be dynamic bindings,
and affected variable references refer to the current dynamic binding
rather than to the current local binding.
For example:
\begin{lisp}
(defun hack (thing *mod*)~~~~~~~;{\rm The binding of the parameter} \\
~~(declare (special *mod*))~~~~~; {\rm \cd{*mod*} is visible to \cd{hack1},} \\
~~(hack1 (car thing)))~~~~~~~~~~; {\rm but not that of \cd{thing}} \\
 \\
(defun hack1 (arg) \\
~~(declare (special *mod*))~~~~~;{\rm Declare references to \cd{*mod*}} \\
~~~~~~~~~~~~~~~~~~~~~~~~~~~~~~~~; {\rm within \cd{hack1} to be special} \\
~~(if (atom arg) *mod* \\
~~~~~~(cons (hack1 (car arg)) (hack1 (cdr arg)))))
\end{lisp}
Note that it is conventional, though not required, to give special
variables names that begin and end with an asterisk.	

A \cd{special} declaration does {\it not} affect bindings pervasively.
Inner bindings of a variable implicitly shadow
a \cd{special} declaration and must be explicitly re-declared to
be special.
(However, a \cd{special} proclamation {\it does} pervasively affect bindings;
this exception is made for reasons of
convenience and compatibility with MacLisp.)
For example:
\begin{lisp}
(proclaim '(special x))~~~~~;{\rm \cd{x} is always special} \\
 \\
(defun example (x y) \\
~~(declare (special y)) \\
~~(let ((y 3) (x (* x 2))) \\
~~~~(print (+ y (locally (declare (special y)) y))) \\
~~~~(let ((y 4)) (declare (special y)) (foo x))))
\end{lisp}
In the contorted code above, the outermost and innermost bindings of
\cd{y} are special and therefore dynamically scoped, but the middle
binding is lexically scoped.  The two arguments to \cd{+} are different,
one being the value, which is \cd{3}, of the lexically bound variable
\cd{y}, and the other being the value of the special variable named \cd{y}
(a binding of which happens, coincidentally, to lexically surround it at
an outer level).  All the bindings of \cd{x} and references to \cd{x}
are special, however, because of the proclamation that \cd{x} is
always \cd{special}.

As a matter of style, use of \cd{special} proclamations should be
avoided.  The \cd{defvar} and \cd{defparameter} macros
are the conventional means for proclaiming special variables
in a program.

\item[\cd{type}]
\cd{(type {\it type} {\it var1} {\it var2} ...)} affects
only variable bindings and specifies that the
variables mentioned will take on values only of the specified type.
In particular, values assigned to the variables by \cd{setq},
as well as the initial values of the variables, must be of
the specified type.

\begin{new}
X3J13 voted in January 1989
\issue{DECLARE-TYPE-FREE}
to alter the interpretation of type declarations.
They are not to be construed to affect ``only variable bindings.''
The new rule for a declaration of a variable to
have a specified type is threefold:
\begin{itemize}
\item It is an error if, during the execution
of any reference to that variable within the scope of the declaration,
the value of the variable is not of the declared type.
\item It is an error if, during the execution
of a \cd{setq} of that variable within the scope of the declaration,
the new value for the variable is not of the declared type.
\item It is an error if, at any moment that execution enters the scope
of the declaration, the value of the variable is not of the
declared type.
\end{itemize}
One may think of a type declaration \cd{(declare (type face bodoni))}
as implicitly changing every reference to \cd{bodoni} within the scope
of the declaration to \cd{(the~face bodoni)}; changing every expression
{\it exp} assigned to \cd{bodoni} within the scope of the declaration
to \cd{(the~face {\it exp})}; and implicitly executing \cd{(the~face bodoni)}
every time execution enters the scope of the declaration.

These new rules make type declarations much more useful.  Under first
edition rules, a type declaration was useless if not associated with
a variable binding; declarations such as in
\begin{lisp}
(locally \\*
~~(declare (type (byte 8) x y)) \\*
~~(+ x y))
\end{lisp}
at best had no effect and at worst were erroneous, depending on one's
interpretation of the first edition.  Under the interpretation approved
by X3J13, such declarations have ``the obvious natural interpretation.''

X3J13 noted that if nested type declarations refer to the same variable,
then all of them have effect; the value of the variable must be a member of the
intersection of the declared types.

Nested type declarations could occur as a result of either macro expansion
or carefully crafted code.  There are three cases.  First,
the inner type might be a subtype of the outer one:
\begin{lisp}
(defun compare (apples oranges) \\*
~~(declare (type number apples oranges)) \\
~~(cond ((typep apples 'fixnum) \\*
~~~~~~~~~;; The programmer happens to know that, thanks to \\*
~~~~~~~~~;; constraints imposed by the caller, if APPLES \\*
~~~~~~~~~;; is a fixnum, then ORANGES will be also, and \\*
~~~~~~~~~;; therefore wishes to avoid the unnecessary cost \\*
~~~~~~~~~;; of checking ORANGES.~~Nevertheless the compiler \\*
~~~~~~~~~;; should be informed to allow it to optimize code. \\
~~~~~~~~~(locally (declare (type fixnum apples oranges))) \\*
~~~~~~~~~~~~~~~~~~;; Maybe the compiler could have figured \\*
~~~~~~~~~~~~~~~~~~;; out by flow analysis that APPLES must \\*
~~~~~~~~~~~~~~~~~~;; be a fixnum here, but it doesn't hurt \\*
~~~~~~~~~~~~~~~~~~;; to say it explicitly. \\*
~~~~~~~~~~~(< apples oranges))) \\
~~~~~~~~((or (complex apples) \\*
~~~~~~~~~~~~~(complex oranges)) \\*
~~~~~~~~~(error "Not yet implemented.~~Sorry.")) \\*
~~~~~~~~...))
\end{lisp}
This is the case most likely to arise in code written completely by hand.

Second, the outer type might be a subtype of the inner one.  In this
case the inner declaration has no additional practical effect, but
it is harmless.  This is
likely to occur if code declares a variable to be of a very specific type
and then passes it to a macro that then declares it to be of a less
specific type.

Third, the inner and outer declarations might be for types that
overlap, neither being a subtype of the other.  This is likely to occur
only as a result of macro expansion.  For example, user code might
declare a variable to be of type \cd{integer}, and a macro might
later declare it to be of type \cd{(or fixnum package)}; in this case
a compiler could intersect the two types to determine that in this
instance the variable may hold only fixnums.

The reader should note that the following code fragment is,
perhaps astonishingly, {\it not in error} under the interpretation approved by
X3J13:
\begin{lisp}
(let ((james .007) \\*
~~~~~~(maxwell 86)) \\*
~~(flet ((spy-swap () \\*
~~~~~~~~~~~(rotatef james maxwell))) \\*
~~~~(locally (declare (integer maxwell)) \\*
~~~~~~(spy-swap) \\*
~~~~~~(view-movie "The Sound of Music") \\*
~~~~~~(spy-swap) \\*
~~~~~~maxwell))) \\*
~\EV\ 86~~{\rm (after a couple of hours of Julie Andrews)}
\end{lisp}
The variable \cd{maxwell} is declared to be an integer over the {\it scope}
of the type declaration, not over its {\it extent}.  Indeed \cd{maxwell}
takes on the non-integer value \cd{.007} while the Trapp family make their
escape, but because no
reference to \cd{maxwell} within the scope of the declaration
ever produces a non-integer value, the code
is correct.

Now the assignment to \cd{maxwell} during the first call
to \cd{spy-swap}, and the reference to \cd{maxwell} during the second call,
{\it do} involve non-integer values, but they occur within the body of
\cd{spy-swap}, which is {\it not} in the scope of the type declaration!
One could put the declaration in a different place so as to include
\cd{spy-swap} in the scope:
\begin{lisp}
(let ((james .007) \\*
~~~~~~(maxwell 86)) \\*
~~(locally (declare (integer maxwell)) \\*
~~~~(flet ((spy-swap () \\*
~~~~~~~~~~~~~(rotatef james maxwell))) \\*
~~~~~~(spy-swap)~~~~~~~~~~~~~~~~~~~~~~~~~~~~~~~~~~~;{\rm Bug!}\\*
~~~~~~(view-movie "The Sound of Music") \\*
~~~~~~(spy-swap) \\*
~~~~~~maxwell)))
\end{lisp}
and then the code is indeed in error.
\end{new}

\begin{new}
X3J13 also voted in January 1989
\issue{FUNCTION-TYPE-ARGUMENT-TYPE-SEMANTICS}
to alter the meaning of the
\cd{function} type specifier when used in \cd{type} declarations
(see section~\ref{SPECIALIZED-TYPE-SPECIFIER-SECTION}).
\end{new}


\item[{\it type}]
\cd{({\it type} {\it var1} {\it var2} ...)}
is an abbreviation for
\cd{(type {\it type} {\it var1} {\it var2} ...)},
provided that {\it type} is one of the symbols appearing
in table~\ref{TYPE-SYMBOLS-TABLE}.

\begin{new}
Observe that this covers the particularly common case of declaring
numeric variables:
\begin{lisp}
(declare (single-float mass dx dy dz) \\*
~~~~~~~~~(double-float acceleration sum))
\end{lisp}
In many implementations there is also some advantage to declaring variables
to have certain specialized vector types such as \cd{base-string}.
\end{new}

\item[\cd{ftype}]
\cd{(ftype {\it type} {\it function-name-1} {\it function-name-2} ...)}
specifies that the named functions will be of the functional type
{\it type}, an example of which follows.
For example:
\begin{lisp}
(declare (ftype (function (integer list) t) nth) \\*
~~~~~~~~~(ftype (function (number) float) sin cos))
\end{lisp}
Note that rules of lexical scoping are observed; if one of the functions
mentioned has a lexically apparent local definition
(as made by \cd{flet} or \cd{labels}), then the declaration
applies to that local definition and not to the global function definition.

\begin{newer}
X3J13 voted in March 1989 \issue{FUNCTION-NAME} to extend \cd{ftype}
declaration specifiers
to accept any function-name (a symbol or a list
whose {\it car} is \cd{setf}---see section~\ref{FUNCTION-NAME-SECTION}).
Thus one may write
\begin{lisp}
(declaim (ftype (function (list) t) (setf cadr)))
\end{lisp}
to indicate the type of the \cd{setf} expansion function for \cd{cadr}.
\end{newer}

\begin{new}
X3J13 voted in January 1989
\issue{FUNCTION-TYPE-ARGUMENT-TYPE-SEMANTICS}
to alter the meaning of the
\cd{function} type specifier when used in \cd{ftype} declarations
(see section~\ref{SPECIALIZED-TYPE-SPECIFIER-SECTION}).
\end{new}
\end{flushdesc}

\begin{obsolete}
\begin{flushdesc}
\item[\cd{function}]
\cd{(function {\it name} {\it arglist} {\it result-type1} {\it result-type2} ...)}
is entirely equivalent to
\begin{lisp}
\cd{(ftype (function {\it arglist} {\it result-type1} {\it result-type2} ...) {\it name})}
\end{lisp}
but may be more convenient for some purposes.
For example:
\begin{lisp}
(declare (function nth (integer list) t) \\
~~~~~~~~~(function sin (number) float) \\
~~~~~~~~~(function cos (number) float))
\end{lisp}
The syntax mildly resembles that of \cd{defun}: a function-name,
then an argument list, then a specification of results.

Note that rules of lexical scoping are observed; if one of the functions
mentioned has a lexically apparent local definition
(as made by \cd{flet} or \cd{labels}), then the declaration
applies to that local definition and not to the global function definition.
\end{flushdesc}
\end{obsolete}

\begin{new}
X3J13 voted in January 1989
\issue{DECLARE-FUNCTION-AMBIGUITY}
to remove this interpretation
of the \cd{function} declaration specifier from the language.
Instead, a declaration specifier
\begin{lisp}
(function {\it var1} {\it var2} ...)
\end{lisp}
is to be treated simply as an abbreviation for
\begin{lisp}
(type function {\it var1} {\it var2} ...)
\end{lisp}
just as for all other symbols appearing in table~\ref{TYPE-SYMBOLS-TABLE}.

X3J13 noted that although \cd{function} appears in
table~\ref{TYPE-SYMBOLS-TABLE}, the first edition also discussed it
explicitly, with a different meaning,
without noting whether the differing
interpretation was to replace or augment the
interpretation regarding table~\ref{TYPE-SYMBOLS-TABLE}.  Unfortunately
there is an ambiguous case: the declaration
\begin{lisp}
(declare (function foo nil string))
\end{lisp}
can be construed to abbreviate either
\begin{lisp}
(declare (ftype (function () string) foo))
\end{lisp}
or
\begin{lisp}
(declare (type function foo nil string))
\end{lisp}
The latter could perhaps be rejected on semantic grounds: it would be an
error to declare \cd{nil}, a constant, to be of type \cd{function}.
In any case, X3J13 determined that the ice was too thin here;
the possibility of confusion is not worth the convenience of
an abbreviation for \cd{ftype} declarations.
The change also makes the language more consistent.
\end{new}

\begin{flushdesc}
\item[\cd{inline}]
\cd{(inline {\it function1} {\it function2} ...)} specifies that
it is desirable for the compiler to open-code
calls to the specified functions; that is, the code for a specified function
should be integrated into the calling routine, appearing in-line
in place of a procedure call.  This may achieve
extra speed at the expense of debuggability (calls to functions
compiled in-line cannot be traced, for example).
This declaration is pervasive.
Remember that
a compiler is free to ignore this declaration.

Note that rules of lexical scoping are observed; if one of the functions
mentioned has a lexically apparent local definition
(as established by \cd{flet} or \cd{labels}), then the declaration
applies to that local definition and not to the global function definition.

\begin{newer}
X3J13 voted in October 1988 \issue{PROCLAIM-INLINE-WHERE}
to clarify that during compilation the \cd{inline} declaration specifier
serves two distinct purposes: it indicates not only that affected calls
to the specified functions should be expanded in-line, but also that
affected definitions of the specified functions must be recorded for
possible use in performing such expansions.

Looking at it the other way,
the compiler is not required to save function definitions against the
possibility of future expansions unless the functions have already been
proclaimed to be \cd{inline}.  If a function is proclaimed (or declaimed)
\cd{inline}
before some call to that function but the current definition of that
function was established before the proclamation was processed,
it is implementation-dependent whether that call will be expanded in-line.
(Of course, it is implementation-dependent anyway, because a compiler
is always free to ignore \cd{inline} declaration specifiers.
However, the intent of the committee is clear: for best results,
the user is advised to put any \cd{inline} proclamation of
a function before any definition of or call to that function.)

Consider these examples:
\begin{lisp}
(defun huey (x) (+ x 100))~~~~~~~~~;{\rm Compiler need not remember this} \\*
(declaim (inline huey dewey)) \\*
(defun dewey (y) (huey (sqrt y)))~~;{\rm Call to \cd{huey} unlikely to be expanded} \\*
(defun louie (z) (dewey (/ z)))~~~~;{\rm Call to \cd{dewey} likely to be expanded}
\end{lisp}
\goodbreak

X3J13 voted in March 1989 \issue{FUNCTION-NAME} to extend \cd{inline}
declaration specifiers
to accept any function-name (a symbol or a list
whose {\it car} is \cd{setf}---see section~\ref{FUNCTION-NAME-SECTION}).
Thus one may write \cd{(declare (inline (setf cadr)))} to indicate
that the \cd{setf}
expansion function for \cd{cadr} should be compiled in-line.
\end{newer}

\item[\cd{notinline}]
\cd{(notinline {\it function1} {\it function2} ...)} specifies that it is
{\it undesirable} to compile the specified functions in-line.
This declaration is pervasive.
A compiler is {\it not} free to ignore this declaration.

Note that rules of lexical scoping are observed; if one of the functions
mentioned has a lexically apparent local definition
(as made by \cd{flet} or \cd{labels}), then the declaration
applies to that local definition and not to the global function definition.

\begin{newer}
X3J13 voted in March 1989 \issue{FUNCTION-NAME} to extend \cd{notinline}
declaration specifiers
to accept any function-name (a symbol or a list
whose {\it car} is \cd{setf}---see section~\ref{FUNCTION-NAME-SECTION}).
Thus one may write \cd{(declare (notinline (setf cadr)))} to indicate
that the \cd{setf}
expansion function for \cd{cadr} should not be compiled in-line.
\end{newer}

\begin{new}
X3J13 voted in January 1989
\issue{ALLOW-LOCAL-INLINE}
to clarify that the proper way to define a function \cd{gnards}
that is not \cd{inline} by default, but for which a local
declaration \cd{(declare (inline~gnards))} has half a chance of
actually compiling \cd{gnards} in-line, is as follows:
\begin{lisp}
(declaim (inline gnards)) \\*
\\*
(defun gnards ...) \\*
\\*
(declaim (notinline gnards))
\end{lisp}
The point is that the first declamation informs the compiler that
the definition of \cd{gnards} may be needed later for in-line expansion,
and the second declamation prevents any expansions unless and until it is
overridden.

While an implementation is never required to perform in-line expansion,
many implementations that do support such expansion will not
process \cd{inline} requests successfully unless definitions are
written with these proclamations in the manner shown above.
\end{new}

\item[\cd{ignore}]
\cd{(ignore {\it var1} {\it var2} ... {\it varn})} affects only variable bindings
and specifies that the bindings
of the specified variables are never used.  It is desirable for a compiler
to issue a warning if a variable so declared is ever referred to
or is also declared special, or if a variable is lexical, never referred to,
and not declared to be ignored.

\item[\cd{optimize}]
\cd{(optimize ({\it quality1} {\it value1}) ({\it quality2} {\it value2})...)}
advises the compiler that each {\it quality} should be given attention
according to the specified corresponding {\it value}.
A quality is a symbol; standard qualities
include \cd{speed} (of the object code), \cd{space} (both code size and
run-time space), \cd{safety} (run-time error checking),
and \cd{compilation-speed} (speed of the compilation process).
\begin{newer}
X3J13 voted in October 1988 \issue{OPTIMIZE-DEBUG-INFO} to add
the standard quality \cd{debug} (ease of debugging).
\end{newer}
Other qualities may be recognized by particular implementations.
A {\it value} should be a non-negative integer, normally in the range
\cd{0} to \cd{3}.  The value \cd{0} means that the quality is totally
unimportant, and \cd{3} that the quality is extremely important;
\cd{1} and \cd{2} are intermediate values, with \cd{1} the ``normal''
or ``usual'' value.
One may abbreviate \cd{({\it quality} 3)} to simply {\it quality}.
This declaration is pervasive.
For example:
\begin{lisp}
(defun often-used-subroutine (x y) \\*
~~(declare (optimize (safety 2))) \\*
~~(error-check x y) \\*
~~(hairy-setup x) \\
~~(do ((i 0 (+ i 1)) \\*
~~~~~~~(z x (cdr z))) \\*
~~~~~~((null z) i) \\
~~~~;; This inner loop really needs to burn. \\*
~~~~(declare (optimize speed)) \\*
~~~~(declare (fixnum i)) \\*
~~~~)))
\end{lisp}

\item[\cd{declaration}]
\cd{(declaration {\it name1} {\it name2} ...)} advises the compiler
that each {\it namej} is a valid but non-standard declaration name.
The purpose of this is to tell one compiler not to issue warnings
for declarations meant for another compiler or other program processor.

\begin{obsolete}
This kind of declaration may be used only as a proclamation.
For example:
\begin{lisp}
(proclaim '(declaration author \\*
~~~~~~~~~~~~~~~~~~~~~~~~target-language \\*
~~~~~~~~~~~~~~~~~~~~~~~~target-machine)) \\
 \\
(proclaim '(target-language ada)) \\
 \\
(proclaim '(target-machine IBM-650))
\end{lisp}
\newpage%manual
\begin{lisp}
(defun strangep (x) \\*
~~(declare (author "Harry Tweeker")) \\*
~~(member x '(strange weird odd peculiar)))
\end{lisp}
\end{obsolete}

\begin{newer}
X3J13 voted in June 1989 \issue{PROCLAIM-ETC-IN-COMPILE-FILE}
to introduce the new macro \cd{declaim}, which is guaranteed
to be recognized appropriately by the compiler and is often more convenient
than \cd{proclaim} for establishing global declarations.

The \cd{declaration} declaration specifier may be used with \cd{declaim}
as well as \cd{proclaim}.  The preceding examples would be better written
using \cd{declaim}, to ensure that the compiler will process them properly.

\begin{lisp}
(declaim (declaration author \\*
~~~~~~~~~~~~~~~~~~~~~~target-language \\*
~~~~~~~~~~~~~~~~~~~~~~target-machine)) \\
 \\
(declaim (target-language ada) \\*
~~~~~~~~~(target-machine IBM-650)) \\
 \\
(defun strangep (x) \\*
~~(declare (author "Harry Tweeker")) \\*
~~(member x '(strange weird odd peculiar)))
\end{lisp}
\end{newer}
\end{flushdesc}

\begin{newer}
X3J13 voted in March 1989 \issue{DYNAMIC-EXTENT} to introduce a new
declaration specifier \cd{dynamic-extent} for variables,
and voted in June 1989 \issue{DYNAMIC-EXTENT-FUNCTION}
to extend it to handle function-names as well.
\begin{flushdesc}
\item[\cd{dynamic-extent}]

\cd{(dynamic-extent {\it item1} {\it item2} ... {\it itemn})}
declares that certain variables or function-names refer to data objects
whose extents may be regarded as dynamic; that is, the declaration
may be construed as a guarantee on the part of the programmer that
the program will behave correctly even if the data objects have only
dynamic extent rather than the usual indefinite extent.

Each {\it item} may be either a variable name or \cd{(function {\it f\/})}
where {\it f} is a function-name (see section~\ref{FUNCTION-NAME-SECTION}).
(Of course, \cd{(function {\it f\/})} may be abbreviated in the usual way
as \cd{\#'{\it f}}.)

  It is permissible for an implementation simply to ignore this declaration.
  In implementations that do not ignore it, the compiler (or interpreter)
  is free to make whatever optimizations are appropriate given this
  information; the most common optimization is to stack-allocate the
  initial value of the object. The data types that can be optimized in this manner
  may vary from implementation to implementation.

The meaning of this declaration can be stated more precisely.
We say that
object {\it x} is an {\it otherwise inaccessible part}
    of {\it y} if and only if making {\it y} inaccessible would make {\it x} inaccessible.
    (Note that every object is an otherwise inaccessible part of itself.)
  Now suppose that construct {\it c} contains a \cd{dynamic-extent} declaration for
  variable (or function) {\it v} (which need not be bound by {\it c}).  Consider the values
  ${\it w}_1, \ldots, {\it w}_{\hbox{\scriptsize\it n}}$ taken on by {\it v} during the course of some execution of
  {\it c}.  The declaration asserts that if some object {\it x}
  is an otherwise inaccessible part of ${\it w}_{\hbox{\scriptsize\it j}}$
  whenever ${\it w}_{\hbox{\scriptsize\it j}}$ becomes the value of {\it v},
  then just after execution of
  $c$ terminates {\it x} will be either inaccessible or
  still an otherwise inaccessible part of the value of {\it v}.
  If this assertion is ever violated, the consequences are undefined.

  In some implementations, it is
  possible to allocate data structures in a way that will make them
  easier to reclaim than by general-purpose garbage collection
  (for example, on the stack or in some temporary area).  The \cd{dynamic-extent}
  declaration is designed to give the implementation the information
  necessary to exploit such techniques.

For example, in the code fragment
\begin{lisp}
(let ((x (list 'a1 'b1 'c1)) \\*
~~~~~~(y (cons 'a2 (cons 'b2 (cons 'c2 'd2))))) \\*
~~(declare (dynamic-extent x y)) \\*
~~...)
\end{lisp}
it is not difficult to prove that
the otherwise inaccessible parts of \cd{x} include the three conses constructed by \cd{list},
and that the otherwise inaccessible parts of \cd{y} include three other
conses manufactured by the three calls to \cd{cons}.
Given the presence of the \cd{dynamic-extent} declaration, a compiler would be
justified in stack-allocating these six conses and reclaiming their storage
on exit from the \cd{let} form.

  Since stack allocation of the initial value entails knowing at the
  object's creation time that the object can be stack-allocated, it is
  not generally useful to declare \cd{dynamic-extent} for variables
  that have no lexically apparent initial value. For example,
\begin{lisp}
(defun f () \\*
~~(let ((x (list 1 2 3))) \\*
~~~~(declare (dynamic-extent x)) \\*
~~~~...))
\end{lisp}
  would permit a compiler to stack-allocate the
  list in \cd{x}. However,
\begin{lisp}
(defun g (x) (declare (dynamic-extent x)) ...) \\*
(defun f () (g (list 1 2 3)))
\end{lisp}
  could not typically permit a similar optimization in \cd{f} because of the
  possibility of later redefinition of \cd{g}.
  Only an implementation careful enough to recompile \cd{f}
  if the definition of \cd{g} were to change incompatibly could stack-allocate
  the list argument to \cd{g} in \cd{f}.

  Other interesting cases are
\begin{lisp}
(declaim (inline g)) \\*
(defun g (x) (declare (dynamic-extent x)) ...) \\*
(defun f () (g (list 1 2 3)))
\end{lisp}
and
\begin{lisp}
(defun f () \\*
~~(flet ((g (x) (declare (dynamic-extent x)) ...)) \\*
~~~~(g (list 1 2 3))))
\end{lisp}
  In each case some compilers might realize the optimization is possible and others
  might not.

  An interesting variant of this is the so-called {\it stack-allocated rest list},
  which can be achieved (in implementations supporting the optimization) by
\begin{lisp}
(defun f (\&rest x) \\*
~~(declare (dynamic-extent x)) \\*
~~...)
\end{lisp}
  Note here that although the initial value of \cd{x} is not explicitly present,
  nevertheless in the usual implementation strategy the
  function \cd{f} is responsible for assembling the list for \cd{x} from the passed arguments,
  so the \cd{f} function can be optimized by a compiler to construct a 
  stack-allocated list instead of a heap-allocated list.

Some Common Lisp functions take other functions as arguments; frequently
the argument function is a so-called {\it downward funarg}, that is, a functional
argument that is passed only downward and whose extent may therefore be dynamic.
\begin{lisp}
(flet ((gd (x) (atan (sinh x)))) \\*
~~(declare (dynamic-extent \#'gd))~~~~~;{\rm \cd{mapcar} won't hang on to \cd{gd}}\\*
~~(mapcar \#'gd my-list-of-numbers))
\end{lisp}


The following three examples are in error, since in each case
the value of \cd{x} is used outside of its
extent.
\begin{lisp}
(length (let ((x (list 1 2 3))) \\*
~~~~~~~~~~(declare (dynamic-extent x)) \\*
~~~~~~~~~~x))~~~~~~~~~~~~~~~~~~~~~~~~~~~~~~~~~~~~;{\rm Wrong}
\end{lisp}
The preceding code is obviously incorrect, because the cons cells making
up the list in \cd{x} might be deallocated (thanks to the declaration)
before \cd{length} is called.
\begin{lisp}
(length (list (let ((x (list 1 2 3))) \\*
~~~~~~~~~~~~~~~~(declare (dynamic-extent x)) \\*
~~~~~~~~~~~~~~~~x)))~~~~~~~~~~~~~~~~~~~~~~~~~~~~~;{\rm Wrong}
\end{lisp}
In this second case it is less obvious that
the code is incorrect, because one might argue that
the cons cells making
up the list in \cd{x} have no effect on the result to be computed by \cd{length}.
Nevertheless the code briefly violates the assertion implied by the declaration
and is therefore incorrect.  (It is not difficult to imagine a perfectly
sensible implementation of a garbage collector that might become confused
by a cons cell containing a dangling pointer to a list that was once stack-allocated
but then deallocated.)
\begin{lisp}
(progn (let ((x (list 1 2 3))) \\*
~~~~~~~~~(declare (dynamic-extent x)) \\*
~~~~~~~~~x)~~~~~~~~~~~~~~~~~~~~~~~~~~~~~~~~~~~~~~;{\rm Wrong} \\*
~~~~~~~(print "Six dollars is your change have a nice day NEXT!"))
\end{lisp}
In this third case it is even less obvious that
the code is incorrect, because the value of \cd{x}
returned from the \cd{let} construct is discarded right away by the \cd{progn}.
Indeed it is, but ``right away'' isn't fast enough.
The code briefly violates the assertion implied by the declaration
and is therefore incorrect.  (If the code is being interpreted,
the interpreter might hang on to the value returned by the \cd{let}
for some time before it is eventually discarded.)

Here is one last example, one that has little practical import but
is theoretically quite instructive.
\begin{lisp}
(dotimes (j 10) \\*
~~(declare (dynamic-extent j)) \\*
~~(setq foo 3)~~~~~~~~~~~~~~~~~~~~~;{\rm Correct} \\*
~~(setq foo j))~~~~~~~~~~~~~~~~~~~~;{\rm Erroneous---but why? (see text)}
\end{lisp}
Since \cd{j} is an integer by the
definition of \cd{dotimes}, but \cd{eq} and \cd{eql} are not necessarily equivalent for
integers, what are the otherwise inaccessible parts of \cd{j}, which this declaration
requires the body of the \cd{dotimes} not to ``save''?  If the value of \cd{j} is \cd{3},
and the body does \cd{(setq foo~3)}, is that an error?  The answer is no, but
the interesting thing is that it depends on the implementation-dependent
behavior of \cd{eq} on numbers.  In an implementation where \cd{eq} and \cd{eql} are
equivalent for \cd{3}, then \cd{3} is not an otherwise inaccessible part because
\cd{(eq~j (+~2~1))} is true,
and therefore there is another way to access the object besides
going through \cd{j}.  On the other hand, in an implementation where \cd{eq} and
\cd{eql} are not equivalent for \cd{3}, then the particular \cd{3} that is the value of
\cd{j} is an otherwise inaccessible part, but any other \cd{3} is not.
Thus \cd{(setq foo~3)} is valid
but \cd{(setq foo~j)} is erroneous.  Since \cd{(setq foo~j)} is erroneous in some
implementations, it is erroneous in all portable programs, but some other
implementations may not be able to detect the error.  (If this conclusion seems
strange, it may help to replace \cd{3} everywhere
in the preceding argument with some obvious
bignum such as \cd{375374638837424898243} and to replace
\cd{10} with some even larger bignum.)

  The \cd{dynamic-extent} declaration should be used with great care.
  It makes possible great performance improvements in some situations, but
  if the user misdeclares
  something and consequently the implementation
  returns a pointer into the stack (or stores it in the heap),
  an undefined situation may result and the integrity of the Lisp storage
  mechanism may be compromised. Debugging these situations may be tricky.
  Users who have asked for this feature have indicated a willingness
  to deal with such problems; nevertheless, I do not encourage
  casual users to use this declaration.
\end{flushdesc}
\end{newer}

An implementation is free to support other (implementation-dependent)
declaration specifiers as well.
On the other hand, a Common Lisp compiler is free to
ignore entire classes of declaration specifiers (for example,
implementation-dependent declaration specifiers
not supported by that compiler's
implementation), except for the \cd{declaration} declaration specifier.
Compiler implementors are encouraged, however, to
program the compiler to issue by default a warning if the compiler finds
a declaration specifier of a kind it never uses.  Such a warning is required
in any case
if a declaration specifier is not one of those defined above and has not been
declared in a \cd{declaration} declaration.

\section{Type Declaration for Forms}

Frequently it is useful to declare that the value produced
by the evaluation of some form will be of a particular type.
Using \cd{declare} one can declare the type of the value
held by a bound variable, but there is no easy way to declare
the type of the value of an unnamed form.  For this purpose the \cd{the}
special form is defined; \cd{(the {\it type} {\it form})} means
that the value of {\it form} is declared to be of type {\it type}.

\begin{defspec}
the value-type form

The {\it form} is evaluated; whatever it produces is returned by
the \cd{the} form.  In addition, it is an error if what is produced
by the {\it form} does not conform to the data type specified by {\it value-type}
(which is not evaluated).  (A given implementation may or may not
actually check for this error.  Implementations are encouraged to make an
explicit error check when running interpretively.)  In effect, this
declares that the user undertakes to guarantee that the values of
the form will always be of the specified type.
For example:
\begin{lisp}
(the string (copy-seq x))~~~~~;{\rm The result will be a string} \\
(the integer (+ x 3))~~~~~~~~~;{\rm The result of \cd{+} will be an integer} \\
(+ (the integer x) 3)~~~~~~~~~;{\rm The value of \cd{x} will be an integer} \\
(the (complex rational) (* z 3)) \\
(the (unsigned-byte 8) (logand x mask))
\end{lisp}
The \cd{values} type specifier may be used to indicate the types
of multiple values:
\begin{lisp}
(the (values integer integer) (floor x y)) \\
(the (values string t) \\
~~~~~(gethash the-key the-string-table))
\end{lisp}

\begin{newer}
X3J13 voted in June 1989 \issue{THE-AMBIGUITY}
to clarify that {\it value-type} may be any valid type specifier whatsoever.
The point is that a type specifier need not be one suitable for
discrimination but only for declaration.

In the case that the {\it form} produces exactly one value and {\it value-type}
is not a \cd{values} type specifier, one may describe a \cd{the} form
as being entirely equivalent to
\begin{lisp}
(let ((\#1=\#:temp {\it form})) (declare (type {\it value-type} \#1\#)) \#1\#)
\end{lisp}
A more elaborate expression could be written to describe the case where
{\it value-type} is a \cd{values} type specifier.
\end{newer}

\beforenoterule
\begin{incompatibility}
This construct is borrowed from the Interlisp DECL
package; Interlisp, however, allows an implicit \cd{progn} after the type
specifier rather than just a single form.  The MacLisp \cd{fixnum-identity}
and \cd{flonum-identity} constructs can be expressed as \cd{(the fixnum {\it x})}
and \cd{(the single-float {\it x})}.
\end{incompatibility}
\afternoterule
\end{defspec}
      % Declarations
%Part{Symbol, Root = "CLM.MSS"}
%Chapter of Common Lisp Manual.  Copyright 1984, 1988, 1989 Guy L. Steele Jr.

\clearpage\def\pagestatus{ULTIMATE}


\chapter{Symbols Символы}
\label{symbol}

A Lisp symbol is a data object that has three user-visible
components:
\begin{itemize}
\item
The \emph{property list} is a list that effectively provides each symbol
with many modifiable named components.

\item
The \emph{print name} must be a string, which is the sequence of
characters used to identify the symbol.  Symbols are of great use
because a symbol can be located once its name is given
(typed, say, on a keyboard).
One may ordinarily not alter a symbol's print name.
\end{itemize}

\begin{newer}
X3J13 voted in March 1989 \issue{CHARACTER-PROPOSAL}
to specify it is an error to alter a print name.
\end{newer}
\begin{itemize}
\item
The \emph{package cell} must refer to a package object.
A package is a data structure
used to locate a symbol once given the symbol's name.
A symbol is uniquely identified
by its name only when considered relative to a package.  A symbol may
appear in many packages, but it can be \emph{owned} by at most one package.
The package cell points to the owner, if any.
Package cells are discussed along with packages in chapter~\ref{XPACK}.
\end{itemize}

Lisp'овые символы является объектами данных, которые имеют три элемента, видимых
для пользователя:
\begin{itemize}
\item
\emph{Список свойств} является списком, который позволяет хранить в символе
именованные изменяемые данные.

\item
\emph{Выводимое имя} должно быть строкой, которая является последовательностью
строковых символов, идентифицирующей символ. Символы несут большую пользу, так
как они могут быть обозначены просто заданным именем (например, напечатанным на
клавиатуре).
Выводимое имя изменять нельзя.

\item
\emph{Ячейка пакета} должна ссылаться на объект пакета.
Пакет является структурой данных, используемой для группирования имен символов.
Символ уникально идентифицируется по имени, только когда рассматривается
относительно пакета. Символ может встречаться в нескольких пакетах, но
\emph{родительским} пакетом может быть как максимум только один.
Ячейка пакета ссылается на родительский пакет, если он есть.
Ячейки пакетов обсуждаются в главе~\ref{XPACK}.
\end{itemize}

A symbol may actually have other components for use by the
implementation.  One of the more important uses of symbols is as
names for program variables; it is frequently desirable for the
implementor to use certain components of a symbol to implement
the semantics of variables.  See \cdf{symbol-value}
and \cdf{symbol-function}.
However, there are several possible
implementation strategies, and so such possible components are not
described here.

Символ может также содержать другие элементы, которые используются
реализацией. Еще одна важная функция --- это использование символов в качестве
имен переменных.  Желательно, чтобы разработчик использовал такие элементы
символа для реализации семантики переменных. Смотрите \cdf{symbol-value} и
\cdf{symbol-function}. Однако, существует несколько стратегии реализации, и
такие возможные элементы символов здесь не описаны.

\section{The Property List Список свойств}

Since its inception, Lisp has associated with each symbol
a kind of tabular data structure
called a \emph{property list} (\emph{plist} for short).  A property list contains
zero or more entries; each entry associates with a key
(called the \emph{indicator}), which is typically
a symbol, an arbitrary Lisp object (called the \emph{value} or,
sometimes, the \emph{property}).
There are no duplications among the indicators; a property list may only
have one property at a time with a given name.  In this way, given
a symbol and an indicator (another symbol), an associated value can be
retrieved.

Начиная с самого создания, Lisp для каждого символа ассоциирует табличную
структуру данных, называемую \emph{список свойств} (для краткости \emph{plist}).
Список свойств содержит ноль и более элементов. Каждый элемент содержит ключ
(называемым \emph{индикатором}), который чаще всего является символом, и
ассоциированным с ним значением (называемым иногда \emph{свойством}), которое
может быть любым Lisp'овым объектом.
Среди индикаторов не может быть дубликатов. Список свойств может иметь только
одно свойство для данного имени. Таким образом, значение может получено с
помощью двух символов: исходного символа и индикатора.

A property
list is very similar in purpose to an association list. The difference
is that a property list is an object with a unique identity; the
operations for adding and removing property-list entries are destructive
operations that alter the property list rather than making a new one.
Association lists, on the other hand, are normally augmented
non-destructively (without side effects) by adding new entries to the
front (see \cdf{acons} and \cdf{pairlis}).

Список свойств по целевому назначению очень похож на ассоциативный
список. Различие в том, что список свойств является единственно подлинным
объектом. Операции добавления и удаления элементов деструктивны, то есть при их
использовании изменяется старый список, и нового списка свойств не создается.
Ассоциативные список, наоборот, обычно изменяются недеструктивно (без побочных
эффектов) с помощью добавления в начало новых элементов (смотрите \cdf{acons} и
\cdf{pairlis}).

A property list is implemented as a memory cell
containing a list with an even number (possibly zero) of elements.
(Usually this memory cell is the property-list cell of a symbol,
but any memory cell acceptable to \cdf{setf} can be used
if \cdf{getf} and \cdf{remf} are used.)
Each pair of elements in the list constitutes an entry;
the first item is the indicator, and the second is the
value.  Because property-list functions are given the symbol
and not the list itself, modifications to the property list
can be recorded by storing back into the property-list cell of the symbol.

Список свойств реализуется, как ячейка памяти, содержащая список с четным
(возможно нулевым) количеством аргументов.
(Обычно эта ячейка памяти является ячейкой списка свойств в символе, но в
принципе подходит любая ячейка памяти, к которой можно применить \cdf{setf})
Каждая пара элементов в списке составляет строку.
Первый в паре это индикатор, а второй --- значение. Так как функции для списка
свойств используют символ, а не сам список, то изменения этого списка свойств
может быть записаны сохранением обратно в ячейку списка свойств символа. FIXME

When a symbol is created, its property list is initially empty.
Properties are created by using \cdf{get} within a \cdf{setf} form.

Когда создается символ, его список свойств пуст. Свойства создаются с помощью
\cdf{get} внутри формы \cdf{setf}.

Common Lisp does not use a symbol's property list as extensively as earlier
Lisp implementations did.  Less-used data, such as compiler,
debugging, and documentation information, is kept on property lists
in Common Lisp.

Common Lisp не использует список свойств символа так интенсивно, как это делали
ранние реализации Lisp'а. В Common Lisp'е нечасто используемые данные, такие как
отладочная информация, информация для компилятора и документация, хранятся в списках
свойств.

\beforenoterule
\begin{incompatibility}
In older Lisp implementations, the
print name, value, and function definition of a symbol were kept on its
property list.  The value cell was introduced into MacLisp and Interlisp
to speed up access to variables; similarly for the
print-name cell and function cell (MacLisp does not use a function cell).
Recent Lisp implementations such as Spice Lisp,
Lisp Machine Lisp, and NIL have
introduced all of these cells plus the
package cell.
None of the MacLisp system property names
(\cdf{expr}, \cdf{fexpr}, \cdf{macro}, \cdf{array},
\cdf{subr}, \cdf{lsubr}, \cdf{fsubr}, and in former times \cdf{value} and
\cdf{pname}) exist in Common Lisp.

In Common Lisp, the notion of ``disembodied property list''
introduced in MacLisp is eliminated.  It tended to be used for
rather kludgy things, and in Lisp Machine Lisp is often associated with
the use of locatives (to make it ``off by one'' for searching
alternating keyword lists).  In Common Lisp special \cdf{setf}-like
property-list functions are introduced: \cdf{getf}
and \cdf{remf}.
\end{incompatibility}
\afternoterule

\begin{defun}[Function]
get symbol indicator &optional default

\cdf{get} searches the property list of
\emph{symbol} for an indicator \cdf{eq} to \emph{indicator}.
The first argument must be a symbol.
If one is found, then the corresponding value is returned;
otherwise \emph{default} is returned.

\cdf{get} ищет в списке свойств символа \emph{symbol} индикатор равный \cdf{eq}
индикатору \emph{indicator}.
Первый аргумент должен быть символом.
Если такое свойство найдено, возвращается ее значение. Иначе возвращается
значение \emph{default}.

If \emph{default} is not specified,
then {\false} is used for \emph{default}.

Если \emph{default} не указано, тогда для значения по-умолчанию используется
{\false}.

Note that there is no way to distinguish an absent property from
one whose value is \emph{default}.
\begin{lisp}
(get x y) \EQ\ (getf (symbol-plist x) y)
\end{lisp}
Suppose that the property list of \cdf{foo} is \cd{(bar t baz 3 hunoz "Huh?")}.
Then, for example:
\begin{lisp}
(get 'foo 'baz) \EV\ 3 \\
(get 'foo 'hunoz) \EV\ "Huh?" \\
(get 'foo 'zoo) \EV\ {\false}
\end{lisp}

Следует отметить, что способа отличить значения по-умолчанию и такое же значение
свойства нет:
\begin{lisp}
(get x y) \EQ\ (getf (symbol-plist x) y)
\end{lisp}
Допустим, что список свойств символа \cd{foo} является \cd{(bar t baz 3 hunoz
  "Huh?")}.
Тогда, например:
\begin{lisp}
(get 'foo 'baz) \EV\ 3 \\
(get 'foo 'hunoz) \EV\ "Huh?" \\
(get 'foo 'zoo) \EV\ {\false}
\end{lisp}
\beforenoterule
\begin{incompatibility}
In MacLisp, the first argument to \cdf{get} could
be a list, in which case the \emph{cdr} of the list was treated
as a so-called ``disembodied property list.''
The first argument to \cdf{get}
could also be any other object, in which case \cdf{get} would 
always return {\nil}.  In Common Lisp, it is an error to give anything
but a symbol as the first argument to \cdf{get}.

What Common Lisp calls \cdf{get}, Interlisp calls \cdf{getprop}.

What MacLisp and Interlisp call \cdf{putprop} is accomplished
in Common Lisp by using \cdf{get} with \cdf{setf}.
\end{incompatibility}
\afternoterule

\cdf{setf} may be used with \cdf{get} to create a new property-value
pair, possibly replacing an old pair with the same property name.
For example:
\begin{lisp}
(get 'clyde 'species) \EV\ {\false} \\
(setf (get 'clyde 'species) 'elephant) \EV\ elephant \\
\textrm{and now} (get 'clyde 'species) \EV\ elephant
\end{lisp}
The \emph{default} argument may be
specified to \cdf{get} in this context; it is ignored by \cdf{setf} but
may be useful in such macros as \cdf{push} that are related to \cdf{setf}:
\begin{lisp}
(push item (get sym 'token-stack '(initial-item)))
\end{lisp}
means approximately the same as
\begin{lisp}
(setf (get sym 'token-stack '(initial-item)) \\
~~~~~~(cons item (get sym 'token-stack '(initial-item))))
\end{lisp}
which in turn would be treated as simply
\begin{lisp}
(setf (get sym 'token-stack) \\
~~~~~~(cons item (get sym 'token-stack '(initial-item))))
\end{lisp}

\cdf{setf} может использоваться вместе с \cdf{get} для созданя новой пары
свойства, возожно замещая старую пару с тем же именем.
Например:
\begin{lisp}
(get 'clyde 'species) \EV\ {\false} \\
(setf (get 'clyde 'species) 'elephant) \EV\ elephant \\
\textrm{и теперь} (get 'clyde 'species) \EV\ elephant
\end{lisp}
В данном контексте может быть указан аргумент \emph{default}. Он игнорируется в
\cdf{setf}, но может быть полезен в таких макросах, как \cdf{push}, которые
связаны с \cdf{setf}:
\begin{lisp}
(push item (get sym 'token-stack '(initial-item)))
\end{lisp}
означает то же, что и
\begin{lisp}
(setf (get sym 'token-stack '(initial-item)) \\
~~~~~~(cons item (get sym 'token-stack '(initial-item))))
\end{lisp}
а если упростить, то
\begin{lisp}
(setf (get sym 'token-stack) \\
~~~~~~(cons item (get sym 'token-stack '(initial-item))))
\end{lisp}

\begin{newer}
X3J13 voted in March 1989 \issue{REMF-DESTRUCTION-UNSPECIFIED}
to clarify the permissible side effects of certain operations;
\cd{(setf (get \emph{symbol} \emph{indicator}) \emph{newvalue})}
is required to behave exactly the same as
\cd{(setf (getf (symbol-plist \emph{symbol}) \emph{indicator}) \emph{newvalue})}.
\end{newer}

\end{defun}

\begin{defun}[Function]
remprop symbol indicator

This removes from \emph{symbol} the property with an indicator \cdf{eq}
to \emph{indicator}.  The property indicator and the corresponding
value are removed by destructively splicing the property
list.  It returns {\false} if no such property was found,
or non-{\false} if a property was found.
\begin{lisp}
(remprop x y) \EQ\ (remf (symbol-plist x) y)
\end{lisp}
For example, if the property list of \cdf{foo} is initially
\begin{lisp}
(color blue height 6.3 near-to bar)
\end{lisp}
then the call
\begin{lisp}
(remprop 'foo 'height)
\end{lisp}
returns a non-{\nil} value after altering \cdf{foo}'s property list to be
\begin{lisp}
(color blue near-to bar)
\end{lisp}

Эта функция удаляет из символа \emph{symbol} свойство с индикатором, равным
\cdf{eq} индикатору \emph{indicator}. Индикатор свойства и соответствующее
значение удаляется из списка деструктивной склейкой списка свойств. Она
возвращает {\false}, если указанного свойства не было, или не-{\false}, если
свойство было.
\begin{lisp}
(remprop x y) \EQ\ (remf (symbol-plist x) y)
\end{lisp}
Например, если список свойств \cd{foo} равен
\begin{lisp}
(color blue height 6.3 near-to bar)
\end{lisp}
тогда вызов
\begin{lisp}
(remprop 'foo 'height)
\end{lisp}
вернет значение не-{\nil}, после изменения списка свойств символа \cd{foo} на
\begin{lisp}
(color blue near-to bar)
\end{lisp}

\begin{newer}
X3J13 voted in March 1989 \issue{REMF-DESTRUCTION-UNSPECIFIED}
to clarify the permissible side effects of certain operations;
\cd{(remprop \emph{symbol} \emph{indicator})}
is required to behave exactly the same as
\cd{(remf (symbol-plist \emph{symbol}) \emph{indicator})}.
\end{newer}
\end{defun}

\begin{defun}[Function]
symbol-plist symbol

This returns the list that contains the property pairs of \emph{symbol};
the contents of the property-list cell are extracted and returned.

Note that using \cdf{get} on the result of \cdf{symbol-plist} does \emph{not} work.
One must give the symbol itself to \cdf{get} or else
use the function \cdf{getf}.

\cdf{setf} may be used with \cdf{symbol-plist} to destructively replace
the entire property list of a symbol.  This is a relatively dangerous
operation, as it may destroy important information that
the implementation may happen to store in property lists.
Also, care must be taken that the new
property list is in fact a list of even length.
\beforenoterule
\begin{incompatibility}
In MacLisp, this function is called \cdf{plist};
in Interlisp, it is called \cdf{getproplist}.
\end{incompatibility}
\afternoterule

Эта функция возвращает список, который содержит пары свойств для символа
\emph{symbol}. Извлекается содержимое ячейки списка свойств и возвращается в
качестве результата.

Следует отметить, что использование \cdf{get} с результатом \cdf{symbol-plist}
\emph{не} будет работать. Необходимо передавать в \cdf{get} символ, или же
использовать \cdf{getf}.

С \cdf{symbol-plist} может использоваться \cdf{setf} для деструктивной замены
списка свойств символа. Это относительно опасная операция, так как может
уничтожить важную информацию, которую, возможно, хранила там реализация.
Также, позаботтесь о том, чтобы новый список свойств содержал четное количество
элементов.
\end{defun}

\begin{defun}[Function]
getf place indicator &optional default

\cdf{getf} searches the property list stored in \emph{place}
for an indicator \cdf{eq} to \emph{indicator}.
If one is found, then the corresponding value is returned;
otherwise \emph{default} is returned.  If \emph{default} is not specified,
then {\false} is used for \emph{default}.
Note that there is no way to distinguish an absent property from
one whose value is \emph{default}.
Often \emph{place} is computed from
a generalized variable acceptable to \cdf{setf}.

\cdf{getf} ищет индикатор равный \cdf{eq} \emph{indicator} в списке свойств,
находящимся в \emph{place}. Если он найден, тогда возвращается соответствующее
значение. Иначе возвращается \emph{default}. Если \emph{default} не задан, то
значение по-умолчанию используется {\falsee}.
Следует отметить, что метода определения, вернулось ли значение по умолчанию или
это значение свойства, нет.
Часто \emph{place} вычисляется из обобщенной переменной, принимаемой функцией
\cdf{setf}.

\cdf{setf} may be used with \cdf{getf}, in which case the \emph{place} must
indeed be acceptable as a \emph{place} to \cdf{setf}.  The effect is to
add a new property-value pair, or update an existing pair,
in the property list kept in the \emph{place}.
The \emph{default} argument may be
specified to \cdf{getf} in this context; it is ignored by \cdf{setf} but
may be useful in such macros as \cdf{push} that are related to \cdf{setf}.
See the description of \cdf{get} for an example of this.

\cdf{setf} может использоваться вместе \cdf{getf}, и в этом случае \emph{place}
должно быть обобщенным, чтобы его можно было передать в \cdf{setf}. Целью
является добавление новой пары свойства, или изменению уже существующей пары, в
списке свойств, хранящимся в \emph{place}.
В данном контексте может быть использован аргумент \emph{default}. Он
игнорируется функцией \cdf{setf}, но может быть полезен в таких макросах, как
\cdf{push}, которые связаны с \cdf{setf}.
Смотрите описание \cdf{get} для примера.

\begin{newer}
X3J13 voted in March 1989 \issue{REMF-DESTRUCTION-UNSPECIFIED}
to clarify the permissible side effects of certain operations;
\cdf{setf} used with \cdf{getf} is permitted to perform a \cdf{setf}
on the \emph{place} or on any part, \emph{car} or \emph{cdr}, of the
top-level list structure held by that \emph{place}.
\end{newer}

\begin{newer}
X3J13 voted in March 1988 \issue{PUSH-EVALUATION-ORDER}
to clarify order of evaluation (see section~\ref{SETF-SECTION}).
\end{newer}

\beforenoterule
\begin{incompatibility}
The Interlisp function \cdf{listget} is similar to \cdf{getf}.
The Interlisp function \cdf{listput} is similar to using \cdf{getf}
with \cdf{setf}.
\end{incompatibility}
\afternoterule
\end{defun}

\begin{defmac}
remf place indicator

This removes from the property list stored in \emph{place}
the property with an indicator \cdf{eq}
to \emph{indicator}.  The property indicator and the corresponding
value are removed by destructively splicing the property
list.  \cdf{remf} returns {\false} if no such property was found,
or some non-{\nil} value if a property was found.
The form \emph{place} may be any generalized variable acceptable to \cdf{setf}.
See \cdf{remprop}.

Эта функция в списке свойств, хранящимся в \emph{place}, удаляет свойства,
индикатор которого равен \cdf{eq} аргументу \emph{indicator}. Индикатор свойства
и соответствующее значение удаляется из списка деструктивной
склейкой. \cdf{remf} возвращает {\false}, если свойства не было в списке, и
некоторое не-{\nil} значение, если свойство было.
Форма \emph{place} может быть любой обобщенной переменной, принимаемой
\cdf{setf}.
Смотрите \cdf{remprop}.

\begin{newer}
X3J13 voted in March 1989 \issue{REMF-DESTRUCTION-UNSPECIFIED}
to clarify the permissible side effects of certain operations;
\cdf{remf} is permitted to perform a \cdf{setf}
on the \emph{place} or on any part, \emph{car} or \emph{cdr}, of the
top-level list structure held by that \emph{place}.
\end{newer}

\begin{newer}
X3J13 voted in March 1988 \issue{PUSH-EVALUATION-ORDER}
to clarify order of evaluation (see section~\ref{SETF-SECTION}).
\end{newer}
\end{defmac}

\begin{defun}[Function]
get-properties place indicator-list

\cdf{get-properties} is like \cdf{getf}, except that the second
argument is a list of indicators.  \cdf{get-properties} searches the
property list stored in \emph{place} for any of the indicators in
\emph{indicator-list} until it finds the first property
in the property list whose indicator is one of
the elements of \emph{indicator-list}.  Normally \emph{place} is computed from
a generalized variable acceptable to \cdf{setf}.

\cdf{get-properties} похожа на \cdf{getf} за исключением того, что второй
аргумент является списком индикаторов. \cdf{get-properties} ищет первый
встретившийся индикатор, который есть в списке индикаторов. Обычно \emph{place}
вычисляется из обобщенной переменной, которая может использоваться в \cdf{setf}.

\cdf{get-properties} returns three values.
If any property was found, then
the first two values are the indicator
and value for the first property whose indicator was in \emph{indicator-list},
and the third is that tail of the property list whose \emph{car}
was the indicator (and whose \emph{cadr} is therefore the value).
If no property was found, all three values are {\nil}.
Thus the third value serves as a flag indicating success or failure
and also allows the search to be restarted, if desired, after the property
was found.

\cdf{get-properties} возвращает три значения.
Если было найдено свойство, то первые два значения являеются индикатором и
значением для первого свойства, чей индикатор присутствовал в списке
\emph{indicator-list}, и третье значение является остатком списка свойств,
\emph{car} элемент которого был индикатором (и соответственно \emph{cadr}
элемент --- значением).
Если ни одного свойства не было найдено, то все три значения равны {\nil}.
Третье значение является флагом успешности или неудачи, и позволяет продолжить
поиск свойств в оставшемся списке.
\end{defun}

\section{The Print Name Выводимое имя}
Every symbol has an associated string called the \emph{print name}.
This string is used as the external representation of the symbol:
if the characters in the string are typed in to \cdf{read}
(with suitable escape conventions for certain characters),
it is interpreted as a reference to that symbol
(if it is interned); and if the symbol is printed, \cdf{print} types out the
print name.
For more information, see the sections on the \emph{reader}
(section~\ref{READER})
and \emph{printer} (section~\ref{PRINTER}).

Каждый символ имеет ассоциированную строку, называемую \emph{выводимое имя}.
Строка используется для вывода отображения символа:
если символы в строке будут напечатаны в функцию \cdf{read} (с необходимым, где
надо, экранированием), они будут интерпретированы как ссылка на этот символ
(если он интернирован). Если символ выводится куда-либо, \cdf{print} печатает
его выводимое имя.
Для подробностей, смотрите раздел о \emph{читателе} (раздел~\ref{READER}) и
\emph{писателе} (раздел~\ref{PRINTER}).

\begin{defun}[Function]
symbol-name sym

This returns the print name of the symbol \emph{sym}.
For example:
\begin{lisp}
(symbol-name 'xyz) \EV\ "XYZ"
\end{lisp}
It is an extremely bad idea to modify a string being used as the print name of
a symbol.  Such a modification may tremendously confuse
the function \cdf{read} and the package system.

Данная функция возвращает выводимое имя для символа \emph{sym}.
Например:
\begin{lisp}
(symbol-name 'xyz) \EV\ "XYZ"
\end{lisp}
Это не очень хорошая идея, изменять строку, которая используется, как выводимое
имя символа. Такое изменение может сильно запутать функцию \cdf{read} и систему
пакетов.

\begin{newer}
X3J13 voted in March 1989 \issue{CHARACTER-PROPOSAL}
to specify that it is an error to modify a string being used
as the print name of a symbol.
\end{newer}

\end{defun}

\section{Creating Symbols Создание символов}

Symbols can be used in two rather different ways.
An \emph{interned} symbol is one that is indexed by its print name
in a catalogue called a \emph{package}.
A request to locate a symbol with that print name results
in the same (\cdf{eq}) symbol.  Every time input is read with the
function \cdf{read},
and that print name appears, it is read as the same symbol.
This property of symbols makes them appropriate to use as names for
things and as hooks on which to hang permanent data objects
(using the property list, for example).

Символы могут использоваться двумя различными способами.
\emph{Интернированный} символ один из них, который определяется его выводимым
именем и каталогом, называемым \emph{пакет}.
A request to locate a symbol with that print name results
in the same (\cdf{eq}) symbol.  Every time input is read with the
function \cdf{read},
and that print name appears, it is read as the same symbol.
This property of symbols makes them appropriate to use as names for
things and as hooks on which to hang permanent data objects
(using the property list, for example).

Interned symbols are normally created automatically; the first time
something (such as the function \cdf{read})
asks the package system for a symbol with a given print name,
that symbol is automatically created.  The function used to ask for
an interned symbol is \cdf{intern}, or one of the functions
related to \cdf{intern}.

Интернированные символы обычно создаются автоматически. Во время первого
обращения к символу в системе пакетов (с помощью \cdf{read}, например), данный
символ создается автоматически. Для обращения к
интернированному символу используется функция \cdf{inter}, или другая с ней
связанная.

Although interned symbols are the most commonly
used, they will not be discussed further here.  For more information,
see chapter~\ref{XPACK}.

Несмотря на то, что интернированные символы используются чаще всего, они не
будут больше здесь рассматриваться. Для подробной информации смотрите
главу~\ref{XPACK}.

An \emph{uninterned} symbol is a symbol used simply as a data object,
with no special cataloguing (it belongs to no particular package).
An uninterned symbol is printed as \cd{\#:} followed by its
print name.
The following are some functions for creating uninterned symbols.

\emph{Неинтернированный} символ является символом, используемым в качестве
объекта данных, без связи с каталогом (он не имеет родительского пакета).
Неинтернированный символ выводится как \cd{\#:} с последующим выводимым именем.


\begin{defun}[Function]
make-symbol print-name

\cd{(make-symbol \emph{print-name})} creates a new uninterned symbol, whose
print name is the string \emph{print-name}.  The value and function bindings will
be unbound and the property list will be empty.

The string actually installed in the symbol's print-name component
may be the given string \emph{print-name} or may be a copy of it,
at the implementation's discretion.  The user should not assume
that \cd{(symbol-name (make-symbol x))} is \cdf{eq} to \cdf{x}, but also
should not alter a string once it has been given as an argument
to \cdf{make-symbol}.


\cd{(make-symbol \emph{print-name})} создает новый неинтернированный символ, у
которого выводимое имя является строкой \emph{print-name}. Полученный символ не
имеет значения функции (unbound) и пустой список свойств.

Строка, переданная в аргументе, может использоваться сразу, либо сначала
копироваться. Это зависит от реализации.
Пользователь не может полагаться на то, что \cd{(symbol-name
  (make-symbol x))} равно \cdf{eq} \cdf{x}, но и также не может изменять строку
переданную в качестве аргумента для \cdf{make-symbol}.

\beforenoterule
\begin{implementation}
An implementation might choose, for example,
to copy the string to some read-only area, in the expectation that
it will never be altered.
\end{implementation}
\afternoterule
\end{defun}

\begin{defun}[Function]
copy-symbol sym &optional copy-props

This returns a new uninterned symbol with the same print name
as \emph{sym}.

Эта функция возвращает новый неинтернированный символ с тем же выводимым именем,
что \emph{sym}.

\begin{newer}
X3J13 voted in March 1989 \issue{COPY-SYMBOL-PRINT-NAME}
that the print name of the new symbol is required to be
the same only in the sense of \cdf{string=}; in other words,
an implementation is permitted (but not required)
to make a copy of the print name.
User programs should not assume that the print names of the old and new symbols
will be \cdf{eq}, although they may happen to be \cdf{eq} in some implementations.
\end{newer}


If \emph{copy-props} is non-{\nil}, then the initial
value and function definition of the new symbol will
be the same as those of \emph{sym}, and the property list of
the new symbol will be a copy of \emph{sym}'s.

Если \emph{copy-props} не-{\nil}, тогда начальное значение и определение функции
нового символа будут те же, что и в переданном \emph{sym}, а список свойств будет
скопирован из исходного символа.

\begin{newer}
X3J13 voted in March 1989 \issue{COPY-SYMBOL-COPY-PLIST}
to clarify that only the top-level conses of the
property list are copied; it is as if \cd{(copy-list (symbol-plist \emph{sym}))}
were used as the property list of the new symbol.
\end{newer}

If \emph{copy-props}
is {\nil} (the default), then the new symbol will be unbound and undefined, and
its property list will be empty.

Если \emph{copy-props} является {\nil} (по-умолчанию), тогда новый символ не
будет связан, не будет иметь определения функции, и список свойств будет пуст.
\end{defun}

\begin{defun}[Function]
gensym &optional x

\cdf{gensym} invents a print name and creates a new symbol with that print name.
It returns the new, uninterned symbol.

The invented print name consists of a prefix
(which defaults to \cdf{G}), followed by the decimal representation of a number.

\cdf{gensym} создает выводимое имя и создает новый символ с этим именем.
Она возвращает новый неинтернированный символ.

Созданное имя содержит префикс (по-умолчанию \cd{G}), с последующим десятичным
представлением числа.
\begin{obsolete}
The number is increased by 1 every time \cdf{gensym} is called.

If the argument \emph{x} is present and is an integer, then \emph{x} must be
non-negative, and the internal counter is set to \emph{x} for future use;
otherwise the internal counter is incremented.
If \emph{x} is a string, then that string is made the default prefix
for this and future calls to \cdf{gensym}.
After handling the argument, \cdf{gensym} creates a
symbol as it would with no argument.
For example:
\begin{lisp}
(gensym) \EV\ G7 \\
(gensym "FOO-") \EV\ FOO-8 \\
(gensym 32) \EV\ FOO-32 \\
(gensym) \EV\ FOO-33 \\
(gensym "GARBAGE-") \EV\ GARBAGE-34
\end{lisp}
\end{obsolete}

\cdf{gensym} is usually used to create a symbol that should not normally
be seen by the user and whose print name is unimportant except to
allow easy distinction by eye between two such symbols.
The optional argument is rarely supplied.
The name comes from ``generate symbol,'' and the symbols produced by it
are often called ``gensyms.''

\cdf{gensym} обычно используется для создания символа, который не виден
пользователю, и его имя не имеет важности.
Необязательный аргумент используется нечасто. Имя образовано от <<генерация
символа>>, и символы созданные, таким образом, часто называются <<gensyms>>.

\beforenoterule
\begin{incompatibility}
In earlier versions of Lisp, such as MacLisp and
Interlisp, the print name of a gensym was of fixed length, consisting
of a single letter and a fixed-length
decimal representation with leading zeros
if necessary, for example, \cd{G0007}.  This convention was
motivated by an implementation consideration, namely that the name
should fit into a single machine word, allowing a quick and clever
implementation.  Such considerations are less relevant in Common Lisp.
The consistent use of mnemonic prefixes can make it easier
for the programmer, when debugging, to determine what code generated
a particular symbol.  The elimination of the fixed-length decimal
representation prevents the same name from being used twice
unless the counter is explicitly reset.
\end{incompatibility}
\afternoterule

If it is desirable
for the generated symbols to be interned, and yet guaranteed to be
symbols distinct from all others,
then the function \cdf{gentemp}
may be more appropriate to use.

Если необходимо, чтобы сгенерированные символы были интернированными и отличными
от существующих символов, тогда удобно использовать функцию \cdf{gentemp}.

\begin{newer}
X3J13 voted in March 1989 \issue{GENSYM-NAME-STICKINESS}
to alter the specification of \cdf{gensym} so that supplying an
optional argument (whether a string or a number) does \emph{not} alter
the internal state maintained by \cdf{gensym}.
Instead, the internal
counter is made explicitly available as a variable named \cd{*gensym-counter*}.

If a string argument is given to \cdf{gensym}, that string is used as the prefix;
otherwise ``\cdf{G}'' is used.  If a number is provided, its decimal
representation is used, but the internal counter is unaffected.
X3J13 deprecates the use of a number as an argument.
\end{newer}
\end{defun}

\begin{newer}
\begin{defun}[Variable]
*gensym-counter*

X3J13 voted in March 1989 \issue{GENSYM-NAME-STICKINESS}
to add \cd{*gensym-counter*}, which
holds the state of the \cdf{gensym} counter; that is, \cdf{gensym}
uses the decimal representation of its value as part of the generated name
and then increments its value.

The initial value of this variable is implementation-dependent
but will be a non-negative integer.

The user may assign to or bind this variable at any time, but its value
must always be a non-negative integer.

Переменная хранит состояние счетчика для функции \cdf{gensym}. \cdf{gensym}
использует десятичное представление этого значения в качестве части имени
генерируемого символа, а затем наращивает этот счетчик.

Первоначальное значение этой переменной зависит от реализации, но должно быть
неотрицательным целым.

Пользователь в любое время может присваивать или связывать это переменную, но
значение должно быть неотрицательным целым.
\end{defun}
\end{newer}

\begin{defun}[Function]
gentemp &optional prefix package

\cdf{gentemp}, like \cdf{gensym}, creates and returns a new symbol.
\cdf{gentemp} differs from \cdf{gensym} in that it interns the symbol
(see \cdf{intern}) in the \emph{package} (which defaults to the current
package; see \cdf{*package*}).  \cdf{gentemp} guarantees the symbol
will be a new one not already existing in the package.  It does this
by using a counter as \cdf{gensym} does, but if the generated symbol
is not really new, then the process is repeated until a new one is created.
There is no provision for resetting the \cdf{gentemp} counter.
Also, the prefix for \cdf{gentemp} is not remembered from one call
to the next; if \emph{prefix} is omitted, the default prefix \cdf{T} is used.

\cdf{gentemp}, как и \cdf{gensym}, создает и возвращает новый символ.
\cdf{gentemp} отличается от \cdf{gensym} в том, что возвращает интернированный
символ (смотрите \cdf{intern}) в пакете \emph{package} (который по-умолчанию
является текущим, смотрите \cdf{*package*}). \cdf{gentemp} гарантирует, что
символ будет новым, и не существовал ранее в указанном пакете. Она также
использует счетчик, однако если полученный символ уже существует счетчик
наращивается, и действия повторяются, пока не будет найдено имя еще не
существующего символа.
Сбросить счетчик невозможно.
Кроме того, префикс для \cdf{gentemp} не сохраняется между вызовами. Если
аргумент \emph{prefix} опущен, то используется значение по-умолчанию \cdf{T}.
\end{defun}

\begin{defun}[Function]
symbol-package sym

Given a symbol \emph{sym}, \cdf{symbol-package} returns the contents of the
package cell of that symbol.  This will be a package object or {\nil}.

Для заданного символа возвращает содержимое ячейки пакета. Результат может быть
объектом пакета или {\nil}.
\end{defun}

\begin{defun}[Function]
keywordp object

The argument may be any Lisp object.  The predicate \cdf{keywordp} is true
if the argument is a symbol and that
symbol is a keyword (that is, belongs to the keyword
package).  Keywords are those symbols that are written with
a leading colon.  Every keyword is a constant, in the sense
that it always evaluates to itself.  See \cdf{constantp}.

Аргумент может быть любым Lisp'овым объектом. Предикат \cdf{keywordp} истинен,
если аргумент является символом и этот символ является ключевым (значит, что
принадлежит пакату ключевых символов). Ключевые символы --- это символы, которые
записываются с двоеточием в начала. Каждый ключевой символ является константой,
в смысле, что выполняются сами в себя. Смотрите \cdf{constantp}.
\end{defun}
      % Functions on symbols
%Part{XPACK, Root = "CLM.MSS"}
%Chapter of Common Lisp Manual.  Copyright 1984, 1988, 1989 Guy L. Steele Jr.

\clearpage\def\pagestatus{ULTIMATE}

\chapter{Packages Пакеты}
\label{XPACK}

One problem with earlier Lisp systems is the use of a single name space
for all symbols.  In large Lisp systems, with modules written by many
different programmers, accidental name collisions become a serious
problem.  Common Lisp addresses this problem through the \emph{package system},
derived from an earlier package system developed for
Lisp Machine Lisp \cite{BLUE-LISPM}.
In addition to preventing name-space conflicts, the
package system makes the modular structure of large Lisp systems more
explicit.

Одна проблемка была с ранними реализациями Lisp систем --- это использование
одного пространства имен для всех символов. В больших Lisp'овых системах, с
модулями написанными разными программистами, случайные совпадения имен стали
серьезной проблемой. Common Lisp решает эту проблему с помощью \emph{системы
  пакетов}, производной от ранней системы пакетов, разработанной для Lisp
Machine \cite{BLUE-LISPM}.
В дополнения к избежанию конфликтов име, система пакетов делает модульную
структуру больших Lisp систем более явной.

A \emph{package} is a data structure that establishes a mapping from print
names (strings) to symbols.  The package thus replaces the ``oblist'' or
``obarray'' machinery of earlier Lisp systems.  At any given time one
package is current, and this package is used by the Lisp reader in
translating strings into symbols.  The current package is, by definition,
the one that is the
value of the global variable \cdf{*package*}.  It is possible to refer to
symbols in packages other than the current one through the use of
\emph{package qualifiers} in the printed representation of the symbol.
For example, \cd{foo:bar}, when seen by the reader,
refers to the symbol whose name is
\cdf{bar} in the package whose name is \cdf{foo}.
(Actually, this is true only if \cdf{bar} is an external symbol of \cdf{foo},
that is, a symbol that is supposed to be visible outside of \cdf{foo}.
A reference to an internal symbol requires the intentionally
clumsier syntax \cd{foo::bar}.)

\emph{Пакет} --- это структура данных, которая устанавливает связь между
выводимыми именами (строкой) и символами. Таким образом пакет заменяет
<<oblist>> или <<obarray>> из ранних версий Lisp'овых систем. В любое время
только один пакет является текущим, и этот пакет используется Lisp'овым
читателем при преобразовании строк в символы. Текущий пакет является, по
определению, значение глобальной переменной \cdf{*package*}. С помощью
\emph{имени пакета} в выводимом имени символа существует возможность
ссылаться на символы других, а не текущего, пакетов.
Например, когда \cd{foo:bar} будет прочтен Common Lisp'ом, то будет ссылаться на
символ \cd{bar} из пакета \cd{foo}.
(А точнее, это верно только, если \cd{bar} является экспортированным символом из
\cd{foo}, то есть символом, который является видимым извне \cd{foo}. Ссылка на
внутренний символ требует удваивания двоеточего: \cd{foo::bar}.)

The string-to-symbol mappings available in a given package are divided
into two classes, \emph{external} and \emph{internal}.  We refer to the
symbols accessible via these mappings as being \emph{external} and
\emph{internal} symbols of the package in question, though really it is the
mappings that are different and not the symbols themselves.  Within a
given package, a name refers to one symbol or to none; if it does refer
to a symbol, then it is either external or internal in that
package, but not both.

Отображение строк в символы доступное в данном пакете делиться на два вида:
\emph{внешнее} и \emph{внутреннее}. Речь идет о символах доступных с помощью
этих отображений, как о \emph{внешних} и \emph{внутренних} символах пакета, хотя
на самом деле различаются отображения, а не символы.
Внутри заданного пакета, имя ссылается на один или ноль символов. Если оно
ссылается на символ, тогда этот символ в данном пакете внешний или внутренний в,
но не одновременно. 

External symbols are part of the package's public interface to other
packages.  External symbols are supposed to be chosen with some care and are
advertised to users of the package.  Internal symbols are for internal
use only, and these symbols are normally hidden from other packages.
Most symbols are created as internal symbols; they become external only
if they appear explicitly in an \cdf{export} command for the package.

Внешние символы являются частью интерфейса пакета доступного в других
пакетах. Внешние символы должны быть выбраны с особой тщательностью и
публиковаться для пользователей этого пакета. Внутренние символы
предусмотрены только для внутреннего использования, и эти символы обычно скрыты
от других пакетов. Большинство символов создаются как внутренние. Они становятся
внешними только, если явно передаются в команду \cdf{export}.

A symbol may appear in many packages.  It will always have the same
name wherever it appears, but it may be external in some packages
and internal in others.  On the other hand,
the same name (string) may refer to different symbols in
different packages.

Символ может встречаться во многих пакетах. Он будет всегда иметь одно и то же
имя, но в некоторых пакетах может быть внешним, а в других внутренним. С другой
стороны, одно и то же имя (строка) может ссылаться на различные символы в
различных пакетах.

Normally, a symbol that appears in one or more packages
will be \emph{owned} by one particular package, called the \emph{home package}
of the symbol; that package is said to \emph{own} the symbol.
Every symbol has a component called the \emph{package cell}
that contains a pointer to its home package.
A symbol that is owned by some package is said to be \emph{interned}.
Some symbols are not owned by any package; such a symbol
is said to be \emph{uninterned}, and its package cell contains {\false}.

Обычно, символ, который встречается в одном и более пакетах, будет иметь только
один родительский пакет, называемый \emph{домашний пакет} символа. Говорится,
что такой пакет владеет символом.
Все символы содержат компонент, называется \emph{ячейка пакета}, которая хранит
указатель на домашний пакет.
Символ, который имеет домашний пакет, называется \emph{интернированным}.
Некоторые символы не имеют домашнего пакета. Они называются
неинтернированными. Их ячейка пакета содержит значение {\false}.

Packages may be built up in layers.  From the point of view of a
package's user, the package is a single collection of mappings from
strings into internal and external symbols.  However, some of these
mappings may be established within the package itself, while other
mappings are inherited from other packages via the \cdf{use-package}
construct.  (The mechanisms responsible for this inheritance are
described below.)  In what follows, we will refer to a symbol as being
\emph{accessible} in a package if it can be referred to
without a package qualifier when that package is current,
regardless of whether the mapping occurs within
that package or via inheritance.   We will refer to a symbol as being
\emph{present} in a package if the mapping is in the package itself and is
not inherited from somewhere else.  Thus a symbol present in a package is accessible,
but an accessible symbol is not necessarily present.

Пакеты могут быть представлены как слои. С этой точки зрения, для пользователя,
пакет выглядит как коллекция отображений из строк во внутренние и внешние
символы. Однако, некоторые из этих отображений могут устанавливаться внутри
самого пакета, тогда как другие отображения наследуются из других пакетов с
помощью конструкции \cdf{use-package}. (Механизм такого наследования описан
ниже.) В дальнейшем, мы будем называть символ, \emph{доступным} в пакете, если
на него можно сослаться без указания квалификатора пакета, вне зависимости от
того унаследовано ли отображение символов, или установлено текущим пакетом. И мы
будем называть символ \emph{родным} в пакете, если отображение установлено
самим пакетом, а не унаследовано. Таким образом настоящий символ в пакете
является доступным, но доступный символ не обязательно является настоящим.

A symbol is said to be \emph{interned in a package} if it is
accessible in that package and also is owned (by either that package
or some other package).  Normally all the symbols accessible in
a package will in fact be owned by some package,
but the terminology is useful when
discussing the pathological case of an accessible but unowned (uninterned)
symbol.

Символ называется \emph{интернированным в пакет}, если он доступен в этом
пакете и имеет некоторый родительский пакет. Обычно все символы
доступные в пакете будут фактически в собственности некоторого пакета, но такая
терминология полезна, когда описываются патологические случаи доступности
никому не принадлежащего (неинтернированного) символа.

As a verb, to \emph{intern} a symbol in a package means to cause the
symbol to be interned in the package if it was not already;
this process is performed by the function \cdf{intern}.
If the symbol was previously unowned, then the package it is being
interned in becomes its owner (home package); but
if the symbol was previously owned by another package, that other package
continues to own the symbol.

Как глагол, \emph{интернировать} символ в пакет, означает сделать так, чтобы
пакет стал владельцем символа, если этого было не так.
Этот процесс выполняется функцией \cdf{intern}.
Если символ прежде был бесхозным (никому не принадлежал), тогда пакет становится
его владельцем (домашним пакетом). Но если символ уже принадлежал кому-то, то
принадлежность не меняется.

To \emph{unintern} a symbol from the package means to cause it to be not
present in the package
and, additionally, to cause the symbol to be uninterned if the
package was the home package (owner) of the symbol.
This process is performed by the function \cdf{unintern}.

\emph{Дезинтернировать} символ из пакета означает убрать владение символом
данным пакетом, при условии, что он был домашним. Данный процесс выполняется
функцией \cdf{unintern}.

\section{Consistency Rules Правила согласования}

Package-related bugs can be very subtle and confusing: things are not
what they appear to be.  The Common Lisp package system is designed with
a number of safety features to prevent most of the common bugs that
would otherwise occur in normal use.  This may seem over-protective, but
experience with earlier package systems has shown that such safety
features are needed.

Ошибки связанные с пакетами могут быть очень тонкими и запутанными. Система
пакетов Common Lisp'а спроектирована с рядом безопасных функций для
предотвращения большинства распространенных ошибок, которые могли бы быть при
обычном использовании. Может показаться, что защита используется излишне,
однако опыт предыдущих систем пакетов показал, что такие меры необходимы.

In dealing with the package system, it is useful to keep in mind the
following consistency rules, which remain in force as long as the value
of \cdf{*package*} is not changed by the user:

При работе с системой пакетов, полезно держать в памяти следующие правила,
которые остаются в силе пока пользователи не изменил значение \cdf{*package*}:

\begin{itemize}
\item
\emph{Read-read consistency:} Reading the same print name always results in
the same (\cdf{eq}) symbol.

\emph{Согласованность чтения-чтения:} Чтение одного и того же имени приводит к
одному и тому же символу (\cdf{eq} вернет истину).

\item
\emph{Print-read consistency:} An interned symbol always prints as a
sequence of characters that, when read back in, yields the same (\cdf{eq})
symbol.

\emph{Согласованность вывода-чтения:} Интернированный символ всегда выводится,
как последовательность строковых символов, которые при повторном чтении дают
исходный символ (\cdf{eq} вернет истину).

\item
\emph{Print-print consistency:} If two interned symbols are not \cdf{eq}, then
their printed representations will be different sequences of
characters.

\emph{Print-print consistency:} Если два интернированных символа не равны
\cdf{eq}, тогда их печатаемые отображения будут различными последовательностями
строковых символов.
\end{itemize}

These consistency rules remain true in spite of any amount of implicit
interning caused by typing in Lisp forms, loading files, etc.  This has
the important implication that, as long as the current package
is not changed, results are reproducible regardless of
the order of loading files or the exact history of what symbols were
typed in when.  The rules can only be violated by explicit action:
changing the value of \cdf{*package*}, forcing some action by continuing
from an error, or calling one of the ``dangerous'' functions
\cdf{unintern}, \cdf{unexport}, \cdf{shadow},
\cdf{shadowing-import}, or \cdf{unuse-package}.

Эти правила согласованности остаются в силе несмотря на любое количество неявных
случаев интернирования в Lisp'овых формах, загрузках файлов и так далее. Важное
значение заключается в том, что пока текущий пакет не меняется, результаты
воспроизводимы вне зависимости от порядка загрузки файлов или истории вводимых
символов. Правила могут быть нарушены только явным действием: изменением
значения \cdf{*package*}, продолжением выполнения после ошибки, или вызовом
одной из <<опасных>> функций \cdf{unintern}, \cdf{unexport}, \cdf{shadow},
\cdf{shadowing-import} или \cdf{unuse-package}. 

\section{Package Names Имена пакетов}
\label{PACKAGE-NAMES-SECTION}

Each package has a name (a string) and perhaps some nicknames.  These
are assigned when the package is created, though they can be changed
later.  A package's name should be something long and self-explanatory,
like \cdf{editor}; there might be a nickname that is shorter and easier to
type, such as \cdf{ed}.

Каждый пакет имеет имя (строку) и, возможно, несколько псевдонимов. Они
присваиваются во время, когда пакет создается, однако, могут быть изменены
позднее. Имя пакета должно быть длинным и информативным, например
\cdf{editor}. Псевдоним должен быть коротким и простым в написании, например
\cdf{ed}.

There is a single name space for packages.  The function
\cdf{find-package} translates a package name or nickname into the
associated package.  The function \cdf{package-name} returns the name of a
package.  The function \cdf{package-nicknames} returns a list of all
nicknames for a package.  The function \cdf{rename-package} removes a
package's current name and nicknames and replaces them with new ones
specified by the user.  Package renaming is occasionally useful when, for
development purposes, it is desirable to load two versions of a package
into the same Lisp.  One can load the first version, rename it,
and then load the other version, without getting a lot of name conflicts.

Для имен пакетов существует только одно пространство имен. Функция
\cdf{find-package} транслирует имя или псевдоним пакета в связанный объект
пакета. Функция \cdf{package-name} возвращает имя пакета. Функция
\cdf{package-nicknames} возвращает список всех псевдонимов для пакета. Функция
\cdf{rename-package} удаляет текущее имя пакета и псевдонимы и заменяет их на
указанные пользователем. Переименование пакета изредка бывает полезным,
например, для разработки, когда необходимо загрузить две версии одного пакета в
Lisp систему. Можно загрузить первую версию, переименовать ее, и затем загрузить
другую верси, без разрешения конфликтов имен.

When the Lisp reader sees a qualified symbol, it handles the package-name
part in the same way as the symbol part with respect to capitalization.
Lowercase characters in the package name are converted to corresponding
uppercase characters
unless preceded by the escape character \cd{{\Xbackslash}} or
surrounded by \cd{|} characters.  The lookup done by the
\cdf{find-package} function is case-sensitive, like that done for
symbols.  Note that \cd{|Foo|:|Bar|} refers to a symbol whose name is
\cdf{Bar} in a package whose name is \cdf{Foo}.  By contrast,
\cd{|Foo:Bar|} refers to a seven-character symbol that has a colon in its name
(as well as two uppercase letters and four lowercase letters)
and is interned in the current package.  Following the convention used
in this book for symbols, we show ordinary package names using
lowercase letters, even though the name string is internally represented
with uppercase letters.

Когда Lisp считыватель встречают полное имя символа, он обрабатывает часть для
имени пакета, также как и часть для имени символа, возводя все неэкранированные
строковые имволы в верхний регистр. Экранирование строковых символов
производится с помощью символов \cd{{\Xbackslash}} или \cd{|}. Поиск,
осуществляемый функцией \cdf{find-package}, является регистрозависимым, также
как и для символов. Следует отметить, что \cd{|Foo|:|Bar|} ссылается на символ,
имя которого \cd{Bar}, в пакете \cd{Foo}. Для сравнения \cd{|Foo:Bar|} ссылается
на семизначный символ, имя которого содержит доветочие (а также две заглавные и
четыре прописные буквы) и интернирован в текущий пакет. В данной книге символы и
пакеты указываются без экранирования строковыми символами только в нижнем
регистре, при этом внутри Lisp машины они будут переведены в верхний регистр.

Most of the functions that require a package-name argument from the
user accept either a symbol or a string.  If a symbol is supplied,
its print name will be used; the print name will already have undergone
case-conversion by the usual rules.  If a string is supplied, it
must be so capitalized as to match exactly the
string that names the package.

Большинство функции, которые принимают имя пакета, могут принимать или символ,
или строку. Если указан символ, то используется его выводимое имя, которое
подвергается обычным преобразованиям в верхний регистр. Если указана строка, она
должна быть преобразована для полного совпадения с именем пакета.

\begin{new}
X3J13 voted in January 1989
\issue{PACKAGE-FUNCTION-CONSISTENCY}
to clarify that one may use either a package object or
a package name (symbol or string) in any of the following
situations:
\begin{itemize}
\item the \cd{:use} argument to \cdf{make-package}
\item the first argument to \cdf{package-use-list}, \cdf{package-used-by-list},
\cdf{package-name}, \cdf{package-nicknames},
\cdf{in-package}, \cdf{find-package},
\cdf{rename-package}, or \cdf{delete-package},


\item the second argument to \cdf{intern}, \cdf{find-symbol},
\cdf{unintern}, \cdf{export}, \cdf{unexport}, \cdf{import}, \cdf{shadowing-import},
or \cdf{shadow}
\item the first argument, or a member of the list that is the first argument,
to \cdf{use-package} or \cdf{unuse-package}
\item the value of the \emph{package} given to \cdf{do-symbols},
\cdf{do-external-symbols}, or \cdf{do-all-symbols}
\item a member of the \emph{package-list} given to \cdf{with-package-iterator}
\end{itemize}
Note that the first argument to \cdf{make-package} must still be a package
name and not an actual package; it makes no sense to create an already
existing package.  Similarly, package nicknames must always be expressed
as package names and not as package objects.  If \cdf{find-package} is
given a package object instead of a name, it simply returns that package.
\end{new}

\section{Translating Strings to Symbols Преобразование строк в символы}
\label{STRING-TO-SYMBOL-SECTION}

The value of the special variable \cdf{*package*} must always be a package
object (not a name).  Whatever package object is currently the
value of \cdf{*package*} is referred to as the \emph{current package}.

Значение специальной переменной \cdf{*package*} должно всегда быть объектом
пакета (не именем). Вне зависимости от объекта пакета в \cdf{*package*}, он
называется \emph{текущим пакетом}.

When the Lisp reader has, by parsing, obtained a string of characters
thought to name a symbol, that name is looked up in the current package.
This lookup may involve looking in other packages whose external symbols
are inherited by the current package.  If the name is found,
the corresponding symbol is returned.  If the name is not found
(that is, there is no symbol of that name accessible in the current package),
a new symbol is created for it and is placed in the current package as an
internal symbol.  Moreover, the current package becomes the owner
(home package) of the symbol, and so the symbol becomes interned
in the current package.
If the name is later read again while this same package is
current, the same symbol will then be found and returned.

Когда Lisp'овый считыватель получает строку для символа, он ищет его имя в
текущем пакете.
Данный поиск может привести к поиску в других пакетах, экспортированные символы
которых унаследованы текущим пакетом. Если имя найдено, то возвращается
соответствующий символ. Если имя не найдено (то есть, в текущем пакете не
существует соответствующего доступного символа), то создается новый символ и
помещается в текущий пакет. Если точнее, то текущий пакет становится владельцем
(домашним пакетом) символа.
Если это имя будет прочитано еще раз позже и в этом же пакете, то этот уже
созданный символ будет возвращен.

Often it is desirable to refer to an external symbol in some package
other than the current one.  This is done through the use of a
\emph{qualified name}, consisting of a package name, then a colon, then the
name of the symbol.  This causes the symbol's name to be looked up
in the specified package, rather than in the current one.  For example,
\cd{editor:buffer} refers to the external symbol named \cdf{buffer}
accessible in the package named \cdf{editor}, regardless of whether
there is a symbol named \cdf{buffer} in the current package.  If there
is no package named \cdf{editor}, or if no symbol named \cdf{buffer}
is accessible in \cdf{editor}, or if \cdf{buffer} is an internal
symbol in \cdf{editor}, the Lisp reader will signal
a correctable error to ask the user for instructions.

Часто необходимо сослаться на внешний символ в некотором другом, нетекущем
пакете. Это может быть сделано с помощью \emph{полного имени}, включающем имя
пакета, затем двоеточение, и наконец имя символа. Это приводит к поиску символа
в указанном, а не текущем пакете. Например, \cd{editor:buffer} ссылается на
внешний символ с именем \cdf{buffer} доступный из пакета с именем \cdf{editor},
вне зависимости от того, если в текущем пакете символ с таким же именем.
Если пакета с именем \cdf{editor} или символа с именем \cdf{buffer} в указанном
пакете, Lisp'овые считыватель сгенерирует сигнал с ошибкой и возможностью
исправить ошибку.

On rare occasions, a user may need to refer to an \emph{internal} symbol of
some package other than the current one.  It is illegal to do this with
the colon qualifier, since accessing an internal symbol of some other
package is usually a mistake.  However, this operation is legal if
a doubled colon
\cd{::} is used as the separator in place of the usual single colon.  If
\cd{editor::buffer} is seen, the effect is exactly the same as
reading \cdf{buffer} with \cdf{*package*} temporarily rebound to the
package whose name is \cdf{editor}.  This special-purpose qualifier
should be used with caution.

В редких случаях, пользователь может нуждаться в ссылке на \emph{внутренний}
символ некоторого нетекущего пакета. Это нельзя сделать с помощью двоеточего,
так как данная запись позволяет ссылаться только на внешние символы. Однако, это
можно сделать с помощью двойного двоеточего \cd{::}, указанного вместо
одинарного. Если используется \cd{editor::buffer}, то эффект такой же, как если
бы произошла попытка найти символ с именем \cdf{buffer} и \cdf{*package*} была
связана с объетом с именем \cd{editor}. Двойное двоеточие должно использоваться
с осторожностью.

The package named \cdf{keyword} contains all keyword symbols used by the
Lisp system itself and by user-written code.  Such symbols must be
easily accessible from any package, and name conflicts are not an issue
because these symbols are used only as labels and never to carry
package-specific values or properties.  Because keyword symbols are used
so frequently, Common Lisp provides a special reader syntax for them.
Any symbol preceded by a colon but no package name (for example
\cd{:foo}) is added to (or looked up in) the \cdf{keyword} package as
an \emph{external} symbol.  The \cdf{keyword} package is also treated
specially in that whenever a symbol is added to the \cdf{keyword} package
the symbol is always made external; the symbol
is also automatically declared to be a constant
(see \cdf{defconstant}) and made to
have itself as its value.  This is why every keyword evaluates to
itself.  As a matter of style, keywords should always be accessed using
the leading-colon convention; the user should never import or inherit
keywords into any other package.  It is an error to try to apply
\cdf{use-package} to the \cdf{keyword} package.

Пакет с именем \cdf{keyword} содержит все ключевые символы используемые
Lisp'овой системой и пользовательским кодом. Такие символы должны быть легко
доступны из любого пакета, и конфликт имен не является проблемой, так как эти
символы используются только в качестве меток и не содержат значений. Так как
ключевые символы используются часто, то Common Lisp для них предоставляет
специальный синтаксис. Любой символ с двоеточием в начале и без имени пакета
(например \cd{:foo}) добавляется (или ищется) в пакете \cdf{keyword} как
\emph{внешний} символ. Пакет \cdf{keyword} также отличается тем, что при
добавлении в него символа, последний автоматически становится внешним. Символ
также автоматически декларируется как константа (смотрите \cdf{defconstant}) и
его значением становится он сам.
В целях стиля, ключевые символы должны всегда быть доступны с помощью двоеточего
в начале имени. Пользователь никогда не должен импортировать или наследовать
ключевые символы в другие пакета. Попытка использовать \cdf{use-package} для
\cdf{keyword} пакета является ошибкой.

Each symbol contains a package cell that is used to record the home
package of the symbol, or {\false} if the symbol is uninterned.  This cell
may be accessed by using the function \cdf{symbol-package}.
When an interned
symbol is printed, if it is a symbol in the keyword package,
then it is printed with a preceding colon; otherwise, if it is accessible
(directly or by inheritance) in the current package, it is printed
without any qualification; otherwise, it is printed with the name of the
home package as the qualifier, using \cd{:} as the separator if the
symbol is external and \cd{::} if not.

Каждый символ содержит ячейку пакета, которая используется для записи домашнего
пакета символа, или {\false}, если пакет неинтернированный. Эта ячейка доступна
с помощью функции \cdf{symbol-package}.
Когда интернированный символ печатается, если это символ в пакете ключевых
символов, тогда он выводится с двоеточием в начале, иначе, если он доступен
(напрямую или отнаследованно) в текущем пакете, он печатается без имени пакета,
иначе он печатается полностью, с именем пакета, именем символа и \cd{:} в
качестестве разделителя для внешнего символа, и \cd{::} для внутреннего.

A symbol whose package slot contains {\false} (that is, has no home
package)
is printed preceded by \cd{\#:}.  It is possible, by the
use of \cdf{import} and \cdf{unintern}, to create a symbol that has no
recorded home package but that in fact is accessible in some package.
The system does not check for this pathological case, and such symbols
will always be printed preceded by \cd{\#:}.

Символ, у которого слот (ячейка) пакета содержит {\false} (то есть, домашний
пакет отсутствует) печатается с \cd{\#:} вначале. С использованием \cdf{import}
и \cdf{uninter} возможно создать символ, который не имеет домашнего пакета, но
фактически доступен в некоторых пакетах.
Lisp система не проверяет такие патологические случаи, и такие символы будут
всегда печататься с предшевствующими \cd{\#:}.

In summary, the following four uses of symbol qualifier syntax are defined.

В целом, синтаксис имен символов может быть выражен в следующих четырех примерах.

\begin{flushdesc}
\item[\cd{foo:bar}]
When read, looks up \cd{BAR} among the external symbols of
the package named \cd{FOO}.  Printed when symbol \cd{bar} is external in its
home package \cd{foo} and is not accessible in the current package.

При прочтении, выполняется поиск символа \cd{BAR} среди внешних символов пакета
\cd{FOO}. При выводе, когда символ \cd{bar} является внешним в домашнем пакете
\cd{foo} и недоступен в текущем пакете.

\item[\cd{foo::bar}]
When read, interns \cd{BAR} as if \cd{FOO} were the
current package.  Printed when symbol \cd{bar} is internal in its home package
\cd{foo} and is not accessible in the current package.

При прочтении, интернирует символ \cd{BAR}, как если бы пакет \cd{FOO} являлся
текущим. При выводе, когда символ \cd{bar} является внутренним в его домашнем
пакете \cd{foo} и недоступен в текущем пакете.

\item[\cd{:bar}]
When read, interns \cd{BAR} as an external symbol in the
\cd{keyword} package and makes it evaluate to itself.  Printed when
the home package of symbol \cd{bar} is \cd{keyword}.

При прочтении, интернирует \cd{BAR} как внешний символ в пакете \cd{keyword} и
выполняет его самого в себя. При выводе, когда \cd{keyword} является домашним
пакетом для символа.
    
\item[\cd{\#:bar}]
When read, creates a new uninterned symbol named \cd{BAR}.
Printed when the symbol \cd{bar} is uninterned (has no home package),
even in the pathological case that \cd{bar} is uninterned but
nevertheless somehow accessible in the current package.

При прочтении, создает новый неинтернированный символ с именем \cd{BAR}.
При выводе, когда символ \cd{bar} не имеет домашнего пакета.
\end{flushdesc}

All other uses of colons within names of symbols are not defined by
Common Lisp but are reserved for implementation-dependent use; this
includes names that end in a colon, contain two or more colons, or
consist of just a colon.

Все другие использования двоеточего внутри имен символов не определены Common
Lisp'ом, но зарезервированы для реализаций. Сюда включены имена с двоеточием в
конце, или содержащими два и более двоеточих или просто состоящими из
двоеточего.

\section{Exporting and Importing Symbols Экспортирование и импортирование
  символов} 
\label{EXPORT-IMPORT-SECTION}

Symbols from one package may be made accessible in another package in
two ways.

Символы из одного пакета могут стать доступными в другом пакете двумя способами.

First, any individual symbol may be added to a package by use
of the function \cdf{import}.  The form \cd{(import 'editor:buffer)} takes
the external symbol named \cdf{buffer} in the \cdf{editor} package (this
symbol was located when the form was read by the Lisp reader) and adds
it to the current package as an internal symbol.  The symbol is then
present in the current package.  The imported symbol is
not automatically exported from the current package, but if it is
already present and external, then the fact that it
is external is not changed.  After the call to
\cdf{import} it is possible to refer to \cdf{buffer} in the importing package
without any qualifier.  The status of \cdf{buffer} in the package named
\cdf{editor} is unchanged, and \cdf{editor} remains the home package for
this symbol.  Once imported, a symbol is \emph{present} in the
importing package and can be removed only by calling \cdf{unintern}.

Первый способ, каждый отдельный символ может быть добавлен в пакет использованием
функции \cdf{import}. Форма \cd{(import 'editor:buffer)} принимает внешний
символ с именем \cd{buffer} в пакете \cd{editor} (этот символ распознается Lisp
считывателем) и добавляет его в текущий пакет в качестве внутреннего
символа. Символ становится доступным в текущем пакете. Импортированный символ
автоматически не экспортируется из текущего пакета, но если он уже существовал в
пакете и был внешним, то это свойство не меняется. После вызова \cdf{import}
в импортирующем пакете появляется возможность ссылаться на \cdf{buffer} без
указания полного имени. Свойства \cdf{buffer} в пакете \cdf{editor} не меняются,
\cdf{editor} продолжает оставаться домашним пакетом для этого символа. Будучи
импортированным, символ будет присутствовать в пакете и может быть удален только
с помощью вызова \cdf{unintern}.

If the symbol is already present in the importing package, \cdf{import}
has no effect.  If a distinct symbol with the name \cdf{buffer} is
accessible in the importing package (directly or by inheritance), then a
correctable error is signaled, as described in
section~\ref{NAME-CONFLICTS-SECTION}, because \cdf{import} avoids letting
one symbol shadow another.

Если символ уже присутствует в импортирующем пакете, \cdf{import} ничего не
делает. Если другой символ с таким же именем \cdf{buffer} уже доступен в
импортирующем пакете (напрямую или унаследован), тогда сигнализируется ошибка с
возможностью исправления ситуации, как написано в
разделе~\ref{NAME-CONFLICT-SECTION}, потому \cdf{import} не допускает сокрытия
одного символа другим.

A symbol is said to be \emph{shadowed} by another symbol in
some package if the first symbol would be accessible by inheritance
if not for the presence of the second symbol.
To import a symbol without the possibility
of getting an
error because of shadowing,
use the function \cdf{shadowing-import}.  This inserts
the symbol into the specified package as an internal symbol, regardless
of whether another symbol of the same name will be shadowed by this
action.
If a different symbol of the same name is already present
in the package, that symbol will first be uninterned from the package
(see \cdf{unintern}).  The new symbol is
added to the package's shadowing-symbols list.  \cdf{shadowing-import}
should be used with caution.  It changes the state of the package system
in such a way that the consistency rules do not hold across the change.

Символ называется скрытым другим символом в некотором пакете. Это когда первый
символ был бы доступен через наследование, если бы не присутствие второго
символа.
Для импортирования символа без ошибки скрытия, используйте функцию
\cdf{shadowing-import}. Она вставляет символ в указанный пакет, как внутренний
символ, вне зависимости от того, происходит ли скрытие другого символа с тем же
именем.
Если другой символ с тем же именем присутствовал в пакете, тогда этот символ
сначала удаляется из пакета \cdf{unintern}. Новый символ добавляется в список
скрывающих символов FIXME. \cdf{shadowing-import} должна использоваться
аккуратно. Она изменяем состояние системы пакетов так, что правила согласования
могут перестать работать.

The second mechanism is provided by the function \cdf{use-package}.  This
causes a package to inherit all of the external symbols of some other
package.  These symbols become accessible as \emph{internal} symbols of the
using package.  That is, they can be referred to without a qualifier
while this package is current, but they are not passed along to any
other package that uses this package.  Note that \cdf{use-package},
unlike \cdf{import}, does not cause any new symbols to be \emph{present}
in the current package but only makes them \emph{accessible} by inheritance.
\cdf{use-package} checks carefully for
name conflicts between the newly imported symbols and those already
accessible in the importing package.  This is described in detail in
section~\ref{NAME-CONFLICTS-SECTION}.

Второй способ предоставляется функцией \cdf{use-package}. Эта функция делает
так, что пакет наследует все внешние символы некоторого другого пакета. Эти
символы становятся доступными, как \emph{внутренние} символы, используемого
пакета. То есть, на них можно ссылаться без указаная пакета внутри текущего
пакета, но они не становятся доступными в других пакетах, использующих
данный. Следует отметить, что \cdf{use-package}, в отличие от \cdf{import}, не
делает новые символы \emph{родственными} в текущем пакете, а делает их только
\emph{доступными} с помощью наследования отображения символов.
\cdf{use-package} проверяет конфликты имен между импортируемыми и уже доступными
символами в импортирующем пакете. Это подробнее описано в
разделе~\ref{NAME-CONFLICT-SECTION}.

Typically a user, working by default in the \cdf{user} package, will
load a number of packages into Lisp to provide an augmented working
environment, and then call \cdf{use-package} on each of these packages
to allow easy access to their external symbols.
\cdf{unuse-package} undoes the effects of a previous \cdf{use-package}.  The
external symbols of the used package are no longer inherited.  However,
any symbols that have been imported into the using package continue to
be present in that package.

Обычно пользователь, по-умолчанию работая в пакете \cdf{user}, будет загружать
ряд пакетов в Lisp систему для предоставления расширенного рабочего окружения, и
затем вызывать \cdf{use-package} для каждого из этих пакетов для простого
доступа к их внешним символами.
\cdf{unuse-package} производит обратные действия относительно
\cdf{use-package}. Внешние символы используемые пакетом престают
наследоваться. Однако, любые импортированные символы, остаются доступными.

There is no way to inherit the \emph{internal} symbols of another package;
to refer to an internal symbol, the user must either make that symbol's home
package current, use a qualifier, or import that symbol into the current
package.

Несуществует способа наследовать \emph{внутренние} символы другого пакета. Для
ссылки на внутренний символ, пользователь должен поменять домашний пакет для
данного символа на текущий, использовать полное имя (вместе с пакетом) или
импортировать этот символ в текущий пакет.

\begin{newer}
The distinction between
external and internal symbols is a primary means of hiding names
so that one program does not tread on the namespace of another.

Различие между внешними и внутренними символами прежде всего означает скрытие
имен, так чтобы одна программа не могла использовать пространство имен другой
программы.
\end{newer}

When \cdf{intern} or some other function wants to look up a symbol in a
given package, it first looks for the symbol among the external and
internal symbols of the package itself; then it looks through the
external symbols of the used packages in some unspecified order.  The
order does not matter; according to the rules for handling name
conflicts (see below), if conflicting symbols appear in two or more
packages inherited by package \emph{X}, a symbol of this name must also appear
in \emph{X} itself as a shadowing symbol.  Of course, implementations are free
to choose other, more efficient ways of implementing this search, as
long as the user-visible behavior is equivalent to what is described
here.

Когда \cdf{intern} или некоторые другие функций хотят найти символ в заданном
пакете, они сначала ищут символ среди внешних и внутренних символов текущего
пакета, затем они в неопределенном порядке ищут среди внешних символов
используемых пакетов. Порядок не имеет значение. В соответствии с правилами
разрешения конфликтов имен (смотрите ниже), если конфликтующие символы существуют
в двух и более пакетах, унаследованных пакетом \emph{X}, символ с этим именем
должен также быть в \emph{X}, как скрытый символ. Конечно, реализации могут
выбрать другой, более эффективный способ для реализации такого поиска, не
изменяя поведение ранее описанного интерфейса.

The function \cdf{export} takes a symbol that is accessible in some
specified package (directly or by inheritance) and makes it an external
symbol of that package.  If the symbol is already accessible as an
external symbol in the package, \cdf{export} has no effect.  If the symbol
is directly present in the package as an internal symbol, it is simply
changed to external status.  If it is accessible as an internal symbol
via \cdf{use-package}, the symbol is first imported into the package, then
exported.  (The symbol is then present in the specified package
whether or not the package
continues to use the package through which the symbol was originally
inherited.)  If the symbol is not
accessible at all in the specified package, a correctable error is
signaled that, upon continuing, asks the user whether the symbol
should be imported.

Функция \cdf{export} принимает символ, который доступен в некотором указанном
пакете (напрямую или унаслеован из другого пакета) и делает его внешним символом
этого пакета. Если символ уже доступен как внешний, \cdf{export} ничего не
делает. Если символ представлен, как внутренний \emph{родственный} символ, его
статус просто меняется на внешний. Если он доступен, как внутренний символ,
полученный с помощью \cdf{use-package}, символ сначала импортируется в пакет, а
затем делается внешним. (После этого символ становиться импортированным в пакет
и остается в нем, вне зависимости от того будет ли использована
\cdf{unuse-package} или нет). Если символ вообще недоступен в указанном пакете,
то сигнализируется ошибка с возможностью решить проблему, а, именно, какой
символ должен быть импортирован.

The function \cdf{unexport} is provided mainly as a way to undo erroneous
calls to \cdf{export}.  It works only on symbols directly present
in the current package, switching them back to internal status.  If
\cdf{unexport} is given a symbol already accessible as an internal
symbol in the current package, it does nothing; if it is given a symbol
not accessible in the package at all, it signals an error.

Функция \cdf{unexport} предоставляет как способ откатить ошибочное
экспортирование символа. Она работает только для символов напрямую
представленных в текущем пакете, меняя их свойство на <<внутреннее>>.
Если \cdf{unexport} получает символ, который в текущем пакете уже является
внутренним, она ничего не делает. Если получает вообще недоступный символ, то
сигнализирует ошибку.

\section{Name Conflicts Конфликты имен}
\label{NAME-CONFLICTS-SECTION}

A fundamental invariant of the package system is that within one package
any particular name can refer to at most one symbol.  A \emph{name conflict}
is said to occur when there is more than one candidate symbol and it is
not obvious which one to choose.  If the system does not always choose
the same way, the read-read consistency rule would be violated.  For
example, some programs or data might have been read in under a certain
mapping of the name to a symbol.  If the mapping changes to a different
symbol, and subsequently additional programs or data are read,
then the two programs will
not access the same symbol even though they use the same name.  Even if
the system did always choose the same way, a name conflict is likely to
result in a mapping from names to symbols different from what was expected by
the user, causing programs to execute incorrectly.  Therefore, any time
a name conflict is about to occur,
an error is signaled.  The user may continue
from the error and tell the package system how to resolve the conflict.

It may be that the same symbol is accessible to a package through more than
one path.  For example, the symbol might be
an external symbol of more than one used package, or the symbol
might be directly present in a package and also inherited from
another package.
In such cases there is no name conflict.
The same identical symbol cannot conflict with itself.
Name conflicts occur only between distinct symbols with
the same name.

The creator of a package can tell the system in advance how to resolve a
name conflict through the use of \emph{shadowing}.  Every package has a
list of shadowing symbols.  A shadowing symbol takes precedence over any
other symbol of the same name that would otherwise be accessible to the
package.  A name conflict involving a shadowing symbol is always
resolved in favor of the shadowing symbol, without signaling an error
(except for one instance involving \cdf{import} described below).  The
functions \cdf{shadow} and \cdf{shadowing-import} may be used to declare
shadowing symbols.

Name conflicts are detected when they become possible, that is, when the
package structure is altered.  There is no need to check for name
conflicts during every name lookup.

The functions \cdf{use-package}, \cdf{import}, and \cdf{export} check for name
conflicts.  \cdf{use-package} makes the external symbols of the package
being used accessible to the using package; each of these symbols is
checked for name conflicts with the symbols already accessible.
\cdf{import} adds a single symbol to the internals of a package, checking
for a name conflict with an existing symbol either present in the
package or accessible to it.  \cdf{import} signals a name conflict error
even if the conflict is with a shadowing symbol, the rationale being
that the user has given two explicit and inconsistent directives.
\cdf{export} makes a single
symbol accessible to all the packages that use the package from which
the symbol is exported.  All of these packages are checked for
name conflicts:  \cd{(export \emph{s} \emph{p})} does
\cd{(find-symbol (symbol-name \emph{s}) \emph{q})} for each package \emph{q}
in \cd{(package-used-by-list \emph{p})}.  Note that in the usual case of
an \cdf{export} during the initial definition of a package, the
result of \cdf{package-used-by-list}
will be {\false} and the name-conflict checking
will take negligible time.

The function \cdf{intern}, which is the one used most
frequently by the Lisp reader for looking up names of symbols,
does not need to do any name-conflict checking, because it
never creates a new symbol if there is already an accessible symbol with
the name given.

\cdf{shadow} and \cdf{shadowing-import} never signal a name-conflict error
because the user, by calling these functions, has specified how any
possible conflict is to be resolved.  \cdf{shadow} does name-conflict
checking to the extent that it checks whether a distinct existing symbol with
the specified name is accessible and, if so, whether it is directly
present in the package or inherited.  In the latter case, a new symbol
is created to shadow it.  \cdf{shadowing-import} does name-conflict
checking to the extent that it checks whether a distinct existing
symbol with the same name is accessible; if so, it is shadowed by
the new symbol, which implies that it must be uninterned
if it was directly present in the package.

\cdf{unuse-package}, \cdf{unexport}, and \cdf{unintern} (when the symbol being
uninterned is not a shadowing symbol) do not need to do any
name-conflict checking because they only remove symbols from a package;
they do not make any new symbols accessible.

Giving a shadowing symbol to \cdf{unintern} can uncover a name conflict that had
previously been resolved by the shadowing.  If package A uses packages
B and C, A contains a shadowing symbol \cdf{x}, and B and C each contain external
symbols named \cdf{x}, then removing the shadowing symbol \cdf{x}
from A will reveal a name
conflict between \cd{b:x} and \cd{c:x} if those two symbols are distinct.
In this case \cdf{unintern} will signal an error.

Aborting from a name-conflict error leaves the original symbol accessible.
Package functions always signal name-conflict errors before making any
change to the package structure.  When multiple changes are to be made,
however, for example when \cdf{export} is given a list of symbols, it is
permissible for the implementation to process each change separately,
so that aborting from a name
conflict caused by the second symbol in the list will not unexport the
first symbol in the list.  However, aborting from a name-conflict error
caused by \cdf{export} of a single symbol will not leave that symbol accessible
to some packages and inaccessible to others; with respect to
each symbol processed, \cdf{export}
behaves as if it were an atomic operation.

Continuing from a name-conflict error should offer the user a chance to
resolve the name conflict in favor of either of the candidates.  The
package structure should be altered to reflect the resolution of the
name conflict, via \cdf{shadowing-import}, \cdf{unintern}, or \cdf{unexport}.

A name conflict in \cdf{use-package} between a symbol directly present in the
using package and an external symbol of the used package may be resolved
in favor of the first symbol by making it a shadowing symbol, or in favor
of the second symbol by uninterning the first symbol from the using
package.  The latter resolution is dangerous if the symbol to be
uninterned is an external symbol of the using package, since it
will cease to be an external symbol.

A name conflict in \cdf{use-package} between two external symbols inherited
by the using package from other packages may be resolved in favor of
either symbol by importing it into the using package and making it a
shadowing symbol.

A name conflict in \cdf{export} between the symbol being exported and a
symbol already present in a package that would inherit the
newly exported symbol may be resolved in favor of the exported symbol
by uninterning the other one, or in favor of the already-present
symbol by making it a shadowing symbol.

A name conflict in \cdf{export} or \cdf{unintern} due to a package
inheriting two distinct symbols with the same name from two other
packages may be resolved in favor of either symbol by importing it into
the using package and making it a shadowing symbol, just as with
\cdf{use-package}.

A name conflict in \cdf{import} between the symbol being imported and a
symbol inherited from some other package may be resolved in favor of the
symbol being imported by making it a shadowing symbol, or in favor
of the symbol already accessible by not doing the \cdf{import}.  A
name conflict in \cdf{import} with a symbol already present in the
package may be resolved by uninterning that symbol, or by not
doing the \cdf{import}.

Good user-interface style dictates that \cdf{use-package} and \cdf{export},
which can cause many name conflicts simultaneously, first check for
all of the name conflicts before presenting any of them to the user.
The user may then choose to resolve all of them wholesale or to resolve
each of them individually, the latter requiring a lot of
interaction but permitting
different conflicts to be resolved different ways.

Implementations may offer other ways of resolving name conflicts.
For instance, if the symbols that conflict are not being used as
objects but only as names for functions, it may be possible to ``merge''
the two symbols by putting the function definition onto both symbols.
References to either symbol for purposes of calling a function would be
equivalent.  A similar merging operation can be done for variable values
and for things stored on the property list.  In Lisp Machine Lisp, for example, one can
also \emph{forward} the value, function, and property cells so that future
changes to either symbol will propagate to the other one.  Some other
implementations are able to do this with value cells but not with
property lists.  Only the user can know whether this way of resolving
a name conflict is adequate, because it will work only if
the use of two non-\cdf{eq}
symbols with the same name will not prevent the correct operation of
the program.  The value of offering symbol merging as a way of resolving
name conflicts is that it can avoid the need to throw away the whole
Lisp world, correct the package-definition forms
that caused the error, and start over from scratch.

\section{Built-in Packages Встроенные пакеты}

\begin{obsolete}
\noindent
The following packages, at least, are built into every Common Lisp system.

\begin{flushdesc}
\item[\cdf{lisp}]
The package named \cdf{lisp} contains the primitives of the
Common Lisp system.  Its external symbols include all of the
user-visible functions and global variables that are present in the
Common Lisp system, such as \cdf{car}, \cdf{cdr}, and \cdf{*package*}.
Almost all other packages will want to use \cdf{lisp} so that these
symbols will be accessible without qualification.

\item[\cdf{user}]
The \cdf{user} package is, by default, the current package at the time
a Common Lisp system starts up.  This package uses the \cdf{lisp} package.
\end{flushdesc}
\end{obsolete}

\begin{newer}
X3J13 voted in March 1989 \issue{LISP-PACKAGE-NAME} to specify that
the forthcoming ANSI Common Lisp will use the package name \cdf{common-lisp}
instead of \cdf{lisp} and the package name \cdf{common-lisp-user}
instead of \cdf{user}.  The purpose is to allow a single Lisp system
to support both ``old'' Common Lisp and ``new'' ANSI Common Lisp
simultaneously despite the fact that in some cases
the two languages use the same
names for incompatible purposes.  (That's what packages are for!)

\begin{flushdesc}
\item[\cdf{common-lisp}]
The package named \cdf{common-lisp} contains the primitives of the
ANSI Common Lisp system (as opposed to a Common Lisp system based
on the 1984 specification).  Its external symbols include all of the
user-visible functions and global variables that are present in the
ANSI Common Lisp system, such as \cdf{car}, \cdf{cdr}, and \cdf{*package*}.
Note, however, that the home package of such symbols is not
necessarily the \cdf{common-lisp} package (this makes it easier for
symbols such as \cdf{t} and \cdf{lambda} to be shared between
the \cdf{common-lisp} package and another package, possibly one named \cdf{lisp}).
Almost all other packages ought to use \cdf{common-lisp} so that these
symbols will be accessible without qualification.
This package has the nickname \cdf{cl}.

\item[\cdf{common-lisp-user}]
The \cdf{common-lisp-user} package is, by default,
the current package at the time an ANSI Common Lisp system starts up.
This package uses the \cdf{common-lisp} package
and has the nickname \cdf{cl-user}.
It may contain other implementation-dependent symbols
and may use other implementation-specific packages.

Пакет \cdf{common-lisp-ise} является, по-умолчанию, текущим пакетом во время
запуска ANSI Common Lisp системы.
Этот пакет использует пакет \cdf{common-lisp} и имеет псевдоним \cdf{cl-user}.
В зависимости от реализации он также может содержать другие символы и
использовать платформоспецифичные пакеты.
\end{flushdesc}
\end{newer}

\begin{flushdesc}
\item[\cdf{keyword}]
This package contains all of the keywords used by built-in
or user-defined Lisp functions.  Printed symbol representations
that start with a colon are interpreted as referring to symbols
in this package, which are always external symbols.  All symbols in this
package are treated as constants that evaluate to themselves, so that the
user can type \cd{:foo} instead of \cd{':foo}.

Этот пакет содержит все ключевые символы, используемые встроенными или
пользовательскими Lisp'овыми функциями. Имя символа, начинающегося с двоеточего,
интерпретируется, как ссылка на символ из этого пакета, который всегда является
внешним. Все символы в этом пакете является константами, которые вычисляются
сами в себя, поэтому пользователь может записывать \cd{:foo} вместо \cd{':foo}.
\end{flushdesc}

\begin{obsolete}
\begin{flushdesc}
\item[\cdf{system}]
This package name is reserved to the implementation.
Normally this is used to contain names of implementation-dependent
system-interface functions.  This package uses \cdf{lisp} and has the
nickname \cdf{sys}. 
\end{flushdesc}
\end{obsolete}

\begin{new}
X3J13 voted in January 1989
\issue{PACKAGE-CLUTTER}
to modify the requirements on the built-in packages
so as to limit what may appear in the \cdf{common-lisp} package
and to lift the requirement that every implementation have a package
named \cdf{system}.  The details are as follows.

Not only must the \cdf{common-lisp} package in any given implementation
contain all the external symbols prescribed by the standard;
the \cdf{common-lisp} package moreover may not contain any external symbol
that is not prescribed by the standard.  However, the \cdf{common-lisp}
package may contain additional internal symbols, depending on the
implementation.

An external symbol of the \cdf{common-lisp} package may not have a function,
macro, or special form definition, or a top-level value,
or a \cdf{special} proclamation, or a type definition, unless specifically
permitted by the standard.  Programmers may validly rely on this fact;
for example, \cdf{fboundp} is guaranteed to be false for all
external symbols of the \cdf{common-lisp} package except those explicitly
specified in the standard to name functions, macros, and special forms.
Similarly, \cdf{boundp} will be false of all such external symbols
except those documented to be variables or constants.

Portable programs
may use external symbols in the \cdf{common-lisp} package that are not documented
to be constants or variables as names of local lexical
variables with the presumption that the implementation has not
proclaimed such variables to be special; this legitimizes the common
practice of using such names as \cdf{list} and \cdf{string} as names
for local variables.

A valid implementation may initially have properties on any symbol,
or dynamically put new properties on symbols (even user-created symbols),
as long as no property indicator used for this purpose is
an external symbol of any package defined by the standard
or a symbol that is accessible from the \cdf{common-lisp-user} package or any
package defined by the user.

This vote eliminates the requirement that every implementation have
a predefined package named \cdf{system}.  An implementation may
provide any number of predefined packages; these should be described
in the documentation for that implementation.

The \cdf{common-lisp-user} package may contain symbols not described by the standard
and may use other implementation-specific packages.
\end{new}

\begin{newer}
X3J13 voted in March 1989 \issue{LISP-SYMBOL-REDEFINITION}
to restrict user programs from performing certain actions that
might interfere with built-in facilities or interact badly
with them.
Except where explicitly allowed, the consequences are undefined if any
of the following actions are performed on a symbol in the \cdf{common-lisp}
package.
\begin{itemize}
\item binding or altering its value (lexically or dynamically)
\item defining or binding it as a function
\item defining or binding it as a macro
\item defining it as a type specifier (\cdf{defstruct}, \cdf{defclass}, \cdf{deftype})
\item defining it as a structure (\cdf{defstruct})
\item defining it as a declaration
\item dsing it as a symbol macro
\item altering its print name
\item altering its package
\item tracing it
\item declaring or proclaiming it special or lexical
\item declaring or proclaiming its \cdf{type} or \cdf{ftype}
\item removing it from the package \cdf{common-lisp}
\end{itemize}
X3J13 also voted in June 1989 \issue{DEFINE-COMPILER-MACRO}
to add to this list the item
\begin{itemize}
\item defining it as a compiler macro
\end{itemize}

If such a symbol is not globally defined as a variable or a constant,
a user program is allowed to lexically bind it and declare the \cdf{type} of that binding.

If such a symbol is not defined as a function, macro, or special form,
a user program is allowed to (lexically) bind it as a function and to declare the
\cdf{ftype} of that binding and to trace that binding.

If such a symbol is not defined as a function, macro, or special form,
a user program is allowed to (lexically) bind it as a macro.

As an example, the behavior of the code fragment
\begin{lisp}
(flet ((open (filename \&key direction) \\*
~~~~~~~~~(format t "{\Xtilde}\%OPEN was called.")  \\*
~~~~~~~~~(open filename :direction direction))) \\*
~~(with-open-file (x "frob" :direction ':output)  \\*
~~~~(format t "{\Xtilde}\%Was OPEN called?")))
\end{lisp}
is undefined.  Even in a ``reasonable'' implementation,
for example, the macro expansion of \cdf{with-open-file} might refer
to the \cdf{open} function and might not.  However, the preceding rules eliminate
the burden of deciding whether an implementation is reasonable. The code
fragment violates the rules; officially its behavior is therefore
completely undefined, and that's that.

Note that ``altering the property list'' is not in the list of
proscribed actions, so a user program \emph{is} permitted to
add properties to or remove properties from
symbols in the \cdf{common-lisp} package.
\end{newer}

\section{Package System Functions and Variables Функции и переменные для системы
пакетов}
\label{PACKAGE-FUNCTIONS-SECTION}

Some of the functions and variables in this section
are described in previous sections but are included here
for completeness.

Некоторые из функций и переменных в этом разделе были описаны в предыдущих
разделах, но включены сюда для завершенности.

\begin{obsolete}
It is up to each implementation's compiler to ensure that when a
compiled file is loaded, all of the symbols in the file end up in the
same packages that they would occupy if the Lisp source file were
loaded.  In most compilers, this will be accomplished by treating
certain package operations as though they are surrounded by
\cd{(eval-when (compile load eval) ...)}; see \cdf{eval-when}.
These operations are
\cdf{make-package}, \cdf{in-package}, \cdf{shadow}, \cdf{shadowing-import},
\cdf{export}, \cdf{unexport}, \cdf{use-package}, \cdf{unuse-package}, and \cdf{import}.
To guarantee proper compilation in all Common Lisp
implementations, these functions should appear only at top level within
a file.  As a matter of style, it is suggested that each file contain
only one package, and that all of the package setup forms appear near
the start of the file.  This is discussed in more detail, with examples,
in section~\ref{PACKAGE-EXAMPLE-SECTION}.
\end{obsolete}

\begin{newer}
X3J13 voted in March 1989 \issue{IN-PACKAGE-FUNCTIONALITY}
to cancel the specifications of the preceding paragraph
in order to support a model of file compilation in which the
compiler need never take special note of ordinary function calls;
only special forms and macros are recognized as affecting the state
of the compilation process.
As part of this change \cdf{in-package} was changed to be a macro
rather than a function and its functionality was restricted.
The actions of
\cdf{shadow}, \cdf{shadowing-import}, \cdf{use-package},
\cdf{import}, \cdf{intern}, and \cdf{export} for compilation
purposes may be accomplished with the new macro \cdf{defpackage}.
\end{newer}

\beforenoterule
\begin{implementation}
In the past, some Lisp compilers have read
the entire file into Lisp before processing any of the forms.  
Other compilers have arranged for
the loader to do all of its intern operations before evaluating any of the
top-level forms.  Neither of these techniques will work in a
straightforward way in Common Lisp because of the presence of multiple
packages.
\end{implementation}
\afternoterule

For the functions described here, all optional arguments named
\emph{package} default to the current value of \cdf{*package*}.  Where a
function takes an argument that is either a symbol or a list of symbols,
an argument of {\false} is treated as an empty list of symbols.  Any
argument described as a package name may be either a string or a symbol.
If a symbol is supplied, its print name will be used as the package
name; if a string is supplied, the user must take care to specify the
same capitalization used in the package name, normally all uppercase.

Для описанных здесь функций, все необязательные аргументы с именем
\emph{package} имеют зачение по-умолчанию \cdf{*package*}. Там, где функция
принимает аргумент, который может быть символом или списком символов, значение
{\false} расценивается, как пустой список символов. Любой аргумент, описанный
как имя пакета, может быть символом или строкой.
Если указан символ, то используется его выводимое имя. Если строка, то
пользователь должен позаботиться о преобразовании регистра символов в верхний
там, где это необходимо.

\begin{defun}[Variable]
*package*

The value of this variable must be a package; this package is said to be
the current package.  The initial value of \cdf{*package*} is the \cdf{user}
package.

Значение этой переменной должно быть объектом пакета. Этот пакет называется
текущим. Первоначальное значение \cdf{*package*} является пакетом \cdf{common-lisp-user}.
\begin{newer}
X3J13 voted in March 1989 \issue{LISP-PACKAGE-NAME} to specify that
the forthcoming ANSI Common Lisp will use the package name \cdf{common-lisp-user}
instead of \cdf{user}.
\end{newer}

The function \cdf{load} rebinds \cdf{*package*} to its current value.  If
some form in the file changes the value of \cdf{*package*} during loading,
the old value will be restored when the loading is completed.

Функция \cdf{load} пересвязывает \cdf{*package*} в текущее значение. Если
некоторая форма в файле во время загрузки изменяет значение \cdf{*package*},
старое значение будет восстановлено после завершения загрузки.

\begin{newer}
X3J13 voted in October 1988 \issue{COMPILE-FILE-PACKAGE}
to require \cdf{compile-file} to rebind \cdf{*package*} to its current value.
\end{newer}
\end{defun}

\begin{defun}[Function]
make-package package-name &key :nicknames :use

This creates and returns a new package with the specified package name.  As
described above, this argument may be either a string or a symbol.  The
\cd{:nicknames} argument must be a list of strings to be used as
alternative names for the package.  Once again, the user may supply
symbols in place of the strings, in which case the print names of the
symbols are used.  These names and nicknames must not conflict with
any existing package names; if they do, a correctable error is
signaled.

The \cd{:use} argument is a list of packages or the names (strings or
symbols) of packages whose external symbols are to be inherited by the
new package.  These packages must already exist.  If not supplied,
\cd{:use} defaults to a list of one package, the \cdf{lisp} package.

Это функция создает и возвращает новый пакет с указанным именем. Как было
описано выше, аргументом может быть или символ, или строка. Аргумент
\cd{:nicknames} должен быть списком строк, которые будут псевдонимами. И здесь
пользователь вместо строк может указывать символы, в случае которых будут
использоваться выводимые имена. Это имя и псевдонимы не могут конфликтовать с
уже имеющимися именами пакетов. Если конфликт произошел, сигнализируется ошибка
с возможностью исправления.

\begin{newer}
X3J13 voted in March 1989 \issue{LISP-PACKAGE-NAME} to specify that
the forthcoming ANSI Common Lisp will use the package name \cdf{common-lisp}
instead of \cdf{lisp}.
\end{newer}

\begin{new}
X3J13 voted in January 1989
\issue{MAKE-PACKAGE-USE-DEFAULT}
to change the specification of \cdf{make-package} so that the default value
for the \cd{:use} argument is unspecified.  Portable code should
specify \cd{:use~'("COMMON-LISP")} explicitly.

\beforenoterule
\begin{rationale}
Many existing implementations of Common Lisp happen to have violated
the first edition specification, providing as the default not only
the \cdf{lisp} package but also (or instead) a package containing
implementation-dependent language extensions.
This is for good reason: usually it is much
more convenient to the user for the default \cd{:use} list to be
the entire, implementation-dependent, extended language rather
than only the facilities specified in this book.  The X3J13 vote
simply legitimizes existing practice.
\end{rationale}
\afternoterule
\end{new}
\end{defun}


\begin{obsolete}
\begin{defun}[Function]
in-package package-name &key :nicknames :use

The \cdf{in-package} function is intended to be placed at the start of a
file containing a subsystem that is to be loaded into some package other
than \cdf{user}.

If there is not already a package named \emph{package-name}, this
function is similar to \cdf{make-package}, except that after the
new package is created, \cdf{*package*} is set to it.  This binding will
remain in force until changed by the user (perhaps with another
\cdf{in-package} call) or until the \cdf{*package*} variable reverts to its
old value at the completion of a \cdf{load} operation.

If there is an existing package whose name is \emph{package-name}, the
assumption is that the user is re-loading a file after making some
changes.  The existing package is augmented to reflect any new nicknames
or new packages in the \cd{:use} list (with the usual error checking), and
\cdf{*package*} is then set to this package.
\end{defun}
\end{obsolete}

\begin{new}
X3J13 voted in January 1989
\issue{RETURN-VALUES-UNSPECIFIED}
to specify that \cdf{in-package} returns the new package, that is, the
value of \cdf{*package*} after the operation has been executed.
\end{new}

\begin{newer}
X3J13 voted in March 1989 \issue{LISP-PACKAGE-NAME} to specify that
the forthcoming ANSI Common Lisp will use the package name \cdf{common-lisp-user}
instead of \cdf{user}.
\end{newer}

\begin{newer}
X3J13 voted in March 1989 \issue{IN-PACKAGE-FUNCTIONALITY}
to restrict the functionality of \cdf{in-package} and to make it a macro.
This is an incompatible change.

    Making \cdf{in-package} a macro rather than a function means that there
    is no need to require \cdf{compile-file} to handle it specially.  Since
    \cdf{defpackage} is also defined to have side
    effects on the compilation environment,
    there is no need to require any of the package functions to be treated
    specially by the compiler.

\begin{defmac}
in-package name

This macro causes \cdf{*package*} to be set to the package named \emph{name},
    which must be a symbol or string.  The \emph{name} is not evaluated.
    An error is signaled if the
    package does not already exist.  Everything this macro does is also
    performed at compile time if the call appears at top level.

Этот макрос пересвязывает переменную \cdf{*package*} с пакетом, имя которого
указано в параметре \emph{name}. Параметр \emph{name} может быть строкой или
символом. Форма \emph{name} не вычисляется.
В случае отсутствия пакета, сигнализируется ошибка.
Кроме того, в случае вызова в качестве формы верхнего уровня, этот макрос
работает и во время компиляции.
\end{defmac}
\end{newer}


\begin{defun}[Function]
find-package name

The \emph{name} must be a string that is the name or nickname for a
package.  This argument may also be a symbol, in which case the symbol's
print name is used.  The package with that name or nickname
is returned; if no such package exists, \cdf{find-package} returns {\false}.
The matching of names observes case (as in \cdf{string=}).

Параметр \emph{name} должен быть строкой, которая является именем или
псевдонимом искомого пакета. Этот параметр может также быть символов, в случае
которого используется выводимое имя. В результате возвращается объект пакета с
указанным именем или псевдонимом. Если пакет найден не был, то
\cdf{find-package} возвращает {\false}.
Сравнение имен регистрозависимо (как в \cdf{string=}).

\begin{new}
X3J13 voted in January 1989
\issue{PACKAGE-FUNCTION-CONSISTENCY}
to allow \cdf{find-package} to accept a package object, in which case
the package is simply returned (see section~\ref{PACKAGE-NAMES-SECTION}).
\end{new}
\end{defun}

\begin{defun}[Function]
package-name package

The argument must be a package.  This function returns the string that
names that package.

Аргумент должен быть объектом пакета. Данная функция возвращает строку имени
указанного пакета.

\begin{new}
X3J13 voted in January 1989
\issue{PACKAGE-FUNCTION-CONSISTENCY}
to allow \cdf{package-name} to accept a package name or nickname, in which case
the primary name of the package so specified is returned
(see section~\ref{PACKAGE-NAMES-SECTION}).
\end{new}

\begin{new}
X3J13 voted in January 1989
\issue{PACKAGE-DELETION}
to add a function to delete packages.
One consequence of this vote is that \cdf{package-name}
will return \cdf{nil} instead of a package name if applied
to a deleted package object.  See \cdf{delete-package}.
\end{new}
\end{defun}

\begin{defun}[Function]
package-nicknames package

The argument must be a package.  This function returns the list of
nickname strings for that package, not including the primary name.

Аргумент должен быть объектом пакета. Эта функция возвращает список псевдонимов
для заданного пакета, не включая главное имя.
\begin{new}
X3J13 voted in January 1989
\issue{PACKAGE-FUNCTION-CONSISTENCY}
to allow \cdf{package-nicknames} to accept a package name or nickname,
in which case the nicknames of the package so specified are returned
(see section~\ref{PACKAGE-NAMES-SECTION}).
\end{new}
\end{defun}

\begin{defun}[Function]
rename-package package new-name &optional new-nicknames

The old name and all of the old nicknames of \emph{package} are eliminated
and are replaced by \emph{new-name} and \emph{new-nicknames}.  The
\emph{new-name} argument is a string or symbol; the \emph{new-nicknames}
argument, which defaults to {\false}, is a list of strings or symbols.

Старое имя и все старые псевдонимы пакета \emph{package} удаляются и заменяются
на \emph{new-name} и \emph{new-nicknames}. Аргумент \emph{new-name} может быть
строкой или символом. Аргумент \emph{new-nicknames}, который по-умолчанию
{\false}, является списком строк или символов.
\begin{new}
X3J13 voted in January 1989
\issue{PACKAGE-FUNCTION-CONSISTENCY}
to clarify that the \emph{package} argument may be either a package object
or a package name (see section~\ref{PACKAGE-NAMES-SECTION}).
\end{new}

\begin{new}
X3J13 voted in January 1989
\issue{RETURN-VALUES-UNSPECIFIED}
to specify that \cdf{rename-package} returns \emph{package}.
\end{new}
\end{defun}

\begin{defun}[Function]
package-use-list package

A list of other packages used by the argument package is returned.

Данная функция возвращает список пакетов, используемых указанным в параметре пакетом.
\begin{new}
X3J13 voted in January 1989
\issue{PACKAGE-FUNCTION-CONSISTENCY}
to clarify that the \emph{package} argument may be either a package object
or a package name (see section~\ref{PACKAGE-NAMES-SECTION}).
\end{new}
\end{defun}

\begin{defun}[Function]
package-used-by-list package

A list of other packages that use the argument package is returned.

Данная функция возвращает список пакетов, использующих указанный в параметре пакет.
\begin{new}
X3J13 voted in January 1989
\issue{PACKAGE-FUNCTION-CONSISTENCY}
to clarify that the \emph{package} argument may be either a package object
or a package name (see section~\ref{PACKAGE-NAMES-SECTION}).
\end{new}
\end{defun}

\begin{defun}[Function]
package-shadowing-symbols package

A list is returned of symbols that have been declared as shadowing
symbols in this package by \cdf{shadow} or \cdf{shadowing-import}.  All
symbols on this list are present in the specified package.

Данная функция возвращает список символов, которые были задекларированы, как
скрывающие символы, с помощью \cdf{shadow} или \cdf{shadowing-import}. Все
символы в этом списке является \emph{родственными} указанному пакету.

\begin{new}
X3J13 voted in January 1989
\issue{PACKAGE-FUNCTION-CONSISTENCY}
to clarify that the \emph{package} argument may be either a package object
or a package name (see section~\ref{PACKAGE-NAMES-SECTION}).
\end{new}
\end{defun}

\begin{defun}[Function]
list-all-packages 

This function returns a list of all packages that currently exist in the
Lisp system.

Эта функция возвращает список всех пакетов, которые существуют в Lisp'овой
системе. 
\end{defun}


\begin{new}
\begin{defun}[Function]
delete-package package

X3J13 voted in January 1989
\issue{PACKAGE-DELETION}
to add the \cdf{delete-package} function, which
deletes the specified \emph{package} from all package system data structures.
The \emph{package} argument may be either a package or the name of a package.

If \emph{package} is a name but there is currently no package of that name,
a correctable error is signaled.  Continuing from the error makes
no deletion attempt but merely returns \cdf{nil} from the call to
\cdf{delete-package}.

If \emph{package} is a package object that has already been deleted,
no error is signaled and no deletion is attempted; instead,
\cdf{delete-package} immediately returns \cdf{nil}.

If the package specified for deletion is currently used by other packages,
a correctable error is signaled.  Continuing from this error,
the effect of the function \cdf{unuse-package} is performed on all
such other packages so as to remove their dependency on the
specified package, after which \cdf{delete-package} proceeds to
delete the specified package as if no other package had been using it.

If any symbol had the specified package as its home package before
the call to \cdf{delete-package}, then its home package is unspecified
(that is, the contents of its package cell are unspecified)
after the \cdf{delete-package} operation has been completed.
Symbols in the deleted package are not modified in any other way.

The name and nicknames of the \emph{package} cease to be recognized package
names.  The package object is still a package, but anonymous; \cdf{packagep} will
be true of it, but \cdf{package-name} applied to it will return \cdf{nil}.

The effect of any other package operation on a deleted package object
is undefined.  In particular, an attempt to locate a symbol within a
deleted package (using \cdf{intern} or \cdf{find-symbol}, for example)
will have unspecified results.

\cdf{delete-package} returns \cdf{t} if the deletion succeeds,
and \cdf{nil} otherwise.
\end{defun}
\end{new}

\begin{defun}[Function]
intern string &optional package

The \emph{package}, which defaults to the current package, is
searched for a symbol with the name specified by the \emph{string}
argument.  This search will include inherited symbols, as described
in section~\ref{EXPORT-IMPORT-SECTION}.
If a symbol with the specified name is found, it is returned.
If no such symbol is found, one is created and is installed in the
specified package as an internal symbol (as an external symbol
if the package is the \cdf{keyword} package); the specified package becomes the
home package of the created symbol.


\begin{newer}
X3J13 voted in March 1989 \issue{CHARACTER-PROPOSAL}
to specify that \cdf{intern} may in effect perform the
search using a copy of the argument string in which
some or all of the implementation-defined
attributes have been removed from the characters of the string.
It is implementation-dependent which attributes are removed.
\end{newer}

Two values are returned.  The first is the symbol that was found or
created.  The second value is {\false} if no pre-existing symbol was found,
and takes on one of three values if a symbol was found:
\begin{indentdesc}{6pc}
\item[\cd{:internal}]
The symbol was directly present in the package as an internal symbol.

\item[\cd{:external}]
The symbol was directly present as an external symbol.

\item[\cd{:inherited}]
The symbol was inherited via \cdf{use-package} (which
implies that the symbol is internal).
\end{indentdesc}

\begin{new}
X3J13 voted in January 1989
\issue{PACKAGE-FUNCTION-CONSISTENCY}
to clarify that the \emph{package} argument may be either a package object
or a package name (see section~\ref{PACKAGE-NAMES-SECTION}).
\end{new}

\beforenoterule
\begin{incompatibility}
Conceptually, \cdf{intern} translates a
string to a symbol.  In MacLisp and several other dialects of Lisp,
\cdf{intern} can take either a string or a symbol as its argument; in the 
latter case, the symbol's print name is extracted and used as the string.  
However, this leads to some confusing issues about what to do if
\cdf{intern} finds a symbol that is not \cdf{eq} to the argument symbol.  To
avoid such confusion, Common Lisp requires the argument to be a string.
\end{incompatibility}
\afternoterule
\end{defun}

\begin{defun}[Function]
find-symbol string &optional package

This is identical to \cdf{intern}, but it never creates a new symbol.  If
a symbol with the specified name is found in the specified package,
directly or by inheritance, the symbol found is returned as the first
value and the second value is as specified for \cdf{intern}.  If the
symbol is not accessible in the specified package, both values are
{\false}.

\begin{new}
X3J13 voted in January 1989
\issue{PACKAGE-FUNCTION-CONSISTENCY}
to clarify that the \emph{package} argument may be either a package object
or a package name (see section~\ref{PACKAGE-NAMES-SECTION}).
\end{new}
\end{defun}

\begin{defun}[Function]
unintern symbol &optional package

If the specified symbol is present in the specified \emph{package}, it is
removed from that package and also from the package's shadowing-symbols
list if it is present there.  Moreover, if the \emph{package} is the home
package for the symbol, the symbol is made to have no home package.
Note that in some circumstances the symbol may continue to be accessible
in the specified package by inheritance.
\cdf{unintern} returns {\true} if it actually removed a symbol,
and {\false} otherwise.

\cdf{unintern} should be used with caution.  It changes the state of the
package system in such a way that the consistency rules do not hold
across the change.

\begin{new}
X3J13 voted in January 1989
\issue{PACKAGE-FUNCTION-CONSISTENCY}
to clarify that the \emph{package} argument may be either a package object
or a package name (see section~\ref{PACKAGE-NAMES-SECTION}).
\end{new}

\beforenoterule
\begin{incompatibility}
The equivalent of this in MacLisp is \cdf{remob}.
\end{incompatibility}
\afternoterule
\end{defun}

\begin{defun}[Function]
export symbols &optional package

The \emph{symbols} argument should be a list of symbols, or possibly a single
symbol.  These symbols become accessible as external symbols in
\emph{package} (see section~\ref{EXPORT-IMPORT-SECTION}).
\cdf{export} returns {\true}.

By convention, a call to \cdf{export} listing all exported symbols is
placed near the start of a file to advertise which of the symbols
mentioned in the file are intended to be used by other programs.

\begin{new}
X3J13 voted in January 1989
\issue{PACKAGE-FUNCTION-CONSISTENCY}
to clarify that the \emph{package} argument may be either a package object
or a package name (see section~\ref{PACKAGE-NAMES-SECTION}).
\end{new}
\end{defun}

\begin{defun}[Function]
unexport symbols &optional package

The argument should be a list of symbols, or possibly a single symbol.
These symbols become internal symbols in \emph{package}.
It is an error to unexport a symbol from the \cdf{keyword} package
(see section~\ref{EXPORT-IMPORT-SECTION}).
\cdf{unexport} returns {\true}.

\begin{new}
X3J13 voted in January 1989
\issue{PACKAGE-FUNCTION-CONSISTENCY}
to clarify that the \emph{package} argument may be either a package object
or a package name (see section~\ref{PACKAGE-NAMES-SECTION}).
\end{new}
\end{defun}

\begin{defun}[Function]
import symbols &optional package

The argument should be a list of symbols, or possibly a single symbol.
These symbols become internal symbols in \emph{package} and can therefore
be referred to without having to use qualified-name (colon) syntax.
\cdf{import} signals a
correctable error if any of the imported symbols has the same name as
some distinct symbol already accessible in the package
(see section~\ref{EXPORT-IMPORT-SECTION}).
\cdf{import} returns {\true}.

\begin{newer}
X3J13 voted in June 1987 \issue{IMPORT-SETF-SYMBOL-PACKAGE}
to clarify that if any symbol to be imported has no home package
then \cdf{import} sets the home package of the symbol to the
\emph{package} to which the symbol is being imported.
\end{newer}

\begin{new}
X3J13 voted in January 1989
\issue{PACKAGE-FUNCTION-CONSISTENCY}
to clarify that the \emph{package} argument may be either a package object
or a package name (see section~\ref{PACKAGE-NAMES-SECTION}).
\end{new}
\end{defun}

\begin{defun}[Function]
shadowing-import symbols &optional package

This is like \cdf{import}, but it does not signal an error even if the
importation of a symbol would shadow some symbol already accessible in
the package.  In addition to being imported, the symbol is placed on the
shadowing-symbols list of \emph{package}
(see section~\ref{NAME-CONFLICTS-SECTION}).
\cdf{shadowing-import} returns {\true}.

\cdf{shadowing-import} should be used with
caution.  It changes the state of the package system in such a way that
the consistency rules do not hold across the change.

\begin{new}
X3J13 voted in January 1989
\issue{PACKAGE-FUNCTION-CONSISTENCY}
to clarify that the \emph{package} argument may be either a package object
or a package name (see section~\ref{PACKAGE-NAMES-SECTION}).
\end{new}
\end{defun}

\begin{defun}[Function]
shadow symbols &optional package

The argument should be a list of symbols, or possibly a single symbol.
The print name of each symbol is extracted, and the specified \emph{package} is
searched for a symbol of that name.  If such a symbol is present in this
package (directly, not by inheritance), then nothing is done.  Otherwise,
a new symbol is created with this print name, and it is inserted in the
\emph{package} as an internal symbol.  The symbol is also placed on the
shadowing-symbols list of the \emph{package}
(see section~\ref{NAME-CONFLICTS-SECTION}).
\cdf{shadow} returns {\true}.

\begin{newer}
X3J13 voted in March 1988 \issue{SHADOW-ALREADY-PRESENT}
to change \cdf{shadow} to accept strings as well as well as symbols
(a string in the \emph{symbols} list being treated as a print name),
and to clarify that if a symbol of specified name is already in
the \emph{package} but is not yet on the shadowing-symbols list
for that \emph{package}, then \cdf{shadow} does add it to the shadowing-symbols
list rather than simply doing nothing.
\end{newer}

\cdf{shadow} should be used with
caution.  It changes the state of the package system in such a way that
the consistency rules do not hold across the change.

\begin{new}
X3J13 voted in January 1989
\issue{PACKAGE-FUNCTION-CONSISTENCY}
to clarify that the \emph{package} argument may be either a package object
or a package name (see section~\ref{PACKAGE-NAMES-SECTION}).
\end{new}
\end{defun}

\begin{defun}[Function]
use-package packages-to-use &optional package

The \emph{packages-to-use} argument should be a list of packages or package
names, or possibly a single package or package name.  These packages are
added to the use-list of \emph{package} if they are not there already.  All
external symbols in the packages to use become accessible in \emph{package}
as internal symbols
(see section~\ref{EXPORT-IMPORT-SECTION}).
It is an error to try to use the \cdf{keyword} package.
\cdf{use-package} returns {\true}.

\begin{new}
X3J13 voted in January 1989
\issue{PACKAGE-FUNCTION-CONSISTENCY}
to clarify that the \emph{package} argument may be either a package object
or a package name (see section~\ref{PACKAGE-NAMES-SECTION}).
\end{new}
\end{defun}

\begin{defun}[Function]
unuse-package packages-to-unuse &optional package

The \emph{packages-to-unuse} argument should be a list of packages or
package names, or possibly a single package or package name.  These
packages are removed from the use-list of \emph{package}.
\cdf{unuse-package} returns {\true}.

\begin{new}
X3J13 voted in January 1989
\issue{PACKAGE-FUNCTION-CONSISTENCY}
to clarify that the \emph{package} argument may be either a package object
or a package name (see section~\ref{PACKAGE-NAMES-SECTION}).
\end{new}
\end{defun}

\begin{new}
X3J13 voted in January 1989
\issue{DEFPACKAGE}
to add a macro \cdf{defpackage} to the
language to make it easier to create new packages, alleviating the burden
on the programmer to perform the various setup operations in exactly the
correct sequence.

\begin{defmac}
defpackage defined-package-name {option}*

This creates a new package, or modifies an existing one, whose name
is \emph{defined-package-name}.  The \emph{defined-package-name}
may be a string or a symbol;
if it is a symbol, only its print name matters, and not what package, if any,
the symbol happens to be in.
The newly created or modified package is returned as the value of
the \cdf{defpackage} form.

Each standard \emph{option} is a list of a keyword (the name of the option)
and associated arguments.  No part of a \cdf{defpackage} form is evaluated.
Except for the \cd{:size} option, more than one option of the same
kind may occur within the same \cdf{defpackage} form.

The standard options for \cdf{defpackage} are as follows.
In every case, any option argument called \emph{package-name}
or \emph{symbol-name}
may be a string or a symbol;
if it is a symbol, only its print name matters, and not what package, if any,
the symbol happens to be in.

\begin{flushdesc}
\item[\cd{(:size \emph{integer})}]
This specifies approximately the number of symbols expected to be in the
package.  This is purely an efficiency hint to the storage allocator,
so that implementations using hash tables as part
of the package data structure (the usual technique) will not
have to incrementally expand the package as new symbols are added to it
(for example, as a large file is read while ``in'' that package).

\item[\cd{(:nicknames \Mstar\emph{package-name})}]
The specified names become nicknames of the package being defined.
If any of the specified nicknames already refers to an existing
package, a continuable error is signaled exactly as for the
function \cdf{make-package}.

\item[\cd{(:shadow \Mstar\emph{symbol-name})}]
Symbols with the specified names are created as shadows
in the package being defined, just as with the function \cdf{shadow}.

\item[\cd{(:shadowing-import-from \emph{package-name} \Mstar\emph{symbol-name})}]
Symbols with the specified names are located in the specified package.
These symbols are imported into the package being defined, shadowing
other symbols if necessary, just as with the function \cdf{shadowing-import}.
In no case will symbols be created in a package other than
the one being defined;
a continuable error is signaled if for any \emph{symbol-name} there
is no symbol of that name accessible in the package named \emph{package-name}.

\item[\cd{(:use \Mstar\emph{package-name})}]
The package being defined is made to ``use'' (inherit from)
the packages specified by this option, just as with
the function \cdf{use-package}.
If no \cd{:use} option is supplied, then a default list is assumed
as for \cdf{make-package}.

X3J13 voted in January 1989
\issue{MAKE-PACKAGE-USE-DEFAULT}
to change the specification of \cdf{make-package} so that the default value
for the \cd{:use} argument is unspecified.  This change affects \cdf{defpackage}
as well.  Portable code should specify \cd{(:use~'("COMMON-LISP"))} explicitly.

\item[\cd{(:import-from \emph{package-name} \Mstar\emph{symbol-name})}]
Symbols with the specified names are located in the specified package.
These symbols are imported into the package being defined,
just as with the function \cdf{import}.
In no case will symbols be created in a package other than
the one being defined;
a continuable error is signaled if for any \emph{symbol-name} there
is no symbol of that name accessible in the package named \emph{package-name}.

\item[\cd{(:intern \Mstar\emph{symbol-name})}]
Symbols with the specified names are located or created
in the package being defined, just as with the function
\cdf{intern}.  Note that the action of this option may be
affected by a \cd{:use}
option, because an inherited symbol will be used in preference
to creating a new one.

\item[\cd{(:export \Mstar\emph{symbol-name})}]
Symbols with the specified names are located or created
in the package being defined and then exported, just as with the function
\cdf{export}.  Note that the action of this option may be
affected by a \cd{:use}, \cd{:import-from}, or \cd{:shadowing-import-from}
option, because an inherited or imported symbol will be used in preference
to creating a new one.
\end{flushdesc}

The order in which options appear in a \cdf{defpackage} form does not matter;
part of the convenience of \cdf{defpackage} is that it sorts out the options
into the correct order for processing.
Options are processed in the following order:
\begin{tabbing}
1.~~\cd{:shadow} and \cd{:shadowing-import-from} \\
2.~~\cd{:use} \\
3.~~\cd{:import-from} and \cd{:intern} \\
4.~~\cd{:export}
\end{tabbing}
Shadows are established first in order to avoid spurious name conflicts
when use links are established.  Use links must occur before importing
and interning so that those operations may refer to normally inherited
symbols rather than creating new ones.  Exports are performed last so that
symbols created by any of the other options, in particular,
shadows and imported symbols, may be exported.  Note that exporting an
inherited symbol implicitly imports it first
(see section~\ref{EXPORT-IMPORT-SECTION}).

If no package named \emph{defined-package-name} already exists,
\cdf{defpackage} creates it.  If such a package does already exist,
then no new package is created.  The existing package is
modified, if possible, to reflect the new definition.  The results are
undefined if the new definition is not consistent with the current
state of the package.

An error is signaled if more than one \cd{:size} option appears.

An error is signaled if the same \cdf{symbol-name} argument (in the sense
of comparing names with \cdf{string=}) appears more than once among
the arguments to all the \cd{:shadow}, \cd{:shadowing-import-from},
\cd{:import-from}, and \cd{:intern} options.

An error is signaled if the same \cdf{symbol-name} argument (in the sense
of comparing names with \cdf{string=}) appears more than once among
the arguments to all the \cd{:intern} and \cd{:export} options.

Other kinds of name conflicts are handled in the same manner that
the underlying operations \cdf{use-package}, \cdf{import}, and \cdf{export}
would handle them.

Implementations may support other \cdf{defpackage} options.
Every implementation should signal an error on encountering
a \cdf{defpackage} option it does not support.

The function \cdf{compile-file} should treat top-level \cdf{defpackage}
forms in the same way it would treat top-level calls to package-affecting
functions (as described at the beginning of
section~\ref{PACKAGE-FUNCTIONS-SECTION}).

Here is an example of a call to \cdf{defpackage} that ``plays it safe''
by using only strings as names.
\begin{lisp}
(cl:defpackage "MY-VERY-OWN-PACKAGE" \\*
~~(:size 496) \\*
~~(:nicknames "MY-PKG" "MYPKG" "MVOP") \\*
~~(:use "COMMON-LISP") \\
~~(:shadow "CAR" "CDR") \\*
~~(:shadowing-import-from "BRAND-X-LISP" "CONS") \\*
~~(:import-from "BRAND-X-LISP" "GC" "BLINK-FRONT-PANEL-LIGHTS") \\*
~~(:export "EQ" "CONS" "MY-VERY-OWN-FUNCTION"))
\end{lisp}
The preceding \cdf{defpackage} example is designed to operate correctly
even if the package current when the form is read happens not to
``use'' the \cdf{common-lisp} package.  (Note the use in this example
of the nickname \cdf{cl} for the \cdf{common-lisp} package.)
Moreover, neither reading in nor evaluating
this \cdf{defpackage} form will ever create any symbols in the
current package.  Note too the use of uppercase letters in the strings.

Here, for the sake of contrast, is a rather similar use of
\cdf{defpackage} that ``plays the whale'' by using all sorts of
permissible syntax.
\begin{lisp}
(defpackage my-very-own-package \\*
~~(:export :EQ common-lisp:cons my-very-own-function) \\*
~~(:nicknames "MY-PKG" \#:MyPkg) \\*
~~(:use "COMMON-LISP") \\
~~(:shadow "CAR") \\*
~~(:size 496) \\*
~~(:nicknames mvop) \\
~~(:import-from "BRAND-X-LISP" "GC" Blink-Front-Panel-Lights) \\
~~(:shadow common-lisp::cdr) \\*
~~(:shadowing-import-from "BRAND-X-LISP" CONS))
\end{lisp}
This example has exactly the same effect on the newly created package
but may create useless symbols in other packages.
The use of explicit package tags is particularly confusing;
for example, this \cdf{defpackage} form will cause the symbol
\cdf{cdr} to be shadowed \emph{in the new package}; it will not be
shadowed in the package \cdf{common-lisp}.  The fact that the name ``\cdf{CDR}''
was specified by a package-qualified reference to a symbol in the
\cdf{common-lisp} package is a red herring.
The moral is that the syntactic flexibility of \cdf{defpackage},
as in other parts of Common Lisp,
yields considerable convenience when used with commonsense competence,
but unutterable confusion when used with Malthusian profusion.

\beforenoterule
\begin{implementation}
An implementation of \cdf{defpackage} might choose to transform
all the \emph{package-name} and \emph{symbol-name} arguments
into strings at macro expansion time, rather than at the time
the resulting expansion is executed, so that even if source code
is expressed in terms of strange symbols in the \cdf{defpackage} form,
the binary file resulting from compiling the source code would
contain only strings.  The purpose of this is simply to minimize
the creation of useless symbols in production code.  This technique
is permitted as an implementation strategy but is not a
behavior required by the specification of \cdf{defpackage}.
\end{implementation}
\afternoterule

Note that \cdf{defpackage} is not capable by itself of defining
mutually recursive packages, for example two packages each of
which uses the other.  However, nothing prevents one from using
\cdf{defpackage} to perform much of the initial setup and then
using functions such as \cdf{use-package}, \cdf{import}, and \cdf{export}
to complete the links.

The purpose of \cdf{defpackage} is to encourage the user to
put the entire definition of a package and its relationships to
other packages in a single place.  It may also encourage the designer
of a large system to place the definitions of all relevant packages
into a single file (say) that can be loaded before loading or compiling
any code that depends on those packages.  Such a file, if carefully
constructed, can simply be loaded into the \cdf{common-lisp-user} package.

Implementations and programming environments may also be better able
to support the programming process (if only by providing better
error checking) through global knowledge of the intended package setup.
\end{defmac}
\end{new}

\begin{defun}[Function]
find-all-symbols string-or-symbol

\cdf{find-all-symbols}
searches every package in the Lisp system to find
every symbol whose print name is the
specified string.  A list of all such symbols found is returned.
This search is case-sensitive.
If the argument is a symbol, its print name supplies
the string to be searched for.
\end{defun}

\begin{defmac}
do-symbols (var [package [result-form]])
           {declaration}* {tag | statement}*

\cdf{do-symbols} provides straightforward iteration over the symbols of a
package.  The body is performed once for each symbol accessible in the
\emph{package}, in no particular order, with the variable \emph{var} bound to
the symbol.  Then \emph{result-form} (a single form, \emph{not} an implicit
\cdf{progn}) is evaluated, and the result is the value of the
\cdf{do-symbols} form.  (When the \emph{result-form} is evaluated, the control
variable \emph{var} is still bound and has the value {\false}.)  If the
\emph{result-form} is omitted, the result is {\false}.  \cdf{return} may be used
to terminate the iteration prematurely.  If execution of the body affects
which symbols are contained in the \emph{package}, other than possibly to
remove the symbol currently the value of \emph{var} by using \cdf{unintern},
the effects are unpredictable.

\begin{new}
X3J13 voted in January 1989
\issue{PACKAGE-FUNCTION-CONSISTENCY}
to clarify that the \emph{package} argument may be either a package object
or a package name (see section~\ref{PACKAGE-NAMES-SECTION}).
\end{new}

\begin{new}
X3J13 voted in March 1988
\issue{DO-SYMBOLS-DUPLICATES}
to specify that the body of a \cdf{do-symbols}
form may be executed more than once for the same accessible symbol, and users
should take care to allow for this possibility.

The point is that the same symbol might be accessible via more than one
chain of inheritance, and it is implementationally costly to eliminate
such duplicates.  Here is an example:
\begin{lisp}
(setq *a* (make-package 'a))~~~~~~;\textrm{Implicitly uses package \cdf{common-lisp}} \\*
(setq *b* (make-package 'b))~~~~~~;\textrm{Implicitly uses package \cdf{common-lisp}} \\*
(setq *c* (make-package 'c :use '(a b))) \\
\\
(do-symbols (x *c*) (print x))~~~~;\textrm{Symbols in package \cdf{common-lisp}} \\*
~~~~~~~~~~~~~~~~~~~~~~~~~~~~~~~~~~;~\textrm{might be printed once or twice here}
\end{lisp}

X3J13 voted in January 1989
\issue{MAPPING-DESTRUCTIVE-INTERACTION}
to restrict user side effects; see section~\ref{STRUCTURE-TRAVERSAL-SECTION}.
\end{new}
\begin{new}
Note that the \cdf{loop} construct provides a kind of \cdf{for} clause that
can iterate over the symbols of a package (see chapter~\ref{LOOP}).
\end{new}
\end{defmac}

\begin{defmac}
do-external-symbols (var [package [result]])
                    {declaration}* {tag | statement}*

\cdf{do-external-symbols} is just like \cdf{do-symbols}, except that only
the external symbols of the specified package are scanned.
\begin{new}
The clarification voted by X3J13
in March 1988 for \cdf{do-symbols}
\issue{DO-SYMBOLS-DUPLICATES},
regarding redundant executions of the body for the same symbol,
applies also to \cdf{do-external-symbols}.
\end{new}

\begin{new}
X3J13 voted in January 1989
\issue{PACKAGE-FUNCTION-CONSISTENCY}
to clarify that the \emph{package} argument may be either a package object
or a package name (see section~\ref{PACKAGE-NAMES-SECTION}).
\end{new}

\begin{new}
X3J13 voted in January 1989
\issue{MAPPING-DESTRUCTIVE-INTERACTION}
to restrict user side effects; see section~\ref{STRUCTURE-TRAVERSAL-SECTION}.
\end{new}
\end{defmac}

\begin{defmac}
do-all-symbols (var [result-form])
               {declaration}* {tag | statement}*

This is similar to \cdf{do-symbols} but executes the body once for every
symbol contained in every package.  (This will not process every symbol
whatsoever, because a symbol not accessible in any package will not
be processed.  Normally, uninterned symbols are not accessible in any package.)
It is \emph{not} in general
the case that each symbol is processed only once, because a symbol may
appear in many packages.

\begin{new}
The clarification voted by X3J13
in March 1988 for \cdf{do-symbols}
\issue{DO-SYMBOLS-DUPLICATES},
regarding redundant executions of the body for the same symbol,
applies also to \cdf{do-all-symbols}.
\end{new}

\begin{new}
X3J13 voted in January 1989
\issue{PACKAGE-FUNCTION-CONSISTENCY}
to clarify that the \emph{package} argument may be either a package object
or a package name (see section~\ref{PACKAGE-NAMES-SECTION}).
\end{new}

\begin{new}
X3J13 voted in January 1989
\issue{MAPPING-DESTRUCTIVE-INTERACTION}
to restrict user side effects; see section~\ref{STRUCTURE-TRAVERSAL-SECTION}.
\end{new}
\end{defmac}

\begin{new}
X3J13 voted in January 1989
\issue{HASH-TABLE-PACKAGE-GENERATORS}
to add a new macro \cdf{with-package-iterator} to the language.

\begin{defmac}
with-package-iterator (mname package-list {symbol-type}+)
                      {\,form}*

The name \emph{mname} is bound and defined as if by \cdf{macrolet},
with the body \emph{form\/}s as its lexical scope, to be a ``generator macro''
such that each invocation of \cd{(\emph{mname})} will
return a symbol and that successive invocations
will eventually deliver, one by one, all the symbols
from the packages that are elements of the list that is the value of the
expression \emph{package-list} (which is evaluated exactly once).

Each element of the \emph{package-list} value
may be either a package or the name of a package.
As a further convenience, if the \emph{package-list} value
is itself a package or the name of a package, it is treated
as if a singleton list containing that value had been provided.
If the \emph{package-list} value is \cdf{nil}, it is considered
to be an empty list of packages.

At each invocation of the generator macro, there are two possibilities.
If there is yet another unprocessed symbol, then
four values are returned: \cdf{t}, the symbol,
a keyword
indicating the accessibility of the symbol within the package (see below), and
the package from which the symbol was accessed.
If there are no more unprocessed symbols in the
list of packages, then one value is returned: \cdf{nil}.

When the generator macro returns a symbol as its second value, the
fourth value is always one of the packages present or named in the
\emph{package-list} value, and the third value is a keyword indicating
accessibility:
\cd{:internal} means present in the package and not exported;
\cd{:external} means present and exported;
and \cd{:inherited} means not present (thus not shadowed) but inherited
from some package used by the package that is the fourth value.

Each \emph{symbol-type} in an invocation of \cdf{with-package-iterator}
is not evaluated.  More than one may be present; their order does not
matter.  They indicate the accessibility types of interest.
A symbol is not returned by the generator macro unless its actual
accessibility matches one of the \emph{symbol-type} indicators.
The standard \emph{symbol-type} indicators are \cd{:internal},
\cd{:external}, and \cd{:inherited}, but implementations are permitted
to extend the syntax of \cdf{with-package-iterator} by recognizing
additional symbol accessibility types.  An error is signaled
if no \emph{symbol-type} is supplied, or if any supplied \emph{symbol-type}
is not recognized by the implementation.

The order in which symbols are produced by successive invocations
of the generator macro is not necessarily correlated in any way
with the order of the packages in the \emph{package-list}.
When more than one package is in the \emph{package-list},
symbols accessible from more than one package may be produced
once or more than once.  Even when only one package is specified,
symbols inherited in multiple ways via used packages may be
produced once or more than once.

The implicit interior state of the iteration over the list of packages
and the symbols within them has dynamic extent.
It is an error to invoke the generator macro
once the \cdf{with-package-iterator} form has been exited.

Any number of invocations of \cdf{with-package-iterator}
and related macros may be nested, and the generator macro of an
outer invocation may be called from within an inner invocation
(provided, of course, that its name is visible or otherwise made available).

X3J13 voted in January 1989
\issue{MAPPING-DESTRUCTIVE-INTERACTION}
to restrict user side effects; see section~\ref{STRUCTURE-TRAVERSAL-SECTION}.

\beforenoterule
\begin{rationale}
This facility is a bit more flexible in some ways than \cdf{do-symbols}
and friends.
In particular, it makes it possible to implement \cdf{loop}
clauses for iterating over packages in a way that is both portable
and efficient (see chapter~\ref{LOOP}).
\end{rationale}
\afternoterule
\end{defmac}
\end{new}

\section{Modules}

A \emph{module} is a Common Lisp subsystem that is loaded from one or more
files.  A module is normally loaded as a single unit, regardless of how
many files are involved.  A module may consist of one package or several
packages.  The file-loading process is necessarily
implementation-dependent, but Common Lisp provides some very simple
portable machinery for naming modules, for keeping track of which modules
have been loaded, and for loading modules as a unit.

\begin{new}
X3J13 voted in January 1989
\issue{REQUIRE-PATHNAME-DEFAULTS}
to eliminate the entire module facility from
the language; that is, the variable \cd{*modules*} and the functions
\cdf{provide} and \cdf{require} are deleted.
X3J13 commented that the file-loading feature of \cdf{require} is not
portable, and that the remaining functionality is easily implemented
by user code.  (I will add that in any case the specification of
\cdf{require} is so vague that different implementations are likely to
have differing behavior.)
\end{new}

\begin{defun}[Variable]
*modules*

The variable \cd{*modules*} is a list of names of the modules
that have been loaded into the Lisp system so far.
This list is used by the functions \cdf{provide} and \cdf{require}.
\end{defun}

\begin{defun}[Function]
provide module-name \\
require module-name &optional pathname

Each module has a unique name (a string).  The \cdf{provide} and \cdf{require}
functions accept either a string or a symbol as the \emph{module-name}
argument.  If a symbol is provided, its print name is used as the module
name.  If the module consists of a single package, it is customary for
the package and module names to be the same.

The \cdf{provide}
function adds a new module name to the list of modules
maintained in the variable \cd{*modules*}, thereby indicating that
the module in question has been loaded.

The \cdf{require} function tests whether a module is already present
(using a case-sensitive comparison); if the module is not present,
\cdf{require} proceeds to load the appropriate
file or set of files.  The \emph{pathname} argument, if present, is a single
pathname or a list of pathnames whose files are to be loaded in order,
left to right.  If the \emph{pathname} argument is {\false} or is not provided,
the system will attempt to determine, in some
system-dependent manner, which files to load.
This will typically involve some central
registry of module names and the associated file lists.

\begin{new}
X3J13 voted in March 1988
not to permit symbols as pathnames
\issue{PATHNAME-SYMBOL} and
to specify exactly which streams may be used as pathnames
\issue{PATHNAME-STREAM} (see section~\ref{PATHNAME-FUNCTIONS}).
Of course, this is moot if \cdf{require} is not in the language.
\end{new}

\begin{new}
X3J13 voted in January 1989
\issue{RETURN-VALUES-UNSPECIFIED}
to specify that the values returned by \cdf{provide} and \cdf{require}
are implementation-dependent.  Of course, this is moot
if \cdf{provide} and \cdf{require} are not in the language.
\end{new}


\beforenoterule
\begin{implementation}
One way to implement such a registry on
many operating systems is simply to use a distinguished ``library''
directory within the file system, where the name of each file
is the same as the module it contains.
\end{implementation}
\afternoterule
\end{defun}

\section{An Example}
\label{PACKAGE-EXAMPLE-SECTION}

\begin{obsolete}\noindent
Most users will want to load and use packages but will never need to
build one.  Often a user will load a number of packages into the
\cdf{user} package whenever using Common Lisp.  Typically an implementation
might provide some sort of initialization file  mechanism to make such setup
automatic when the Lisp starts up.  Table~\ref{INIT-FILE-TABLE}
shows such an initialization file, one that simply
causes other facilities to be loaded.
\end{obsolete}

\begin{newer}
X3J13 voted in March 1989 \issue{LISP-PACKAGE-NAME} to specify that
the forthcoming ANSI Common Lisp will use the package name \cdf{common-lisp-user}
instead of \cdf{user}.
\end{newer}

\begin{table}[t]
\caption{An Initialization File}
\label{INIT-FILE-TABLE}
\begin{lisp}
;;;; Lisp init file for I. Newton. \\
 \\
;;; Set up the USER package the way I like it. \\
 \\
(require 'calculus)~~~~~~~~~~~~~;I use CALCULUS a lot; load it. \\
(use-package 'calculus)~~~~~~~~~;Get easy access to its \\
~~~~~~~~~~~~~~~~~~~~~~~~~~~~~~~~; exported symbols. \\
 \\
(require 'newtonian-mechanics)~~;Same thing for NEWTONIAN-MECHANICS \\
(use-package 'newtonian-mechanics) \\
 \\
;;; I just want a few things from RELATIVITY, \\
;;; and other things conflict. \\
;;; Import only what I need into the USER package. \\
 \\
(require 'relativity) \\
(import '(relativity:speed-of-light \\
~~~~~~~~~~relativity:ignore-small-errors)) \\
 \\
;;; These are worth loading, but I will use qualified names, \\
;;; such as PHLOGISTON:MAKE-FIRE-BOTTLE, to get at any symbols \\
;;; I might need from these packages. \\
 \\
(require 'phlogiston) \\
(require 'alchemy) \\
 \\
;;; End of Lisp init file for I. Newton.
\end{lisp}
\end{table}

When each of two files uses some symbols from the other, the author
of those files must be
careful to arrange the contents of the file in the proper order.
Typically each file contains a single package that is a complete module.
The contents of such a file should include the following items, in
order:

\begin{enumerate}
\item
A call to \cdf{provide} that announces the module name.

\item
A call to \cdf{in-package} that establishes the package.

\item
A call to \cdf{shadow} that establishes any local symbols that will shadow
symbols that would otherwise be inherited from packages that this
package will use.

\item
A call to \cdf{export} that establishes all of this package's external
symbols.

\item
Any number of calls to \cdf{require} to load other modules that the
contents of this file might want to use or refer to.  (Because the
calls to \cdf{require} follow the calls to \cdf{in-package},
\cdf{shadow}, and \cdf{export}, it is possible for the packages that may
be loaded to refer to external symbols in this package.)

\item
Any number of calls to \cdf{use-package}, to make external
symbols from other packages accessible in this package.

\item
Any number of calls to \cdf{import}, to make
symbols from other packages present in this package.

\item
Finally, the definitions making up the contents of this package/module.
\end{enumerate}

The following mnemonic sentence may be helpful in remembering
the proper order of these calls:
\begin{center}
{\bf Put in seven extremely random user interface commands.}
\end{center}
Each word of the sentence corresponds to one item in the above ordering:
\begin{tabbing}
\hskip 6pc\=\hskip 8pc\=\kill
\>Put\>\cdf{Provide} \\*
\>IN\>\cdf{IN-package} \\*
\>Seven\>\cdf{Shadow} \\*
\>EXtremely\>\cdf{EXport} \\*
\>Random\>\cdf{Require} \\*
\>USEr\>\cdf{USE-package} \\*
\>Interface\>\cdf{Import} \\*
\>COmmands\>COntents of package/module
\end{tabbing}


\begin{table}
\caption{File \protect\cdf{alchemy}}
\label{ALCHEMY-FILE-TABLE}
\begin{lisp}
;;;; Alchemy functions, written and maintained by Merlin, Inc. \\
 \\[4pt]
(provide 'alchemy)~~~~~~~~~~~~~~~~~~;The module is named ALCHEMY. \\
(in-package 'alchemy)~~~~~~~~~~~~~~~;So is the package. \\
 \\[4pt]
;;; There is nothing to shadow. \\
 \\[4pt]
;;; Here is the external interface. \\
 \\[4pt]
(export '(lead-to-gold gold-to-lead  \\
~~~~~~~~~~antimony-to-zinc elixir-of-life)) \\
 \\[4pt]
;;; This package/module needs a function from \\
;;; the PHLOGISTON package/module. \\
 \\[4pt]
(require 'phlogiston) \\
 \\[4pt]
;;; We don't frequently need most of the external symbols from \\
;;; PHLOGISTON, so it's not worth doing a USE-PACKAGE on it. \\
;;; We'll just use qualified names as needed.~~But we use \\
;;; one function, MAKE-FIRE-BOTTLE, a lot, so import it. \\
;;; It's external in PHLOGISTON and so can be referred to \\
;;; here using ":" qualified-name syntax. \\
 \\[4pt]
(import '(phlogiston:make-fire-bottle)) \\
 \\[4pt]
;;; Now for the real contents of this file. \\
 \\[4pt]
(defun lead-to-gold (x) \\
~~"Takes a quantity of lead and returns gold." \\
~~(when (> (phlogiston:heat-flow 5 x x)~~;Using a qualified symbol \\
~~~~~~~~~~~3) \\
~~~~(make-fire-bottle x))~~~~~~~~~~~~~~~~;Using an imported symbol \\
~~(gild x)) \\
 \\[4pt]
;;; And so on ...
\end{lisp}
\vfill
\end{table}

\begin{table}
\caption{File \protect\cdf{phlogiston}}
\label{PHLOGISTON-FILE-TABLE}
\begin{lisp}
;;;; Phlogiston functions, by Thermofluidics, Ltd. \\
 \\[4pt]
(provide 'phlogiston)~~~~~~~~~~~~~;The module is named PHLOGISTON. \\
(in-package 'phlogiston)~~~~~~~~~~;So is the package. \\
 \\[4pt]
;;; There is nothing to shadow. \\
 \\[4pt]
;;; Here is the external interface. \\
 \\[4pt]
(export '(heat-flow cold-flow mix-fluids separate-fluids \\
~~~~~~~~~~burn make-fire-bottle)) \\
 \\[4pt]
;;; This file uses functions from the ALCHEMY package/module. \\
 \\[4pt]
(require 'alchemy) \\
 \\[4pt]
;;; We use alchemy functions a lot, so use the package. \\
;;; This will allow symbols exported from the ALCHEMY package \\
;;; to be referred to here without the need for qualified names. \\
 \\[4pt]
(use-package 'alchemy) \\
 \\[4pt]
;;; No calls to IMPORT are needed here. \\
 \\[4pt]
;;; The real contents of this package/module. \\
\\[4pt]
(defvar *feeling-weak* nil) \\
 \\[4pt]
(defun heat-flow (amount x y) \\
~~"Make some amount of heat flow from x to y." \\
~~(when *feeling-weak* \\
~~~~(quaff (elixir-of-life)))~~~~~;No qualifier is needed. \\
~~(push-heat amount x y)) \\
 \\[4pt]
;;; And so on ...
\end{lisp}
\vfill
\end{table}

The sentence says what it helps you to do.

\begin{new}
The most distressing aspect of the X3J13 vote to eliminate
\cdf{provide} and \cdf{require}
\issue{REQUIRE-PATHNAME-DEFAULTS}
is of course that it completely ruins the mnemonic sentence.
\end{new}


Now, suppose for the sake of example
that the \cdf{phlogiston} and \cdf{alchemy} packages are
single-file, single-package modules as described above.  The \cdf{phlogiston}
package needs to use the \cdf{alchemy} package, and the \cdf{alchemy} package
needs to use several
external symbols from the \cdf{phlogiston} package.
The definitions in the \cdf{alchemy} and \cdf{phlogiston} files
(see tables~\ref{ALCHEMY-FILE-TABLE} and~\ref{PHLOGISTON-FILE-TABLE})
allow a
user to specify \cdf{require} statements for either of these modules, or for
both of them in either order, and all relevant information will be
loaded automatically and in the correct order.

\begin{obsolete}
For very large modules whose contents are spread over several files
(the \cdf{lisp} package is an example), it is recommended that the user
create the package and declare all of the shadows and external symbols
in a separate file, so that this can be loaded before anything that
might use symbols from this package.
\end{obsolete}

\begin{new}
Indeed, the \cdf{defpackage} macro
approved by X3J13 in January 1989
\issue{DEFPACKAGE}
encourages the use of such a separate file.
(By the way,
X3J13 voted in March 1989 \issue{LISP-PACKAGE-NAME} to specify that
the forthcoming ANSI Common Lisp will use the package name \cdf{common-lisp}
instead of \cdf{lisp}.)
Let's take a look at a revision
of I.~Newton's files using \cdf{defpackage}.
\end{new}

\begin{table}[t]
\caption{An Initialization File When \protect\cdf{defpackage} Is Used}
\label{DEFPACKAGE-INIT-FILE-TABLE}
\begin{lisp}
;;;; Lisp init file for I. Newton. \\
 \\
;;; Set up the USER package the way I like it. \\
 \\
(load "calculus")~~~~~~~~~~~~~~~;I use CALCULUS a lot; load it. \\
(use-package 'calculus)~~~~~~~~~;Get easy access to its \\
~~~~~~~~~~~~~~~~~~~~~~~~~~~~~~~~; exported symbols. \\
 \\
(load "newtonian-mechanics")~~~~;Ditto for NEWTONIAN-MECHANICS \\
(use-package 'newtonian-mechanics) \\
 \\
;;; I just want a few things from RELATIVITY, \\
;;; and other things conflict. \\
;;; Import only what I need into the USER package. \\
 \\
(load "relativity") \\
(import '(relativity:speed-of-light \\
~~~~~~~~~~relativity:ignore-small-errors)) \\
 \\
;;; These are worth loading, but I will use qualified names, \\
;;; such as PHLOGISTON:MAKE-FIRE-BOTTLE, to get at any symbols \\
;;; I might need from these packages. \\
 \\
(load "phlogiston") \\
(load "alchemy") \\
 \\
;;; End of Lisp init file for I. Newton.
\end{lisp}
\end{table}
\begin{new}
The new version of the initialization file avoids using \cdf{require};
instead, we assume that \cdf{load} will do the job
(see table~\ref{DEFPACKAGE-INIT-FILE-TABLE}).

The other files have each been split into two parts, one that
establishes the package and one that defines the contents.
This example uses a simple convention that for any file
named, say, ``\cdf{foo}'' the file named ``\cdf{foo-package}''
contains the necessary \cdf{defpackage} and/or other package-establishing
code.  The idiom
\begin{lisp}
(unless (find-package "FOO") \\
~~(load "foo-package"))
\end{lisp}
is conventionally used to load a package definition but only if the package
has not already been defined.  (This is a bit clumsy, and there are other
ways to arrange things so that a package is defined no more than once.)

The file \cdf{alchemy-package} is shown in
table~\ref{DEFPACKAGE-ALCHEMY-PACKAGE-TABLE}.
The tricky point here is that the \cdf{alchemy} and \cdf{phlogiston}
packages contain mutual references (each imports from the other),
and so \cdf{defpackage} alone cannot do the job.  Therefore
the \cdf{phlogiston} package is not mentioned in a \cd{:use} option
in the \cdf{defpackage} for the \cdf{alchemy} package.  Instead,
the function \cdf{use-package} is called explicitly, after the package definition
for \cdf{phlogiston} has been loaded.  Note that this file has
been coded with excruciating care so as to operate correctly even if
the package current when the file is loaded does not inherit from
the \cdf{common-lisp} package. In particular, the standard load-package-definition
idiom has been peppered with package qualifiers:
\begin{lisp}
(cl:unless (cl:find-package "PHLOGISTON") \\
~~(cl:load "phlogiston-package"))
\end{lisp}
Note the use of the nickname \cdf{cl} for the \cdf{common-lisp} package.

The \cdf{alchemy} file, shown in table~\ref{DEFPACKAGE-ALCHEMY-FILE-TABLE},
simply loads the \cdf{alchemy} package definition,
makes that package current, and then defines the ``real contents''
of the package.
\end{new}
\begin{table}[t]
\caption{File \protect\cdf{alchemy-package} Using \protect\cdf{defpackage}}
\label{DEFPACKAGE-ALCHEMY-PACKAGE-TABLE}
\begin{lisp}
;;;; Alchemy package, written and maintained by Merlin, Inc. \\
 \\
(cl:defpackage "ALCHEMY" \\
~~(:export "LEAD-TO-GOLD" "GOLD-TO-LEAD" \\
~~~~~~~~~~~"ANTIMONY-TO-ZINC" "ELIXIR-OF-LIFE") \\
~~) \\
 \\
;;; This package needs a function from the PHLOGISTON package. \\
;;; Load the definition of the PHLOGISTON package if necessary. \\
 \\
(cl:unless (cl:find-package "PHLOGISTON") \\
~~(cl:load "phlogiston-package")) \\
 \\
;;; We don't frequently need most of the external symbols from \\
;;; PHLOGISTON, so it's not worth doing a USE-PACKAGE on it. \\
;;; We'll just use qualified names as needed.  But we use \\
;;; one function, MAKE-FIRE-BOTTLE, a lot, so import it. \\
;;; It's external in PHLOGISTON and so can be referred to \\
;;; here using ":" qualified-name syntax. \\
 \\
(cl:import '(phlogiston:make-fire-bottle))
\end{lisp}
\vskip\ruletonoteskip
\hrule
\vskip\ruletonoteskip\null
\caption{File \protect\cdf{alchemy} Using \protect\cdf{defpackage}}
\label{DEFPACKAGE-ALCHEMY-FILE-TABLE}
\begin{lisp}
;;;; Alchemy functions, written and maintained by Merlin, Inc. \\
 \\
(unless (find-package "ALCHEMY") \\
~~(load "alchemy-package")) \\
 \\
(in-package 'alchemy) \\
 \\
(defun lead-to-gold (x) \\
~~"Takes a quantity of lead and returns gold." \\
~~(when (> (phlogiston:heat-flow 5 x x)~~;Using a qualified symbol \\
~~~~~~~~~~~3) \\
~~~~(make-fire-bottle x))~~~~~~~~~~~~~~~~;Using an imported symbol \\
~~(gild x)) \\
 \\
;;; And so on ...
\end{lisp}
\end{table}



\begin{table}[t]
\caption{File \protect\cdf{phlogiston-package} Using \protect\cdf{defpackage}}
\label{DEFPACKAGE-PHLOGISTON-PACKAGE-TABLE}
\begin{lisp}
;;;; Phlogiston package definition, by Thermofluidics, Ltd. \\
 \\
;;; This package uses functions from the ALCHEMY package. \\
 \\
(cl:unless (cl:find-package "ALCHEMY") \\
~~(cl:load "alchemy-package")) \\
 \\
(cl:defpackage "PHLOGISTON" \\
~~(:use "COMMON-LISP" "ALCHEMY") \\
~~(:export "HEAT-FLOW" \\
~~~~~~~~~~~"COLD-FLOW" \\
~~~~~~~~~~~"MIX-FLUIDS" \\
~~~~~~~~~~~"SEPARATE-FLUIDS" \\
~~~~~~~~~~~"BURN" \\
~~~~~~~~~~~"MAKE-FIRE-BOTTLE") \\
~~)
\end{lisp}
\end{table}

\begin{table}[b]
\caption{File \protect\cdf{phlogiston} Using \protect\cdf{defpackage}}
\label{DEFPACKAGE-PHLOGISTON-FILE-TABLE}
\begin{lisp}
;;;; Phlogiston functions, by Thermofluidics, Ltd. \\
 \\
(unless (find-package "PHLOGISTON") \\
~~(load "phlogiston-package")) \\
 \\
(in-package 'phlogiston) \\
\\
(defvar *feeling-weak* nil) \\
\\
(defun heat-flow (amount x y) \\
~~"Make some amount of heat flow from x to y." \\
~~(when *feeling-weak* \\
~~~~(quaff (elixir-of-life)))~~~~~;No qualifier is needed. \\
~~(push-heat amount x y)) \\
 \\
;;; And so on ...
\end{lisp}
\end{table}
\begin{new}

The file \cdf{phlogiston-package} is shown in
table~\ref{DEFPACKAGE-PHLOGISTON-PACKAGE-TABLE}.
This one is a little more straightforward than the file \cdf{alchemy-package},
because the latter bears the responsibility for breaking the
circular package references.
This file simply makes sure that the \cdf{alchemy} package is defined
and then performs a \cdf{defpackage} for the \cdf{phlogiston} package.

The \cdf{phlogiston} file, shown in table~\ref{DEFPACKAGE-PHLOGISTON-FILE-TABLE},
simply loads the \cdf{phlogiston} package definition,
makes that package current, and then defines the ``real contents''
of the package.



Let's look at the question of package circularity in
this example a little more closely.
Suppose that the file \cdf{alchemy-package} is loaded first.
It defines the \cdf{alchemy} package and then loads
file \cdf{phlogiston-package}.  That file in turn finds that the
package \cdf{alchemy} has already been defined and therefore does
not attempt to load file \cdf{alchemy-package} again; it merely
defines package \cdf{phlogiston}.  The file \cdf{alchemy-package}
then has a chance to import \cd{phlogiston:make-fire-bottle} and everything
is fine.

On the other hand, suppose that the file \cdf{phlogiston-package} is
loaded first.  It finds that the
package \cdf{alchemy} has \emph{not} already been defined, and therefore
it immediately loads file \cdf{alchemy-package}.
That file in turn defines the \cdf{alchemy} package; then it
finds that package \cdf{phlogiston} is not yet defined and so
loads file \cdf{phlogiston-package} \emph{again} (indeed, in nested fashion).
This time file \cdf{phlogiston-package} \emph{does} find that the
package \cdf{alchemy} has already been defined, so it simply defines
package \cdf{phlogiston} and terminates.
The file \cdf{alchemy-package} then imports \cd{phlogiston:make-fire-bottle}
and terminates.
Finally, the outer loading of file \cdf{phlogiston-package}
\emph{re-defines} package \cdf{phlogiston}.  Oh, dear.  Fortunately the
two definitions of package \cdf{phlogiston} agree in every detail, so
everything ought to be all right.  Still, it looks a bit dicey; \emph{I}
certainly don't have the same warm, fuzzy feeling that I would if
no package were defined more than once.

\clearpage%manual

Conclusion: \cdf{defpackage} goes a long way, but it certainly doesn't
solve all the possible problems of package and file management.
Neither did \cdf{require} and \cdf{provide}.  Perhaps further experimentation
will yield facilities appropriate for future standardization.
\end{new}
       % Packages
%Part{Number, Root = "CLM.MSS"}
%Chapter of Common Lisp Manual.  Copyright 1984, 1988, 1989 Guy L. Steele Jr.

\clearpage\def\pagestatus{FINAL PROOF}

\begingroup
\def\arcsinh{\mathop{\rm arcsinh}\nolimits}
\def\arccosh{\mathop{\rm arccosh}\nolimits}
\def\arctanh{\mathop{\rm arctanh}\nolimits}
\def\cis{\mathop{\rm cis}\nolimits}
\def\phase{\mathop{\rm phase}\nolimits}


\chapter{Numbers}
\label{NUMBER}


Common Lisp provides several different representations for numbers.
These representations may be divided into four categories: integers,
ratios, floating-point numbers, and complex numbers.  Many numeric
functions will accept any kind of number; they are {\it generic}.  Other
functions accept only certain kinds of numbers.

\begin{new}
Note that this remark, predating the design of the Common Lisp Object System,
uses the term ``generic'' in a generic sense and not necessarily
in the technical sense used by CLOS
(see chapter~\ref{DTYPES}).
\end{new}



In general, numbers in Common Lisp are not true objects; \cdf{eq} cannot
be counted upon to operate on them reliably.  In particular,
it is possible that the expression
\begin{lisp}
(let ((x z) (y z)) (eq x y))
\end{lisp}
may be false rather than true if the value of \cdf{z} is a number.

\beforenoterule
\begin{rationale}
This odd breakdown of \cdf{eq} in the case of numbers
allows the implementor enough design freedom to produce exceptionally
efficient numerical code on conventional architectures.
MacLisp requires this freedom, for example, in order to produce compiled
numerical code equal in speed to Fortran.
Common Lisp makes this same restriction,
if not for this freedom, then at least for the sake of compatibility.
\end{rationale}
\afternoterule

If two objects are to be compared for ``identity,'' but either might be
a number, then the predicate \cdf{eql} is probably appropriate;
if both objects are known to be numbers, then \cdf{=}
may be preferable.

\section{Precision, Contagion, and Coercion}
\label{PRECISION-CONTAGION-COERCION-SECTION}

In general,
computations with floating-point numbers are only approximate.
The {\it precision} of a floating-point number is not necessarily
correlated at all with the {\it accuracy} of that number.
For instance, 3.142857142857142857 is a more precise approximation
to $\pi$ than 3.14159, but the latter is more accurate.
The precision refers to the number of bits retained in the representation.
When an operation combines a short floating-point number with a long one,
the result will be a long floating-point number.  This rule is made
to ensure that as much accuracy as possible is preserved; however,
it is by no means a guarantee.
Common Lisp numerical routines do assume, however, that the accuracy of
an argument does not exceed its precision.  Therefore
when two small floating-point numbers
are combined, the result will always be a small floating-point number.
This assumption can be overridden by first explicitly converting
a small floating-point number to a larger representation.
(Common Lisp never converts automatically from a larger size to a smaller one.)

Rational computations cannot overflow in the usual sense
(though of course there may not be enough storage
to represent one), as integers and ratios may in principle be of any magnitude.
Floating-point computations may get exponent overflow or underflow;
this is an error.

\vskip 0pt plus 24pt

\begin{newer}
X3J13 voted in June 1989 \issue{FLOAT-UNDERFLOW}
to address certain problems relating to floating-point overflow and
underflow, but certain parts of the proposed solution were not adopted, namely
to add the macro \cdf{without-floating-underflow-traps} to the language and to
require certain behavior of floating-point overflow and underflow.
The committee agreed that this area of the language requires more
discussion before a solution is standardized.

For the record, the proposal that was considered and rejected
(for the nonce) introduced a macro
\cd{without-\discretionary{}{}{}floating-\discretionary{}{}{}underflow-\discretionary{}{}{}traps}
that would execute its body in such a way that, within its dynamic extent,
a floating-point underflow
must not signal an error but instead must produce
either a denormalized number or zero as the result.
The rejected proposal also specified the following treatment of overflow and underflow:
\begin{itemize}
\item A floating-point computation that overflows should signal
  an error of type \cdf{floating-point-overflow}.
\item Unless the dynamic extent of a use of
  \cdf{without-floating-underflow-traps}, a floating-point computation that
  underflows should signal an error of type \cdf{floating-point-underflow}.  A
  result that can be represented only in denormalized form must be considered an
  underflow in implementations that support denormalized floating-point
  numbers.
\end{itemize}
These points refer to conditions \cdf{floating-point-overflow}
and \cd{floating-\discretionary{}{}{}point-\discretionary{}{}{}underflow}
that were approved by X3J13
and are described in section~\ref{PREDEFINED-CONDITIONS-SECTION}.
\end{newer}

\vskip 0pt plus 24pt

When rational and floating-point numbers are compared or combined by
a numerical function, the rule of {\it floating-point contagion}
is followed: when a rational meets a floating-point number,\vadjust{\penalty-10000}
the rational is first converted to a floating-point number of
the same format.  For functions such as \cdf{+}
that take more than two arguments,
it may be that part of the operation is carried out exactly using
rationals and then the rest is done using floating-point arithmetic.

\begin{new}
X3J13 voted in January 1989
\issue{CONTAGION-ON-NUMERICAL-COMPARISONS}
to apply the rule of floating-point
contagion stated above to the case of {\it combining} rational and floating-point numbers.
For {\it comparing}, the following rule is to be used instead:
When a rational number and a floating-point number are to be compared
by a numerical function, in effect the floating-point number is first
converted to a rational number as if by the function \cdf{rational},
and then an exact comparison of two rational numbers is performed.
It is of course valid to use a more efficient implementation than
actually calling the function \cdf{rational}, as long as the result
of the comparison is the same.  In the case of complex numbers, the
real and imaginary parts are handled separately.

\beforenoterule
\begin{rationale}
In general, accuracy cannot be preserved in combining operations, but
it can be preserved in comparisons, and preserving it makes that part
of Common Lisp algebraically a bit more tractable.  In particular,
this change prevents the breakdown of transitivity.
Let \cdf{a} be the result of \cd{(/~10.0 single-float-epsilon)}, and
let \cdf{j} be the result of \cd{(floor~a)}.  (Note that \cd{(=~a~(+~a~1.0))}
is true, by the definition of \cdf{single-float-epsilon}.)
Under the old rules,
all of \cd{(<=~a~j)}, \cd{(<~j~(+~j~1))}, and \cd{(<=~(+~j~1)~a)}
would be true; transitivity would then imply that \cd{(<~a~a)} ought to be
true, but of course it is false, and therefore transitivity fails.
Under the new rule, however, \cd{(<=~(+~j~1)~a)} is false.
\end{rationale}
\afternoterule
\end{new}

For functions that are mathematically associative (and possibly commutative),
a Common Lisp implementation may process the arguments in any manner consistent
with associative (and possibly commutative) rearrangement.
This does not affect the order in which the argument forms
are evaluated, of course; that order is always left to right,
as in all Common Lisp function calls.  What is left loose is the
order in which the argument values are processed.
The point of all this is that implementations may differ in 
which automatic coercions are applied because of differing
orders of argument processing.  As an example, consider this
expression:
\begin{lisp}
(+ 1/3 2/3 1.0D0 1.0 1.0E-15)
\end{lisp}
One implementation might process the arguments from left to right,
first adding \cd{1/3} and \cd{2/3} to get \cd{1}, then converting that
to a double-precision floating-point number for combination
with \cd{1.0D0}, then successively converting and adding \cd{1.0} and
\cd{1.0E-15}.  Another implementation might process the arguments
from right to left, first performing a single-precision floating-point addition
of \cd{1.0} and \cd{1.0E-15} (and probably losing some accuracy
in the process!), then converting the sum to double precision
and adding \cd{1.0D0}, then converting \cd{2/3} to double-precision
floating-point and adding it, and then converting \cd{1/3} and adding that.
A third implementation might first scan all the arguments, process
all the rationals first to keep that part of the computation exact,
then find an argument of the largest floating-point format among all
the arguments and add that, and then add in all other arguments,
converting each in turn (all in a perhaps misguided attempt to make
the computation as accurate as possible).  In any case, all three
strategies are legitimate.  The user can of course control the order of
processing explicitly by writing several calls; for example:
\begin{lisp}
(+ (+ 1/3 2/3) (+ 1.0D0 1.0E-15) 1.0)
\end{lisp}
The user can also control all coercions simply by writing calls
to coercion functions explicitly.

In general, then, the type of the result of a numerical function
is a floating-point number of the largest format among all the
floating-point arguments to the function; but if the arguments
are all rational, then the result is rational (except for functions
that can produce mathematically irrational results, in which case
a single-format floating-point number may result).

There is a separate rule of complex contagion.
As a rule, complex numbers never result from a numerical function
unless one or more of the
arguments is complex.  (Exceptions to this
rule occur among the irrational and transcendental functions,
specifically \cdf{expt}, \cdf{log}, \cdf{sqrt},
\cdf{asin}, \cdf{acos}, \cdf{acosh}, and \cdf{atanh};
see section~\ref{TRANSCENDENTAL-SECTION}.)
When a non-complex number meets a complex number, the non-complex
number is in effect first converted to a complex number by providing an
imaginary part of zero.

If any computation produces a result that is a ratio of
two integers such that the denominator evenly divides the
numerator, then the result is immediately converted to the equivalent
integer.  This is called the rule of {\it rational canonicalization}.

If the result of any computation would be a complex rational
with a zero imaginary part, the result is immediately
converted to a non-complex rational number by taking the
real part.  This is called the rule of {\it complex canonicalization}.
Note that this rule does {\it not} apply to complex numbers whose components
are floating-point numbers.  Whereas \cd{\#C(5 0)} and \cd{5} are not
distinct values in Common Lisp (they are always \cdf{eql}),
\cd{\#C(5.0 0.0)} and \cd{5.0} are always distinct values in Common Lisp
(they are never \cdf{eql}, although they are \cdf{equalp}).

\section{Predicates on Numbers}

Each of the following functions tests a single number for
a specific property.
Each function requires that its argument be
a number; to call one with a non-number is an error.

\begin{defun}[Function]
zerop number

This predicate is true if {\it number} is zero (the integer zero,
a floating-point zero, or a complex zero), and is false otherwise.
Regardless of whether an implementation provides distinct representations
for positive and negative floating-point zeros,
\cd{(zerop -0.0)} is always true.
It is an error if the argument {\it number} is not a number.
\end{defun}

\begin{defun}[Function]
plusp number

This predicate is true if {\it number} is strictly greater than zero,
and is false otherwise.
It is an error if the argument {\it number} is not a non-complex number.
\end{defun}

\begin{defun}[Function]
minusp number

This predicate is true if {\it number} is strictly less than zero,
and is false otherwise.
Regardless of whether an implementation provides distinct representations
for positive and negative floating-point zeros,
\cd{(minusp -0.0)} is always false.
(The function \cdf{float-sign} may be used to distinguish a negative zero.)
It is an error if the argument {\it number} is not a non-complex number.
\end{defun}

\begin{defun}[Function]
oddp integer

This predicate is true if the argument {\it integer} is odd (not divisible
by 2), and otherwise is false.  It is an error if the argument is not
an integer.
\end{defun}

\begin{defun}[Function]
evenp integer

This predicate is true if the argument {\it integer} is even (divisible
by 2), and otherwise is false.  It is an error if the argument is not
an integer.
\end{defun}

See also the data-type predicates \cdf{integerp},
\cdf{rationalp}, \cdf{floatp}, \cdf{complexp}, and \cdf{numberp}.

\section{Comparisons on Numbers}

Each of the functions in this section requires that its arguments all be
numbers; to call one with a non-number is an error.  Unless otherwise
specified, each works on all types of numbers, automatically performing
any required coercions when arguments are of different types.

\begin{defun}[Function]
= number &rest more-numbers \\
/= number &rest more-numbers \\
< number &rest more-numbers \\
> number &rest more-numbers \\
<= number &rest more-numbers \\
>= number &rest more-numbers

These functions each take one or more arguments.  If the sequence
of arguments satisfies a certain condition:
\begin{tabbing}
\hskip 5pc\=\kill
\cdf{=}\>all the same \\
\cd{/=}\>all different \\
\cdf{<}\>monotonically increasing \\
\cdf{>}\>monotonically decreasing \\
\cdf{<=}\>monotonically nondecreasing \\
\cdf{>=}\>monotonically nonincreasing
\end{tabbing}
then the predicate is true, and otherwise is false.
Complex numbers may be compared using \cdf{=} and \cd{/=},
but the others require non-complex arguments.
Two complex numbers are considered equal by \cdf{=}
if their real parts are equal and their imaginary parts are equal
according to \cdf{=}.
A complex number may be compared with a non-complex number with \cdf{=} or \cd{/=}.
For example:
\begin{lisp}
\hskip 0.5\textwidth\=\kill
(= 3 3) {\rm is true.}\>(/= 3 3) {\rm is false.} \\
(= 3 5) {\rm is false.}\>(/= 3 5) {\rm is true.} \\
(= 3 3 3 3) {\rm is true.}\>(/= 3 3 3 3) {\rm is false.} \\
(= 3 3 5 3) {\rm is false.}\>(/= 3 3 5 3) {\rm is false.} \\
(= 3 6 5 2) {\rm is false.}\>(/= 3 6 5 2) {\rm is true.} \\
(= 3 2 3) {\rm is false.}\>(/= 3 2 3) {\rm is false.} \\
(< 3 5) {\rm is true.}\>(<= 3 5) {\rm is true.} \\
(< 3 -5) {\rm is false.}\>(<= 3 -5) {\rm is false.} \\
(< 3 3) {\rm is false.}\>(<= 3 3) {\rm is true.} \\
(< 0 3 4 6 7) {\rm is true.}\>(<= 0 3 4 6 7) {\rm is true.} \\
(< 0 3 4 4 6) {\rm is false.}\>(<= 0 3 4 4 6) {\rm is true.} \\
(> 4 3) {\rm is true.}\>(>= 4 3) {\rm is true.} \\
(> 4 3 2 1 0) {\rm is true.}\>(>= 4 3 2 1 0) {\rm is true.} \\
(> 4 3 3 2 0) {\rm is false.}\>(>= 4 3 3 2 0) {\rm is true.} \\
(> 4 3 1 2 0) {\rm is false.}\>(>= 4 3 1 2 0) {\rm is false.} \\
(= 3) {\rm is true.}\>(/= 3) {\rm is true.} \\
(< 3) {\rm is true.}\>(<= 3) {\rm is true.} \\
(= 3.0 \#C(3.0 0.0)) {\rm is true.}\>(/= 3.0 \#C(3.0 1.0)) {\rm is true.} \\
(= 3 3.0) {\rm is true.}\>(= 3.0s0 3.0d0) {\rm is true.} \\
(= 0.0 -0.0) {\rm is true.}\>(= 5/2 2.5) {\rm is true.} \\
(> 0.0 -0.0) {\rm is false.}\>(= 0 -0.0) {\rm is true.}
\end{lisp}
With two arguments, these functions perform the usual arithmetic
comparison tests.
With three or more arguments, they are useful for range checks,
as shown in the following example:
\begin{lisp}
(<= 0 x 9)~~~~~~~~~~~~~~~~~~~~~~;\textrm{true if \cdf{x} is between 0 and 9, inclusive} \\
(< 0.0 x 1.0)~~~~~~~~~~~~~~~~~~~;\textrm{true if \cdf{x} is between 0.0 and 1.0, exclusive} \\
(< -1 j (length s))~~~~~~~~~~~~~;\textrm{true if \cdf{j} is a valid index for \cdf{s}} \\
(<= 0 j k (- (length s) 1))~~~~~;\textrm{true if \cdf{j} and \cdf{k} are each valid} \\
~~~~~~~~~~~~~~~~~~~~~~~~~~~~~~~~;\textrm{indices for \cdf{s} and $\cdf{j}\leq\cdf{k}$}
\end{lisp}

\beforenoterule
\begin{rationale}
The ``unequality'' relation is called \cd{/=} rather than
\cd{<>}
(the name used in Pascal) for two reasons.  First, \cd{/=} of more than two
arguments is not the same as the \cdf{or} of \cdf{<} and \cdf{>} of those same
arguments.  Second, unequality is meaningful for complex numbers even though
\cdf{<} and \cdf{>} are not.  For both reasons it would be misleading to
associate unequality with the names of \cdf{<} and \cdf{>}.
\end{rationale}
\betweennoterule
\begin{incompatibility}
In Common Lisp, the comparison operations
perform ``mixed-mode'' comparisons: \cd{(= 3 3.0)} is true.  In MacLisp,
there must be exactly two arguments, and they must be either both fixnums
or both floating-point numbers.  To compare two numbers for numerical
equality and type equality, use \cdf{eql}.
\end{incompatibility}
\afternoterule
\end{defun}

\begin{defun}[Function]
max number &rest more-numbers \\
min number &rest more-numbers

The arguments may be any non-complex numbers.
\cdf{max} returns the argument that is greatest (closest
to positive infinity).
\cdf{min} returns the argument that is least (closest to
negative infinity).

For \cdf{max},
if the arguments are a mixture of rationals and floating-point
numbers, and the largest argument
is a rational, then the implementation is free to
produce either that rational or its floating-point approximation;
if the largest argument is a floating-point number of a smaller format
than the largest format of any floating-point argument,
then the implementation is free to
return the argument in its given format or expanded to the larger format.
More concisely, the implementation has the choice of returning the largest
argument as is or applying the rules of floating-point contagion,
taking all the arguments into consideration for contagion purposes.
Also, if two or more of the arguments are equal, then any one
of them may be chosen as the value to return.
Similar remarks apply to \cdf{min} (replacing ``largest argument'' by
``smallest argument'').

\begin{lisp}
\hskip 0.5\textwidth\=\kill
(max 6 12) \EV\ 12\>(min 6 12) \EV\ 6 \\
(max -6 -12) \EV\ -6\>(min -6 -12) \EV\ -12 \\
(max 1 3 2 -7) \EV\ 3\>(min 1 3 2 -7) \EV\ -7 \\
(max -2 3 0 7) \EV\ 7\>(min -2 3 0 7) \EV\ -2 \\
(max 3) \EV\ 3\>(min 3) \EV\ 3 \\
(max 5.0 2) \EV\ 5.0\>(min 5.0 2) \EV\ 2 {\it or} 2.0 \\
(max 3.0 7 1) \EV\ 7 {\it or} 7.0\>(min 3.0 7 1) \EV\ 1 {\it or} 1.0 \\
(max 1.0s0 7.0d0) \EV\ 7.0d0 \\
(min 1.0s0 7.0d0) \EV\ 1.0s0 {\it or} 1.0d0 \\
(max 3 1 1.0s0 1.0d0) \EV\ 3 {\it or} 3.0d0 \\
(min 3 1 1.0s0 1.0d0) \EV\ 1 {\it or} 1.0s0 {\it or} 1.0d0
\end{lisp}
\end{defun}

\section{Arithmetic Operations}

Each of the functions in this section requires that its arguments all be
numbers; to call one with a non-number is an error.  Unless otherwise
specified, each works on all types of numbers, automatically performing
any required coercions when arguments are of different types.

\begin{defun}[Function]
+ &rest numbers

This returns the sum of the arguments.  If there are no arguments, the result
is \cd{0}, which is an identity for this operation.

\beforenoterule
\begin{incompatibility}
While \cdf{+} is compatible with its use in Lisp Machine Lisp,
it is incompatible with MacLisp, which uses \cdf{+} for fixnum-only
addition.
\end{incompatibility}
\afternoterule
\end{defun}

\begin{defun}[Function]
- number &rest more-numbers

The function \cdf{-}, when given one argument, returns the negative
of that argument.

The function \cdf{-}, when given more than one argument, successively subtracts
from the first argument all the others, and returns the result.
For example, \cd{(- 3 4 5)} \EV\ \cd{-6}.

\beforenoterule
\begin{incompatibility}
While \cdf{-} is compatible with its use in Lisp Machine Lisp,
it is incompatible with MacLisp, which uses \cdf{-} for fixnum-only
subtraction.
Also, \cdf{-} differs from \cdf{difference} as used in most Lisp
systems in the case of one argument.
\end{incompatibility}
\afternoterule
\end{defun}

\begin{defun}[Function]
* &rest numbers

This returns the product of the arguments.
If there are no arguments, the result
is \cd{1}, which is an identity for this operation.

\beforenoterule
\begin{incompatibility}
While \cdf{*} is compatible with its use in Lisp Machine Lisp,
it is incompatible with MacLisp, which uses \cdf{*} for fixnum-only
multiplication.
\end{incompatibility}
\afternoterule
\end{defun}

\begin{defun}[Function]
/ number &rest more-numbers

The function \cdf{/}, when given more than one argument, successively divides
the first argument by all the others and returns the result.

\begin{new}%CORR
It is generally accepted that it is an error for any argument other than the
first to be zero.
\end{new}

With one argument, \cdf{/} reciprocates the argument.

\begin{new}%CORR
It is generally accepted that it is an error in this case for the argument
to be zero.
\end{new}

\cdf{/} will produce a ratio if the mathematical quotient of two integers
is not an exact integer.  For example:
\begin{lisp}
(/ 12 4) \EV\ 3 \\
(/ 13 4) \EV\ 13/4 \\
(/ -8) \EV\ -1/8 \\
(/ 3 4 5) \EV\ 3/20
\end{lisp}
To divide one integer by another producing an integer result,
use one of the functions \cdf{floor}, \cdf{ceiling}, \cdf{truncate},
or \cdf{round}.

If any argument is a floating-point number,
then the rules of floating-point contagion apply.

\beforenoterule
\begin{incompatibility}
What \cdf{/} does is totally unlike what the usual
\cd{//} or \cdf{quotient} operator does.  In most Lisp systems,
\cdf{quotient} behaves like \cdf{/} except when dividing integers,
in which case it behaves like \cdf{truncate} of two arguments;
this behavior is mathematically intractable, leading to such
anomalies as
\begin{lisp}
(quotient 1.0 2.0) \EV\ 0.5   {\rm but}   (quotient 1 2) \EV\ 0
\end{lisp}
In contrast, the Common Lisp function \cdf{/} produces these results:
\begin{lisp}
(/ 1.0 2.0) \EV\ 0.5          {\rm and}   (/ 1 2) \EV\ 1/2
\end{lisp}
In practice \cdf{quotient} is used only when one is sure that both arguments
are integers, {\it or} when one is sure that at least one argument
is a floating-point number.  \cdf{/} is tractable for its purpose
and works for {\it any} numbers.
\end{incompatibility}
\afternoterule
\end{defun}

\begin{defun}[Function]
1+ number \\
1- number

\cd{(1+ {\it x})} is the same as \cd{(+ {\it x} 1)}.

\cd{(1- {\it x})} is the same as \cd{(- {\it x} 1)}.
Note that the short name may be confusing: \cd{(1- {\it x})} does {\it not} mean
$1-{\it x}$; rather, it means ${\it x}-1$.

\beforenoterule
\begin{rationale}
These are included primarily for compatibility with MacLisp
and Lisp Machine Lisp.  Some programmers prefer always to write \cd{(+ x 1)} and
\cd{(- x 1)} instead of \cd{(1+ x)} and \cd{(1- x)}.
\end{rationale}
\betweennoterule
\begin{implementation}
Compiler writers are very strongly encouraged to ensure
that \cd{(1+ x)} and \cd{(+ x 1)} compile into identical code, and
similarly for \cd{(1- x)} and \cd{(- x 1)}, to avoid pressure on a Lisp
programmer to write possibly less clear code for the sake of efficiency.
This can easily be done as a source-language transformation.
\end{implementation}
\afternoterule
\end{defun}

\begin{defmac}
incf place [delta] \\
decf place [delta]

The number produced by the form {\it delta}
is added to (\cdf{incf}) or subtracted from (\cdf{decf})
the number in the generalized variable named by {\it place},
and the sum is stored back into {\it place} and returned.
The form {\it place} may be any form acceptable
as a generalized variable to \cdf{setf}.
If {\it delta} is not supplied, then the number in {\it place} is changed
by \cd{1}.
For example:
\begin{lisp}
\hskip 9pc\=\kill
(setq n 0) \\
(incf n) \EV\ 1\>{\rm and now} n \EV\ 1 \\
(decf n 3) \EV\ -2\>{\rm and now} n \EV\ -2 \\
(decf n -5) \EV\ 3\>{\rm and now} n \EV\ 3 \\
(decf n) \EV\ 2\>{\rm and now} n \EV\ 2
\end{lisp}
The effect of \cd{(incf {\it place} {\it delta})}
is roughly equivalent to
\begin{lisp}
(setf {\it place} (+ {\it place} {\it delta}))
\end{lisp}
except that the latter would evaluate any subforms of {\it place}
twice, whereas \cdf{incf} takes care to evaluate them only once.
Moreover, for certain {\it place} forms \cdf{incf} may be
significantly more efficient than the \cdf{setf} version.
\begin{newer}
X3J13 voted in March 1988 \issue{PUSH-EVALUATION-ORDER}
to clarify order of evaluation (see section~\ref{SETF-SECTION}).
\end{newer}
\end{defmac}

\begin{defun}[Function]
conjugate number

This returns the complex conjugate of {\it number}.  The conjugate
of a non-complex number is itself.  For a complex number \cdf{z},
\begin{lisp}
(conjugate z) \EQ\ (complex (realpart z) (- (imagpart z)))
\end{lisp}
For example:
\begin{lisp}
(conjugate \#C(3/5 4/5)) \EV\ \#C(3/5 -4/5) \\
(conjugate \#C(0.0D0 -1.0D0)) \EV\ \#C(0.0D0 1.0D0) \\
(conjugate 3.7) \EV\ 3.7
\end{lisp}
\end{defun}

\begin{defun}[Function]
gcd &rest integers

This returns the greatest common divisor of all the arguments,
which must be integers.  The result of \cdf{gcd} is always a non-negative
integer.
If one argument is given, its absolute value is returned.
If no arguments are given, \cdf{gcd} returns \cd{0},
which is an identity for this operation.
For three or more arguments,
\begin{lisp}
(gcd {\it a} {\it b} {\it c} ... {\it z}) \EQ\ (gcd (gcd {\it a} {\it b}) {\it c} ... {\it z})
\end{lisp}

Here are some examples of the use of \cdf{gcd}:
\begin{lisp}
(gcd 91 -49) \EV\ 7 \\
(gcd 63 -42 35) \EV\ 7 \\
(gcd 5) \EV\ 5 \\
(gcd -4) \EV\ 4 \\
(gcd) \EV\ 0
\end{lisp}
\end{defun}

\begin{defun}[Function]
lcm integer &rest more-integers

This returns the least common multiple of its arguments,
which must be integers.
The result of \cdf{lcm} is always a non-negative integer.
For two arguments that are not both zero,
\begin{lisp}
(lcm {\it a} {\it b}) \EQ\ (/ (abs (* {\it a} {\it b})) (gcd {\it a} {\it b}))
\end{lisp}
If one or both arguments are zero,
\begin{lisp}
(lcm {\it a} 0) \EQ\ (lcm 0 {\it a}) \EQ\ 0
\end{lisp}

For one argument, \cdf{lcm} returns the absolute value of that argument.
For three or more arguments,
\begin{lisp}
(lcm {\it a} {\it b} {\it c} ... {\it z}) \EQ\ (lcm (lcm {\it a} {\it b}) {\it c} ... {\it z})
\end{lisp}

Some examples:
\begin{lisp}
(lcm 14 35) \EV\ 70 \\
(lcm 0 5) \EV\ 0 \\
(lcm 1 2 3 4 5 6) \EV\ 60
\end{lisp}

Mathematically, \cd{(lcm)} should return infinity.  Because Common Lisp
does not have a representation for infinity, \cdf{lcm}, unlike \cdf{gcd},
always requires at least one argument.

\begin{new}
X3J13 voted in January 1989
\issue{LCM-NO-ARGUMENTS}
to specify that \cd{(lcm)~\EV~1}.

This is one of my biggest boners.  The identity for \cdf{lcm} is of course 1,
not infinity, and so \cd{(lcm)} ought to have been defined to return 1.
Sorry about that, though in point of fact very few users have complained
to me that this mistake in the first edition has cramped their programming style.
\end{new}
\end{defun}

\section{Irrational and Transcendental Functions}
\label{TRANSCENDENTAL-SECTION}

Common Lisp provides no data type that can accurately represent irrational
numerical values.
The functions in this section are described as if the results
were mathematically accurate, but actually they all produce floating-point
approximations to the true mathematical result in the general case.
In some places
mathematical identities are set forth that are intended to elucidate the
meanings of the functions; however, two mathematically identical
expressions may be computationally different because of errors
inherent in the floating-point approximation process.

When the arguments to
a function in this section are all rational and the true mathematical result
is also (mathematically) rational, then unless otherwise noted
an implementation is free to return either an accurate result of
type \cdf{rational} or a single-precision floating-point approximation.
If the arguments are all rational but the result cannot be expressed
as a rational number, then a single-precision floating-point
approximation is always returned.

\begin{newer}
X3J13 voted in March 1989
\issue{COMPLEX-RATIONAL-RESULT}
to clarify that the provisions of the previous paragraph apply to complex
numbers.  If the arguments to a function are all of type
\cd{(or~rational (complex rational))} and the true mathematical result
is (mathematically) a complex number with rational real
and imaginary parts, then unless otherwise noted
an implementation is free to return either an accurate result of
type \cd{(or rational (complex rational))}
or a single-precision floating-point approximation
of type \cdf{single-float} (permissible only if the imaginary part
of the true mathematical result is zero) or \cd{(complex single-float)}.
If the arguments are all of type
\cd{(or~rational (complex rational))} but the result cannot be expressed
as a rational or complex rational number, then the returned value
will be of type \cdf{single-float} (permissible only if the imaginary part
of the true mathematical result is zero) or \cd{(complex single-float)}.
\end{newer}

The rules of floating-point contagion and complex contagion are 
effectively obeyed by all the functions in this section except \cdf{expt},
which treats some cases of rational exponents specially.
When, possibly after contagious conversion, all of the arguments are of
the same floating-point or complex floating-point type,
then the result will be of that same type unless otherwise noted.

\beforenoterule
\begin{implementation}
There is a ``floating-point cookbook'' by
Cody and Waite \cite{CODY-AND-WAITE} that may be a useful aid
in implementing the functions defined in this section.
\end{implementation}
\afternoterule

\subsection{Exponential and Logarithmic Functions}

Along with the usual one-argument and two-argument exponential and
logarithm functions, \cdf{sqrt} is considered to be an exponential
function, because it raises a number to the power 1/2.

\begin{defun}[Function]
exp number

Returns {\it e} raised to the power {\it number},
where {\it e} is the base of the natural logarithms.
\end{defun}

\begin{defun}[Function]
expt base-number power-number

Returns {\it base-number} raised to the power {\it power-number}.
If the {\it base-number} is of type \cdf{rational} and the {\it power-number} is
an \cdf{integer},
the calculation will be exact and the result will be of type \cdf{rational};
otherwise a floating-point approximation may result.

\begin{newer}
X3J13 voted in March 1989
\issue{COMPLEX-RATIONAL-RESULT}
to clarify that provisions similar to those of the previous paragraph apply to complex
numbers.  If the {\it base-number} is of type \cd{(complex~rational)}
and the {\it power-number} is
an \cdf{integer},
the calculation will also be exact and the result will be of type
\cd{(or~rational (complex rational))};
otherwise a floating-point or complex floating-point approximation may result.
\end{newer}

When {\it power-number} is \cd{0} (a zero of type integer),
then the result is always the value 1 in the type of {\it base-number},
even if the {\it base-number} is zero (of any type).  That is:
\begin{lisp}
(expt {\it x} 0) \EQ\ (coerce 1 (type-of {\it x}))
\end{lisp}
If the {\it power-number} is a zero of any other data type,
then the result is also the value 1, in the type of the arguments
after the application of the contagion rules, with one exception:
it is an error if {\it base-number} is zero when the {\it power-number}
is a zero not of type integer.

Implementations of \cdf{expt} are permitted to use different algorithms
for the cases of a rational {\it power-number} and a floating-point
{\it power-number}; the motivation is that in many cases greater accuracy
can be achieved for the case of a rational {\it power-number}.
For example, \cd{(expt pi 16)} and \cd{(expt pi 16.0)} may yield
slightly different results if the first case is computed by repeated squaring
and the second by the use of logarithms.  Similarly, an implementation
might choose to compute \cd{(expt x 3/2)} as if it had
been written \cd{(sqrt (expt x 3))}, perhaps producing a more accurate
result than would \cd{(expt x 1.5)}.  It is left to the implementor
to determine the best strategies.

\begin{new}
X3J13 voted in January 1989
\issue{EXPT-RATIO}
to clarify that the preceding remark is in
error, because \cd{(sqrt (expt~x~3))} does not produce the same value
as \cd{(expt~x~3/2)} in most cases, and to specify that the
specification of the principal value of \cdf{expt} as given in section~\ref{BRANCH-CUTS-SECTION}
should be regarded as definitive.

As an example of the difficulty, let
$ {\it x}=\cis {2 \pi \over 3}= -{1 \over 2} + {\sqrt{3} \over 2}{\it i} $.
Then $ \sqrt{{\it x}^3} = \sqrt{1} = 1 $, but
$ {\it x}^{3 / 2} = {\it e}^{(3/2) \log {\hbox{\scriptsize\it x}}}
   = {\it e}^{(3/2) (2\pi/3) i} = {\it e}^{\pi i} = -1 $.
Another example is $x=-1$; then $ \sqrt{x^3} = \sqrt{-1} = i$, but
$ x^{3/2} = e^{(3/2) \log {x}} = e^{(3/2) \pi i} = -i $.
\end{new}

The result of \cdf{expt} can be a complex number, even when neither argument
is complex, if {\it base-number} is negative and {\it power-number}
is not an integer.  The result is always the principal complex value.
Note that \cd{(expt -8 1/3)} is not permitted to return \cd{-2};
while \cd{-2} is indeed one of the cube roots of \cd{-8}, it is
not the principal cube root, which is a complex number
approximately equal to \cd{\#C(1.0 1.73205)}.

\begin{new}%CORR
{\it Notice of correction.}  The first edition gave the incorrect value
\cd{\#C(0.5 1.73205)} for the principal cube root of \cd{-8}.  The correct
value is \cd{\#C(1.0 1.73205)}, that is, $1+\sqrt{3}i$.  I simply don't know what
I was thinking of!
\end{new}
\end{defun}


\begin{defun}[Function]
log number &optional base

Returns the logarithm of {\it number} in the base {\it base},
which defaults to {\it e}, the base of the natural logarithms.
For example:
\begin{lisp}
(log 8.0 2) \EV\ 3.0 \\
(log 100.0 10) \EV\ 2.0
\end{lisp}
The result of \cd{(log 8 2)} may be either \cd{3} or \cd{3.0}, depending on the
implementation.

Note that \cdf{log} may return a complex result when given a non-complex
argument if the argument is negative.  For example:
\begin{lisp}
(log -1.0) \EQ\ (complex 0.0 (float pi 0.0))
\end{lisp}

\begin{new}
X3J13 voted in January 1989
\issue{IEEE-ATAN-BRANCH-CUT}
to specify certain floating-point behavior when minus zero is supported.
As a part of that vote it approved a mathematical definition of complex logarithm
in terms of real logarithm, absolute value,
arc tangent of two real arguments, and the phase function as
\begin{tabbing}
\hskip 10pc\=\kill
Logarithm\>$ \log \left| {\it z} \right| + {\it i} \phase {\it z} $
\end{tabbing}
This specifies the branch cuts precisely whether minus zero is supported or not;
see \cdf{phase} and \cdf{atan}.
\end{new}
\end{defun}

\begin{defun}[Function]
sqrt number

Returns the principal square root of {\it number}.
If the {\it number} is not complex but is negative, then the result
will be a complex number.
For example:
\begin{lisp}
(sqrt 9.0) \EV\ 3.0 \\
(sqrt -9.0) \EV\ \#c(0.0 3.0)
\end{lisp}
The result of \cd{(sqrt 9)} may be either \cd{3} or \cd{3.0}, depending on the
implementation.  The result of \cd{(sqrt -9)} may be either \cd{\#c(0 3)}
or \cd{\#c(0.0 3.0)}.

\begin{new}
X3J13 voted in January 1989
\issue{IEEE-ATAN-BRANCH-CUT}
to specify certain floating-point behavior when minus zero is supported.
As a part of that vote it approved a mathematical definition of complex square root
in terms of complex logarithm and exponential functions as
\begin{tabbing}
\hskip 10pc\=\kill
Square root\>$ {\it e}^{(\log z)/2} $
\end{tabbing}
This specifies the branch cuts precisely whether minus zero is supported or not;
see \cdf{phase} and \cdf{atan}.
\end{new}
\end{defun}

\begin{defun}[Function]
isqrt integer

Integer square root: the argument must be a non-negative integer, and the
result is the greatest integer less than or equal to the exact positive
square root of the argument.
For example:
\begin{lisp}
(isqrt 9) \EV\ 3 \\
(isqrt 12) \EV\ 3 \\
(isqrt 300) \EV\ 17 \\
(isqrt 325) \EV\ 18
\end{lisp}
\end{defun}

\subsection{Trigonometric and Related Functions}

Some of the functions in this section, such as \cdf{abs}
and \cdf{signum}, are apparently unrelated
to trigonometric functions when considered as functions of
real numbers only.  The way in which they are extended to
operate on complex numbers makes the trigonometric connection clear.

\begin{defun}[Function]
abs number

Returns the absolute value of the argument.  For a non-complex number {\it x},
\begin{lisp}
(abs {\it x}) \EQ\ (if (minusp {\it x}) (- {\it x}) {\it x})
\end{lisp}
and the result is always of the same type as the argument.

For a complex number {\it z}, the absolute value may be computed as
\begin{lisp}
(sqrt (+ (expt (realpart {\it z}) 2) (expt (imagpart {\it z}) 2)))
\end{lisp}

\beforenoterule
\begin{implementation}
The careful implementor will not use this formula directly
for all complex numbers
but will instead handle very large or very small components specially
to avoid intermediate overflow or underflow.
\end{implementation}
\afternoterule

For example:
\begin{lisp}
(abs \#c(3.0 -4.0)) \EV\ 5.0
\end{lisp}
The result of \cd{(abs \#c(3 4))} may be either \cd{5} or \cd{5.0},
depending on the implementation.
\end{defun}

\begin{defun}[Function]
phase number

The phase of a number is the angle part of its polar representation
as a complex number.  That is,
\begin{lisp}
(phase {\it z}) \EQ\ (atan (imagpart {\it z}) (realpart {\it z}))
\end{lisp}
\begin{obsolete}
The result is in radians, in the range $-\pi$ (exclusive)
to $\pi$ (inclusive).  The phase of a positive non-complex number
is zero; that of a negative non-complex number is $\pi$.
The phase of zero is arbitrarily defined to be zero.
\end{obsolete}

\begin{new}
X3J13 voted in January 1989
\issue{IEEE-ATAN-BRANCH-CUT}
to specify certain floating-point behavior when minus zero is supported;
\cdf{phase} is still defined in terms of \cdf{atan} as above,
but thanks to a change in \cdf{atan} the range of \cdf{phase}
becomes $-\pi$ {\it inclusive} to $\pi$ inclusive.  The value $-\pi$
results from an argument\vadjust{\penalty-10000}
whose real part is negative and whose imaginary
part is minus zero.  The \cdf{phase} function therefore has a branch cut
along the negative real axis.  The phase of $+0+0{\it i}$ is $+0$, of $+0-0{\it i}$ is $-0$,
of $-0+0{\it i}$ is $+\pi$, and of $-0-0{\it i}$ is $-\pi$.
\end{new}

If the argument is a complex floating-point number, the result
is a floating-point number of the same type as the components of
the argument.
If the argument is a floating-point number, the result is a
floating-point number of the same type.
If the argument is a rational number or complex rational number, the result
is a single-format floating-point number.
\end{defun}

\begin{defun}[Function]
signum number

By definition,
\begin{lisp}
(signum {\it x}) \EQ\ (if (zerop {\it x}) {\it x} (/ {\it x} (abs {\it x})))
\end{lisp}
For a rational number, \cdf{signum} will return one of \cd{-1}, \cd{0}, or \cd{1}
according to whether the number is negative, zero, or positive.
For a floating-point number, the result will be a floating-point number
of the same format whose value is $-1$, $0$, or $1$.
For a complex number {\it z}, \cd{(signum {\it z})} is a complex number of
the same phase but with unit magnitude, unless {\it z} is a complex zero,
in which case the result is {\it z}.
For example:
\begin{lisp}
(signum 0) \EV\ 0 \\
(signum -3.7L5) \EV\ -1.0L0 \\
(signum 4/5) \EV\ 1 \\
(signum \#C(7.5 10.0)) \EV\ \#C(0.6 0.8) \\
(signum \#C(0.0 -14.7)) \EV\ \#C(0.0 -1.0)
\end{lisp}
For non-complex rational numbers, \cdf{signum} is a rational function,
but it may be irrational for complex arguments.
\end{defun}

\begin{defun}[Function]
sin radians \\
cos radians \\
tan radians

\cdf{sin} returns the sine of the argument, \cdf{cos} the cosine,
and \cdf{tan} the tangent.  The argument is in radians.
The argument may be complex.
\end{defun}

\begin{defun}[Function]
cis radians

This computes $ {\it e} ^ {i \cdot radians} $.
The name \cdf{cis} means ``cos + {\it i} sin,'' because
$ {\it e} ^{ i \theta } = \cos \theta + {\it i} \sin \theta $.
The argument is in
radians and may be any non-complex number.  The result is a complex
number whose real part is the cosine of the argument and whose imaginary
part is the sine.  Put another way, the result is a complex number whose
phase is the equal to the argument (mod $2\pi$)
and whose magnitude is unity.

\beforenoterule
\begin{implementation}
Often it is cheaper to calculate the sine and cosine
of a single angle together than to perform two disjoint calculations.
\end{implementation}
\afternoterule
\end{defun}

\begin{defun}[Function]
asin number \\
acos number

\cdf{asin} returns the arc sine of the argument, and \cdf{acos} the arc cosine.
The result is in radians.  The argument may be complex.

The arc sine and arc cosine functions may be defined mathematically for
an argument {\it z} as follows:
\begin{tabbing}
\hskip 10pc\=\kill
Arc sine\>$ -{\it i} \log \left({\it i}{\it z} + \sqrt{1-{\it z}^2}\right) $ \\[2pt]
Arc cosine\>$ -{\it i} \log \left({\it z} + {\it i}\sqrt{1-{\it z}^2}\right) $
\end{tabbing}
Note that the result of \cdf{asin} or \cdf{acos} may be
complex even if the argument is not complex; this occurs
when the absolute value of the argument is greater than 1.

\begin{newer}
Kahan \cite{KAHAN-COMPLEX-FNS} suggests for \cdf{acos} the
defining formula
\begin{tabbing}
\hskip 10pc\=\kill
Arc cosine\>$  \displaystyle { 2 \log \left( \sqrt{{1+{\it z} \over 2}} + {\it i} \sqrt{{1-{\it z} \over 2}} \right) \over i}$
\end{tabbing}
or even the much simpler $ (\pi/2)-\arcsin {\it z} $.  Both equations are mathematically
equivalent to the formula shown above.
\end{newer}

\beforenoterule
\begin{implementation}
These formulae are mathematically correct, assuming
completely accurate computation.  They may be terrible methods for
floating-point computation.  Implementors should consult a good text on
numerical analysis.  The formulae given above are not necessarily
the simplest ones for real-valued computations, either; they are chosen
to define the branch cuts in desirable ways for the complex case.
\end{implementation}
\afternoterule
\end{defun}

\begin{defun}[Function]
atan y &optional x

An arc tangent is calculated and the result is returned in radians.

With two arguments {\it y} and {\it x}, neither argument may be complex.
The result is the arc tangent of the quantity {\it y/x}.
The signs of {\it y} and {\it x} are used to derive quadrant
information; moreover, {\it x} may be zero provided
{\it y} is not zero.  The value of \cdf{atan} is always between
$-\pi$ (exclusive) and $\pi$ (inclusive).
The following table details various special cases.

\begin{flushleft}
\begin{tabular*}{\linewidth}{@{}l@{\extracolsep{\fill}}llc@{}}
\multicolumn{2}{c}{Condition}&Cartesian Locus&Range of Result \\
\hlinesp
\hbox to 0.4in{${\it y}=0$\hss}&\hbox to 0.4in{${\it x}>0$\hss}&\hbox to 1.6in{Positive {\it x}-axis\hss}&$0$ \\
${\it y}>0$&${\it x}>0$&Quadrant I&$0 < {\rm result} < \pi/2$ \\
${\it y}>0$&${\it x}=0$&Positive {\it y}-axis&$\pi/2$ \\
${\it y}>0$&${\it x}<0$&Quadrant II&$\pi/2 < {\rm result} < \pi$ \\
${\it y}=0$&${\it x}<0$&Negative {\it x}-axis&$\pi$ \\
${\it y}<0$&${\it x}<0$&Quadrant III&$-\pi < {\rm result} < -\pi/2$ \\
${\it y}<0$&${\it x}=0$&Negative {\it y}-axis&$-\pi/2$ \\
${\it y}<0$&${\it x}>0$&Quadrant IV&$-\pi/2 < {\rm result} < 0$ \\
${\it y}=0$&${\it x}=0$&Origin&error \\
\hline
\end{tabular*}
\end{flushleft}
\vskip 0pt plus 6pt\relax%manual
\begin{new}
X3J13 voted in January 1989
\issue{IEEE-ATAN-BRANCH-CUT}
to specify certain floating-point behavior when minus zero is supported.
When there is a minus zero, the preceding table must be modified slightly:

\begin{flushleft}
\begin{tabular*}{\linewidth}{@{}l@{\extracolsep{\fill}}llc@{}}
\multicolumn{2}{c}{Condition}&Cartesian Locus&Range of Result \\
\hlinesp
\hbox to 0.4in{${\it y}=+0$\hss}&\hbox to 0.4in{${\it x}>0$\hss}&\hbox to 1.6in{Just above positive {\it x}-axis\hss}&$+0$ \\
${\it y}>0$&${\it x}>0$&Quadrant I&$+0 < {\rm result} < \pi/2$ \\
${\it y}>0$&${\it x}=\pm 0$&Positive {\it y}-axis&$\pi/2$ \\
${\it y}>0$&${\it x}<0$&Quadrant II&$\pi/2 < {\rm result} < \pi$ \\
${\it y}=+0$&${\it x}<0$&Just below negative {\it x}-axis&$\pi$ \\
${\it y}=-0$&${\it x}<0$&Just above negative {\it x}-axis&$\pi$ \\
${\it y}<0$&${\it x}<0$&Quadrant III&$-\pi < {\rm result} < -\pi/2$ \\
${\it y}<0$&${\it x}=\pm 0$&Negative {\it y}-axis&$-\pi/2$ \\
${\it y}<0$&${\it x}>0$&Quadrant IV&$-\pi/2 < {\rm result} < -0$ \\
${\it y}=-0$&${\it x}>0$&Just below positive {\it x}-axis&$-0$ \\
${\it y}=+0$&${\it x}=+0$&Near origin&$+0$ \\
${\it y}=-0$&${\it x}=+0$&Near origin&$-0$ \\
${\it y}=+0$&${\it x}=-0$&Near origin&$\pi$ \\
${\it y}=-0$&${\it x}=-0$&Near origin&$-\pi$ \\
\hline
\end{tabular*}
\end{flushleft}
\vskip 0pt plus 6pt\relax%manual

Note that the case ${\it y}=0,{\it x}=0$ is an error in the absence of minus zero,
but the four cases ${\it y}=\pm 0,{\it x}=\pm 0$ are defined in the presence of minus zero.
\end{new}

\begin{obsolete}
With only one argument {\it y}, the argument may be complex.
The result is the arc tangent of {\it y}, which may be defined by
the following formula:
\begin{tabbing}
\hskip 10pc\=\kill
Arc tangent\>$ -{\it i}\log \left((1+{\it i}{\it y}) \sqrt{1/(1+{\it y}^2)}\right) $
\end{tabbing}
\end{obsolete}

\newpage%manual

\beforenoterule
\begin{implementation}
This formula is mathematically correct, assuming
completely accurate computation.  It may be a terrible method for
floating-point computation.  Implementors should consult a good text on
numerical analysis.  The formula given above is not necessarily
the simplest one for real-valued computations, either; it is chosen
to define the branch cuts in desirable ways for the complex case.
\end{implementation}
\afternoterule

\begin{new}
X3J13 voted in January 1989
\issue{COMPLEX-ATAN-BRANCH-CUT}
to replace the preceding formula with the formula
\begin{tabbing}
\hskip 10pc\=\kill
Arc tangent\>$ \displaystyle { \log (1+{\it i}{\it y}) - \log (1-{\it i}{\it y}) \over 2{\it i} } $
\end{tabbing}
This change alters the direction of continuity for the
branch cuts, which alters the result returned by \cdf{atan}
only for arguments on the imaginary axis that
are of magnitude greater than 1.
See section~\ref{BRANCH-CUTS-SECTION} for further details.
\end{new}

For a non-complex argument {\it y}, the result is non-complex and lies between
$-\pi/2$ and $\pi/2$ (both exclusive).

\beforenoterule
\begin{incompatibility}
MacLisp has a function called \cdf{atan} whose
range is from 0 to $2\pi$.  Almost every other programming language
(ANSI Fortran, IBM PL/1, Interlisp) has a two-argument arc tangent
function with range $-\pi$ to $\pi$.
Lisp Machine Lisp provides two two-argument
arc tangent functions, \cdf{atan} (compatible with MacLisp)
and \cd{atan2} (compatible with all others).

Common Lisp makes two-argument \cdf{atan} the standard one
with range $-\pi$ to $\pi$.  Observe that this makes
the one-argument and two-argument versions of \cdf{atan} compatible
in the sense that the branch cuts do not fall in different places.
The Interlisp one-argument function \cdf{arctan} has a range
from 0 to $\pi$, while nearly every other programming language
provides the range $-\pi/2$ to $\pi/2$ for
one-argument arc tangent!
Nevertheless, since Interlisp uses the standard two-argument
version of arc tangent, its branch cuts are inconsistent anyway.
\end{incompatibility}
\afternoterule
\end{defun}

\begin{defun}[Constant]
pi

This global variable has as its value the best possible approximation to
$\pi$ in {\it long} floating-point format.
For example:
\begin{lisp}
(defun sind (x)~~~~~;{\rm The argument is in degrees} \\
~~(sin (* x (/ (float pi x) 180))))
\end{lisp}
An approximation to $\pi$ in some other precision can
be obtained by writing \cd{(float pi {\it x})}, where {\it x} is a
floating-point number of the desired precision,
or by writing \cd{(coerce pi {\it type})}, where {\it type} is the
name of the desired type, such as \cdf{short-float}.
\end{defun}

\penalty-10000\relax

\begin{defun}[Function]
sinh number \\
cosh number \\
tanh number \\
asinh number \\
acosh number \\
atanh number

\begin{obsolete}\noindent
These functions compute the hyperbolic sine, cosine, tangent,
arc sine, arc cosine, and arc tangent functions, which are mathematically
defined for an argument {\it z} as follows:

\begin{tabbing}
\hskip 10pc\=\kill
Hyperbolic sine\>$ ({\it e}^{z}-{\it e}^{-z})/2 $ \\
Hyperbolic cosine\>$ ({\it e}^{z}+{\it e}^{-z})/2 $ \\
Hyperbolic tangent\>$ ({\it e}^{z}-{\it e}^{-z})/({\it e}^{z}+{\it e}^{-z}) $ \\[2pt]
Hyperbolic arc sine\>$ \log \left({\it z}+\sqrt{1+{\it z}^2}\right) $ \\[2pt]
Hyperbolic arc cosine\>$ \log \left({\it z}+({\it z}+1)\sqrt{({\it z}-1)/({\it z}+1)}\right) $ \\[2pt]
Hyperbolic arc tangent\>$ \log \left((1+{\it z})\sqrt{1-1/{\it z}^2}\right) $\`{\bf WRONG!}
\end{tabbing}
\end{obsolete}

\begin{newer}
WARNING!  {\it The formula shown above for hyperbolic arc tangent is incorrect.}
It is not a matter of incorrect branch cuts; it simply does not compute anything
like a hyperbolic arc tangent.  This unfortunate error in the first edition
was the result of mistranscribing a (correct) APL formula from Penfield's paper
\cite{APL-BRANCH-CUTS}.  The formula should have been transcribed as
\begin{tabbing}
\hskip 10pc\=\kill
Hyperbolic arc tangent\>$ \log \left((1+{\it z})\sqrt{1/(1-{\it z}^2)}\right) $
\end{tabbing}
A proposal was submitted to X3J13 in September 1989 to replace the
formulae for \cdf{acosh} and \cdf{atanh}.
See section~\ref{BRANCH-CUTS-SECTION} for further discussion.
\end{newer}

Note that the result of \cdf{acosh} may be
complex even if the argument is not complex; this occurs
when the argument is less than 1.
Also, the result of \cdf{atanh} may be
complex even if the argument is not complex; this occurs
when the absolute value of the argument is greater than 1.

\beforenoterule
\begin{implementation}
These formulae are mathematically correct, assuming
completely accurate computation.  They may be terrible methods for
floating-point computation.  Implementors should consult a good text on
numerical analysis.  The formulae given above are not necessarily
the simplest ones for real-valued computations, either; they are chosen
to define the branch cuts in desirable ways for the complex case.
\end{implementation}
\afternoterule
\end{defun}

\subsection{Branch Cuts, Principal Values, and Boundary Conditions in the Complex Plane}
\label{BRANCH-CUTS-SECTION}



Many of the irrational and transcendental functions are multiply defined
in the complex domain; for example, there are in general an infinite
number of complex values for the logarithm function.  In each such
case, a principal value must be chosen for the function to return.
In general, such values cannot be chosen so as to make the range
continuous; lines in the domain
called {\it branch cuts} must be defined, which in turn
define the discontinuities in the range.

Common Lisp defines the branch cuts, principal values, and boundary
conditions for the complex functions following
a proposal for complex functions in APL \cite{APL-BRANCH-CUTS}.
The contents of this section are borrowed largely from that proposal.

\beforenoterule
\begin{incompatibility}
The branch cuts defined here differ in a few very minor
respects from those advanced by W.~Kahan, who considers not only the
``usual'' definitions but also the special modifications necessary for
{IEEE} proposed floating-point arithmetic, which has infinities and
minus zero as explicit computational objects.  For example, he proposes
that $\sqrt{-4+0{\it i}}=2{\it i}$, but $\sqrt{-4-0{\it i}}=-2{\it i}$.

It may be that the differences between the APL proposal and Kahan's
proposal will be ironed out.  If so, Common Lisp may be
changed as necessary to be compatible with these other groups.  Any changes
from the specification below are likely to be quite minor,
probably concerning primarily questions of which side of a branch cut
is continuous with the cut itself.
\end{incompatibility}
\afternoterule

\begin{new}
Indeed, X3J13 voted in January 1989
\issue{COMPLEX-ATAN-BRANCH-CUT}
to alter the direction of continuity for
the branch cuts of \cdf{atan}, and also
\issue{IEEE-ATAN-BRANCH-CUT}
to address the treatment of branch cuts
in implementations that have a distinct floating-point minus zero.

The treatment of minus zero centers in two-argument \cdf{atan}.
If there is no minus zero, then the branch cut runs just below the negative real
axis as before, and the range of two-argument \cdf{atan} is $(-\pi,\pi]$.
If there is a minus zero, however, then the branch cut runs precisely on the negative real
axis, skittering between pairs of numbers of the form $-{\it x}\pm 0{\it i}$,
and the range of two-argument \cdf{atan} is $[-\pi,\pi]$.

The treatment of minus zero by all other irrational and transcendental functions
is then specified by defining those functions in terms of two-argument \cdf{atan}.
First, \cdf{phase} is defined in terms of two-argument \cdf{atan}, and
complex \cdf{abs} in terms of real \cdf{sqrt};
then complex \cdf{log} is defined in terms of \cdf{phase}, \cdf{abs}, and real \cdf{log};
then complex \cdf{sqrt} in terms of complex \cdf{log};
and finally all others are defined in terms of these.

Kahan \cite{KAHAN-COMPLEX-FNS} treats these matters in some detail and also
suggests specific algorithms for implementing irrational and transcendental functions
in IEEE standard floating-point arithmetic \cite{IEEE-PROPOSED-FLOATING-POINT-STANDARD}.

Remarks in the first edition about the direction of the continuity of branch
cuts continue to hold in the absence of minus zero and may be ignored if minus zero
is supported; since all branch cuts happen to run along the principal axes, they
run {\it between} plus zero and minus zero, and so each sort of zero is associated
with the obvious quadrant.
\end{new}

\begin{flushdesc}
\item[\cdf{sqrt}]
The branch cut for square root lies along the negative real axis,
continuous with quadrant II.
The range consists of the right half-plane, including the non-negative
imaginary axis and excluding the negative imaginary axis.

\begin{new}
X3J13 voted in January 1989
\issue{IEEE-ATAN-BRANCH-CUT}
to specify certain floating-point behavior when minus zero is supported.
As a part of that vote it approved a mathematical definition of complex square root:
\begin{tabbing}
$ \sqrt{{\it z}} = {\it e}^{(\log z)/2} $
\end{tabbing}
This defines the branch cuts precisely, whether minus zero is supported or not.
\end{new}

\item[\cdf{phase}]
The branch cut for the phase function lies along the negative real
axis, continuous with quadrant II.  The range consists of that portion of
the real axis between $-\pi$ (exclusive) and $\pi$ (inclusive).

\begin{new}
X3J13 voted in January 1989
\issue{IEEE-ATAN-BRANCH-CUT}
to specify certain floating-point behavior when minus zero is supported.
As a part of that vote it approved a mathematical definition of phase:
\begin{tabbing}
$ \phase {\it z} = \arctan (\Im {\it z}, \Re {\it z}) $
\end{tabbing}
where $\Im {\it z}$ is the imaginary part of ${\it z}$ and $\Re {\it z}$ the real part of ${\it z}$.
This defines the branch cuts precisely, whether minus zero is supported or not.
\end{new}

\item[\cdf{log}]
The branch cut for the logarithm function of one argument (natural
logarithm) lies along the negative real axis, continuous with quadrant II.
The domain excludes the origin.  For a complex number ${\it z}$,
$\log {\it z}$ is defined to be
\begin{tabbing}
$ \log {\it z} = \left(\log \left|{\it z}\right|\right)+{\it i} (\phase {\it z}) $
\end{tabbing}
Therefore the range of the one-argument logarithm function is that strip
of the complex plane containing numbers with imaginary parts between
$-\pi$ (exclusive) and $\pi$ (inclusive).

\begin{new}
The X3J13 vote on minus zero
\issue{IEEE-ATAN-BRANCH-CUT}
would alter that exclusive bound of $-\pi$  to be inclusive if minus zero is supported.
\end{new}

The two-argument logarithm function is defined as $ \log_{b} {\it z}=(\log {\it z})/(\log {\it b}) $.
This defines the principal values precisely.  The range of the two-argument
logarithm function is the entire complex plane.
It is an error if ${\it z}$ is zero.  If ${\it z}$ is non-zero and ${\it b}$ is zero,
the logarithm is taken to be zero.

\item[\cdf{exp}]
The simple exponential function has no branch cut.

\item[\cdf{expt}]
The two-argument exponential function is defined
as $ {\it b}^{x}={\it e}^{x \log b } $.
This defines the principal values precisely.  The range of the
two-argument exponential function is the entire complex plane.  Regarded
as a function of ${\it x}$, with ${\it b}$ fixed, there is no branch cut.
Regarded as a function of {\it b}, with ${\it x}$ fixed, there is in general
a branch cut along the negative real axis, continuous with quadrant II.
The domain excludes the origin.
By definition, $0^0=1$.  If ${\it b}=0$ and the real part of ${\it x}$ is strictly
positive, then ${\it b}^{x}=0$.
For all other values of ${\it x}$, $0^{x}$
is an error.

\item[\cdf{asin}]
The following definition for arc sine determines the range and
branch cuts:
\begin{tabbing}
$ \arcsin {\it z}=-{\it i} \log \left({\it i}{\it z}+\sqrt{1-{\it z}^2}\right) $
\end{tabbing}

\begin{newer}\noindent
This is equivalent to the formula
\begin{tabbing}
$ \displaystyle \arcsin {\it z} = { \arcsinh {\it i}{\it z} \over {\it i} }$
\end{tabbing}
recommended by Kahan \cite{KAHAN-COMPLEX-FNS}.
\end{newer}

The branch cut for the arc sine function is in two pieces:
one along the negative real axis to the left of $-1$
(inclusive), continuous with quadrant II, and one along the positive real
axis to the right of 1 (inclusive), continuous with quadrant IV.  The
range is that strip of the complex plane containing numbers whose real
part is between $-\pi/2$ and $\pi/2$.  A number with real
part equal to $-\pi/2$ is in the range if and only if its imaginary
part is non-negative; a number with real part equal to $\pi/2$ is in
the range if and only if its imaginary part is non-positive.

\item[\cdf{acos}]
The following definition for arc cosine determines the range and
branch cuts:
\begin{tabbing}
$ \arccos {\it z}=-{\it i} \log \left({\it z}+{\it i} \sqrt{1-{\it z}^2}\right) $
\end{tabbing}
or, which is equivalent,
\begin{tabbing}
$ \arccos {\it z}={\pi \over 2}-\arcsin {\it z} $
\end{tabbing}
The branch cut for the arc cosine function is in two pieces:
one along the negative real axis to the left of $-1$
(inclusive), continuous with quadrant II, and one along the positive real
axis to the right of 1 (inclusive), continuous with quadrant IV.  
This is the same branch cut as for arc sine.
The range is that strip of the complex plane containing numbers whose real
part is between zero and $\pi$.  A number with real
part equal to zero is in the range if and only if its imaginary
part is non-negative; a number with real part equal to $\pi$ is in
the range if and only if its imaginary part is non-positive.

\item[\cdf{atan}]
The following definition for (one-argument) arc tangent determines the
range and branch cuts:
\begin{obsolete}
\noindent\leavevmode\vtop{
\begin{tabbing}
$ \arctan {\it z}=-{\it i} \log \left((1+{\it i} {\it z}) \sqrt{1/(1+{\it z}^2)}\right)$
\end{tabbing}
}

Beware of simplifying this formula; ``obvious'' simplifications are likely
to alter the branch cuts or the values on the branch cuts incorrectly.

The branch cut for the arc tangent function is in two pieces:
one along the positive imaginary axis above {\it i}
(exclusive), continuous with quadrant II, and one along the negative imaginary
axis below $-{\it i}$ (exclusive), continuous with quadrant IV.  
The points {\it i} and $-{\it i}$ are excluded from the domain.
The range is that strip of the complex plane containing numbers whose real
part is between $-\pi/2$ and $\pi/2$.  A number with real
part equal to $-\pi/2$ is in the range if and only if its imaginary
part is strictly positive; a number with real part equal to $\pi/2$ is in
the range if and only if its imaginary part is strictly negative.  Thus the range of
the arc tangent function is identical to that of the arc sine function with the points
$-\pi/2$ and $\pi/2$ excluded.
\end{obsolete}

\begin{new}
X3J13 voted in January 1989
\issue{COMPLEX-ATAN-BRANCH-CUT}
to replace the formula shown above with the formula
\begin{tabbing}
$ \displaystyle \arctan {\it z} = { \log (1+{\it i}{\it z}) - \log (1-{\it i}{\it z}) \over 2{\it i} }$
\end{tabbing}
This is equivalent to the formula
\begin{tabbing}
$ \displaystyle \arctan {\it z} = { \arctanh {\it i}{\it z} \over {\it i} }$
\end{tabbing}
recommended by Kahan \cite{KAHAN-COMPLEX-FNS}.
It causes the upper branch cut to be continuous with
quadrant I rather than quadrant II, and the lower branch cut to
be continuous with quadrant III rather than quadrant IV; otherwise it agrees with the
formula of the first edition.  Therefore this change alters the result returned by \cdf{atan}
only for arguments on the positive imaginary axis that
are of magnitude greater than 1.  The full description for this new formula is as follows.

The branch cut for the arc tangent function is in two pieces:
one along the positive imaginary axis above {\it i}
(exclusive), continuous with quadrant I, and one along the negative imaginary
axis below $-{\it i}$ (exclusive), continuous with quadrant III.  
The points {\it i} and $-{\it i}$ are excluded from the domain.
The range is that strip of the complex plane containing numbers whose real
part is between $-\pi/2$ and $\pi/2$.  A number with real
part equal to $-\pi/2$ is in the range if and only if its imaginary
part is strictly negative; a number with real part equal to $\pi/2$ is in
the range if and only if its imaginary part is strictly positive.  Thus the range of
the arc tangent function is {\it not} identical to that of the arc sine function.
\end{new}

\item[\cdf{asinh}]
The following definition for the inverse hyperbolic sine determines
the range and branch cuts:
\begin{tabbing}
$ \arcsinh {\it z}=\log \left({\it z}+\sqrt{1+{\it z}^2}\right)$
\end{tabbing}
The branch cut for the inverse hyperbolic sine function is in two pieces:
one along the positive imaginary axis above {\it i}
(inclusive), continuous with quadrant I, and one along the negative imaginary
axis below $-{\it i}$ (inclusive), continuous with quadrant III.
The range is that strip of the complex plane containing numbers whose imaginary
part is between $-\pi/2$ and $\pi/2$.  A number with imaginary
part equal to $-\pi/2$ is in the range if and only if its real
part is non-positive; a number with imaginary part equal to $\pi/2$ is in
the range if and only if its real part is non-negative.

\item[\cdf{acosh}]
The following definition for the inverse hyperbolic cosine
determines the range and branch cuts:
\begin{tabbing}
$ \arccosh {\it z}=\log \left({\it z}+({\it z}+1)\sqrt{({\it z}-1)/({\it z}+1)}\right) $
\end{tabbing}

\begin{newer}
Kahan \cite{KAHAN-COMPLEX-FNS} suggests the formula
\begin{tabbing}
$ \arccosh {\it z}=2 \log \left(  \sqrt{({\it z}+1)/2} + \sqrt{({\it z}-1)/2} \right) $
\end{tabbing}
pointing out that it yields the same principal value but eliminates
a gratuitous removable singularity at ${\it z}=-1$.
A proposal was submitted to X3J13 in September 1989 to replace the
formula \cdf{acosh} with that recommended by Kahan.
There is a good possibility that it will be adopted.
\end{newer}

The branch cut for the inverse hyperbolic cosine function
lies along the real axis to the left of 1 (inclusive), extending
indefinitely along the negative real axis, continuous with quadrant II
and (between 0 and 1) with quadrant I.
The range is that half-strip of the complex plane containing numbers whose
real part is non-negative and whose imaginary
part is between $-\pi$ (exclusive) and $\pi$ (inclusive).
A number with real part zero is in the range 
if its imaginary part is between zero (inclusive) and $\pi$ (inclusive).

\item[\cdf{atanh}]
The following definition for the inverse hyperbolic tangent
determines the range and branch cuts:
\begin{obsolete}
\noindent\leavevmode\vtop{
\begin{tabbing}
$ \arctanh {\it z}=\log \left((1+{\it z})\sqrt{1-1/{\it z}^2}\right) $\`{\bf WRONG!}
\end{tabbing}
}
\end{obsolete}

\begin{newer}
WARNING!  {\it The formula shown above for hyperbolic arc tangent is incorrect.}
It is not a matter of incorrect branch cuts; it simply does not compute anything
like a hyperbolic arc tangent.  This unfortunate error in the first edition
was the result of mistranscribing a (correct) APL formula from Penfield's paper
\cite{APL-BRANCH-CUTS}.  The formula should have been transcribed as
\begin{tabbing}
$ \arctanh {\it z}=\log \left((1+{\it z})\sqrt{1/(1-{\it z}^2)}\right) $
\end{tabbing}
\end{newer}

\begin{obsolete}
Beware of simplifying this formula; ``obvious'' simplifications are
likely to alter the branch cuts or the values on the branch cuts
incorrectly.

The branch cut for the inverse hyperbolic tangent function
is in two pieces: one along the negative real axis to the left of
$-1$ (inclusive), continuous with quadrant III, and one along
the positive real axis to the right of 1 (inclusive), continuous with
quadrant I.  The points $-1$ and 1 are excluded from the
domain.
The range is that strip of the complex plane containing
numbers whose imaginary part is between $-\pi/2$ and
$\pi/2$.  A number with imaginary part equal to $-\pi/2$
is in the range if and only if its real part is strictly negative; a number with
imaginary part equal to $\pi/2$ is in the range if and only if its real
part is strictly positive.  Thus the range of the inverse
hyperbolic tangent function is identical to
that of the inverse hyperbolic sine function with the points
$-\pi {\it i}/2$ and $\pi {\it i}/2$ excluded.
\end{obsolete}

\begin{newer}
A proposal was submitted to X3J13 in September 1989 to replace the
formula \cdf{atanh} with that recommended by Kahan \cite{KAHAN-COMPLEX-FNS}:
\begin{tabbing}
$ \displaystyle \arctanh {\it z}= { \left(\log(1+{\it z}) - \log(1-{\it z})\right) \over 2 } $
\end{tabbing}
There is a good possibility that it will be adopted.  If it is, the complete
description of the branch cuts of \cdf{atanh} will then be as follows.

The branch cut for the inverse hyperbolic tangent function
is in two pieces: one along the negative real axis to the left of
$-1$ (inclusive), continuous with quadrant II, and one along
the positive real axis to the right of 1 (inclusive), continuous with
quadrant IV.  The points $-1$ and 1 are excluded from the
domain.
The range is that strip of the complex plane containing
numbers whose imaginary part is between $-\pi/2$ and
$\pi/2$.  A number with imaginary part equal to $-\pi/2$
is in the range if and only if its real part is strictly positive; a number with
imaginary part equal to $\pi/2$ is in the range if and only if its real
part is strictly negative.  Thus the range of the inverse
hyperbolic tangent function is {\it not} the same as
that of the inverse hyperbolic sine function.
\end{newer}
\end{flushdesc}

With these definitions, the following useful identities are obeyed
throughout the applicable portion of the complex domain, even on
the branch cuts:


\begin{flushleft}
\begin{tabular*}{\textwidth}{@{}l@{\extracolsep{\fill}}ll@{}}
$\sin {\it i}{\it z} = {\it i} \sinh {\it z}$&$\sinh {\it i}{\it z} = {\it i} \sin {\it z}$&$\arctan {\it i}{\it z} = {\it i} \arctanh {\it z}$ \\
$\cos {\it i}{\it z} = \cosh {\it z}$&$\cosh {\it i}{\it z} = \cos {\it z}$&$\arcsinh {\it i}{\it z} = {\it i} \arcsin {\it z}$ \\
$\tan {\it i}{\it z} = {\it i} \tanh {\it z}$&$\arcsin {\it i}{\it z} = {\it i} \arcsinh {\it z}$&$\arctanh {\it i}{\it z} = {\it i} \arctan {\it z}$
\end{tabular*}
\end{flushleft}

\begin{new}
I thought it would be useful to provide some graphs illustrating the behavior
of the irrational and transcendental functions in the complex plane.
It also provides an opportunity to show off the Common Lisp code that
was used to generate them.

Imagine the complex plane to be decorated
as follows.  The real and imaginary axes are painted with thick lines.
Parallels from the axes on both sides at distances of 1, 2, and 3 are painted
with thin lines; these parallels are doubly infinite lines, as are the axes.
Four annuli (rings) are painted in gradated shades of gray.  Ring 1, the inner ring,
consists of points whose radial distances from the origin lie in the range
$[1/4,1/2]$; ring 2 is in the radial range
$[3/4, 1]$; ring 3, in the range
$[\pi /2, 2]$; and ring 4, in the range $[3, \pi]$.
Ring {\it j} is divided into $2^{j+1}$ equal sectors, with each sector
painted a different shade of gray, darkening as one proceeds counterclockwise
from the positive real axis.

We can illustrate the behavior of a numerical function ${\it f}$ by considering how
it maps the complex plane to itself.  More specifically, consider each
point ${\it z}$ of the decorated plane.  We decorate a new plane by coloring
the point ${\it f}({\it z})$ with the same color that point ${\it z}$ had in the original
decorated plane.  In other words, the newly decorated plane illustrates
how the ${\it f}$ maps the axes, other horizontal and vertical lines, and annuli.

In each figure we will show only a fragment of the complex plane,
with the real axis horizontal in the usual manner ($-\infty$ to the left, $+\infty$
to the right) and the imaginary axis vertical ($-\infty {\it i}$ below, $+\infty {\it i}$
above).  Each fragment shows a region containing points whose real and imaginary
parts are in the range $[-4.1, 4.1]$.  The axes of the new plane are shown as very
thin lines, with large tick marks at integer coordinates and somewhat smaller
tick marks at multiples of $\pi/2$.

Figure~\ref{IDENTITY-PLOT} shows the result of plotting the \cdf{identity} function
(quite literally); the graph exhibits the decoration of the original plane.

Figures~\ref{SECOND-PLOT} through~\ref{LAST-PLOT} show the graphs for the functions
\cdf{sqrt}, \cdf{exp}, \cdf{log}, \cdf{sin}, \cdf{asin}, \cdf{cos}, \cdf{acos}, \cdf{tan}, \cdf{atan},
\cdf{sinh}, \cdf{asinh}, \cdf{cosh}, \cdf{acosh}, \cdf{tanh}, and \cdf{atanh}, and
as a bonus, the graphs for the functions $\sqrt{1-{\it z}^2}$,
$\sqrt{1+{\it z}^2}$, $({\it z}-1)/({\it z}+1)$, and $(1+{\it z})/(1-{\it z})$.  All of these are related
to the trigonometric functions in various ways.  For example, if
${\it f}({\it z})=({\it z}-1)/({\it z}+1)$, then $\tanh {\it z} = {\it f}({\it e}^{2z})$, and if ${\it g}({\it z})=\sqrt{1-{\it z}^2}$, then
$\cos {\it z} = {\it g}(\sin {\it z})$.  It is instructive to examine the graph for $\sqrt{1-{\it z}^2}$
and try to visualize how it transforms the graph for \cdf{sin} into the graph for~\cdf{cos}.

Each figure is accompanied by a commentary on what maps to what and other interesting
features.  None of this material is terribly new; much of it may be found in any
good textbook on complex analysis.  I believe that the particular form in which the
graphs are presented is novel, as well as the fact that the graphs have been generated
as PostScript \cite{ADOBE-POSTSCRIPT} code by Common Lisp code.  This PostScript
code was then fed directly to the typesetting equipment that set the pages for this book.
Samples of this PostScript code follow the figures themselves,
after which the code for the entire program is presented.



In the commentaries that accompany the figures I
sometimes speak of mapping the points $\pm\infty$ or $\pm\infty i$.
When I say that function $f$ maps $+\infty$ to a certain point $z$, I mean that
\begin{tabbing}
$ z=\lim_{x\rightarrow +\infty} f(x+0\, i\/) $
\end{tabbing}
Similarly, when I say that $f$ maps
$-\infty i$ to $z$, I mean that
\begin{tabbing}
$ z = \lim_{y\rightarrow -\infty} f(0+yi\/) $
\end{tabbing}
In other words, I am considering a limit as one travels out along one of the
main axes.  I~also speak in a similar manner of mapping {\it to} one of these
infinities.
\end{new}
\clearpage

\begingroup

\ifx \HCode\Undef
\newcommand{\numplot}[1]{\includegraphics{#1-plot}\\}
\else
\newcommand{\numplot}[1]{\includegraphics{#1-plot.png}\\}
\fi

\newdimen\foodimen
\foodimen=\textheight
\setbox0=\vbox{\hrule}
\advance\foodimen by -2\ht0
\advance\foodimen by -16pt


\begin{figure}
\vbox to \foodimen{
\caption{Initial Decoration of the Complex Plane (Identity Function)}
\label{IDENTITY-PLOT}
\numplot{identity}
\small\noindent
This figure was produced in exactly the same manner as succeeding figures,
simply by plotting the function \cdf{identity} instead of a numerical function.
Thus the first of these figures was produced by the last function of the first edition.
I knew it would come in handy someday!
}
\end{figure}


\clearpage

\begin{figure}
\vbox to \foodimen{
\caption{Illustration of the Range of the Square Root Function}
\label{SECOND-PLOT}
\numplot{sqrt}
\small\noindent
The \cdf{sqrt} function maps the complex plane into the right half
of the plane by slitting it along the negative real axis and then sweeping
it around as if half-closing a folding fan.
The fan also shrinks, as if it were made of cotton and had gotten
wetter at the periphery than at the center.
The positive real axis is mapped onto itself.  The negative real axis is mapped
onto the positive imaginary axis (but if minus zero is supported, then
$-{\it x}+0{\it i}$ is mapped onto the positive imaginary axis and $-{\it x}-0{\it i}$ onto
the negative imaginary axis, assuming ${\it x}>0$).  The positive imaginary axis
is mapped onto the northeast diagonal, and the negative imaginary axis
onto the southeast diagonal.  More generally, lines are mapped to rectangular hyperbolas
(or fragments thereof\,) centered on the origin;
lines through the origin are mapped to degenerate
hyperbolas (perpendicular lines through the origin).
}
\end{figure}

\clearpage

\begin{figure}
\vbox to \foodimen{
\caption{Illustration of the Range of the Exponential Function}
\numplot{exp}
\small\noindent
The \cdf{exp} function maps horizontal lines to radii and maps vertical
lines to circles centered at the origin.
The origin is mapped to 1.  (It is instructive to compare
this graph with those of other functions
that map the origin to 1, for example $(1+{\it z})/(1-{\it z})$, $\cos {\it z}$, and $\sqrt{1-{\it z}^2}$.)
The entire real axis is mapped to the positive
real axis, with $-\infty$ mapping to the origin and $+\infty$ to itself.
The imaginary axis is mapped to the unit circle with infinite multiplicity (period $2\pi$);
therefore the mapping of the imaginary infinities $\pm\infty {\it i}$ is indeterminate.
It follows that the entire left half-plane is mapped to the interior of the unit circle,
and the right half-plane is mapped to the exterior of the unit circle.
A line at any angle other than horizontal or vertical is mapped to a
logarithmic spiral (but this is not illustrated here).
}
\end{figure}

\clearpage

\begin{figure}
\vbox to \foodimen{
\caption{Illustration of the Range of the Natural Logarithm Function}
\numplot{log}
\small\noindent
The \cdf{log} function, which is the inverse of \cdf{exp}, naturally maps radial lines to
horizontal lines and circles centered at the origin to vertical lines.
The interior of the unit circle is thus mapped to the entire left half-plane,
and the exterior of the unit circle is mapped to the right half-plane.
The positive real axis is mapped to the entire real axis, and the negative
real axis to a horizontal line of height $\pi$.  The positive and negative
imaginary axes are mapped to horizontal lines of height $\pm\pi/2$.
The origin is mapped to $-\infty$.

}
\end{figure}

\clearpage

\begin{figure}
\vbox to \foodimen{
\caption{Illustration of the Range of the Function $({\it z}-1)/({\it z}+1)$}
\numplot{minus-over-plus}
\small\noindent
A line is a degenerate circle with infinite radius;
when I say ``circles'' here I also mean lines.
Then $({\it z}-1)/({\it z}+1)$ maps circles into circles.
All circles through $-1$ become lines; all lines become
circles through $1$.
The real axis is mapped onto itself: 1 to
the origin, the origin to $-1$, $-1$ to infinity, and infinity to 1.
The imaginary axis becomes the unit circle; {\it i} is mapped to itself,
as is $-{\it i}$.  Thus the entire right half-plane is mapped to the interior
of the unit circle, the unit circle interior to the left half-plane,
the left half-plane to the unit circle exterior, and the unit circle exterior
to the right half-plane.  Imagine the complex plane to be a vast sea.
The Colossus of Rhodes straddles the origin, its left foot on {\it i} and its right foot on $-{\it i}$.
It bends down and briefly paddles water between its legs so furiously that the water
directly beneath is pushed out into the entire area behind it; much that was
behind swirls forward to either side; and all that was before is
sucked in to lie between its feet.
}
\end{figure}

\clearpage

\begin{figure}
\vbox to \foodimen{
\caption{Illustration of the Range of the Function $(1+{\it z})/(1-{\it z})$}
\numplot{plus-over-minus}
\small\noindent
The function ${\it h}({\it z})=(1+{\it z})/(1-{\it z})$
is the inverse of ${\it f}({\it z})=({\it z}-1)/({\it z}+1)$; that is,
${\it h}({\it f}({\it z}))={\it f}({\it h}({\it z}))={\it z}$. At first glance,
the graph of ${\it h}$ appears to be that of ${\it f}$
flipped left-to-right, or perhaps reflected in the origin, but careful
consideration of the shaded annuli reveals that this is not so; something more
subtle is going on.  Note that
${\it f}({\it f}({\it z}))={\it h}({\it h}({\it z}))={\it g}({\it z})=-1/{\it z}$.
The functions ${\it f}$, ${\it g}$, ${\it h}$, and the identity function
thus form a group under composition, isomorphic to the group
of the cyclic permutations of the points $-1$, $0$, $1$, and $\infty$, as
indeed these functions accomplish the four possible cyclic permutations
on those points.  This function group is a subset of the group of bilinear
transformations $({\it a}{\it z}+{\it b})/({\it c}{\it z}+{\it d})$,
all of which are conformal (angle-preserving) and map circles
onto circles.  Now, doesn't that tangle of circles through $-1$ look like something
the cat got into?
}
\end{figure}

\clearpage

\begin{figure}
\vbox to \foodimen{
\caption{Illustration of the Range of the Sine Function}
\numplot{sin}
\small\noindent
We are used to seeing \cdf{sin} looking like a wiggly ocean wave,
graphed vertically as a function of the real axis only.  Here is a different view.
The entire real axis is mapped to the segment $[-1, 1]$ of the real axis
with infinite multiplicity (period $2\pi$).  The imaginary axis is mapped to itself
as if by \cdf{sinh} considered as a real function.  The origin is mapped to itself.
Horizontal lines are mapped to ellipses with foci at $\pm1$ (note that two horizontal
lines equidistant from the real axis will map onto the same ellipse).
Vertical lines are mapped to hyperbolas with the same foci.  There is a curious accident:
the ellipse for horizontal
lines at distance $\pm1$ from the real axis appears to intercept the real axis at
$\pm\pi/2\approx \pm1.57\ldots$ but this is not so; the intercepts are actually at
$\pm({\it e}+1/{\it e})/2\approx \pm1.54\ldots\,\hbox{}$.
}
\end{figure}

\clearpage

\begin{figure}
\vbox to \foodimen{
\caption{Illustration of the Range of the Arc Sine Function}
\numplot{asin}
\small\noindent
Just as \cdf{sin} grabs horizontal lines and bends them into elliptical loops around
the origin, so its inverse \cdf{asin} takes annuli and yanks them more
or less horizontally straight.  Because sine is not injective,
its inverse as a function cannot be surjective.  This is just a highfalutin
way of saying that the range of the \cdf{asin} function doesn't
cover the entire plane but only a strip $\pi$ wide; arc sine as a one-to-many relation
would cover the plane with an infinite number of copies of this strip side by side,
looking for all the world like the tail of a peacock with an infinite number of feathers.
The imaginary axis is mapped to itself as if by \cdf{asinh} considered as a real function.
The real axis is mapped to a bent path, turning corners at $\pm\pi/2$ (the points to
which $\pm 1$ are mapped); $+\infty$ is mapped to $\pi/2-\infty {\it i}$, and
$-\infty$ to $-\pi/2+\infty {\it i}$.
}
\end{figure}

\clearpage

\begin{figure}
\vbox to \foodimen{
\caption{Illustration of the Range of the Cosine Function}
\numplot{cos}
\small\noindent
We are used to seeing \cdf{cos} looking exactly like \cdf{sin}, a wiggly ocean wave,
only displaced.  Indeed the complex mapping of \cdf{cos} is also similar
to that of \cdf{sin}, with horizontal and vertical lines mapping to the same ellipses
and hyperbolas with foci at $\pm 1$, although mapping to them in a different
manner, to be sure.
The entire real axis is again mapped to the segment $[-1, 1]$ of the real axis,
but each half of the imaginary axis is mapped to the real axis to the right of 1
(as if by \cdf{cosh} considered as a real function).  Therefore $\pm\infty {\it i}$
both map to $+\infty$.
The origin is mapped to 1.  Whereas \cdf{sin} is an odd function, \cdf{cos} is an
even function; as a result {\it two} points in each annulus, one the negative
of the other, are mapped to the same shaded point in this graph; the shading shown here
is taken from points in the original upper half-plane.
}
\end{figure}

\clearpage

\begin{figure}
\vbox to \foodimen{
\caption{Illustration of the Range of the Arc Cosine Function}
\numplot{acos}
\small\noindent
The graph of \cdf{acos} is very much like that of \cdf{asin}.
One might think that our nervous peacock has shuffled half a step
to the right, but the shading on the annuli shows that we have instead caught
the bird exactly in mid-flight while doing a cartwheel.
This is easily understood if we recall that $\arccos {\it z}=(\pi/2)-\arcsin {\it z}$;
negating $\arcsin {\it z}$ rotates it upside down, and adding the result to $\pi/2$
translates it $\pi/2$ to the right.
The imaginary axis is mapped upside down to the vertical line at $\pi/2$.
The point $+1$ is mapped to the origin, and $-1$ to $\pi$.
The image of the real axis is again cranky; $+\infty$ is mapped to $+\infty {\it i}$,
and $-\infty$ to $\pi-\infty {\it i}$.

}
\end{figure}

\clearpage

\begin{figure}
\vbox to \foodimen{
\caption{Illustration of the Range of the Tangent Function}
\numplot{tan}
\small\noindent
The usual graph of \cdf{tan} as a real function looks like an infinite chorus
line of disco dancers, left hands pointed skyward and right hands to the floor.
The \cdf{tan} function is the quotient of \cdf{sin} and \cdf{cos}
but it doesn't much look like either except for having period $2\pi$.
This goes for the complex plane as well, although the swoopy loops produced
from the annulus between $\pi/2$ and $2$ look vaguely like
those from the graph of \cdf{sin} inside out.
The real axis is mapped onto itself with infinite multiplicity (period $2\pi$).
The imaginary axis is mapped backwards onto $[-{\it i},{\it i}]$:
$+\infty {\it i}$ is mapped to $-{\it i}$ and $-\infty {\it i}$ to $+{\it i}$.
Horizontal lines below or above the real axis
become circles surrounding $+{\it i}$ or $-{\it i}$, respectively.
Vertical lines become circular arcs from $+{\it i}$ to $-{\it i}$;
two vertical lines separated by $(2{\it k}+1)\pi$ for integer ${\it k}$
together become a complete circle.  It seems that two arcs
shown hit the real axis at $\pm\pi/2=\pm 1.57\ldots$ but that is a coincidence;
they really hit the axis at $\pm\tan 1= 1.55\ldots\,\hbox{}$.
}
\end{figure}

\clearpage

\begin{figure}
\vbox to \foodimen{
\caption{Illustration of the Range of the Arc Tangent Function}
\numplot{xatan}
\small\noindent
All I can say is that this peacock is a horse of another color.
At first glance, the axes seem to map in the same way as for \cdf{asin} and
\cdf{acos}, but look again: this time it's the imaginary axis doing weird things.
All infinities map multiply to the points
$(2{\it k}+1)\pi/2$; within the strip of principal values we may say that
the real axis is mapped to the interval $[-\pi/2,+\pi/2]$ and therefore
$-\infty$ is mapped to $-\pi/2$ and $+\infty$ to $+\pi/2$.
The point $+{\it i}$ is mapped to $+\infty {\it i}$, and $-{\it i}$ to $-\infty {\it i}$, and
so the imaginary axis is mapped into three pieces: the segment
$[-\infty {\it i},-{\it i}]$ is mapped to $[\pi/2,\pi/2-\infty {\it i}]$; the segment
$[-{\it i},{\it i}]$ is mapped to the imaginary axis $[-\infty {\it i},+\infty {\it i}]$; and the segment
$[+{\it i},+\infty {\it i}]$ is mapped to $[-\pi/2+\infty {\it i},-\pi/2]$.
}
\end{figure}

\clearpage

\begin{figure}
\vbox to \foodimen{
\caption{Illustration of the Range of the Hyperbolic Sine Function}
\numplot{sinh}
\small\noindent
It would seem that the graph of \cdf{sinh} is merely that of \cdf{sin}
rotated 90 degrees.  If that were so, then we would have $\sinh {\it z} = {\it i} \sin {\it z}$.
Careful inspection of the shading, however, reveals that this is not quite the case;
in both graphs the lightest and darkest shades, which initially are adjacent to
the positive real axis, remain adjacent to the positive real axis in both cases.
To derive the graph of \cdf{sinh} from \cdf{sin} we must therefore first
rotate the complex plane by $-90$ degrees, then apply \cdf{sin}, then
rotate the result by 90 degrees. In other words, $\sinh {\it z} = {\it i} \sin (-{\it i\/}){\it z}$;
consistently replacing ${\it z}$ with ${\it i}{\it z}$ in this formula yields the familiar identity
$\sinh {\it i}{\it z} = {\it i} \sin {\it z}$.
}
\end{figure}

\clearpage

\begin{figure}
\vbox to \foodimen{
\caption{Illustration of the Range of the Hyperbolic Arc Sine Function}
\numplot{asinh}
\small\noindent
The peacock sleeps.  Because $\arcsinh {\it i}{\it z} = {\it i} \arcsin {\it z}$,
the graph of \cdf{asinh}
is related to that of \cdf{asin} by pre- and post-rotations of the complex plane
in the same way as for \cdf{sinh} and \cdf{sin}.
}
\end{figure}

\clearpage

\begin{figure}
\vbox to \foodimen{
\caption{Illustration of the Range of the Hyperbolic Cosine Function}
\numplot{cosh}
\small\noindent
The graph of \cdf{cosh} does {\it not} look like that of \cdf{cos} rotated
90 degrees; instead it looks like that of \cdf{cos} unrotated.
That is because $\cosh {\it i}{\it z}$ is not equal to ${\it i} \cos {\it z}$;
rather, $\cosh {\it i}{\it z} = \cos {\it z}$.
Interpreted, that means that the shading is pre-rotated but there is no
post-rotation.
}
\end{figure}

\clearpage

\begin{figure}
\vbox to \foodimen{
\caption{Illustration of the Range of the Hyperbolic Arc Cosine Function}
\numplot{acosh}
\small\noindent
Hmm---I'd rather not say what happened to this peacock.
This feather looks a bit mangled.  Actually it is all right---the principal
value for \cdf{acosh} is so chosen that its graph does not look simply
like a rotated version of the graph of \cdf{acos}, but if all values were
shown, the two graphs would fill the plane in repeating patterns related
by a rotation.

}
\end{figure}

\clearpage

\begin{figure}
\vbox to \foodimen{
\caption{Illustration of the Range of the Hyperbolic Tangent Function}
\numplot{tanh}
\small\noindent
The diagram for \cdf{tanh} is simply that of \cdf{tan} turned on its ear:
${\it i} \tan {\it z} = \tanh {\it i}{\it z}$.
The imaginary axis is mapped onto itself with infinite multiplicity (period $2\pi$),
and the real axis is mapped onto the segment $[-1,+1]$:
$+\infty$ is mapped to $+1$, and $-\infty$ to $-1$.
Vertical lines to the left or right of the real axis
are mapped to circles surrounding $-1$ or $1$, respectively.
Horizontal lines are mapped to circular arcs anchored at $-1$ and $+1$;
two horizontal lines separated by a distance $(2{\it k}+1)\pi$ for integer ${\it k}$ are
together mapped into a complete circle.  How do we know these really are circles?
Well, $\tanh {\it z} = ((\exp 2{\it z})-1)/((\exp 2{\it z})+1)$, which is the composition of the
bilinear transform $({\it z}-1)/({\it z}+1)$, the exponential $\exp {\it z}$,
and the magnification $2{\it z}$.
Magnification maps lines to lines of the same slope; the exponential maps
horizontal lines to circles and vertical lines to radial lines;
and a bilinear transform maps generalized circles (including lines) to
generalized circles.  Q.E.D.
}
\end{figure}

\clearpage

\begin{figure}
\vbox to \foodimen{
\caption{Illustration of the Range of the Hyperbolic Arc Tangent Function}
\label{ATANH-PLOT}
\numplot{really-good-atanh}
\small\noindent
A sleeping peacock of another color: $\arctanh {\it i}{\it z} = {\it i} \arctan {\it z}$.
}
\end{figure}

\clearpage

\begin{figure}
\vbox to \foodimen{
\caption{Illustration of the Range of the Function $\protect\sqrt{1-{\it z}^2}$}
\numplot{sqrt-one-minus-sq}
\small\noindent
Here is a curious graph indeed for so simple a function!
The origin is mapped to 1.  The real axis segment $[0,1]$ is
mapped backwards (and non-linearly) into itself; the segment $[1,+\infty]$
is mapped non-linearly onto the positive imaginary axis.
The negative real axis is mapped to the same points as the positive real axis.
Both halves of the imaginary axis are mapped into $[1,+\infty]$ on the real axis.
Horizontal lines become vaguely vertical, and
vertical lines become vaguely horizontal.
Circles centered at the origin are transformed into Cassinian \hbox{(half-)ovals}; the unit
circle is mapped to a \hbox{(half-)lemniscate} of Bernoulli.  The outermost annulus appears
to have its {\it inner} edge at $\pi$ on the real axis and its {\it outer}
edge at 3 on the imaginary axis, but this is another accident; the intercept
on the real axis, for example, is not really at $\pi\approx 3.14\ldots$ but
at $\sqrt{1-(3{\it i\/})^2}=\sqrt{10}\approx 3.16\ldots\,\hbox{}$.

}
\end{figure}

\clearpage

\begin{figure}
\vbox to \foodimen{
\caption{Illustration of the Range of the Function $\protect\sqrt{1+{\it z}^2}$}
\label{LAST-PLOT}
\numplot{sqrt-one-plus-sq}
\small\noindent
The graph of ${\it q}({\it z})=\sqrt{1+{\it z}^2}$ looks like that
of ${\it p}({\it z})=\sqrt{1-{\it z}^2}$ except for
the shading.  You might not expect ${\it p}$ and ${\it q}$ to be related in the same
way that \cdf{cos} and \cdf{cosh} are, but after a little reflection (or perhaps
I should say, after turning it around in one's mind) one can see that
${\it q}({\it i}{\it z})={\it p}({\it z})$.  This formula is indeed of exactly the same form as
$\cosh {\it i}{\it z} = \cos {\it z}$.  The function
$\sqrt{1+{\it z}^2}$ maps both halves of the real axis into $[1,+\infty]$ on the real axis.
The segments $[0,{\it i}]$ and $[0,-{\it i}]$ of the imaginary axis are each mapped
backwards onto segment $[0,1]$ of the real axis; $[{\it i},+\infty {\it i}]$ and
$[-{\it ,}-\infty {\it i}]$ are each mapped onto the positive imaginary axis
(but if minus zero is supported then opposite sides of the
imaginary axis map to opposite halves of the imaginary axis---for example,
$q(+0+2{\it i\/})=\sqrt{5}{\it i}$ but $q(-0+2{\it i\/})=-\sqrt{5}{\it i}$).
}
\end{figure}

\clearpage
\endgroup
\begin{new}
Here is a sample of the PostScript code that generated
figure~\ref{IDENTITY-PLOT}, showing the initial scaling,
translation, and clipping parameters; the code for one
sector of the innermost annulus; and the code for the negative
imaginary axis.  Comment lines indicate how path or boundary
segments were generated separately and then spliced (in order to
allow for the places that a singularity might lurk, in which case
the generating code can ``inch up'' to the problematical argument
value).

The size of the entire PostScript file for the
\cdf{identity} function was about 68 kilobytes (2757 lines, including comments).
The smallest files
were the plots for \cdf{atan} and \cdf{atanh}, about 65 kilobytes apiece;
the largest were the plots for \cdf{sin}, \cdf{cos}, \cdf{sinh}, and \cdf{cosh},
about 138 kilobytes apiece.

\begingroup\small  \topsep 0pt plus 6pt \relax
\begin{lisp}
\\
\% PostScript file for plot of function IDENTITY \\
\% Plot is to fit in a region 4.666666666666667 inches square \\
\% ~showing axes extending 4.1 units from the origin. \\
 \\
40.97560975609756 40.97560975609756 scale \\
4.1 4.1 translate \\
newpath \\
~~-4.1 -4.1 moveto \\
~~4.1 -4.1 lineto \\
~~4.1 4.1 lineto \\
~~-4.1 4.1 lineto \\
~~closepath \\
clip \\
\% Moby grid for function IDENTITY \\
\% Annulus 0.25 0.5 4 0.97 0.45 \\
\% Sector from 4.7124 to 6.2832 (quadrant 3) \\
newpath \\
~~0.0 -0.25 moveto \\
~~0.0 -0.375 lineto \\
~~\%middle radial \\
~~0.0 -0.375 lineto \\
~~0.0 -0.5 lineto \\
~~\%end radial \\
~~0.0 -0.5 lineto \\
~~0.092 -0.4915 lineto \\
~~0.1843 -0.4648 lineto \\
~~0.273 -0.4189 lineto \\
~~0.3536 -0.3536 lineto \\
~~\%middle circumferential \\
~~0.3536 -0.3536 lineto \\
~~0.413 -0.2818 lineto \\
~~0.4594 -0.1974 lineto \\
~~0.4894 -0.1024 lineto \\
~~0.5 0.0 lineto \\
~~\%end circumferential \\
~~0.5 0.0 lineto \\
~~0.375 0.0 lineto \\
~~\%middle radial \\
~~0.375 0.0 lineto \\
~~0.25 0.0 lineto \\
~~\%end radial \\
~~0.25 0.0 lineto \\
~~0.2297 -0.0987 lineto \\
~~0.1768 -0.1768 lineto \\
~~\%middle circumferential \\
~~0.1768 -0.1768 lineto \\
~~0.0922 -0.2324 lineto \\
~~0.0 -0.25 lineto \\
~~\%end circumferential \\
~~closepath \\
currentgray~~~0.45 setgray~~~fill~~~setgray
\end{lisp}
{\rm {\lbrack}2598 lines omitted{\rbrack}}
\begin{lisp}
\% Vertical line from (0.0, -0.5) to (0.0, 0.0) \\
newpath \\
~~0.0 -0.5 moveto \\
~~0.0 0.0 lineto \\
0.05 setlinewidth~~~1 setlinecap~~stroke \\
\% Vertical line from (0.0, -0.5) to (0.0, -1.0) \\
newpath \\
~~0.0 -0.5 moveto \\
~~0.0 -1.0 lineto \\
0.05 setlinewidth~~~1 setlinecap~~stroke \\
\% Vertical line from (0.0, -2.0) to (0.0, -1.0) \\
newpath \\
~~0.0 -2.0 moveto \\
~~0.0 -1.0 lineto \\
0.05 setlinewidth~~~1 setlinecap~~stroke \\
\% Vertical line from (0.0, -2.0) to (0.0, -1.1579208923731617E77) \\
newpath \\
~~0.0 -2.0 moveto \\
~~0.0 -6.3553 lineto \\
~~0.0 -6.378103166302659 lineto \\
~~0.0 -6.378103166302659 lineto \\
~~0.0 -6.378103166302659 lineto \\
0.05 setlinewidth   1 setlinecap  stroke
\end{lisp}
{\rm {\lbrack}84 lines omitted{\rbrack}}
\begin{lisp}
\% End of PostScript file for plot of function IDENTITY
\end{lisp}
\endgroup


\penalty-10000%manual

Here is the program that generated the PostScript code for
the graphs shown in figures~\ref{IDENTITY-PLOT} through~\ref{LAST-PLOT}.
It contains a mixture of fairly general mechanisms and {\it ad hoc} kludges
for plotting functions of a single complex argument while gracefully handling
extremely large and small values,
branch cuts, singularities, and periodic behavior.
The aim was to provide a simple user interface that would not
require the caller to provide special advice for each function
to be plotted.
The file for figure~\ref{IDENTITY-PLOT}, for example, was generated
by the call \cd{(picture~'identity)}, which resulted in the writing of
a file named \cd{identity-plot.ps}.

The program assumes that any periodic behavior will have a period that is a multiple of
$2\pi$; that branch cuts will fall along the real or imaginary axis;
and that singularities or very large or small values
will occur only at the origin, at $\pm 1$ or $\pm {\it i}$,
or on the boundaries of the annuli (particularly those with radius $\pi/2$ or $\pi$).
The central function is \cdf{parametric-path}, which accepts four arguments:
two real numbers that are the endpoints of an interval of real numbers,
a function that maps this interval into a path in the complex plane,
and the function to be plotted; the task of \cdf{parametric-path} is to
generate PostScript code (a series of \cdf{lineto} operations)
that will plot an approximation to the image of the parametric path
as transformed by the function to be plotted.
Each of the functions \cdf{hline}, \cdf{vline}, \cdf{-hline}, \cdf{-vline}, \cdf{radial},
and \cdf{circumferential} takes appropriate parameters
and returns a function suitable for use as the third argument
to \cdf{parametric-path}.
There is some code that defends against errors
(by using \cdf{ignore-errors}) and against certain peculiarities of
IEEE floating-point arithmetic (the code that checks for not-a-number (NaN) results).

The program is offered here without further comment or apology.

\vskip 0pt plus 10pt%manual
\hrule width 0pt\relax

\begingroup\small \topsep 0pt plus 10pt \relax
\begin{lisp}
\\
(defparameter units-to-show 4.1) \\*
(defparameter text-width-in-picas 28.0) \\*
(defparameter device-pixels-per-inch 300) \\*
(defparameter pixels-per-unit \\*
~~(* (/ (/ text-width-in-picas 6) \\*
~~~~~~~~(* units-to-show 2)) \\*
~~~~~device-pixels-per-inch)) \\
\\
(defparameter big (sqrt (sqrt most-positive-single-float))) \\*
(defparameter tiny (sqrt (sqrt least-positive-single-float))) \\
\\
(defparameter path-really-losing 1000.0) \\
(defparameter path-outer-limit (* units-to-show (sqrt 2) 1.1)) \\
(defparameter path-minimal-delta (/ 10 pixels-per-unit)) \\
(defparameter path-outer-delta (* path-outer-limit 0.3)) \\
(defparameter path-relative-closeness 0.00001) \\*
(defparameter back-off-delta 0.0005)
\end{lisp}
\newpage%manual
\begin{lisp}
(defun comment-line (stream \&rest stuff) \\*
~~(format stream "{\Xtilde}\%\% ") \\*
~~(apply \#'format stream stuff) \\*
~~(format t "{\Xtilde}\%\% ") \\*
~~(apply \#'format t stuff)) \\
\\
(defun parametric-path (from to paramfn plotfn) \\*
~~(assert (and (plusp from) (plusp to))) \\*
~~(flet ((domainval (x) (funcall paramfn x)) \\*
~~~~~~~~~(rangeval (x) (funcall plotfn (funcall paramfn x))) \\*
~~~~~~~~~(losing (x) (or (null x) \\*
~~~~~~~~~~~~~~~~~~~~~~~~~(/= (realpart x) (realpart x))~~;NaN? \\*
~~~~~~~~~~~~~~~~~~~~~~~~~(/= (imagpart x) (imagpart x))~~;NaN? \\*
~~~~~~~~~~~~~~~~~~~~~~~~~(> (abs (realpart x)) path-really-losing) \\*
~~~~~~~~~~~~~~~~~~~~~~~~~(> (abs (imagpart x)) path-really-losing)))) \\
~~~~(when (> to 1000.0) \\*
~~~~~~(let ((f0 (rangeval from)) \\*
~~~~~~~~~~~~(f1 (rangeval (+ from 1))) \\*
~~~~~~~~~~~~(f2 (rangeval (+ from (* 2 pi)))) \\*
~~~~~~~~~~~~(f3 (rangeval (+ from 1 (* 2 pi)))) \\*
~~~~~~~~~~~~(f4 (rangeval (+ from (* 4 pi))))) \\
~~~~~~~~(flet ((close (x y) \\*
~~~~~~~~~~~~~~~~~(or (< (careful-abs (- x y)) path-minimal-delta) \\*
~~~~~~~~~~~~~~~~~~~~~(< (careful-abs (- x y)) \\*
~~~~~~~~~~~~~~~~~~~~~~~~(* (+ (careful-abs x) (careful-abs y)) \\*
~~~~~~~~~~~~~~~~~~~~~~~~~~~path-relative-closeness))))) \\
~~~~~~~~~~(when (and (close f0 f2) \\*
~~~~~~~~~~~~~~~~~~~~~(close f2 f4) \\*
~~~~~~~~~~~~~~~~~~~~~(close f1 f3) \\*
~~~~~~~~~~~~~~~~~~~~~(or (and (close f0 f1) \\*
~~~~~~~~~~~~~~~~~~~~~~~~~~~~~~(close f2 f3)) \\*
~~~~~~~~~~~~~~~~~~~~~~~~~(and (not (close f0 f1)) \\*
~~~~~~~~~~~~~~~~~~~~~~~~~~~~~~(not (close f2 f3))))) \\*
~~~~~~~~~~~~(format t "{\Xtilde}\&Periodicity detected.") \\*
~~~~~~~~~~~~(setq to (+ from (* (signum (- to from)) 2 pi))))))) \\
~~~~~(let ((fromrange (ignore-errors (rangeval from))) \\*
~~~~~~~~~~(torange (ignore-errors (rangeval to)))) \\*
~~~~~~(if (losing fromrange) \\*
~~~~~~~~~~(if (losing torange) \\*
~~~~~~~~~~~~~~'() \\*
~~~~~~~~~~~~~~(parametric-path (back-off from to) to paramfn plotfn)) \\
~~~~~~~~~~(if (losing torange) \\*
~~~~~~~~~~~~~~(parametric-path from (back-off to from) paramfn plotfn) \\*
~~~~~~~~~~~~~~(expand-path (refine-path (list from to) \#'rangeval) \\*
~~~~~~~~~~~~~~~~~~~~~~~~~~~\#'rangeval))))))
\end{lisp}
\vskip 0pt plus 10pt \newpage%manual
\begin{lisp}
(defun back-off (point other) \\*
~~(if (or (> point 10.0) (< point 0.1)) \\*
~~~~~~(let ((sp (sqrt point))) \\*
~~~~~~~~(if (or (> point sp other) (< point sp other)) \\*
~~~~~~~~~~~~sp \\*
~~~~~~~~~~~~(* sp (sqrt other)))) \\*
~~~~~~(+ point (* (signum (- other point)) back-off-delta)))) \\
\\
(defun careful-abs (z) \\*
~~(cond ((or (> (realpart z) big) \\*
~~~~~~~~~~~~~(< (realpart z) (- big)) \\*
~~~~~~~~~~~~~(> (imagpart z) big) \\*
~~~~~~~~~~~~~(< (imagpart z) (- big))) \\*
~~~~~~~~~big) \\
~~~~~~~~((complexp z) (abs z)) \\*
~~~~~~~~((minusp z) (- z)) \\*
~~~~~~~~(t z))) \\      %intentional extra \\
\end{lisp}
\begin{lisp}
(defparameter max-refinements 5000) \\
\\
(defun refine-path (original-path rangevalfn) \\*
~~(flet ((rangeval (x) (funcall rangevalfn x))) \\*
~~~~(let ((path original-path)) \\*
~~~~~~(do ((j 0 (+ j 1))) \\*
~~~~~~~~~~((null (rest path))) \\
~~~~~~~~(when (zerop (mod (+ j 1) max-refinements)) \\*
~~~~~~~~~~~~~~(break "Runaway path")) \\
~~~~~~~~(let* ((from (first path)) \\*
~~~~~~~~~~~~~~~(to (second path)) \\*
~~~~~~~~~~~~~~~(fromrange (rangeval from)) \\*
~~~~~~~~~~~~~~~(torange (rangeval to)) \\*
~~~~~~~~~~~~~~~(dist (careful-abs (- torange fromrange))) \\*
~~~~~~~~~~~~~~~(mid (* (sqrt from) (sqrt to))) \\*
~~~~~~~~~~~~~~~(midrange (rangeval mid))) \\
~~~~~~~~~~(cond ((or (and (far-out fromrange) (far-out torange)) \\*
~~~~~~~~~~~~~~~~~~~~~(and (< dist path-minimal-delta) \\*
~~~~~~~~~~~~~~~~~~~~~~~~~~(< (abs (- midrange fromrange)) \\*
~~~~~~~~~~~~~~~~~~~~~~~~~~~~~path-minimal-delta) \\*
~~~~~~~~~~~~~~~~~~~~~~~~~~;; Next test is intentionally asymmetric to \\*
~~~~~~~~~~~~~~~~~~~~~~~~~~;;~~avoid problems with periodic functions. \\*
~~~~~~~~~~~~~~~~~~~~~~~~~~(< (abs (- (rangeval (/ (+ to (* from 1.5)) \\*
~~~~~~~~~~~~~~~~~~~~~~~~~~~~~~~~~~~~~~~~~~~~~~~~~~2.5)) \\*
~~~~~~~~~~~~~~~~~~~~~~~~~~~~~~~~~~~~~fromrange)) \\*
~~~~~~~~~~~~~~~~~~~~~~~~~~~~~path-minimal-delta))) \\*
~~~~~~~~~~~~~~~~~(pop path)) \\
~~~~~~~~~~~~~~~~((= mid from) (pop path)) \\*
~~~~~~~~~~~~~~~~((= mid to) (pop path)) \\*
~~~~~~~~~~~~~~~~(t (setf (rest path) (cons mid (rest path))))))))) \\*
~~original-path) \\
\\
(defun expand-path (path rangevalfn) \\*
~~(flet ((rangeval (x) (funcall rangevalfn x))) \\*
~~~~(let ((final-path (list (rangeval (first path))))) \\*
~~~~~~(do ((p (rest path) (cdr p))) \\*
~~~~~~~~~~((null p) \\*
~~~~~~~~~~~(unless (rest final-path) \\*
~~~~~~~~~~~~~(break "Singleton path")) \\*
~~~~~~~~~~~(reverse final-path)) \\
~~~~~~~~(let ((v (rangeval (car p)))) \\*
~~~~~~~~~~(cond ((and (rest final-path) \\*
~~~~~~~~~~~~~~~~~~~~~~(not (far-out v)) \\*
~~~~~~~~~~~~~~~~~~~~~~(not (far-out (first final-path))) \\*
~~~~~~~~~~~~~~~~~~~~~~(between v (first final-path) \\*
~~~~~~~~~~~~~~~~~~~~~~~~~~~~~~~~~(second final-path))) \\*
~~~~~~~~~~~~~~~~~(setf (first final-path) v)) \\
~~~~~~~~~~~~~~~~((null (rest p))~~~;Mustn't omit last point \\*
~~~~~~~~~~~~~~~~~(push v final-path)) \\
~~~~~~~~~~~~~~~~((< (abs (- v (first final-path))) path-minimal-delta)) \\
~~~~~~~~~~~~~~~~((far-out v) \\*
~~~~~~~~~~~~~~~~~(unless (and (far-out (first final-path)) \\*
~~~~~~~~~~~~~~~~~~~~~~~~~~~~~~(< (abs (- v (first final-path))) \\*
~~~~~~~~~~~~~~~~~~~~~~~~~~~~~~~~~path-outer-delta)) \\*
~~~~~~~~~~~~~~~~~~~(push (* 1.01 path-outer-limit (signum v)) \\*
~~~~~~~~~~~~~~~~~~~~~~~~~final-path))) \\*
~~~~~~~~~~~~~~~~(t (push v final-path)))))))) \\
\\
(defun far-out (x) \\*
~~(> (careful-abs x) path-outer-limit)) \\
\\
(defparameter between-tolerance 0.000001) \\
\\
(defun between (p q r) \\*
~~(let ((px (realpart p)) (py (imagpart p)) \\*
~~~~~~~~(qx (realpart q)) (qy (imagpart q)) \\*
~~~~~~~~(rx (realpart r)) (ry (imagpart r))) \\
~~~~(and (or (<= px qx rx) (>= px qx rx)) \\*
~~~~~~~~~(or (<= py qy ry) (>= py qy ry)) \\*
~~~~~~~~~(< (abs (- (* (- qx px) (- ry qy)) \\*
~~~~~~~~~~~~~~~~~~~~(* (- rx qx) (- qy py)))) \\*
~~~~~~~~~~~~between-tolerance))))
\end{lisp}
\vskip 0pt plus 10pt \newpage%manual
\begin{lisp}
(defun circle (radius) \\*
~~\#'(lambda (angle) (* radius (cis angle)))) \\
\\
(defun hline (imag) \\*
~~\#'(lambda (real) (complex real imag))) \\
\\
(defun vline (real) \\*
~~\#'(lambda (imag) (complex real imag))) \\
\\
(defun -hline (imag) \\*
~~\#'(lambda (real) (complex (- real) imag))) \\
\\
(defun -vline (real) \\*
~~\#'(lambda (imag) (complex real (- imag)))) \\
\\
(defun radial (phi quadrant) \\*
~~\#'(lambda (rho) (repair-quadrant (* rho (cis phi)) quadrant))) \\
\\
(defun circumferential (rho quadrant) \\*
~~\#'(lambda (phi) (repair-quadrant (* rho (cis phi)) quadrant))) \\
\\
;;; Quadrant is 0, 1, 2, or 3, meaning I, II, III, or IV. \\
\\
(defun repair-quadrant (z quadrant) \\*
~~(complex (* (+ (abs (realpart z)) tiny) \\*
~~~~~~~~~~~~~~(case quadrant (0 1.0) (1 -1.0) (2 -1.0) (3 1.0))) \\*
~~~~~~~~~~~(* (+ (abs (imagpart z)) tiny) \\*
~~~~~~~~~~~~~~(case quadrant (0 1.0) (1 1.0) (2 -1.0) (3 -1.0))))) \\
\\
(defun clamp-real (x) \\*
~~(if (far-out x) \\*
~~~~~~(* (signum x) path-outer-limit) \\*
~~~~~~(round-real x))) \\
\\
(defun round-real (x) \\*
~~(/ (round (* x 10000.0)) 10000.0)) \\
\\
(defun round-point (z) \\*
~~(complex (round-real (realpart z)) (round-real (imagpart z)))) \\
\\
(defparameter hiringshade 0.97) \\*
(defparameter loringshade 0.45) \\
\\
(defparameter ticklength 0.12) \\*
(defparameter smallticklength 0.09)
\end{lisp}
\vskip 0pt plus 10pt \newpage%manual
\begin{lisp}
;;; This determines the pattern of lines and annuli to be drawn. \\*
(defun moby-grid (\&optional (fn 'sqrt) (stream t)) \\*
~~(comment-line stream "Moby grid for function {\Xtilde}S" fn) \\
~~(shaded-annulus 0.25 0.5 4 hiringshade loringshade fn stream) \\*
~~(shaded-annulus 0.75 1.0 8 hiringshade loringshade fn stream) \\*
~~(shaded-annulus (/ pi 2) 2.0 16 hiringshade loringshade fn stream) \\*
~~(shaded-annulus 3 pi 32 hiringshade loringshade fn stream) \\
~~(moby-lines :horizontal 1.0 fn stream) \\*
~~(moby-lines :horizontal -1.0 fn stream) \\*
~~(moby-lines :vertical 1.0 fn stream) \\*
~~(moby-lines :vertical -1.0 fn stream) \\
~~(let ((tickline 0.015) \\*
~~~~~~~~(axisline 0.008)) \\
~~~~(flet ((tick (n) (straight-line (complex n ticklength) \\*
~~~~~~~~~~~~~~~~~~~~~~~~~~~~~~~~~~~~(complex n (- ticklength)) \\*
~~~~~~~~~~~~~~~~~~~~~~~~~~~~~~~~~~~~tickline \\*
~~~~~~~~~~~~~~~~~~~~~~~~~~~~~~~~~~~~stream)) \\
~~~~~~~~~~~(smalltick (n) (straight-line (complex n smallticklength) \\*
~~~~~~~~~~~~~~~~~~~~~~~~~~~~~~~~~~~~~~~~~(complex n (- smallticklength)) \\*
~~~~~~~~~~~~~~~~~~~~~~~~~~~~~~~~~~~~~~~~~tickline \\*
~~~~~~~~~~~~~~~~~~~~~~~~~~~~~~~~~~~~~~~~~stream))) \\
~~~~~~(comment-line stream "Real axis") \\*
~~~~~~(straight-line \#c(-5 0) \#c(5 0) axisline stream) \\*
~~~~~~(dotimes (j (floor units-to-show)) \\*
~~~~~~~~(let ((q (+ j 1))) (tick q) (tick (- q)))) \\
~~~~~~(dotimes (j (floor units-to-show (/ pi 2))) \\*
~~~~~~~~(let ((q (* (/ pi 2) (+ j 1)))) \\*
~~~~~~~~~~(smalltick q) \\*
~~~~~~~~~~(smalltick (- q))))) \\
~~~~(flet ((tick (n) (straight-line (complex ticklength n) \\*
~~~~~~~~~~~~~~~~~~~~~~~~~~~~~~~~~~~~(complex (- ticklength) n) \\*
~~~~~~~~~~~~~~~~~~~~~~~~~~~~~~~~~~~~tickline \\*
~~~~~~~~~~~~~~~~~~~~~~~~~~~~~~~~~~~~stream)) \\
~~~~~~~~~~~(smalltick (n) (straight-line (complex smallticklength n) \\*
~~~~~~~~~~~~~~~~~~~~~~~~~~~~~~~~~~~~~~~~~(complex (- smallticklength) n) \\*
~~~~~~~~~~~~~~~~~~~~~~~~~~~~~~~~~~~~~~~~~tickline \\*
~~~~~~~~~~~~~~~~~~~~~~~~~~~~~~~~~~~~~~~~~stream))) \\
~~~~~~(comment-line stream "Imaginary axis") \\*
~~~~~~(straight-line \#c(0 -5) \#c(0 5) axisline stream) \\*
~~~~~~(dotimes (j (floor units-to-show)) \\*
~~~~~~~~(let ((q (+ j 1))) (tick q) (tick (- q)))) \\
~~~~~~(dotimes (j (floor units-to-show (/ pi 2))) \\*
~~~~~~~~(let ((q (* (/ pi 2) (+ j 1)))) \\*
~~~~~~~~~~(smalltick q) \\*
~~~~~~~~~~(smalltick (- q)))))))
\end{lisp}
\vskip 0pt plus 10pt \newpage%manual
\begin{lisp}
(defun straight-line (from to wid stream) \\*
~~(format stream \\*
~~~~~~~~~~"{\Xtilde}\%newpath~~{\Xtilde}S {\Xtilde}S moveto~~{\Xtilde}S {\Xtilde}S lineto~~{\Xtilde}S {\Xtilde} \\
~~~~~~~~~~~setlinewidth~~1~~setlinecap~~stroke" \\*
~~~~~~~~~~(realpart from) \\*
~~~~~~~~~~(imagpart from) \\*
~~~~~~~~~~(realpart to) \\*
~~~~~~~~~~(imagpart to) \\*
~~~~~~~~~~wid)) \\
\end{lisp}
\begin{lisp}
;;; This function draws the lines for the pattern. \\
(defun moby-lines (orientation signum plotfn stream) \\*
~~(let ((paramfn (ecase orientation \\*
~~~~~~~~~~~~~~~~~~~(:horizontal (if (< signum 0) \#'-hline \#'hline)) \\*
~~~~~~~~~~~~~~~~~~~(:vertical (if (< signum 0) \#'-vline \#'vline))))) \\
~~~~(flet ((foo (from to other wid) \\*
~~~~~~~~~~~~~(ecase orientation \\*
~~~~~~~~~~~~~~~(:horizontal \\*
~~~~~~~~~~~~~~~~(comment-line stream \\*
~~~~~~~~~~~~~~~~~~~~~~~~~~~~~~"Horizontal line from ({\Xtilde}S, {\Xtilde}S) to ({\Xtilde}S, {\Xtilde}S)" \\*
~~~~~~~~~~~~~~~~~~~~~~~~~~~~~~(round-real (* signum from)) \\*
~~~~~~~~~~~~~~~~~~~~~~~~~~~~~~(round-real other) \\*
~~~~~~~~~~~~~~~~~~~~~~~~~~~~~~(round-real (* signum to)) \\*
~~~~~~~~~~~~~~~~~~~~~~~~~~~~~~(round-real other))) \\
~~~~~~~~~~~~~~~(:vertical \\*
~~~~~~~~~~~~~~~~(comment-line stream \\*
~~~~~~~~~~~~~~~~~~~~~~~~~~~~~~"Vertical line from ({\Xtilde}S, {\Xtilde}S) to ({\Xtilde}S, {\Xtilde}S)" \\*
~~~~~~~~~~~~~~~~~~~~~~~~~~~~~~(round-real other) \\*
~~~~~~~~~~~~~~~~~~~~~~~~~~~~~~(round-real (* signum from)) \\*
~~~~~~~~~~~~~~~~~~~~~~~~~~~~~~(round-real other) \\*
~~~~~~~~~~~~~~~~~~~~~~~~~~~~~~(round-real (* signum to))))) \\
~~~~~~~~~~~~~(postscript-path \\*
~~~~~~~~~~~~~~~stream \\*
~~~~~~~~~~~~~~~(parametric-path from \\*
~~~~~~~~~~~~~~~~~~~~~~~~~~~~~~~~to \\*
~~~~~~~~~~~~~~~~~~~~~~~~~~~~~~~~(funcall paramfn other) \\*
~~~~~~~~~~~~~~~~~~~~~~~~~~~~~~~~plotfn)) \\*
~~~~~~~~~~~~~(postscript-penstroke stream wid))) \\
~~~~~~(let* ((thick 0.05) \\*
~~~~~~~~~~~~~(thin 0.02)) \\*
~~~~~~~~;; Main axis \\*
~~~~~~~~(foo 0.5 tiny 0.0 thick) \\*
~~~~~~~~(foo 0.5 1.0 0.0 thick) \\*
~~~~~~~~(foo 2.0 1.0 0.0 thick) \\*
~~~~~~~~(foo 2.0 big 0.0 thick) \\
~~~~~~~~;; Parallels at 1 and -1 \\*
~~~~~~~~(foo 2.0 tiny 1.0 thin) \\*
~~~~~~~~(foo 2.0 big 1.0 thin) \\*
~~~~~~~~(foo 2.0 tiny -1.0 thin) \\*
~~~~~~~~(foo 2.0 big -1.0 thin) \\
~~~~~~~~;; Parallels at 2, 3, -2, -3 \\*
~~~~~~~~(foo tiny big 2.0 thin) \\*
~~~~~~~~(foo tiny big -2.0 thin) \\*
~~~~~~~~(foo tiny big 3.0 thin) \\*
~~~~~~~~(foo tiny big -3.0 thin))))) \\
\\
(defun splice (p q) \\*
~~(let ((v (car (last p))) \\*
~~~~~~~~(w (first q))) \\
~~~~(and (far-out v) \\*
~~~~~~~~~(far-out w) \\*
~~~~~~~~~(>= (abs (- v w)) path-outer-delta) \\
~~~~~~~~~;; Two far-apart far-out points.~~Try to walk around \\*
~~~~~~~~~;;~~outside the perimeter, in the shorter direction. \\*
~~~~~~~~~(let* ((pdiff (phase (/ v w))) \\*
~~~~~~~~~~~~~~~~(npoints (floor (abs pdiff) (asin .2))) \\*
~~~~~~~~~~~~~~~~(delta (/ pdiff (+ npoints 1))) \\*
~~~~~~~~~~~~~~~~(incr (cis delta))) \\
~~~~~~~~~~~(do ((j 0 (+ j 1)) \\*
~~~~~~~~~~~~~~~~(p (list w "end splice") (cons (* (car p) incr) p))) \\*
~~~~~~~~~~~~~~~((= j npoints) (cons "start splice" p))))))) \\
\end{lisp}
\begin{lisp}
;;; This function draws the annuli for the pattern. \\
(defun shaded-annulus (inner outer sectors firstshade lastshade fn stream) \\*
~~(assert (zerop (mod sectors 4))) \\*
~~(comment-line stream "Annulus {\Xtilde}S {\Xtilde}S {\Xtilde}S {\Xtilde}S {\Xtilde}S" \\*
~~~~~~~~~~~~~~~~(round-real inner) (round-real outer) \\*
~~~~~~~~~~~~~~~~sectors firstshade lastshade) \\
~~(dotimes (jj sectors) \\*
~~~~(let ((j (- sectors jj 1))) \\
~~~~~~(let* ((lophase (+ tiny (* 2 pi (/ j sectors)))) \\*
~~~~~~~~~~~~~(hiphase (* 2 pi (/ (+ j 1) sectors))) \\*
~~~~~~~~~~~~~(midphase (/ (+ lophase hiphase) 2.0)) \\*
~~~~~~~~~~~~~(midradius (/ (+ inner outer) 2.0)) \\*
~~~~~~~~~~~~~(quadrant (floor (* j 4) sectors))) \\
~~~~~~~~(comment-line stream "Sector from {\Xtilde}S to {\Xtilde}S (quadrant {\Xtilde}S)" \\*
~~~~~~~~~~~~~~~~~~~~~~(round-real lophase) \\*
~~~~~~~~~~~~~~~~~~~~~~(round-real hiphase) \\*
~~~~~~~~~~~~~~~~~~~~~~quadrant) \\
~~~~~~~~(let ((p0 (reverse (parametric-path midradius \\*
~~~~~~~~~~~~~~~~~~~~~~~~~~~~~~~~~~~~~~~~~~~~inner \\*
~~~~~~~~~~~~~~~~~~~~~~~~~~~~~~~~~~~~~~~~~~~~(radial lophase quadrant) \\*
~~~~~~~~~~~~~~~~~~~~~~~~~~~~~~~~~~~~~~~~~~~~fn))) \\
~~~~~~~~~~~~~~(p1 (parametric-path midradius \\*
~~~~~~~~~~~~~~~~~~~~~~~~~~~~~~~~~~~outer \\*
~~~~~~~~~~~~~~~~~~~~~~~~~~~~~~~~~~~(radial lophase quadrant) \\*
~~~~~~~~~~~~~~~~~~~~~~~~~~~~~~~~~~~fn)) \\*
~~~~~~~~~~~~~~(p2 (reverse (parametric-path midphase \\*
~~~~~~~~~~~~~~~~~~~~~~~~~~~~~~~~~~~~~~~~~~~~lophase \\*
~~~~~~~~~~~~~~~~~~~~~~~~~~~~~~~~~~~~~~~~~~~~(circumferential outer \\*
~~~~~~~~~~~~~~~~~~~~~~~~~~~~~~~~~~~~~~~~~~~~~~~~~~~~~~~~~~~~~quadrant) \\*
~~~~~~~~~~~~~~~~~~~~~~~~~~~~~~~~~~~~~~~~~~~~fn))) \\
~~~~~~~~~~~~~~(p3 (parametric-path midphase \\*
~~~~~~~~~~~~~~~~~~~~~~~~~~~~~~~~~~~hiphase \\*
~~~~~~~~~~~~~~~~~~~~~~~~~~~~~~~~~~~(circumferential outer quadrant) \\*
~~~~~~~~~~~~~~~~~~~~~~~~~~~~~~~~~~~fn)) \\
~~~~~~~~~~~~~~(p4 (reverse (parametric-path midradius \\*
~~~~~~~~~~~~~~~~~~~~~~~~~~~~~~~~~~~~~~~~~~~~outer \\*
~~~~~~~~~~~~~~~~~~~~~~~~~~~~~~~~~~~~~~~~~~~~(radial hiphase quadrant) \\*
~~~~~~~~~~~~~~~~~~~~~~~~~~~~~~~~~~~~~~~~~~~~fn))) \\
~~~~~~~~~~~~~~(p5 (parametric-path midradius \\*
~~~~~~~~~~~~~~~~~~~~~~~~~~~~~~~~~~~inner \\*
~~~~~~~~~~~~~~~~~~~~~~~~~~~~~~~~~~~(radial hiphase quadrant) \\*
~~~~~~~~~~~~~~~~~~~~~~~~~~~~~~~~~~~fn)) \\
~~~~~~~~~~~~~~(p6 (reverse (parametric-path midphase \\*
~~~~~~~~~~~~~~~~~~~~~~~~~~~~~~~~~~~~~~~~~~~~hiphase \\*
~~~~~~~~~~~~~~~~~~~~~~~~~~~~~~~~~~~~~~~~~~~~(circumferential inner \\*
~~~~~~~~~~~~~~~~~~~~~~~~~~~~~~~~~~~~~~~~~~~~~~~~~~~~~~~~~~~~~quadrant) \\*
~~~~~~~~~~~~~~~~~~~~~~~~~~~~~~~~~~~~~~~~~~~~fn))) \\
~~~~~~~~~~~~~~(p7 (parametric-path midphase \\*
~~~~~~~~~~~~~~~~~~~~~~~~~~~~~~~~~~~lophase \\*
~~~~~~~~~~~~~~~~~~~~~~~~~~~~~~~~~~~(circumferential inner quadrant) \\*
~~~~~~~~~~~~~~~~~~~~~~~~~~~~~~~~~~~fn))) \\
~~~~~~~~~~(postscript-closed-path stream \\*
~~~~~~~~~~~~(append \\*
~~~~~~~~~~~~~~p0 (splice p0 p1) '("middle radial") \\*
~~~~~~~~~~~~~~p1 (splice p1 p2) '("end radial") \\
~~~~~~~~~~~~~~p2 (splice p2 p3) '("middle circumferential") \\
~~~~~~~~~~~~~~p3 (splice p3 p4) '("end circumferential") \\
~~~~~~~~~~~~~~p4 (splice p4 p5) '("middle radial") \\
~~~~~~~~~~~~~~p5 (splice p5 p6) '("end radial") \\
~~~~~~~~~~~~~~p6 (splice p6 p7) '("middle circumferential") \\
~~~~~~~~~~~~~~p7 (splice p7 p0) '("end circumferential") \\*
~~~~~~~~~~~~~~)))
\end{lisp}
\vskip 0pt plus 10pt \newpage%manual
\begin{lisp}
~~~~~~~~(postscript-shade stream \\*
~~~~~~~~~~~~~~~~~~~~~~~~~~(/ (+ (* firstshade (- (- sectors 1) j)) \\*
~~~~~~~~~~~~~~~~~~~~~~~~~~~~~~~~(* lastshade j)) \\*
~~~~~~~~~~~~~~~~~~~~~~~~~~~~~(- sectors 1))))))) \\
\\
(defun postscript-penstroke (stream wid) \\*
~~(format stream "{\Xtilde}\%{\Xtilde}S setlinewidth~~~1 setlinecap~~stroke" \\
~~~~~~~~~~wid)) \\
\\
(defun postscript-shade (stream shade) \\*
~~(format stream "{\Xtilde}\%currentgray~~~{\Xtilde}S setgray~~~fill~~~setgray" \\
~~~~~~~~~~shade)) \\
\\
(defun postscript-closed-path (stream path) \\*
~~(unless (every \#'far-out (remove-if-not \#'numberp path)) \\*
~~~~(postscript-raw-path stream path) \\*
~~~~(format stream "{\Xtilde}\%~~closepath"))) \\
\\
(defun postscript-path (stream path) \\*
~~(unless (every \#'far-out (remove-if-not \#'numberp path)) \\*
~~~~(postscript-raw-path stream path))) \\
\\
;;; Print a path as a series of PostScript "lineto" commands. \\*
(defun postscript-raw-path (stream path) \\*
~~(format stream "{\Xtilde}\%newpath") \\*
~~(let ((fmt "{\Xtilde}\%~~{\Xtilde}S {\Xtilde}S moveto")) \\*
~~~~(dolist (pt path) \\*
~~~~~~(cond ((stringp pt) \\*
~~~~~~~~~~~~~(format stream "{\Xtilde}\%~~\%{\Xtilde}A" pt)) \\
~~~~~~~~~~~~(t (format stream \\*
~~~~~~~~~~~~~~~~~~~~~~~fmt \\*
~~~~~~~~~~~~~~~~~~~~~~~(clamp-real (realpart pt)) \\*
~~~~~~~~~~~~~~~~~~~~~~~(clamp-real (imagpart pt))) \\*
~~~~~~~~~~~~~~~(setq fmt "{\Xtilde}\%~~{\Xtilde}S {\Xtilde}S lineto")))))) \\
\\
;;; Definitions of functions to be plotted that are not \\*
;;; standard Common Lisp functions. \\*
\\*
(defun one-plus-over-one-minus (x) (/ (+ 1 x) (- 1 x))) \\
\\
(defun one-minus-over-one-plus (x) (/ (- 1 x) (+ 1 x))) \\
\\
(defun sqrt-square-minus-one (x) (sqrt (- 1 (* x x)))) \\
\\
(defun sqrt-one-plus-square (x) (sqrt (+ 1 (* x x))))
\end{lisp}
\vskip 0pt plus 10pt \newpage%manual
\begin{lisp}
;;; Because X3J13 voted for a new definition of the atan function, \\*
;;; the following definition was used in place of the atan function \\*
;;; provided by the Common Lisp implementation I was using. \\*
\\*
(defun good-atan (x) \\*
~~(/ (- (log (+ 1 (* x \#c(0 1)))) \\*
~~~~~~~~(log (- 1 (* x \#c(0 1))))) \\*
~~~~~\#c(0 2))) \\
\\
;;; Because the first edition had an erroneous definition of atanh, \\*
;;; the following definition was used in place of the atanh function \\*
;;; provided by the Common Lisp implementation I was using. \\*
\\*
(defun really-good-atanh (x) \\*
~~(/ (- (log (+ 1 x)) \\
~~~~~~~~(log (- 1 x))) \\
~~~~~2)) \\
\end{lisp}
\begin{lisp}
;;; This is the main procedure that is intended to be called by a user. \\*
(defun picture (\&optional (fn \#'sqrt)) \\*
~~(with-open-file (stream (concatenate 'string \\*
~~~~~~~~~~~~~~~~~~~~~~~~~~~~~~~~~~~~~~~(string-downcase (string fn)) \\*
~~~~~~~~~~~~~~~~~~~~~~~~~~~~~~~~~~~~~~~"-plot.ps") \\*
~~~~~~~~~~~~~~~~~~~~~~~~~~:direction :output) \\
~~~~(format stream "\% PostScript file for plot of function {\Xtilde}S{\Xtilde}\%" fn) \\*
~~~~(format stream "\% Plot is to fit in a region {\Xtilde}S inches square{\Xtilde}\%" \\*
~~~~~~~~~~~~(/ text-width-in-picas 6.0)) \\
~~~~(format stream \\*
~~~~~~~~~~~~"\%~~showing axes extending {\Xtilde}S units from the origin.{\Xtilde}\%" \\*
~~~~~~~~~~~~units-to-show) \\
~~~~(let ((scaling (/ (* text-width-in-picas 12) (* units-to-show 2)))) \\*
~~~~~~(format stream "{\Xtilde}\%{\Xtilde}S {\Xtilde}:*{\Xtilde}S scale" scaling)) \\
~~~~(format stream "{\Xtilde}\%{\Xtilde}S {\Xtilde}:*{\Xtilde}S translate" units-to-show) \\
~~~~(format stream "{\Xtilde}\%newpath") \\
~~~~(format stream "{\Xtilde}\%~~{\Xtilde}S {\Xtilde}S moveto" (- units-to-show) (- units-to-show)) \\
~~~~(format stream "{\Xtilde}\%~~{\Xtilde}S {\Xtilde}S lineto" units-to-show (- units-to-show)) \\
~~~~(format stream "{\Xtilde}\%~~{\Xtilde}S {\Xtilde}S lineto" units-to-show units-to-show) \\
~~~~(format stream "{\Xtilde}\%~~{\Xtilde}S {\Xtilde}S lineto" (- units-to-show) units-to-show) \\
~~~~(format stream "{\Xtilde}\%~~closepath") \\
~~~~(format stream "{\Xtilde}\%clip") \\
~~~~(moby-grid fn stream) \\
~~~~(format stream \\*
~~~~~~~~~~~~"{\Xtilde}\%\% End of PostScript file for plot of function {\Xtilde}S" \\*
~~~~~~~~~~~~fn) \\
~~~~(terpri stream)))
\end{lisp}
\endgroup
\end{new}

\section{Type Conversions and Component Extractions on Numbers}

While most arithmetic functions will operate on any kind of number,
coercing types if necessary, the following functions are provided to
allow specific conversions of data types to be forced when desired.

\begin{defun}[Function]
float number &optional other

This converts any non-complex number to a floating-point number.
With no second argument, if {\it number} is already a floating-point
number, then {\it number} is returned;
otherwise a \cdf{single-float} is produced.
If the argument {\it other} is provided, then it must be a floating-point
number, and {\it number} is converted to the same format as {\it other}.
See also \cdf{coerce}.
\end{defun}

\begin{defun}[Function]
rational number \\
rationalize number

Each of these functions converts any non-complex number to a rational
number.  If the argument is already rational, it is returned.
The two functions differ in their treatment of floating-point numbers.

\cdf{rational} assumes that the floating-point number is completely accurate
and returns a rational number mathematically equal to the precise
value of the floating-point number.

\cdf{rationalize} assumes that the
floating-point number is accurate only to the precision of the
floating-point representation and may return any rational number for
which the floating-point number is the best available approximation of
its format; in doing this it attempts to keep both numerator and
denominator small.

It is always the case that
\begin{lisp}
(float (rational {\it x}) {\it x}) \EQ\ {\it x}
\end{lisp}
and
\begin{lisp}
(float (rationalize {\it x}) {\it x}) \EQ\ {\it x}
\end{lisp}
That is, rationalizing a floating-point number by either method
and then converting it back
to a floating-point number of the same format produces the original number.
What distinguishes the two functions is that \cdf{rational} typically
has a simple, inexpensive implementation, whereas \cdf{rationalize} goes
to more trouble to produce a result that is more pleasant to view and
simpler to compute with for some purposes.
\end{defun}

\begin{defun}[Function]
numerator rational \\
denominator rational

These functions take a rational number (an integer or ratio)
and return as an integer the numerator or denominator of the canonical
reduced form of the rational.  The numerator of an integer is that integer;
the denominator of an integer is \cd{1}.  Note that
\begin{lisp}
(gcd (numerator {\it x}) (denominator {\it x})) \EV\ 1
\end{lisp}
The denominator will always be a strictly positive integer;
the numerator may be any integer.
For example:
\begin{lisp}
(numerator (/ 8 -6)) \EV\ -4 \\
(denominator (/ 8 -6)) \EV\ 3
\end{lisp}
\end{defun}

There is no \cdf{fix} function in Common Lisp because there are several
interesting ways to convert non-integral values to integers.
These are provided by the functions below, which perform not only
type conversion but also some non-trivial calculations as well.

\begin{defun}[Function]
floor number &optional divisor \\
ceiling number &optional divisor \\
truncate number &optional divisor \\
round number &optional divisor

In the simple one-argument case,
each of these functions converts its argument {\it number}
(which must not be complex) to an integer.
If the argument is already an integer, it is returned directly.
If the argument is a ratio or floating-point number, the functions use
different algorithms for the conversion.

\cdf{floor} converts its argument by truncating toward negative
infinity; that is, the result is the largest integer that is not larger
than the argument.

\cdf{ceiling} converts its argument by truncating toward positive
infinity; that is, the result is the smallest integer that is not smaller
than the argument.

\cdf{truncate} converts its argument by truncating toward zero;
that is, the result is the integer of the same sign as the argument
and which has the greatest integral
magnitude not greater than that of the argument.

\cdf{round} converts its argument by rounding to the nearest
integer; if {\it number} is exactly halfway between two integers
(that is, of the form $integer+0.5$), then it is rounded to the one that
is even (divisible by 2).

The following table shows what the four functions produce when given
various arguments.

\begin{flushleft}
\cf
\begin{tabular}{@{}ccccc@{}}
{\rm Argument}&floor&ceiling&truncate&round\\
\hlinesp
~2.6& ~2& ~3& ~2& ~3 \\
~2.5& ~2& ~3& ~2& ~2 \\
~2.4& ~2& ~3& ~2& ~2 \\
~0.7& ~0& ~1& ~0& ~1 \\
~0.3& ~0& ~1& ~0& ~0 \\
-0.3& -1& ~0& ~0& ~0 \\
-0.7& -1& ~0& ~0& -1 \\
-2.4& -3& -2& -2& -2 \\
-2.5& -3& -2& -2& -2 \\
-2.6& -3& -2& -2& -3 \\
\hline
\end{tabular}
\end{flushleft}

If a second argument {\it divisor} is supplied, then the result
is the appropriate type of rounding or truncation applied to the
result of dividing the {\it number} by the {\it divisor}.
For example, \cd{(floor 5~2)}~\EQ~\cd{(floor (/~5~2))} but is potentially more
efficient.
\begin{new}%CORR
This statement is not entirely accurate; one should instead say that
\cd{(values (floor 5~2))}~\EQ~\cd{(values (floor (/~5~2)))},
because there is a second value to consider, as discussed below.
In other words, the first values returned by the two forms will be the same, but
in general the second values will differ.  Indeed, we have
\begin{lisp}
(floor 5 2) \EV\ 2 {\rm and} 1 \\
(floor (/ 5 2)) \EV\ 2 {\rm and} 1/2
\end{lisp}
for this example.
\end{new}
The {\it divisor} may be any non-complex number.
\begin{new}%CORR
It is generally accepted that it is an error for the {\it divisor} to be zero.
\end{new}
The one-argument case is exactly like the two-argument case where the second
argument is \cd{1}.

\begin{newer}
In other words, the one-argument case returns an integer and fractional part
for the {\it number}: \cd{(truncate 5.3) \EV\ 5.0 {\rm and} 0.3}, for example.
\end{newer}
Each of the functions actually returns {\it two} values,
whether given one or two arguments.  The second
result is the remainder and may be obtained using
\cdf{multiple-value-bind} and related constructs.
If any of these functions is given two arguments ${\it x}$ and ${\it y}$
and produces results ${\it q}$ and ${\it r}$, then ${\it q} \cdot {\it y}+{\it r}={\it x}$.
The first result ${\it q}$ is always an integer.
The remainder ${\it r}$ is an integer if both arguments are integers,
is rational if both arguments are rational,
and is floating-point if either argument is floating-point.
One consequence is that
in the one-argument case the remainder is always a number of the same type
as the argument.

When only one argument is given, the two results are exact;
the mathematical sum of the two results is always equal to the
mathematical value of the argument.

\goodbreak

\beforenoterule
\begin{incompatibility}
The names of the functions \cdf{floor}, \cdf{ceiling},
\cdf{truncate}, and \cdf{round} are more accurate than names like \cdf{fix}
that have heretofore been used in various Lisp systems.
The names used here are compatible with standard mathematical
terminology (and with PL/1, as it happens).  In Fortran
\cdf{ifix} means \cdf{truncate}.  Algol 68 provides \cdf{round}
and uses \cdf{entier} to mean \cdf{floor}.
In MacLisp, \cdf{fix} and \cdf{ifix} both
mean \cdf{floor} (one is generic, the other flonum-in/fixnum-out).
In Interlisp, \cdf{fix} means \cdf{truncate}.
In Lisp Machine Lisp, \cdf{fix} means \cdf{floor} and \cdf{fixr} means \cdf{round}.
Standard Lisp provides a \cdf{fix} function but does not
specify precisely what it does.  The existing usage
of the name \cdf{fix} is so confused that it seemed best to avoid it
altogether.

The names and definitions given here have recently been adopted
by Lisp Machine Lisp, and MacLisp and NIL (New Implementation of Lisp) seem likely to follow suit.
\end{incompatibility}
\afternoterule
\end{defun}

\begin{defun}[Function]
mod number divisor \\
rem number divisor

\cdf{mod} performs the operation \cdf{floor} on its two arguments
and returns the {\it second} result of \cdf{floor} as its only result.
Similarly,
\cdf{rem} performs the operation \cdf{truncate} on its arguments
and returns the {\it second} result of \cdf{truncate} as its only result.

\cdf{mod} and \cdf{rem} are therefore the usual modulus
and remainder functions when applied to two integer arguments.
In general, however, the arguments may be integers or floating-point
numbers.
\begin{lisp}
\hskip 0.5\textwidth\=\kill
(mod 13 4) \EV\ 1\>(rem 13 4) \EV\ 1 \\
(mod -13 4) \EV\ 3\>(rem -13 4) \EV\ -1 \\
(mod 13 -4) \EV\ -3\>(rem 13 -4) \EV\ 1 \\
(mod -13 -4) \EV\ -1\>(rem -13 -4) \EV\ -1 \\
(mod 13.4 1) \EV\ 0.4\>(rem 13.4 1) \EV\ 0.4 \\
(mod -13.4 1) \EV\ 0.6\>(rem -13.4 1) \EV\ -0.4
\end{lisp}

\beforenoterule
\begin{incompatibility}
The Interlisp function \cdf{remainder} is essentially
equivalent to the Common Lisp function \cdf{rem}.  The MacLisp function \cdf{remainder}
is like \cdf{rem} but accepts only integer arguments.
\end{incompatibility}
\afternoterule
\end{defun}

\begin{defun}[Function]
ffloor number &optional divisor \\
fceiling number &optional divisor \\
ftruncate number &optional divisor \\
fround number &optional divisor

These functions are just like \cdf{floor}, \cdf{ceiling}, \cdf{truncate}, and
\cdf{round}, except that the result (the first result of two) is always a
floating-point number rather than an integer.  It is roughly as if
\cdf{ffloor} gave its arguments to \cdf{floor}, and then applied \cdf{float} to
the first result before passing them both back.  In practice, however,
\cdf{ffloor} may be implemented much more efficiently.  Similar remarks
apply to the other three functions.  If the first argument is a
floating-point number, and the second argument is not a floating-point
number of longer format, then the first result will be a floating-point
number of the same type as the first argument.
For example:
\begin{lisp}
(ffloor -4.7) \EV\ -5.0 and 0.3 \\
(ffloor 3.5d0) \EV\ 3.0d0 and 0.5d0
\end{lisp}
\end{defun}

\begin{defun}[Function]
decode-float float \\
scale-float float integer \\
float-radix float \\
float-sign float1 &optional float2 \\
float-digits float \\
float-precision float \\
integer-decode-float float

The function \cdf{decode-float} takes a floating-point number
and returns three values.

The first value is a new floating-point number of the same format
representing the significand; the second value is an integer
representing the exponent; and the third value is a floating-point
number of the same format indicating the sign ($-1.0$ or $1.0$).
Let {\it b} be the radix for the floating-point representation;
then \cdf{decode-float} divides the argument by an integral power of {\it b}
so as to bring its value between 1/{\it b} (inclusive) and 1 (exclusive)
and returns the quotient as the first value.
If the argument is zero, however, the result
is equal to the absolute value of the argument (that is, if there is a negative
zero, its significand is considered to be a positive~zero).

The second value of \cdf{decode-float} is
the integer exponent {\it e} to which {\it b} must be raised
to produce the appropriate power for the division.
If the argument is zero, any integer value may be returned, provided
that the identity shown below for \cdf{scale-float} holds.

The third value of \cdf{decode-float} is a floating-point number,
of the same format as the argument, whose absolute value is 1
and whose sign matches that of the argument.

The function \cdf{scale-float} takes a floating-point number {\it f}
(not necessarily between 1/{\it b} and 1) and
an integer {\it k}, and returns \cd{(* {\it f} (expt (float {\it b} {\it f}) {\it k}))}.
(The use of \cdf{scale-float} may be much more efficient than using
exponentiation and multiplication and avoids intermediate
overflow and underflow if the final result is representable.)

Note that
\begin{lisp}
(multiple-value-bind (signif expon sign) \\*
~~~~~~~~~~~~~~~~~~~~~(decode-float {\it f}) \\*
~~(scale-float signif expon)) \\*
\EQ\ (abs {\it f})
\end{lisp}
and
\begin{lisp}
(multiple-value-bind (signif expon sign) \\*
~~~~~~~~~~~~~~~~~~~~~(decode-float {\it f}) \\*
~~(* (scale-float signif expon) sign)) \\*
\EQ\ {\it f}
\end{lisp}

The function \cdf{float-radix} returns (as an integer)
the radix {\it b} of the floating-point argument.

The function \cdf{float-sign} returns a floating-point number ${\it z}$ such
that ${\it z}$ and {\it float1} have the same sign and also such that
${\it z}$ and {\it float2} have the same absolute value.
The argument {\it float2} defaults to the value of \cd{(float 1 {\it float1})};
\cd{(float-sign x)} therefore always produces a \cd{1.0} or \cd{-1.0}
of appropriate format
according to the sign of \cdf{x}.  (Note that if an implementation
has distinct representations for negative zero and positive zero,
then \cd{(float-sign -0.0)} \EV\ \cd{-1.0}.)

The function \cdf{float-digits} returns, as a non-negative integer,
the number of radix-{\it b} digits
used in the representation of its argument (including any implicit
digits, such as a ``hidden bit'').
The function \cdf{float-precision}
returns, as a non-negative integer,
the number of significant radix-{\it b} digits present in the
argument; if the argument is (a floating-point)
zero, then the result is (an integer) zero.
For normalized floating-point numbers, the results of \cdf{float-digits}
and \cdf{float-precision}
will be the same, but the precision will be less than the
number of representation digits for a denormalized or zero number.

The function \cdf{integer-decode-float} is similar to \cdf{decode-float}
but for its first value returns,
as an \cdf{integer}, the significand scaled so as to be an integer.
For an argument {\it f}, this integer will be strictly less than
\begin{lisp}
\cd{(expt {\it b} (float-precision {\it f}))}
\end{lisp}
but no less than
\begin{lisp}
\cd{(expt {\it b} (- (float-precision {\it f}) 1))}
\end{lisp}
except that if {\it f} is zero, then the integer value will be zero.

The second value bears the same relationship to the first value
as for \cdf{decode-float}:
\begin{lisp}
(multiple-value-bind (signif expon sign) \\*
~~~~~~~~~~~~~~~~~~~~~(integer-decode-float {\it f}) \\*
~~(scale-float (float signif {\it f}) expon)) \\*
\EQ\ (abs {\it f})
\end{lisp}

The third value of \cdf{integer-decode-float} will be \cd{1} or \cd{-1}.

\beforenoterule
\begin{rationale}
These functions allow the writing of machine-independent,
or at least machine-parameterized, floating-point software of reasonable
efficiency.
\end{rationale}
\afternoterule
\end{defun}

\begin{defun}[Function]
complex realpart &optional imagpart

The arguments must be non-complex numbers; a number is returned
that has {\it realpart} as its real part and {\it imagpart} as its imaginary
part, possibly converted according to the rule of floating-point
contagion (thus both components will be of the same type).
If {\it imagpart} is not specified,
then \cd{(coerce 0 (type-of {\it realpart}))} is
effectively used.  Note that if both the {\it realpart} and {\it imagpart} are
rational and the {\it imagpart} is zero, then the result is just the
{\it realpart} because of the rule of canonical representation
for complex rationals.  It follows that the result of \cdf{complex}
is not always a complex number; it may be simply a \cdf{rational}.
\end{defun}

\begin{defun}[Function]
realpart number \\
imagpart number

These return the real and imaginary parts of a complex number.  If
{\it number} is a non-complex number, then \cdf{realpart} returns its
argument {\it number} and \cdf{imagpart}
returns \cd{(* 0 {\it number})}, which
has the effect that the imaginary part of a rational is \cd{0} and that of
a floating-point number is a floating-point zero of the same format.

\begin{newer}
A clever way to multiply a complex number {\it z} by {\it i} is to write
\begin{lisp}
(complex (- (imagpart {\it z})) (realpart {\it z}))
\end{lisp}
instead of \cd{(*~{\it z} \#c(0~1))}.  This cleverness is not always
gratuitous; it may be of particular importance in the presence of minus
zero.  For example, if we are using IEEE standard floating-point arithmetic
and $z=4+0{\it i}$, the result of the clever expression is $-0+4{\it i}$, a true
$90^\circ$ rotation of ${\it z}$, whereas the result of \cd{(*~{\it z} \#c(0~1))}
is likely to be
\begin{tabbing}
$ (4+0{\it i\/})(+0+{\it i\/}) = ((4)(+0) - (+0)(1)) + ((4)(1) + (+0)(+0)){\it i} $ \\
\hskip2pc$ = ((+0)-(+0))+((4)+(+0)){\it i} = +0+4{\it i} $
\end{tabbing}
which could
land on the wrong side of a branch cut, for example.
\end{newer}
\end{defun}


\section{Logical Operations on Numbers}

The logical operations in this section require integers
as arguments; it is an error to supply a non-integer as an argument.
The functions all treat integers as if
they were represented in two's-complement notation.

\beforenoterule
\begin{implementation}
Internally, of course, an implementation of
Common Lisp may or may not use a two's-complement representation.
All that is necessary is that the logical operations
perform calculations so as to give this appearance to the user.
\end{implementation}
\afternoterule

The logical operations provide a convenient way to represent
an infinite vector of bits.  Let such a conceptual vector be
indexed by the non-negative integers.  Then bit ${\it j}$ is assigned
a ``weight'' $2^{j}$.
Assume that only a finite number of bits are 1's
or only a finite number of bits are 0's.
A vector with only a finite number of one-bits is represented
as the sum of the weights of the one-bits, a positive integer.
A vector with only a finite number of zero-bits is represented
as \cd{-1} minus the sum of the weights of the zero-bits, a negative integer.

This method of using integers to represent bit-vectors can in turn
be used to represent sets.  Suppose that some (possibly countably
infinite) universe of discourse
for sets is mapped into the non-negative integers.
Then a set can be represented as a bit vector; an element is in the
set if the bit whose index corresponds to that element is a one-bit.
In this way all finite sets can be represented (by positive
integers), as well as all sets whose complements are finite
(by negative integers).  The functions \cdf{logior}, \cdf{logand},
and \cdf{logxor} defined below then compute the union,
intersection, and symmetric difference operations on sets
represented in this way.

\begin{defun}[Function]
logior &rest integers

This returns the bit-wise logical {\it inclusive or} of its arguments.
If no argument is given, then the result is zero,
which is an identity for this operation.
\end{defun}

\begin{defun}[Function]
logxor &rest integers

This returns the bit-wise logical {\it exclusive or} of its arguments.
If no argument is given, then the result is zero,
which is an identity for this operation.
\end{defun}

\begin{defun}[Function]
logand &rest integers

This returns the bit-wise logical {\it and} of its arguments.
If no argument is given, then the result is \cd{-1},
which is an identity for this operation.
\end{defun}

\begin{defun}[Function]
logeqv &rest integers

This returns the bit-wise logical {\it equivalence} (also known as {\it exclusive nor})
of its arguments.
If no argument is given, then the result is \cd{-1},
which is an identity for this operation.
\end{defun}

\begin{defun}[Function]
lognand integer1 integer2 \\
lognor integer1 integer2 \\
logandc1 integer1 integer2 \\
logandc2 integer1 integer2 \\
logorc1 integer1 integer2 \\
logorc2 integer1 integer2

These are the other six non-trivial bit-wise logical operations
on two arguments.  Because they are not associative,
they take exactly two arguments rather than any non-negative number
of arguments.
\begin{lisp}
(lognand {\it n1} {\it n2}) \EQ\ (lognot (logand {\it n1} {\it n2})) \\[2pt]
(lognor {\it n1} {\it n2}) \EQ\ (lognot (logior {\it n1} {\it n2})) \\[2pt]
(logandc1 {\it n1} {\it n2}) \EQ\ (logand (lognot {\it n1}) {\it n2}) \\[2pt]
(logandc2 {\it n1} {\it n2}) \EQ\ (logand {\it n1} (lognot {\it n2})) \\[2pt]
(logorc1 {\it n1} {\it n2}) \EQ\ (logior (lognot {\it n1}) {\it n2}) \\[2pt]
(logorc2 {\it n1} {\it n2}) \EQ\ (logior {\it n1} (lognot {\it n2}))
\end{lisp}
\end{defun}

The ten bit-wise logical operations on two integers are summarized
in the following table:
\begin{flushleft}
\cf
\begin{tabular}{@{}rlllll@{}}
{\it integer1}&0&0&1&1 \\
{\it integer2}&0&1&0&1&{\rm Operation Name} \\
\hlinesp
\hbox{logand~~}&0&0&0&1&{\rm and} \\
\hbox{logior~~}&0&1&1&1&{\rm inclusive or} \\
\hbox{logxor~~}&0&1&1&0&{\rm exclusive or} \\
\hbox{logeqv~~}&1&0&0&1&{\rm equivalence (exclusive nor)} \\
\hbox{lognand~}&1&1&1&0&{\rm not-and} \\
\hbox{lognor~~}&1&0&0&0&{\rm not-or} \\
\hbox{logandc1}&0&1&0&0&{\rm and complement of {\it integer1} with {\it integer2}} \\
\hbox{logandc2}&0&0&1&0&{\rm and {\it integer1} with complement of {\it integer2}} \\
\hbox{logorc1~}&1&1&0&1&{\rm or complement of {\it integer1} with {\it integer2}} \\
\hbox{logorc2~}&1&0&1&1&{\rm or {\it integer1} with complement of {\it integer2}} \\
\hline
\end{tabular}
\end{flushleft}


\begin{defun}[Function][Constant]
boole op integer1 integer2 \\
boole-clr \\
boole-set \\
boole-1 \\
boole-2 \\
boole-c1 \\
boole-c2 \\
boole-and \\
boole-ior \\
boole-xor \\
boole-eqv \\
boole-nand \\
boole-nor \\
boole-andc1 \\
boole-andc2 \\
boole-orc1 \\
boole-orc2

The function \cdf{boole} takes an operation {\it op} and two integers,
and returns an integer produced by performing the logical operation
specified by {\it op} on the two integers.  The precise values of
the sixteen constants are implementation-dependent, but they are
suitable for use as the first argument to \cdf{boole}:

\vskip 0pt minus 4pt

\begin{flushleft}
\cf
\begin{tabular}{@{}rlllll@{}}
{\it integer1}&0&0&1&1 \\
{\it integer2}&0&1&0&1&{\rm Operation Performed} \\
\hlinesp
\hbox{boole-clr~~}&0&0&0&0&{\rm always 0} \\
\hbox{boole-set~~}&1&1&1&1&{\rm always 1} \\
\hbox{boole-1~~~~}&0&0&1&1&{\it integer1} \\
\hbox{boole-2~~~~}&0&1&0&1&{\it integer2} \\
\hbox{boole-c1~~~}&1&1&0&0&{\rm complement of {\it integer1}} \\
\hbox{boole-c2~~~}&1&0&1&0&{\rm complement of {\it integer2}} \\
\hbox{boole-and~~}&0&0&0&1&{\rm and} \\
\hbox{boole-ior~~}&0&1&1&1&{\rm inclusive or} \\
\hbox{boole-xor~~}&0&1&1&0&{\rm exclusive or} \\
\hbox{boole-eqv~~}&1&0&0&1&{\rm equivalence (exclusive nor)} \\
\hbox{boole-nand~}&1&1&1&0&{\rm not-and} \\
\hbox{boole-nor~~}&1&0&0&0&{\rm not-or} \\
\hbox{boole-andc1}&0&1&0&0&{\rm and complement of {\it integer1} with {\it integer2}} \\
\hbox{boole-andc2}&0&0&1&0&{\rm and {\it integer1} with complement of {\it integer2}} \\
\hbox{boole-orc1~}&1&1&0&1&{\rm or complement of {\it integer1} with {\it integer2}} \\
\hbox{boole-orc2~}&1&0&1&1&{\rm or {\it integer1} with complement of {\it integer2}} \\
\hline
\end{tabular}
\end{flushleft}

\cdf{boole} can therefore compute all sixteen logical functions on two
arguments.  In general,
\begin{lisp}
(boole boole-and x y) \EQ\ (logand x y)
\end{lisp}
and the latter is more perspicuous.  However, \cdf{boole} is useful when it
is necessary to parameterize a procedure so that it can use
one of several logical operations.
\end{defun}

\begin{defun}[Function]
lognot integer

This returns the bit-wise logical {\it not} of its argument.
Every bit of the result is the complement of the corresponding bit
in the argument.
\begin{lisp}
(logbitp {\it j} (lognot {\it x})) \EQ\ (not (logbitp {\it j} {\it x}))
\end{lisp}
\end{defun}

\begin{defun}[Function]
logtest integer1 integer2

\cdf{logtest} is a predicate that is true if any of
the bits designated by the 1's in {\it integer1} are 1's in {\it integer2}.
\begin{lisp}
(logtest {\it x} {\it y}) \EQ\ (not (zerop (logand {\it x} {\it y})))
\end{lisp}
\end{defun}

\begin{defun}[Function]
logbitp index integer

\cdf{logbitp} is true if the bit in {\it integer} whose index
is {\it index} (that is, its weight is $2^{index}$) is a one-bit;
otherwise it is false.
For example:
\begin{lisp}
(logbitp 2 6) {\rm is true} \\
(logbitp 0 6) {\rm is false} \\
(logbitp {\it k} {\it n}) \EQ\ (ldb-test (byte 1 {\it k}) {\it n})
\end{lisp}
\begin{new}
X3J13 voted in January 1989
\issue{ARGUMENTS-UNDERSPECIFIED}
to clarify that the {\it index} must be a non-negative integer.
\end{new}
\end{defun}

\begin{defun}[Function]
ash integer count

This function shifts {\it integer} arithmetically left by {\it count} bit
positions if {\it count} is positive,
or right by $-\hbox{\it count}$ bit positions if {\it count} is negative.
The sign of the result is always the same as the sign of {\it integer}.

Mathematically speaking, this operation performs the computation
${\it floor}({\it integer}\cdot2^{count}$).

Logically, this moves all of the bits in {\it integer} to the left,
adding zero-bits at the bottom, or moves them to the right,
discarding bits.  (In this context the question of what gets shifted
in on the left is irrelevant; integers, viewed as strings of bits,
are ``half-infinite,'' that is, conceptually extend infinitely far to the left.)
For example:
\begin{lisp}
(logbitp {\it j} (ash {\it n} {\it k})) \EQ\ (and (>= {\it j} {\it k}) (logbitp (- {\it j} {\it k}) {\it n}))
\end{lisp}
\end{defun}

\begin{defun}[Function]
logcount integer

The number of bits in {\it integer} is determined and returned.
If {\it integer} is positive, the \cd{1}-bits in its binary
representation are counted.  If {\it integer} is negative,
the \cd{0}-bits in its two's-complement binary representation are counted.
The result is always a non-negative integer.
For example:
\begin{lisp}
(logcount 13) \EV\ 3~~~~~~;{\rm Binary representation is} ...0001101 \\
(logcount -13) \EV\ 2~~~~~;{\rm Binary representation is} ...1110011 \\
(logcount 30) \EV\ 4~~~~~~;{\rm Binary representation is} ...0011110 \\
(logcount -30) \EV\ 4~~~~~;{\rm Binary representation is} ...1100010
\end{lisp}
The following identity always holds:
\begin{lisp}
(logcount x) \EQ\ (logcount (- (+ x 1))) \\
~~~~~~~~~~~~~\EQ\ (logcount (lognot x))
\end{lisp}
\end{defun}

\begin{defun}[Function]
integer-length integer

This function performs the computation
\begin{tabbing}
$ {\it ceiling}(\log_2 ({\bf if}\; {\it integer} < 0 \;{\bf then} \;
    -{\it integer} \;{\bf else}\; {\it integer}+1)) $
\end{tabbing}
This is useful in two different ways.
First, if {\it integer} is non-negative, then its value can be represented
in unsigned binary form in a field whose width in bits is
no smaller than \cd{(integer-length {\it integer})}.
Second, regardless of the sign of {\it integer}, its value can be
represented in signed binary two's-complement form in a field
whose width in bits is no smaller than \cd{(+ (integer-length {\it integer}) 1)}.
For example:
\begin{lisp}
(integer-length 0) \EV\ 0 \\
(integer-length 1) \EV\ 1 \\
(integer-length 3) \EV\ 2 \\
(integer-length 4) \EV\ 3 \\
(integer-length 7) \EV\ 3 \\
(integer-length -1) \EV\ 0 \\
(integer-length -4) \EV\ 2 \\
(integer-length -7) \EV\ 3 \\
(integer-length -8) \EV\ 3
\end{lisp}

\beforenoterule
\begin{incompatibility}
This function is similar to the MacLisp
function \cdf{haulong}.  One may define \cdf{haulong} as
\begin{lisp}
(haulong x) \EQ\ (integer-length (abs x))
\end{lisp}
\end{incompatibility}
\afternoterule
\end{defun}


\section{Byte Manipulation Functions}

Several functions are provided for dealing with an arbitrary-width field of
contiguous bits appearing anywhere in an integer.
Such a contiguous set of bits is called a {\it byte}.
Here the term {\it byte} does not imply some fixed number of bits
(such as eight), rather a field of arbitrary and user-specifiable width.

The byte-manipulation functions use objects called {\it byte specifiers} to
designate a specific byte position within an integer.
The representation of a byte specifier is implementation-dependent;
in particular, it may or may not be a number.
It is sufficient to know that the function \cdf{byte} will construct one,
and that the byte-manipulation functions will accept them.
The function \cdf{byte} accepts two integers representing
the {\it position} and {\it size} of the byte and returns
a byte specifier.
Such a specifier designates a byte whose width is {\it size}
and whose bits have weights $ 2^{position+size-1} $
through $ 2^{position} $.

\begin{defun}[Function]
byte size position

\cdf{byte} takes two integers representing the size and position
of a byte and returns a byte specifier suitable for use
as an argument to byte-manipulation functions.
\end{defun}

\begin{defun}[Function]
byte-size bytespec \\
byte-position bytespec

Given a byte specifier, \cdf{byte-size} returns the size specified as an
integer; \cdf{byte-position} similarly returns the position.
For example:
\begin{lisp}
(byte-size (byte {\it j} {\it k})) \EQ\ {\it j} \\
(byte-position (byte {\it j} {\it k})) \EQ\ {\it k}
\end{lisp}
\end{defun}

\begin{defun}[Function]
ldb bytespec integer

{\it bytespec} specifies a byte of {\it integer} to be extracted.
The result is returned as a non-negative integer.
For example:
\begin{lisp}
(logbitp {\it j} (ldb (byte {\it s} {\it p}) {\it n})) \EQ\ (and (< {\it j} {\it s}) (logbitp (+ {\it j} {\it p}) {\it n}))
\end{lisp}
The name of the function \cdf{ldb} means ``load byte.''

\beforenoterule
\begin{incompatibility}
The MacLisp function \cdf{haipart} can be
implemented in terms of \cdf{ldb} as follows:
\begin{lisp}
(defun haipart (integer count) \\
~~(let ((x (abs integer))) \\
~~~~(if (minusp count) \\
~~~~~~~~(ldb (byte (- count) 0) x) \\
~~~~~~~~(ldb (byte count (max 0 (- (integer-length x) count))) \\
~~~~~~~~~~~~~x))))
\end{lisp}
\end{incompatibility}
\afternoterule

If the argument {\it integer} is specified by a form that is a {\it place} form
acceptable to \cdf{setf}, then
\cdf{setf} may be used with \cdf{ldb} to modify
a byte within the integer that is stored
in that {\it place}.
The effect is to perform a \cdf{dpb} operation
and then store the result back into the {\it place}.
\end{defun}

\begin{defun}[Function]
ldb-test bytespec integer

\cdf{ldb-test} is a predicate that is true if any of
the bits designated by the byte specifier {\it bytespec} are 1's in {\it integer};
that is, it is true if the designated field is non-zero.
\begin{lisp}
(ldb-test {\it bytespec} {\it n}) \EQ\ (not (zerop (ldb {\it bytespec} {\it n})))
\end{lisp}
\end{defun}

\begin{defun}[Function]
mask-field bytespec integer

This is similar to \cdf{ldb}; however, the result contains
the specified byte
of {\it integer} in the position specified by {\it bytespec},
rather than in position 0 as with \cdf{ldb}.
The result therefore agrees with {\it integer} in the byte specified
but has zero-bits everywhere else.
For example:
\begin{lisp}
(ldb {\it bs} (mask-field {\it bs} {\it n})) \EQ\ (ldb {\it bs} {\it n}) \\
 \\
(logbitp {\it j} (mask-field (byte {\it s} {\it p}) {\it n})) \\
~~~\EQ\ (and (>= {\it j} {\it p}) (< {\it j} (+ {\it p} {\it s})) (logbitp {\it j} {\it n})) \\
 \\
(mask-field {\it bs} {\it n}) \EQ\ (logand {\it n} (dpb -1 {\it bs} 0))
\end{lisp}

If the argument {\it integer} is specified by a form that is a {\it place} form
acceptable to \cdf{setf},
then \cdf{setf} may be used with \cdf{mask-field}
to modify a byte within the integer that is stored
in that {\it place}.
The effect is to perform a \cdf{deposit-field} operation
and then store the result back into the {\it place}.
\end{defun}

\begin{defun}[Function]
dpb newbyte bytespec integer

This returns a number that is the same as {\it integer} except in the
bits specified by {\it bytespec}.  Let {\it s} be the size specified
by {\it bytespec}; then the low {\it s} bits of {\it newbyte} appear in
the result in the byte specified by {\it bytespec}.
The integer {\it newbyte} is therefore interpreted as
being right-justified, as if it were the result of \cdf{ldb}.
For example:
\begin{lisp}
(logbitp {\it j} (dpb {\it m} (byte {\it s} {\it p}) {\it n})) \\
~~\EQ\ (if \=(and (>= {\it j} {\it p}) (< {\it j} (+ {\it p} {\it s}))) \\
\>(logbitp (- {\it j} {\it p}) {\it m}) \\
\>(logbitp {\it j} {\it n}))
\end{lisp}
The name of the function \cdf{dpb} means ``deposit byte.''
\end{defun}

\begin{defun}[Function]
deposit-field newbyte bytespec integer

This function is to \cdf{mask-field} as \cdf{dpb} is to \cdf{ldb}.
The result is an integer that contains the bits of {\it newbyte}
within the byte specified by {\it bytespec}, and elsewhere contains the bits
of {\it integer}.
For example:
\begin{lisp}
(logbitp {\it j} (deposit-field {\it m} (byte {\it s} {\it p}) {\it n})) \\
~~~\EQ\ (if \=(and (>= {\it j} {\it p}) (< {\it j} (+ {\it p} {\it s}))) \\
\>(logbitp {\it j} {\it m}) \\
\>(logbitp {\it j} {\it n}))
\end{lisp}

\beforenoterule
\begin{implementation}
If the {\it bytespec} is a constant, one may of course
construct, at compile time, an equivalent mask {\it m}, for example
by computing \cd{(deposit-field -1 {\it bytespec} 0)}.  Given
this mask {\it m}, one may then compute
\begin{lisp}
(deposit-field {\it newbyte} {\it bytespec} {\it integer})
\end{lisp}
by computing
\begin{lisp}
(logior (logand {\it newbyte} {\it m}) (logand {\it integer} (lognot {\it m})))
\end{lisp}
where the result of \cd{(lognot {\it m})} can of course also be computed
at compile time.  However, the following expression
may also be used and may require fewer
temporary registers in some situations:
\begin{lisp}
(logxor {\it integer} (logand {\it m} (logxor {\it integer} {\it newbyte})))
\end{lisp}
A related, though possibly less useful, trick is that
\begin{lisp}
(let ((z (logand (logxor x y) m))) \\
~~(setq x (logxor z x)) \\
~~(setq y (logxor z y)))
\end{lisp}
interchanges those bits of \cdf{x} and \cdf{y} for which the mask \cdf{m} is
\cd{1}, and leaves alone those bits of \cdf{x} and \cdf{y} for which \cdf{m} is
\cd{0}.
\end{implementation}
\afternoterule
\end{defun}

\section{Random Numbers}
\label{RANDOM}

The Common Lisp facility for generating pseudo-random numbers has
been carefully defined to make its use reasonably portable.
While two implementations may produce different series
of pseudo-random numbers, the distribution of values should
be relatively independent of such machine-dependent aspects
as word size.

\begin{defun}[Function]
random number &optional state

\cd{(random {\it n})} accepts a positive number {\it n} and returns
a number of the same kind between zero (inclusive) and {\it n} (exclusive).
The number {\it n} may be an integer or a floating-point number.
An approximately uniform choice distribution is used.
If {\it n} is an integer, each of the possible results
occurs with (approximate) probability 1/{\it n}.
(The qualifier ``approximate'' is used because of implementation
considerations; in practice, the deviation from uniformity should be
quite small.)

The argument {\it state} must be an object of type \cdf{random-state};
it defaults to the value of the variable \cd{*random-state*}.
This object is used to maintain the state of the pseudo-random-number
generator and is altered as a side effect of the \cdf{random} operation.

\beforenoterule
\begin{incompatibility}
\cdf{random} of zero arguments as defined in MacLisp
has been omitted because
its value is too implementation-dependent (limited by fixnum range).
\end{incompatibility}
\betweennoterule
\begin{implementation}
In general, even if \cdf{random} of zero arguments
were defined as in MacLisp,
it is not adequate to define \cd{(random {\it n})} for integral {\it n}
to be simply \cd{(mod (random) {\it n})}; this fails to be uniformly distributed
if {\it n} is larger than the largest number produced by \cdf{random},
or even if {\it n} merely approaches this number.
This is another reason for omitting \cdf{random} of zero arguments in Common Lisp.
Assuming that the underlying mechanism produces ``random bits''
(possibly in chunks such as fixnums), the best approach is to produce
enough random bits to construct an integer {\it k} some number {\it d} of bits
larger than \cd{(integer-length {\it n})} (see \cdf{integer-length}), and
then compute \cd{(mod {\it k} {\it n})}.  The quantity {\it d} should be at
least 7, and preferably 10 or more.

To produce random floating-point numbers in the half-open
range $[{\it A},{\it B})$,
accepted practice (as determined by a look through the
{\it Collected Algorithms from the ACM}, particularly algorithms
133, 266, 294, and 370) is to compute ${\it X}\cdot({\it B}-{\it A})+{\it A}$,
where {\it X} is a floating-point number uniformly distributed over
$[0.0, 1.0)$
and computed by calculating a random integer ${\it N}$ in the range
$[0,{\it M})$
(typically by a multiplicative-congruential or linear-congruential method
mod ${\it M}$) and then setting ${\it X}={\it N}/{\it M}$.  See also \cite{KNUTH-VOLUME-2}.
If one takes ${\it M}=2^{\hbox{\footnotesize\it f}}$, where ${\it f}$ is the length of the significand
of a floating-point number (and it is in fact common to choose ${\it M}$
to be a power of 2), then this method is equivalent to the following
assembly-language-level procedure.  Assume the representation
has no hidden bit.  Take a floating-point 0.5,
and clobber its entire significand with random bits.  Normalize the
result if necessary.

For example, on the DEC PDP-10, assume that accumulator \cdf{T} is completely random
(all 36 bits are random).  Then the code sequence
\begin{lisp}
LSH T,-9~~~~~~~~~~~~~~~~~;{\rm Clear high 9 bits; low 27 are random} \\
FSC T,128.~~~~~~~~~~~~~~~;{\rm Install exponent and normalize}
\end{lisp}
will produce in \cdf{T} a random floating-point number uniformly distributed
over $[0.0, 1.0)$.  (Instead of the \cdf{LSH} instruction,
one could do
\begin{lisp}
TLZ T,777000~~~~~~~~~~~~~;{\rm That's 777000 octal}
\end{lisp}
but if the 36 random bits came from a congruential random-number generator,
the high-order bits tend to be ``more random'' than the low-order ones,
and so the \cdf{LSH} would be better for uniform distribution.
Ideally all the bits would be the result of high-quality randomness.)

With a hidden-bit representation, normalization is not a problem,
but dealing with the hidden bit is.  The method can be adapted as follows.
Take a floating-point 1.0 and clobber the explicit significand bits with
random bits; this produces a random floating-point number in
the range $[1.0, 2.0)$.  Then simply subtract 1.0.  In effect, we
let the hidden bit creep in and then subtract it away again.

For example, on the DEC VAX, assume that register \cdf{T} is
completely random (but a little less random than on the PDP-10, as
it has only 32 random bits).  Then the code sequence
\begin{lisp}
INSV \#{\Xcircumflex}X81,\#7,\#9,T~~~~~;{\rm Install correct sign bit and exponent} \\
SUBF \#{\Xcircumflex}F1.0,T~~~~~~~~~~;{\rm Subtract 1.0}
\end{lisp}
will produce in \cdf{T} a random floating-point number uniformly distributed
over $[0.0, 1.0)$.  Again, if the low-order bits are not random enough,
then the instruction
\begin{lisp}
ROTL \#7,T
\end{lisp}
should be performed first.

Implementors may wish to consult reference \cite{ADDITIVE-RANDOMS} for
a discussion of some efficient methods of generating pseudo-random numbers.
\end{implementation}
\afternoterule
\end{defun}

\begin{defun}[Variable]
*random-state*

This variable holds a data structure,
an object of type \cdf{random-state}, that encodes the internal state
of the random-number generator that \cdf{random} uses by default.
The nature
of this data structure is implementation-dependent.  It may be
printed out and successfully read back in, but may or may not function
correctly as a random-number state object in another implementation.
A call to \cdf{random} will perform a side effect on this data structure.
Lambda-binding this variable to a different random-number state object
will correctly save and restore the old state object.
\end{defun}

\begin{defun}[Function]
make-random-state &optional state

This function returns a new object of type \cdf{random-state},
suitable for use as the value of the variable \cd{*random-state*}.
If {\it state} is {\false} or omitted, \cdf{make-random-state} returns a {\it copy}
of the current random-number state object (the value of
the variable \cd{*random-state*}).  If {\it state} is a state object,
a copy of that state object is returned.  If {\it state} is {\true},
then a new state object is returned that has been ``randomly''
initialized by some means (such as by a time-of-day clock).

\beforenoterule
\begin{rationale}
Common Lisp purposely provides no way to initialize a \cdf{random-state}
object from a user-specified ``seed.''  The reason for this is that
the number of bits of state information in a \cdf{random-state} object
may vary widely from one implementation to another, and there is no
simple way to guarantee that any user-specified seed value will be
``random enough.''  Instead, the initialization of \cdf{random-state}
objects is left to the implementor in the case where the argument {\true}
is given to \cdf{make-random-state}.

To handle the common situation of executing the same program many times
in a reproducible manner, where that program uses \cdf{random}, the following
procedure may be used:
\begin{enumerate}
\item
Evaluate \cd{(make-random-state t)} to create a \cdf{random-state} object.

\item
Write that object to a file, using \cdf{print}, for later use.

\item
Whenever the program is to be run, first use \cdf{read} to create
a copy of the \cdf{random-state} object from the printed representation
in the file.
Then use the \cdf{random-state} object newly created by the \cdf{read} operation
to initialize the random-number generator for the program.
\end{enumerate}
It is for the sake of this procedure for reproducible execution that
implementations are required to provide a read/print syntax for objects
of type \cdf{random-state}.

It is also possible to make copies of a \cdf{random-state} object
directly without going through the print/read process, simply by
using the \cdf{make-random-state} function to copy the object; this allows
the same sequence of random numbers to be generated many times
within a single program.
\end{rationale}
\betweennoterule
\begin{implementation}
A recommended way to implement the type \cdf{random-state}
is effectively to use the machinery for \cdf{defstruct}.
The usual structure syntax may then be used for printing \cdf{random-state}
objects; one might look something like
\begin{lisp}
\#S(RANDOM-STATE DATA \#(14 49 98436589 786345 8734658324 ...))
\end{lisp}
where the components are of course completely implementation-dependent.
\end{implementation}
\afternoterule
\end{defun}

\begin{defun}[Function]
random-state-p object

\cdf{random-state-p} is true if its argument is a random-state object,
and otherwise is false.
\begin{lisp}
(random-state-p {\it x}) \EQ\ (typep {\it x} 'random-state)
\end{lisp}
\end{defun}

\section{Implementation Parameters}

The values of the named constants defined in this section are
implementation-dependent.  They may be useful for parameterizing
code in some situations.

\begin{defun}[Constant]
most-positive-fixnum \\
most-negative-fixnum

The value of \cdf{most-positive-fixnum} is that fixnum closest in value to
positive infinity provided by the implementation.

The value of \cdf{most-negative-fixnum} is that fixnum closest in value to
negative infinity provided by the implementation.

\begin{new}
X3J13 voted in January 1989
\issue{FIXNUM-NON-PORTABLE}
to specify that \cdf{fixnum} must be a supertype
of the type \cd{(signed-byte 16)}, and additionally that the value
of \cdf{array-dimension-limit} must be a fixnum.  This implies that the value
of \cdf{most-negative-fixnum} must be less than or equal to $-2^{15}$,
and the value of \cdf{most-positive-fixnum} must be greater than or equal to
both $2^{15}-1$ and the value of \cdf{array-dimension-limit}.
\end{new}
\end{defun}

\begin{defun}[Constant]
most-positive-short-float \\
least-positive-short-float \\
least-negative-short-float \\
most-negative-short-float

The value of \cdf{most-positive-short-float} is that short-format
floating-point number closest in value to (but not equal to)
positive infinity provided by the implementation.

The value of \cdf{least-positive-short-float} is that positive short-format
floating-point number closest in value to (but not equal to) zero provided by
the implementation.

The value of \cdf{least-negative-short-float} is that negative short-format
floating-point number closest in value to (but not equal to) zero provided by
the implementation.  (Note that even if an implementation supports
minus zero as a distinct short floating-point value,
\cdf{least-negative-short-float} must not be minus zero.)

\begin{newer}
X3J13 voted in June 1989 \issue{FLOAT-UNDERFLOW}
to clarify that these definitions are to be taken quite literally.
In implementations that support denormalized numbers,
the values of \cdf{least-positive-short-float} and
\cdf{least-negative-short-float} may be denormalized.
\end{newer}

The value of \cdf{most-negative-short-float} is that short-format
floating-point number closest in value to (but not equal to)
negative infinity provided by the implementation.
\end{defun}


\begin{defun}[Constant]
most-positive-single-float \\
least-positive-single-float \\
least-negative-single-float \\
most-negative-single-float \\
most-positive-double-float \\
least-positive-double-float \\
least-negative-double-float \\
most-negative-double-float \\
most-positive-long-float \\
least-positive-long-float \\
least-negative-long-float \\
most-negative-long-float

These are analogous to the constants defined above for short-format
floating-point numbers.
\end{defun}

\bigskip

\begin{newer}
\begin{defun}[Constant]
least-positive-normalized-short-float \\
least-negative-normalized-short-float

X3J13 voted in June 1989 \issue{FLOAT-UNDERFLOW}
to add these constants to the language.

The value of \cdf{least-positive-normalized-short-float} is that positive normalized
short-format
floating-point number closest in value to (but not equal to) zero provided by
the implementation.  In implementations that do not support denormalized numbers
this may be the same as the value of
\cd{least-\discretionary{}{}{}positive-\discretionary{}{}{}short-float}.

The value of \cdf{least-negative-normalized-short-float} is that negative normalized short-format
floating-point number closest in value to (but not equal to) zero provided by
the implementation.
(Note that even if an implementation supports
minus zero as a distinct short floating-point value,
\cd{least-\discretionary{}{}{}negative-\discretionary{}{}{}normalized-\discretionary{}{}{}short-float} must not be minus zero.)
In implementations that do not support denormalized numbers
this may be the same as the value of \cd{least-\discretionary{}{}{}positive-\discretionary{}{}{}short-float}.
\end{defun}

\begin{defun}[Constant]
least-positive-normalized-single-float \\
least-negative-normalized-single-float \\
least-positive-normalized-double-float \\
least-negative-normalized-double-float \\
least-positive-normalized-long-float \\
least-negative-normalized-long-float

These are analogous to the constants defined above for short-format
floating-point numbers.
\end{defun}
\end{newer}

\begin{defun}[Constant]
short-float-epsilon \\
single-float-epsilon \\
double-float-epsilon \\
long-float-epsilon

These constants have as value, for each floating-point format,
the smallest positive floating-point number {\it e} of that format such that
the expression
\begin{lisp}
(not (= (float 1 {\it e}) (+ (float 1 {\it e}) {\it e})))
\end{lisp}
is true when actually evaluated.
\end{defun}

\begin{defun}[Constant]
short-float-negative-epsilon \\
single-float-negative-epsilon \\
double-float-negative-epsilon \\
long-float-negative-epsilon

These constants have as value, for each floating-point format,
the smallest positive floating-point number {\it e} of that format such that
the expression
\begin{lisp}
(not (= (float 1 {\it e}) (- (float 1 {\it e}) {\it e})))
\end{lisp}
is true when actually evaluated.
\end{defun}

\endgroup
      % Functions on numbers
%Part{Char, Root = "CLM.MSS"}
%%% Chapter of Common Lisp Manual.  Copyright 1984, 1987, 1988, 1989 Guy L. Steele Jr.

\clearpage\def\pagestatus{FINAL PROOF}

\chapter{Characters}

Common Lisp provides a character data type; objects of this type
represent printed symbols such as letters.

In general, characters in Common Lisp are not true objects; \cdf{eq} cannot
be counted upon to operate on them reliably.  In particular,
it is possible that the expression
\begin{lisp}
(let ((x z) (y z)) (eq x y))
\end{lisp}
may be false rather than true, if the value of \cdf{z} is a character.

\beforenoterule
\begin{rationale}
This odd breakdown of \cdf{eq} in the case of characters
allows the implementor enough design freedom to produce exceptionally
efficient code on conventional architectures.  In this respect the
treatment of characters exactly parallels that of numbers, as described
in chapter~\ref{NUMBER}.
\end{rationale}
\afternoterule

\begin{table}
\caption{Standard Character Labels, Glyphs, and Descriptions}
\label{STANDARD-CHAR-REPERTOIRE-TABLE}
\tabcolsep0pt
\def\arraystretch{1.1}

\ifx \HCode\Undef
% not tex4ht ...

\begin{tabular*}{\textwidth}{@{}l@{\extracolsep{\fill}}llllllll@{}}
           &&&\cd{SM05}&\cd{{\Xatsign}}&{\rm commercial at}&\cd{SD13}&\cd{{\Xbq}}&{\rm grave accent} \\
\cd{SP02}&\cd{!}&{\rm exclamation mark}&\cd{LA02}&\cdf{A}&{\rm capital A}&\cd{LA01}&\cdf{a}&{\rm small a} \\
\cd{SP04}&\cd{"}&{\rm quotation mark}&\cd{LB02}&\cdf{B}&{\rm capital B}&\cd{LB01}&\cdf{b}&{\rm small b} \\
\cd{SM01}&\cd{\#}&{\rm number sign}&\cd{LC02}&\cdf{C}&{\rm capital C}&\cd{LC01}&\cdf{c}&{\rm small c} \\
\cd{SC03}&\cd{\$}&{\rm dollar sign}&\cd{LD02}&\cdf{D}&{\rm capital D}&\cd{LD01}&\cdf{d}&{\rm small d} \\
\cd{SM02}&\cd{\%}&{\rm percent sign}&\cd{LE02}&\cdf{E}&{\rm capital E}&\cd{LE01}&\cdf{e}&{\rm small e} \\
\cd{SM03}&\cd{\&}&{\rm ampersand}&\cd{LF02}&\cdf{F}&{\rm capital F}&\cd{LF01}&\cdf{f}&{\rm small f} \\
\cd{SP05}&\cd{'}&{\rm apostrophe}&\cd{LG02}&\cdf{G}&{\rm capital G}&\cd{LG01}&\cdf{g}&{\rm small g} \\
\cd{SP06}&\cd{(}&{\rm left parenthesis}&\cd{LH02}&\cdf{H}&{\rm capital H}&\cd{LH01}&\cdf{h}&{\rm small h} \\
\cd{SP07}&\cd{)}&{\rm right parenthesis}&\cd{LI02}&\cdf{I}&{\rm capital I}&\cd{LI01}&\cdf{i}&{\rm small i} \\
\cd{SM04}&\cdf{*}&{\rm asterisk}&\cd{LJ02}&\cdf{J}&{\rm capital J}&\cd{LJ01}&\cdf{j}&{\rm small j} \\
\cd{SA01}&\cdf{+}&{\rm plus sign}&\cd{LK02}&\cdf{K}&{\rm capital K}&\cd{LK01}&\cdf{k}&{\rm small k} \\
\cd{SP08}&\cd{,}&{\rm comma}&\cd{LL02}&\cdf{L}&{\rm capital L}&\cd{LL01}&\cdf{l}&{\rm small l} \\
\cd{SP10}&\cdf{-}&{\rm hyphen or minus sign}&\cd{LM02}&\cdf{M}&{\rm capital M}&\cd{LM01}&\cdf{m}&{\rm small m} \\
\cd{SP11}&\cd{.}&{\rm period or full stop}&\cd{LN02}&\cdf{N}&{\rm capital N}&\cd{LN01}&\cdf{n}&{\rm small n} \\
\cd{SP12}&\cdf{/}&{\rm solidus}&\cd{LO02}&\cdf{O}&{\rm capital O}&\cd{LO01}&\cdf{o}&{\rm small o} \\
\cd{ND10}&\cd{0}&{\rm digit 0}&\cd{LP02}&\cdf{P}&{\rm capital P}&\cd{LP01}&\cdf{p}&{\rm small p} \\
\cd{ND01}&\cd{1}&{\rm digit 1}&\cd{LQ02}&\cdf{Q}&{\rm capital Q}&\cd{LQ01}&\cdf{q}&{\rm small q} \\
\cd{ND02}&\cd{2}&{\rm digit 2}&\cd{LR02}&\cdf{R}&{\rm capital R}&\cd{LR01}&\cdf{r}&{\rm small r} \\
\cd{ND03}&\cd{3}&{\rm digit 3}&\cd{LS02}&\cdf{S}&{\rm capital S}&\cd{LS01}&\cdf{s}&{\rm small s} \\
\cd{ND04}&\cd{4}&{\rm digit 4}&\cd{LT02}&\cdf{T}&{\rm capital T}&\cd{LT01}&\cdf{t}&{\rm small t} \\
\cd{ND05}&\cd{5}&{\rm digit 5}&\cd{LU02}&\cdf{U}&{\rm capital U}&\cd{LU01}&\cdf{u}&{\rm small u} \\
\cd{ND06}&\cd{6}&{\rm digit 6}&\cd{LV02}&\cdf{V}&{\rm capital V}&\cd{LV01}&\cdf{v}&{\rm small v} \\
\cd{ND07}&\cd{7}&{\rm digit 7}&\cd{LW02}&\cdf{W}&{\rm capital W}&\cd{LW01}&\cdf{w}&{\rm small w} \\
\cd{ND08}&\cd{8}&{\rm digit 8}&\cd{LX02}&\cdf{X}&{\rm capital X}&\cd{LX01}&\cdf{x}&{\rm small x} \\
\cd{ND09}&\cd{9}&{\rm digit 9}&\cd{LY02}&\cdf{Y}&{\rm capital Y}&\cd{LY01}&\cdf{y}&{\rm small y} \\
\cd{SP13}&\cd{:}&{\rm colon}&\cd{LZ02}&\cdf{Z}&{\rm capital Z}&\cd{LZ01}&\cdf{z}&{\rm small z} \\
\cd{SP14}&\cd{;}&{\rm semicolon}&\cd{SM06}&\cd{{\Xlbracket}}&{\rm left square bracket}&\cd{SM11}&\cd{{\Xlbrace}}&{\rm left curly bracket} \\
\cd{SA03}&\cdf{<}&{\rm less-than sign}&\cd{SM07}&\cd{{\Xbackslash}}&{\rm reverse solidus}&\cd{SM13}&\cd{|}&{\rm vertical bar} \\
\cd{SA04}&\cdf{=}&{\rm equals sign}&\cd{SM08}&\cd{{\Xrbracket}}&{\rm right square bracket}&\cd{SM14}&\cd{{\Xrbrace}}&{\rm right curly bracket} \\
\cd{SA05}&\cdf{>}&{\rm greater-than sign}&\cd{SD15}&\cd{{\Xcircumflex}}&{\rm circumflex accent}&\cd{SD19}&\cd{{\Xtilde}}&{\rm tilde} \\
\cd{SP15}&\cd{?}&{\rm question mark}&\cd{SP09}&\cd{{\Xunderscore}}&{\rm low line}&
\end{tabular*}

\else
%... tex4ht ...

 Table unavailable 

\fi

\vfill
\begin{small}
\noindent
The characters in this table plus the space and newline characters make up
the standard Common Lisp character repertoire (type \cdf{standard-char}).
The character labels and character descriptions shown here are taken
from ISO standard 6937/2 .  The first character of the label
categorizes the character as Latin, Numeric, or Special.
\end{small}
\end{table}

If two objects are to be compared for ``identity,'' but either might be
a character, then the predicate \cdf{eql} is probably appropriate.

\begin{newer}
X3J13 voted in March 1989 \issue{CHARACTER-PROPOSAL}
to approve the following definitions and terminology for use in
discussing character facilities in Common Lisp.

A {\it character repertoire} defines a collection of characters
independent of their specific rendered image or font.  (This corresponds
to the mathematical notion of a set, but the term {\it character set}
is avoided here because it has been used in the past to mean
both what is here called a repertoire and what is here called a coded
character set.)
Character repertoires are specified independent of coding and their characters
are identified only with a unique {\it character label},
a graphic symbol, and a character description.
As an example, table~\ref{STANDARD-CHAR-REPERTOIRE-TABLE}
shows the character labels, graphic symbols, and character descriptions for
all of the characters in the repertoire \cdf{standard-char}
except for \cd{\#{\Xbackslash}Space} and \cd{\#{\Xbackslash}Newline}.

Every Common Lisp implementation must support the standard character repertoire
as well as repertoires named \cdf{base-char}, \cdf{extended-char},
and \cdf{character}.  Other repertoires may be supported as well.
X3J13 voted in June 1989 \issue{MORE-CHARACTER-PROPOSAL} to specify that names of
repertoires may be used as type specifiers.  Such types must be subtypes of \cdf{character};
that is, in a given implementation
the repertoire named \cdf{character} must encompass all the character objects supported
by that implementation.

A {\it coded character set} is a character repertoire plus an {\it encoding}
that provides a bijective mapping between each character in the set and a number
(typically a non-negative integer)
that serves as the character representation.
There are numerous internationally standardized coded character sets.

A character may be included in one or more character repertoires.
Similarly, a character may be included in one or more coded character sets.

To ensure that each character is uniquely defined, we may use a universal registry of
characters that incorporates a collection of distinguished repertoires
called {\it character scripts} that form an exhaustive partition of all characters.
That is, each character is included in exactly one character script.
(Draft ISO 10646 Coded Character Set Standard, if eventually approved as a standard,
may become the practical realization of this universal registry.)

(X3J13 voted in June 1989 \issue{MORE-CHARACTER-PROPOSAL} to specify that
an implementation must document the character scripts it supports.
For each script the documentation should discuss character labels,
glyphs, and descriptions; any canonicalization processes performed
by the reader that result in treating distinct characters as equivalent;
any canonicalization performed by \cdf{format} in processing directives;
the behavior of \cdf{char-upcase}, \cdf{char-downcase}, and the predicates
\cdf{alpha-char-p}, \cdf{upper-case-p}, \cdf{lower-case-p}, \cdf{both-case-p},
\cdf{graphic-char-p}, \cdf{alphanumericp}, \cdf{char-equal}, \cdf{char-not-equal},
\cdf{char-lessp}, \cdf{char-greaterp}, \cdf{char-not-greaterp}, and \cdf{char-not-lessp}
for characters in the script; and behavior with respect to input and output,
including coded character sets and external coding schemes.)

In Common Lisp a {\it character} data object is identified by its {\it character code},
a unique numerical code.  Each character code is composed from a character script
and a character label.  The convention by which a character script and
character label compose a character code is implementation dependent.
[X3J13 did not approve all parts of the proposal from its Subcommittee
on Characters.  As a result, some features that were approved appear to
have no purpose.  X3J13 wished to support the standardization by ISO of character
scripts and coded character sets but declined to design facilities for use in
Common Lisp until there has been more progress by ISO in this area.
The approval of the terminology for scripts and labels gives a hint
to implementors of likely directions for Common Lisp in the future.]

A character object that is classified as {\it graphic}, or displayable,
has an associated {\it glpyh}.  The glyph is the visual representation
of the character.  All other character data objects are classified as
{\it non-graphic}.

This terminology assigns names to Common Lisp concepts
in a manner consistent with
related concepts discussed in various ISO standards for coded
character sets and provides a demarcation between standardization
activities.  For example, facilities for manipulating characters,
character scripts, and coded character sets are properly defined
by a Common Lisp standard, but Common Lisp should not define
standard character sets or standard character scripts.
\end{newer}

\section{Character Attributes}

Every character has three attributes: code, bits, and font.
The code attribute is intended to distinguish among the printed glyphs
and formatting functions for characters.  The bits attribute allows extra
flags to be associated with a character.  The font attribute permits
a specification of the style of the glyphs (such as italics).

\begin{new}
The treatment of character attributes in Common Lisp has not been
entirely successful.  The font attribute has not been widely used,
for two reasons.  First, a single integer, limited in most
implementations to 255 at most, is not an adequate, convenient, or portable
representation for a font.  Second, in many applications where font
information matters it is more convenient or more efficient to represent
font information as shift codes that apply to many characters, rather than
attaching font information separately to each character.

As for the bits attribute, it was intended to support
character input from extended keyboards having extra ``shift'' keys.
This, in turn, was imagined to support the programming of a portable
EMACS-like editor in Common Lisp.  (The EMACS command set
is most convenient when the keyboard has separate ``control'' and
``meta'' keys.)   The bits attribute has been used in the implementation
of such editors and other interactive interfaces.  However, software
that relies crucially on these extended characters will not be portable
to Common Lisp implementations that do not support them.

X3J13 voted in March 1989 \issue{CHARACTER-PROPOSAL}
and in June 1989 \issue{MORE-CHARACTER-PROPOSAL}
to revise considerably the treatment
of characters in the language.  The bits and font attributes are eliminated;
instead a character may have {\it implementation-defined attributes}.
The treatment of such attributes by existing character-handling functions
is carefully constrained by certain rules.

Implementations are free to
continue to support bits and font attributes, but they are
formally regarded as implementation-defined attributes.
The rules are generally consistent with the previous
treatment of the bits and font attributes.
My guess is that
the font attribute as currently defined will wither away,
but the bits attribute as defined by the first edition will
continue to be supported as a {\it de facto} standard extension,
because it fills a useful small purpose.
\end{new}

\begin{defun}[Constant]
char-code-limit

The value of \cdf{char-code-limit} is a non-negative
integer that is the upper exclusive bound on values produced
by the function \cdf{char-code}, which returns the {\it code} component
of a given character; that is, the values returned by \cdf{char-code}
are non-negative and strictly less than the value of
\cdf{char-code-limit}.

\begin{new}
Common Lisp does not at present explicitly guarantee that all integers between
zero and the value of \cdf{char-code-limit} are valid character codes, and so
it is wise in any case for the programmer to assume that the space of
assigned character codes may be sparse.
\end{new}
\end{defun}

\begin{obsolete}
\begin{defun}[Constant]
char-font-limit

The value of \cdf{char-font-limit} is a non-negative
integer that is the upper exclusive bound on values produced
by the function \cdf{char-font}, which returns the {\it font} component
of a given character; that is, the values returned by \cdf{char-font}
are non-negative and strictly less than the value of
\cdf{char-font-limit}.

\beforenoterule
\begin{implementation}
No Common Lisp implementation is required to support
non-zero font attributes; if it does not, then \cdf{char-font-limit}
should be \cd{1}.
\end{implementation}
\afternoterule
\end{defun}
\end{obsolete}

\begin{newer}
X3J13 voted in March 1989 \issue{CHARACTER-PROPOSAL}
to eliminate \cdf{char-font-limit}.

Experience has shown that numeric codes are not an especially
convenient, let alone portable, representation for font information.
A system based on typeface names, type styles, and point sizes would be much better.
(Macintosh software developers made the same discovery and have recently
converted to a new font identification scheme.)
\end{newer}

\begin{obsolete}
\begin{defun}[Constant]
char-bits-limit

The value of \cdf{char-bits-limit} is a non-negative
integer that is the upper exclusive bound on values produced
by the function \cdf{char-bits}, which returns the {\it bits} component
of a given character; that is, the values returned by \cdf{char-bits}
are non-negative and strictly less than the value of
\cdf{char-bits-limit}.  Note that the value of \cdf{char-bits-limit}
will be a power of 2.

\beforenoterule
\begin{implementation}
No Common Lisp implementation is required to support
non-zero bits attributes; if it does not, then \cdf{char-bits-limit}
should be \cd{1}.
\end{implementation}
\afternoterule
\end{defun}
\end{obsolete}

\begin{newer}
X3J13 voted in March 1989 \issue{CHARACTER-PROPOSAL}
to eliminate \cdf{char-bits-limit}.
\end{newer}

\section{Predicates on Characters}

The predicate \cdf{characterp} may be used to determine
whether any Lisp object is a character object.

\begin{defun}[Function]
standard-char-p char

The argument {\it char} must be a character object.
\cdf{standard-char-p} is true if the argument is a ``standard character,''
that is, an object of type \cdf{standard-char}.

Note that any character with a non-zero bits or
font attribute is non-standard.
\end{defun}

\begin{defun}[Function]
graphic-char-p char

The argument {\it char} must be a character object.
\cdf{graphic-char-p} is true if the argument is a ``graphic'' (printing)
character, and false if it is a ``non-graphic'' (formatting or control)
character.  Graphic characters have a standard textual representation
as a single glyph, such as \cdf{A} or \cdf{*} or \cdf{=}.
By convention, the space character is considered to be graphic.
Of the standard characters
all but \cd{\#{\Xbackslash}Newline} are graphic.
The semi-standard characters
\cd{\#{\Xbackslash}Backspace}, \cd{\#{\Xbackslash}Tab}, \cd{\#{\Xbackslash}Rubout}, \cd{\#{\Xbackslash}Linefeed}, \cd{\#{\Xbackslash}Return},
and \cd{\#{\Xbackslash}Page} are not graphic.

Programs may assume that
graphic characters of font 0 are all of the same width
when printed, for example, for purposes of columnar
formatting.  (This does not prohibit the use of a variable-pitch font
as font 0, but merely implies that every implementation of Common Lisp
must provide {\it some} mode of operation in which font 0 is
a fixed-pitch font.)
Portable programs should assume that, in general,
non-graphic characters and characters of
other fonts may be of varying widths.

Any character with a non-zero bits attribute is non-graphic.
\end{defun}

\begin{obsolete}
\begin{defun}[Function]
string-char-p char

The argument {\it char} must be a character object.
\cdf{string-char-p} is true if {\it char} can be stored into
a string, and otherwise is false.
Any character that satisfies \cdf{standard-char-p}
also satisfies \cdf{string-char-p}; others may also.
\end{defun}
\end{obsolete}

\begin{newer}
X3J13 voted in March 1989 \issue{CHARACTER-PROPOSAL}
to eliminate \cdf{string-char-p}.
\end{newer}

\begin{defun}[Function]
alpha-char-p char

The argument {\it char} must be a character object.
\cdf{alpha-char-p} is true if the argument is an alphabetic
character, and otherwise is false.

If a character is alphabetic, then it is perforce graphic.
Therefore any character with a non-zero bits attribute cannot be alphabetic.
Whether a character is alphabetic may depend on its font number.

Of the standard characters (as defined by \cdf{standard-char-p}),
the letters \cdf{A} through \cdf{Z} and \cdf{a} through \cdf{z} are alphabetic.
\end{defun}

\begin{defun}[Function]
upper-case-p char \\
lower-case-p char \\
both-case-p char

The argument {\it char} must be a character object.

\cdf{upper-case-p} is true if the argument is an uppercase
character, and otherwise is false.

\cdf{lower-case-p} is true if the argument is a lowercase
character, and otherwise is false.

\cdf{both-case-p} is true if the argument is an uppercase character and
there is a corresponding lowercase character (which can be obtained
using \cdf{char-downcase}), or if the argument is a lowercase character and
there is a corresponding uppercase character (which can be obtained
using \cdf{char-upcase}).

If a character is either uppercase or lowercase, it is necessarily
alphabetic (and therefore is graphic, and therefore has a zero bits
attribute).  However, it is permissible in theory for an alphabetic
character to be neither uppercase nor lowercase (in a non-Roman font,
for example).

Of the standard characters (as defined by \cdf{standard-char-p}),
the letters \cdf{A} through \cdf{Z} are uppercase and \cdf{a}
through \cdf{z} are lowercase.
\end{defun}

\begin{defun}[Function]
digit-char-p char &optional (radix 10)

The argument {\it char} must be a character object,
and {\it radix} must be a non-negative integer.
If {\it char} is not a digit of the radix
specified by {\it radix}, then \cdf{digit-char-p} is
false; otherwise it returns
a non-negative integer that is the ``weight'' of {\it char} in that radix.

Digits are necessarily graphic characters.

Of the standard characters (as defined by \cdf{standard-char-p}),
the characters \cd{0} through \cd{9}, \cdf{A} through \cdf{Z},
and \cdf{a} through \cdf{z}
are digits.  The weights of \cd{0} through \cd{9} are the integers 0 through 9,
and of \cdf{A} through \cdf{Z} (and also \cdf{a} through \cdf{z}) are 10 through 35.
\cdf{digit-char-p} returns the weight for one of these digits if and only if
its weight is strictly less than {\it radix}.  Thus, for example,
the digits for radix 16 are
\begin{lisp}
0  1  2  3  4  5  6  7  8  9  A  B  C  D  E  F
\end{lisp}

Here is an example of the use of \cdf{digit-char-p}:
\begin{lisp}
(defun convert-string-to-integer (str \&optional (radix 10)) \\
~~"Given a digit string and optional radix, return an integer." \\
~~(do ((j 0 (+ j 1)) \\
~~~~~~~(n 0 (+ (* n radix) \\
~~~~~~~~~~~~~~~(or (digit-char-p (char str j) radix) \\
~~~~~~~~~~~~~~~~~~~(error "Bad radix-{\Xtilde}D digit: {\Xtilde}C" \\
~~~~~~~~~~~~~~~~~~~~~~~~~~radix \\
~~~~~~~~~~~~~~~~~~~~~~~~~~(char str j)))))) \\
~~~~~~((= j (length str)) n)))
\end{lisp}
\end{defun}

\begin{defun}[Function]
alphanumericp char

The argument {\it char} must be a character object.
\cdf{alphanumericp} is true if {\it char} is either alphabetic
or numeric.  By definition,
\begin{lisp}
(alphanumericp x) \\
~~~\EQ\ (or (alpha-char-p x) (not (null (digit-char-p x))))
\end{lisp}
Alphanumeric characters are therefore necessarily graphic
(as defined by the predicate \cdf{graphic-char-p}).

Of the standard characters (as defined by \cdf{standard-char-p}),
the characters \cd{0} through \cd{9}, \cdf{A} through \cdf{Z},
and \cdf{a} through \cdf{z} are alphanumeric.
\end{defun}

\begin{defun}[Function]
char= character &rest more-characters \\
char/= character &rest more-characters \\
char< character &rest more-characters \\
char> character &rest more-characters \\
char<= character &rest more-characters \\
char>= character &rest more-characters

The arguments must all be character objects.
These functions compare the objects using the implementation-dependent
total ordering on characters, in a manner analogous to numeric
comparisons by \cdf{=} and related functions.

The total ordering on characters is guaranteed to have the following
properties:
\begin{itemize}
\item
The standard alphanumeric characters obey the following partial ordering:
\begin{lisp}
A<B<C<D<E<F<G<H<I<J<K<L<M<N<O\hbox{<P<Q<R<S<T<U<V<W<X<Y<Z} \\
\hbox to 0pt{a<b<c<d<e<f<g<h<i<j<k<l<m<n<o<\hss}~~~~~~~~~~~~~~~~~~~~~~~~~~~~~~p<q<r<s<t<u<v<w<x<y<z \\
0<1<2<3<4<5<6<7<8<9 \\
{\it either} 9<A {\it or} Z<0 \\
{\it either} 9<a {\it or} z<0
\end{lisp}
This implies that alphabetic ordering holds within each case (upper and
lower), and that the digits as a group
are not interleaved with letters.  However, the ordering
or possible interleaving of
uppercase letters and lowercase letters is unspecified.
(Note that both the ASCII and the EBCDIC character sets
conform to this specification.  As it happens, neither ordering
interleaves uppercase and lowercase letters:
in the ASCII ordering, \cd{9<A} and \cd{Z<a},
whereas in the EBCDIC ordering \cd{z<A} and \cd{Z<0}.)
\end{itemize}

\begin{obsolete}
\begin{itemize}
\item
If two characters have the same bits and font attributes,
then their ordering by \cdf{char<} is consistent with the numerical
ordering by the predicate \cdf{<} on their code attributes.

\item
If two characters differ in any attribute (code, bits, or font), then they
are different.
\end{itemize}
\end{obsolete}

\begin{newer}
X3J13 voted in March 1989 \issue{CHARACTER-PROPOSAL}
to replace the notion of bits and font attributes with
that of implementation-defined attributes.

\begin{itemize}
\item
If two characters have identical implementation-defined attributes,
then their ordering by \cdf{char<} is consistent with the numerical
ordering by the predicate \cdf{<} on their codes, and similarly
for \cdf{char>}, \cdf{char<=}, and \cdf{char>=}.

\item
If two characters differ in any implementation-defined
attribute, then they are not \cdf{char=}.
\end{itemize}
\end{newer}

The total ordering is not necessarily the same as the total
ordering on the integers produced by applying \cdf{char-int} to the
characters (although it is a reasonable implementation technique to
use that ordering).

While alphabetic characters of a given case must be
properly ordered, they need not be contiguous; thus \cd{(char<= \#{\Xbackslash}a x
\#{\Xbackslash}z)} is {\it not} a valid way of determining whether or not \cdf{x} is a
lowercase letter.  That is why a separate
\cdf{lower-case-p} predicate is provided.

\begin{lisp}
(char= \#{\Xbackslash}d \#{\Xbackslash}d) {\rm is true.} \\
(char/= \#{\Xbackslash}d \#{\Xbackslash}d) {\rm is false.} \\
(char= \#{\Xbackslash}d \#{\Xbackslash}x) {\rm is false.} \\
(char/= \#{\Xbackslash}d \#{\Xbackslash}x) {\rm is true.} \\
(char= \#{\Xbackslash}d \#{\Xbackslash}D) {\rm is false.} \\
(char/= \#{\Xbackslash}d \#{\Xbackslash}D) {\rm is true.} \\
(char= \#{\Xbackslash}d \#{\Xbackslash}d \#{\Xbackslash}d \#{\Xbackslash}d) {\rm is true.} \\
(char/= \#{\Xbackslash}d \#{\Xbackslash}d \#{\Xbackslash}d \#{\Xbackslash}d) {\rm is false.} \\
(char= \#{\Xbackslash}d \#{\Xbackslash}d \#{\Xbackslash}x \#{\Xbackslash}d) {\rm is false.} \\
(char/= \#{\Xbackslash}d \#{\Xbackslash}d \#{\Xbackslash}x \#{\Xbackslash}d) {\rm is false.} \\
(char= \#{\Xbackslash}d \#{\Xbackslash}y \#{\Xbackslash}x \#{\Xbackslash}c) {\rm is false.} \\
(char/= \#{\Xbackslash}d \#{\Xbackslash}y \#{\Xbackslash}x \#{\Xbackslash}c) {\rm is true.} \\
(char= \#{\Xbackslash}d \#{\Xbackslash}c \#{\Xbackslash}d) {\rm is false.} \\
(char/= \#{\Xbackslash}d \#{\Xbackslash}c \#{\Xbackslash}d) {\rm is false.} \\
(char< \#{\Xbackslash}d \#{\Xbackslash}x) {\rm is true.} \\
(char<= \#{\Xbackslash}d \#{\Xbackslash}x) {\rm is true.} \\
(char< \#{\Xbackslash}d \#{\Xbackslash}d) {\rm is false.} \\
(char<= \#{\Xbackslash}d \#{\Xbackslash}d) {\rm is true.} \\
(char< \#{\Xbackslash}a \#{\Xbackslash}e \#{\Xbackslash}y \#{\Xbackslash}z) {\rm is true.} \\
(char<= \#{\Xbackslash}a \#{\Xbackslash}e \#{\Xbackslash}y \#{\Xbackslash}z) {\rm is true.} \\
(char< \#{\Xbackslash}a \#{\Xbackslash}e \#{\Xbackslash}e \#{\Xbackslash}y) {\rm is false.} \\
(char<= \#{\Xbackslash}a \#{\Xbackslash}e \#{\Xbackslash}e \#{\Xbackslash}y) {\rm is true.} \\
(char> \#{\Xbackslash}e \#{\Xbackslash}d) {\rm is true.} \\
(char>= \#{\Xbackslash}e \#{\Xbackslash}d) {\rm is true.} \\
(char> \#{\Xbackslash}d \#{\Xbackslash}c \#{\Xbackslash}b \#{\Xbackslash}a) {\rm is true.} \\
(char>= \#{\Xbackslash}d \#{\Xbackslash}c \#{\Xbackslash}b \#{\Xbackslash}a) {\rm is true.} \\
(char> \#{\Xbackslash}d \#{\Xbackslash}d \#{\Xbackslash}c \#{\Xbackslash}a) {\rm is false.} \\
(char>= \#{\Xbackslash}d \#{\Xbackslash}d \#{\Xbackslash}c \#{\Xbackslash}a) {\rm is true.} \\
(char> \#{\Xbackslash}e \#{\Xbackslash}d \#{\Xbackslash}b \#{\Xbackslash}c \#{\Xbackslash}a) {\rm is false.} \\
(char>= \#{\Xbackslash}e \#{\Xbackslash}d \#{\Xbackslash}b \#{\Xbackslash}c \#{\Xbackslash}a) {\rm is false.} \\
(char> \#{\Xbackslash}z \#{\Xbackslash}A) {\rm may be true or false.} \\
(char> \#{\Xbackslash}Z \#{\Xbackslash}a) {\rm may be true or false.}
\end{lisp}

There is no requirement that \cd{(eq c1 c2)} be true merely because
\cd{(char= c1 c2)} is true.  While \cdf{eq} may distinguish two character
objects that \cdf{char=} does not, it is distinguishing them not
as {\it characters}, but in some sense on the basis of a lower-level
implementation characteristic.
(Of course, if \cd{(eq c1 c2)} is true,
then one may expect \cd{(char= c1 c2)} to be true.)
However, \cdf{eql} and \cdf{equal}
compare character objects in the same
way that \cdf{char=} does.
\end{defun}

\begin{defun}[Function]
char-equal character &rest more-characters \\
char-not-equal character &rest more-characters \\
char-lessp character &rest more-characters \\
char-greaterp character &rest more-characters \\
char-not-greaterp character &rest more-characters \\
char-not-lessp character &rest more-characters

\begin{obsolete}
The predicate \cdf{char-equal} is like \cdf{char=}, and similarly
for the others, except according to a different ordering such that
differences of bits attributes and case are ignored,
and font information is taken into account in an implementation-dependent
manner.
\end{obsolete}

\begin{newer}
X3J13 voted in March 1989 \issue{CHARACTER-PROPOSAL}
to replace the notion of bits and font attributes with
that of implementation-defined attributes.  The effect, if any,
of each such attribute on the behavior of
\cdf{char-equal}, \cdf{char-not-equal}, \cdf{char-lessp}, \cdf{char-greaterp},
\cdf{char-not-greaterp}, and \cdf{char-not-lessp} must be specified
as part of the definition of that attribute.
\end{newer}


For the standard characters, the ordering is such that
\cd{A=a}, \cd{B=b}, and so on, up to \cd{Z=z}, and furthermore either
\cd{9<A} or \cd{Z<0}.
For example:
\begin{lisp}
(char-equal \#{\Xbackslash}A \#{\Xbackslash}a) {\rm is true.} \\
(char= \#{\Xbackslash}A \#{\Xbackslash}a) {\rm is false.} \\
(char-equal \#{\Xbackslash}A \#{\Xbackslash}Control-A) {\rm is true.}
\end{lisp}
\begin{obsolete}
The ordering may depend on the font information. For example, an implementation
might decree that \cd{(char-equal \#{\Xbackslash}p \#{\Xbackslash}{\it p})} be true, but that
\cd{(char-equal \#{\Xbackslash}p \#{\Xbackslash}}$\pi$\cd{)} be false
(where \cd{\#{\Xbackslash}}$\pi$ is a
lowercase \cdf{p} in some font).  Assuming italics to be in font 1
and the Greek alphabet in font 2, this is the same as saying that
\cd{(char-equal \#0{\Xbackslash}p \#1{\Xbackslash}p)} may be true and at the same time
\cd{(char-equal \#0{\Xbackslash}p \#2{\Xbackslash}p)} may be false.
\end{obsolete}
\end{defun}

\section{Character Construction and Selection}

These functions may be used to extract attributes of a character
and to construct new characters.

\begin{defun}[Function]
char-code char

The argument {\it char} must be a character object.
\cdf{char-code} returns the code attribute of the character object;
this will be a non-negative integer less than the (normal) value of
the variable \cdf{char-code-limit}.

\begin{new}
This is usually what you need in order to treat a character as an
index into a vector.  The length of the vector should then be
equal to \cdf{char-code-limit}.  Be careful how you initialize this
vector; remember that you cannot necessarily
expect all non-negative integers less than
\cdf{char-code-limit} to be valid character codes.
\end{new}
\end{defun}


\begin{obsolete}
\begin{defun}[Function]
char-bits char

The argument {\it char} must be a character object.
\cdf{char-bits} returns the bits attribute of the character object;
this will be a non-negative integer less than the (normal) value of
the variable \cdf{char-bits-limit}.
\end{defun}
\end{obsolete}

\begin{newer}
X3J13 voted in March 1989 \issue{CHARACTER-PROPOSAL}
to eliminate \cdf{char-bits}.
\end{newer}


\begin{obsolete}
\begin{defun}[Function]
char-font char

The argument {\it char} must be a character object.
\cdf{char-font} returns the font attribute of the character object;
this will be a non-negative integer less than the (normal) value of
the variable \cdf{char-font-limit}.
\end{defun}
\end{obsolete}

\begin{newer}
X3J13 voted in March 1989 \issue{CHARACTER-PROPOSAL}
to eliminate \cdf{char-font}.
\end{newer}
\medskip
\begin{new}
The references to the ``normal'' values of the ``variables''
\cdf{char-code-limit},
\cdf{char-bits-limit}, and \cdf{char-font-limit} in the descriptions
of \cdf{char-code}, \cdf{char-bits}, and \cdf{char-font} were an oversight on
my part.  Early in the design of Common Lisp they were indeed variables,
but they are at present defined to be constants, and their values therefore
are always normal and should not change.  But this point is now moot.
\end{new}

\begin{defun}[Function]
code-char code &optional (bits 0) (font 0)

\begin{obsolete}
All three arguments must be non-negative integers.
If it is possible in the implementation to construct a character
object whose code attribute is {\it code}, whose bits attribute is {\it bits},
and whose font attribute is {\it font}, then such an object is returned;
otherwise {\false} is returned.

For any integers {\it c}, {\it b}, and {\it f}, if \cd{(code-char {\it c} {\it b} {\it f})}
is not {\false} then
\begin{lisp}
(char-code (code-char {\it c} {\it b} {\it f})) \EV\ {\it c} \\
(char-bits (code-char {\it c} {\it b} {\it f})) \EV\ {\it b} \\
(char-font (code-char {\it c} {\it b} {\it f})) \EV\ {\it f}
\end{lisp}
If the font and bits attributes of a character object \cdf{c} are zero,
then it is the case that
\begin{lisp}
(char= (code-char (char-code c)) c)
\end{lisp}
is true.
\end{obsolete}
\begin{newer}
X3J13 voted in March 1989 \issue{CHARACTER-PROPOSAL}
to eliminate the {\it bits} and {\it font} arguments from
the specification of \cdf{code-char}.
\end{newer}
\end{defun}


\begin{obsolete}
\begin{defun}[Function]
make-char char &optional (bits 0) (font 0)

The argument {\it char} must be a character,
and {\it bits} and {\it font} must be non-negative integers.
If it is possible in the implementation to construct a character
object whose code attribute is the same as
the code attribute of {\it char},
whose bits attribute is {\it bits},
and whose font attribute is {\it font}, then such an object is returned;
otherwise {\false} is returned.

If {\it bits} and {\it font} are zero, then \cdf{make-char} cannot fail.
This implies that for every character object one can ``turn off''
its bits and font attributes.
\end{defun}
\end{obsolete}

\begin{newer}
X3J13 voted in March 1989 \issue{CHARACTER-PROPOSAL}
to eliminate \cdf{make-char}.
\end{newer}

\section{Character Conversions}


These functions perform various transformations on characters,
including case conversions.

\begin{defun}[Function]
character object

The function \cdf{character} coerces its argument to be a character
if possible; see \cdf{coerce}.
\begin{lisp}
(character x) \EQ\ (coerce x 'character)
\end{lisp}
\end{defun}


\begin{defun}[Function]
char-upcase char \\
char-downcase char

The argument {\it char} must be a character object.
\cdf{char-upcase} attempts to convert its argument to an uppercase
equivalent; \cdf{char-downcase} attempts to convert its argument
to a lowercase equivalent.

\begin{obsolete}
\cdf{char-upcase} returns a character object with the same font
and bits attributes as {\it char}, but with possibly a different code
attribute.  If the code is different from {\it char}'s, then the predicate
\cdf{lower-case-p} is true of {\it char}, and \cdf{upper-case-p}
is true of the result character.  Moreover, if \cd{(char= (char-upcase x) x)}
is {\it not} true, then it is true that
\begin{lisp}
(char= (char-downcase (char-upcase x)) x)
\end{lisp}
Similarly,
\cdf{char-downcase} returns a character object with the same font
and bits attributes as {\it char}, but with possibly a different code
attribute.  If the code is different from {\it char}'s, then the predicate
\cdf{upper-case-p} is true of {\it char}, and \cdf{lower-case-p}
is true of the result character.  Moreover, if \cd{(char= (char-downcase x) x)}
is {\it not} true, then it is true that
\begin{lisp}
(char= (char-upcase (char-downcase x)) x)
\end{lisp}
Note that the action of \cdf{char-upcase} and \cdf{char-downcase} may
depend on the bits and font attributes of the character.  In particular,
they have no effect on a character with a non-zero bits attribute,
because such characters are by definition not alphabetic.
See \cdf{alpha-char-p}.
\end{obsolete}

\begin{newer}
X3J13 voted in March 1989 \issue{CHARACTER-PROPOSAL}
to replace the notion of bits and font attributes with
that of implementation-defined attributes.  The effect of
\cdf{char-upcase} and \cdf{char-downcase} is to preserve
implementation-defined attributes.
\end{newer}

\end{defun}

\begin{defun}[Function]
digit-char weight &optional (radix 10) (font 0)

All arguments must be integers.  \cdf{digit-char}
determines whether or not it is possible to construct
a character object
whose font attribute is {\it font}, and whose {\it code} is such that the
result character has the weight {\it weight} when considered as
a digit of the radix {\it radix} (see the predicate \cdf{digit-char-p}).
It returns such a character if that is possible, and otherwise returns {\false}.

\cdf{digit-char} cannot return {\false} if {\it font} is zero,
{\it radix} is between 2 and 36 inclusive, and {\it weight} is non-negative
and less than {\it radix}.

If more than one character object can encode
such a weight in the given radix, one will be chosen consistently
by any given implementation; moreover, among the standard characters,
uppercase letters are preferred to lowercase letters.
For example:
\begin{lisp}
(digit-char 7) \EV\ \#{\Xbackslash}7 \\
(digit-char 12) \EV\ {\false} \\
(digit-char 12 16) \EV\ \#{\Xbackslash}C~~~~~;{\rm not} \#{\Xbackslash}c \\
(digit-char 6 2) \EV\ {\false} \\
(digit-char 1 2) \EV\ \#{\Xbackslash}1
\end{lisp}
Note that no argument is provided for specifying the {\it bits} component
of the returned character, because a digit cannot have a non-zero
{\it bits} component.  The reasoning is that every digit is graphic
(see \cdf{digit-char-p}) and no graphic character has a non-zero
{\it bits} component (see \cdf{graphic-char-p}).

\begin{newer}
X3J13 voted in March 1989 \issue{CHARACTER-PROPOSAL}
to eliminate the {\it font} argument from the specification of \cdf{digit-char}.
\end{newer}
\end{defun}

\begin{defun}[Function]
char-int char

The argument {\it char} must be a character object.
\cdf{char-int} returns a non-negative integer encoding the character object.

If the font and bits attributes of {\it char} are zero, then
\cdf{char-int} returns the same integer \cdf{char-code} would.
Also,
\begin{lisp}
(char= c1 c2) \EQ\ (= (char-int c1) (char-int c2))
\end{lisp}
for characters \cd{c1} and \cd{c2}.

This function is provided primarily for the purpose of hashing characters.
\end{defun}


\begin{obsolete}
\begin{defun}[Function]
int-char integer

The argument must be a non-negative integer.
\cdf{int-char} returns a character object \cdf{c} such that
\cd{(char-int c)} is equal to {\it integer}, if possible; otherwise
\cdf{int-char} returns false.
\end{defun}
\end{obsolete}

\begin{newer}
X3J13 voted in March 1989 \issue{CHARACTER-PROPOSAL}
to eliminate \cdf{int-char}.
\end{newer}

\begin{defun}[Function]
char-name char

The argument {\it char} must be a character object.
If the character has a name, then that name (a string) is returned;
otherwise {\false} is returned.  All characters that have
zero font and bits attributes and that are non-graphic
(do not satisfy the predicate \cdf{graphic-char-p}) have names.
Graphic characters may or may not have names.

The standard newline and space characters have the respective
names \cdf{Newline} and \cdf{Space}.
The semi-standard characters have the names
\cdf{Tab}, \cdf{Page}, \cdf{Rubout}, \cdf{Linefeed}, \cdf{Return}, and \cdf{Backspace}.


Characters that have names can be notated as \cd{\#{\Xbackslash}} followed
by the name.  (See section~\ref{SHARP-SIGN-MACRO-CHARACTER-SECTION}.)
Although the name may be written in any case,
it is stylish to capitalize it thus: \cd{\#{\Xbackslash}Space}.

\cdf{char-name} will only locate ``simple'' character names;
it will not construct names such as \cdf{Control-Space} on the
basis of the character's bits attribute.

\begin{new}
The easiest way to get a name that includes the bits attribute of
a character \cdf{c} is \cd{(format nil "{\Xtilde}:C" c)}.
\end{new}
\end{defun}

\begin{defun}[Function]
name-char name

The argument \cdf{name} must be an object coerceable to a string
as if by the function \cdf{string}.
If the name is the same as the name of a character object
(as determined by \cdf{string-equal}), that object
is returned; otherwise {\false} is returned.
\end{defun}

\begin{obsolete}

\section{Character Control-Bit Functions}

Common Lisp provides explicit names for four bits of the bits attribute:
{\it Control}, {\it Meta}, {\it Hyper}, and {\it Super}.  The following
definitions are provided for manipulating these.
Each Common Lisp implementation provides these functions for compatibility,
even if it does not support any or all of the bits named below.
\end{obsolete}


\begin{obsolete}
\begin{defun}[Constant]
char-control-bit \\
char-meta-bit \\
char-super-bit \\
char-hyper-bit

The values of these named constants are the ``weights'' (as integers) for
the four named control bits.  The weight of the control bit is \cd{1};
of the meta bit, \cd{2}; of the super bit, \cd{4}; and of the hyper bit, \cd{8}.

If a given implementation of Common Lisp does not support a particular bit,
then the corresponding constant is zero instead.
\end{defun}
\end{obsolete}

\begin{newer}
X3J13 voted in March 1989 \issue{CHARACTER-PROPOSAL}
to eliminate all four of the constants
\cdf{char-control-bit}, \cdf{char-meta-bit}, \cdf{char-super-bit},
and \cdf{char-hyper-bit}.
\end{newer}
\medskip
\begin{new}
When Common Lisp was first designed, keyboards with ``extra bits'' were
relatively rare.  The bits attribute was originally designed to support input
from keyboards in use at Stanford and M.I.T.~circa 1981.

Since that time such extended keyboards have come into wider use.
Notable here are the keyboards associated with certain
personal computers and workstations.  For example, in some specific applications
the {\it command} and {\it option} keys of Apple Macintosh keyboards have
had the connotations of {\it control} and {\it meta}.  Macintosh II
extended keyboards also have keys marked {\it control} whose use
is analogous to that of {\it hyper} on the old M.I.T.~keyboards.
IBM PC personal computer keyboards have {\it alt} keys that function
much like {\it meta} keys; similarly, keyboards on Sun workstations
have keys very much like {\it meta} keys but labelled {\it left} and {\it right}.
\end{new}

\begin{obsolete}
\begin{defun}[Function]
char-bit char name

\cdf{char-bit} takes a character object {\it char} and the name of a bit,
and returns non-{\false} if the bit of that name is set in {\it char},
or {\false} if the bit is not set in {\it char}.
For example:
\begin{lisp}
(char-bit \#{\Xbackslash}Control-X \cd{:control}) \EV\ {\it true}
\end{lisp}
Valid values for {\it name}
are implementation-dependent, but typically are \cd{:control},
\cd{:meta}, \cd{:hyper}, and \cd{:super}.
It is an error to give \cdf{char-bit} the name of a bit not supported
by the implementation.

If the argument {\it char} is specified by a form that is a {\it place} form
acceptable to \cdf{setf},
then \cdf{setf} may be used with \cdf{char-bit}
to modify a bit of the character stored in that
{\it place}.
The effect is to perform a \cdf{set-char-bit} operation
and then store the result back into the {\it place}.
\end{defun}
\end{obsolete}

\begin{newer}
X3J13 voted in March 1989 \issue{CHARACTER-PROPOSAL}
to eliminate \cdf{char-bit}.
\end{newer}


\begin{obsolete}
\begin{defun}[Function]
set-char-bit char name newvalue

\cdf{char-bit} takes a character object {\it char}, the name of a bit,
and a flag.  A character is returned that is just like {\it char}
except that the named bit is set or reset according to whether
{\it newvalue} is non-{\false} or {\false}.
Valid values for {\it name}
are implementation-dependent, but typically are \cd{:control},
\cd{:meta}, \cd{:hyper}, and \cd{:super}.
For example:
\begin{lisp}
(set-char-bit \#{\Xbackslash}X \cd{:control} t) \EV\ \#{\Xbackslash}Control-X \\
(set-char-bit \#{\Xbackslash}Control-X \cd{:control} t) \EV\ \#{\Xbackslash}Control-X \\
(set-char-bit \#{\Xbackslash}Control-X \cd{:control} {\false}) \EV\ \#{\Xbackslash}X
\end{lisp}
\end{defun}
\end{obsolete}

\begin{newer}
X3J13 voted in March 1989 \issue{CHARACTER-PROPOSAL}
to eliminate \cdf{set-char-bit}.
\end{newer}
        % Functions on characters
%Part{KSeque, Root = "CLM.MSS"}
%%% Chapter of Common Lisp Manual.  Copyright 1984, 1988, 1989 Guy L. Steele Jr.

\clearpage\def\pagestatus{FINAL PROOF}

\ifx \rulang\Undef

\chapter{Sequences}
\label{KSEQUE}

The type \cdf{sequence} encompasses both lists and vectors (one-dimensional
arrays).
While these are different data structures with different structural
properties leading to different algorithmic uses, they do have a common
property: each contains an ordered set of elements.
Note that {\nil} is considered to be a sequence of length zero.

Some operations are useful on both lists and arrays
because they deal with ordered sets of elements.  One may ask the number
of elements, reverse the ordering, extract a subsequence, and so on.  For
such purposes Common Lisp provides a set of generic functions on sequences.

Note that this remark, predating the design of the Common Lisp Object System,
uses the term ``generic'' in a generic sense, and not necessarily
in the technical sense used by CLOS
(see chapter~\ref{DTYPES}).

\begin{flushleft}
\cf
\begin{tabular*}{\textwidth}{@{}l@{\extracolsep{\fill}}lll@{}}
elt&reverse&map&remove \\
length&nreverse&some&remove-duplicates \\
subseq&concatenate&every&delete \\
copy-seq&position&notany&delete-duplicates \\
fill&find&notevery&substitute \\
replace&sort&reduce&nsubstitute \\
count&merge&search&mismatch
\end{tabular*}
\end{flushleft}
Some of these operations come in more than one version.
Such versions are indicated by adding a suffix (or occasionally a prefix)
to the basic name of the operation.
In addition, many operations accept one or more optional keyword
arguments that can modify the operation in various ways.

If the operation requires testing sequence elements according to
some criterion, then the criterion may be specified in one of two ways.
The basic operation accepts an item,
and elements are tested for being \cdf{eql} to that item.
(A test other than \cdf{eql} can be specified by the \cd{:test}
or \cd{:test-not} keyword.  It is an error to use both
of these keywords in the same call.)
The variants formed by adding \cdf{-if} and \cdf{-if-not}
to the basic operation name do not take an item,
but instead a one-argument predicate,
and elements are tested for satisfying or not satisfying the predicate.
As an example,
\begin{lisp}
(remove \emph{item} \emph{sequence})
\end{lisp}
returns a copy of \emph{sequence} from which all elements \cdf{eql} to \emph{item}
have been removed;
\begin{lisp}
(remove \emph{item} \emph{sequence} \cd{:test} \#'equal)
\end{lisp}
returns a copy of \emph{sequence} from which all elements \cdf{equal} to \emph{item}
have been removed;
\begin{lisp}
(remove-if \#'numberp \emph{sequence})
\end{lisp}
returns a copy of \emph{sequence} from which all numbers have been removed.

If an operation tests elements of a sequence in any manner,
the keyword argument \cd{:key}, if not {\false}, should be a function
of one argument that will extract from an element the part to be tested
in place of the whole element.
For example, the effect of the MacLisp expression
\cd{(assq item seq)} could be obtained by
\begin{lisp}
(find \emph{item} \emph{sequence} \cd{:test} \#'eq \cd{:key} \#'car)
\end{lisp}
This searches for the first element of \emph{sequence} whose \emph{car} is \cdf{eq}
to \emph{item}.
\begin{newer}
X3J13 voted in June 1988 \issue{FUNCTION-TYPE} to allow the \cd{:key} function
to be only of type \cdf{symbol} or \cdf{function}; a lambda-expression
is no longer acceptable as a functional argument.  One must use the
\cdf{function} special operator or the abbreviation \cd{\#'} before
a lambda-expression that appears as an  explicit argument form.
\end{newer}

For some operations it can be useful to specify the direction
in which the sequence is conceptually processed.  In this case the basic
operation normally processes the sequence in the forward direction,
and processing in the reverse direction is indicated by a non-{\false}
value for the keyword argument \cd{:from-end}.  (The processing order
specified by the \cd{:from-end} is purely conceptual.  Depending on
the object to be processed and on the implementation, the actual processing
order may be different.  For this reason a user-supplied \emph{test} function
should be free of side effects.)

Many operations allow the specification of a subsequence to be operated
upon.  Such operations have keyword arguments
called \cd{:start} and \cd{:end}.  These arguments should be integer indices
into the sequence, with $\emph{start}\leq\emph{end}$
(it is an error if $\emph{start}>\emph{end}$).  They indicate
the subsequence starting with and \emph{including} element \emph{start}
and up to but \emph{excluding} element \emph{end}.  The length of the subsequence
is therefore $\emph{end}-\emph{start}$.  If \emph{start} is omitted,
it defaults to zero; and if \emph{end} is omitted or {\false}, it defaults to
the length of the sequence.
Therefore if both \emph{start} and \emph{end} are omitted, the entire sequence
is processed by default.
For the most part, subsequence specification
is permitted purely for the sake of efficiency;
one could simply call \cdf{subseq} instead to extract the subsequence
before operating on it.  Note, however, that operations that
calculate indices
return indices into the original sequence, not into the subsequence:
\begin{lisp}
(position \#{\Xbackslash}b "foobar" \cd{:start} 2 \cd{:end} 5) \EV\ 3 \\
(position \#{\Xbackslash}b (subseq "foobar" 2 5)) \EV\ 1
\end{lisp}
If two sequences are involved, then
the keyword arguments
\cd{:start1}, \cd{:end1}, \cd{:start2}, and \cd{:end2} are used to
specify separate subsequences for each sequence.

\begin{newer}
X3J13 voted in June 1988 \issue{SUBSEQ-OUT-OF-BOUNDS}
(and further clarification was voted in January 1989
\issue{RANGE-OF-START-AND-END-PARAMETERS})
to specify that these rules apply not
only to all built-in functions that have keyword parameters named
\cd{:start}, \cd{:start1}, \cd{:start2}, \cd{:end}, \cd{:end1},
or \cd{:end2} but also to functions such as \cdf{subseq}
that take required or optional parameters that are documented
as being named \emph{start} or \emph{end}.
\begin{itemize}
\item A ``start'' argument must always be a non-negative integer and
defaults to zero if not supplied; it is not permissible to pass \cdf{nil}
as a ``start'' argument.
\item An ``end'' argument must be either a
non-negative integer or \cdf{nil} (which indicates the end of the
sequence) and defaults to \cdf{nil}
if not supplied; therefore supplying \cdf{nil} is equivalent to
not supplying such an argument.
\item If the ``end'' argument is an integer, it must be no greater than the
active length of the corresponding sequence
(as returned by the function \cdf{length}).
\item The default value for the ``end'' argument is the active length
of the corresponding sequence.
\item The ``start'' value (after defaulting, if necessary) must not be greater than the
corresponding ``end'' value (after defaulting, if necessary).
\end{itemize}
This may be summarized as follows.
Let \emph{x} be the sequence within which indices are to be considered.  Let \emph{s} be
the ``start'' argument for that sequence of any standard function,
whether explicitly specified or defaulted, through omission, to
zero.  Let \emph{e} be the ``end'' argument for that sequence
of any standard function, whether explicitly specified or defaulted, through
omission or an explicitly passed \cdf{nil} value, to the active length of \emph{x}, as
returned by \cdf{length}.  Then it is an error if the test
\cd{(<=~0~\emph{s}~\emph{e}~(length \emph{x}))}
is not true.
\end{newer}

For some functions, notably \cdf{remove} and \cdf{delete}, the keyword argument
\cd{:count} is used to specify how many occurrences of the item should
be affected.  If this is {\false} or is not supplied, all matching items are
affected.

In the following function descriptions, an element \emph{x} of a sequence
``satisfies the test'' if any of the following holds:
\begin{itemize}
\item
A basic function was called,
\emph{testfn} was specified by the keyword \cd{:test}, and
\cd{(funcall \emph{testfn} \emph{item} (\emph{keyfn} \emph{x}))} is true.

\item
A basic function was called,
\emph{testfn} was specified by the keyword \cd{:test-not}, and
\cd{(funcall \emph{testfn} \emph{item} (\emph{keyfn} \emph{x}))} is false.

\item
An \cdf{-if} function was called, and
\cd{(funcall \emph{predicate} (\emph{keyfn} \emph{x}))} is true.

\item
An \cdf{-if-not} function was called, and
\cd{(funcall \emph{predicate} (\emph{keyfn} \emph{x}))} is false.
\end{itemize}
In each case \emph{keyfn} is the
value of the \cd{:key} keyword argument (the default being the identity
function).  See, for example, \cdf{remove}.

In the following function descriptions,
two elements \emph{x} and \emph{y} taken from sequences ``match'' if
either of the following holds:
\begin{itemize}
\item
\emph{testfn} was specified by the keyword \cd{:test}, and
\cd{(funcall \emph{testfn} (\emph{keyfn} \emph{x}) (\emph{keyfn} \emph{y}))} is true.

\item
\emph{testfn} was specified by the keyword \cd{:test-not}, and
\cd{(funcall \emph{testfn} (\emph{keyfn} \emph{x}) (\emph{keyfn} \emph{y}))} is false.
\end{itemize}
See, for example, \cdf{search}.

\begin{newer}
X3J13 voted in June 1988 \issue{FUNCTION-TYPE} to allow the \emph{testfn}
or \cdf{predicate}
to be only of type \cdf{symbol} or \cdf{function}; a lambda-expression
is no longer acceptable as a functional argument.  One must use the
\cdf{function} special operator or the abbreviation \cd{\#'} before
a lambda-expression that appears as an  explicit argument form.
\end{newer}

You may depend on the order in which arguments
are given to \emph{testfn}; this permits the use of non-commutative
test functions in a predictable manner.
The order of the arguments to \emph{testfn} corresponds
to the order in which those arguments (or the sequences containing
those arguments)
were given to the sequence function in question.
If a sequence function gives two elements from the same
sequence argument to \emph{testfn}, they are given in the same order in
which they appear in the sequence.

Whenever a sequence function must construct and return
a new vector, it always returns a \emph{simple}
vector (see section~\ref{ARRAY-TYPE-SECTION}).
Similarly, any strings constructed will be simple strings.

\begin{defun}[Function]
complement fn

Returns a function whose value is the same as that of \cdf{not}
applied to the result of applying the function \emph{fn} to the same
arguments.  One could define \cdf{complement} as follows:
\begin{lisp}
(defun complement (fn) \\*
~~\#'(lambda (\&rest arguments) \\*
~~~~~~(not (apply fn arguments))))
\end{lisp}

One intended use of \cdf{complement} is to supplant the use of
\cd{:test-not} arguments and \cdf{-if-not} functions.
\begin{lisp}
(remove-if-not \#'virtuous senators) {\EQ} \\*
~~~(remove-if (complement \#'virtuous) senators) \\
\\
(remove-duplicates telephone-book \\*
~~~~~~~~~~~~~~~~~~~:test-not \#'mismatch) {\EQ} \\*
~~~(remove-duplicates telephone-book \\*
~~~~~~~~~~~~~~~~~~~~~~:test (complement \#'mismatch))
\end{lisp}
\end{defun}

\section{Simple Sequence Functions}

Most of the following functions perform simple operations on a single
sequence; \cdf{make-sequence} constructs a new sequence.

\begin{defun}[Function]
elt sequence index

This returns the element of \emph{sequence} specified by \emph{index},
which must be a non-negative integer less than the length of the \emph{sequence}
as returned by \cdf{length}.
The first element of a sequence has index \cd{0}.

(Note that \cdf{elt} observes the fill pointer in those vectors that have
fill pointers.  The array-specific function \cdf{aref} may be used
to access vector elements that are beyond the vector's fill pointer.)

\cdf{setf} may be used with \cdf{elt} to destructively replace
a sequence element with a new value.
\end{defun}

\begin{defun}[Function]
subseq sequence start &optional end

This returns the subsequence of \emph{sequence} specified by \emph{start} and
\emph{end}.
\cdf{subseq} \emph{always} allocates a new sequence for a result; it never
shares storage with an old sequence.  The result subsequence is always of
the same type as the argument \emph{sequence}.

\cdf{setf} may be used with \cdf{subseq} to destructively replace
a subsequence with a sequence of new values; see also \cdf{replace}.
\end{defun}

\begin{defun}[Function]
copy-seq sequence

A copy is made of the argument \emph{sequence}; the result is \cdf{equalp}
to the argument but not \cdf{eq} to it.
\begin{lisp}
(copy-seq \emph{x}) \EQ\ (subseq \emph{x} 0)
\end{lisp}
but the name \cdf{copy-seq} is more perspicuous when applicable.
\end{defun}

\begin{defun}[Function]
length sequence

The number of elements in \emph{sequence} is returned as a non-negative integer.
If the sequence is a vector with a fill pointer,
the ``active length'' as specified by the fill pointer is returned
(see section~\ref{FILL-POINTER}).
\end{defun}

\begin{defun}[Function]
reverse sequence

The result is a new sequence of the same kind as \emph{sequence},
containing the same elements but in reverse order.
The argument is not modified.
\end{defun}

\begin{defun}[Function]
nreverse sequence

The result is a sequence containing the same elements as \emph{sequence}
but in reverse order.  The argument may be destroyed and re-used to
produce the result.  The result may or may not be \cdf{eq} to the
argument, so it is usually wise to say something like
\cd{(setq x (nreverse x))}, because simply \cd{(nreverse x)} is not
guaranteed to leave a reversed value in \cdf{x}.
\begin{newer}
X3J13 voted in March 1989 \issue{REMF-DESTRUCTION-UNSPECIFIED}
to clarify the permissible side effects of certain operations.
When the \emph{sequence} is a list,
\cdf{nreverse} is permitted to perform a \cdf{setf} on any part,
\emph{car} or \emph{cdr}, of the top-level list structure of that list.
When the \emph{sequence} is an array,
\cdf{nreverse} is permitted to re-order the elements of the given array
in order to produce the resulting array.
\end{newer}
\end{defun}

\begin{defun}[Function]
make-sequence type size &key :initial-element

This returns a sequence of type \emph{type} and of length \emph{size}, each of
whose elements
has been initialized to the \cd{:initial-element} argument.
If specified, the \cd{:initial-element} argument must be an object that
can be an element of a sequence of type \emph{type}.
For example:
\begin{lisp}
(make-sequence '(vector double-float) \\*
~~~~~~~~~~~~~~~100 \\*
~~~~~~~~~~~~~~~:initial-element 1d0)
\end{lisp}
If an \cd{:initial-element} argument is not specified, then the sequence will
be initialized in an implementation-dependent way.

\begin{new}
X3J13 voted in January 1989
\issue{ARGUMENTS-UNDERSPECIFIED}
to clarify that the \emph{type} argument
must be a type specifier, and the \emph{size} argument
must be a non-negative integer less than the value of
\cdf{array-dimension-limit}.
\end{new}

\begin{newer}
X3J13 voted in June 1989 \issue{SEQUENCE-TYPE-LENGTH} to specify that
\cdf{make-sequence} should signal an error if the sequence \emph{type} specifies the
number of elements and the \emph{size} argument is different.
\end{newer}

\begin{newer}
X3J13 voted in March 1989 \issue{CHARACTER-PROPOSAL}
to specify that if \emph{type} is \cdf{string}, the result is the same
as if \cdf{make-string} had been called with the same \emph{size}
and \cd{:initial-element} arguments.
\end{newer}
\end{defun}

\section{Concatenating, Mapping, and Reducing Sequences}

The functions in this section each operate on an arbitrary number of
sequences except for \cdf{reduce}, which is included here because
of its conceptual relationship to the mapping functions.

\begin{defun}[Function]
concatenate result-type &rest sequences

The result is a new sequence that contains all the elements of all the
sequences in order.  All of the sequences are copied from; the result
does not share any structure with any of the argument sequences (in this
\cdf{concatenate} differs from \cdf{append}).  The type of the result is
specified by \emph{result-type}, which must be a subtype of \cdf{sequence},
as for the function \cdf{coerce}.
It must be possible for every element of the argument sequences to be an
element of a sequence of type \emph{result-type}.

If only one \emph{sequence} argument is provided
and it has the type specified by \emph{result-type},
\cdf{concatenate} is required to copy the argument rather than simply
returning it.  If a copy is not required, but only possibly type conversion,
then the \cdf{coerce} function may be appropriate.

\begin{newer}
X3J13 voted in June 1989 \issue{SEQUENCE-TYPE-LENGTH} to specify that
\cdf{concatenate} should signal an error if the sequence type specifies the
number of elements and the sum of the argument lengths is different.
\end{newer}
\end{defun}

\begin{defun}[Function]
map result-type function sequence &rest more-sequences

The \emph{function} must take as many arguments as there are sequences
provided; at least one sequence must be provided.
The result of \cdf{map} is a sequence such that element \emph{j} is the result
of applying \emph{function} to element \emph{j} of each of the argument
sequences.  The result sequence is as long as the shortest of the
input sequences.

If the \emph{function} has side effects, it can count on being called
first on all the elements numbered \cd{0}, then on all those
numbered \cd{1}, and so on.

The type of the result sequence is specified by the argument \emph{result-type}
(which must be a subtype of the type \cdf{sequence}),
as for the function \cdf{coerce}.
In addition, one may specify {\nil} for the result type, meaning that no
result sequence is to be produced; in this case the \emph{function} is invoked
only for effect, and \cdf{map} returns {\nil}.  This gives an effect similar
to that of \cdf{mapc}.

\begin{newer}
X3J13 voted in June 1989 \issue{SEQUENCE-TYPE-LENGTH} to specify that
\cdf{map} should signal an error if the sequence type specifies the number of
elements and the minimum of the argument lengths is different.
\end{newer}

\begin{new}
X3J13 voted in January 1989
\issue{MAPPING-DESTRUCTIVE-INTERACTION}
to restrict user side effects; see section~\ref{STRUCTURE-TRAVERSAL-SECTION}.
\end{new}

\noindent
For example:
\begin{lisp}
(map 'list \#'- '(1 2 3 4)) \EV\ (-1 -2 -3 -4) \\
(map 'string \\
~~~~~\#'(lambda (x) (if (oddp x) \#{\Xbackslash}1 \#{\Xbackslash}0)) \\
~~~~~'(1 2 3 4)) \\
~~~\EV\ "1010"
\end{lisp}
\end{defun}


\begin{defun}[Function]
map-into result-sequence function &rest sequences

Function \cdf{map-into} destructively modifies the \emph{result-sequence} to
contain the results of applying \emph{function} to corresponding elements of
the argument \emph{sequences} in turn; it then
returns \emph{result-sequence}.

The arguments \emph{result-sequence}
and each element of \emph{sequences} can each be
either a list or a vector (one-dimensional array).
The \emph{function} must accept at least as many arguments as the
number of argument \emph{sequences} supplied to \cdf{map-into}.
If \emph{result-sequence} and
the other argument \emph{sequences} are not all the same length, the iteration
terminates when the shortest sequence is exhausted.  If \emph{result-sequence}
is a vector with a fill pointer, the fill pointer is ignored when
deciding how many iterations to perform, and afterwards the
fill pointer is set to the number of times the \emph{function} was applied.

If the \emph{function} has side effects, it can count on being called
first on all the elements numbered \cd{0}, then on all those
numbered \cd{1}, and so on.

If \emph{result-sequence} is longer than the shortest element of \emph{sequences},
extra elements at the end of \emph{result-sequence} are unchanged.

The function \cdf{map-into} differs from \cdf{map} in that it modifies
an existing sequence rather than creating a new one.  In addition,
\cdf{map-into} can be called with only two arguments (\emph{result-sequence}
and \emph{function}), while \cdf{map} requires at least three arguments.

If \emph{result-sequence} is \cdf{nil}, \cdf{map-into} immediately returns
\cdf{nil}, because \cdf{nil} is a sequence of length zero.
\end{defun}


\begin{defun}[Function]
some predicate sequence &rest more-sequences \\
every predicate sequence &rest more-sequences \\
notany predicate sequence &rest more-sequences \\
notevery predicate sequence &rest more-sequences

These are all predicates.
The \emph{predicate} must take as many arguments as there are sequences
provided.  The \emph{predicate} is first applied to the elements
with index \cd{0} in each of the sequences, and possibly then to
the elements with index \cd{1}, and so on, until a termination
criterion is met or the end of the shortest of the \emph{sequences} is reached.

If the \emph{predicate} has side effects, it can count on being called
first on all the elements numbered \cd{0}, then on all those
numbered \cd{1}, and so on.

\cdf{some} returns as soon as any invocation of \emph{predicate}
returns a non-{\false} value; \cdf{some} returns that value.
If the end of a sequence is reached, \cdf{some} returns {\false}.
Thus, considered as a predicate, it is true if \emph{some} invocation of
\emph{predicate} is true.

\cdf{every} returns {\false} as soon as any invocation of \emph{predicate}
returns {\false}.
If the end of a sequence is reached, \cdf{every} returns a non-{\false} value.
Thus, considered as a predicate, it is true if \emph{every} invocation of
\emph{predicate} is true.

\cdf{notany} returns {\false} as soon as any invocation of \emph{predicate}
returns a non-{\false} value.
If the end of a sequence is reached, \cdf{notany} returns a non-{\false} value.
Thus, considered as a predicate, it is true if \emph{no} invocation of
\emph{predicate} is true.

\cdf{notevery} returns a non-{\false} value as soon as any invocation
of \emph{predicate} returns {\false}.  If the end of a sequence is reached,
\cdf{notevery} returns
{\false}.  Thus, considered as a predicate, it is true if \emph{not every} invocation of
\emph{predicate} is true.

\begin{new}
X3J13 voted in January 1989
\issue{MAPPING-DESTRUCTIVE-INTERACTION}
to restrict user side effects; see section~\ref{STRUCTURE-TRAVERSAL-SECTION}.
\end{new}
\end{defun}

\begin{defun}[Function]
reduce function sequence &key :from-end :start :end :initial-value

The \cdf{reduce} function combines all the elements of a sequence using
a binary operation; for example, using \cdf{+} one can add up all the
elements.

The specified subsequence of the \emph{sequence} is combined
or ``reduced'' using the
\emph{function}, which must accept two arguments.  The reduction is
left-associative, unless the \cd{:from-end} argument is true (it defaults
to {\nil}), in which case it is right-associative.  If an
\cd{:initial-value} argument is given, it is logically placed before the
subsequence (after it if \cd{:from-end} is true) and included in the
reduction operation.

If the specified subsequence contains exactly one element
and the keyword argument \cd{:initial-value}
is not given, then that element
is returned and the \emph{function} is not called.
If the specified subsequence is empty and an \cd{:initial-value} is given,
then the \cd{:initial-value} is returned
and the \emph{function} is not called.

If the specified subsequence is empty and no \cd{:initial-value} is given,
then the \emph{function} is called with zero
arguments, and \cdf{reduce} returns whatever the function does.  (This is
the only case where the \emph{function} is called with other than two
arguments.)

\begin{lisp}
(reduce \#'+ '(1 2 3 4)) \EV\ 10 \\
(reduce \#'- '(1 2 3 4)) \EQ\ (- (- (- 1 2) 3) 4) \EV\ -8 \\
(reduce \#'- '(1 2 3 4) :from-end t)~~~~~;\textrm{Alternating sum} \\
~~~\EQ\ (- 1 (- 2 (- 3 4))) \EV\ -2 \\
(reduce \#'+ '()) \EV\ 0 \\
(reduce \#'+ '(3)) \EV\ 3 \\
(reduce \#'+ '(foo)) \EV\ foo \\
(reduce \#'list '(1 2 3 4)) \EV\ (((1 2) 3) 4) \\
(reduce \#'list '(1 2 3 4) :from-end t) \EV\ (1 (2 (3 4))) \\
(reduce \#'list '(1 2 3 4) :initial-value 'foo) \\
~~~\EV\ ((((foo 1) 2) 3) 4) \\
(reduce \#'list '(1 2 3 4) \\
~~~~~~~~:from-end t :initial-value 'foo) \\
~~~\EV\ (1 (2 (3 (4 foo))))
\end{lisp}
If the \emph{function} produces side effects, the order of the calls
to the \emph{function} can be correctly predicted from the reduction ordering
demonstrated above.

The name ``reduce'' for this function is borrowed from {APL}.

\begin{new}
X3J13 voted in March 1988 \issue{REDUCE-ARGUMENT-EXTRACTION}
to extend the \cdf{reduce} function to take
an additional keyword argument named \cd{:key}.  As usual, this argument
defaults to the identity function.  The value of this argument must be
a function that accepts at least one argument.  This function is applied once
to each element of the
sequence that is to participate in the reduction operation, in the order
implied by the \cd{:from-end} argument; the values returned by this
function are combined by the reduction \emph{function}.
However, the \cd{:key} function is \emph{not} applied
to the \cd{:initial-value} argument (if any).
\end{new}

\begin{new}
X3J13 voted in January 1989
\issue{MAPPING-DESTRUCTIVE-INTERACTION}
to restrict user side effects; see section~\ref{STRUCTURE-TRAVERSAL-SECTION}.
\end{new}
\end{defun}

\section{Modifying Sequences}

Each of these functions alters the contents of a sequence or produces
an altered copy of a given sequence.

\begin{defun}[Function]
fill sequence item &key :start :end

The \emph{sequence} is destructively modified by replacing each element of
the subsequence specified by the \cd{:start} and \cd{:end} parameters
with the \emph{item}.  The \emph{item} may be any Lisp object but must be a
suitable element for the \emph{sequence}.  The \emph{item} is stored into all
specified components of the \emph{sequence}, beginning at the one specified
by the \cd{:start} index (which defaults to zero), up to but not
including the one specified by the \cd{:end} index (which defaults to the
length of the sequence).  \cdf{fill} returns the modified \emph{sequence}.
For example:
\begin{lisp}
(setq x (vector 'a 'b 'c 'd 'e)) \EV\ \#(a b c d e) \\
(fill x 'z \cd{:start} 1 \cd{:end} 3) \EV\ \#(a z z d e) \\
~~\textrm{and now} x \EV\ \#(a z z d e) \\
(fill x 'p) \EV\ \#(p p p p p) \\
~~\textrm{and now} x \EV\ \#(p p p p p)
\end{lisp}
\end{defun}

\begin{defun}[Function]
replace sequence1 sequence2 &key :start1 :end1 :start2~:end2

The sequence \emph{sequence1} is destructively modified by copying successive
elements into it from \emph{sequence2}.  The elements of
\emph{sequence2} must be of a type that may be stored into
\emph{sequence1}.  The subsequence of \emph{sequence2}
specified by \cd{:start2} and \cd{:end2} is copied into the
subsequence of \emph{sequence1} specified by \cd{:start1} and \cd{:end1}.
(The arguments \cd{:start1} and \cd{:start2} default to zero.
The arguments \cd{:end1} and \cd{:end2} default
to {\false}, meaning the end of the appropriate sequence.)
If these subsequences are not of the same length, then
the shorter length determines how many elements are copied; the extra
elements near the end of the longer subsequence are not involved in the
operation.
The number of elements copied may be expressed as:
\begin{lisp}
(min (- \emph{end1} \emph{start1}) (- \emph{end2} \emph{start2}))
\end{lisp}
The value returned by \cdf{replace} is the modified \emph{sequence1}.

If \emph{sequence1} and \emph{sequence2} are the same (\cdf{eq}) object
and the region being modified overlaps the region being copied
from, then it is as if the entire source region were copied to another
place and only then copied back into the target region.
However, if \emph{sequence1} and \emph{sequence2} are \emph{not} the same,
but the region being modified overlaps the region being copied from
(perhaps because of shared list structure or displaced arrays),
then after the \cdf{replace} operation
the subsequence of \emph{sequence1} being modified will have
unpredictable contents.
\end{defun}

\begin{defun}[Function]
remove item sequence &key :from-end :test :test-not :start :end :count :key \\
remove-if predicate sequence &key :from-end :start :end :count :key \\
remove-if-not predicate sequence &key :from-end :start :end :count :key

The result is a sequence of the same kind as the argument \emph{sequence}
that has the same elements except that those in the subsequence
delimited by \cd{:start} and \cd{:end} and satisfying the test (see
above) have been removed.  This is a non-destructive operation; the result
is a copy of the input \emph{sequence}, save that some elements are not
copied.  Elements not removed occur in the same order in the result
as they did in the argument.

The \cd{:count} argument, if supplied, limits the number of elements
removed; if more than \cd{:count} elements satisfy the test,
then of these elements only the leftmost are removed,
as many as specified by \cd{:count}.

\begin{new}
X3J13 voted in January 1989
\issue{RANGE-OF-COUNT-KEYWORD}
to clarify that the \cd{:count} argument must be either \cdf{nil}
or an integer, and that supplying a negative integer produces the
same behavior as supplying zero.
\end{new}

A non-{\false} \cd{:from-end} specification
matters only when the \cd{:count} argument
is provided; in that case only the rightmost \cd{:count} elements satisfying
the test are removed.
For example:
\begin{lisp}
(remove 4 '(1 2 4 1 3 4 5)) \EV\ (1 2 1 3 5) \\
(remove 4 '(1 2 4 1 3 4 5) \cd{:count} 1) \EV\ (1 2 1 3 4 5) \\
(remove 4 '(1 2 4 1 3 4 5) \cd{:count} 1 \cd{:from-end} t) \\
~~~\EV\ (1 2 4 1 3 5) \\
(remove 3 '(1 2 4 1 3 4 5) \cd{:test} \#'>) \EV\ (4 3 4 5) \\
(remove-if \#'oddp '(1 2 4 1 3 4 5)) \EV\ (2 4 4) \\
(remove-if \#'evenp '(1 2 4 1 3 4 5) \cd{:count} 1 \cd{:from-end} t) \\
~~~\EV\ (1 2 4 1 3 5)
\end{lisp}
The result of \cdf{remove} may share
with the argument \emph{sequence}; a list result may share a tail
with an input list, and the result may be \cdf{eq} to the input \emph{sequence}
if no elements need to be removed.

\begin{new}
X3J13 voted in January 1989
\issue{MAPPING-DESTRUCTIVE-INTERACTION}
to restrict user side effects; see section~\ref{STRUCTURE-TRAVERSAL-SECTION}.
\end{new}
\end{defun}

\begin{defun}[Function]
delete item sequence &key :from-end :test :test-not :start~:end~:count~:key \\
delete-if predicate sequence &key :from-end :start~:end~:count~:key \\
delete-if-not predicate sequence &key :from-end :start~:end~:count~:key

This is the destructive counterpart to \cdf{remove}.
The result is a sequence of the same kind as the argument \emph{sequence}
that has the same elements except that those in the subsequence
delimited by \cd{:start} and \cd{:end} and satisfying the test (see
above) have been deleted.  This is a destructive operation.
The argument \emph{sequence} may be destroyed and used to construct
the result; however, the result may or may not be \cdf{eq} to \emph{sequence}.
Elements not deleted occur in the same order in the result
as they did in the argument.

The \cd{:count} argument, if supplied, limits the number of elements
deleted; if more than \cd{:count} elements satisfy the test,
then of these elements only the leftmost are deleted,
as many as specified by \cd{:count}.

\begin{new}
X3J13 voted in January 1989
\issue{RANGE-OF-COUNT-KEYWORD}
to clarify that the \cd{:count} argument must be either \cdf{nil}
or an integer, and that supplying a negative integer produces the
same behavior as supplying zero.
\end{new}

A non-{\false} \cd{:from-end} specification
matters only when the \cd{:count} argument
is provided; in that case only the rightmost \cd{:count} elements satisfying
the test are deleted.
For example:
\begin{lisp}
(delete 4 '(1 2 4 1 3 4 5)) \EV\ (1 2 1 3 5) \\
(delete 4 '(1 2 4 1 3 4 5) \cd{:count} 1) \EV\ (1 2 1 3 4 5) \\
(delete 4 '(1 2 4 1 3 4 5) \cd{:count} 1 \cd{:from-end} t) \\
~~~\EV\ (1 2 4 1 3 5) \\
(delete 3 '(1 2 4 1 3 4 5) \cd{:test} \#'>) \EV\ (4 3 4 5) \\
(delete-if \#'oddp '(1 2 4 1 3 4 5)) \EV\ (2 4 4) \\
(delete-if \#'evenp '(1 2 4 1 3 4 5) \cd{:count} 1 \cd{:from-end} t) \\
~~~\EV\ (1 2 4 1 3 5)
\end{lisp}

\begin{new}
X3J13 voted in January 1989
\issue{MAPPING-DESTRUCTIVE-INTERACTION}
to restrict user side effects; see section~\ref{STRUCTURE-TRAVERSAL-SECTION}.
\end{new}

\begin{newer}
X3J13 voted in March 1989 \issue{REMF-DESTRUCTION-UNSPECIFIED}
to clarify the permissible side effects of certain operations.
When the \emph{sequence} is a list,
\cdf{delete} is permitted to perform a \cdf{setf} on any part,
\emph{car} or \emph{cdr}, of the top-level list structure of that list.
When the \emph{sequence} is an array,
\cdf{delete} is permitted to alter the dimensions of the given array
and to slide some of its elements into new positions without permuting them
in order to produce the resulting array.

Furthermore, \cd{(delete-if \emph{predicate} \emph{sequence}~...)}
is required to behave exactly like
\begin{lisp}
(delete nil \emph{sequence} \\*
~~~~~~~~:test \#'(lambda (unused item) \\*
~~~~~~~~~~~~~~~~~~~(declare (ignore unused)) \\*
~~~~~~~~~~~~~~~~~~~(funcall \emph{predicate} item)) \\*
~~~~~~~~...)
\end{lisp}
\end{newer}
\end{defun}

\begin{defun}[Function]
remove-duplicates sequence &key :from-end :test :test-not :start :end :key \\
delete-duplicates sequence &key :from-end :test :test-not :start :end :key

The elements of \emph{sequence} are compared pairwise, and if any two match,
then the one occurring earlier in the sequence
is discarded (but if the \cd{:from-end} argument is true, then the one
later in the sequence is discarded).
The result is a sequence of the same kind as the
argument sequence with enough elements removed so that no two of the remaining
elements match.  The order of the elements remaining in the result
is the same as the order in which they appear in \emph{sequence}.

\cdf{remove-duplicates} is the non-destructive version
of this operation.
The result of \cdf{remove-duplicates} may share
with the argument \emph{sequence}; a list result may share a tail
with an input list, and the result may be \cdf{eq} to the input \emph{sequence}
if no elements need to be removed.

\cdf{delete-duplicates} may destroy the argument \emph{sequence}.

Some examples:
\begin{lisp}
(remove-duplicates '(a b c b d d e)) \EV\ (a c b d e) \\
(remove-duplicates '(a b c b d d e) \cd{:from-end} t) \EV\ (a b c d e) \\
(remove-duplicates '((foo \#{\Xbackslash}a) (bar \#{\Xbackslash}\%) (baz \#{\Xbackslash}A)) \\
~~~~~~~~~~~~~~~~~~~\cd{:test} \#'char-equal \cd{:key} \#'cadr) \\
~~~\EV\ ((bar \#{\Xbackslash}\%) (baz \#{\Xbackslash}A)) \\
(remove-duplicates '((foo \#{\Xbackslash}a) (bar \#{\Xbackslash}\%) (baz \#{\Xbackslash}A)) \\
~~~~~~~~~~~~~~~~~~~\cd{:test} \#'char-equal \cd{:key} \#'cadr \cd{:from-end} t) \\
~~~\EV\ ((foo \#{\Xbackslash}a) (bar \#{\Xbackslash}\%))
\end{lisp}

These functions are useful for converting a sequence into a canonical
form suitable for representing a set.

\begin{new}
X3J13 voted in January 1989
\issue{MAPPING-DESTRUCTIVE-INTERACTION}
to restrict user side effects; see section~\ref{STRUCTURE-TRAVERSAL-SECTION}.
\end{new}

\begin{newer}
X3J13 voted in March 1989 \issue{REMF-DESTRUCTION-UNSPECIFIED}
to clarify the permissible side effects of certain operations.
When the \emph{sequence} is a list,
\cdf{delete-duplicates} is permitted to perform a \cdf{setf} on any part,
\emph{car} or \emph{cdr}, of the top-level list structure of that list.
When the \emph{sequence} is an array,
\cdf{delete-duplicates} is permitted to alter the dimensions of the given array
and to slide some of its elements into new positions without permuting them
in order to produce the resulting array.
\end{newer}
\end{defun}

\begin{defun}[Function]
substitute newitem olditem sequence &key :from-end :test :test-not :start :end :count :key \\
substitute-if newitem test sequence &key :from-end :start~:end :count :key \\
substitute-if-not newitem test sequence &key :from-end :start :end :count :key

The result is a sequence of the same kind as the argument \emph{sequence}
that has the same elements except that those in the subsequence
delimited by \cd{:start} and \cd{:end} and satisfying the test (see
above) have been replaced by \emph{newitem}.  This is a non-destructive
operation; the result is a copy of the input \emph{sequence}, save that some
elements are changed.

The \cd{:count} argument, if supplied, limits the number of elements
altered; if more than \cd{:count} elements satisfy the test,
then of these elements only the leftmost are replaced,
as many as specified by \cd{:count}.

\begin{new}
X3J13 voted in January 1989
\issue{RANGE-OF-COUNT-KEYWORD}
to clarify that the \cd{:count} argument must be either \cdf{nil}
or an integer, and that supplying a negative integer produces the
same behavior as supplying zero.
\end{new}

A non-{\false} \cd{:from-end} specification
matters only when the \cd{:count} argument
is provided; in that case only the rightmost \cd{:count} elements satisfying
the test are replaced.
For example:
\begin{lisp}
(substitute 9 4 '(1 2 4 1 3 4 5)) \EV\ (1 2 9 1 3 9 5) \\
(substitute 9 4 '(1 2 4 1 3 4 5) \cd{:count} 1) \EV\ (1 2 9 1 3 4 5) \\
(substitute 9 4 '(1 2 4 1 3 4 5) \cd{:count} 1 \cd{:from-end} t) \\
~~~\EV\ (1 2 4 1 3 9 5) \\
(substitute 9 3 '(1 2 4 1 3 4 5) \cd{:test} \#'>) \EV\ (9 9 4 9 3 4 5) \\
(substitute-if 9 \#'oddp '(1 2 4 1 3 4 5)) \EV\ (9 2 4 9 9 4 9) \\
(substitute-if 9 \#'evenp '(1 2 4 1 3 4 5) \cd{:count} 1 \cd{:from-end} t) \\
~~~\EV\ (1 2 4 1 3 9 5)
\end{lisp}
The result of \cdf{substitute} may share
with the argument \emph{sequence}; a list result may share a tail
with an input list, and the result may be \cdf{eq} to the input \emph{sequence}
if no elements need to be changed.

See also \cdf{subst}, which performs substitutions throughout a tree.

\begin{new}
X3J13 voted in January 1989
\issue{MAPPING-DESTRUCTIVE-INTERACTION}
to restrict user side effects; see section~\ref{STRUCTURE-TRAVERSAL-SECTION}.
\end{new}
\end{defun}

\begin{defun}[Function]
nsubstitute newitem olditem sequence &key :from-end :test :test-not :start :end :count :key \\
nsubstitute-if newitem test sequence &key :from-end :start~:end :count :key \\
nsubstitute-if-not newitem test sequence &key :from-end :start :end :count :key

This is the destructive counterpart to \cdf{substitute}.
The result is a sequence of the same kind as the argument \emph{sequence}
that has the same elements except that those in the subsequence
delimited by \cd{:start} and \cd{:end} and satisfying the test (see
above) have been replaced by \emph{newitem}.  This is a destructive operation.
The argument \emph{sequence} may be destroyed and used to construct
the result; however, the result may or may not be \cdf{eq} to \emph{sequence}.

See also \cdf{nsubst}, which performs destructive
substitutions throughout a tree.
\begin{new}
X3J13 voted in January 1989
\issue{MAPPING-DESTRUCTIVE-INTERACTION}
to restrict user side effects; see section~\ref{STRUCTURE-TRAVERSAL-SECTION}.
\end{new}

\begin{newer}
X3J13 voted in March 1989 \issue{REMF-DESTRUCTION-UNSPECIFIED}
to clarify the permissible side effects of certain operations.
When the \emph{sequence} is a list,
\cdf{nsubstitute} or \cdf{nsubstitute-if}
is required to perform a \cdf{setf} on any
\emph{car} of the top-level list structure of that list
whose old contents must be replaced with \emph{newitem}
but is forbidden to perform a \cdf{setf} on any \cdf{cdr} of the list.
When the \emph{sequence} is an array,
\cdf{nsubstitute} or \cdf{nsubstitute-if}
is required to perform a \cdf{setf} on any element of the array
whose old contents must be replaced with \emph{newitem}.
These functions, therefore, may successfully be
used solely for effect, the caller discarding the returned value
(though some programmers find this stylistically distasteful).
\end{newer}
\end{defun}

\section{Searching Sequences for Items}

Each of these functions searches a sequence to locate one or more
elements satisfying some test.

\begin{defun}[Function]
find item sequence &key :from-end :test :test-not :start~:end :key \\
find-if predicate sequence &key :from-end :start :end :key \\
find-if-not predicate sequence &key :from-end :start~:end~:key

If the \emph{sequence} contains an element satisfying the test,
then the leftmost such element
is returned; otherwise {\false} is returned.

If \cd{:start} and \cd{:end} keyword arguments are given,
only the specified subsequence of \emph{sequence} is searched.

If a non-{\false} \cd{:from-end} keyword argument is specified, then the result is
the \emph{rightmost} element satisfying the test.

\begin{new}
X3J13 voted in January 1989
\issue{MAPPING-DESTRUCTIVE-INTERACTION}
to restrict user side effects; see section~\ref{STRUCTURE-TRAVERSAL-SECTION}.
\end{new}
\end{defun}

\begin{defun}[Function]
position item sequence &key :from-end :test :test-not :start~:end~:key \\
position-if predicate sequence &key :from-end :start~:end~:key \\
position-if-not predicate sequence &key :from-end :start~:end~:key

If the \emph{sequence} contains an element satisfying the test,
then the index within the sequence of the leftmost such element
is returned as a non-negative integer; otherwise {\false} is returned.

If \cd{:start} and \cd{:end} keyword arguments are given,
only the specified subsequence of \emph{sequence} is searched.
However, the index returned is relative to the entire sequence,
not to the subsequence.

If a non-{\false} \cd{:from-end} keyword argument is specified, then the result is
the index of the \emph{rightmost} element satisfying the test.  (The index
returned, however, is an index from the left-hand end, as usual.)

\begin{new}
X3J13 voted in January 1989
\issue{MAPPING-DESTRUCTIVE-INTERACTION}
to restrict user side effects; see section~\ref{STRUCTURE-TRAVERSAL-SECTION}.
\end{new}
\end{defun}

Here is a simple piece of code that uses several of the sequence
functions, notably \cdf{position-if} and \cdf{find-if},
to process strings.  Note one use of \cdf{loop} as well.
\begin{lisp}
(defun debug-palindrome (s) \\*
~~(flet ((match (x) (char-equal (first x) (third x)))) \\*
~~~~(let* ((pairs (loop for c across s \\*
~~~~~~~~~~~~~~~~~~~~~~~~for j from 0 \\*
~~~~~~~~~~~~~~~~~~~~~~~~when (alpha-char-p c) \\*
~~~~~~~~~~~~~~~~~~~~~~~~~~collect (list c j))) \\*
~~~~~~~~~~~(quads (mapcar \#'append pairs (reverse pairs))) \\*
~~~~~~~~~~~(diffpos (position-if (complement \#'match) quads))) \\
~~~~~~(when diffpos \\*
~~~~~~~~(let* ((diff (elt quads diffpos)) \\*
~~~~~~~~~~~~~~~(same (find-if \#'match quads \\*
~~~~~~~~~~~~~~~~~~~~~~~~~~~~~~:start (+ diffpos 1)))) \\
~~~~~~~~~~(if same \\*
~~~~~~~~~~~~~~(format nil \\*
~~~~~~~~~~~~~~~~~~~~~~"/{\Xtilde}A/ (at {\Xtilde}D) is not the reverse of /{\Xtilde}A/" \\*
~~~~~~~~~~~~~~~~~~~~~~(subseq s (second diff) (second same)) \\*
~~~~~~~~~~~~~~~~~~~~~~(second diff) \\*
~~~~~~~~~~~~~~~~~~~~~~(subseq s (+ (fourth same) 1) \\*
~~~~~~~~~~~~~~~~~~~~~~~~~~~~~~~~(+ (fourth diff) 1))) \\*
~~~~~~~~~~~~~~"This palindrome is completely messed up!"))))))
\end{lisp}
Here is an example of its behavior.
\begin{lisp}
(setq panama~~~~~;\textrm{A putative palindrome?} \\*
~~~~~~"A man, a plan, a canoe, pasta, heros, rajahs, \\*
~~~~~~~a coloratura, maps, waste, percale, macaroni, a gag, \\*
~~~~~~~a banana bag, a tan, a tag, a banana bag again \\*
~~~~~~~(or a camel), a crepe, pins, Spam, a rut, a Rolo, \\*
~~~~~~~cash, a jar, sore hats, a peon, a canal--Panama!")
\end{lisp}
\begin{lisp}
(debug-palindrome panama) \\*
~~\EV\ "/wast/ (at 73) is not the reverse of /, pins/" \\
\\
(replace panama "snipe" :start1 73)~~~~~;\textrm{Repair it} \\*
~~\EV\ "A man, a plan, a canoe, pasta, heros, rajahs, \\*
~~~~~~~a coloratura, maps, snipe, percale, macaroni, a gag, \\*
~~~~~~~a banana bag, a tan, a tag, a banana bag again \\*
~~~~~~~(or a camel), a crepe, pins, Spam, a rut, a Rolo, \\*
~~~~~~~cash, a jar, sore hats, a peon, a canal--Panama!" \\
\\
(debug-palindrome panama) \EV\ nil~~~~~;\textrm{Copacetic---a true palindrome} \\
\\
(debug-palindrome "Rubber baby buggy bumpers") \\*
~~\EV\ "/Rubber / (at 0) is not the reverse of /umpers/" \\
\\
(debug-palindrome "Common Lisp: The Language") \\*
~~\EV\ "/Commo/ (at 0) is not the reverse of /guage/" \\
\\
(debug-palindrome "Complete mismatches are hard to find") \\*
~~\EV\ \\
~~"/Complete mism/ (at 0) is not the reverse of /re hard to find/" \\
\\
(debug-palindrome "Waltz, nymph, for quick jigs vex Bud") \\*
~~\EV\ "This palindrome is completely messed up!" \\
\\
(debug-palindrome "Doc, note: I dissent.~~A fast never \\*
~~~~~~~~~~~~~~~~~~~prevents a fatness.~~I diet on cod.") \\*
~~\EV\nil~~~~~;\textrm{Another winner} \\
\\
(debug-palindrome "Top step's pup's pet spot") \EV\ nil
\end{lisp}

\begin{defun}[Function]
count item sequence &key :from-end :test :test-not :start~:end~:key \\
count-if predicate sequence &key :from-end :start~:end~:key \\
count-if-not predicate sequence &key :from-end :start~:end~:key

The result is always a non-negative integer, the number of
elements in the specified subsequence of \emph{sequence} satisfying
the test.

The \cd{:from-end} argument does not affect the result returned;
it is accepted purely for compatibility with other sequence functions.

\begin{new}
X3J13 voted in January 1989
\issue{MAPPING-DESTRUCTIVE-INTERACTION}
to restrict user side effects; see section~\ref{STRUCTURE-TRAVERSAL-SECTION}.
\end{new}
\end{defun}

\begin{defun}[Function]
mismatch sequence1 sequence2 &key :from-end :test :test-not :key :start1 :start2 :end1 :end2

The specified subsequences of
\emph{sequence1} and \emph{sequence2} are compared element-wise.
If they are of equal length and match in every element, the result is
{\false}.  Otherwise, the result is a non-negative integer.
This result is the index within
\emph{sequence1} of the leftmost position at which the two
subsequences fail to match; or,
if one subsequence is shorter than and a matching prefix of the other,
the result is the index
relative to \emph{sequence1} beyond the last position tested.

If a non-{\false} \cd{:from-end} keyword argument is given, then
\emph{one plus} the index of the \emph{rightmost}
position in which the sequences differ is returned.  In effect, the (sub)sequences
are aligned at their right-hand ends; then, the last elements are compared,
the penultimate elements, and so on.  The index returned is again
an index relative to \emph{sequence1}.

\begin{new}
X3J13 voted in January 1989
\issue{MAPPING-DESTRUCTIVE-INTERACTION}
to restrict user side effects; see section~\ref{STRUCTURE-TRAVERSAL-SECTION}.
\end{new}
\end{defun}

\begin{defun}[Function]
search sequence1 sequence2 &key :from-end :test :test-not :key :start1 :start2 :end1 :end2

A search is conducted for a subsequence of \emph{sequence2} that
element-wise matches \emph{sequence1}.
If there is no such subsequence, the result is {\false}; if there is,
the result is the index into \emph{sequence2} of the leftmost element
of the leftmost such matching subsequence.

If a non-{\false} \cd{:from-end} keyword argument is given,
the index of the leftmost
element of the \emph{rightmost} matching subsequence is returned.

The implementation may choose to search the sequence in any order;
there is no guarantee on the number of times the test is made.
For example, \cdf{search} with a non-{\nil} \cd{:from-end}
argument might actually search a list from left to right
instead of from right to left (but in either case would return
the rightmost matching subsequence, of course).  Therefore it is a good
idea for a user-supplied predicate to be free of side effects.

\begin{new}
X3J13 voted in January 1989
\issue{MAPPING-DESTRUCTIVE-INTERACTION}
to restrict user side effects; see section~\ref{STRUCTURE-TRAVERSAL-SECTION}.
\end{new}
\end{defun}

\section{Sorting and Merging}

These functions may destructively modify argument sequences
in order to put a sequence into sorted order or to merge two
already sorted sequences.

\begin{defun}[Function]
sort sequence predicate &key :key \\
stable-sort sequence predicate &key :key

\indexterm{sorting}
The \emph{sequence} is destructively sorted according to an order determined by
the \emph{predicate}.  The \emph{predicate} should take two
arguments, and return non-{\false} if and only if the first argument is
strictly less than the second (in some appropriate sense). 
If the first argument is greater than or equal to the second
(in the appropriate sense), then the \emph{predicate} should return {\false}.

The \cdf{sort} function determines the relationship between two elements
by giving keys extracted from the elements to the \emph{predicate}.
The \cd{:key} argument, when applied to an element, should return
the key for that element.  The \cd{:key} argument defaults to the identity
function, thereby making the element itself be the key.

The \cd{:key} function should not have any side effects.
A useful example of a \cd{:key} function would be a component
selector function for a \cdf{defstruct} structure, used in sorting
a sequence of structures.
\begin{lisp}
(sort \emph{a} \emph{p} \cd{:key} \emph{s})
   \EQ\ (sort \emph{a} \#'(lambda (x y) (\emph{p} (\emph{s} x) (\emph{s} y))))
\end{lisp}
While the above two expressions are equivalent, the first may be more
efficient in some implementations for certain types of arguments.  For
example, an implementation may choose to apply \emph{s} to each
item just once, putting the resulting keys into a separate table, and
then sort the parallel tables, as opposed to applying
\emph{s} to an item every time just before applying the \emph{predicate}.

If the \cd{:key} and \emph{predicate} functions always return, then the
sorting operation will always terminate, producing a sequence containing
the same elements as the original sequence (that is, the result is a
permutation of \emph{sequence}).  This is guaranteed even if the
\emph{predicate} does not really consistently represent a total order
(in which case the elements will be scrambled in some unpredictable
way, but no element will be lost).  If
the \cd{:key} function consistently returns meaningful keys,
and the \emph{predicate}
does reflect some total ordering criterion on those keys, then the
elements of the result sequence will be properly sorted according
to that ordering.

The sorting operation performed by \cdf{sort} is not guaranteed \emph{stable}.
Elements considered equal by the \emph{predicate} may or may not
stay in their original order.  (The \emph{predicate} is assumed to
consider two elements \emph{x} and \emph{y} to be equal if
\cd{(funcall \emph{predicate} \emph{x} \emph{y})} and
\cd{(funcall \emph{predicate} \emph{y} \emph{x})} are both false.)
The function \cdf{stable-sort} guarantees
stability but may be slower than \cdf{sort} in some situations.

The sorting operation may be destructive in all cases.  In the case of an
array argument, this is accomplished by permuting the elements in place.
In the case of a list, the list is
destructively reordered in the same manner as for
\cdf{nreverse}.  Thus if the argument should not be destroyed, the
user must sort a copy of the argument.

Should execution of the \cd{:key} function or the \emph{predicate} cause an error,
the state of the list or array being sorted is
undefined.  However, if the error is corrected, the sort will, of
course, proceed correctly. 

Note that since sorting requires many comparisons, and thus
many calls to the \emph{predicate}, sorting will be much faster if the
\emph{predicate} is a compiled function rather than interpreted. 

An example:
\begin{lisp}
(setq foovector (sort foovector \#'string-lessp \cd{:key} \#'car))
\end{lisp}
If \cdf{foovector} contained these items before the sort
\begin{lisp}
("Tokens" "The Lion Sleeps Tonight") \\
("Carpenters" "Close to You") \\
("Rolling Stones" "Brown Sugar") \\
("Beach Boys" "I Get Around") \\
("Mozart" "Eine Kleine Nachtmusik" (K 525)) \\
("Beatles" "I Want to Hold Your Hand")
\end{lisp}
then after the sort \cdf{foovector} would contain
\begin{lisp}
("Beach Boys" "I Get Around") \\
("Beatles" "I Want to Hold Your Hand") \\
("Carpenters" "Close to You") \\
("Mozart" "Eine Kleine Nachtmusik" (K 525)) \\
("Rolling Stones" "Brown Sugar") \\
("Tokens" "The Lion Sleeps Tonight")
\end{lisp}

\begin{new}
X3J13 voted in January 1989
\issue{MAPPING-DESTRUCTIVE-INTERACTION}
to restrict user side effects; see section~\ref{STRUCTURE-TRAVERSAL-SECTION}.
\end{new}
\end{defun}

\begin{defun}[Function]
merge result-type sequence1 sequence2 predicate &key :key

The sequences \emph{sequence1} and \emph{sequence2} are destructively
merged according to an order determined by
the \emph{predicate}.  The result is a sequence of type \emph{result-type},
which must be a subtype of \cdf{sequence}, as for the function \cdf{coerce}.
The \emph{predicate} should take two
arguments and return non-{\false} if and only if the first argument is
strictly less than the second (in some appropriate sense). 
If the first argument is greater than or equal to the second
(in the appropriate sense), then the \emph{predicate} should return {\false}.

The \cdf{merge} function determines the relationship between two elements
by giving keys extracted from the elements to the \emph{predicate}.
The \cd{:key} function, when applied to an element, should return
the key for that element; the \cd{:key} function defaults to the identity
function, thereby making the element itself be the key.

The \cd{:key} function should not have any side effects.
A useful example of a \cd{:key} function would be a component
selector function for a \cdf{defstruct} structure, used to merge
a sequence of structures.

If the \cd{:key} and \emph{predicate} functions always return, then the
merging operation will always terminate.
The result of merging two sequences \emph{x} and \emph{y} is a new sequence
\emph{z}, such that the length of \emph{z} is the sum of the lengths of \emph{x}
and \emph{y}, and \emph{z} contains all the elements of \emph{x} and \emph{y}.
If \emph{x1} and \emph{x2} are two elements of \emph{x}, and \emph{x1} precedes
\emph{x2} in \emph{x}, then \emph{x1} precedes \emph{x2} in \emph{z}, and similarly for
elements of \emph{y}.  In short, \emph{z} is an \emph{interleaving} of \emph{x}
 and \emph{y}.

Moreover, if \emph{x} and \emph{y} were correctly sorted according to the
\emph{predicate}, then \emph{z} will also be correctly sorted,
as shown in this example.
\begin{lisp}
(merge 'list '(1 3 4 6 7) '(2 5 8) \#'<) \EV\ (1 2 3 4 5 6 7 8)
\end{lisp}
If \emph{x} or \emph{y} is not so sorted then \emph{z} will not be sorted,
but will nevertheless be an interleaving of \emph{x} and \emph{y}.

The merging operation is guaranteed
\emph{stable}; if two or more elements are considered equal by the
\emph{predicate}, then the elements from \emph{sequence1} will
precede those from \emph{sequence2} in the result.
(The \emph{predicate} is assumed to
consider two elements \emph{x} and \emph{y} to be equal if
\cd{(funcall \emph{predicate} \emph{x} \emph{y})} and
\cd{(funcall \emph{predicate} \emph{y} \emph{x})} are both false.)
For example:
\begin{lisp}
(merge 'string "BOY" "nosy" \#'char-lessp) \EV\ "BnOosYy"
\end{lisp}
The result can \emph{not} be \cd{"BnoOsYy"}, \cd{"BnOosyY"}, or \cd{"BnoOsyY"}.
The function \cdf{char-lessp} ignores case, and so considers
the characters \cdf{Y} and \cdf{y} to be equal, for example;
the stability property then guarantees that the character from the
first argument (\cdf{Y}) must precede the one from the second
argument (\cdf{y}).

\begin{newer}
X3J13 voted in June 1989 \issue{SEQUENCE-TYPE-LENGTH} to specify that
\cdf{merge} should signal an error if the sequence type specifies the number of
elements and the sum of the lengths of the two sequence arguments is
different.
\end{newer}

\begin{new}
X3J13 voted in January 1989
\issue{MAPPING-DESTRUCTIVE-INTERACTION}
to restrict user side effects; see section~\ref{STRUCTURE-TRAVERSAL-SECTION}.
\end{new}
\end{defun}

%RUSSIAN
\else

\chapter{Последовательности}
\label{KSEQUE}

Тип \cdf{sequence} объединяет списки и вектора (одномерные массивы).
Хотя это две различные структуры данных с различными структурными
свойствами, приводящими к алгоритмически различному использованию, они имеют
общее свойство: каждая хранит упорядоченное множество элементов.

Некоторые операции полезны и для списков и для массивов, потому что они
взаимодействуют с упорядоченными множествами элементов. Можно запросить
количество элементов, изменить порядок элементов на противоположный, извлечь
подмножество (подпоследовательность) и так далее. Для таких целей Common Lisp
предлагает ряд общих для всех типов последовательностей функций.

\begin{flushleft}
\begin{tabular*}{\textwidth}{@{}l@{\extracolsep{\fill}}lll@{}}
elt&reverse&map&remove \\
length&nreverse&some&remove-duplicates \\
subseq&concatenate&every&delete \\
copy-seq&position&notany&delete-duplicates \\
fill&find&notevery&substitute \\
replace&sort&reduce&nsubstitute \\
count&merge&search&mismatch
\end{tabular*}
\end{flushleft}
Некоторые из этих операций имеют более одной версии.
Такие версии отличаются суффиксом (или префиксом) от имени базовой функции.
Кроме того, многие операции принимают один или более необязательных именованных
аргументов, которые могут изменить поведение операции.

Если операция требует проверки элементов последовательности на совпадение с
некоторым условием, тогда это условие может быть указано одним из двух способов.
Основная операция принимает объект и сравнивает с ним каждый элемент
последовательности на равенство \cdf{eql}.
(Операция сравнения может быть задана в именованном параметре \cd{:test} или
\cd{:test-not}. Использование обоих параметров одновременно является ошибкой.)
Другие варианты операции образуются с добавлением префиксов \cdf{-if} и
\cdf{-if-not}. Эти операции, в отличие от базовой, принимают не объект, а
предикат с одним аргументом. В этом случае проверяется истинность или ложность
предиката для каждого элемента последовательности.
В качестве примера,
\begin{lisp}
(remove \emph{item} \emph{sequence})
\end{lisp}
возвращает копию последовательности \emph{sequence}, в которой удалены все
элементы равные \cdf{eql} объекту \emph{emph}.
\begin{lisp}
(remove \emph{item} \emph{sequence} \cd{:test} \#'equal)
\end{lisp}
возвращает копию последовательности \emph{sequence}, в которой удалены все
элементы равные \cdf{equal} объекту \emph{emph}.
\begin{lisp}
(remove-if \#'numberp \emph{sequence})
\end{lisp}
возвращает копию последовательности \emph{sequence}, в которой удалены все
числа.

Если операция любым методом проверяет элементы последовательности, 
то если именованный параметр \cdf{:key} не равняется {\false}, то он должен быть
функцией одного аргумента, которая будет извлекать из элемента необходимую для
проверки часть.
Например,
\begin{lisp}
(find \emph{item} \emph{sequence} \cd{:test} \#'eq \cd{:key} \#'car)
\end{lisp}
Это выражение ищет первый элемент последовательности \emph{sequence}, \emph{car}
элемент которого равен \cdf{eq} объекту \emph{item}.

\begin{newer}
X3J13 voted in June 1988 \issue{FUNCTION-TYPE} to allow the \cd{:key} function
to be only of type \cdf{symbol} or \cdf{function}; a lambda-expression
is no longer acceptable as a functional argument.  One must use the
\cdf{function} special operator or the abbreviation \cd{\#'} before
a lambda-expression that appears as an  explicit argument form.
\end{newer}

Для некоторых операций было бы удобно указать направление обработки
последовательности. В этом случае базовые операции обычно обрабатывают
последовательность в прямом направлении. Обратное направление обработки
указывается с помощью не-{\false} значения для именованного параметра
\cd{:from-end}. (Порядок обработки указываемый в \cd{:from-end} чисто
концептуальный. В зависимости от обрабатываемого объекта и реализации,
действительный порядок обработки может отличаться. Поэтому пользовательские
функции \emph{test} не должны иметь побочных эффектов.)

Много операций позволяют указать подпоследовательность для обработки. Такие
операции имеют именованные параметры \cd{:start} и \cd{:end}. Эти аргумент
должны быть индексами внутри последовательности, и
$\emph{start}\leq\emph{end}$. Ситуация $\emph{start}>\emph{end}$ является
ошибкой. Эти параметры указывают подпоследовательность начиная с позиции \emph{start}
включительно и заканчивая позицией \emph{end} невключительно. Таким образом
длина подпоследовательности равна $\emph{end}-\emph{start}$. Если параметр
\emph{start} опущен, используется значение по-умолчанию ноль. Если параметр
\emph{end} опущен, используется значение по-умолчанию длина последовательности.
В большинстве случаев, указание подпоследовательности допускается исключительно
ради эффективности. Вместо этого можно было бы просто вызвать
\cdf{subseq}. Однако, следует отметить, что операция, которая вычисляет индексы
для подпоследовательности, возвращает индексы исходной последовательности, а не
подпоследовательности:
\begin{lisp}
(position \#{\Xbackslash}b "foobar" \cd{:start} 2 \cd{:end} 5) \EV\ 3 \\
(position \#{\Xbackslash}b (subseq "foobar" 2 5)) \EV\ 1
\end{lisp}
Если в операции участвует две последовательности, тогда ключевые параметры
\cd{:start1}, \cd{:end1}, \cd{start2} и \cd{:end2} используются для указания
отдельных подпоследовательностей для каждой последовательности.

\begin{newer}
X3J13 voted in June 1988 \issue{SUBSEQ-OUT-OF-BOUNDS}
(and further clarification was voted in January 1989
\issue{RANGE-OF-START-AND-END-PARAMETERS})
to specify that these rules apply not
only to all built-in functions that have keyword parameters named
\cd{:start}, \cd{:start1}, \cd{:start2}, \cd{:end}, \cd{:end1},
or \cd{:end2} but also to functions such as \cdf{subseq}
that take required or optional parameters that are documented
as being named \emph{start} or \emph{end}.
\begin{itemize}
\item A ``start'' argument must always be a non-negative integer and
defaults to zero if not supplied; it is not permissible to pass \cdf{nil}
as a ``start'' argument.
\item An ``end'' argument must be either a
non-negative integer or \cdf{nil} (which indicates the end of the
sequence) and defaults to \cdf{nil}
if not supplied; therefore supplying \cdf{nil} is equivalent to
not supplying such an argument.
\item If the ``end'' argument is an integer, it must be no greater than the
active length of the corresponding sequence
(as returned by the function \cdf{length}).
\item The default value for the ``end'' argument is the active length
of the corresponding sequence.
\item The ``start'' value (after defaulting, if necessary) must not be greater than the
corresponding ``end'' value (after defaulting, if necessary).
\end{itemize}
This may be summarized as follows.
Let \emph{x} be the sequence within which indices are to be considered.  Let \emph{s} be
the ``start'' argument for that sequence of any standard function,
whether explicitly specified or defaulted, through omission, to
zero.  Let \emph{e} be the ``end'' argument for that sequence
of any standard function, whether explicitly specified or defaulted, through
omission or an explicitly passed \cdf{nil} value, to the active length of \emph{x}, as
returned by \cdf{length}.  Then it is an error if the test
\cd{(<=~0~\emph{s}~\emph{e}~(length \emph{x}))}
is not true.
\end{newer}

Для некоторых функций, в частности \cdf{remove} и \cdf{delete}, ключевой
параметр \cd{:count} используется для указания, как много подходящих элементов
должны обрабатываться. Если он равен {\false} или не указан, обрабатываются все
подходящие элементы.

В следующих описаниях функций, элемент \emph{x} последовательности
<<удовлетворяет условию>>, если любое из следующих выражений верно:
\begin{itemize}
\item
Была вызвана базовая функция, 
функция \emph{testfn} заданная в параметре \cd{:test}, и 
выражение \cd{(funcall \emph{testfn} \emph{item} (\emph{keyfn} \emph{x}))}
истинно.

\item
Была вызвана базовая функция, 
функция \emph{testfn}  заданная в параметре \cd{:test-not}, и
выражение \cd{(funcall \emph{testfn} \emph{item} (\emph{keyfn} \emph{x}))}
ложно.

\item
Была вызвана функция \cdf{-if}, и выражение
\cd{(funcall \emph{predicate} (\emph{keyfn} \emph{x}))} истинно.

\item
Была вызвана функция \cdf{-if-not}, и выражение
\cd{(funcall \emph{predicate} (\emph{keyfn} \emph{x}))} ложно
\end{itemize}
В каждом случае, функция \emph{keyfn} является значением параметра \cd{:key}
(по-умолчанию функцией эквивалентности). Для примера, смотрите \cdf{remove}.

В следующих описаниях функций
два элемента \emph{x} и \emph{y}, взятых из последовательности,
<<эквивалентны>>, если одно из следующих выражений верно:
\begin{itemize}
\item
Функция \emph{testfn} указана в параметре \cd{:test}, и 
выражение
\cd{(funcall \emph{testfn} (\emph{keyfn} \emph{x}) (\emph{keyfn}  \emph{y}))}
истинно.

\item
Функция \emph{testfn} указана в параметре \cd{:test-not}, и 
выражение
\cd{(funcall \emph{testfn} (\emph{keyfn} \emph{x}) (\emph{keyfn} \emph{y}))}
ложно.
\end{itemize}
Для примера, смотрите \cdf{search}.

\begin{newer}
X3J13 voted in June 1988 \issue{FUNCTION-TYPE} to allow the \emph{testfn}
or \cdf{predicate}
to be only of type \cdf{symbol} or \cdf{function}; a lambda-expression
is no longer acceptable as a functional argument.  One must use the
\cdf{function} special operator or the abbreviation \cd{\#'} before
a lambda-expression that appears as an  explicit argument form.
\end{newer}

Вы можете рассчитывать на порядок передачи аргументов в функцию \emph{testfn}.
Это позволяет использовать некоммутативную функцию проверки в стиле предиката.
Порядок аргументов в функцию \emph{testfn} соответствует порядку, в котором эти
аргументы (или последовательности их содержащие) были переданы в рассматриваемую
функцию.
Если рассматриваемая функция передаёт два элемента из одной последовательности в
функцию \emph{testfn}, то аргументы передаются в том же порядке, в котором были
в последовательности.

Если функция должна создать и вернуть новый вектор, то она всегда возвращает
\emph{простой} вектор (смотрите раздел~\ref{ARRAY-TYPE-SECTION}).
Таким же образом, любые создаваемые строки будут простыми строками.

\begin{defun}[Function]
complement fn

Returns a function whose value is the same as that of \cdf{not}
applied to the result of applying the function \emph{fn} to the same
arguments.  One could define \cdf{complement} as follows:

Данная функция возвращает другую функцию, значение которой является
\emph{отрицанием} функции \emph{fn}. Функцию \cdf{complement} можно было
определить так:
\begin{lisp}
(defun complement (fn) \\*
~~\#'(lambda (\&rest arguments) \\*
~~~~~~(not (apply fn arguments))))
\end{lisp}

One intended use of \cdf{complement} is to supplant the use of
\cd{:test-not} arguments and \cdf{-if-not} functions.

Можно использовать функцию \cdf{complement} для эмуляции \cd{-if-not} функций
или аргумента \cd{:test-not}.
\begin{lisp}
(remove-if-not \#'virtuous senators) {\EQ} \\*
~~~(remove-if (complement \#'virtuous) senators) \\
\\
(remove-duplicates telephone-book \\*
~~~~~~~~~~~~~~~~~~~:test-not \#'mismatch) {\EQ} \\*
~~~(remove-duplicates telephone-book \\*
~~~~~~~~~~~~~~~~~~~~~~:test (complement \#'mismatch))
\end{lisp}
\end{defun}

\section{Простые функции для последовательностей}

Большинство следующих функций выполняют простые операции над одиночными
последовательностями. \cdf{make-sequence} создаёт новую последовательность.

\begin{defun}[Функция]
elt sequence index

Функция возвращает элемент последовательности \emph{sequence}, указанный
индексом \emph{index}. Индекс должен быть неотрицательным целым числом меньшим
чем длина последовательности \emph{sequence}. Длина последовательности, в свою
очередь, вычисляется функцией \cdf{length}.
Первый элемент последовательности имеет индекс \cd{0}.

(Следует отметить, что \cdf{elt} учитывает указатель заполнения для векторов,
его имеющих. Для доступа ко всем элементам таких векторов используется функция
для массивов \cdf{aref}.)

Для изменения элемента последовательности может использоваться функция
\cdf{setf} в связке с \cdf{elt}.
\end{defun}

\begin{defun}[Функция]
subseq sequence start &optional end

Данная функция возвращает подпоследовательность последовательности
\emph{sequence} начиная с позиции \emph{start} и заканчивая позицией
\emph{end}. 
\cdf{subseq} \emph{всегда} создаёт новую последовательность для
результата. Возвращённая подпоследовательность всегда имеет тот же тип, что и
исходная последовательность.

Для изменения подпоследовательности элементов можно использовать \cdf{setf} в
связке с \cdf{subseq}. Смотрите также \cdf{replace}.
\end{defun}

\begin{defun}[Функция]
copy-seq sequence

Функция создаёт копию аргумента \emph{sequence}. Результат равен \cdf{equalp}
аргументу. Однако, результат не не равен \cdf{eq} аргументу.
\begin{lisp}
(copy-seq \emph{x}) \EQ\ (subseq \emph{x} 0)
\end{lisp}
но имя \cdf{copy-seq} лучше передаёт смысл операции.
\end{defun}

\begin{defun}[Функция]
length sequence

Функция возвращает количество элементов последовательности
\emph{sequence}. Результат является неотрицательным целым числом.
Если последовательность является вектором с указателем заполнения, возвращается
<<активная длина>>, то есть длина заданная указателем заполнения (смотрите
раздел~\ref{FILL-POINTER}).
\end{defun}

\begin{defun}[Функция]
reverse sequence

Результатом является новая последовательность такого же типа, что и
последовательность \emph{sequence}. Элементы итоговой последовательности
размещаются в обратном порядке.
Аргумент при этом не модифицируется.
\end{defun}

\begin{defun}[Функция]
nreverse sequence

Результатом является последовательность, содержащая те же элементы, что и
последовательность \emph{sequence}, но в обратном порядке. Аргумент может быть
уничтожен и использован для возврата результата. Результат может быть или не
быть равен \cdf{eq} аргументу. Поэтому обычно используют явное присваивание
\cd{(setq x (nreverse x))}, так как просто \cd{(nreverse x)} не гарантирует
возврат преобразованной последовательности в \cdf{x}.

\begin{newer}
X3J13 voted in March 1989 \issue{REMF-DESTRUCTION-UNSPECIFIED}
to clarify the permissible side effects of certain operations.
When the \emph{sequence} is a list,
\cdf{nreverse} is permitted to perform a \cdf{setf} on any part,
\emph{car} or \emph{cdr}, of the top-level list structure of that list.
When the \emph{sequence} is an array,
\cdf{nreverse} is permitted to re-order the elements of the given array
in order to produce the resulting array.
\end{newer}
\end{defun}

\begin{defun}[Функция]
make-sequence type size &key :initial-element

Данная функция возвращает последовательность типа \emph{type} и длинной
\emph{size}, каждый из элементов которой содержит значение аргумента
\cd{:initial-element}.
Если указан аргумент \cd{:initial-element}, он должен принадлежать типу элемента
последовательности \emph{type}.
Например:
\begin{lisp}
(make-sequence '(vector double-float) \\*
~~~~~~~~~~~~~~~100 \\*
~~~~~~~~~~~~~~~:initial-element 1d0)
\end{lisp}
Если аргумент \cd{:initial-element} не указан, тогда последовательность
инициализируется алгоритмом реализации.

\begin{new}
X3J13 voted in January 1989
\issue{ARGUMENTS-UNDERSPECIFIED}
to clarify that the \emph{type} argument
must be a type specifier, and the \emph{size} argument
must be a non-negative integer less than the value of
\cdf{array-dimension-limit}.
\end{new}

\begin{newer}
X3J13 voted in June 1989 \issue{SEQUENCE-TYPE-LENGTH} to specify that
\cdf{make-sequence} should signal an error if the sequence \emph{type} specifies the
number of elements and the \emph{size} argument is different.
\end{newer}

\begin{newer}
X3J13 voted in March 1989 \issue{CHARACTER-PROPOSAL}
to specify that if \emph{type} is \cdf{string}, the result is the same
as if \cdf{make-string} had been called with the same \emph{size}
and \cd{:initial-element} arguments.
\end{newer}
\end{defun}

\section{Объединение, отображение и приведение последовательностей}

Функции в этом разделе могут обрабатывать произвольное количество
последовательностей за исключением функции \cdf{reduce}, которая была включена
сюда из-за концептуальной связи с функциями отображения.

\begin{defun}[Функция]
concatenate result-type &rest sequences

Результатом является новая последовательность, которая содержит по порядку все
элементы указанных последовательностей. Результат не содержит связей с
аргументами (в этом \cdf{concatenate} отличается от \cdf{append}). Тип
результата указывается в аргументе \emph{result-type}, который должен быть
подтипом \cdf{sequence}, как для функции \cdf{coerce}.
Необходимо, чтобы элементы исходных последовательностей принадлежали указанному
типу \emph{result-type}.

Если указана только одна последовательность, и она имеет тот же тип, что
указан в \emph{result-type}, \cdf{concatenate} всё равно копирует аргумент и
возвращает новую последовательность. Если вам не требуется копия, а просто
необходимо преобразовать тип последовательности, лучше использовать
\cdf{coerce}.

\begin{newer}
X3J13 voted in June 1989 \issue{SEQUENCE-TYPE-LENGTH} to specify that
\cdf{concatenate} should signal an error if the sequence type specifies the
number of elements and the sum of the argument lengths is different.
\end{newer}
\end{defun}

\begin{defun}[Функция]
map result-type function sequence &rest more-sequences

Функция \emph{function} должна принимать только аргументов, сколько
последовательностей было передано в \cdf{map}. 
Результат функции \cdf{map} --- последовательность, элементы которой являются
результатами применения функции \emph{function} к соответствующим элементам
исходных последовательностей. Длина итоговой последовательности равна длине
самой короткой исходной.

Если функция \emph{function} имеет побочные эффекты, она может рассчитывать на
то, что сначала будет вызвана со всеми элементами в \cd{0}-ой позиции, затем со
всеми в \cd{1}-ой и так далее.

Тип итоговой последовательности указывается в аргументе \emph{result-type} (и
должен быть подтипом \cdf{sequence}).
Кроме того, для типа можно указать {\nil}, и это означает, что результата быть
не должно. В таком случае функция \emph{function} вызывается только для побочных
эффектов, и \cdf{map} возвращает {\nil}. В этом случае \cdf{map} похожа на
\cdf{mapc}.

\begin{newer}
X3J13 voted in June 1989 \issue{SEQUENCE-TYPE-LENGTH} to specify that
\cdf{map} should signal an error if the sequence type specifies the number of
elements and the minimum of the argument lengths is different.
\end{newer}

\begin{new}
X3J13 voted in January 1989
\issue{MAPPING-DESTRUCTIVE-INTERACTION}
to restrict user side effects; see section~\ref{STRUCTURE-TRAVERSAL-SECTION}.
\end{new}

Пользователь ограничен в создании побочных действий так, как это описано в
разделе~\ref{STRUCTURE-TRAVERSAL-SECTION}

Например:
\begin{lisp}
(map 'list \#'- '(1 2 3 4)) \EV\ (-1 -2 -3 -4) \\
(map 'string \\
~~~~~\#'(lambda (x) (if (oddp x) \#{\Xbackslash}1 \#{\Xbackslash}0)) \\
~~~~~'(1 2 3 4)) \\
~~~\EV\ "1010"
\end{lisp}
\end{defun}


\begin{defun}[Функция]
map-into result-sequence function &rest sequences

Функция \cdf{map-into} вычисляет последовательность с помощью применения функции
\emph{function} к соответствующим элементам исходных последовательностей и
помещает результат в последовательность \emph{result-sequence}.
Функция возвращается \emph{result-sequence}.

Аргументы \emph{result-sequence} и все \emph{sequences} должны быть списками
или векторами (одномерными массивами).
Функция \emph{function} должна принимать столько аргументов, сколько было
указано исходных последовательностей.
Если \emph{result-sequence} и другие аргументы \emph{sequences} не одинаковой
длины, цикл закончится на самой короткой последовательности. Если
\emph{result-sequence} является вектором с указателем заполнения, то этот
указатель во время цикла игнорируется, а после завершения устанавливается на то
количество элементов, которые были получены от функции \emph{function}.

Если функция \emph{function} имеет побочные эффекты, она может рассчитывать на
то, что сначала будет вызвана со всеми элементами в \cd{0}-ой позиции, затем со
всеми в \cd{1}-ой и так далее.

Функция \cdf{map-into} отличается от \cdf{map} тем, что модифицирует уже
имеющуюся последовательность, а не создаёт новую. Кроме того, \cdf{map-into}
может быть вызвана только с двумя аргументами (\emph{result-sequence} и
\emph{function}), тогда как \cdf{map} требует как минимум три параметра.

Если \emph{result-sequence} является \cdf{nil}, \cdf{map-into} немедленно
возвращает \cdf{nil}, потому что длина \cdf{nil} последовательности равняется
нулю.
\end{defun}


\begin{defun}[Функция]
some predicate sequence &rest more-sequences \\
every predicate sequence &rest more-sequences \\
notany predicate sequence &rest more-sequences \\
notevery predicate sequence &rest more-sequences

Всё это предикаты.
Функция \emph{predicate} должна принимать столько аргументов, сколько было
указано последовательностей. Функция \emph{predicate} сначала применяется к
элементам \cd{0}-ой позиции, затем, возможно, к элементам \cd{1}-ой позиции, и
так далее, до тех пор пока не сработает условие завершения цикла или не
закончится одна из последовательностей \emph{sequences}.

Если \emph{predicate} имеет побочные эффекты, он может рассчитывать на
то, что сначала будет вызвана со всеми элементами в \cd{0}-ой позиции, затем со 
всеми в \cd{1}-ой и так далее.

\cdf{some} завершается, как только применение \emph{predicate} вернёт не-{\false}
значение. Тогда \cdf{some} возвращает значение, на котором произошла остановка.
Если цикл достиг конца последовательности и ни одно применение \emph{predicate}
не вернуло не-{\false}, тогда возвращается значение {\false}.
Таким образом, можно сказать, что предикат \cdf{some} истинен, если
\emph{какой-либо (some)} вызов \emph{predicate} истинен.

\cdf{every} возвращает {\false}, как только применение \emph{predicate}
возвращает {\false}.
Если цикл достиг конца последовательности, \cdf{every} возвращает не-{\false}.
Таким образом, можно сказать, что предикат \cdf{every} истинен, если
\emph{каждый (every)} вызов \emph{predicate} истинен.

\cdf{notany} возвращает {\false}, как только применение \emph{predicate} 
возвращает не-{\false}.
Если цикл достиг конца последовательности, \cdf{notany} возвращает не-{\false}.
Таким образом, можно сказать, что предикат \cdf{notany} истинен, если
\emph{ни какой (notany)} вызов \emph{predicate} не истинен.

\cdf{notevery} возвращает не-{\false}, как только применение \emph{predicate} 
возвращает {\false}.
Если цикл достиг конца последовательности, \cdf{notevery} возвращает {\false}.
Таким образом, можно сказать, что предикат \cdf{notevery} истинен, если
\emph{ни каждый (notany)} вызов \emph{predicate} истинен.

\begin{new}
X3J13 voted in January 1989
\issue{MAPPING-DESTRUCTIVE-INTERACTION}
to restrict user side effects; see section~\ref{STRUCTURE-TRAVERSAL-SECTION}.
\end{new}

Пользователь ограничен в создании побочных действий так, как это описано в
разделе~\ref{STRUCTURE-TRAVERSAL-SECTION}
\end{defun}

\begin{defun}[Функция]
reduce function sequence &key :from-end :start :end :initial-value

Функция \cdf{reduce} объединяет все элементы последовательности использую
заданную (бинарную binary) операцию. Например \cdf{+} может суммировать все
элементы последовательности.

Указанная подпоследовательность последовательности \emph{sequence} объединяется
или <<редуцируется>> с помощью функции \emph{function}, которая должна принимать
два аргумента. Приведение (редуцирование, объединение) является
левоассоциативным, если только \cd{:from-end} не содержит истину, в последнем
случае операция становиться правоассоциативной. По-умолчанию
\cd{:from-end} равно {\nil}.
Если задан аргумент \cd{:initial-value}, то он логически помещается перед
подпоследовательностью (или после в случае истинности \cd{:from-end}) и
включается в операцию редуцирования.

Если указанная подпоследовательность содержит только один элемент и параметр
\cd{:initial-value} не задан, тогда этот элемент возвращается, и функция
\emph{function} ни разу не вызывается.
Если заданная подпоследовательность пуста, и задан параметр \cd{:initial-value},
тогда возвращается \cd{:initial-value}, и функция \emph{function} не вызывается.

Если заданная подпоследовательность пуста, и параметр \cd{:initial-value} не задан,
тогда функция \emph{function} вызывается без аргументов, и \cdf{reduce}
возвращает то, что вернула эта функция. Это единственное исключение из правила о
том, что функция \emph{function} вызывается с двумя аргументами.

\begin{lisp}
(reduce \#'+ '(1 2 3 4)) \EV\ 10 \\
(reduce \#'- '(1 2 3 4)) \EQ\ (- (- (- 1 2) 3) 4) \EV\ -8 \\
(reduce \#'- '(1 2 3 4) :from-end t)~~~~~;\textrm{Альтернативная сумма} \\
~~~\EQ\ (- 1 (- 2 (- 3 4))) \EV\ -2 \\
(reduce \#'+ '()) \EV\ 0 \\
(reduce \#'+ '(3)) \EV\ 3 \\
(reduce \#'+ '(foo)) \EV\ foo \\
(reduce \#'list '(1 2 3 4)) \EV\ (((1 2) 3) 4) \\
(reduce \#'list '(1 2 3 4) :from-end t) \EV\ (1 (2 (3 4))) \\
(reduce \#'list '(1 2 3 4) :initial-value 'foo) \\
~~~\EV\ ((((foo 1) 2) 3) 4) \\
(reduce \#'list '(1 2 3 4) \\
~~~~~~~~:from-end t :initial-value 'foo) \\
~~~\EV\ (1 (2 (3 (4 foo))))
\end{lisp}
Если функция \emph{function} имеет побочные эффекта, можно положится на порядок
вызовов функции так, как было продемонстрировано выше.

Имя <<reduce>> было позаимствовано из {APL}.

\begin{new}
X3J13 voted in March 1988 \issue{REDUCE-ARGUMENT-EXTRACTION}
to extend the \cdf{reduce} function to take
an additional keyword argument named \cd{:key}.  As usual, this argument
defaults to the identity function.  The value of this argument must be
a function that accepts at least one argument.  This function is applied once
to each element of the
sequence that is to participate in the reduction operation, in the order
implied by the \cd{:from-end} argument; the values returned by this
function are combined by the reduction \emph{function}.
However, the \cd{:key} function is \emph{not} applied
to the \cd{:initial-value} argument (if any).
\end{new}

\begin{new}
X3J13 voted in January 1989
\issue{MAPPING-DESTRUCTIVE-INTERACTION}
to restrict user side effects; see section~\ref{STRUCTURE-TRAVERSAL-SECTION}.
\end{new}

Пользователь ограничен в создании побочных действий так, как это описано в
разделе~\ref{STRUCTURE-TRAVERSAL-SECTION}
\end{defun}

\section{Модификация последовательностей}

Каждая из этих функций или модифицирует последовательность, или возвращает
модифицированную копию.

\begin{defun}[Функция]
fill sequence item &key :start :end

Функция модифицирует последовательность, заменяя каждый элемент
подпоследовательности, обозначенной с помощью параметров \cd{:start} и
\cd{:end}, объектом \emph{item}. \emph{item} может быть любым Lisp'овыми 
объектом, подходящим для типа элементов последовательности \emph{sequence}. 
Объект \emph{item} сохраняется в всех указанных компонентах
последовательности \emph{sequence}, начиная с позиции \cd{:start} (по-умолчанию
равна 0) и заканчивая невключительно позицией \cd{:end} (по-умолчанию равна
длине последовательности). \cd{fill} возвращает изменённую последовательность.
Например:
\begin{lisp}
(setq x (vector 'a 'b 'c 'd 'e)) \EV\ \#(a b c d e) \\
(fill x 'z \cd{:start} 1 \cd{:end} 3) \EV\ \#(a z z d e) \\
~~\textrm{and now} x \EV\ \#(a z z d e) \\
(fill x 'p) \EV\ \#(p p p p p) \\
~~\textrm{and now} x \EV\ \#(p p p p p)
\end{lisp}
\end{defun}

\begin{defun}[Функция]
replace sequence1 sequence2 &key :start1 :end1 :start2~:end2

Функция модифицирует последовательность \emph{sequence1} копируя в неё элементы
из последовательности \emph{sequence2}. Элементы \emph{sequence2} для
принадлежать типу элементов \emph{sequence1}. Подпоследовательность
\emph{sequence2}, указанная с помощью параметров \cd{:start2} и \cd{:end2},
копируется в подпоследовательность \emph{sequence2}, указанную с помощью
параметров \cd{:start1} и \cd{:end1}. Аргументы \cd{:start1} и \cd{:start2}
по-умолчанию равны нулю. Аргументы \cd{:end1} и \cd{:end2} по-умолчанию
{\false}, что означает длины соответствующих последовательностей.
Если указанные подпоследовательности имеют не равные длины, тогда для наиболее
короткая из них определяет копируемые элементы. Лишние элементы
последовательностей функцией не обрабатываются.
Количество копируемых элементов можно вычислить так:
\begin{lisp}
(min (- \emph{end1} \emph{start1}) (- \emph{end2} \emph{start2}))
\end{lisp}
Значением функции \cdf{replace} является модифицированная последовательность
\emph{sequence1}.

Если обе последовательности равны (\cdf{eq}), и интервал для модификации
перекрывается с исходным интервалом, тогда поведение такое, как если бы все
исходные данные сохранялись во временном месте, а затем сохранялись в итоговый
интервал.
Однако, если последовательности не равны, но итоговый интервал пересекается с
исходным (возможно, при использовании \emph{соединённых} массивов), тогда
после выполнения \cdf{replace}, итоговое содержимое не определено.
\end{defun}

\begin{defun}[Функция]
remove item sequence &key :from-end :test :test-not :start :end :count :key \\
remove-if predicate sequence &key :from-end :start :end :count :key \\
remove-if-not predicate sequence &key :from-end :start :end :count :key

Результатом является последовательность того же типа, что и последовательность
\emph{sequence}. Однако, итоговая последовательность не будет содержать элементы
в интервале \cd{:start}-\cd{:end}, которые удовлетворяли условию.
Результат является копией входящей последовательности \emph{sequence} без
исключённых элементов. Неудалённые элементы сохраняются в таком же порядке.

Если указан аргумент \cd{:count}, то удаляться будет только это количество. Если
количество удаляемых элементов превышает параметр, тогда будут удалены только
самые левые в количестве равном \cd{:count}.

\begin{new}
X3J13 voted in January 1989
\issue{RANGE-OF-COUNT-KEYWORD}
to clarify that the \cd{:count} argument must be either \cdf{nil}
or an integer, and that supplying a negative integer produces the
same behavior as supplying zero.
\end{new}

Если при использовании \cd{:count} для параметра
\cd{:from-end} указано не-{\false} значение, тогда удаление элементов будет
происходить справа в количестве \cd{:count}.
Например:
\begin{lisp}
(remove 4 '(1 2 4 1 3 4 5)) \EV\ (1 2 1 3 5) \\
(remove 4 '(1 2 4 1 3 4 5) \cd{:count} 1) \EV\ (1 2 1 3 4 5) \\
(remove 4 '(1 2 4 1 3 4 5) \cd{:count} 1 \cd{:from-end} t) \\
~~~\EV\ (1 2 4 1 3 5) \\
(remove 3 '(1 2 4 1 3 4 5) \cd{:test} \#'>) \EV\ (4 3 4 5) \\
(remove-if \#'oddp '(1 2 4 1 3 4 5)) \EV\ (2 4 4) \\
(remove-if \#'evenp '(1 2 4 1 3 4 5) \cd{:count} 1 \cd{:from-end} t) \\
~~~\EV\ (1 2 4 1 3 5)
\end{lisp}
Результат \cdf{remove} может быть \emph{соединён} с исходной
последовательностью. Также результат может быть равен \cdf{eq} исходной
последовательности, если ни одного элементы не было удалено.

\begin{new}
X3J13 voted in January 1989
\issue{MAPPING-DESTRUCTIVE-INTERACTION}
to restrict user side effects; see section~\ref{STRUCTURE-TRAVERSAL-SECTION}.
\end{new}

Пользователь ограничен в создании побочных действий так, как это описано в
разделе~\ref{STRUCTURE-TRAVERSAL-SECTION}
\end{defun}

\begin{defun}[Функция]
delete item sequence &key :from-end :test :test-not :start~:end~:count~:key \\
delete-if predicate sequence &key :from-end :start~:end~:count~:key \\
delete-if-not predicate sequence &key :from-end :start~:end~:count~:key

Данная функция в отличие от \cdf{remove} модифицирует исходную
последовательность. Результатом является последовательность того же типа, что и последовательность
\emph{sequence}. Однако, итоговая последовательность не будет содержать элементы
в интервале \cd{:start}-\cd{:end}, которые удовлетворяли условию.
Результат является копией входящей последовательности \emph{sequence} без
исключённых элементов. Неудалённые элементы сохраняются в таком же порядке.
Последовательность \emph{sequence} может быть модифицирована, поэтому результат
может быть равен \cdf{eq} или нет исходной последовательности.

Если указан аргумент \cd{:count}, то удаляться будет только это количество. Если
количество удаляемых элементов превышает параметр, тогда будут удалены только
самые левые в количестве равном \cd{:count}.

\begin{new}
X3J13 voted in January 1989
\issue{RANGE-OF-COUNT-KEYWORD}
to clarify that the \cd{:count} argument must be either \cdf{nil}
or an integer, and that supplying a negative integer produces the
same behavior as supplying zero.
\end{new}

Если при использовании \cd{:count} для параметра
\cd{:from-end} указано не-{\false} значение, тогда удаление элементов будет
происходить справа в количестве \cd{:count}.
Например:
\begin{lisp}
(delete 4 '(1 2 4 1 3 4 5)) \EV\ (1 2 1 3 5) \\
(delete 4 '(1 2 4 1 3 4 5) \cd{:count} 1) \EV\ (1 2 1 3 4 5) \\
(delete 4 '(1 2 4 1 3 4 5) \cd{:count} 1 \cd{:from-end} t) \\
~~~\EV\ (1 2 4 1 3 5) \\
(delete 3 '(1 2 4 1 3 4 5) \cd{:test} \#'>) \EV\ (4 3 4 5) \\
(delete-if \#'oddp '(1 2 4 1 3 4 5)) \EV\ (2 4 4) \\
(delete-if \#'evenp '(1 2 4 1 3 4 5) \cd{:count} 1 \cd{:from-end} t) \\
~~~\EV\ (1 2 4 1 3 5)
\end{lisp}

\begin{new}
X3J13 voted in January 1989
\issue{MAPPING-DESTRUCTIVE-INTERACTION}
to restrict user side effects; see section~\ref{STRUCTURE-TRAVERSAL-SECTION}.
\end{new}

Пользователь ограничен в создании побочных действий так, как это описано в
разделе~\ref{STRUCTURE-TRAVERSAL-SECTION}

\begin{newer}
X3J13 voted in March 1989 \issue{REMF-DESTRUCTION-UNSPECIFIED}
to clarify the permissible side effects of certain operations.
When the \emph{sequence} is a list,
\cdf{delete} is permitted to perform a \cdf{setf} on any part,
\emph{car} or \emph{cdr}, of the top-level list structure of that list.
When the \emph{sequence} is an array,
\cdf{delete} is permitted to alter the dimensions of the given array
and to slide some of its elements into new positions without permuting them
in order to produce the resulting array.

Furthermore, \cd{(delete-if \emph{predicate} \emph{sequence}~...)}
is required to behave exactly like
\begin{lisp}
(delete nil \emph{sequence} \\*
~~~~~~~~:test \#'(lambda (unused item) \\*
~~~~~~~~~~~~~~~~~~~(declare (ignore unused)) \\*
~~~~~~~~~~~~~~~~~~~(funcall \emph{predicate} item)) \\*
~~~~~~~~...)
\end{lisp}
\end{newer}
\end{defun}

\begin{defun}[Функция]
remove-duplicates sequence &key :from-end :test :test-not :start :end :key \\
delete-duplicates sequence &key :from-end :test :test-not :start :end :key

Функция попарно сравнивает элементы последовательности \emph{sequence}, и если
они равны, тогда первый из них удаляется (если параметр \cd{:from-end} равен
истине, то удаляется последний).
Результат является последовательностью того же типа, что и исходная, с
удалёнными повторяющимися элементами. Порядок следования элементов в итоге такой
же как в исходной последовательности.

\cdf{remove-duplicates} является не модифицирующей версией этой операции.
Результат \cdf{remove-duplicates} может быть \emph{соединён} с исходной
последовательностью. Также результат может быть равен \cdf{eq} исходной
последовательности, если ни одного элементы не было удалено.

\cdf{delete-duplicates} может модифицировать аргумент \emph{sequence}.

Например:
\begin{lisp}
(remove-duplicates '(a b c b d d e)) \EV\ (a c b d e) \\
(remove-duplicates '(a b c b d d e) \cd{:from-end} t) \EV\ (a b c d e) \\
(remove-duplicates '((foo \#{\Xbackslash}a) (bar \#{\Xbackslash}\%) (baz \#{\Xbackslash}A)) \\
~~~~~~~~~~~~~~~~~~~\cd{:test} \#'char-equal \cd{:key} \#'cadr) \\
~~~\EV\ ((bar \#{\Xbackslash}\%) (baz \#{\Xbackslash}A)) \\
(remove-duplicates '((foo \#{\Xbackslash}a) (bar \#{\Xbackslash}\%) (baz \#{\Xbackslash}A)) \\
~~~~~~~~~~~~~~~~~~~\cd{:test} \#'char-equal \cd{:key} \#'cadr \cd{:from-end} t) \\
~~~\EV\ ((foo \#{\Xbackslash}a) (bar \#{\Xbackslash}\%))
\end{lisp}

Эти функции полезны для преобразования последовательности в каноническую форму
представления множества.

\begin{new}
X3J13 voted in January 1989
\issue{MAPPING-DESTRUCTIVE-INTERACTION}
to restrict user side effects; see section~\ref{STRUCTURE-TRAVERSAL-SECTION}.
\end{new}

Пользователь ограничен в создании побочных действий так, как это описано в
разделе~\ref{STRUCTURE-TRAVERSAL-SECTION}

\begin{newer}
X3J13 voted in March 1989 \issue{REMF-DESTRUCTION-UNSPECIFIED}
to clarify the permissible side effects of certain operations.
When the \emph{sequence} is a list,
\cdf{delete-duplicates} is permitted to perform a \cdf{setf} on any part,
\emph{car} or \emph{cdr}, of the top-level list structure of that list.
When the \emph{sequence} is an array,
\cdf{delete-duplicates} is permitted to alter the dimensions of the given array
and to slide some of its elements into new positions without permuting them
in order to produce the resulting array.
\end{newer}
\end{defun}

\begin{defun}[Функция]
substitute newitem olditem sequence &key :from-end :test :test-not :start :end :count :key \\
substitute-if newitem test sequence &key :from-end :start~:end :count :key \\
substitute-if-not newitem test sequence &key :from-end :start :end :count :key

Результатом является последовательность такого же типа что и исходная
\emph{sequence}
за исключением того, что удовлетворяющие условию элементы в интервале
\cd{:start}-\cd{:end} будут заменены на объект \emph{newitem}. Эта операция
создаёт копию исходной последовательности с некоторыми изменёнными элементами.

Если указан аргумент \cd{:count}, то изменяться будет только это количество
элементов. Если 
количество изменяемых элементов превышает параметр, тогда будут удалены только
самые левые в количестве равном \cd{:count}.

\begin{new}
X3J13 voted in January 1989
\issue{RANGE-OF-COUNT-KEYWORD}
to clarify that the \cd{:count} argument must be either \cdf{nil}
or an integer, and that supplying a negative integer produces the
same behavior as supplying zero.
\end{new}

Если при использовании \cd{:count} для параметра
\cd{:from-end} указано не-{\false} значение, тогда изменение элементов будет
происходить справа в количестве \cd{:count}.
Например:
\begin{lisp}
(substitute 9 4 '(1 2 4 1 3 4 5)) \EV\ (1 2 9 1 3 9 5) \\
(substitute 9 4 '(1 2 4 1 3 4 5) \cd{:count} 1) \EV\ (1 2 9 1 3 4 5) \\
(substitute 9 4 '(1 2 4 1 3 4 5) \cd{:count} 1 \cd{:from-end} t) \\
~~~\EV\ (1 2 4 1 3 9 5) \\
(substitute 9 3 '(1 2 4 1 3 4 5) \cd{:test} \#'>) \EV\ (9 9 4 9 3 4 5) \\
(substitute-if 9 \#'oddp '(1 2 4 1 3 4 5)) \EV\ (9 2 4 9 9 4 9) \\
(substitute-if 9 \#'evenp '(1 2 4 1 3 4 5) \cd{:count} 1 \cd{:from-end} t) \\
~~~\EV\ (1 2 4 1 3 9 5)
\end{lisp}
Результат \cdf{substitute} может быть \emph{соединён} с исходной
последовательностью. Также результат может быть равен \cdf{eq} исходной
последовательности, если ни одного элементы не было изменено.

Смотрите также \cdf{subst}, которая осуществляет замену в древовидной структуре.

\begin{new}
X3J13 voted in January 1989
\issue{MAPPING-DESTRUCTIVE-INTERACTION}
to restrict user side effects; see section~\ref{STRUCTURE-TRAVERSAL-SECTION}.
\end{new}

Пользователь ограничен в создании побочных действий так, как это описано в
разделе~\ref{STRUCTURE-TRAVERSAL-SECTION}
\end{defun}

\begin{defun}[Функция]
nsubstitute newitem olditem sequence &key :from-end :test :test-not :start :end :count :key \\
nsubstitute-if newitem test sequence &key :from-end :start~:end :count :key \\
nsubstitute-if-not newitem test sequence &key :from-end :start :end :count :key

Эта функция является деструктивным аналогом для \cdf{substitute}. Это значит,
что они модифицирует свой аргумент.
Результатом является последовательность такого же типа что и исходная
\emph{sequence}
за исключением того, что удовлетворяющие условию элементы в интервале
\cd{:start}-\cd{:end} будут заменены на объект \emph{newitem}.
Последовательность \emph{sequence} может быть модифицирована, поэтому результат
может быть равен \cdf{eq} или нет исходной последовательности.

Смотрите также \cdf{nsubst}, которая осуществляет деструктивную замену в
древовидной структуре.

\begin{new}
X3J13 voted in January 1989
\issue{MAPPING-DESTRUCTIVE-INTERACTION}
to restrict user side effects; see section~\ref{STRUCTURE-TRAVERSAL-SECTION}.
\end{new}

\begin{newer}
X3J13 voted in March 1989 \issue{REMF-DESTRUCTION-UNSPECIFIED}
to clarify the permissible side effects of certain operations.
When the \emph{sequence} is a list,
\cdf{nsubstitute} or \cdf{nsubstitute-if}
is required to perform a \cdf{setf} on any
\emph{car} of the top-level list structure of that list
whose old contents must be replaced with \emph{newitem}
but is forbidden to perform a \cdf{setf} on any \cdf{cdr} of the list.
When the \emph{sequence} is an array,
\cdf{nsubstitute} or \cdf{nsubstitute-if}
is required to perform a \cdf{setf} on any element of the array
whose old contents must be replaced with \emph{newitem}.
These functions, therefore, may successfully be
used solely for effect, the caller discarding the returned value
(though some programmers find this stylistically distasteful).
\end{newer}
\end{defun}

\section{Поиск элементов последовательностей}

Каждая из этих функций ищет в последовательности элемент, удовлетворяющий
некоторому условию.

\begin{defun}[Функция]
find item sequence &key :from-end :test :test-not :start~:end :key \\
find-if predicate sequence &key :from-end :start :end :key \\
find-if-not predicate sequence &key :from-end :start~:end~:key

Если последовательность \emph{sequence} содержит элемент, удовлетворяющий
условию, тогда возвращается первый найденный слева элемент, иначе возвращается
{\false}.

Если заданы параметры \cd{:start} и \cd{:end}, тогда поиск осуществляется в этом
интервале.

Если \cd{:from-end} указан в не-{\false}, тогда результатом функции будет
найденный элемент справа.

\begin{new}
X3J13 voted in January 1989
\issue{MAPPING-DESTRUCTIVE-INTERACTION}
to restrict user side effects; see section~\ref{STRUCTURE-TRAVERSAL-SECTION}.
\end{new}

Пользователь ограничен в создании побочных действий так, как это описано в
разделе~\ref{STRUCTURE-TRAVERSAL-SECTION}
\end{defun}

\begin{defun}[Функция]
position item sequence &key :from-end :test :test-not :start~:end~:key \\
position-if predicate sequence &key :from-end :start~:end~:key \\
position-if-not predicate sequence &key :from-end :start~:end~:key

Если последовательность \emph{sequence} содержит элемент, удовлетворяющий
условию, тогда возвращается позиция найденного элемента слева, иначе возвращается
{\false}.

Если заданы параметры \cd{:start} и \cd{:end}, тогда поиск осуществляется в этом
интервале. Однако возвращаемый индекс будет относиться ко всей
последовательности в целом.

Если \cd{:from-end} указан в не-{\false}, тогда результатом функции будет
индекс найденного элемент справа. (Однако, возвращаемый индекс, будет, как
обычно, принадлежать нумерации слева направо.)

\begin{new}
X3J13 voted in January 1989
\issue{MAPPING-DESTRUCTIVE-INTERACTION}
to restrict user side effects; see section~\ref{STRUCTURE-TRAVERSAL-SECTION}.
\end{new}

Пользователь ограничен в создании побочных действий так, как это описано в
разделе~\ref{STRUCTURE-TRAVERSAL-SECTION}
\end{defun}

Ниже приведены несколько примеров использования нескольких функций,
преимущественно \cdf{position-if} и \cdf{find-if}, для обработки строк.
Также используется \cdf{loop}.
\begin{lisp}
(defun debug-palindrome (s) \\*
~~(flet ((match (x) (char-equal (first x) (third x)))) \\*
~~~~(let* ((pairs (loop for c across s \\*
~~~~~~~~~~~~~~~~~~~~~~~~for j from 0 \\*
~~~~~~~~~~~~~~~~~~~~~~~~when (alpha-char-p c) \\*
~~~~~~~~~~~~~~~~~~~~~~~~~~collect (list c j))) \\*
~~~~~~~~~~~(quads (mapcar \#'append pairs (reverse pairs))) \\*
~~~~~~~~~~~(diffpos (position-if (complement \#'match) quads))) \\
~~~~~~(when diffpos \\*
~~~~~~~~(let* ((diff (elt quads diffpos)) \\*
~~~~~~~~~~~~~~~(same (find-if \#'match quads \\*
~~~~~~~~~~~~~~~~~~~~~~~~~~~~~~:start (+ diffpos 1)))) \\
~~~~~~~~~~(if same \\*
~~~~~~~~~~~~~~(format nil \\*
~~~~~~~~~~~~~~~~~~~~~~"/{\Xtilde}A/ (at {\Xtilde}D) is not the reverse of /{\Xtilde}A/" \\*
~~~~~~~~~~~~~~~~~~~~~~(subseq s (second diff) (second same)) \\*
~~~~~~~~~~~~~~~~~~~~~~(second diff) \\*
~~~~~~~~~~~~~~~~~~~~~~(subseq s (+ (fourth same) 1) \\*
~~~~~~~~~~~~~~~~~~~~~~~~~~~~~~~~(+ (fourth diff) 1))) \\*
~~~~~~~~~~~~~~"This palindrome is completely messed up!"))))))
\end{lisp}
Вот пример поведения этой функции
\begin{lisp}
(setq panama~~~~~;\textrm{Предполагаемый палиндром?} \\*
~~~~~~"A man, a plan, a canoe, pasta, heros, rajahs, \\*
~~~~~~~a coloratura, maps, waste, percale, macaroni, a gag, \\*
~~~~~~~a banana bag, a tan, a tag, a banana bag again \\*
~~~~~~~(or a camel), a crepe, pins, Spam, a rut, a Rolo, \\*
~~~~~~~cash, a jar, sore hats, a peon, a canal--Panama!")
\end{lisp}
\begin{lisp}
(debug-palindrome panama) \\*
~~\EV\ "/wast/ (at 73) is not the reverse of /, pins/" \\
\\
(replace panama "snipe" :start1 73)~~~~~;\textrm{Восстановить} \\*
~~\EV\ "A man, a plan, a canoe, pasta, heros, rajahs, \\*
~~~~~~~a coloratura, maps, snipe, percale, macaroni, a gag, \\*
~~~~~~~a banana bag, a tan, a tag, a banana bag again \\*
~~~~~~~(or a camel), a crepe, pins, Spam, a rut, a Rolo, \\*
~~~~~~~cash, a jar, sore hats, a peon, a canal--Panama!" \\
\\
(debug-palindrome panama) \EV\ nil~~~~~;\textrm{Первоклассный---истинный палиндром} \\
\\
(debug-palindrome "Rubber baby buggy bumpers") \\*
~~\EV\ "/Rubber / (at 0) is not the reverse of /umpers/" \\
\\
(debug-palindrome "Common Lisp: The Language") \\*
~~\EV\ "/Commo/ (at 0) is not the reverse of /guage/" \\
\\
(debug-palindrome "Complete mismatches are hard to find") \\*
~~\EV\ \\
~~"/Complete mism/ (at 0) is not the reverse of /re hard to find/" \\
\\
(debug-palindrome "Waltz, nymph, for quick jigs vex Bud") \\*
~~\EV\ "This palindrome is completely messed up!" \\
\\
(debug-palindrome "Doc, note: I dissent.~~A fast never \\*
~~~~~~~~~~~~~~~~~~~prevents a fatness.~~I diet on cod.") \\*
~~\EV\nil~~~~~;\textrm{Другой победитель} \\
\\
(debug-palindrome "Top step's pup's pet spot") \EV\ nil
\end{lisp}

\begin{defun}[Функция]
count item sequence &key :from-end :test :test-not :start~:end~:key \\
count-if predicate sequence &key :from-end :start~:end~:key \\
count-if-not predicate sequence &key :from-end :start~:end~:key

Результатом является неотрицательное целое число, указывающее на количество
элементов из указанной подпоследовательности, удовлетворяющих условию.

\cd{:from-end} не оказывает никакого воздействия, и оставлен только для
совместимости с другими функциями.

\begin{new}
X3J13 voted in January 1989
\issue{MAPPING-DESTRUCTIVE-INTERACTION}
to restrict user side effects; see section~\ref{STRUCTURE-TRAVERSAL-SECTION}.
\end{new}

Пользователь ограничен в создании побочных действий так, как это описано в
разделе~\ref{STRUCTURE-TRAVERSAL-SECTION}
\end{defun}

\begin{defun}[Функция]
mismatch sequence1 sequence2 &key :from-end :test :test-not :key :start1 :start2 :end1 :end2

Функция поэлементно сравнивает указанные подпоследовательности.
Если их длины равны и соответственно равны между собой каждые их элементы, то
функция возвращает {\false}. Иначе, функция возвращает неотрицательное целое
число.
Это число указывает на первый элемент слева, который не совпал с элементом другой
последовательности \emph{sequence2}. Или же, в случае если одна
подпоследовательность короче другой, но все элементы были одинаковыми, индекс
указывает на последний проверенный элемент.

Если параметр \cd{:from-end} равен не-{\false} значению, тогда возвращается
увеличенная на 1 позиция первого не совпавшего элемента справа. Фактически,
(под)последовательности выравниваются по своим правым концам, затем сравниваются
последние элементы, затем предпоследние и так далее. Возвращаемый индекс
принадлежит первой последовательности \emph{sequence1}.

\begin{new}
X3J13 voted in January 1989
\issue{MAPPING-DESTRUCTIVE-INTERACTION}
to restrict user side effects; see section~\ref{STRUCTURE-TRAVERSAL-SECTION}.
\end{new}

Пользователь ограничен в создании побочных действий так, как это описано в
разделе~\ref{STRUCTURE-TRAVERSAL-SECTION}
\end{defun}

\begin{defun}[Функция]
search sequence1 sequence2 &key :from-end :test :test-not :key :start1 :start2 :end1 :end2

Функция осуществляет поэлементный поиск последовательности \emph{sequence1} в
последовательности \emph{sequence1}.
Если поиск не увенчался успехом, результатом является значение {\false}. Иначе,
результат является индексом первого совпавшего элемента слева во второй
последовательности \emph{sequence2}.

Если аргумент \cd{:from-end} равен не-{\false}, поиск осуществляется справа, и
результат является индексом элемента слева в первой совпавшей
подпоследовательности справа.

Реализация может выбирать алгоритм поиска на своё усмотрение. Количество
проводимых сравнений не может быть указано точно.
Например, \cdf{search} при \cd{:from-end} равном не-{\nil} может фактически
производить поиск слева направо, но результатом будет всегда индекс самой правой
совпавшей последовательности. Поэтому пользователю желательно использовать
предикаты без побочных эффектов.

\begin{new}
X3J13 voted in January 1989
\issue{MAPPING-DESTRUCTIVE-INTERACTION}
to restrict user side effects; see section~\ref{STRUCTURE-TRAVERSAL-SECTION}.
\end{new}

Пользователь ограничен в создании побочных действий так, как это описано в
разделе~\ref{STRUCTURE-TRAVERSAL-SECTION}
\end{defun}

\section{Сортировка и слияние}

Эти функции могут модифицировать исходные данные:
сортировать или соединять две уже отсортированные последовательности.

\begin{defun}[Функция]
sort sequence predicate &key :key \\
stable-sort sequence predicate &key :key

Функция сортирует последовательность \emph{sequence} в порядке, определяемом
предикатом \emph{predicate}. Результат помещается в исходную
последовательность. Предикат \emph{predicate} должен принимать два аргумента,
возвращать не-{\false} тогда и только тогда, когда первый аргумент строго меньше
второго (в подходящем для этого смысле).
Если первый аргумент больше или равен второму (в подходящем для этого смысле),
тогда предикат \emph{predicate} должен вернуть {\false}.

Функция \cdf{sort} устанавливает отношение между двумя элементами с помощью
предиката \emph{predicate}, применённого к извлечённой из элемента ключевой
части. Функция \cd{:key}, применённая к элементу, должна возвращать его ключевую
часть.
Аргумент \cd{:key} по-умолчанию равен функции эквивалентности, тем самым
возвращая весь элемент.

Функция \cd{:key} не должна иметь побочных эффектов.
Полезным примером функции \cd{:key} может быть функция-селектор для некоторой
структуры (созданной с помощью \cdf{defstruct}), используемая для сортировки
последовательности структур.
\begin{lisp}
(sort \emph{a} \emph{p} \cd{:key} \emph{s})
   \EQ\ (sort \emph{a} \#'(lambda (x y) (\emph{p} (\emph{s} x) (\emph{s} y))))
\end{lisp}
Тогда как оба выражения эквивалентны, для некоторых реализаций и определённых
типов первое выражение может оказаться эффективнее.
Например, реализация может выбрать сначала вычислить все ключевые части,
поместить их в таблицу, а затем параллельно сортировать таблицы.

Если функции \cd{:key} и \emph{predicate} всегда возвращают управление, тогда
операция сортировки будет всегда возвращать последовательность, содержащую такое
же количество элементов как и в исходной (результат является просто
последовательностью с переставленными элементами).
Это поведение гарантируется, даже если предикат \emph{predicate} в
действительности не производит сравнение (в таком случае элементы будут
расположены в неопределённом порядке, но ни один из них не будет потерян). Если
функция \cd{:key} последовательно возвращает необходимые ключевые части, и
\emph{predicate} отображает некоторый критерий для упорядочивания данных частей,
тогда элементы итоговой последовательности будут корректно отсортированы в
соответствие с этим критерием.

Операция сортировки, выполняемая с помощью \cdf{sort}, не гарантированно
\emph{постоянна}.
Элементы, рассматриваемые предикатом \emph{predicate} как равные, могут остаться
или нет в оригинальном порядке.
(Предполагается, что \emph{predicate} рассматривает два элемента \emph{x} и
\emph{y} равными, если 
\cd{(funcall \emph{predicate} \emph{x} \emph{y})} и
\cd{(funcall \emph{predicate} \emph{y} \emph{x})} являются ложными.)
Функция \cdf{stable-sort} гарантирует постоянность, но в некоторых ситуациях
может быть медленнее чем \cdf{sort}.

Операция сортировки может быть деструктивное во всех случаях. В случае аргумента
массива, она производится перестановкой элементов.
В случае аргумента списка, список деструктивно переупорядочивается похожим
образом как в \cdf{nreverse}.
Таким образом, если аргумент не должен быть изменён, пользователь должен
сортировать копию исходной последовательности.

Если применение функций \cd{:key} или \emph{predicate} вызывает ошибку, то
состояние сортируемого массива или списка не определено.
Однако, если ошибка может быть исправлена, сортировка, конечно же, будет
завершена корректно.

Следует отметить, что сортировка требует много сравнений и следовательно много
вызовов предиката \emph{predicate}, сортировка будет более быстрой, если
\emph{predicate} является компилируемой, а не интерпретируемой функцией.

Например:
\begin{lisp}
(setq foovector (sort foovector \#'string-lessp \cd{:key} \#'car))
\end{lisp}
Если \cdf{foovector} содержит эти данные перед сортировкой
\begin{lisp}
("Tokens" "The Lion Sleeps Tonight") \\
("Carpenters" "Close to You") \\
("Rolling Stones" "Brown Sugar") \\
("Beach Boys" "I Get Around") \\
("Mozart" "Eine Kleine Nachtmusik" (K 525)) \\
("Beatles" "I Want to Hold Your Hand")
\end{lisp}
тогда после неё, \cdf{foovector} будет содержать
\begin{lisp}
("Beach Boys" "I Get Around") \\
("Beatles" "I Want to Hold Your Hand") \\
("Carpenters" "Close to You") \\
("Mozart" "Eine Kleine Nachtmusik" (K 525)) \\
("Rolling Stones" "Brown Sugar") \\
("Tokens" "The Lion Sleeps Tonight")
\end{lisp}

\begin{new}
X3J13 voted in January 1989
\issue{MAPPING-DESTRUCTIVE-INTERACTION}
to restrict user side effects; see section~\ref{STRUCTURE-TRAVERSAL-SECTION}.
\end{new}

Пользователь ограничен в создании побочных действий так, как это описано в
разделе~\ref{STRUCTURE-TRAVERSAL-SECTION}
\end{defun}

\begin{defun}[Функция]
merge result-type sequence1 sequence2 predicate &key :key

Функция деструктивно соединяет две последовательности \emph{sequence1} и
\emph{sequence2} в порядке определяемом предикатом \emph{predicate}.
Результат является последовательностью типа \emph{result-type}, который должен
быть подтипом \cdf{sequence}.
Предикат \emph{predicate} должен принимать два аргумента,
возвращать не-{\false} тогда и только тогда, когда первый аргумент строго меньше
второго (в подходящем для этого смысле).
Если первый аргумент больше или равен второму (в подходящем для этого смысле),
тогда предикат \emph{predicate} должен вернуть {\false}.

Функция \cdf{merge} устанавливает отношение между двумя элементами с помощью
предиката \emph{predicate}, применённого к извлечённой из элемента ключевой
части. Функция \cd{:key}, применённая к элементу, должна возвращать его ключевую
часть.
Аргумент \cd{:key} по-умолчанию равен функции эквивалентности, тем самым
возвращая весь элемент.

Функция \cd{:key} не должна иметь побочных эффектов.
Полезным примером функции \cd{:key} может быть функция-селектор для некоторой
структуры (созданной с помощью \cdf{defstruct}), используемая для сортировки
последовательности структур.

Если функции \cd{:key} и \emph{predicate} всегда возвращают управление, тогда
операция сортировки будет всегда завершаться.
Результатом слияния двух последовательностей \emph{x} и \emph{y} является новая
последовательность \emph{z}, у которой длина равняется сумма длин \emph{x} и
\emph{y}. \emph{z} содержит все элементы \emph{x} и \emph{y}.
Если \emph{x1} и \emph{x2} являются элементами \emph{x}, и \emph{x1} стоит
прежде \emph{x2}, тогда в \emph{z} \emph{x1} также будет стоять прежде чем
\emph{x2}. Такое же правило и для \emph{y}. Если коротко, \emph{z} является
\emph{слиянием} \emph{x} и \emph{y}.

Более того, если \emph{x} и \emph{y} были правильно отсортированы в соответствие
с предикатом \emph{predicate}, тогда \emph{z} будет также правильно
отсортирована. Например,
\begin{lisp}
(merge 'list '(1 3 4 6 7) '(2 5 8) \#'<) \EV\ (1 2 3 4 5 6 7 8)
\end{lisp}
Если \emph{x} или \emph{y} не были отсортированы, тогда \emph{z} также не будет
отсортирована. Однако слияние в любом случае произойдёт.

Операция слияния гарантированно \emph{постоянна}.
Если два или более элементов рассматриваются предикатом \emph{predicate} как
равные, тогда в результате элементы из \emph{sequence1} будут предшествовать
элементам из \emph{sequence2}.
(Предполагается, что \emph{predicate} рассматривает два элемента \emph{x} и
\emph{y} равными, если 
\cd{(funcall \emph{predicate} \emph{x} \emph{y})} и
\cd{(funcall \emph{predicate} \emph{y} \emph{x})} являются ложными.)
Например:
\begin{lisp}
(merge 'string "BOY" "nosy" \#'char-lessp) \EV\ "BnOosYy"
\end{lisp}
Результат не может быть \cd{"BnoOsYy"}, \cd{"BnOosyY"} или \cd{"BnoOsyY"}.
Функция \cdf{char-lessp} игнорирует регистр, таким образом, например, символы \cd{Y} и
\cd{y} равны. Свойство постоянности гарантирует, что символ из первого аргумента
(\cd{Y}) должен предшествовать символу из второго аргумента (\cd{y}).

\begin{newer}
X3J13 voted in June 1989 \issue{SEQUENCE-TYPE-LENGTH} to specify that
\cdf{merge} should signal an error if the sequence type specifies the number of
elements and the sum of the lengths of the two sequence arguments is
different.
\end{newer}

\begin{new}
X3J13 voted in January 1989
\issue{MAPPING-DESTRUCTIVE-INTERACTION}
to restrict user side effects; see section~\ref{STRUCTURE-TRAVERSAL-SECTION}.
\end{new}

Пользователь ограничен в создании побочных действий так, как это описано в
разделе~\ref{STRUCTURE-TRAVERSAL-SECTION}
\end{defun}

\fi      % Functions on sequences (keyword style)
%Part{List, Root = "CLM.MSS"}
%%% Chapter of Common Lisp Manual.  Copyright 1984, 1988, 1989 Guy L. Steele Jr.

\clearpage\def\pagestatus{FINAL PROOF}

\ifx \rulang\Undef

\chapter{Lists}

A \emph{cons}, or dotted pair, is a compound data object having two components
called the \emph{car} and \emph{cdr}.  Each component may be any Lisp object.
A \emph{list} is a chain of conses
linked by \emph{cdr} fields; the chain is terminated by some atom
(a non-cons object).
An ordinary list is terminated by {\nil}, the empty list
(also written {\emptylist}).
A list whose \emph{cdr} chain is terminated by some non-{\nil} atom is called
a \emph{dotted list}.

The recommended predicate for testing for the end of a list is \cdf{endp}.

\section{Conses Cons-ячейки}

These are the basic operations on conses viewed as pairs rather than
as the constituents of a list.

\begin{defun}[Function]
car list

This returns the \emph{car} of \emph{list}, which must be a cons or {\emptylist};
that is, \emph{list} must satisfy the predicate \cdf{listp}.
By definition, the \emph{car} of {\emptylist} is {\emptylist}.
If the cons is regarded as the first cons of a list, then \cdf{car}
returns the first element of the list.
For example:
\begin{lisp}
(car '(a b c)) \EV\ a
\end{lisp}
See \cdf{first}.
The \emph{car} of a cons may be altered by using \cdf{rplaca} or \cdf{setf}.
\end{defun}

\begin{defun}[Function]
cdr list

This returns the \emph{cdr} of \emph{list}, which must be a cons or {\emptylist};
that is, \emph{list} must satisfy the predicate \cdf{listp}.
By definition, the \emph{cdr} of {\emptylist} is {\emptylist}.
If the cons is regarded as the first cons of a list, then \cdf{cdr}
returns the rest of the list, which is a list with all elements
but the first of the original list.
For example:
\begin{lisp}
(cdr '(a b c)) \EV\ (b c)
\end{lisp}
See \cdf{rest}.
The \emph{cdr} of a cons may be altered by using \cdf{rplacd} or \cdf{setf}.
\end{defun}

\begin{defun}[Function]
caar list \\
cadr list \\
cdar list \\
cddr list \\
caaar list \\
caadr list \\
cadar list \\
caddr list \\
cdaar list \\
cdadr list \\
cddar list \\
cdddr list \\
caaaar list \\
caaadr list \\
caadar list \\
caaddr list \\
cadaar list \\
cadadr list \\
caddar list \\
cadddr list \\
cdaaar list \\
cdaadr list \\
cdadar list \\
cdaddr list \\
cddaar list \\
cddadr list \\
cdddar list \\
cddddr list

All of the compositions of up to four \cdf{car} and \cdf{cdr} operations
are defined as separate Common Lisp functions.
The names of these functions begin with \cdf{c} and end with \cdf{r},
and in between is a sequence of \cdf{a} and \cdf{d} letters
corresponding to
the composition performed by the function. 
For example:
\begin{lisp}
(cddadr x) \textrm{is the same as} (cdr (cdr (car (cdr x))))
\end{lisp}
If the argument is regarded as a list, then \cdf{cadr} returns
the second element of the list, \cdf{caddr} the third, and \cdf{cadddr}
the fourth.  If the first element of a list is a list, then
\cdf{caar} is the first element of the sublist, \cdf{cdar} is the
rest of that sublist, and \cdf{cadar} is the second element of the sublist,
and so on.

As a matter of style, it is often preferable to define a function or
macro to access part of a complicated data structure, rather than to use
a long \cdf{car}/\cdf{cdr} string.  For example, one might define
a macro to extract the list of parameter variables from a lambda-expression:
\begin{lisp}
(defmacro lambda-vars (lambda-exp) `(cadr ,lambda-exp))
\end{lisp}
and then use \cdf{lambda-vars} for this purpose instead of \cdf{cadr}.
See also \cdf{defstruct}, which will automatically define
new record data types and access functions for instances of them.

Any of these functions may be used to specify a \emph{place} for \cdf{setf}.
\end{defun}
	
\begin{defun}[Function]
cons x y

\cdf{cons} is the primitive function to create a new \emph{cons} whose
\emph{car} is \emph{x} and whose \emph{cdr} is \emph{y}.
For example:
\begin{lisp}
(cons 'a 'b) \EV\ (a . b) \\
(cons 'a (cons 'b (cons 'c '{\emptylist}))) \EV\ (a b c) \\
(cons 'a '(b c d)) \EV\ (a b c d)
\end{lisp}
\cdf{cons} may be thought of as creating a \emph{cons}, or as adding a new element
to the front of a list.
\end{defun}

\begin{defun}[Function]
tree-equal x y &key :test :test-not

This is a predicate that is true if \emph{x} and \emph{y} are
isomorphic trees with identical leaves, that is, if \emph{x} and \emph{y}
are atoms that satisfy the test (by default \cdf{eql}),
or if they are both conses and their \emph{car}'s are \cdf{tree-equal}
and their \emph{cdr}'s are \cdf{tree-equal}.
Thus \cdf{tree-equal} recursively compares conses (but not any other objects
that have components).  See \cdf{equal}, which does recursively
compare certain other structured objects, such as strings.

\begin{new}
X3J13 voted in January 1989
\issue{MAPPING-DESTRUCTIVE-INTERACTION}
to restrict user side effects; see section~\ref{STRUCTURE-TRAVERSAL-SECTION}.
\end{new}
\end{defun}

\section{Lists}

The following functions perform various operations on lists.

The list is one of the original Lisp data types.  The very name ``Lisp''
is an abbreviation for ``LISt Processing.''

\goodbreak

\begin{defun}[Function]
endp object

The predicate \cdf{endp} is the recommended way to test for the end
of a list.  It is false of conses, true of {\nil}, and an error for
all other arguments.

\beforenoterule
\begin{implementation}
Implementations are encouraged to signal an
error, especially in the interpreter, for a non-list argument.
The \cdf{endp} function is defined so as to allow compiled code
to perform simply an atom check or a null check if speed is more
important than safety.
\end{implementation}
\afternoterule
\end{defun}

\begin{defun}[Function]
list-length list

\cdf{list-length} returns, as an integer, the length of \emph{list}.
\cdf{list-length} differs from \cdf{length} when the \emph{list} is
circular; \cdf{length} may fail to return, whereas \cdf{list-length}
will return {\nil}.
For example:
\begin{lisp}
(list-length '{\emptylist}) \EV\ 0 \\
(list-length '(a b c d)) \EV\ 4 \\
(list-length '(a (b c) d)) \EV\ 3 \\
(let ((x (list 'a b c))) \\
~~(rplacd (last x) x) \\
~~(list-length x)) \EV\ {\nil}
\end{lisp}
\cdf{list-length} could be implemented as follows:
\begin{lisp}
(defun list-length (x) \\
~~(do ((n 0 (+ n 2))~~~~~~~~~~~~;\textrm{Counter} \\
~~~~~~~(fast x (cddr fast))~~~~~;\textrm{Fast pointer: leaps by 2} \\
~~~~~~~(slow x (cdr slow)))~~~~~;\textrm{Slow pointer: leaps by 1} \\
~~~~~~(nil) \\
~~~~;; If fast pointer hits the end, return the count. \\
~~~~(when (endp fast) (return n)) \\
~~~~(when (endp (cdr fast)) (return (+ n 1))) \\
~~~~;; If fast pointer eventually equals slow pointer, \\
~~~~;;  then we must be stuck in a circular list. \\
~~~~;; (A deeper property is the converse: if we are \\
~~~~;;  stuck in a circular list, then eventually the \\
~~~~;;  fast pointer will equal the slow pointer. \\
~~~~;;  That fact justifies this implementation.) \\
~~~~(when (and (eq fast slow) (> n 0)) (return nil))))
\end{lisp}
See \cdf{length}, which will return the length of any sequence.
\end{defun}

\begin{defun}[Function]
nth n list

\cd{(nth \emph{n} \emph{list})} returns the \emph{n}th element of \emph{list}, where
the \emph{car} of the list is the ``zeroth'' element.
The argument \emph{n} must be a non-negative integer.
If the length of the list is not greater than \emph{n}, then the result
is {\emptylist}, that is, {\false}.
(This is consistent with the idea that the \emph{car} and \emph{cdr}
of {\emptylist} are each {\emptylist}.)
For example:
\begin{lisp}
(nth 0 '(foo bar gack)) \EV\ foo \\
(nth 1 '(foo bar gack)) \EV\ bar \\
(nth 3 '(foo bar gack)) \EV\ {\emptylist}
\end{lisp}

\cdf{nth} may be used to specify a \emph{place} to \cdf{setf};
when \cdf{nth} is used in this way, the argument \emph{n} must be less
than the length of the \emph{list}.

Note that the arguments to \cdf{nth} are reversed from the order
used by most other sequence selector functions such as \cdf{elt}.
\end{defun}


\begin{defun}[Function]
first list \\
second list \\
third list \\
fourth list \\
fifth list \\
sixth list \\
seventh list \\
eighth list \\
ninth list \\
tenth list

These functions are sometimes convenient for accessing particular
elements of a list.  \cdf{first} is the same as \cdf{car},
\cdf{second} is the same as \cdf{cadr}, \cdf{third} is the
same as \cdf{caddr}, and so on.
Note that the ordinal numbering used here is one-origin,
as opposed to the zero-origin numbering used by \cdf{nth}:
\begin{lisp}
(fifth x) \EQ\ (nth 4 x)
\end{lisp}

\cdf{setf} may be used with each of these functions to store
into the indicated position of a list.
\end{defun}

\begin{defun}[Function]
rest list

\cdf{rest} means the same as \cdf{cdr} but mnemonically complements \cdf{first}.
\cdf{setf} may be used with \cdf{rest} to replace the \emph{cdr} of a list
with a new value.
\end{defun}

\begin{defun}[Function]
nthcdr n list

\cd{(nthcdr \emph{n} \emph{list})} performs the \cdf{cdr} operation \emph{n} times
on \emph{list}, and returns the result.
For example:
\begin{lisp}
(nthcdr 0 '(a b c)) \EV\ (a b c) \\
(nthcdr 2 '(a b c)) \EV\ (c) \\
(nthcdr 4 '(a b c)) \EV\ {\emptylist}
\end{lisp}
In other words, it returns the \emph{n}th \emph{cdr} of the list.

\begin{lisp}
(car (nthcdr n x)) \EQ\ (nth n x)
\end{lisp}
The argument \emph{n} must be a non-negative integer.
\end{defun}

\begin{defun}[Function]
last list &optional (n 1)

\cdf{last} returns the tail of the \emph{list}
consisting of the last \emph{n} conses of \emph{list}.  The \emph{list} may
be a dotted list.  It is an error if the \emph{list} is circular.

The argument \emph{n} must be a non-negative integer.
If \emph{n} is zero, then the atom that terminates the \emph{list}
is returned.  If \emph{n} is not less than the number of cons cells
making up the \emph{list}, then the \emph{list} itself is returned.

For example:
\begin{lisp}
(setq x '(a b c d)) \\
(last x) \EV\ (d) \\
(rplacd (last x) '(e f)) \\
x \EV\ '(a b c d e f) \\
(last x 3) \EV\ (d e f) \\
(last '()) \EV\ () \\
(last '(a b c . d)) \EV\ (c . d) \\
(last '(a b c . d) 0) \EV\ d \\
(last '(a b c . d) 2) \EV\ (b c . d) \\
(last '(a b c . d) 1729) \EV\ (a b c . d)
\end{lisp}
\end{defun}

\begin{defun}[Function]
list &rest args

\cdf{list} constructs and returns a list of its arguments.
For example:
\begin{lisp}
(list 3 4 'a (car '(b . c)) (+ 6 -2)) \EV\ (3 4 a b 4)
\end{lisp}

\begin{lisp}
(list) \EV\ () \\
(list (list 'a 'b) (list 'c 'd 'e)) \EV\ ((a b) (c d e))
\end{lisp}
\end{defun}

\begin{defun}[Function]
list* arg &rest others

\cdf{list*} is like \cdf{list} except that the last \emph{cons}
of the constructed list is ``dotted.''  The last argument to \cdf{list*}
is used as the \emph{cdr} of the last cons constructed;
this need not be an atom.  If it is not an atom,
then the effect is to add several new elements to the front of a list.
For example:
\begin{lisp}
(list* 'a 'b 'c 'd) \EV\ (a b c . d)
\end{lisp}
This is like
\begin{lisp}
(cons 'a (cons 'b (cons 'c 'd)))
\end{lisp}
Also:
\begin{lisp}
(list* 'a 'b 'c '(d e f)) \EV\ (a b c d e f) \\*
(list* x) \EQ\ x
\end{lisp}
\end{defun}

\begin{defun}[Function]
make-list size &key :initial-element

This creates and returns a list containing \emph{size} elements, each
of which is initialized to the \cd{:initial-element}
argument (which defaults to {\false}).
\emph{size} should be a non-negative integer.
For example:
\begin{lisp}
(make-list 5) \EV\ ({\false} {\false} {\false} {\false} {\false}) \\
(make-list 3 \cd{:initial-element} 'rah) \EV\ (rah rah rah)
\end{lisp}
\end{defun}

\begin{defun}[Function]
append &rest lists

The arguments to \cdf{append} are lists.  The result is a list that is the
concatenation of the arguments.
The arguments are not destroyed.
For example:
\begin{lisp}
(append '(a b c) '(d e f) '{\emptylist} '(g)) \EV\ (a b c d e f g)
\end{lisp}
Note that \cdf{append} copies the top-level list structure of each of its
arguments \emph{except} the last.
The function \cdf{concatenate} can perform a similar operation, but always
copies all its arguments.  See also \cdf{nconc}, which is like \cdf{append}
but destroys all arguments but the last.

The last argument actually need not be a list but may be any Lisp object,
which becomes the tail end of the constructed list.
For example, \cd{(append '(a b c) 'd)} \EV\ \cd{(a b c . d)}.

\cd{(append \emph{x} '{\emptylist})} is an idiom once frequently used to copy the
list \emph{x}, but the \cdf{copy-list} function is more appropriate to this
task.
\end{defun}

\begin{defun}[Function]
copy-list list

This returns a list that is \cdf{equal} to \emph{list}, but not \cdf{eq}.
Only the top level of list structure is copied; that is, \cdf{copy-list}
copies in the \emph{cdr} direction but not in the \emph{car} direction.
If the list is ``dotted,'' that is, \cd{(cdr (last \emph{list}))}
is a non-{\nil} atom, this will be true of the returned list also.
See also \cdf{copy-seq} and \cdf{copy-tree}.
\end{defun}

\begin{defun}[Function]
copy-alist list

\cdf{copy-alist} is for copying association lists.  The top level of
list structure of \emph{list} is copied, just as for \cdf{copy-list}.
In addition, each element of \emph{list} that is a cons is replaced
in the copy by a new cons with the same \emph{car} and \emph{cdr}.
\end{defun}

\begin{defun}[Function]
copy-tree object

\cdf{copy-tree} is for copying trees of conses.
The argument \emph{object} may be any Lisp object.
If it is not a cons, it is returned; otherwise
the result is a new cons of the results of calling \cdf{copy-tree}
on the \emph{car} and \emph{cdr} of the argument.  In other words,
all conses in the tree are copied recursively, stopping
only when non-conses are encountered.
Circularities and the sharing of substructure are \emph{not} preserved.
\end{defun}

\begin{defun}[Function]
revappend x y

\cd{(revappend \emph{x} \emph{y})} is exactly the same as 
\cd{(append (reverse \emph{x}) \emph{y})} except that it is potentially more
efficient.  Both \emph{x} and \emph{y} should be lists.
The argument \emph{x} is copied, not destroyed.
Compare this with \cdf{nreconc}, which destroys its first argument.
\end{defun}

\begin{defun}[Function]
nconc &rest lists

\cdf{nconc} takes lists as arguments.  It returns a list that is the arguments
concatenated together.  The arguments are changed rather than copied.
(Compare this with \cdf{append}, which copies arguments rather than
destroying them.)
For example:
\begin{lisp}
(setq x '(a b c)) \\
(setq y '(d e f)) \\
(nconc x y) \EV\ (a b c d e f) \\
x \EV\ (a b c d e f)
\end{lisp}
Note, in the example, that the value of \cdf{x} is now different,
since its last cons has been \cdf{rplacd}'d to the value of \cdf{y}.
If one were then to evaluate \cd{(nconc x y)} again,
it would yield a piece of ``circular'' list
structure, whose printed representation would be
\cd{(a b c d e f d e f d e f ...)}, repeating forever;
if the \cdf{*print-circle*} switch were non-{\nil},
it would be printed as \cd{(a b c . \#1=(d e f . \#1\#))}.

The side-effect behavior of \cdf{nconc} is specified by a recursive relationship
outlined in the following table, in which a call to \cdf{nconc} matching
the earliest possible
pattern on the left is required to have side-effect behavior
equivalent to the corresponding expression on the right.
\begin{flushleft}
\begin{tabular}{@{}ll@{}}
\cd{(nconc)}&\cd{nil~~~~~;}\textrm{No side effects} \\
\cd{(nconc nil . \emph{r})~~~~}&\cd{(nconc . \emph{r})} \\
\cd{(nconc \emph{x})}&\emph{x} \\
\cd{(nconc \emph{x} \emph{y})}&\cd{(let ((p \emph{x}) (q \emph{y}))} \\
                                  &\cd{~~(rplacd (last p) q)} \\
                                  &\cd{~~p)} \\
\cd{(nconc \emph{x} \emph{y} . \emph{r})}&\cd{(nconc (nconc \emph{x} \emph{y})
  . \emph{r})}
\end{tabular} 
\end{flushleft}
\end{defun}

\begin{defun}[Function]
nreconc x y

\cd{(nreconc \emph{x} \emph{y})} is exactly the same as 
\cd{(nconc (nreverse \emph{x}) \emph{y})} except that it is potentially more
efficient.  Both \emph{x} and \emph{y} should be lists.
The argument \emph{x} is destroyed.
Compare this with \cdf{revappend}.

\begin{lisp}
(setq planets '(jupiter mars earth venus mercury)) \\
(setq more-planets '(saturn uranus pluto neptune)) \\
(nreconc more-planets planets) \\
\`\EV\ (neptune pluto uranus saturn jupiter mars earth venus mercury) \\
~~\textrm{and now the value of \cdf{more-planets} is not well defined}
\end{lisp}

\cd{(nreconc \emph{x} \emph{y})} is permitted and
required to have side-effect behavior
equivalent to that of \cd{(nconc (nreverse \emph{x})~\emph{y})}.
\end{defun}

\begin{defmac}
push item place

The form \emph{place} should be the name of a generalized variable
containing a list; \emph{item} may refer to any Lisp object.  The \emph{item}
is consed onto the front of the list, and the augmented list is stored
back into \emph{place} and returned.
The form \emph{place} may be any form acceptable as a
generalized variable to \cdf{setf}.  If the list held in \emph{place} is
viewed as a push-down stack, then \cdf{push} pushes an element onto the top
of the stack.
For example:
\begin{lisp}
(setq x '(a (b c) d)) \\
(push 5 (cadr x)) \EV\ (5 b c)  \textrm{and now} x \EV\ (a (5 b c) d)
\end{lisp}
The effect of \cd{(push \emph{item} \emph{place})}
is roughly equivalent to
\begin{lisp}
(setf \emph{place} (cons \emph{item} \emph{place}))
\end{lisp}
except that the latter would evaluate any subforms of \emph{place}
twice, while \cdf{push} takes care to evaluate them only once.
Moreover, for certain \emph{place} forms \cdf{push} may be
significantly more efficient than the \cdf{setf} version.

Note that \emph{item} is fully evaluated before any part of \emph{place}
is evaluated.
\end{defmac}

\begin{defmac}
pushnew item place &key :test :test-not :key

The form \emph{place} should be the name of a generalized variable
containing a list; \emph{item} may refer to any Lisp object.  If the
\emph{item} is not already a member of the list (as determined by
comparisons using the \cd{:test} predicate, which defaults to \cdf{eql}),
then the \emph{item} is consed onto the front of the list, and
the augmented list is stored back into \emph{place} and returned; otherwise
the unaugmented list is returned.  The form \emph{place} may be
any form acceptable as a generalized variable to \cdf{setf}.  If the
list held in \emph{place} is viewed as a set, then \cdf{pushnew} adjoins an
element to the set; see \cdf{adjoin}.

The keyword arguments to \cdf{pushnew}
follow the conventions for the generic sequence
functions.  See chapter~\ref{KSEQUE}.
In effect, these keywords are simply passed on to the \cdf{adjoin} function.

\cdf{pushnew} returns the new contents of the \emph{place}.
For example:
\begin{lisp}
(setq x '(a (b c) d)) \\
(pushnew 5 (cadr x)) \EV\ (5 b c)   \textrm{and now} x \EV\ (a (5 b c) d) \\
(pushnew 'b (cadr x)) \EV\ (5 b c)  \textrm{and \cdf{x} is unchanged}
\end{lisp}
The effect of
\begin{lisp}
(pushnew \emph{item} \emph{place} \cd{:test} \emph{p})
\end{lisp}
is roughly equivalent to
\begin{lisp}
(setf \emph{place} (adjoin \emph{item} \emph{place} \cd{:test} \emph{p}))
\end{lisp}
except that the latter would evaluate any subforms of
\emph{place} twice, while \cdf{pushnew} takes care to evaluate them only once.
Moreover, for certain \emph{place} forms \cdf{pushnew} may be
significantly more efficient than the \cdf{setf} version.

Note that \emph{item} is fully evaluated before any part of \emph{place}
is evaluated.
\end{defmac}

\begin{defmac}
pop place

The form \emph{place} should be the name of a generalized variable
containing a list.  The result of \cdf{pop} is the \cdf{car} of the contents
of \emph{place}, and as a side effect the \cdf{cdr} of the contents is stored
back into \emph{place}.  The form \emph{place} may be any form acceptable as a
generalized variable to \cdf{setf}.  If the list held in \emph{place} is
viewed as a push-down stack, then \cdf{pop} pops an element from the top of
the stack and returns it.
For example:
\begin{lisp}
(setq stack '(a b c)) \\
(pop stack) \EV\ a  \textrm{and now} stack \EV\ (b c)
\end{lisp}
The effect of \cd{(pop \emph{place})} is roughly equivalent to
\begin{lisp}
(prog1 (car \emph{place}) (setf \emph{place} (cdr \emph{place})))
\end{lisp}
except that the latter would evaluate any subforms of \emph{place}
three times, while \cdf{pop} takes care to evaluate them only once.
Moreover, for certain \emph{place} forms \cdf{pop} may be
significantly more efficient than the \cdf{setf} version.
\end{defmac}

\begin{defun}[Function]
butlast list &optional n

This creates and returns a list with the same elements as \emph{list},
excepting the last \emph{n} elements.
\emph{n} defaults to 1.  The argument is not destroyed.
If the \emph{list} has fewer than \emph{n} elements, then {\emptylist} is returned.
For example:
\begin{lisp}
(butlast '(a b c d)) \EV\ (a b c) \\
(butlast '((a b) (c d))) \EV\ ((a b)) \\
(butlast '(a)) \EV\ {\emptylist} \\
(butlast nil) \EV\ {\emptylist}
\end{lisp}
The name is from the phrase ``all elements but the last.''
\end{defun}

\begin{defun}[Function]
nbutlast list &optional n

This is the destructive version of \cdf{butlast}; it changes the \emph{cdr} of
the cons \emph{n}+1 from the end of the \emph{list} to {\nil}.  \emph{n} defaults to 1.
If the \emph{list} has fewer than \emph{n} elements, then \cdf{nbutlast}
returns {\emptylist}, and the argument is not modified.  (Therefore
one normally writes \cd{(setq a (nbutlast a))} rather than simply
\cd{(nbutlast a)}.)
For example:
\begin{lisp}
(setq foo '(a b c d)) \\
(nbutlast foo) \EV\ (a b c) \\
foo \EV\ (a b c) \\
(nbutlast '(a)) \EV\ {\emptylist} \\
(nbutlast '{\nil}) \EV\ {\emptylist}
\end{lisp}
\end{defun}

\begin{defun}[Function]
ldiff list sublist

\emph{list} should be a list, and \emph{sublist} should be a sublist
of \emph{list}, that is, one of the conses that make up \emph{list}.
\cdf{ldiff} (meaning ``list difference'') will return a new (freshly consed)
list, whose elements are those elements of \emph{list} that appear before
\emph{sublist}.  If \emph{sublist} is not a tail of \emph{list}
(and in particular if \emph{sublist} is {\nil}),
then a copy of the entire \emph{list} is returned.
The argument \emph{list} is not destroyed.
For example:
\begin{lisp}
(setq x '(a b c d e)) \\
(setq y (cdddr x)) \EV\ (d e) \\
(ldiff x y) \EV\ (a b c) \\[4pt]
\textrm{but} (ldiff '(a b c d) '(c d)) \EV\ (a b c d)
\end{lisp}
since the sublist was not \cdf{eq} to any part of the list.
\end{defun}

\section{Alteration of List Structure}

The functions \cdf{rplaca} and \cdf{rplacd}
may be used to make alterations in already existing
list structure, that is, to change the \emph{car} or \emph{cdr} of an
existing cons.
One may also use \cdf{setf} in conjunction with \cdf{car} and \cdf{cdr}.

The structure is not copied but is destructively altered;
hence caution should be exercised when using these functions, as
strange side effects can occur if portions of list structure become
shared.
The \cdf{nconc}, \cdf{nreverse}, \cdf{nreconc},
and \cdf{nbutlast} functions, already
described,
have the same property, as do certain of the generic sequence
functions such as \cdf{delete}.
However, they are normally not
used for this side effect; rather, the list-structure modification
is purely for efficiency, and compatible non-modifying functions
are provided.

\begin{defun}[Function]
rplaca x y

\cd{(rplaca \emph{x} \emph{y})} changes the \emph{car} of \emph{x} to \emph{y} and returns
(the modified) \emph{x}.  \emph{x} must be a cons, but \emph{y} may be any
Lisp object.
For example:
\begin{lisp}
(setq g '(a b c)) \\
(rplaca (cdr g) 'd) \EV\ (d c) \\
\textrm{Now} g \EV\ (a d c)
\end{lisp}
\end{defun}

\begin{defun}[Function]
rplacd x y

\cd{(rplacd \emph{x} \emph{y})} changes the \emph{cdr} of \emph{x} to \emph{y} and returns
(the modified) \emph{x}.  \emph{x} must be a cons, but \emph{y} may be
any Lisp object.
For example:
\begin{lisp}
(setq x '(a b c)) \\
(rplacd x 'd) \EV\ (a . d) \\
\textrm{Now} x \EV\ (a . d)
\end{lisp}
\end{defun}

The functions \cdf{rplaca} and \cdf{rplacd} go back to the earliest
origins of Lisp, along with \cdf{car}, \cdf{cdr}, and \cdf{cons}.
Nowadays, however, they seem to be falling by the wayside.
More and more Common Lisp programmers use \cdf{setf} for nearly
all structure modifications: \cd{(rplaca x~y)} is rendered
as \cd{(setf (car x)~y)} or perhaps as \cd{(setf (first x)~y)}.
Even more likely is that a \cdf{defstruct} structure or a CLOS class
is used in place of a list, if the data structure is at all complicated;
in this case \cdf{setf} is used with a slot accessor.

\section{Substitution of Expressions}

\indexterm{substitution}
A number of functions are provided for performing substitutions
within a tree.  All take a tree and a description
of old subexpressions to be replaced by new ones.
They come in non-destructive and destructive varieties
and specify substitution either by two arguments or by an association list.

The naming conventions for these functions and for their keyword
arguments generally follow the conventions for the generic sequence
functions.  See chapter~\ref{KSEQUE}.

\begin{defun}[Function]
subst new old tree &key :test :test-not :key \\
subst-if new test tree &key :key \\
subst-if-not new test tree &key :key

\cd{(subst \emph{new} \emph{old} \emph{tree})} makes a copy of \emph{tree},
substituting \emph{new} for every subtree or leaf of \emph{tree}
(whether the subtree or leaf is a \emph{car} or a \emph{cdr} of its parent)
such that \emph{old} and the subtree or leaf satisfy the test.  It
returns the modified copy of \emph{tree}.  The original \emph{tree} is
unchanged, but the result tree may share with parts of the argument
\emph{tree}.

For example:
\begin{lisp}
(subst 'tempest 'hurricane \\
~~~~~~~'(shakespeare wrote (the hurricane))) \\
~~~\EV\ (shakespeare wrote (the tempest)) \\
\\
(subst 'foo '{\nil} '(shakespeare wrote (twelfth night))) \\
~~~\EV\ (shakespeare wrote (twelfth night . foo) . foo) \\
\\
(subst '(a . cons) '(old . pair) \\
~~~~~~~'((old . spice) ((old . shoes) old . pair) (old . pair)) \\
~~~~~~~\cd{:test} \#'equal) \\
~~~\EV\ ((old . spice) ((old . shoes) a . cons) (a . cons))
\end{lisp}
This function is not destructive; that is, it does not change
the \emph{car} or \emph{cdr} of any already existing list structure.
One possible definition of \cdf{subst}:
\begin{lisp}
(defun subst (old new tree \cd{\&rest} x \cd{\&key} test test-not key) \\*
~~(cond ((satisfies-the-test old tree :test test \\*
~~~~~~~~~~~~~~~~~~~~~~~~~~~~~:test-not test-not :key key) \\*
~~~~~~~~~new) \\*
~~~~~~~~((atom tree) tree) \\
~~~~~~~~(t (let ((a (apply \#'subst old new (car tree) x)) \\*
~~~~~~~~~~~~~~~~~(d (apply \#'subst old new (cdr tree) x))) \\
~~~~~~~~~~~~~(if (and (eql a (car tree)) \\*
~~~~~~~~~~~~~~~~~~~~~~(eql d (cdr tree))) \\*
~~~~~~~~~~~~~~~~~tree \\*
~~~~~~~~~~~~~~~~~(cons a d))))))
\end{lisp}
See also \cdf{substitute}, which substitutes for top-level elements
of a sequence.

\begin{new}
X3J13 voted in January 1989
\issue{MAPPING-DESTRUCTIVE-INTERACTION}
to restrict user side effects; see section~\ref{STRUCTURE-TRAVERSAL-SECTION}.
\end{new}
\end{defun}

\begin{defun}[Function]
nsubst new old tree &key :test :test-not :key \\
nsubst-if new test tree &key :key \\
nsubst-if-not new test tree &key :key

\cdf{nsubst} is a destructive version of \cdf{subst}.  The list structure of
\emph{tree} is altered by destructively replacing with \emph{new}
each leaf or subtree of the \emph{tree} such that \emph{old} and the leaf
or subtree satisfy the test.

\begin{new}
X3J13 voted in January 1989
\issue{MAPPING-DESTRUCTIVE-INTERACTION}
to restrict user side effects; see section~\ref{STRUCTURE-TRAVERSAL-SECTION}.
\end{new}
\end{defun}

\begin{defun}[Function]
sublis alist tree &key :test :test-not :key

\cdf{sublis} makes substitutions for objects in a tree
(a structure of conses).
The first argument to \cdf{sublis} is an association list.
The second argument is the tree in which
substitutions are to be made, as for \cdf{subst}.
\cdf{sublis} looks at all subtrees and leaves of the tree;
if a subtree or leaf appears as a key in the association
list (that is, the key and the subtree or leaf satisfy the test),
it is replaced by the object with which it is associated.
This operation is non-destructive.  In effect, \cdf{sublis} can
perform several \cdf{subst} operations simultaneously.
For example:
\begin{lisp}
(sublis '((x . 100) (z . zprime)) \\*
~~~~~~~~'(plus x (minus g z x p) 4 . x)) \\*
~~~\EV\ (plus 100 (minus g zprime 100 p) 4 . 100) \\*
 \\*
(sublis '(((+ x y) . (- x y)) ((- x y) . (+ x y))) \\*
~~~~~~~~'(* (/ (+ x y) (+ x p)) (- x y)) \\*
~~~~~~~~:test \#'equal) \\*
~~~\EV\ (* (/ (- x y) (+ x p)) (+ x y))
\end{lisp}

\begin{new}
X3J13 voted in January 1989
\issue{MAPPING-DESTRUCTIVE-INTERACTION}
to restrict user side effects; see section~\ref{STRUCTURE-TRAVERSAL-SECTION}.
\end{new}
\end{defun}

\begin{defun}[Function]
nsublis alist tree &key :test :test-not :key

\cdf{nsublis} is like \cdf{sublis} but destructively modifies the relevant
parts of the \emph{tree}.

\begin{new}
X3J13 voted in January 1989
\issue{MAPPING-DESTRUCTIVE-INTERACTION}
to restrict user side effects; see section~\ref{STRUCTURE-TRAVERSAL-SECTION}.
\end{new}
\end{defun}

\section{Using Lists as Sets}

Common Lisp includes functions that allow a list of items to be
treated as a \emph{set}.
There are functions to add, remove, and search for items in a list,
based on various criteria.
There are also set union, intersection, and difference functions.

The naming conventions for these functions and for their keyword
arguments generally follow the conventions that apply to the generic sequence
functions.  See chapter~\ref{KSEQUE}.

\begin{defun}[Function]
member item list &key :test :test-not :key \\
member-if predicate list &key :key \\
member-if-not predicate list &key :key

The \emph{list} is searched for an element that satisfies the test.
If none is found, {\false} is returned;
otherwise, the tail of \emph{list} beginning
with the first element that satisfied the test is returned.
The \emph{list} is searched on the top level only. 
These functions are suitable for use as predicates.

For example:
\begin{lisp}
(member 'snerd '(a b c d)) \EV\ {\false} \\
(member-if \#'numberp '(a \#{\Xbackslash}Space 5/3 foo)) \EV\ (5/3 foo) \\
(member 'a '(g (a y) c a d e a f)) \EV\ (a d e a f)
\end{lisp}
Note, in the last example,
that the value returned by \cdf{member} is \cdf{eq} to the portion of the list
beginning with \cdf{a}.
Thus \cdf{rplaca} on the result of \cdf{member} may be used
to alter the found list element,
if a check is first made that \cdf{member} did not return {\false}.

See also \cdf{find} and \cdf{position}.

\begin{new}
X3J13 voted in January 1989
\issue{MAPPING-DESTRUCTIVE-INTERACTION}
to restrict user side effects; see section~\ref{STRUCTURE-TRAVERSAL-SECTION}.
\end{new}
\end{defun}

\begin{defun}[Function]
tailp sublist list

\cdf{tailp} is true if and only if there exists an integer \emph{n} such that
\begin{lisp}
(eql \emph{sublist} (nthcdr \emph{n} \emph{list}))
\end{lisp}
\emph{list} may be a dotted list (implying that
implementations must use \cdf{atom} and not \cdf{endp} to check for
the end of the \emph{list}).
\end{defun}

\begin{defun}[Function]
adjoin item list &key :test :test-not :key

\cdf{adjoin} is used to add an element to a set, provided that
it is not already a member.  The equality test defaults to \cdf{eql}.
\begin{lisp}
(adjoin \emph{item} \emph{list}) \EQ\ (if (member \emph{item} \emph{list}) \emph{list} (cons \emph{item} \emph{list}))
\end{lisp}
In general, the test may be any predicate; the \emph{item} is added to the
list only if there is no element of the list that ``satisfies the test.''

\cdf{adjoin} deviates from the usual rules described in chapter~\ref{KSEQUE}
for the treatment of arguments named \emph{item} and \cd{:key}.
If a \cd{:key} function is specified, it is applied to \emph{item}
as well as to each element of the list.  The rationale is that
if the \emph{item} is not yet in the list, it soon will be, and so
the test is more properly viewed as being between two elements
rather than between a separate \emph{item} and an element.
\begin{lisp}
(adjoin \emph{item} \emph{list} :key \emph{fn}) \\
~~\EQ\ (if (member (funcall \emph{fn} \emph{item}) \emph{list}
  :key \emph{fn}) \emph{list} (cons \emph{item} \emph{list})) 
\end{lisp}
See \cdf{pushnew}.

\begin{new}
X3J13 voted in January 1989
\issue{MAPPING-DESTRUCTIVE-INTERACTION}
to restrict user side effects; see section~\ref{STRUCTURE-TRAVERSAL-SECTION}.
\end{new}
\end{defun}

\begin{defun}[Function]
union list1 list2 &key :test :test-not :key \\
nunion list1 list2 &key :test :test-not :key

\cdf{union} takes two lists and returns a new list containing
everything that is an element of either of the \emph{lists}.
If there is a duplication between two lists,
only one of the duplicate instances will be in the result.
If either of the arguments has duplicate entries within it,
the redundant entries
may or may not appear in the result.
For example:
\begin{lisp}
(union '(a b c) '(f a d)) \\
~~~\EV\ (a b c f d) \textrm{or} (b c f a d) \textrm{or} (d f a b c) \textrm{or} ... \\
 \\
(union '((x 5) (y 6)) '((z 2) (x 4)) :key \#'car) \\
~~~\EV\ ((x 5) (y 6) (z 2)) \textrm{or} ((x 4) (y 6) (z 2)) \textrm{or} ...
\end{lisp}

There is no guarantee that the order of elements in the result will
reflect the ordering of the arguments in any particular way.
The implementation is therefore free to use any of a variety of strategies.
The result list may share cells with, or be \cdf{eq} to, either of the arguments
if appropriate.

In general, the test may be any predicate, and the union operation may be
described as follows.  For all possible ordered pairs consisting of one
element from \emph{list1} and one element from \emph{list2}, the test is used
to determine whether they ``match.''  For every matching pair, at least
one of the two elements of the pair will be in the result.  Moreover, any
element from either list that matches no element of the other will appear
in the result.  All this is very general, but probably not particularly
useful unless the test is an equivalence relation.

The \cd{:test-not} argument can be useful when the test function
is the logical negation of an equivalence test.  A good example
of this is the function \cdf{mismatch}, which is logically inverted
so that possibly useful information can be returned if the arguments do not
match.  This additional ``useful information'' is discarded in the following
example; \cdf{mismatch} is used purely as a predicate.
\begin{lisp}
(union '(\#(a b) \#(5 0 6) \#(f 3)) \\
~~~~~~~'(\#(5 0 6) (a b) \#(g h)) \\
~~~~~~~:test-not \\
~~~~~~~\#'mismatch) \\
~~~\EV\ (\#(a b) \#(5 0 6) \#(f 3) \#(g h))~~~~~;\textrm{One possible result} \\
~~~\EV\ ((a b) \#(f 3) \#(5 0 6) \#(g h))~~~~~~;\textrm{Another possible result}
\end{lisp}
Using \cd{\cd{:test-not} \#'mismatch} differs from using
\cd{\cd{:test} \#'equalp}, for example, because \cdf{mismatch}
will determine that \cd{\#(a b)} and \cd{(a b)} are the same,
while \cdf{equalp} would regard them as not the same.

\cdf{nunion} is the destructive version of \cdf{union}.
It performs the same operation but may destroy the argument lists,
perhaps in order to use their cells to construct the result.

\begin{new}
X3J13 voted in January 1989
\issue{MAPPING-DESTRUCTIVE-INTERACTION}
to restrict user side effects; see section~\ref{STRUCTURE-TRAVERSAL-SECTION}.
\end{new}

\begin{newer}
X3J13 voted in March 1989 \issue{REMF-DESTRUCTION-UNSPECIFIED}
to clarify the permissible side effects of certain operations;
\cdf{nunion} is permitted to perform a \cdf{setf} on any part,
\emph{car} or \emph{cdr}, of the top-level list structure of 
any of the argument lists.
\end{newer}
\end{defun}

\begin{defun}[Function]
intersection list1 list2 &key :test :test-not :key \\
nintersection list1 list2 &key :test :test-not :key

\cdf{intersection} takes two lists and returns a new list containing
everything that is an element of both argument lists.
If either list has duplicate entries, the redundant entries
may or may not appear in the result.
For example:
\begin{lisp}
(intersection '(a b c) '(f a d)) \EV\ (a)
\end{lisp}
There is no guarantee that the order of elements in the result will
reflect the ordering of the arguments in any particular way.
The implementation is therefore free to use any of a variety of strategies.
The result list may share cells with, or be \cdf{eq} to, either of the arguments
if appropriate.

In general, the test may be any predicate, and the intersection operation
may be described as follows.  For all possible ordered pairs consisting of
one element from \emph{list1} and one element from \emph{list2}, the test is
used to determine whether they ``match.''  For every matching pair,
exactly one of the two elements of the pair will be put in the result.
No element from either list appears in the result that does not match
an element from the other list.
All this is very general, but probably
not particularly useful unless the test is an equivalence relation.

\cdf{nintersection} is the destructive version of \cdf{intersection}.
It performs the same operation, but may destroy \emph{list1},
perhaps in order to use its cells to construct the result.
(The argument \emph{list2} is \emph{not} destroyed.)

\begin{new}
X3J13 voted in January 1989
\issue{MAPPING-DESTRUCTIVE-INTERACTION}
to restrict user side effects; see section~\ref{STRUCTURE-TRAVERSAL-SECTION}.
\end{new}

\begin{newer}
X3J13 voted in March 1989 \issue{REMF-DESTRUCTION-UNSPECIFIED}
to clarify the permissible side effects of certain operations;
\cdf{nintersection} is permitted to perform a \cdf{setf} on any part,
\emph{car} or \emph{cdr}, of the top-level list structure of 
any of the argument lists.
\end{newer}
\end{defun}

\begin{defun}[Function]
set-difference list1 list2 &key :test :test-not :key \\
nset-difference list1 list2 &key :test :test-not :key

\cdf{set-difference} returns a list of elements of \emph{list1}
that do not appear in \emph{list2}.  This operation is
not destructive.

There is no guarantee that the order of elements in the result will
reflect the ordering of the arguments in any particular way.
The implementation is therefore free to use any of a variety of strategies.
The result list may share cells with, or be \cdf{eq} to, either of the arguments
if appropriate.

In general, the test may be any predicate, and the set difference operation
may be described as follows.  For all possible ordered pairs consisting of
one element from \emph{list1} and one element from \emph{list2}, the test is
used to determine whether they ``match.''  An element of \emph{list1}
appears in the result if and only if it does not match any element
of \emph{list2}. This is very general and permits interesting applications.
For example, one can remove from a list of strings all those strings
containing one of a given list of characters:
\begin{lisp}
;; Remove all flavor names that contain "c" or "w". \\
(set-difference '("strawberry" "chocolate" "banana" \\
~~~~~~~~~~~~~~~~~~"lemon" "pistachio" "rhubarb") \\
~~~~~~~~~~~~~~~~'(\#{\Xbackslash}c \#{\Xbackslash}w) \\
~~~~~~~~~~~~~~~~:test \\
~~~~~~~~~~~~~~~~\#'(lambda (s c) (find c s))) \\
~~~\EV\ ("banana" "rhubarb" "lemon")~~~~~;\textrm{One possible ordering}
\end{lisp}

\cdf{nset-difference} is the destructive version of \cdf{set-difference}.
This operation may destroy \emph{list1}.

\begin{new}
X3J13 voted in January 1989
\issue{MAPPING-DESTRUCTIVE-INTERACTION}
to restrict user side effects; see section~\ref{STRUCTURE-TRAVERSAL-SECTION}.
\end{new}
\end{defun}

\begin{defun}[Function]
set-exclusive-or list1 list2 &key :test :test-not :key \\
nset-exclusive-or list1 list2 &key :test :test-not :key

\cdf{set-exclusive-or} returns a list of elements that appear
in exactly one of \emph{list1} and \emph{list2}.
This operation is not destructive.

There is no guarantee that the order of elements in the result will
reflect the ordering of the arguments in any particular way.
The implementation is therefore free to use any of a variety of strategies.
The result list may share cells with, or be \cdf{eq} to, either of the arguments
if appropriate.

In general, the test may be any predicate, and the \cdf{set-exclusive-or} operation
may be described as follows.  For all possible ordered pairs consisting of
one element from \emph{list1} and one element from \emph{list2}, the test is
used to determine whether they ``match.''  The result contains precisely
those elements of \emph{list1} and \emph{list2} that appear in no matching pair.

\cdf{nset-exclusive-or} is the destructive version of \cdf{set-exclusive-or}.
Both lists may be destroyed in producing the result.

\begin{new}
X3J13 voted in January 1989
\issue{MAPPING-DESTRUCTIVE-INTERACTION}
to restrict user side effects; see section~\ref{STRUCTURE-TRAVERSAL-SECTION}.
\end{new}

\begin{newer}
X3J13 voted in March 1989 \issue{REMF-DESTRUCTION-UNSPECIFIED}
to clarify the permissible side effects of certain operations;
\cdf{nset-exclusive-or} is permitted to perform a \cdf{setf} on any part,
\emph{car} or \emph{cdr}, of the top-level list structure of 
any of the argument lists.
\end{newer}
\end{defun}

\begin{defun}[Function]
subsetp list1 list2 &key :test :test-not :key

\cdf{subsetp} is a predicate that is true if every element of \emph{list1}
appears in (``matches'' some element of) \emph{list2}, and false otherwise.

\begin{new}
X3J13 voted in January 1989
\issue{MAPPING-DESTRUCTIVE-INTERACTION}
to restrict user side effects; see section~\ref{STRUCTURE-TRAVERSAL-SECTION}.
\end{new}
\end{defun}

\section{Association Lists}

An \emph{association list}, or \emph{a-list}, is a data structure
used very frequently in Lisp.  An a-list is a list of pairs (conses);
each pair is an association.  The \emph{car} of a pair is called the \emph{key},
and the \emph{cdr} is called the \emph{datum}.

An advantage of the a-list representation is that an a-list can be
incrementally augmented simply by adding new entries to the front.
Moreover, because the searching function \cdf{assoc} searches the
a-list in order, new entries can ``shadow'' old entries.  If an a-list is
viewed as a mapping from keys to data, then the mapping can be not only
augmented but also altered in a non-destructive manner by adding new
entries to the front of the a-list.

Sometimes an a-list represents a bijective mapping, and it is desirable
to retrieve a key given a datum.  For this purpose, the ``reverse'' searching
function \cdf{rassoc} is provided.  Other variants of a-list searches
can be constructed using the function \cdf{find} or \cdf{member}.

It is permissible to let {\false} be an element of an a-list in place of
a pair.  Such an element is not considered to be a pair but is simply
passed over when the a-list is searched by \cdf{assoc}.

\begin{defun}[Function]
acons key datum a-list

\cdf{acons} constructs a new association list by adding the pair
\cd{(\emph{key} . \emph{datum})} to the old \emph{a-list}.
\begin{lisp}
(acons x y a) \EQ\ (cons (cons x y) a)
\end{lisp}
\end{defun}

\begin{defun}[Function]
pairlis keys data &optional a-list

\cdf{pairlis} takes two lists and makes an association list that associates
elements of the first list to corresponding elements of the second
list.  It is an error if the two lists \emph{keys} and \emph{data} are not of
the same length.  If the optional argument \emph{a-list} is provided, then the
new pairs are added to the front of it.

The new pairs may appear in the resulting a-list in any order;
in particular, either forward or backward order is permitted.
Therefore the result of the call
\begin{lisp}
(pairlis '(one two) '(1 2) '((three . 3) (four . 19)))
\end{lisp}
might be
\begin{lisp}
((one . 1) (two . 2) (three . 3) (four . 19))
\end{lisp}
but could equally well be
\begin{lisp}
((two . 2) (one . 1) (three . 3) (four . 19))
\end{lisp}
\end{defun}

\begin{defun}[Function]
assoc item a-list &key :test :test-not :key \\
assoc-if predicate a-list &key :key \\
assoc-if-not predicate a-list &key :key

Each of these searches the association list
\emph{a-list}.  The value is the first pair in the a-list such that
the \emph{car} of the pair satisfies the test, or {\false} if there is
no such pair in the a-list.
For example:
\begin{lisp}
(assoc 'r '((a . b) (c . d) (r . x) (s . y) (r . z))) \\
~~~~~~~~\EV\  (r . x) \\
(assoc 'goo '((foo . bar) (zoo . goo))) \EV\ {\false} \\
(assoc '2 '((1 a b c) (2 b c d) (-7 x y z))) \EV\ (2 b c d)
\end{lisp}
It is possible to \cdf{rplacd} the result of \cdf{assoc} \emph{provided}
that it is not {\false},
in order to ``update'' the ``table'' that was \cdf{assoc}'s second argument.
(However, it is often better to update an a-list by adding new pairs
to the front, rather than altering old pairs.)
For example:
\begin{lisp}
(setq values '((x . 100) (y . 200) (z . 50))) \\
(assoc 'y values) \EV\ (y . 200) \\
(rplacd (assoc 'y values) 201) \\
(assoc 'y values) \EV\ (y . 201) \textrm{now}
\end{lisp}
A typical trick is to say
\cd{(cdr (assoc x y))}.
Because the \emph{cdr} of {\false} is guaranteed to be {\false},
this yields {\false} if no pair is found \emph{or} if a pair is
found whose \emph{cdr} is {\false}.  This is useful if {\false} serves
its usual role as a ``default value.''

The two expressions
\begin{lisp}
(assoc \emph{item} \emph{list} :test \emph{fn})
\end{lisp}
and
\begin{lisp}
(find \emph{item} \emph{list} :test \emph{fn} :key \#'car)
\end{lisp}
are equivalent in meaning with one important exception:
if {\nil} appears in the a-list in place of a pair,
and the \emph{item} being searched for is {\nil},
\cdf{find} will blithely compute the \emph{car} of the {\nil} in the a-list,
find that it is equal to the \emph{item}, and return {\nil},
whereas \cdf{assoc} will ignore the {\nil} in the a-list and continue
to search for an actual pair (cons) whose \emph{car} is {\nil}.
See \cdf{find} and \cdf{position}.

\begin{new}
X3J13 voted in January 1989
\issue{MAPPING-DESTRUCTIVE-INTERACTION}
to restrict user side effects; see section~\ref{STRUCTURE-TRAVERSAL-SECTION}.
\end{new}
\end{defun}

\begin{defun}[Function]
rassoc item a-list &key :test :test-not :key \\
rassoc-if predicate a-list &key :key \\
rassoc-if-not predicate a-list &key :key

\cdf{rassoc} is the reverse form of \cdf{assoc}; it searches for
a pair whose \emph{cdr} satisfies the test, rather than the \emph{car}.
If the \emph{a-list} is considered to be a mapping, then \cdf{rassoc}
treats the \emph{a-list} as representing the inverse mapping.
For example:
\begin{lisp}
(rassoc 'a '((a . b) (b . c) (c . a) (z . a))) \EV\ (c . a)
\end{lisp}

The expressions
\begin{lisp}
(rassoc \emph{item} \emph{list} :test \emph{fn})
\end{lisp}
and
\begin{lisp}
(find \emph{item} \emph{list} :test \emph{fn} :key \#'cdr)
\end{lisp}
are equivalent in meaning, except when the \emph{item} is {\nil}
and {\nil} appears in place of a pair in the a-list.  See the discussion
of the function \cdf{assoc}.

\begin{new}
X3J13 voted in January 1989
\issue{MAPPING-DESTRUCTIVE-INTERACTION}
to restrict user side effects; see section~\ref{STRUCTURE-TRAVERSAL-SECTION}.
\end{new}
\end{defun}

%RUSSIAN
\else

\chapter{Списки}

\emph{cons}-ячейка, или пара с точкой, является составным объектом данных,
содержащим два элемента, называемых \emph{car} и \emph{cdr}. Каждый компонент
может является любым Lisp'овым объектом.
\emph{Список (list)} является цепочкой cons-ячеек, соединённых \emph{cdr}
элементами.
Цепочка завершается некоторым не-cons объектом (atom object).
Обычный список завершается объектом {\nil}, или, как его ещё называют, пустым
списком {\emptylist}.
Список, цепочка которого завершается некоторым не-{\nil} объектом, называется
\emph{списком с точкой}.

Для проверки конца списка служит предикат \cdf{endp}.

\section{Cons-ячейки}

Ниже представлены несколько основных операций над cons-ячейками. В данных
случаях cons-ячейки рассматриваются как пары, а не компоненты списка.

\begin{defun}[Функция]
car list

Функция возвращает \emph{car} элемент списка \emph{list}, который должен быть
или \emph{cons} ячейкой или {\emptylist}. То есть аргумент должен удовлетворять
предикату \cdf{listp}.
По определению \emph{car} элемент пустого списка является пустым списком.
Если cons-ячейка является первым звеном списке, то можно сказать, что \cdf{car}
возвращает первый элемент списка.
Например:
\begin{lisp}
(car '(a b c)) \EV\ a
\end{lisp}
Смотрите \cdf{first}.

\emph{car} элемент cons-ячейки может быть изменён с помощью \cdf{rplaca} или
\cdf{setf}.
\end{defun}

\begin{defun}[Функция]
cdr list

Функция возвращает \emph{cdr} элемент списка \emph{list}, который должен быть
или \emph{cons} ячейкой или {\emptylist}. То есть аргумент должен удовлетворять
предикату \cdf{listp}.
По определению \emph{cdr} элемент пустого списка является пустым списком.
Если cons-ячейка является первым звеном списке, то можно сказать, что \cdf{cdr}
возвращает остаток исходного списка без первого элемента.
Например:
\begin{lisp}
(cdr '(a b c)) \EV\ (b c)
\end{lisp}
Смотрите \cdf{rest}.

\emph{cdr} элемент cons-ячейки может быть изменён с помощью \cdf{rplacd} или
\cdf{setf}.
\end{defun}

\begin{defun}[Функция]
caar list \\
cadr list \\
cdar list \\
cddr list \\
caaar list \\
caadr list \\
cadar list \\
caddr list \\
cdaar list \\
cdadr list \\
cddar list \\
cdddr list \\
caaaar list \\
caaadr list \\
caadar list \\
caaddr list \\
cadaar list \\
cadadr list \\
caddar list \\
cadddr list \\
cdaaar list \\
cdaadr list \\
cdadar list \\
cdaddr list \\
cddaar list \\
cddadr list \\
cdddar list \\
cddddr list

Все композиции до четырёх \cdf{car} и \cdf{cdr} операций определены как
самостоятельные функции.
Их имена начинаются с \cdf{c} и заканчиваются \cdf{r}. Середина имени содержит
последовательность букв \cdf{a} и \cdf{d} в соответствие с композиций выполняемых
этими функциями.
Например:
\begin{lisp}
(cddadr x) \textrm{is the same as} (cdr (cdr (car (cdr x))))
\end{lisp}
Если в качестве аргумента указан список, тогда \cdf{cadr} вернёт второй элемент
списка, \cdf{caddr} --- третий, и \cdf{cadddr} --- четвёртый. Если первый
элемент списка является списком, тогда \cdf{caar} вернёт первый элемент этого
подсписка, \cdf{cdar} --- остаток подсписка без первого элемента, \cdf{cadar} --- второй
элемент подсписка и так далее.

В целях стиля, предпочтительнее определить функцию или макрос для
доступа к части сложной структуры данных, а не использовать длинные
\cdf{car}/\cdf{cdr} строки. Например, можно определить макрос для получения
списка параметров лямбда-выражения:
\begin{lisp}
(defmacro lambda-vars (lambda-exp) `(cadr ,lambda-exp))
\end{lisp}
и затем использовать \cdf{lambda-vars} вместо \cdf{cadr}.
Смотрите также \cdf{defstruct}, которая будет автоматически определять новые
типы данных и функции доступа к частям экземпляров этих структур.

Любая из этих функций может использоваться в связке с \cdf{setf}.
\end{defun}
	
\begin{defun}[Функция]
cons x y

\cdf{cons} является (примитивной) функцией для создания новой
\emph{cons}-ячейки, у которой \emph{car} элемент будет \emph{x}, а \emph{cdr}
элемент --- \emph{y}.
Например:
\begin{lisp}
(cons 'a 'b) \EV\ (a . b) \\
(cons 'a (cons 'b (cons 'c '{\emptylist}))) \EV\ (a b c) \\
(cons 'a '(b c d)) \EV\ (a b c d)
\end{lisp}
\cdf{cons} может рассматриваться как для создания \emph{cons}-ячейки, так и для
добавления нового элемента в начало списка.
\end{defun}

\begin{defun}[Функция]
tree-equal x y &key :test :test-not

Данный предикат истинен, если \emph{x} и \emph{y} являются изоморфными деревьями
с идентичными листьями, то есть, если \emph{x} и \emph{y} являются атомами,
которые удовлетворяют проверке (по-умолчанию \cdf{eql}),
или если они оба являются cons-ячейками и их \emph{car} элементы удовлетворяют
\cdf{tree-equal} и \emph{cdr} элементы удовлетворяют \cdf{tree-equal}.
Таким образом \cdf{tree-equal} рекурсивно сравнивает cons-ячейки (но не любой
другой составной объект). Смотрите \cdf{equal}, который рекурсивно сравнивает
другие составные объекты, как например строки.

\begin{new}
X3J13 voted in January 1989
\issue{MAPPING-DESTRUCTIVE-INTERACTION}
to restrict user side effects; see section~\ref{STRUCTURE-TRAVERSAL-SECTION}.
\end{new}
\end{defun}

\section{Списки}

Следующие функции выполняет различные операции над списками.

Список является одним из первых Lisp'овых типов данных. Имя <<Lisp>>
расшифровывается
как <<LISt Processing>>.

\begin{defun}[Функция]
endp object

Предикат \cdf{endp} используется для проверки конца списка. Возвращает ложь для
cons-ячеек, истину для {\nil}, и генерирует ошибку для всех остальных объектов
других типов.

\beforenoterule
\begin{implementation}
Implementations are encouraged to signal an
error, especially in the interpreter, for a non-list argument.
The \cdf{endp} function is defined so as to allow compiled code
to perform simply an atom check or a null check if speed is more
important than safety.
\end{implementation}
\afternoterule
\end{defun}

\begin{defun}[Функция]
list-length list

\cdf{list-length} возвращает длину списка \emph{list}.
\cdf{list-length} отличается от \cdf{length} при использовании с циклическим
списком. В таком случае \cdf{length} может не вернуть управление, тогда как
\cdf{list-length} вернёт {\nil}.
Например:
\begin{lisp}
(list-length '{\emptylist}) \EV\ 0 \\
(list-length '(a b c d)) \EV\ 4 \\
(list-length '(a (b c) d)) \EV\ 3 \\
(let ((x (list 'a b c))) \\
~~(rplacd (last x) x) \\
~~(list-length x)) \EV\ {\nil}
\end{lisp}
\cdf{list-length} может быть реализован так:
\begin{lisp}
(defun list-length (x) \\
~~(do ((n 0 (+ n 2))~~~~~~~~~~~~;\textrm{Счётчик} \\
~~~~~~~(fast x (cddr fast))~~~~~;\textrm{Быстрый указатель: на две позиции вперёд} \\
~~~~~~~(slow x (cdr slow)))~~~~~;\textrm{Медленный указатель: на одну позицию} \\
~~~~~~(nil) \\
~~~~;; Если быстрый указатель дошёл до конца, вернуть длину. \\
~~~~(when (endp fast) (return n)) \\
~~~~(when (endp (cdr fast)) (return (+ n 1))) \\
~~~~;; If fast pointer eventually equals slow pointer, \\
~~~~;;  then we must be stuck in a circular list. \\
~~~~;; (A deeper property is the converse: if we are \\
~~~~;;  stuck in a circular list, then eventually the \\
~~~~;;  fast pointer will equal the slow pointer. \\
~~~~;;  That fact justifies this implementation.) \\
~~~~(when (and (eq fast slow) (> n 0)) (return nil))))
\end{lisp}
Смотрите \cdf{length}, которая возвращает длину любой последовательности.
\end{defun}

\begin{defun}[Функция]
nth n list

\cd{(nth \emph{n} \emph{list})} возвращает \emph{n}-нный элемент списка
\emph{list}. \emph{car} элемент списка принимается за <<нулевой>> элемент.
Аргумент \emph{n} должен быть неотрицательным целым числом.
Если длина списка не больше чем \emph{n}, тогда результат {\emptylist}, или
другими словами {\nil}.
(Это согласовывается с концепцией того, что \emph{car} и \emph{cdr} от
{\emptylist} являются {\emptylist}.)
Например:
\begin{lisp}
(nth 0 '(foo bar gack)) \EV\ foo \\
(nth 1 '(foo bar gack)) \EV\ bar \\
(nth 3 '(foo bar gack)) \EV\ {\emptylist}
\end{lisp}

\cdf{nth} может быть использован в связке с \cdf{setf} для изменения элемента
списка. В этом случае, аргумент \emph{n} должен быть меньше чем длина списка
\emph{list}.

Следует отметить, что порядок аргументов в \cdf{nth} обратный в отличие от
большинства других функций селекторов для последовательностей, таких ка
\cdf{elt}.
\end{defun}

\begin{defun}[Функция]
first list \\
second list \\
third list \\
fourth list \\
fifth list \\
sixth list \\
seventh list \\
eighth list \\
ninth list \\
tenth list

Иногда эти функции удобно использовать для доступа к определёнными элементам
списка.
\cdf{first} то же, что и \cdf{car}, \cdf{second} то же, что и \cdf{cadr},
\cdf{third} то же, что и \cdf{caddr}, и так далее.
Следует отметить, что нумерация начинается с единицы (first) в отличие от
нумерации, которая начинается с нуля и используется в \cdf{nth}.
\begin{lisp}
(fifth x) \EQ\ (nth 4 x)
\end{lisp}

Каждая из этих функций может быть использована в связке \cdf{setf} для изменения
элемента массива. 
\end{defun}

\begin{defun}[Функция]
rest list

\cdf{rest} означает то же, что и \cdf{cdr}, но мнемонически согласуется с
\cdf{first}.
\cdf{rest} может использоваться в связке с \cdf{setf} для изменения элементов
массива.
\end{defun}

\begin{defun}[Функция]
nthcdr n list

\cd{(nthcdr \emph{n} \emph{list})} выполняет для
списка \emph{lisp} операцию \cdf{cdr} \emph{n} раз, и возвращает результат.
Например:
\begin{lisp}
(nthcdr 0 '(a b c)) \EV\ (a b c) \\
(nthcdr 2 '(a b c)) \EV\ (c) \\
(nthcdr 4 '(a b c)) \EV\ {\emptylist}
\end{lisp}
Другими словами, она возвращает \emph{n}-нную \emph{cdr} часть списка.

\begin{lisp}
(car (nthcdr n x)) \EQ\ (nth n x)
\end{lisp}
Аргумент \emph{n} должен быть неотрицательным целым числом.
\end{defun}

\begin{defun}[Функция]
last list &optional (n 1)

\cdf{last} возвращает последние \emph{n} cons-ячеек списка \emph{lisp}. Список
\emph{list} может быть списком с точкой. Передача зацикленного списка является
ошибкой.

Аргумент \emph{n} должен быть неотрицательным целым числом.
Если \emph{n} равен нулю, тогда возвращается последний атом списка
\emph{list}. Если \emph{n} не меньше чем количество cons-ячеек, то возвращается
весь список.

Например:
\begin{lisp}
(setq x '(a b c d)) \\
(last x) \EV\ (d) \\
(rplacd (last x) '(e f)) \\
x \EV\ '(a b c d e f) \\
(last x 3) \EV\ (d e f) \\
(last '()) \EV\ () \\
(last '(a b c . d)) \EV\ (c . d) \\
(last '(a b c . d) 0) \EV\ d \\
(last '(a b c . d) 2) \EV\ (b c . d) \\
(last '(a b c . d) 1729) \EV\ (a b c . d)
\end{lisp}
\end{defun}

\begin{defun}[Функция]
list &rest args

\cdf{list} создаёт и возвращает список, составленный из аргументов.
Например:
\begin{lisp}
(list 3 4 'a (car '(b . c)) (+ 6 -2)) \EV\ (3 4 a b 4)
\end{lisp}

\begin{lisp}
(list) \EV\ () \\
(list (list 'a 'b) (list 'c 'd 'e)) \EV\ ((a b) (c d e))
\end{lisp}
\end{defun}

\begin{defun}[Функция]
list* arg &rest others

\cdf{list*} похожа на \cdf{list} за исключением того, что последняя
\emph{cons}-ячейка создаваемого списка будет <<с точкой>>. Последний аргумент
используется как последний элемент списка, а именно в последней cons-ячейки в
\emph{cdr} элементе. Данный аргумент необязательно должен быть атомом, и если он
не атом, то в результате список будет иметь большую длину чем количество
аргументов.
Например:
\begin{lisp}
(list* 'a 'b 'c 'd) \EV\ (a b c . d)
\end{lisp}
Это то же, что и
\begin{lisp}
(cons 'a (cons 'b (cons 'c 'd)))
\end{lisp}
А также:
\begin{lisp}
(list* 'a 'b 'c '(d e f)) \EV\ (a b c d e f) \\*
(list* x) \EQ\ x
\end{lisp}
\end{defun}

\begin{defun}[Функция]
make-list size &key :initial-element

Функция создаёт и возвращает список содержащий количество \emph{size} элементов,
каждый из которых будет инициализирован значением аргумента
\cd{:initial-element} (который по-умолчанию {\false}).
\emph{size} должен быть неотрицательным целым числом.
Например:
\begin{lisp}
(make-list 5) \EV\ ({\false} {\false} {\false} {\false} {\false}) \\
(make-list 3 \cd{:initial-element} 'rah) \EV\ (rah rah rah)
\end{lisp}
\end{defun}

\begin{defun}[Функция]
append &rest lists

Функция возвращает список содержащий все элементы указанных в аргументах списков.
Аргументы не разрушаются.
Например:
\begin{lisp}
(append '(a b c) '(d e f) '{\emptylist} '(g)) \EV\ (a b c d e f g)
\end{lisp}
Следует отметить, что \cdf{append} копирует верхний уровень всех переданных
списков \emph{за исключением} последнего.
Функция \cdf{concatenate} выполняет похожую операцию, но всегда копирует все
аргументы. Смотрите также \cdf{nconc}, которая похожа на \cdf{append}, но
разрушает все аргументы кроме последнего.

Последний аргумент может быть любым Lisp объектом, и в этом случае этот объект
становится последним элементов итогового списка.
Например, \cd{(append '(a b c) 'd)} \EV\ \cd{(a b c . d)}.

\cd{(append \emph{x} '{\emptylist})} может быть использовано для копирования
списка \emph{x}, однако для этого больше подходит функция \cdf{copy-list}.
\end{defun}

\begin{defun}[Функция]
copy-list list

Функция возвращает список, который равен \cdf{equal} и в то же время не равен
\cdf{eq} списку \emph{list}, 
Копируется только верхний уровень списка, то есть \cdf{copy-list} копирует
только в направлении \emph{cdr} элементов, но не в направлении \emph{car}
элементов.
Если список <<с точкой>>, то есть \cd{(cdr (last \emph{list}))} является
не-{\nil} атомом, тогда итоговый список также будет <<с точкой>>.
Смотрите также \cdf{copy-seq} и \cdf{copy-tree}.
\end{defun}

\begin{defun}[Функция]
copy-alist list

\cdf{copy-alist} копирует ассоциативные списки. При этом, также как и в
\cdf{copy-list}, копируется только верхний уровень списка \emph{lisp}.
Кроме того, каждый элемент списка \emph{list}, являющийся в свою очередь
cons-ячейкой, заменяется новой cons-ячейкой с теми же \emph{car} и \emph{cons}
элементами.
\end{defun}

\begin{defun}[Функция]
copy-tree object

\cdf{copy-tree} копирует древовидно организованные cons-ячейки.
Аргумент \emph{object} может быть любым Lisp'овым объектом.
Если он не является cons-ячейкой, то ничего не произойдёт и данный объект будет
возвращён в качестве результата. В противном случае будет возвращена новая
cons-ячейка, в которой \emph{car} и \emph{cons} элементы будут результатами
рекурсивных вызовов \cdf{copy-tree}. Другими словами, все cons-ячейки будут
рекурсивно скопированы, и рекурсия будет останавливаться только на атомах.
\end{defun}

\begin{defun}[Функция]
revappend x y

\cd{(revappend \emph{x} \emph{y})} похожа на \cd{(append (reverse \emph{x})
   \emph{y})} за исключением того, что она потенциально более
 производительна. Аргументы \emph{x} и \emph{y} должны быть списками.
Аргумент \emph{x} копируется (не разрушается) в отличие от \cdf{nreconc},
которая разрушает первый аргумент.
\end{defun}

\begin{defun}[Функция]
nconc &rest lists

В качестве аргументов \cdf{nconc} принимает списки. Функция соединяет списки и
возвращает результат. При этом аргументы изменяются, а не копируются.
(В сравнении с \cdf{append}, которая копирует аргументы, а не разрушает их.)
Например:
\begin{lisp}
(setq x '(a b c)) \\
(setq y '(d e f)) \\
(nconc x y) \EV\ (a b c d e f) \\
x \EV\ (a b c d e f)
\end{lisp}
Следует отметить, что в примере, значение \cdf{x} отличается от первоначального,
так как последняя cons-ячейка была изменена с помощью \cdf{rplacd} значением
\cdf{y}.
Если сейчас выполнить \cd{(nconc x y)} ещё раз, тогда часть списка зациклится:
\cd{(a b c d e f d e f d e f ...)}, и так до бесконечности.
Если \cdf{*print-circle*} не равен {\nil}, тогда вывод списка будет таким:
\cd{(a b c . \#1=(d e f . \#1\#))}.

Вызов \cdf{nconc}, совпадающий
с наиболее близким шаблоном выражения в левой части приводит к эквивалентным
побочным действиям, как в правой части таблицы.
\begin{flushleft}
\begin{tabular}{@{}ll@{}}
\cd{(nconc)}&\cd{nil~~~~~;}\textrm{Нет побочных эффектов} \\
\cd{(nconc nil . \emph{r})~~~~}&\cd{(nconc . \emph{r})} \\
\cd{(nconc \emph{x})}&\emph{x} \\
\cd{(nconc \emph{x} \emph{y})}&\cd{(let ((p \emph{x}) (q \emph{y}))} \\
                                  &\cd{~~(rplacd (last p) q)} \\
                                  &\cd{~~p)} \\
\cd{(nconc \emph{x} \emph{y} . \emph{r})}&\cd{(nconc (nconc \emph{x} \emph{y})
  . \emph{r})}
\end{tabular} 
\end{flushleft}
\end{defun}

\penalty-10000 %manual

\begin{defun}[Функция]
nreconc x y

\cd{(nreconc \emph{x} \emph{y})} похожа на \cd{(nconc (nreverse \emph{x})
  \emph{y})} за исключением того, что она потенциально эффективнее.
Оба аргумента должны быть списками. 
Аргумент \emph{x} разрушается.
Сравните с \cdf{revappend}.

\begin{lisp}
(setq planets '(jupiter mars earth venus mercury)) \\
(setq more-planets '(saturn uranus pluto neptune)) \\
(nreconc more-planets planets) \\
\`\EV\ (neptune pluto uranus saturn jupiter mars earth venus mercury) \\
~~\textrm{теперь значение \cdf{more-planets} точно не определено}
\end{lisp}

Поведение \cd{(nreconc \emph{x} \emph{y})} совпадает с поведением \cd{(nconc
  (nreverse \emph{x})~\emph{y})} в части побочных эффектов.
\end{defun}

\begin{defmac}
push item place

Форма \emph{place} должна быть именем обобщённое переменной, содержащей
список. \emph{item} может указывать на любой Lisp'овый объект. \emph{item}
вставляется в начало списка и данный список возвращается в качестве результата.
Форма \emph{place} может
быть любой формой, которая подходит для \cdf{setf}.
Если рассматривать список как стек, тогда \cdf{push} добавляет элемент на
вершину стека.
Например:
\begin{lisp}
(setq x '(a (b c) d)) \\
(push 5 (cadr x)) \EV\ (5 b c)  \textrm{и теперь} x \EV\ (a (5 b c) d)
\end{lisp}
Действие от
\begin{lisp}
(push \emph{item} \emph{place})
\end{lisp}
эквивалентно действию
\begin{lisp}
(setf \emph{place} (cons \emph{item} \emph{place}))
\end{lisp}
за исключением того, что \cdf{push} выполняет форму \emph{place} только один раз,
а не три.
Более того, в для некоторых форм \emph{place} \cdf{push} может быть
эффективнее чем версия с \cdf{setf}.

Следует отметить, что \emph{item} вычисляется прежде чем вычисляется
\emph{place}.
\end{defmac}

\begin{defmac}
pushnew item place &key :test :test-not :key

Форма \emph{place} должна быть именем обобщённое переменной, содержащей
список. \emph{item} может указывать на любой Lisp'овый объект. Если \emph{item}
не содержится в списке (этот факт устанавливается с помощью предиката
переданного в \cd{:test}, который по-умолчанию \cdf{eql}), тогда \emph{item}
вставляется в начало списка и данный список возвращается в качестве результата.
В противном случае возвращается исходный список. Форма \emph{place} может
быть любой формой, которая подходит для \cdf{setf}.
Если рассматривать список как множество, тогда \cdf{pushnew} добавляет элемент в
множество. Смотрите \cdf{adjoin}.

Именованные параметры \cdf{pushnew} имеют тот же смысл, что и в функциях для
последовательностей. Смотрите главу~\ref{KSEQUE}.
По сути, данные аргументы идентичны аргументам \cdf{adjoin}.

\cdf{pushnew} возвращает модифицированное содержимое переменной \emph{place}.
Например:
\begin{lisp}
(setq x '(a (b c) d)) \\
(pushnew 5 (cadr x)) \EV\ (5 b c)   \textrm{и теперь} x \EV\ (a (5 b c) d) \\
(pushnew 'b (cadr x)) \EV\ (5 b c)  \textrm{и \cdf{x} не меняется}
\end{lisp}
Действие от
\begin{lisp}
(pushnew \emph{item} \emph{place} \cd{:test} \emph{p})
\end{lisp}
эквивалентно действию
\begin{lisp}
(setf \emph{place} (adjoin \emph{item} \emph{place} \cd{:test} \emph{p}))
\end{lisp}
за исключением того, что \cdf{pushnew} выполняет форму \emph{place} только один раз,
а не три.
Более того, в для некоторых форм \emph{place} \cdf{pushnew} может быть
эффективнее чем версия с \cdf{setf}.

Следует отметить, что \emph{item} вычисляется прежде чем вычисляется
\emph{place}.
\end{defmac}

\begin{defmac}
pop place

Форма \emph{place} должна быть именем обобщённое переменной, содержащей
список. Результатом \cdf{pop} является результат \cdf{car} функции для
переданного списка, и побочным эффектом является то, что в обобщённую переменную
сохраняется результат \cdf{cdr} для списка.
Форма \emph{place} может быть любой формой, которая подходит для
\cdf{setf}. Если рассматривать исходный список как стек, то \cdf{pop} достаёт
элемент из вершины стека и возвращает его.
Например:
\begin{lisp}
(setq stack '(a b c)) \\
(pop stack) \EV\ a  \textrm{and now} stack \EV\ (b c)
\end{lisp}
Действия от \cd{(pop \emph{place})} эквивалентно
\begin{lisp}
(prog1 (car \emph{place}) (setf \emph{place} (cdr \emph{place})))
\end{lisp}
за исключением того, что \cdf{pop} выполняет форму \emph{place} только один раз,
а не три.
Более того, в для некоторых форм \emph{place} \cdf{pop} может быть
эффективнее чем версия с \cdf{setf}. 
\end{defmac}

\begin{defun}[Функция]
butlast list &optional n

Функция создаёт и возвращает список с такими же элементами кроме \emph{n}
последних, что и в списке \emph{list}.
\emph{n} по-умолчанию равно 1. Аргумент не разрушается.
Если длина списка \emph{list} меньше чем \emph{n}, тогда возвращается
{\emptylist}.
Например:
\begin{lisp}
(butlast '(a b c d)) \EV\ (a b c) \\
(butlast '((a b) (c d))) \EV\ ((a b)) \\
(butlast '(a)) \EV\ {\emptylist} \\
(butlast nil) \EV\ {\emptylist}
\end{lisp}
Имя функции образовано от фразы <<all elements but the last>> (<<все элементы
кроме последних>>).
\end{defun}

\begin{defun}[Функция]
nbutlast list &optional n

Это деструктивная версия \cdf{butlast}. Данная функция изменяет \emph{cdr}
элемент cons-ячейки на {\nil}. Искомая cons-ячейка находится на позиции
\emph{n}+1 с конца списка. Если длина списка \emph{list} меньше чем \emph{n},
тогда возвращается {\emptylist}, и аргумент не модифицируется. (Таким образом
можно написать \cd{(setq a (nbutlast a))}, а не \cd{(nbutlast a)}.)
Например:
\begin{lisp}
(setq foo '(a b c d)) \\
(nbutlast foo) \EV\ (a b c) \\
foo \EV\ (a b c) \\
(nbutlast '(a)) \EV\ {\emptylist} \\
(nbutlast '{\nil}) \EV\ {\emptylist}
\end{lisp}
\end{defun}

\begin{defun}[Функция]
ldiff list sublist

Аргумент \emph{list} должен быть списком, и \emph{sublist} должен быть
подсписком \emph{list}.
\cdf{ldiff} (означает <<list difference>>) возвращает новый список, элементы
которого содержат все элементы списка \emph{list} до подсписка
\emph{sublist}. Если \emph{sublist} не является частью \emph{list} (или в
частности равен {\nil}), тогда возвращается копия всего списка \emph{list}.
Аргумент \emph{list} не разрушается.
Например:
\begin{lisp}
(setq x '(a b c d e)) \\
(setq y (cdddr x)) \EV\ (d e) \\
(ldiff x y) \EV\ (a b c) \\[4pt]
\textrm{но} (ldiff '(a b c d) '(c d)) \EV\ (a b c d)
\end{lisp}
так как подсписок не равен \cdf{eq} ни одной части списка.
\end{defun}

\section{Изменение структуры списка}

Для изменения структуры уже имеющегося списка могут использоваться функции
\cdf{rplaca} и \cdf{rplacd}. Данные функции изменяют \emph{car} и \emph{cdr}
элементы cons-ячеек соответственно.
Можно также использовать \cdf{setf} в связке с \cdf{car} и \cdf{cdr}.

Структура списка деструктивно изменяется, а не копируется. Такое поведение может
оказаться неожиданным, особенно при использовании частей списком, на которые
указывают более одной переменной.
Описанные ранее функции \cdf{nconc}, \cdf{nreverse}, \cdf{nreconc},
и \cdf{nbutlast} также деструктивно изменяют список.
Данные функции имеют <<копирующие, а не разрушающие>> аналоги.

\begin{defun}[Функция]
rplaca x y

\cd{(rplaca \emph{x} \emph{y})} изменяет \emph{car} элемент в cons-ячейке
\emph{x} на \emph{y} и возвращает (модифицированную) \emph{x}. \emph{x} должен
быть cons-ячейкой, но \emph{y} может быть любым Lisp'овым объектом.
Например:
\begin{lisp}
(setq g '(a b c)) \\
(rplaca (cdr g) 'd) \EV\ (d c) \\
\textrm{Теперь} g \EV\ (a d c)
\end{lisp}
\end{defun}

\begin{defun}[Функция]
rplacd x y

\cd{(rplacd \emph{x} \emph{y})} изменяет \emph{cdr} элемент в cons-ячейке
\emph{x} на \emph{y} и возвращает (модифицированную) \emph{x}. \emph{x} должен
быть cons-ячейкой, но \emph{y} может быть любым Lisp'овым объектом.
Например:
\begin{lisp}
(setq x '(a b c)) \\
(rplacd x 'd) \EV\ (a . d) \\
\textrm{Теперь} x \EV\ (a . d)
\end{lisp}
\end{defun}

Функции \cdf{rplaca} и \cdf{rplacd} пришли из самых ранних версий Lisp'а, как
\cdf{car}, \cdf{cdr} и \cdf{cons}.
Однако, в настоящее время, они, похоже, отходят на второй план.
Все больше и больше Common Lisp программистов используют \cdf{setf} почти
для всех изменений структур: \cd{(rplaca x~y)} становится
\cd{(setf (car x)~y)} или, возможно, \cd{(setf (first x)~y)}.
Ещё более вероятно, что структура \cdf{defstruct} или CLOS класс используют
вместо списком, если структура данных слишком сложная.
В таком случае \cdf{setf} используется в связке с функцией доступа к слоту.

\section{Замещение выражений}

\indexterm{substitution}
Для выполнения операции замещения в древовидной структуре cons-ячеек
предоставлен ряд функций. Все эти функции принимают дерево и описание того, что
на что необходимо заменить. Функции имеют копирующие и деструктивные версии, а
также версии в которых замещение описывается либо двумя аргументами, либо
ассоциативным списком.

Правила именования для этих функций и для их именованных параметров совпадают с
правилами функций для последовательностей. Смотрите раздел~\ref{KSEQUE}.

\begin{defun}[Функция]
subst new old tree &key :test :test-not :key \\
subst-if new test tree &key :key \\
subst-if-not new test tree &key :key

\cd{(subst \emph{new} \emph{old} \emph{tree})} создаёт копию дерева \emph{tree},
замещая элемент \emph{old} элементом \emph{new}. Замещение происходит в любом
месте дерева. Функция возвращает модифицированную копию дерева \emph{tree}.
Исходный объект \emph{tree} не изменяется, но итоговое дерево может иметь
общие с исходным части.

Например:
\begin{lisp}
(subst 'tempest 'hurricane \\
~~~~~~~'(shakespeare wrote (the hurricane))) \\
~~~\EV\ (shakespeare wrote (the tempest)) \\
\\
(subst 'foo '{\nil} '(shakespeare wrote (twelfth night))) \\
~~~\EV\ (shakespeare wrote (twelfth night . foo) . foo) \\
\\
(subst '(a . cons) '(old . pair) \\
~~~~~~~'((old . spice) ((old . shoes) old . pair) (old . pair)) \\
~~~~~~~\cd{:test} \#'equal) \\
~~~\EV\ ((old . spice) ((old . shoes) a . cons) (a . cons))
\end{lisp}
Эта функция не является деструктивной. Она не изменяет \cdf{car} и \cdf{cdr}
элементы уже существующего дерева.
Можно определить \cdf{subst} так:
\begin{lisp}
(defun subst (old new tree \cd{\&rest} x \cd{\&key} test test-not key) \\*
~~(cond ((satisfies-the-test old tree :test test \\*
~~~~~~~~~~~~~~~~~~~~~~~~~~~~~:test-not test-not :key key) \\*
~~~~~~~~~new) \\*
~~~~~~~~((atom tree) tree) \\
~~~~~~~~(t (let ((a (apply \#'subst old new (car tree) x)) \\*
~~~~~~~~~~~~~~~~~(d (apply \#'subst old new (cdr tree) x))) \\
~~~~~~~~~~~~~(if (and (eql a (car tree)) \\*
~~~~~~~~~~~~~~~~~~~~~~(eql d (cdr tree))) \\*
~~~~~~~~~~~~~~~~~tree \\*
~~~~~~~~~~~~~~~~~(cons a d))))))
\end{lisp}
Смотрите также \cdf{substitute}, которая проводит замещение только для верхнего
уровня списка.
\end{defun}

\begin{defun}[Функция]
nsubst new old tree &key :test :test-not :key \\
nsubst-if new test tree &key :key \\
nsubst-if-not new test tree &key :key

\cdf{nsubst} является деструктивным аналогом \cdf{subst}. В дереве \emph{tree}
любой элемент \emph{old} заменяется на \emph{new}.

\begin{new}
X3J13 voted in January 1989
\issue{MAPPING-DESTRUCTIVE-INTERACTION}
to restrict user side effects; see section~\ref{STRUCTURE-TRAVERSAL-SECTION}.
\end{new}
\end{defun}

\begin{defun}[Функция]
sublis alist tree &key :test :test-not :key

\cdf{sublis} выполняет замещение объектов в дереве (древовидной структуре из
cons-ячеек).
Первый аргумент \cdf{sublis} является ассоциативным списком.
Второй аргумент --- дерево, в котором выполняется замещение.
\cdf{sublis} проходит по всему дереву включая листья, и если элемент встречается
в качестве ключа в ассоциативном списке, то данный элемент заменяет на значение
ключа.
Данная операция не разрушает дерево. \cdf{sublis} может выполнять несколько
\cdf{subst} операций за один раз.
Например:
\begin{lisp}
(sublis '((x . 100) (z . zprime)) \\*
~~~~~~~~'(plus x (minus g z x p) 4 . x)) \\*
~~~\EV\ (plus 100 (minus g zprime 100 p) 4 . 100) \\*
 \\*
(sublis '(((+ x y) . (- x y)) ((- x y) . (+ x y))) \\*
~~~~~~~~'(* (/ (+ x y) (+ x p)) (- x y)) \\*
~~~~~~~~:test \#'equal) \\*
~~~\EV\ (* (/ (- x y) (+ x p)) (+ x y))
\end{lisp}

\begin{new}
X3J13 voted in January 1989
\issue{MAPPING-DESTRUCTIVE-INTERACTION}
to restrict user side effects; see section~\ref{STRUCTURE-TRAVERSAL-SECTION}.
\end{new}
\end{defun}

\begin{defun}[Функция]
nsublis alist tree &key :test :test-not :key

\cdf{nsublis} похожа на \cdf{sublis} но деструктивно модифицирует необходимые
элементы дерева \emph{tree}.

\begin{new}
X3J13 voted in January 1989
\issue{MAPPING-DESTRUCTIVE-INTERACTION}
to restrict user side effects; see section~\ref{STRUCTURE-TRAVERSAL-SECTION}.
\end{new}
\end{defun}

\section{Использование списков как множеств}

Common Lisp содержит функции, которые позволяют обрабатывать списки элементов
как \emph{множества}.
Сюда входят функции добавления, удаления и поиска элементов в списке,
основанного на различных критериях.
Кроме того, включены функции объединения, пересечения и разности.

Правила наименования данных функций и их именованных параметров в основном
следуют правилам именования функций для последовательностей. Смотрите
главу~\ref{KSEQUE}.

\begin{defun}[Функция]
member item list &key :test :test-not :key \\
member-if predicate list &key :key \\
member-if-not predicate list &key :key

Функция осуществляет поиск элемента, удовлетворяющего условию, в списке
\emph{list}.
Если элемент не найдёт, возвращается {\false}.
Иначе возвращается часть списка, начинающаяся с искомого элемента.
Поиск осуществляется только в верхнем уровне списка.
Эти функции могут использоваться в качестве предикатов.

Например:
\begin{lisp}
(member 'snerd '(a b c d)) \EV\ {\false} \\
(member-if \#'numberp '(a \#{\Xbackslash}Space 5/3 foo)) \EV\ (5/3 foo) \\
(member 'a '(g (a y) c a d e a f)) \EV\ (a d e a f)
\end{lisp}
Следует отметить, что в последнем примере значение, возвращённое \cdf{member},
равно \cdf{eq} части списка, которая начинается на \cdf{a}.
Если \cdf{member} вернула не {\false} значение, то для изменения полученного
элемента списка можно использовать \cdf{rplaca}.

Смотрите также \cdf{find} и \cdf{position}.

\begin{new}
X3J13 voted in January 1989
\issue{MAPPING-DESTRUCTIVE-INTERACTION}
to restrict user side effects; see section~\ref{STRUCTURE-TRAVERSAL-SECTION}.
\end{new}
\end{defun}

\begin{defun}[Функция]
tailp sublist list

\cdf{tailp} истинен тогда и только тогда, когда существует такое целое число
\emph{n}, что выполняется
\begin{lisp}
(eql \emph{sublist} (nthcdr \emph{n} \emph{list}))
\end{lisp}
\emph{list} может быть списком с точкой (подразумевается, что реализации могут
использовать \cdf{atom} и не могут \cdf{endp} для проверки конца списка
\emph{list}). FIXME
\end{defun}

\begin{defun}[Функция]
adjoin item list &key :test :test-not :key

\cdf{adjoin} используется для добавления элементов во множество, если этого
элемента во множестве ещё не было. Условие равенства по-умолчанию \cdf{eql}.
\begin{lisp}
(adjoin \emph{item} \emph{list}) \EQ\ (if (member \emph{item} \emph{list}) \emph{list} (cons \emph{item} \emph{list}))
\end{lisp}
Условие равенства может быть любым предикатом. \emph{item} добавляется в список
тогда и только тогда, когда в списке не было ни одного элемента,
<<удовлетворяющего условию>>.

\cdf{adjoin} отклоняется от обычных правил, описанных в главе~\ref{KSEQUE} в
части обработки параметров \emph{item} и \cd{:key}.
Если указана \cd{:key} функция, то она применяется к параметру \emph{item}, также
как и к каждому элементу списка. Обоснование в том, что если \emph{item} ещё не
был в списке и если он там появится, то применение функции \cd{:key} к нему как
элементу списка не будет корректным, если этого не было при его добавлении.
\begin{lisp}
(adjoin \emph{item} \emph{list} :key \emph{fn}) \\
~~\EQ\ (if (member (funcall \emph{fn} \emph{item}) \emph{list}
  :key \emph{fn}) \emph{list} (cons \emph{item} \emph{list})) 
\end{lisp}

Смотрите также \cdf{pushnew}.

\begin{new}
X3J13 voted in January 1989
\issue{MAPPING-DESTRUCTIVE-INTERACTION}
to restrict user side effects; see section~\ref{STRUCTURE-TRAVERSAL-SECTION}.
\end{new}
\end{defun}

\begin{defun}[Функция]
union list1 list2 &key :test :test-not :key \\
nunion list1 list2 &key :test :test-not :key

\cdf{union} принимает два списка и возвращает новый список, содержащий всё, что
является элементами списков \emph{list1} и \emph{list2}.
Если в списках есть дубликаты, то в итоговом будет только один экземпляр.
Например:
\begin{lisp}
(union '(a b c) '(f a d)) \\
~~~\EV\ (a b c f d) \textrm{или} (b c f a d) \textrm{или} (d f a b c) \textrm{или} ... \\
 \\
(union '((x 5) (y 6)) '((z 2) (x 4)) :key \#'car) \\
~~~\EV\ ((x 5) (y 6) (z 2)) \textrm{или} ((x 4) (y 6) (z 2)) \textrm{или} ...
\end{lisp}

Порядок элементов в итоговом списке не обязательно совпадает с порядком
соответствующих элементов в списках.
Итоговый список может иметь общие ячейки с или быть равным \cdf{eq} переданным
аргументам.

Функция \cd{:test} может быть любым предикатом, и операция объединения может
быть описана следующим образом. Для всех возможных упорядоченных пар, состоящих
из одного элемента из списка \emph{list1} и одного элемента из списка
\emph{list2}, предикат устанавливает <<равны>> ли они. Для каждой пары равных
элементов, как минимум один из двух элементов будет помещён в результат. Кроме
того, любой элемент, которые не был равен ни одному другому элементу, также
будет помещён в результат. Это описание может быть полезным при использовании
хитрых функций проверки равенства.

Аргумент \cd{:test-not} может быть полезен, когда функция проверки равенства
является логическим отрицанием проверки равенства. Хороший пример такой функции
это \cdf{mismatch}, которая логически инвертирована так, что если аргументы не
равны, то может быть получена возможная полезная информация. Эта дополнительная
<<полезная информация>> отбрасывается в следующем примере. \cdf{mismatch}
используется только как предикат.
\begin{lisp}
(union '(\#(a b) \#(5 0 6) \#(f 3)) \\
~~~~~~~'(\#(5 0 6) (a b) \#(g h)) \\
~~~~~~~:test-not \\
~~~~~~~\#'mismatch) \\
~~~\EV\ (\#(a b) \#(5 0 6) \#(f 3) \#(g h))~~~~~;\textrm{Возможный результат} \\
~~~\EV\ ((a b) \#(f 3) \#(5 0 6) \#(g h))~~~~~~;\textrm{Другой возможный результат}
\end{lisp}
Использование \cd{\cd{:test-not} \#'mismatch} отличается от использования
\cd{\cd{:test} \#'equalp}, например, потому что \cdf{mismatch} определяет что
\cd{\#(a b)} и \cd{(a b)} одинаковы, тогда как \cdf{equalp} определяет эти
выражения разными.

\cdf{nunion} является деструктивной версией \cdf{union}.
Она выполняет ту же операцию, но может разрушить аргументы, возможно при
использовании их ячеек для построения результата.

\begin{new}
X3J13 voted in January 1989
\issue{MAPPING-DESTRUCTIVE-INTERACTION}
to restrict user side effects; see section~\ref{STRUCTURE-TRAVERSAL-SECTION}.
\end{new}

\begin{newer}
X3J13 voted in March 1989 \issue{REMF-DESTRUCTION-UNSPECIFIED}
to clarify the permissible side effects of certain operations;
\cdf{nunion} is permitted to perform a \cdf{setf} on any part,
\emph{car} or \emph{cdr}, of the top-level list structure of 
any of the argument lists.
\end{newer}
\end{defun}

\begin{defun}[Функция]
intersection list1 list2 &key :test :test-not :key \\
nintersection list1 list2 &key :test :test-not :key

\cdf{intersection} принимает два списка и возвращает новый список содержащий все
элементы, которые есть и в первом и во втором списках одновременно.
Например:
\begin{lisp}
(intersection '(a b c) '(f a d)) \EV\ (a)
\end{lisp}

Порядок элементов в итоговом списке не обязательно совпадает с порядком
соответствующих элементов в списках.
Итоговый список может иметь общие ячейки с или быть равным \cdf{eq} переданным
аргументам.

Функция \cd{:test} может быть любым предикатом, и операция пересечения может
быть описана следующим образом. Для всех возможных упорядоченных пар, состоящих
из одного элемента из списка \emph{list1} и одного элемента из списка
\emph{list2}, предикат устанавливает <<равны>> ли они. Для каждой пары равных
элементов, только один из двух элементов будет помещён в результат. Больше
никаких элементов в итоговом списке не будет. Это описание может быть полезным
при использовании хитрых функций проверки равенства.

\cdf{nintersection} является является деструктивной версией \cdf{intersection}.
Она выполняет ту же операцию, но может разрушить аргумент \emph{list1}, возможно при
использовании их ячеек для построения результата. (Аргумент \emph{list2} не
разрушается.)

\begin{new}
X3J13 voted in January 1989
\issue{MAPPING-DESTRUCTIVE-INTERACTION}
to restrict user side effects; see section~\ref{STRUCTURE-TRAVERSAL-SECTION}.
\end{new}

\begin{newer}
X3J13 voted in March 1989 \issue{REMF-DESTRUCTION-UNSPECIFIED}
to clarify the permissible side effects of certain operations;
\cdf{nintersection} is permitted to perform a \cdf{setf} on any part,
\emph{car} or \emph{cdr}, of the top-level list structure of 
any of the argument lists.
\end{newer}
\end{defun}

\begin{defun}[Функция]
set-difference list1 list2 &key :test :test-not :key \\
nset-difference list1 list2 &key :test :test-not :key

\cdf{set-difference} возвращает список элементов списка \emph{list1}, которые не
встречаются в списке \emph{list2}. Данная операция не разрушает аргументы.

Порядок элементов в итоговом списке не обязательно совпадает с порядком
соответствующих элементов в списке \emph{list1}.
Итоговый список может иметь общие ячейки, или быть равным \cdf{eq} аргументу
\emph{list1}.

\emph{:test} может быть любым предикатом, и операция разности множеств может
быть описана следующим образом. Для всех возможных упорядоченных пар,
состоящих из элементов первого и второго списков, используется предикат для
установки их <<равенства>>. Элемент из списка \emph{list1} помещается в
результат, тогда и только тогда, когда он не равен ни одному элементу списка. Это
позволяет делать очень интересные приложения.
Например, можно удалить из списка строк все строки, содержащие некоторый список символов:
\emph{list2}.
\begin{lisp}
;; Удалить все имена специй содержащие буквы "c" или "w". \\
(set-difference '("strawberry" "chocolate" "banana" \\
~~~~~~~~~~~~~~~~~~"lemon" "pistachio" "rhubarb") \\
~~~~~~~~~~~~~~~~'(\#{\Xbackslash}c \#{\Xbackslash}w) \\
~~~~~~~~~~~~~~~~:test \\
~~~~~~~~~~~~~~~~\#'(lambda (s c) (find c s))) \\
~~~\EV\ ("banana" "rhubarb" "lemon")~~~~~;\textrm{Возможен другой порядок
  элементов}
\end{lisp}

\cdf{nset-difference} является деструктивной версией
\cdf{set-difference}. Данная операция может разрушить \emph{list1}.

\begin{new}
X3J13 voted in January 1989
\issue{MAPPING-DESTRUCTIVE-INTERACTION}
to restrict user side effects; see section~\ref{STRUCTURE-TRAVERSAL-SECTION}.
\end{new}
\end{defun}

\begin{defun}[Функция]
set-exclusive-or list1 list2 &key :test :test-not :key \\
nset-exclusive-or list1 list2 &key :test :test-not :key

\cdf{set-exclusive-or} возвращает список элементов, которые встречаются только в
списке \emph{list1} и только в списке \emph{list2}.
Данная операция не разрушает аргументы.

Порядок элементов в итоговом списке не обязательно совпадает с порядком
соответствующих элементов в списке \emph{list1}.
Итоговый список может иметь общие ячейки, или быть равным \cdf{eq} аргументу
\emph{list1}.

Функция проверки равенства элементов может быть любым предикатом, и операцию
\cdf{set-exclusive-or} можно описать следующим образом. Для всех возможных
упорядоченных пар, содержащих один элемент из списка \emph{list1} и один элемент
из списка \emph{list2}, функция используется для проверки
<<равенства>>. Результат содержит точно те элементы списков \emph{list1} и
\emph{list2}, которые были только в различающихся парах.

\cdf{nset-exclusive-or} является деструктивной версией
\cdf{set-exclusive-or}. Данная операция может разрушить аргументы.

\begin{new}
X3J13 voted in January 1989
\issue{MAPPING-DESTRUCTIVE-INTERACTION}
to restrict user side effects; see section~\ref{STRUCTURE-TRAVERSAL-SECTION}.
\end{new}

\begin{newer}
X3J13 voted in March 1989 \issue{REMF-DESTRUCTION-UNSPECIFIED}
to clarify the permissible side effects of certain operations;
\cdf{nset-exclusive-or} is permitted to perform a \cdf{setf} on any part,
\emph{car} or \emph{cdr}, of the top-level list structure of 
any of the argument lists.
\end{newer}
\end{defun}

\begin{defun}[Функция]
subsetp list1 list2 &key :test :test-not :key

\cdf{subsetp} является предикатом, который истинен, если каждый элемент списка
\emph{list1} встречается в (<<равен>> некоторому элементу в) списке
\emph{list2}, иначе ложен.

\begin{new}
X3J13 voted in January 1989
\issue{MAPPING-DESTRUCTIVE-INTERACTION}
to restrict user side effects; see section~\ref{STRUCTURE-TRAVERSAL-SECTION}.
\end{new}
\end{defun}

\section{Ассоциативные списки}

\emph{Ассоциативный список}, или \emph{a-list}, является структурой данных,
часто используемой в Lisp'е. a-list является списком пар (cons-ячеек). Каждая
пара является ассоциацией. \emph{car} элемент пары называется \emph{key}, и
\emph{cdr} элемент \emph{datum}.

Преимущество a-list представления данных в том, что a-list может быть
постепенно увеличен путём простого добавления в начало новых записей.
Кроме того, поскольку функция поиска \cdf{assoc} по порядку проходит элементы
a-list, то новые записи могут <<скрыть>> старые. Если a-list
рассматривать как отображение ключей в значения, то отображение может быть не
только увеличено, но также изменено с помощью добавления новых записей в начало
a-list.

Иногда a-list представляет биективное (bijective) отображение, и бывает нужно
получить ключ для некоторого значения. Для этих целей используется функция
<<реверсивного>> поиска \cdf{rassoc}. Другие варианты поиска в a-list могут быть
созданы с помощью функции \cdf{find} или \cdf{member}.

Допустимо, чтобы {\false} был элемент a-list вместо пары ключ-значение.
Такой элемент не считается парой и просто пропускается при использовании
функции \cdf{assoc}.

\begin{defun}[Функция]
acons key datum a-list

\cdf{acons} создаёт новый ассоциативный список, с помощью добавления пары
\cd{(\emph{key} . \emph{datum})} к старому \emph{a-list}.
\begin{lisp}
(acons x y a) \EQ\ (cons (cons x y) a)
\end{lisp}
\end{defun}

\begin{defun}[Функция]
pairlis keys data &optional a-list

\cdf{pairlis} принимает два списка и создаёт ассоциативный список, который
связывает элементы первого списка с соответствующими элементами второго. Если
списки не одинаковой длины, это является ошибкой. Если указан необязательный
аргумент \emph{a-list}, тогда новые пары добавляются к нему в начало.

Новые пары могут быть расположены в итоговом списке a-list в любом порядке.
Например:
\begin{lisp}
(pairlis '(one two) '(1 2) '((three . 3) (four . 19)))
\end{lisp}
может быть
\begin{lisp}
((one . 1) (two . 2) (three . 3) (four . 19))
\end{lisp}
или может быть так
\begin{lisp}
((two . 2) (one . 1) (three . 3) (four . 19))
\end{lisp}
\end{defun}

\begin{defun}[Функция]
assoc item a-list &key :test :test-not :key \\
assoc-if predicate a-list &key :key \\
assoc-if-not predicate a-list &key :key

Каждая из этих функций осуществляет поиск в ассоциативном списке
\emph{a-list}. Функция возвращает первую пару, удовлетворяющую условию, или
{\false}, если такой пары не было найдено.
Например:
\begin{lisp}
(assoc 'r '((a . b) (c . d) (r . x) (s . y) (r . z))) \\
~~~~~~~~\EV\  (r . x) \\
(assoc 'goo '((foo . bar) (zoo . goo))) \EV\ {\false} \\
(assoc '2 '((1 a b c) (2 b c d) (-7 x y z))) \EV\ (2 b c d)
\end{lisp}
Если функция вернула пару, можно изменить её значение с помощью \cdf{rplacd}.
(Однако лучше будет добавить новую <<затеняющую>> пару в начало, чем
модифицировать старую.)
Например:
\begin{lisp}
(setq values '((x . 100) (y . 200) (z . 50))) \\
(assoc 'y values) \EV\ (y . 200) \\
(rplacd (assoc 'y values) 201) \\
(assoc 'y values) \EV\ (y . 201) \textrm{теперь}
\end{lisp}
Типичный приём использования \cd{(cdr (assoc x y))}.
Так как \emph{cdr} от {\false} гарантировано вычисляется в {\false}, то в случае
отсутствия нужной пары, будет возвращено значение {\false}. {\false} также
будет возвращено, если значение для ключа равно {\false}. Такое поведение
удобно, если {\false} несёт смысл <<значения по-умолчанию>>.

Два выражения
\begin{lisp}
(assoc \emph{item} \emph{list} :test \emph{fn})
\end{lisp}
и 
\begin{lisp}
(find \emph{item} \emph{list} :test \emph{fn} :key \#'car)
\end{lisp}
эквивалентны за исключением, того что \cdf{assoc} игнорирует значения {\nil} на
месте пар.

Смотрите также \cdf{find} и \cdf{position}.

\begin{new}
X3J13 voted in January 1989
\issue{MAPPING-DESTRUCTIVE-INTERACTION}
to restrict user side effects; see section~\ref{STRUCTURE-TRAVERSAL-SECTION}.
\end{new}
\end{defun}

\begin{defun}[Функция]
rassoc item a-list &key :test :test-not :key \\
rassoc-if predicate a-list &key :key \\
rassoc-if-not predicate a-list &key :key

\cdf{rassoc} является реверсивной формой для \cdf{assoc}. Функция ищет пары, у
которых \emph{cdr} элемент удовлетворяет заданному условию.
Если \emph{a-list} рассматривается как отображение, то \cdf{rassoc} обрабатывает
\emph{a-list} как представление инверсного отображения.
Например:
\begin{lisp}
(rassoc 'a '((a . b) (b . c) (c . a) (z . a))) \EV\ (c . a)
\end{lisp}

Выражения 
\begin{lisp}
(rassoc \emph{item} \emph{list} :test \emph{fn})
\end{lisp}
и
\begin{lisp}
(find \emph{item} \emph{list} :test \emph{fn} :key \#'cdr)
\end{lisp}
эквивалентны за исключением, того что \cdf{rassoc} игнорирует значения {\nil} на
месте пар.

\begin{new}
X3J13 voted in January 1989
\issue{MAPPING-DESTRUCTIVE-INTERACTION}
to restrict user side effects; see section~\ref{STRUCTURE-TRAVERSAL-SECTION}.
\end{new}
\end{defun}

\fi        % Functions on lists
%Part{Hash, Root = "CLM.MSS"}
%%%Chapter of Common Lisp Manual.  Copyright 1984, 1988, 1989 Guy L. Steele Jr.

\clearpage\def\pagestatus{FINAL PROOF}

\chapter{Hash Tables Хеш-таблицы}
\label{HASH}

A hash table is a Lisp object that can efficiently map a given
Lisp object to another Lisp object.
Each hash table has a set of \emph{entries}, each of which associates a
particular \emph{key} with a particular \emph{value}.  The basic functions
that deal with hash tables can create entries, delete entries, and find
the value that is associated with a given key.  Finding the value is
very fast, even if there are many entries, because hashing is used; this
is an important advantage of hash tables over property lists.

Хеш-таблица является Lisp'овыми объектом, который может быстро отображать
заданный Lisp'овый объект в другой Lisp'овый объект. 
\emph{Wikipedia более понятно излагает: это структура данных, реализующая интерфейс
ассоциативного массива, а именно, она позволяет хранить пары (ключ, значение) и
выполнять три операции: операцию добавления новой пары, операцию поиска и
операцию удаления пары по ключу.}
Каждая хеш-таблица содержит множество \emph{элементов}, каждый из которых
содержит \emph{значение} ассоциированное с \emph{ключом}. Базовые функции
взаимодействия с хеш-таблицей монут создавать элементы (пары ключ-значение),
удалять элементы и искать значения по заданному ключу. Поиск значения очень
быстрый, даже при наличии большого количества элементов, потому что используется
хеширование. Это самое важное преимущество хеш-таблиц перед списками свойств.

A given hash table can associate only one \emph{value} with a given
\emph{key}; if you try to add a second \emph{value}, it will replace the
first.  Also, adding a value to a hash table is a destructive operation;
the hash table is modified.  By contrast, association lists can be
augmented non-destructively.

Хеш-таблица может связывать только одно \emph{значение} с заданным
\emph{ключом}. Если вы попробуете добавть второе \emph{значение}, оно заменит
предыдущее. К тому же, добавление значения в хеш-таблицу является деструктивной
операцией, в этом случае хеш-таблица модифицируется. Ассоциативные списки же,
наоборот, могут расширяться недеструктивно.

Hash tables come in three kinds, the difference being whether the keys
are compared with \cdf{eq}, \cdf{eql}, or \cdf{equal}.  In other words, there
are hash tables that hash on Lisp \emph{objects} (using \cdf{eq} or \cdf{eql})
and there are hash tables that hash on \emph{tree structure}
(using \cdf{equal}).

Хеш-таблицы существуют в трех видах, различие между ними в том, как сравниваются
ключи, с помощью \cdf{eq}, \cdf{eql} или \cdf{equal}. Другими словами,
существуют хеш таблицы с ключами, которые используют Lisp'овые \emph{объекты}
(\cdf{eq} или \cdf{eql}) и которые используют \emph{древовидные структуры}
(\cdf{equal}). FIXME

Hash tables are created with the function
\cdf{make-hash-table}, which takes various options, including
which kind of hash table to make (the default being the \cdf{eql} kind).
To look up a key and find
the associated value, use \cdf{gethash}.
New entries are added
to hash tables using \cdf{setf} with \cdf{gethash}.
To remove an entry, use \cdf{remhash}.  Here is a simple example.
\begin{lisp}
(setq a (make-hash-table)) \\
(setf (gethash 'color a) 'brown) \\
(setf (gethash 'name a) 'fred) \\
(gethash 'color a) \EV\ brown \\
(gethash 'name a) \EV\ fred \\
(gethash 'pointy a) \EV\ {\false}
\end{lisp}

Хеш-таблицы создаются функцией \cdf{make-hash-table}, которая принимает
различные параметры, включая тип хеш-таблицы (по-умолчанию тип \cdf{eql}).
Для поиска значения по ключу используйте \cdf{gethash}.
Новые элементы могут быть добавлены с помощью \cdf{gethash} внутри \cdf{setf}.
Для удаления элементов используйте \cdf{remhash}. Вот простой пример.
\begin{lisp}
(setq a (make-hash-table)) \\
(setf (gethash 'color a) 'brown) \\
(setf (gethash 'name a) 'fred) \\
(gethash 'color a) \EV\ brown \\
(gethash 'name a) \EV\ fred \\
(gethash 'pointy a) \EV\ {\false}
\end{lisp}

In this example, the symbols \cdf{color} and \cdf{name} are being used as
keys, and the symbols \cdf{brown} and \cdf{fred} are being used as the
associated values.  The hash table has two items in it, one of which
associates from \cdf{color} to \cdf{brown}, and the other of which
associates from \cdf{name} to \cdf{fred}.

В этом примере, символы \cd{color} и \cd{name} используются в качестве ключей, а
символы \cd{brown} и \cd{fred} в качестве ассоциированных значений. Хеш-таблица
содержит две пары, в одной ключ \cd{color} связан с \cd{brown}, а в другой
\cd{name} с \cd{fred}.

Keys do not have to be symbols; they can be any Lisp object.  Similarly,
values can be any Lisp object.

Ключи необязательно должны быть символами. Они могут быть любыми Lisp'овыми
объектами. Значения также могут быть любыми Lisp'овыми объектами.

\begin{obsolete}
When a hash table is first created, it has a \emph{size}, which is the
maximum number of entries it can hold.  Usually the actual capacity of
the table is somewhat less, since the hashing is not perfectly
collision-free.  With the maximum possible bad luck, the capacity could
be very much less, but this rarely happens.  If so many entries are
added that the capacity is exceeded, the hash table will automatically
grow, and the entries will be \emph{rehashed} (new hash values will be
recomputed, and everything will be rearranged so that the fast hash
lookup still works).  This is transparent to the caller; it all happens
automatically.
\end{obsolete}

\begin{newer}
There is a discrepancy between the preceding description of the
size of a hash table and the description of the \cd{:size} argument
in the specification below
of \cdf{make-hash-table}.

X3J13 voted in June 1989 \issue{HASH-TABLE-SIZE} to regard the
latter description as definitive: the \cd{:size} argument
is approximately the number of entries that can be inserted
without having to enlarge the hash table.  This definition is certainly
more convenient for the user.
\end{newer}

\beforenoterule
\begin{incompatibility}
This hash table facility is compatible with Lisp Machine Lisp.  It
is similar to the hasharray facility of Interlisp, and some of the
function names are the same.  However, it is \emph{not} compatible with
Interlisp.  The exact details and the order of arguments are designed to
be consistent with the rest of MacLisp rather than with
Interlisp.  For instance, the order of arguments to \cdf{maphash} is
different, there is no ``system hash table,'' and there is not
the Interlisp restriction that keys and values may not be {\false}.
\end{incompatibility}
\afternoterule

\section{Hash Table Functions Функции для хеш-таблиц}

This section documents the functions for hash tables, which
use \emph{objects} as keys and associate other objects with them.

Данный раздел описывает функции для хеш-таблиц, которые используют
\emph{объекты} для ключей и ассоциируют другие объекты с этими.

\begin{defun}[Function]
make-hash-table &key :test :size :rehash-size :rehash-threshold

This function creates and returns a new hash table.
The \cd{:test} argument determines how keys are compared;
it must be one of the three values \cd{\#'eq}, \cd{\#'eql}, or \cd{\#'equal},
or one of the three symbols \cdf{eq}, \cdf{eql}, or \cdf{equal}.
If no test is specified, \cdf{eql} is assumed.

Эта функция создает и возвращает новую хеш-таблицу.
Аргумент \cd{:test} определяет как будут сравниваться ключи.
Он может быть одним из трех значений \cd{\#'eq}, \cd{\#'eql} или \cd{\#'equal},
или одним из трех символов \cdf{eq}, \cdf{eql} или \cdf{equal}.
По-умолчанию используется \cdf{eql}.

\begin{new}
X3J13 voted in January 1989
\issue{HASH-TABLE-TESTS}
to add a fourth type of hash table:
the value of \cd{\#'equalp} and the symbol \cdf{equalp} are to be additional
valid possibilities for the \cd{:test} argument.


Note that one consequence of
the vote to change the rules of
floating-point contagion
\issue{CONTAGION-ON-NUMERICAL-COMPARISONS}
(described in section~\ref{PRECISION-CONTAGION-COERCION-SECTION})
is to require \cdf{=}, and therefore also \cdf{equalp},
to compare the values of numbers exactly and not approximately, making
\cdf{equalp} a true equivalence relation on numbers.

Another valuable use of \cdf{equalp} hash tables is case-insensitive
comparison of keys that are strings.
\end{new}

The \cd{:size} argument
sets the initial size of the hash table, in entries.
(The actual size may be rounded up from the size
you specify to the next ``good'' size, for example to make it a prime number.)
You won't necessarily be able to store precisely
this many entries into the table
before it overflows and becomes bigger, but this argument does serve
as a hint to the implementation of approximately
how many entries you intend to store.

Аргумент \cd{:size} устанавливает первоначальный размер хеш-таблицы в парах.
(Указанный размер может быть округлен до <<хорошего>> размера, например, до
первого следующего простого числа.)
Вы можете не сохранять в таблице столько пар, сколько указали. Этот аргумент
служит подсказкой для реализации о том, какое примерно число элементов вы будете
хранить в хеш-таблице.

\begin{new}
X3J13 voted in January 1989
\issue{ARGUMENTS-UNDERSPECIFIED}
to clarify that the \cd{:size} argument
must be a non-negative integer.
\end{new}

\begin{newer}
X3J13 voted in June 1989 \issue{HASH-TABLE-SIZE} to regard the
preceding description of the \cd{:size} argument as definitive: it
is approximately the number of entries that can be inserted
without having to enlarge the hash table.
\end{newer}

The \cd{:rehash-size} argument
specifies how much to increase the size of the hash table when it becomes
full.  This can be an integer greater than zero,
which is the number of entries to add, or
it can be a floating-point number greater than 1,
which is the ratio of the new size to the old size.
The default value for this argument is implementation-dependent.

Аргумент \cd{:rehash-size} указывает, на сколько увеличить размер хеш-таблицы,
когда она по размерам достигнет пределов.
Если это целое число, оно должно быть больше нуля, и будет означать абсолютное
приращение. Если это число с плавающей точкой, оно должно быть больше 1, и будет
означать относительное приращение к предыдущему размеру.
Значение по-умолчанию зависит от реализации.

\begin{obsolete}
The \cd{:rehash-threshold} argument
specifies how full the hash table can get before it must grow.
This can be an integer greater than zero and less than the \cd{:rehash-size}
(in which case it will be scaled whenever the table is grown),
or it can be a floating-point number between zero and 1.
The default value for this argument is implementation-dependent.
\end{obsolete}

\begin{newer}
X3J13 voted in June 1989 \issue{HASH-TABLE-SIZE} to replace
the preceding specification of the \cd{:rehash-threshold} argument
with the following:
The \cd{:rehash-threshold} argument
specifies how full the hash table can get before it must grow.
It may be any \cdf{real} number between \cd{0} and \cd{1}, inclusive.
It indicates the maximum desired level of hash table occupancy.
An implementation is permitted to ignore this argument.
The default value for this argument is implementation-dependent.

Аргумент \cd{:rehash-threshold} указывает, насколько должна наполниться
хеш-таблица, прежде чем она будет увеличина. Он должен быть числом типа
\cdf{real} между \cd{0} и \cd{1}, включительно.
Он показывает максимальный уровень заполнения хеш-таблицы.
Значение по-умолчанию зависит от реализации.
\end{newer}

An example of the use of \cdf{make-hash-table}:
\begin{lisp}
(make-hash-table :rehash-size 1.5 \\*
~~~~~~~~~~~~~~~~~:size (* number-of-widgets 43))
\end{lisp}

Пример использования \cdf{make-hash-table}:
\begin{lisp}
(make-hash-table :rehash-size 1.5 \\*
~~~~~~~~~~~~~~~~~:size (* number-of-widgets 43))
\end{lisp}
\end{defun}

\begin{defun}[Function]
hash-table-p object

\cdf{hash-table-p} is true if its argument is a hash table,
and otherwise is false.
\begin{lisp}
(hash-table-p x) \EQ\ (typep x 'hash-table)
\end{lisp}

\cdf{hash-table-p} возвращает истину, если аргумент является хеш-таблицей. Иначе
возвращает ложь.
\begin{lisp}
(hash-table-p x) \EQ\ (typep x 'hash-table)
\end{lisp}
\end{defun}

\begin{defun}[Function]
gethash key hash-table &optional default

\cdf{gethash} finds the entry in \emph{hash-table} whose key is \emph{key}
and returns the
associated value.  If there is no such entry, \cdf{gethash} returns \emph{default},
which is {\false} if not specified.

\cdf{gethash} actually returns two values, the second being a predicate
value that is true if an entry was found, and false if no entry was found.

\cdf{setf} may be used with \cdf{gethash} to make new entries in a hash
table.  If an entry with the specified \emph{key} already exists, it is
removed before the new entry is added.  The \emph{default} argument may be
specified to \cdf{gethash} in this context; it is ignored by \cdf{setf}
but may be useful in such macros as \cdf{incf} that are related to \cdf{setf}:
\begin{lisp}
(incf (gethash a-key table 0))
\end{lisp}
means approximately the same as
\begin{lisp}
(setf (gethash a-key table 0) \\*
~~~~~~(+ (gethash a-key table 0) 1))
\end{lisp}
which in turn would be treated as simply
\begin{lisp}
(setf (gethash a-key table) \\*
~~~~~~(+ (gethash a-key table 0) 1))
\end{lisp}

\cdf{gethash} ищет элемент в хеш-\emph{hash-table}, чей ключ равен \emph{key} и
возвращает связанное значение. Если элемент не был найден, то возвращается
значение аргумента \emph{default}, который по-умолчанию равен {\false}.

\cdf{gethash} возвращает два значение. Второе значение является предикатом, и
истинно, если значение было найдено, и ложно если нет.

\cdf{setf} может использоваться вместе с \cdf{gethash} для создания в
хеш-таблице новых элементов. Если элемент с заданным ключом \emph{key} уже
существует, он будет удален перед добавлением. В этом контексте может
использоваться аргумент \emph{default}, он игнорируется \cdf{setf}, но может
быть полезным в таких макросах как \cdf{incf}, которые связаны с \cdf{setf}:
\begin{lisp}
(incf (gethash a-key table 0))
\end{lisp}
обозначает то же, что и 
\begin{lisp}
(setf (gethash a-key table 0) \\*
~~~~~~(+ (gethash a-key table 0) 1))
\end{lisp}
что можно преобразовать в
\begin{lisp}
(setf (gethash a-key table) \\*
~~~~~~(+ (gethash a-key table 0) 1))
\end{lisp}
\end{defun}

\begin{defun}[Function]
remhash key hash-table

\cdf{remhash} removes
any entry for \emph{key} in \emph{hash-table}.  This is a predicate
that is true if there was an
entry or false if there was not.

\cdf{remhash} удаляет любой элемент в хеш-таблице \emph{hash-table} ключ,
которого равен параметру \emph{key}. Возвращает истину, если элемент был
удален, и ложь, если элемента уже не существовало.
\end{defun}

\begin{defun}[Function]
maphash function hash-table

For each entry in \emph{hash-table}, \cdf{maphash} calls
\emph{function} on two arguments:
the key of the entry and the value of the entry; \cdf{maphash} then returns \cdf{nil}.
If entries are added to or deleted from the hash table while a \cdf{maphash}
is in progress, the results are unpredictable, with one exception:
if the \emph{function} calls \cdf{remhash} to remove the entry currently
being processed by the \emph{function}, or performs a \cdf{setf} of
\cdf{gethash} on that entry to change the associated value, then those
operations will have the intended effect.
For example:
\begin{lisp}
;;; Alter every entry in MY-HASH-TABLE, replacing the value with \\
;;; its square root.  Entries with negative values are removed. \\
(maphash \#'(lambda (key val) \\
~~~~~~~~~~~~~(if (minusp val) \\
~~~~~~~~~~~~~~~~~(remhash key my-hash-table) \\
~~~~~~~~~~~~~~~~~(setf (gethash key my-hash-table) (sqrt val)))) \\
~~~~~~~~~my-hash-table)
\end{lisp}

Для каждого элемента в хеш-таблице \emph{hash-table}, \cdf{maphash} вызывает
функцию \emph{function} с двумя аргументами:
ключ элемента и
значение элемента.
\cdf{maphash} возвращает \cdf{nil}.
Если во время выполнения \cdf{maphash} в хеш-таблице добавлялись или удалялись
ключи, то результат непредсказуем, но есть исключение:
если функция \emph{function} вызывает \cdf{remhash} для удаления элемента,
который в эту функцию и был передан, или устанавливает новое значение с помощью
\cdf{setf} этому элементу, то эти операции будут выполнены правильно.
Например:
\begin{lisp}
;;; Изменение каждого элемента в MY-HASH-TABLE, с заменой на \\
;;; квадратный корень. Элементы с отрицательными значения удаляются. \\
(maphash \#'(lambda (key val) \\
~~~~~~~~~~~~~(if (minusp val) \\
~~~~~~~~~~~~~~~~~(remhash key my-hash-table) \\
~~~~~~~~~~~~~~~~~(setf (gethash key my-hash-table) (sqrt val)))) \\
~~~~~~~~~my-hash-table)
\end{lisp}

\begin{new}
X3J13 voted in January 1989
\issue{MAPPING-DESTRUCTIVE-INTERACTION}
to restrict user side effects; see section~\ref{STRUCTURE-TRAVERSAL-SECTION}.
\end{new}
\end{defun}

\begin{defun}[Function]
clrhash hash-table

This removes all the entries from \emph{hash-table}
and returns the hash table itself.

Функция удаляет все элемент из хеш-таблицы \emph{hash-table} и возвращает эту
хеш-таблицу.
\end{defun}

\begin{defun}[Function]
hash-table-count hash-table

This returns the number of entries in the \emph{hash-table}.
When a hash table is first created or has been cleared,
the number of entries is zero.

Функция возвращает количество элементов в хеш-таблице \emph{hash-table}.
Когда хеш-таблица только были сделана, или только что очищена, то количество
элементов равно нулю.
\end{defun}


\begin{new}
\begin{defmac}
with-hash-table-iterator (mname hash-table) {\,form}*

X3J13 voted in January 1989
\issue{HASH-TABLE-PACKAGE-GENERATORS}
to add the macro \cdf{with-hash-table-iterator}.

The name \emph{mname} is bound and defined as if by \cdf{macrolet},
with the body \emph{form\/}s as its lexical scope, to be a ``generator macro''
such that successive invocations \cd{(\emph{mname})} will
return entries, one by one, from the hash table that is the value of the
expression \emph{hash-table} (which is evaluated exactly once).

At each invocation of the generator macro, there are two possibilities.
If there is yet another unprocessed entry in the hash table, then
three values are returned: \cdf{t},
the key of the hash table entry, and
the associated value of the hash table entry.
On the other hand, if there are no more unprocessed entries in the
hash table, then one value is returned: \cdf{nil}.

The implicit interior state of the iteration over the hash table
entries has dynamic extent.  While the name \emph{mname} has
lexical scope, it is an error to invoke the generator macro
once the \cdf{with-hash-table-iterator} form has been exited.

Invocations of \cdf{with-hash-table-iterator}
and related macros may be nested, and the generator macro of an
outer invocation may be called from within an inner invocation
(assuming that its name is visible or otherwise made available).

\begin{new}
X3J13 voted in January 1989
\issue{MAPPING-DESTRUCTIVE-INTERACTION}
to restrict user side effects; see section~\ref{STRUCTURE-TRAVERSAL-SECTION}.
\end{new}

\beforenoterule
\begin{rationale}
This facility is a bit more flexible than \cdf{maphash}.
It makes possible a portable and efficient implementation of \cdf{loop}
clauses for iterating over hash tables (see chapter~\ref{LOOP}).
\end{rationale}
\afternoterule
\end{defmac}

\newpage%manual

\begin{lisp}
(setq turtles (make-hash-table :size 9 :test 'eq)) \\*
(setf (gethash 'howard-kaylan turtles) '(musician lead-singer)) \\*
(setf (gethash 'john-barbata turtles) '(musician drummer)) \\*
(setf (gethash 'leonardo turtles) '(ninja leader blue)) \\*
(setf (gethash 'donatello turtles) '(ninja machines purple)) \\*
(setf (gethash 'al-nichol turtles) '(musician guitarist)) \\*
(setf (gethash 'mark-volman turtles) '(musician great-hair)) \\*
(setf (gethash 'raphael turtles) '(ninja cool rude red)) \\*
(setf (gethash 'michaelangelo turtles) '(ninja party-dude orange)) \\*
(setf (gethash 'jim-pons turtles) '(musician bassist)) \\
\\
(with-hash-table-iterator (get-turtle turtles) \\*
~~(labels ((try (got-one \&optional key value) \\*
~~~~~~~~~~~~~(when got-one\`;\textrm{Remember, keys may show up in any order} \\*
~~~~~~~~~~~~~~~(when (eq (first value) 'ninja) \\*
~~~~~~~~~~~~~~~~~(format t "{\Xtilde}\%{\Xtilde}:({\Xtilde}A{\Xtilde}): {\Xtilde}{\Xlbrace}{\Xtilde}A{\Xtilde}{\Xcircumflex}, {\Xtilde}{\Xrbrace}" \\*
~~~~~~~~~~~~~~~~~~~~~~~~~key (rest value))) \\*
~~~~~~~~~~~~~~~(multiple-value-call \#'try (get-turtle))))) \\*
~~~~(multiple-value-call \#'try (get-turtle))))~~~~~;\textrm{Prints 4 lines} \\*
Michaelangelo: PARTY-DUDE, ORANGE \\*
Leonardo: LEADER, BLUE \\*
Raphael: COOL, RUDE, RED \\*
Donatello: MACHINES, PURPLE \\*
~~\EV\ nil
\end{lisp}
\end{new}

\begin{newer}
\begin{defun}[Function]
hash-table-rehash-size hash-table \\
hash-table-rehash-threshold hash-table \\
hash-table-size hash-table \\
hash-table-test hash-table

X3J13 voted in March 1989 \issue{HASH-TABLE-ACCESS}
to add four accessor functions that return values suitable for use in a call to
\cdf{make-hash-table} in order to produce a new hash table with state
corresponding to the current state of the argument hash table.
 
\cdf{hash-table-rehash-size} returns the current rehash size of a hash table.

\cdf{hash-table-rehash-threshold} returns the current rehash threshold.

\cdf{hash-table-size} returns the current size of a hash table.

\cdf{hash-table-test} returns the test used for comparing keys.
If the test is one of the standard test functions, then the result
will always be a symbol, even if the function itself was specified
when the \emph{hash-table} was created.  For example:
\begin{lisp}
(hash-table-test (make-hash-table :test \#'equal)) \EV\ equal
\end{lisp}
Implementations that extend \cdf{make-hash-table} by providing additional
possibilities for the \cd{:test} argument may determine how the
value returned by \cdf{hash-table-test} is related to such additional tests.

Добавлены функции, которые возвращают значения используемые при вызове
\cdf{make-hash-table}.

\cdf{hash-table-rehash-size} возвращает размер приращения хеш-таблицы.

\cdf{hash-table-rehash-threshold} возвращает текущий уровень заполнения.

\cdf{hash-table-size} возвращает рекомендуемый размер хеш-таблицы.

\cdf{hash-table-test} возвращает функцию сравнения используемую для ключей.
Если данная функция одна из стандартных, то результат всегда является символом,
даже если при создании хеш-таблицы было указано иное. Например:
\begin{lisp}
(hash-table-test (make-hash-table :test \#'equal)) \EV\ equal
\end{lisp}
Реализации, которые расширяют \cdf{make-hash-table} дополнительными функциями
сравнения для аргумента \cd{:test}, могут определять, как будет возвращено
значение из \cdf{hash-table-test} для этих дополнительных функций.
\end{defun} 
\end{newer}

\section{Primitive Hash Function Функции хеширования}

The function \cdf{sxhash} is a convenient tool for the user who needs
to create more complicated hashed data structures than are provided by
\cdf{hash-table} objects.

Функция \cdf{sxhash} является удобным инструментом для пользователя,
нуждающегося в создании более сложных хешированных структур данных, чем
предоставляются объектами \cdf{hash-table}.

\begin{defun}[Function]
sxhash object

\cdf{sxhash} computes a hash code for an object and returns the hash code as
a non-negative fixnum.  A property of \cdf{sxhash}
is that \cd{(equal \emph{x} \emph{y})} implies \cd{(= (sxhash \emph{x}) (sxhash \emph{y}))}.

The manner in which the hash code is computed is implementation-dependent
but independent of the particular ``incarnation'' or ``core image.''
Hash values produced
by \cdf{sxhash} may be written out to files, for example, and meaningfully
read in again into an instance of the same implementation.

\cdf{sxhash} вычисляет хеш для объекта и возвращает этот хеш, как
неотрицательное число fixnum. Свойство \cdf{sxhash} заключается в том, что
\cd{(equal \emph{x} \emph{y})} подразумевает \cd{(= (sxhash \emph{x}) (sxhash
  \emph{y}))}.

Механизм вычисления хеша зависит от реализации, но независим от запущенного
экземпляра.
Например, вычисленные \cdf{sxhash} хеши могут быть записаны в файлы, и без
потери информации прочитаны из них в рамках одной реализации.
\end{defun}
        % Hash tables
%Part{Array, Root = "CLM.MSS"}
%%%Chapter of Common Lisp Manual.  Copyright 1984, 1988, 1989 Guy L. Steele Jr.

\clearpage\def\pagestatus{FINAL PROOF}

\ifx \rulang\Undef

\chapter{Arrays}

An array is an object with components arranged according
to a rectilinear coordinate system.
In principle, an
array in Common Lisp may have any number of dimensions, including zero.
(A zero-dimensional array has exactly one element.)
In practice, an implementation may limit the number of dimensions
supported, but
every Common Lisp implementation must support arrays of up to
seven dimensions.
Each dimension is a non-negative integer; if any dimension of an array is zero,
the array has no elements.

An array may be a \emph{general array}, meaning each element may be any Lisp
object, or it may be a \emph{specialized array}, meaning that each element
must be of a given restricted type.

\begin{newer}
X3J13 voted in March 1989 \issue{CHARACTER-PROPOSAL}
to eliminate the type \cdf{string-char} and to redefine the type
\cdf{string} to be the union of one or more specialized vector
types, the types of whose elements are subtypes of the type \cdf{character}.
\end{newer}

\section{Array Creation}

Do not be daunted by the many options of the function \cdf{make-array}.
All that is required to construct an array is a list of
the dimensions; most of the options are for relatively esoteric
applications.

\begin{defun}[Function]
make-array dimensions &key :element-type :initial-element :initial-contents :adjustable :fill-pointer :displaced-to :displaced-index-offset

This is the primitive function for making arrays.  The \emph{dimensions} argument
should be a list of non-negative integers
that are to be the dimensions of the array; the
length of the list will be the dimensionality of the array.  
Each dimension must be smaller than \cdf{array-dimension-limit},
and the product of all the dimensions must be smaller than
\cdf{array-total-size-limit}.
Note that if \emph{dimensions} is {\nil}, then a zero-dimensional array is created.
For convenience when making a one-dimensional array, the single dimension
may be provided as an integer rather than as a list of one integer.

An implementation of Common Lisp may impose a limit on the rank of an array,
but this limit may not be smaller than 7.  Therefore, any Common Lisp
program may assume the use of arrays of rank 7 or less.
The implementation-dependent limit on array rank is reflected in
\cdf{array-rank-limit}.

The keyword arguments for \cdf{make-array} are as follows:

\begin{flushdesc}
\item[\cd{:element-type}]
This argument
should be the name of the type of the elements of the array;
an array is constructed
of the most specialized type that can nevertheless accommodate
elements of the given type.
The type {\true} specifies a general array, one whose elements may
be any Lisp object; this is the default type.

\begin{new}
X3J13 voted in January 1989
\issue{ARRAY-TYPE-ELEMENT-TYPE-SEMANTICS}
to change \cdf{typep} and \cdf{subtypep}
so that the specialized \cdf{array} type specifier
means the same thing for discrimination purposes
as for declaration purposes: it encompasses those arrays
that can result by specifying \emph{element-type} as the element type
to the function \cdf{make-array}.  Therefore we may say
that if \emph{type} is the \cd{:element-type} argument, then
the result will be an array of type \cd{(array \emph{type})};
put another way, for any type \emph{A},
\begin{lisp}
(typep (make-array ... :element-type '\emph{A} ...) \\*
~~~~~~~'(array \emph{A\/})))
\end{lisp}
is always true.
See \cdf{upgraded-array-element-type}.
\end{new}

\item[\cd{:initial-element}]
This argument
may be used to initialize each element of the array.  The value
must be of the type specified by the \cd{:element-type} argument.  If the
\cd{:initial-element} option is omitted, the initial values of the array
elements are undefined (unless the \cd{:initial-contents} or
\cd{:displaced-to} option is used).
The \cd{:initial-element} option may not be used with the
\cd{:initial-contents} or \cd{:displaced-to} option.

\item[\cd{:initial-contents}]
This argument may be used to initialize the
contents of the array.  The value is a nested structure of sequences.  If
the array is zero-dimensional, then the value specifies the single
element.  Otherwise, the value must be a sequence whose length is equal
to the first dimension; each element must be a nested structure for an
array whose dimensions are the remaining dimensions, and so on.
For example:
\begin{lisp}
(make-array '(4 2 3) \\*
~~~~~~~~~~~~:initial-contents \\
~~~~~~~~~~~~'(((a b c) (1 2 3)) \\
~~~~~~~~~~~~~~((d e f) (3 1 2)) \\
~~~~~~~~~~~~~~((g h i) (2 3 1)) \\
~~~~~~~~~~~~~~((j k l) (0 0 0))))
\end{lisp}
The numbers of levels in the structure must equal the rank of the array.
Each leaf of the nested structure
must be of the type specified by the \cd{:type} option.  If the
\cd{:initial-contents} option is omitted, the initial values of the array
elements are undefined (unless the \cd{:initial-element} or
\cd{:displaced-to} option is used).
The \cd{:initial-contents} option may not be used with the
\cd{:initial-element} or \cd{:displaced-to} option.

\item[\cd{:adjustable}]
This argument, if specified and not {\false}, indicates that it
must be possible to alter the array's size dynamically after it is
created.  This argument defaults to {\nil}.

\begin{newer}
X3J13 voted in June 1989
\issue{ADJUST-ARRAY-NOT-ADJUSTABLE}
to clarify that if this argument is non-{\false}
then the predicate \cdf{adjustable-array-p} will necessarily be true when applied
to the resulting array; but if this argument is \cdf{nil} (or omitted) then the
resulting array may or may not be adjustable, depending on the implementation,
and therefore \cdf{adjustable-array-p} may be correspondingly true or false of
the resulting array.  Common Lisp provides no portable way to create a
non-adjustable array, that is, an array for which \cdf{adjustable-array-p} is
guaranteed to be false.
\end{newer}

\item[\cd{:fill-pointer}]
This argument
specifies that the array should have a fill pointer.
If this option is specified and not {\false}, the array must be one-dimensional.
The value is used to initialize the fill pointer for the array.
If the value {\true} is specified, the length of the array is used;
otherwise the value must be an integer between 0 (inclusive)
and the length of the array (inclusive).
This argument defaults to {\nil}.

\item[\cd{:displaced-to}]
This argument, if specified and
not {\false}, specifies that the array will be a \emph{displaced} array.
The argument must then be an array;
\cdf{make-array} will create
an \emph{indirect} or \emph{shared} array that shares its contents with
the specified array.  In this case the \cd{:displaced-index-offset}
option may be useful.
It is an error if the array given as the \cd{:displaced-to} argument
does not have the same \cd{:element-type} as the array being created.
The \cd{:displaced-to} option may not be used with the
\cd{:initial-element}
or \cd{:initial-contents} option.
This argument defaults to {\nil}.

\item[\cd{:displaced-index-offset}]
This argument may be used only in conjunction
with the \cd{:displaced-to} option.
It must be a non-negative integer (it defaults to zero); it is made to be the
index-offset of the created shared array.

When an array A is given as
the \cd{:displaced-to} argument to \cdf{make-array} when creating array B,
then array B is said to be \emph{displaced} to array A.  Now the
total number of elements in an array, called the \emph{total size} of the array,
is calculated as the product of all the dimensions
(see \cdf{array-total-size}).
It is required that the total size of A be no smaller than the sum
of the total size of B plus the offset \emph{n} specified by
the \cd{:displaced-index-offset}
argument.  The effect of displacing is that array B does not have any
elements of its own but instead maps accesses to itself into
accesses to array A. The mapping treats both arrays as if they
were one-dimensional by taking the elements in row-major order,
and then maps an access to element \emph{k} of array B to an access to element
\emph{k}+\emph{n} of array A.
\end{flushdesc}

If \cdf{make-array} is called with each of the \cd{:adjustable}, \cd{:fill-pointer},
and \cd{:displaced-to}
arguments either unspecified or {\nil}, then the
resulting array is guaranteed to be a \emph{simple} array
(see section~\ref{ARRAY-TYPE-SECTION}).

\begin{newer}
X3J13 voted in June 1989
\issue{ADJUST-ARRAY-NOT-ADJUSTABLE}
to clarify that if one or more of the \cd{:adjustable}, \cd{:fill-pointer},
and \cd{:displaced-to} arguments is true, then whether the resulting
array is simple is unspecified.
\end{newer}

Here are some examples of the use of \cdf{make-array}:
\begin{lisp}
;;; Create a one-dimensional array of five elements. \\*
(make-array 5) \\
 \\
;;; Create a two-dimensional array, 3 by 4, with four-bit elements. \\*
(make-array '(3 4) \cd{:element-type} '(mod 16)) \\
 \\
;;; Create an array of single-floats.\\*
(make-array 5 \cd{:element-type} 'single-float)) \\
\\
;;; Making a shared array. \\*
(setq a (make-array '(4 3))) \\
(setq b (make-array 8 :displaced-to a \\*
~~~~~~~~~~~~~~~~~~~~~~:displaced-index-offset 2)) \\
;;; Now it is the case that: \\*
~~~~~~~~(aref b 0) \EQ\ (aref a 0 2) \\*
~~~~~~~~(aref b 1) \EQ\ (aref a 1 0) \\*
~~~~~~~~(aref b 2) \EQ\ (aref a 1 1) \\*
~~~~~~~~(aref b 3) \EQ\ (aref a 1 2) \\*
~~~~~~~~(aref b 4) \EQ\ (aref a 2 0) \\*
~~~~~~~~(aref b 5) \EQ\ (aref a 2 1) \\*
~~~~~~~~(aref b 6) \EQ\ (aref a 2 2) \\*
~~~~~~~~(aref b 7) \EQ\ (aref a 3 0)
\end{lisp}
The last example depends on the fact that arrays are, in effect,
stored in row-major order for purposes of sharing.  Put another way,
the indices for the elements of an array are ordered
lexicographically.

\beforenoterule
\begin{incompatibility}
Both Lisp Machine Lisp, as described in reference \cite{BLUE-LISPM},
and Fortran \cite{DRAFT-FORTRAN-77,ANSI-FORTRAN-77} store arrays in
column-major order.
\end{incompatibility}
\afternoterule
\end{defun}

\begin{defun}[Constant]
array-rank-limit

The value of \cdf{array-rank-limit} is a positive integer that is
the upper exclusive bound on the rank of an array.
This bound depends on the implementation
but will not be smaller than 8; therefore every Common Lisp implementation
supports arrays whose rank is between 0 and 7 (inclusive).
(Implementors are encouraged to make this limit as large as practicable
without sacrificing performance.)
\end{defun}

\begin{defun}[Constant]
array-dimension-limit

The value of \cdf{array-dimension-limit} is a positive integer that is
the upper exclusive bound on each individual dimension of an array.
This bound depends on the implementation
but will not be smaller than 1024.
(Implementors are encouraged to make this limit as large as practicable
without sacrificing performance.)

\begin{new}
X3J13 voted in January 1989
\issue{FIXNUM-NON-PORTABLE}
to specify that the value
of \cd{array-dimension-limit} must be of type \cdf{fixnum}.
This in turn implies that all valid array indices will be fixnums.
\end{new}
\end{defun}

\begin{defun}[Constant]
array-total-size-limit

The value of \cdf{array-total-size-limit} is a positive integer that is
the upper exclusive bound on the total number of elements in an array.
This bound depends on the implementation
but will not be smaller than 1024.
(Implementors are encouraged to make this limit as large as practicable
without sacrificing performance.)

The actual limit on array size imposed by the implementation may vary
according to the \cd{:element-type} of the array; in this case the value of
\cdf{array-total-size-limit} will be the smallest of these individual
limits.
\end{defun}

\begin{defun}[Function]
vector &rest objects

The function \cdf{vector} is a convenient means for creating
a simple general vector with specified initial contents.
It is analogous to the function \cdf{list}.
\begin{lisp}
(vector $\emph{a}_1$ $\emph{a}_2$ ... $\emph{a}_{n}$) \\
~~~\EQ\ (make-array (list $\emph{n}$) :element-type t \\
~~~~~~~~~~~~~:initial-contents (list $\emph{a}_1$ $\emph{a}_2$ ... $\emph{a}_{n}$))
\end{lisp}
\end{defun}

\section{Array Access}

The function \cdf{aref} is normally
used for accessing an element of an array.
Other access functions, such as \cdf{svref}, \cdf{char}, and \cdf{bit},
may be more efficient in specialized circumstances.

\begin{defun}[Function]
aref array &rest subscripts

This accesses and returns the element of \emph{array} specified
by the \emph{subscripts}.  The number of subscripts must
equal the rank of the array, and each subscript must be
a non-negative integer less than the corresponding array dimension.

\cdf{aref} is unusual among the functions that operate on arrays
in that it completely ignores fill pointers.  \cdf{aref} can access
without error any array element, whether active or not.  The generic
sequence function \cdf{elt}, however, observes the fill pointer;
accessing an element beyond the fill pointer with \cdf{elt} is an error.

\begin{new}
Note that this remark, predating the design of the Common Lisp Object System,
uses the term ``generic'' in a generic sense and not necessarily
in the technical sense used by CLOS
(see chapter~\ref{DTYPES}).
\end{new}

\cdf{setf} may be used with \cdf{aref} to destructively replace
an array element with a new value.

Under some circumstances it is desirable to write code that
will extract an element from an array \cdf{a} given a list \cdf{z} of the indices,
in such a way that the code works regardless of the rank of the
array.  This is easy using \cdf{apply}:
\begin{lisp}
(apply \#'aref a z)
\end{lisp}
(The length of the list must of course equal the rank of
the array.)  This construction may be used with \cdf{setf} to alter
the element so selected to some new value \cdf{w}:
\begin{lisp}
(setf (apply \#'aref a z) w)
\end{lisp}
\end{defun}

\begin{defun}[Function]
svref simple-vector index

The first argument must be a simple general vector,
that is, an object of type \cdf{simple-vector}.
The element of the \emph{simple-vector} specified by the integer \emph{index}
is returned.

The \emph{index} must be non-negative and less than
the length of the vector.

\cdf{setf} may be used with \cdf{svref} to destructively replace
a simple-vector element with a new value.

\cdf{svref} is identical to \cdf{aref} except that it requires its first
argument to be a simple vector.  In some implementations of Common Lisp,
\cdf{svref} may be faster than \cdf{aref} in situations where it is applicable.
See also \cdf{schar} and \cdf{sbit}.
\end{defun}

\section{Array Information}

The following functions extract from an array
interesting information other than the elements.

\begin{defun}[Function]
array-element-type array

\cdf{array-element-type} returns a type specifier for the set of objects
that can be stored in the \emph{array}.  This set may be larger than the
set requested when the array was created; for example,
the result of
\begin{lisp}
(array-element-type (make-array 5 :element-type '(mod 5)))
\end{lisp}
could be \cd{(mod 5)}, \cd{(mod 8)}, \cdf{fixnum}, \cdf{t}, or any other
type of which \cd{(mod 5)} is a subtype.  See \cdf{subtypep}.
\end{defun}

\begin{defun}[Function]
array-rank array

This returns the number of dimensions (axes) of \emph{array}.
This will be a non-negative integer.
See \cdf{array-rank-limit}.

\beforenoterule
\begin{incompatibility}
In Lisp Machine Lisp, this is called \cd{array-\#-dims}.
This name causes problems in other Lisp dialects
because of the \cd{\#} character.
\end{incompatibility}
\afternoterule
\end{defun}

\begin{defun}[Function]
array-dimension array axis-number

The length of dimension number \emph{axis-number} of the \emph{array} is returned.
\emph{array} may be any kind of array, and \emph{axis-number} should
be a non-negative integer less than the rank of \emph{array}.
If the \emph{array} is a vector with a fill pointer,
\cdf{array-dimension} returns the total size of the vector,
including inactive elements,
not the size indicated by the fill pointer.
(The function \cdf{length} will return the size indicated
by the fill pointer.)

\beforenoterule
\begin{incompatibility}
This is similar to the Lisp Machine Lisp function
\cdf{array-dimension-n}, but takes its arguments in the other
order, and is zero-origin for consistency
instead of one-origin.  In Lisp Machine Lisp \cd{(array-dimension-n 0)}
returns the length of the array leader.
\end{incompatibility}
\afternoterule
\end{defun}

\begin{defun}[Function]
array-dimensions array

\cdf{array-dimensions} returns a list whose elements are the dimensions
of \emph{array}.
\end{defun}

\begin{defun}[Function]
array-total-size array

\cdf{array-total-size} returns the total number of elements in the \emph{array},
calculated as the product of all the dimensions.
\begin{lisp}
(array-total-size \emph{x}) \\
~~~\EQ\ (apply \#'* (array-dimensions \emph{x})) \\
~~~\EQ\ (reduce \#'* (array-dimensions \emph{x}))
\end{lisp}
Note that the total size of a zero-dimensional array is \cd{1}.
The total size of a one-dimensional array is calculated without regard
for any fill pointer.
\end{defun}

\begin{defun}[Function]
array-in-bounds-p array &rest subscripts

This predicate checks whether the \emph{subscripts} are all
legal subscripts for \emph{array}.  The predicate is true if they
are all legal; otherwise it is false.  The \emph{subscripts} must be integers.
The number of \emph{subscripts} supplied must equal the rank of the array.
Like \cdf{aref}, \cdf{array-in-bounds-p} ignores fill pointers.
\end{defun}

\begin{defun}[Function]
array-row-major-index array &rest subscripts

This function takes an array and valid subscripts for the array
and returns a single non-negative integer less than the total size
of the array that identifies the
accessed element in the row-major ordering of the elements.
The number of \emph{subscripts} supplied must equal the rank of the array.
Each subscript must be a non-negative integer less than the
corresponding array dimension.
Like \cdf{aref}, \cdf{array-row-major-index} ignores fill pointers.

A possible definition of \cdf{array-row-major-index}, with no error checking,
would be
\begin{lisp}
(defun array-row-major-index (a \cd{\&rest} subscripts) \\
~~(apply \#'+ (maplist \#'(lambda (x y) \\
~~~~~~~~~~~~~~~~~~~~~~~~~~(* (car x) (apply \#'* (cdr y)))) \\
~~~~~~~~~~~~~~~~~~~~~~subscripts \\
~~~~~~~~~~~~~~~~~~~~~~(array-dimensions a))))
\end{lisp}
For a one-dimensional array, the result of \cdf{array-row-major-index}
always equals the supplied subscript.
\end{defun}

\begin{defun}[Function]
row-major-aref array index

This allows any array element to be accessed as if the containing array
were one-dimensional.  The \emph{index} must be a non-negative integer
less than the total size of the \emph{array}.  It indexes into the array as
if its elements were arranged one-dimensionally in row-major order.
It may be understood in terms of \cdf{aref} as follows:
\begin{lisp}
(row-major-aref \emph{array} \emph{index}) \EQ \\*
~~(aref (make-array (array-total-size array)) \\*
~~~~~~~~~~~~~~~~~~~~:displaced-to array \\*
~~~~~~~~~~~~~~~~~~~~:element-type (array-element-type array)) \\*
~~~~~~~~index)
\end{lisp}
In other words, one may treat an array as one-dimensional by creating
a new one-dimensional array that is displaced to the old one and then
accessing the new array.
Alternatively, \cdf{aref} may be understood in terms of \cdf{row-major-aref}:
\begin{lisp}
(aref \emph{array} $\emph{i}_0$ $\emph{i}_1$ ... $\emph{i}_{n-1}$) \EQ \\*
~~(row-major-aref array \\*
~~~~~~~~~~~~~~~~~~(array-row-major-index array $\emph{i}_0$ $\emph{i}_1$ ... $\emph{i}_{n-1}$)
\end{lisp}
That is, a multidimensional array access is equivalent to a row-major access
using an equivalent row-major index.

Like \cdf{aref}, \cdf{row-major-aref} completely ignores fill pointers.
A call to \cdf{row-major-setf} is suitable for use as a \emph{place} for
\cdf{setf}.

This operation makes it easier to write code that efficiently processes
arrays of any rank.  Suppose, for example, that one wishes to set every
element of an array \cdf{tennis-scores} to zero.  One might write
\begin{lisp}
(fill (make-array (array-total-size tennis-scores) \\*
~~~~~~~~~~~~~~~~~~:element-type (array-element-type tennis-scores) \\*
~~~~~~~~~~~~~~~~~~:displaced-to tennis-scores) \\*
~~~~~~0)
\end{lisp}
Unfortunately, this incurs the overhead of creating a displaced array,
and \cdf{fill} cannot be applied to multidimensional arrays.
Another approach would be to handle each possible rank separately:
\begin{lisp}
(ecase (array-rank tennis-scores) \\*
~~(0 (setf (aref tennis-scores) 0)) \\
~~(1 (dotimes (i0 (array-dimension tennis-scores 0)) \\*
~~~~~~~(setf (aref tennis-scores i0) 0))) \\
~~(2 (dotimes (i0 (array-dimension tennis-scores 0)) \\*
~~~~~~~(dotimes (i1 (array-dimension tennis-scores 1)) \\*
~~~~~~~~~(setf (aref tennis-scores i0 i1) 0)))) \\*
~~... \\
~~(7 (dotimes (i0 (array-dimension tennis-scores 0)) \\*
~~~~~~~(dotimes (i1 (array-dimension tennis-scores 1)) \\
~~~~~~~~~(dotimes (i2 (array-dimension tennis-scores 1)) \\
~~~~~~~~~~~(dotimes (i3 (array-dimension tennis-scores 1)) \\
~~~~~~~~~~~~~(dotimes (i4 (array-dimension tennis-scores 1)) \\
~~~~~~~~~~~~~~~(dotimes (i5 (array-dimension tennis-scores 1)) \\
~~~~~~~~~~~~~~~~~(dotimes (i6 (array-dimension tennis-scores 1)) \\*
~~~~~~~~~~~~~~~~~~~(setf (aref tennis-scores i0 i1 i2 i3 i4 i5 i6) \\*
~~~~~~~~~~~~~~~~~~~~~~~~~0))))))))) \\*
~~)
\end{lisp}
It is easy to get tired of writing such code.  Furthermore, this approach
is undesirable because some implementations of Common Lisp
will in fact correctly support arrays of rank greater than 7 (though no
implementation is required to do so).  A recursively nested loop does the job,
but it is still pretty hairy:
\begin{lisp}
(labels \\*
~~((grok-any-rank (\&rest indices) \\*
~~~~~(let ((d (- (array-rank tennis-scores) (length indices))) \\*
~~~~~~~(if (= d 0) \\*
~~~~~~~~~~~(setf (apply \#'row-major-aref indices) 0) \\*
~~~~~~~~~~~(dotimes (i (array-dimension tennis-scores (- d 1))) \\*
~~~~~~~~~~~~~(apply \#'grok-any-rank i indices)))))) \\*
~~(grok-any-rank))
\end{lisp}
Whether this code is particularly efficient depends on many implementation
parameters, such as how \cd{\&rest} arguments are handled and how
cleverly calls to \cdf{apply} are compiled.
How much easier it is to use \cdf{row-major-aref}!
\begin{lisp}
(dotimes (i (array-total-size tennis-scores)) \\*
~~(setf (row-major-aref tennis-scores i) 0))
\end{lisp}
Surely this code is sweeter than the honeycomb.
\end{defun}

\begin{defun}[Function]
adjustable-array-p array

This predicate is true if the argument (which must be an array)
is adjustable, and otherwise is false.

\begin{newer}
X3J13 voted in June 1989
\issue{ADJUST-ARRAY-NOT-ADJUSTABLE}
to clarify that \cdf{adjustable-array-p} is true of an array
if and only if \cdf{adjust-array}, when applied to that array,
will return the same array, that is, an array \cdf{eq} to the original array.
If the \cd{:adjustable} argument
to \cdf{make-array} is non-\cdf{nil} when an array is created,
then \cdf{adjustable-array-p} must be true of that array.
If an array is created with the \cd{:adjustable} argument \cdf{nil}
(or omitted), then \cdf{adjustable-array-p} may be true or false of that
array, depending on the implementation.
X3J13 further voted to \emph{define}
the terminology ``adjustable array'' to mean precisely ``an array of
which \cdf{adjustable-array-p} is true.''
See \cdf{make-array} and \cdf{adjust-array}.
\end{newer}
\end{defun}

\section{Functions on Arrays of Bits}

The functions described in this section operate only
on arrays of bits, that is, specialized arrays whose elements
are all \cd{0} or \cd{1}.

\begin{defun}[Function]
bit bit-array &rest subscripts \\
sbit simple-bit-array &rest subscripts

\cdf{bit} is exactly like \cdf{aref} but requires an array of bits,
that is, one of type \cd{(array bit)}.
The result will always be \cd{0} or \cd{1}.
\cdf{sbit} is like \cdf{bit} but additionally requires that the first
argument be a \emph{simple} array (see section~\ref{ARRAY-TYPE-SECTION}).
Note that \cdf{bit} and \cdf{sbit}, unlike \cdf{char} and \cdf{schar},
allow the first argument to be an array of any rank.

\cdf{setf} may be used with \cdf{bit} or \cdf{sbit} to destructively replace
a bit-array element with a new value.

\cdf{bit} and \cdf{sbit} are identical to \cdf{aref} except for the
more specific type requirements on the first argument.
In some implementations of Common Lisp,
\cdf{bit} may be faster than \cdf{aref} in situations where it is applicable,
and \cdf{sbit} may similarly be faster than \cdf{bit}.
\end{defun}

\begin{defun}[Function]
bit-and bit-array1 bit-array2 &optional result-bit-array \\
bit-ior bit-array1 bit-array2 &optional result-bit-array \\
bit-xor bit-array1 bit-array2 &optional result-bit-array \\
bit-eqv bit-array1 bit-array2 &optional result-bit-array \\
bit-nand bit-array1 bit-array2 &optional result-bit-array \\
bit-nor bit-array1 bit-array2 &optional result-bit-array \\
bit-andc1 bit-array1 bit-array2 &optional result-bit-array \\
bit-andc2 bit-array1 bit-array2 &optional result-bit-array \\
bit-orc1 bit-array1 bit-array2 &optional result-bit-array \\
bit-orc2 bit-array1 bit-array2 &optional result-bit-array

These functions perform bit-wise logical operations on bit-arrays.
All of the arguments to any of these functions must be bit-arrays
of the same rank and dimensions.
The result is a bit-array of matching rank and dimensions,
such that any given bit of the result
is produced by operating on corresponding bits from each of the arguments.

If the third argument is {\false} or omitted, a new array is created
to contain the result.  If the third argument is a bit-array,
the result is destructively placed into that array.  If the third
argument is {\true}, then the first argument is also used as the third
argument; that is, the result is placed back in the first array.

The following table indicates what the result bit is for each operation
as a function of the two corresponding argument bits.
\begin{flushleft}
\cf
\begin{tabular*}{\textwidth}{@{}l@{\extracolsep{\fill}}lllll@{}}
~~~\emph{argument1}~~&0&0&1&1 \\
~~~\emph{argument2}~~&0&1&0&1&\emph{Operation name} \\
\hlinesp
bit-and&0&0&0&1&\textrm{and} \\
bit-ior&0&1&1&1&\textrm{inclusive or} \\
bit-xor&0&1&1&0&\textrm{exclusive or} \\
bit-eqv&1&0&0&1&\textrm{equivalence (exclusive nor)} \\
bit-nand&1&1&1&0&\textrm{not-and} \\
bit-nor&1&0&0&0&\textrm{not-or} \\
bit-andc1&0&1&0&0&\textrm{and complement of \emph{argument1} with \emph{argument2}} \\
bit-andc2&0&0&1&0&\textrm{and \emph{argument1} with complement of \emph{argument2}} \\
bit-orc1&1&1&0&1&\textrm{or complement of \emph{argument1} with \emph{argument2}} \\
bit-orc2&1&0&1&1&\textrm{or \emph{argument1} with complement of \emph{argument2}} \\
\hline
\end{tabular*}
\end{flushleft}
For example:
\begin{lisp}
(bit-and \#*1100 \#*1010) \EV\ \#*1000 \\
(bit-xor \#*1100 \#*1010) \EV\ \#*0110 \\
(bit-andc1 \#*1100 \#*1010) \EV\ \#*0100
\end{lisp}
See \cdf{logand} and related functions.
\end{defun}

\begin{defun}[Function]
bit-not bit-array &optional result-bit-array

The first argument must be an array of bits.  A bit-array
of matching rank and dimensions is returned that contains
a copy of the argument
with all the bits inverted.
See \cdf{lognot}.

If the second argument is {\false} or omitted, a new array is created
to contain the result.  If the second argument is a bit-array,
the result is destructively placed into that array.  If the second
argument is {\true}, then the first argument is also used as the second
argument; that is, the result is placed back in the first array.
\end{defun}

\section{Fill Pointers}
\label{FILL-POINTER}

Several functions for manipulating a \emph{fill pointer} are provided
in Common Lisp
to make it easy to incrementally fill in the contents of a vector
and, more generally, to allow efficient varying of the length of a vector.
For example, a string with a fill pointer has most of the characteristics
of a PL/I varying string.

The fill pointer is a non-negative integer no larger than the total
number of elements in the vector (as returned by \cdf{array-dimension});
it is the number of ``active'' or ``filled-in'' elements in the vector.
The fill pointer constitutes the ``active length'' of the vector;
all vector elements whose index is less than the fill pointer are
active, and the others are inactive.
Nearly all functions that operate on the contents of a vector
will operate only on the active elements.  An important exception
is \cdf{aref}, which can be used to access any vector element
whether in the active region of the vector or not.  It is important
to note that vector elements not in the active region are still considered
part of the vector.

\beforenoterule
\begin{implementation}
An implication of this rule is that
vector elements outside the active region may not be garbage-collected.
\end{implementation}
\afternoterule

Only vectors (one-dimensional arrays) may have fill pointers;
multidimensional arrays may not.  (Note, however, that one can create
a multidimensional array that is \emph{displaced} to a vector that has
a fill pointer.)

\begin{defun}[Function]
array-has-fill-pointer-p array

The argument must be an array.  \cdf{array-has-fill-pointer-p} returns
{\true} if the array has a fill pointer, and otherwise returns {\false}.
Note that
\cdf{array-has-fill-pointer-p}
always returns {\false} if
the \emph{array} is not one-dimensional.
\end{defun}

\begin{defun}[Function]
fill-pointer vector

The fill pointer of \emph{vector} is returned.  It is an error if
the \emph{vector} does not have a fill pointer.

\cdf{setf} may be used with \cdf{fill-pointer} to change the fill pointer
of a vector.  The fill pointer of a vector must always be an integer
between zero and the size of the vector (inclusive).
\end{defun}

\begin{defun}[Function]
vector-push new-element vector

\emph{vector} must be a one-dimensional array that has a fill pointer,
and \emph{new-element} may be any object. \cdf{vector-push} attempts to store
\emph{new-element} in the element of the vector designated by the fill
pointer, and to increase the fill pointer by 1.  If the fill pointer does
not designate an element of the vector (specifically, when it gets too
big), it is unaffected and
\cdf{vector-push} returns {\false}.  Otherwise, the store and increment take
place and \cdf{vector-push} returns the \emph{former} value of the fill pointer
(1 less than the one it leaves in the vector); thus the value of
\cdf{vector-push} is the index of the new element pushed.
\begin{new}
It is instructive to compare \cdf{vector-push}, which is a function,
with \cdf{push}, which is a macro that requires a \emph{place} suitable
for \cdf{setf}.  A vector with a fill pointer effectively contains
the place to be modified in its \cdf{fill-pointer} slot.
\end{new}
\end{defun}

\begin{defun}[Function]
vector-push-extend new-element vector &optional extension

\cdf{vector-push-extend} is just like \cdf{vector-push} except
that if the fill pointer gets too large, the vector is extended (using
\cdf{adjust-array}) so that it can contain more elements.
If, however, the vector is not adjustable, then \cdf{vector-push-extend}
signals an error.

\begin{newer}
X3J13 voted in June 1989
\issue{ADJUST-ARRAY-NOT-ADJUSTABLE}
to clarify that \cdf{vector-push-extend} regards an array as
not adjustable if and only if \cdf{adjustable-array-p} is false
of that array.
\end{newer}

The optional argument \emph{extension}, which must be a positive
integer, is the minimum number of elements to be added to the vector if it
must be extended; it defaults to a ``reasonable'' implementation-dependent
value.
\end{defun}

\begin{defun}[Function]
vector-pop vector

\emph{vector} must be a one-dimensional array that has a fill pointer.
If the fill pointer is zero, \cdf{vector-pop} signals an error.
Otherwise the fill pointer is decreased by 1, and the vector element
designated by the new value of the fill pointer is returned.
\end{defun}

\section{Changing the Dimensions of an Array}

This function may be used to resize or reshape an array.
Its options are similar to those of \cdf{make-array}.

\begin{defun}[Function]
adjust-array array new-dimensions &key :element-type :initial-element :initial-contents :fill-pointer :displaced-to :displaced-index-offset

\cdf{adjust-array} takes an array and a number of other arguments
as for \cdf{make-array}.  The number of dimensions
specified by \emph{new-dimensions} must equal the rank of \emph{array}.

\cdf{adjust-array} returns an array of the same type and rank as \emph{array},
with the specified \emph{new-dimensions}.  In effect, the \emph{array} argument
itself is modified to conform to the new specifications, but this may
be achieved either by modifying the \emph{array} or by creating a new
array and modifying the \emph{array} argument to be \emph{displaced} to the
new array.

In the simplest case, one specifies only the \emph{new-dimensions}
and possibly an \cd{:initial-element} argument.
Those elements of \emph{array} that
are still in bounds appear in the new array.  The elements of
the new array that are not in the bounds of \emph{array} are initialized
to the \cd{:initial-element}; if this argument is not provided,
then the initial contents of any new elements are undefined.

If \cd{:element-type} is specified, then \emph{array} must be such that it could have
been originally created with that type; otherwise an error is signaled.
Specifying \cd{:element-type} to \cdf{adjust-array} serves only to require such an
error check.

If \cd{:initial-contents} or \cd{:displaced-to}
is specified, then it is treated as for
\cdf{make-array}.  In this case none of the original contents of
\emph{array} appears in the new array.

If \cd{:fill-pointer} is specified, the fill pointer of the \emph{array}
is reset as specified.  An error is signaled if \emph{array} had no
fill pointer already.

\begin{new}
X3J13 voted in June 1988
\issue{ADJUST-ARRAY-FILL-POINTER}
to clarify the treatment of the \cd{:fill-pointer}
argument as follows.

If the \cd{:fill-pointer} argument is not supplied, then the fill pointer
of the \emph{array} is left alone.  It is an error
to try to adjust the \emph{array} to a total size that is smaller
than its fill pointer.

If the \cd{:fill-pointer} argument is supplied, then its value
must be either an integer, \true, or \false.  If it is an integer,
then it is the new value for the fill pointer;
it must be non-negative and no greater than the new size to which the
\emph{array} is being adjusted.
If it is \true, then the fill pointer is set equal to the new size
for the \emph{array}.  If it is \false, then the fill pointer is
left alone; it is as if the argument had not been supplied.
Again, it is an error
to try to adjust the \emph{array} to a total size that is smaller
than its fill pointer.

An error is signaled if a non-{\false} \cd{:fill-pointer} value
is supplied and the \emph{array} to be adjusted does not already
have a fill pointer.

This extended treatment of the \cd{:fill-pointer}
argument to \cdf{adjust-array} is consistent with the previously
existing treatment of the \cd{:fill-pointer} argument to \cdf{make-array}.
\end{new}

\cdf{adjust-array} may, depending on the implementation and the arguments,
simply alter the given array or create and return a new one.
In the latter case the given array will be altered so as to be displaced
to the new array and have the given new dimensions.

\begin{newer}
X3J13 voted in January 1989
\issue{ADJUST-ARRAY-NOT-ADJUSTABLE}
to allow \cdf{adjust-array} to be applied to any array.
If \cdf{adjust-array} is applied to an array that was
originally created with \cd{:adjustable} true,
the array returned is \cdf{eq} to its first argument.  It is not specified
whether \cdf{adjust-array} returns an array \cdf{eq} to its first argument for any
other arrays.  If the array returned by \cdf{adjust-array} is not \cdf{eq} to its
first argument, the original array is unchanged and does not share
storage with the new array.

Under this new definition, it is wise to treat \cdf{adjust-array}
in the same manner as \cdf{delete} and \cdf{nconc}: one should carefully
retain the returned value, for example by writing
\begin{lisp}
(setq my-array (adjust-array my-array ...))
\end{lisp}
rather than relying solely on a side effect.
\end{newer}

If \cdf{adjust-array} is applied to an \emph{array} that is displaced
to another array \emph{x}, then afterwards neither \emph{array} nor the returned
result is displaced to \emph{x} unless such displacement is explicitly
re-specified in the call to \cdf{adjust-array}.

For example, suppose that the 4-by-4 array \cdf{m} looks like this:
\begin{lisp}
\#2A(~(~alpha~~~~~beta~~~~~~gamma~~~~~delta~) \\
~~~~~(~epsilon~~~zeta~~~~~~eta~~~~~~~theta~) \\
~~~~~(~iota~~~~~~kappa~~~~~lambda~~~~mu~~~~) \\
~~~~~(~nu~~~~~~~~xi~~~~~~~~omicron~~~pi~~~~)~)
\end{lisp}
Then the result of
\begin{lisp}
(adjust-array m '(3 5) :initial-element 'baz)
\end{lisp}
is a 3-by-5 array with contents
\begin{lisp}
\#2A(~(~alpha~~~~~beta~~~~~~gamma~~~~~delta~~~~~baz~) \\
~~~~~(~epsilon~~~zeta~~~~~~eta~~~~~~~theta~~~~~baz~) \\
~~~~~(~iota~~~~~~kappa~~~~~lambda~~~~mu~~~~~~~~baz~)~)
\end{lisp}
Note that if array \cdf{a} is created displaced to array \cdf{b} and subsequently
array \cdf{b} is given to \cdf{adjust-array}, array \cdf{a} will still be
displaced to array \cdf{b}; the effects of this displacement and
the rule of row-major storage order must be taken into account.

\begin{newer}
X3J13 voted in June 1988 \issue{ADJUST-ARRAY-DISPLACEMENT}
to clarify the interaction of \cdf{adjust-array} with array displacement.

Suppose that an array \emph{A} is to be adjusted.  There are four cases
according to whether or not \emph{A} was displaced before adjustment
and whether or not the result is displaced after adjustment.
\begin{itemize}
\item
Suppose \emph{A} is not displaced either before or after.
The dimensions of \emph{A} are altered, and
the contents are rearranged as appropriate.  Additional elements of \emph{A}
are taken from the \cd{:initial-element} argument.
However, the use of the \cd{:initial-contents} argument causes all old
contents to be discarded.

\item
Suppose \emph{A} is not displaced before, but is displaced to array \emph{C} after.
None of the original contents of \emph{A} appears in \emph{A} afterwards; \emph{A}
now contains (some of) the contents of \emph{C}, without any rearrangement of
\emph{C}.

\item
Suppose \emph{A} is displaced to {array \emph{B}} before the call,
and is displaced to array \emph{C} after the call.
(Note that \emph{B} and \emph{C} may be the same array.)
The contents of \emph{B} do not appear in
\emph{A} afterwards (unless such contents also happen to be in \emph{C},
as when \emph{B} and \emph{C} are the same, for example).  If
\cd{:displaced-index-offset} is not specified in the call to
\cdf{adjust-array}, it defaults
to zero; the old offset (into \emph{B}) is not retained.

\item
Suppose \emph{A} is displaced to array \emph{B} before the call,
but is not displaced afterwards.  In this case \emph{A} gets
a new ``data region'' and (some of) the
contents of \emph{B} are copied into it as appropriate to
maintain the existing old contents.  Additional elements of \emph{A}
are taken from the \cd{:initial-element} argument.
However, the use of the \cd{:initial-contents} argument causes all old
contents to be discarded.
\end{itemize}

If array \emph{X} is displaced to array \emph{Y}, and array \emph{Y} is displaced
to array \emph{Z}, and array \emph{Y} is altered by \cdf{adjust-array}, array
\emph{X} must now refer to the adjusted contents of \emph{Y}.  This means that an
implementation may not collapse the chain to make \emph{X} refer to \emph{Z}
directly and forget that the chain of reference passes through array \emph{Y}. 
(Caching techniques are of course permitted, as long as they preserve the
semantics specified here.)

If \emph{X} is displaced to \emph{Y}, it is an error to adjust \emph{Y} in such a
way that it no longer has enough elements to satisfy \emph{X}.  This error may be
signaled at the time of the adjustment, but this is not required.

Note that omitting the \cd{:displaced-to} argument to \cdf{adjust-array} is
equivalent to specifying \cd{:displaced-to~nil}; in either case, the array is
not displaced after the call regardless of whether it was displaced before
the call.
\end{newer}
\end{defun}

%RUSSIAN
\else

\chapter{Массивы}

Массив является объектом с элементами, расположенными в соответствие с
прямолинейной координатной системой. В целом, это можно назвать построчным
хранением объектов многомерных таблиц.
В теории, массив в Common Lisp'е может иметь любое количество измерений, включая
ноль.
(Нульмерный массив измерений имеет только один элемент. FIXME)
На практике, реализация может ограничивать количество измерений,
но должна как минимум поддерживать семь.
Каждое измерение является неотрицательным целым. Если любое из измерений равно
нулю, то массив не имеет элементов.

Массив может быть \emph{общим}. Это означает, что элемент может быть
любым Lisp объектом. Массив также может быть \emph{специализированным}. Это
означает, что элемент ограничен некоторым типом.

\section{Создание массива}

Не волнуйтесь о большом количестве опций для функции \cdf{make-array}.
Все что действительно необходимо, это список размерностей. Остальные же опции
служат для относительно эзотерических программ.

\begin{defun}[Функция]
make-array dimensions &key :element-type :initial-element :initial-contents :adjustable :fill-pointer :displaced-to :displaced-index-offset

Это базовая функция для создания массивов. Аргумент \emph{dimensions} должен
быть списком неотрицательных целых чисел, которые являются измерениями массива. Длина
списка является размерностью массива. Каждое измерение должно быть меньше чем
\cdf{array-dimension-limit}, и произведение всех измерений должно быть меньше
\cdf{array-total-size-limit}.
Следует отметить, что если \emph{dimensions} {\nil}, тогда создаётся массив с
нулевым количеством измерений.
По соглашению, при создании одномерного массива, размер измерений может
быть указан как одиночное целое число, а не список целых чисел.

Реализация Common Lisp'а может иметь ограничение на ранг массива, но это
ограничение должно быть меньше 7. Таким образом, любая Common
Lisp программа может использовать массивы с рангом меньшим или равным 7.

Для \cdf{make-array} существуют следующие именованные параметры:

\begin{flushdesc}

\item[\cd{:element-type}]
Этот аргумент должен быть именем типа элементов массива.
Массив создаётся с наиболее близким типом к указанному, и не может содержать
объекты отличные от этого типа.
Тип {\true} указывает на создание общего массива, элементы которого могут быть
любыми объектами. {\true} является значением по-умолчанию.

\item[\cd{:initial-element}]
Этот аргумент может использоваться для инициализации каждого элемента
массива. Значение должно принадлежать типу указанному в аргументе
\cd{:element-type}. Если \cd{:initial-element} опущен, тогда значение
по-умолчанию для элементов массива не определено (если только не используется
\cd{:initial-contents} или \cd{:displaced-to}).
Опция \cd{:initial-element} может не использоваться, когда указана опция
\cd{:initial-contents} или \cd{:displaced-to}.

\item[\cd{:initial-contents}]
Этот аргумент может использоваться для инициализации содержимого
массива. Значением параметра является структура вложенных
последовательностей. Если массив с нулевым количеством измерений, тогда значение
задаёт одиночный элемент. В противном случае, значение должно быть
последовательностью с длинной, равной размеру первого измерения. Каждый
вложенный элемент также должен быть последовательностью, размером равным
следующему измерению массива.
Например:
\begin{lisp}
(make-array '(4 2 3) \\*
~~~~~~~~~~~~:initial-contents \\
~~~~~~~~~~~~'(((a b c) (1 2 3)) \\
~~~~~~~~~~~~~~((d e f) (3 1 2)) \\
~~~~~~~~~~~~~~((g h i) (2 3 1)) \\
~~~~~~~~~~~~~~((j k l) (0 0 0))))
\end{lisp}
Количество уровней в структуре должно равняться рангу массива.
Каждый лист вложенной структуры должен иметь тип, указанный в опции
\cd{:type}. Если опция \cd{:initial-contents} опущена, первоначальное значение
элементов массива не определено (если только не используется опция
\cd{:initial-element} или \cd{:displaced-to}).
Опция \cd{:initial-contents} может не использоваться, когда указана опция
\cd{:initial-element} или \cd{:displaced-to}.

\item[\cd{:adjustable}]
Если данный аргумент указан и не является {\false}, тогда размер массива после
создания может быть изменён. По-умолчанию параметр равен {\false}.

\item[\cd{:fill-pointer}]
Данный аргумент указывает, что массив должен иметь указатель заполнения.
Если эта опция указана и не является {\false}, тогда массив должен быть
одномерным.
Значение используется для инициализации указателя заполнения массива.
Если указано значение {\true}, тогда используется длина массива.
Иначе, значение должно быть целым числом между нулём (включительно) и длинной
массива (включительно)ю
Аргумент по-умолчанию равен {\nil}.

\item[\cd{:displaced-to}]
Если данный аргумент указан и не является {\false}, то он указывает на то, что
массив будет \emph{соединён} с другим массивом.
Значение аргумента должно быть уже созданным массивом.
\cdf{make-array} создаст \emph{косвенный} или \emph{разделяемый} массив, который
имеет общее содержимое с указанным массивом. В таком случае опция
\cd{:displaced-index-offset} позволяет указать смещение общей части массивов.
Если массив, переданный в аргумент \cd{:displaced-to}, не имеет такого же типа
элементов (\cd{:element-type}), как и создаваемый массив, то генерируется
ошибка.
Опция \cd{:displaced-to} не может использоваться вместе с опцией \cd{:initial-element}
или \cd{:initial-contents}.
По-умолчанию аргумент равен {\nil}.

\item[\cd{:displaced-index-offset}]
Данный аргумент может использоваться только в сочетании с опцией
\cd{:displaced-to}.
Она должна быть неотрицательным целым (по-умолчанию равна нулю). Она означает
смещение создаваемого \emph{разделяемого} массива относительно исходного.

Если при создании массива B массив A передан в качестве аргумента
\cd{:displaced-to} в функцию \cdf{make-array}, тогда массив B называется
\emph{соединённым} с массивом A. Полное количество элементов массива, называемое
\emph{полный размер} массива, вычисляется как произведение размеров всех
измерений (смотрите \cdf{array-total-size}).
Необходимо, чтобы полный размер массива A был не меньше чем сумма \emph{полного
  размера} массива B и смещения \emph{n}, указанного в аргументе
\cd{:displaced-index-offset}.
Смысл соединения массивов в том, что массив B не имеет своих элементов и
операции над ним приводят в действительности к операциям над массивом A.
При отображении элементов массивов, последние рассматриваются как одномерные, с
построчным порядком размещения элементов.
Так, элемент с индексом \emph{k} массива \emph{B} отображается в элемент с
индексом \emph{k}+\emph{n} массива \emph{A}.
\end{flushdesc}

Если \cdf{make-array} вызывается без указания аргументов \cd{:adjustable},
\cd{:fill-pointer} и \cdf{:displaced-to}, или указания их в {\nil}, тогда
получаемый массив гарантированного будет \emph{простым} (смотрите
раздел~\ref{ARRAY-TYPE-SECTION}). 

Вот несколько примеров использования \cdf{make-array}:
\begin{lisp}
;;; Создание одномерного массива с пятью элементами. \\*
(make-array 5) \\
 \\
;;; Создание двумерного массива, 3 на 4, с четырёхбитными элементами. \\*
(make-array '(3 4) \cd{:element-type} '(mod 16)) \\
 \\
;;; Создание массива чисел с плавающей точкой.\\*
(make-array 5 \cd{:element-type} 'single-float)) \\
\\
;;; Создание соединённого массива. \\*
(setq a (make-array '(4 3))) \\
(setq b (make-array 8 :displaced-to a \\*
~~~~~~~~~~~~~~~~~~~~~~:displaced-index-offset 2)) \\
;;; В таком случае: \\*
~~~~~~~~(aref b 0) \EQ\ (aref a 0 2) \\*
~~~~~~~~(aref b 1) \EQ\ (aref a 1 0) \\*
~~~~~~~~(aref b 2) \EQ\ (aref a 1 1) \\*
~~~~~~~~(aref b 3) \EQ\ (aref a 1 2) \\*
~~~~~~~~(aref b 4) \EQ\ (aref a 2 0) \\*
~~~~~~~~(aref b 5) \EQ\ (aref a 2 1) \\*
~~~~~~~~(aref b 6) \EQ\ (aref a 2 2) \\*
~~~~~~~~(aref b 7) \EQ\ (aref a 3 0)
\end{lisp}
\end{defun}

\begin{defun}[Константа]
array-rank-limit

Значение \cdf{array-rank-limit} является положительным целым, которое обозначает
(невключительно) наивысший возможный ранг массива.
Это значение зависит от реализации, но не может быть меньше 8. Таком образом
каждая реализация Common Lisp'а поддерживает массивы, ранг которых лежит между 0
и 7 (включительно).
(Разработчикам предлагается сделать это значение большим настолько, на сколько
это возможно без ущерба в производительности.)
\end{defun}

\begin{defun}[Константа]
array-dimension-limit

Значение \cdf{array-dimension-limit} является положительным целым, которое
обозначает (невключительно) наивысший возможный размер измерения для массива.
Это значение зависит от реализации, но не может быть меньше 1024.
(Разработчикам предлагается сделать это значение большим настолько, на сколько
это возможно без ущерба в производительности.)
\end{defun}

\begin{defun}[Константа]
array-total-size-limit

Значение \cdf{array-total-size-limit} является положительным целым, которое
обозначает (невключительно) наивысшее возможное количество элементов в массиве.
Это значение зависит от реализации, но не может быть меньше 1024.
(Разработчикам предлагается сделать это значение большим настолько, на сколько
это возможно без ущерба в производительности.)

На практике предельный размер массива может зависеть от типа элемента массива, в
таком случае \cdf{array-total-size-limit} будет наименьшим из ряда этих размеров.
\end{defun}

\begin{defun}[Функция]
vector &rest objects

Функция \cdf{vector} служит для создания простых базовых векторов с заданным
содержимым.
Она является аналогом для функции \cdf{list}.
\begin{lisp}
(vector $\emph{a}_1$ $\emph{a}_2$ ... $\emph{a}_{n}$) \\
~~~\EQ\ (make-array (list $\emph{n}$) :element-type t \\
~~~~~~~~~~~~~:initial-contents (list $\emph{a}_1$ $\emph{a}_2$ ... $\emph{a}_{n}$))
\end{lisp}
\end{defun}

\section{Доступ к массиву}

Для доступа к элементам массива обычно используется функция \cdf{aref}.
В отдельных случаях может быть более эффективно использование
других функций, таких как \cdf{svref}, \cdf{char} и \cdf{bit}.

\begin{defun}[Функция]
aref array &rest subscripts

Данная функция возвращает элемент массива \emph{array}, который задан индексами
\emph{subscripts}. Количество индексов должно совпадать с рангом массива, и
каждый индекс должен быть неотрицательным целым числом меньшим чем
соответствующий размер измерения.

\cdf{aref} отличается от остальных функций тем, что полностью игнорирует
указатели заполнения.
\cdf{aref} может получать доступ к любому элементу массива, вне зависимости
является он активным или нет. Однако общая функция для последовательностей
\cdf{elt} учитывает указатель заполнения. Доступ к элементам за указателем
заполнение с помощью \cdf{elt} является ошибкой.

Для изменения элемента массива может использоваться \cdf{setf} в связке с
\cdf{aref}.

В некоторых случаях, удобно писать код, который получает элемент из массива,
например \cdf{a}, используя список индексов, например \cdf{z}.
Это легко сделать используя \cdf{apply}:
\begin{lisp}
(apply \#'aref a z)
\end{lisp}
(Длина списка, конечно же, должна быть равна рангу массива.) Эта конструкция
может использоваться с \cdf{setf} для изменения элемента массива на, например,
\cdf{w}:
\begin{lisp}
(setf (apply \#'aref a z) w)
\end{lisp}
\end{defun}

\begin{defun}[Функция]
svref simple-vector index

Первый аргумент должен быть простым базовым вектором, то есть объектом типа
\cdf{simple-vector}.
В результате возвращается элемент вектора \emph{simple-vector} с индексом
\emph{index}.

\emph{index} должен быть неотрицательным целым меньшим чем длина вектора.

Для изменения элемента вектора может использоваться \cdf{setf} в связке с
\cdf{svref}.

\cdf{svref} идентична \cdf{aref} за исключением того, что требует, чтобы первый
аргумент был простым вектором. В некоторых реализациях Common Lisp'а,
\cdf{svref} может быть быстрее чем \cdf{aref} в
тех случаях, где она применима.
Смотрите также \cdf{schar} и \cdf{sbit}
\end{defun}

\section{Информация о массиве}

Следующие функции извлекают интересную информацию, и это не элементы
массива.

\begin{defun}[Функция]
array-element-type array

\cdf{array-element-type} возвращает спецификатор типа для множества объектов,
которые могут быть сохранены в массиве \emph{array}. Это множество может быть
больше чем то, которое запрашивалось в функции \cdf{make-array}. Например,
результат
\begin{lisp}
(array-element-type (make-array 5 :element-type '(mod 5)))
\end{lisp}
может быть \cd{(mod 5)}, \cd{(mod 8)}, \cdf{fixnum}, \cdf{t} или любой другой
тип, для которого \cd{(mod 5)} является подтипом. Смотрите \cdf{subtypep}.
\end{defun}

\begin{defun}[Функция]
array-rank array

Эта функция возвращает количество измерений (осей) массива \emph{array}.
Результат будет неотрицательным целым.
Смотрите \cdf{array-rank-limit}.
\end{defun}

\begin{defun}[Функция]
array-dimension array axis-number

Данная функция возвращает размер измерения \emph{axis-number} массива
\emph{array}.
\emph{array} может быть любым видом массива, и \emph{axis-number} должен быть
неотрицательным целым меньшим чем ранг массива \emph{array}.
Если \emph{array} является вектором с указателем заполнения,
\cdf{array-dimension} возвращает общий размер вектора, включая неактивные
элементы, а не размер ограниченный указателем заполнения.
(Функция \cdf{length} будет возвращать размер ограниченный указателем
заполнения.) 
\end{defun}

\begin{defun}[Функция]
array-dimensions array

\cdf{array-dimensions} возвращает список, элементы которого являются размерами
измерений массива \emph{array}.
\end{defun}

\begin{defun}[Функция]
array-total-size array

\cdf{array-total-size} возвращает общее количество элементов массива
\emph{array}, которое вычислено как произведение размеров всех измерений.
\begin{lisp}
(array-total-size \emph{x}) \\
~~~\EQ\ (apply \#'* (array-dimensions \emph{x})) \\
~~~\EQ\ (reduce \#'* (array-dimensions \emph{x}))
\end{lisp}
Следует отметить, что общий размер нульмерного (FIXME) массива равен \cd{1}.
Общий размер одномерного массива вычисляется без учёта указателя заполнения.
\end{defun}

\begin{defun}[Функция]
array-in-bounds-p array &rest subscripts

Данный предикат проверяет, являются ли индексы \emph{subscripts} для массива
\emph{array} корректными.
Если они корректны, предикат истинен, иначе ложен. \emph{subscripts} должен
быть целыми числами. Количество индексов \emph{subscripts} должно равняться
рангу массива.
Как и \cdf{aref}, \cdf{array-in-bounds-p} игнорирует указатели заполнения.
\end{defun}

\begin{defun}[Функция]
array-row-major-index array &rest subscripts

Данная функция принимает массив и корректные для него индексы и возвращает
одиночное неотрицательное целое значение меньшее чем общий размер массива,
которое идентифицирует элемент, полагаясь на построчный порядок хранения
элементов.
Количество указанных индексов \emph{subscripts} должно равняться рангу массива.
Каждый индекс должен быть неотрицательным целым числом меньшим чем соответствующий
размер измерения.
Как и \cdf{aref}, \cdf{array-row-major-index} игнорирует указатели заполнения.

Возможно определение \cdf{array-row-major-index}, без проверки на ошибки, может
выглядеть так:
\begin{lisp}
(defun array-row-major-index (a \cd{\&rest} subscripts) \\
~~(apply \#'+ (maplist \#'(lambda (x y) \\
~~~~~~~~~~~~~~~~~~~~~~~~~~(* (car x) (apply \#'* (cdr y)))) \\
~~~~~~~~~~~~~~~~~~~~~~subscripts \\
~~~~~~~~~~~~~~~~~~~~~~(array-dimensions a))))
\end{lisp}
Для одномерного массива, результат \cdf{array-row-major-index} всегда равен
переданному индексу.
\end{defun}

\begin{defun}[Функция]
row-major-aref array index

Данная функция позволяет получить доступ к элементу, как если бы массив был
одномерный.
Аргумент \emph{index} должен быть неотрицательным целым меньшим чем общий размер
массива \emph{array}. Данная функция индексирует массив, как если бы он был
одномерный с построчным порядком.
Эту функцию можно понять в терминах \cdf{aref}:
\begin{lisp}
(row-major-aref \emph{array} \emph{index}) \EQ \\*
~~(aref (make-array (array-total-size array)) \\*
~~~~~~~~~~~~~~~~~~~~:displaced-to array \\*
~~~~~~~~~~~~~~~~~~~~:element-type (array-element-type array)) \\*
~~~~~~~~index)
\end{lisp}
Другими словами, можно обработать массив как одномерный с помощью создания
нового одномерного массива, который \emph{соединён} с исходным, и получить
доступ к новому массиву.
И наоборот, \cdf{aref} может быть описана в терминах \cdf{row-major-aref}:
\begin{lisp}
(aref \emph{array} $\emph{i}_0$ $\emph{i}_1$ ... $\emph{i}_{n-1}$) \EQ \\*
~~(row-major-aref array \\*
~~~~~~~~~~~~~~~~~~(array-row-major-index array $\emph{i}_0$ $\emph{i}_1$ ... $\emph{i}_{n-1}$)
\end{lisp}

Как и \cdf{aref}, \cdf{row-major-aref} полностью игнорирует указатели
заполнения.
Для изменения элемента массива, можно комбинировать вызов \cdf{row-major-aref} с
формой \cdf{setf}.

Эта операция облегчает написание кода, который обрабатывает массивы различных
рангов. Предположим, что необходимо обнулить содержимое массива
\cdf{tennis-scores}. Можно решить это так:
\begin{lisp}
(fill (make-array (array-total-size tennis-scores) \\*
~~~~~~~~~~~~~~~~~~:element-type (array-element-type tennis-scores) \\*
~~~~~~~~~~~~~~~~~~:displaced-to tennis-scores) \\*
~~~~~~0)
\end{lisp}
К сожалению, так как \cdf{fill} не может принимать многомерные массивы, в данном
примере создаётся \emph{соединённый} массив, что является лишней операцией.
Другим способом является отдельная обработка каждого измерения многомерного
массива:
\begin{lisp}
(ecase (array-rank tennis-scores) \\*
~~(0 (setf (aref tennis-scores) 0)) \\
~~(1 (dotimes (i0 (array-dimension tennis-scores 0)) \\*
~~~~~~~(setf (aref tennis-scores i0) 0))) \\
~~(2 (dotimes (i0 (array-dimension tennis-scores 0)) \\*
~~~~~~~(dotimes (i1 (array-dimension tennis-scores 1)) \\*
~~~~~~~~~(setf (aref tennis-scores i0 i1) 0)))) \\*
~~... \\
~~(7 (dotimes (i0 (array-dimension tennis-scores 0)) \\*
~~~~~~~(dotimes (i1 (array-dimension tennis-scores 1)) \\
~~~~~~~~~(dotimes (i2 (array-dimension tennis-scores 1)) \\
~~~~~~~~~~~(dotimes (i3 (array-dimension tennis-scores 1)) \\
~~~~~~~~~~~~~(dotimes (i4 (array-dimension tennis-scores 1)) \\
~~~~~~~~~~~~~~~(dotimes (i5 (array-dimension tennis-scores 1)) \\
~~~~~~~~~~~~~~~~~(dotimes (i6 (array-dimension tennis-scores 1)) \\*
~~~~~~~~~~~~~~~~~~~(setf (aref tennis-scores i0 i1 i2 i3 i4 i5 i6) \\*
~~~~~~~~~~~~~~~~~~~~~~~~~0))))))))) \\*
~~)
\end{lisp}
От такого кода быстро приходит усталость. Кроме того, данный подход не
желателен, так как некоторые реализации Common Lisp'а будут фактически
поддерживать не более 7 измерений. Рекурсивно вложенные циклы справляются с
задачей лучше, но код всё ещё выглядит как лапша:
\begin{lisp}
(labels \\*
~~((grok-any-rank (\&rest indices) \\*
~~~~~(let ((d (- (array-rank tennis-scores) (length indices))) \\*
~~~~~~~(if (= d 0) \\*
~~~~~~~~~~~(setf (apply \#'row-major-aref indices) 0) \\*
~~~~~~~~~~~(dotimes (i (array-dimension tennis-scores (- d 1))) \\*
~~~~~~~~~~~~~(apply \#'grok-any-rank i indices)))))) \\*
~~(grok-any-rank))
\end{lisp}
Является ли этот код эффективным зависит от многих параметров реализации, таких
как способ обработки \cd{\&rest} аргументов и компиляции \cdf{apply} вызовов.
Только посмотрите как просто использовать для задачи \cdf{row-major-aref}!
\begin{lisp}
(dotimes (i (array-total-size tennis-scores)) \\*
~~(setf (row-major-aref tennis-scores i) 0))
\end{lisp}
Нет сомнения, что этот код, слаще любых медовых сот.
\end{defun}

\begin{defun}[Функция]
adjustable-array-p array

Если аргумент, который должен быть массивом, может быть расширен, данный
предикат истинен, иначе ложен.
\end{defun}

\section{Функции для битовых массивов}

Функции описанные в данном разделе работают только с массивами битов, то есть,
со специализированными массивами, элементы которых могут принимать значения
только либо \cd{0}, либо \cd{1}.

\begin{defun}[Функция]
bit bit-array &rest subscripts \\
sbit simple-bit-array &rest subscripts

Функция \cdf{bit} похожа на \cdf{aref}, но принимает только массив битов, то
есть массив типа \cd{(array bit)}.
Результатом всегда является \cd{0} или \cd{1}.
\cdf{sbit} похожа на \cdf{bit}, но дополнительно требует, чтобы первый аргумент
был \emph{простым} массивом (смотрите раздел~\ref{ARRAY-TYPE-SECTION}).
Следует отметить, что \cdf{bit} и \cdf{sbit}, в отличие от \cdf{char} и
\cdf{schar}, могут принимать массив любого ранга.

Для замены элемента массива может использоваться \cdf{setf} в связке с \cdf{bit}
или \cdf{sbit}.

\cdf{bit} и \cdf{sbit} идентичны \cdf{aref} за исключением того, что принимают
только специализированные массивы.
В некоторых реализациях Common Lisp'а \cdf{bit} и \cdf{sbit} могут быть быстрее
чем \cdf{aref} в
тех случаях, где они применимы.
\end{defun}

\begin{defun}[Функция]
bit-and bit-array1 bit-array2 &optional result-bit-array \\
bit-ior bit-array1 bit-array2 &optional result-bit-array \\
bit-xor bit-array1 bit-array2 &optional result-bit-array \\
bit-eqv bit-array1 bit-array2 &optional result-bit-array \\
bit-nand bit-array1 bit-array2 &optional result-bit-array \\
bit-nor bit-array1 bit-array2 &optional result-bit-array \\
bit-andc1 bit-array1 bit-array2 &optional result-bit-array \\
bit-andc2 bit-array1 bit-array2 &optional result-bit-array \\
bit-orc1 bit-array1 bit-array2 &optional result-bit-array \\
bit-orc2 bit-array1 bit-array2 &optional result-bit-array

Эти функции выполняют побитовые логические операции над битовыми массивами.
Все аргументы для этих функций должны быть битовыми массивами одинакового ранга
и измерений.
Результат является битовым массивом с такими же рангом и измерениями, что и
аргументами. Каждый бит получается применением соответствующей операции к битам
исходных массивов.

Если третий аргумент опущен или является {\false}, создаётся новый массив,
который будет содержать результат. Если третий аргумент это битовый массив, то
результат помещается в этот массив. Если третий аргумент является {\true}, тогда
для третьего аргумента используется первый массив. Таким образом результат
помещается в массив переданный первым аргументом.

Следующая таблица отображает результаты применения битовых операций.
\begin{flushleft}
\cf
\begin{tabular*}{\textwidth}{@{}l@{\extracolsep{\fill}}lllll@{}}
~~~\emph{argument1}~~&0&0&1&1 \\
~~~\emph{argument2}~~&0&1&0&1&\emph{Имя операции} \\
\hlinesp
bit-and&0&0&0&1&\textrm{и} \\
bit-ior&0&1&1&1&\textrm{или} \\
bit-xor&0&1&1&0&\textrm{исключающее или} \\
bit-eqv&1&0&0&1&\textrm{равенство (исключающее не-или)} \\
bit-nand&1&1&1&0&\textrm{не-и} \\
bit-nor&1&0&0&0&\textrm{не-или} \\
bit-andc1&0&1&0&0&\textrm{не-\emph{argument1} и \emph{argument2}} \\
bit-andc2&0&0&1&0&\textrm{\emph{argument1} и не-\emph{argument2}} \\
bit-orc1&1&1&0&1&\textrm{не-\emph{argument1} или \emph{argument2}} \\
bit-orc2&1&0&1&1&\textrm{\emph{argument1} или не-\emph{argument2}} \\
\hline
\end{tabular*}
\end{flushleft}
Например:
\begin{lisp}
(bit-and \#*1100 \#*1010) \EV\ \#*1000 \\
(bit-xor \#*1100 \#*1010) \EV\ \#*0110 \\
(bit-andc1 \#*1100 \#*1010) \EV\ \#*0100
\end{lisp}
Смотрите также \cdf{logand} и связанные с ней функции.
\end{defun}

\begin{defun}[Функция]
bit-not bit-array &optional result-bit-array

Первый аргумент должен быть массивом битов. Результатом является битовый массив
с таким же рангом и измерениями и инвертированными битами.
Смотрите также \cdf{lognot}.

Если второй аргумент опущен или равен {\false}, для результата создаётся новый
массив. Если второй аргумент является битовым массивом, результат помещается в
него. Если второй аргумент является {\true}, то для результата используется
массив из первого аргумента. То есть результат помещается в исходный массив.
\end{defun}

\section{Указатели заполнения}
\label{FILL-POINTER}

Common Lisp предоставляет несколько функций для управления \emph{указателями
  заполнения}. Они позволяют последовательно заполнять содержимое вектора. А
если точнее, они позволяют эффективно менять длину вектора.
Например, строка с указателем заполнения имеет большинство характеристик строки
с переменной длиной из PL/I.

Указатель заполнения является неотрицательным целым числом не большим чем общее
количество элементов в векторе (полученное от \cdf{array-dimension}).
Указатель заполнения указывает на <<активные>> или <<заполненные>> элементы
вектора.
Указатель заполнение отображает <<активную длину>> вектора.
Все элементы вектора, индекс которых меньше чем указатель заполнения, являются
активными. Остальные элементы являются неактивными.
Почти все функции, которые оперируют содержимым вектора, будут оперировать
только активными элементами. Исключением является \cdf{aref}, которая может
получать доступ ко всем элементам массива, активным и неактивным. Следует
отметить, что элементы вектора из неактивной части тем не менее являются частью
вектора.

Указатели заполнения могут иметь только вектора (одномерные массивы).
Многомерные массивы не могут иметь указатели заполнения. (Следует отметить,
однако, что можно создать 
многомерный массив \emph{соединённый} с вектором, у которого есть указатель
заполнения.)

\begin{defun}[Функция]
array-has-fill-pointer-p array

Аргумент должен быть массивом. \cdf{array-has-fill-pointer-p} возвращает
{\true}, если массив умеет указатель заполнения, иначе возвращает {\false}.
Следует отметить, что \cdf{array-has-fill-pointer-p} всегда возвращает {\false},
если массив \emph{array} не одномерный.
\end{defun}

\begin{defun}[Функция]
fill-pointer vector

Данная функция возвращает указатель заполнения вектора \emph{vector}. Если
вектор \emph{vector} не имеет указателя заполнения, генерируется ошибка.

Для изменения указателя заполнения вектора может использоваться функция
\cdf{setf} в связке с \cdf{fill-pointer}. Указатель заполнения вектора должен
быть всегда целым числом между нулём и размером вектора (включительно).
\end{defun}

\begin{defun}[Функция]
vector-push new-element vector

Аргумент \emph{vector} должен быть одномерным массивом, имеющим указатель
заполнения, и \emph{new-element} может быть любым объектом.
\cdf{vector-push} пытается сохранить \emph{new-element} в элемент вектора, на
который ссылается указатель заполнения, и увеличить этот указатель на 1. Если
указатель заполнения не определяет элемент вектора (например, когда он
становится слишком большим), то \cdf{vector-push} возвращает {\false}. В
противном случае, если вставка нового элемента произошла, \cdf{vector-push}
возвращает \emph{предыдущее} значение указателя. Таким образом, \cdf{vector-push}
является индексом вставленного элемента.

Можно сравнить \cdf{vector-push}, которая является функцией, с \cdf{push},
который являет макросом, который принимает \emph{место} подходящее для
\cdf{setf}.
Вектор с указателем заполнения содержит такое \emph{место} в слоте
\cdf{fill-pointer}. FIXME
\end{defun}

\begin{defun}[Функция]
vector-push-extend new-element vector &optional extension

\cdf{vector-push-extend} похожа на \cdf{vector-push} за исключением того, что
если указатель заполнения стал слишком большим, длина вектора увеличивается (с
помощью \cdf{adjust-array}), и новый элемент помещается в вектор.
Однако, если вектор не расширяемый, тогда \cdf{vector-push-extend} сигнализирует
ошибку.

Необязательный аргумент \emph{extension}, который должен быть положительным
целым, является минимальным количеством элементов, добавляемых в вектор, если
последний должен быть расширен. По-умолчанию содержит значение зависимое от
реализации.
\end{defun}

\begin{defun}[Функция]
vector-pop vector

Аргумент \emph{vector} должен быть одномерным массивом, который имеет указатель
заполнения.
Если указатель заполнения является нулём, \cdf{vector-pop} сигнализирует ошибку.
В противном случае, указатель заполнения уменьшается на 1, и в качестве значения
функции возвращается обозначенный указателем элемент вектора.
\end{defun}

\section{Изменение измерений массива}

Следующие функции могут использоваться для изменения размера или формы массива.
Их опции почти совпадают с опциями функции \cdf{make-array}.

\begin{defun}[Функция]
adjust-array array new-dimensions &key :element-type :initial-element :initial-contents :fill-pointer :displaced-to :displaced-index-offset

\cdf{adjust-array} принимает массив и те же аргументы, что и для
\cdf{make-array}. Количество измерений, указанных в \emph{new-dimensions} должно
равняться рангу массива \emph{array}.

\cdf{adjust-array} возвращает массив такого же типа и ранга, что и массив
\emph{array}, но с другим измерениями \emph{new-dimensions}. Фактически, аргумент
\emph{array} модифицируется в соответствие с новыми указаниями. Но это может быть
достигнуто двумя путями: модификацией массива \emph{array} или созданием нового
массива и модификацией аргумента \emph{array} для \emph{соединения} его с новым
массивом.

В простейшем случае, можно указать только измерения \emph{new-dimensions} и,
возможно, аргумент \cd{:initial-element}.
Не пустые элементы массива \emph{array} вставляются в новый массив. Элементы
нового массива, которые не были заполнены значениями из старого массива,
получают значение из \cd{:initial-element}. Если аргумент не был указан, то
первоначальное значение элементов не определено.

Если указан \cd{:element-type}, тогда массив \emph{array} должен был быть создан
с указанием такого же типа. В противном случае сигнализируется ошибка.
Указание \cd{:element-type} в \cdf{adjust-array} служит только для проверки на
существование ошибки несовпадения типов.

Если указаны \cd{:initial-contents} или \cd{:displaced-to}, тогда они
обрабатываются как для \cdf{make-array}. В таком случае, 

Если указан \cd{:fill-pointer}, тогда указатель заполнения массива \emph{array}
принимает указанное значение. Если массив \emph{array} не содержал указателя
заполнения сигнализируется ошибка.

\cdf{adjust-array} может, в зависимости от реализации и аргументов, просто
изменить исходный массив или создать и вернуть новый.
В последнем случае исходный массив изменяется, а именно \emph{соединяется} с
новым массивом, и имеет новые измерения.

Если \cd{adjust-array} применяется к массиву \emph{array}, который соединён с
другим массивом \emph{x}, тогда ни массив \emph{array}, ни возвращённый
результат не будет \emph{соединены} с \emph{x}, если только такое соединения не
задано явно в вызове \cdf{adjust-array}.

Например, предположим что массив 4-на-4 \cdf{m} выглядит так:
\begin{lisp}
\#2A(~(~alpha~~~~~beta~~~~~~gamma~~~~~delta~) \\
~~~~~(~epsilon~~~zeta~~~~~~eta~~~~~~~theta~) \\
~~~~~(~iota~~~~~~kappa~~~~~lambda~~~~mu~~~~) \\
~~~~~(~nu~~~~~~~~xi~~~~~~~~omicron~~~pi~~~~)~)
\end{lisp}
Тогда результат выражения
\begin{lisp}
(adjust-array m '(3 5) :initial-element 'baz)
\end{lisp}
является массивом 3-на-5 с содержимым
\begin{lisp}
\#2A(~(~alpha~~~~~beta~~~~~~gamma~~~~~delta~~~~~baz~) \\
~~~~~(~epsilon~~~zeta~~~~~~eta~~~~~~~theta~~~~~baz~) \\
~~~~~(~iota~~~~~~kappa~~~~~lambda~~~~mu~~~~~~~~baz~)~)
\end{lisp}
Следует отметить, что если массив \cd{a} создаётся \emph{соединённым} с массивом
\cd{b} и затем массив \cd{b} передаётся в \cdf{adjust-array}, то массив \cd{a}
все ещё будет \emph{соединён} с массивом \cd{b}. При этом должны быть приняты во
внимание правила \emph{соединения} массивов и построчный порядок следования
элементов.
\end{defun}

\fi       % Hairy array stuff
%Part{String, Root = "CLM.MSS"}
%%%Chapter of Common Lisp Manual.  Copyright 1984, 1988, 1989 Guy L. Steele Jr.


\chapter{Strings Строки}
\def\pagestatus{FINAL PROOF}

A string is a specialized vector (one-dimensional array)
whose elements are characters.

Строка является специализированным вектором (или одномерным массивом), элементы
которого --- строковые символы.

\begin{obsolete}
Specifically, the type \cdf{string}
is identical to the type \cd{(vector string-char)}, which in turn
is the same as \cd{(array string-char (*))}.
\end{obsolete}

\begin{newer}
X3J13 voted in March 1989 \issue{CHARACTER-PROPOSAL}
to eliminate the type \cdf{string-char} and to redefine the type
\cdf{string} to be the union of one or more specialized vector
types, the types of whose elements are subtypes of the type \cdf{character}.
\end{newer}

Any string-specific function defined in this chapter
whose name begins with the prefix \cdf{string}
will accept a symbol instead of a string
as an argument \emph{provided} that the operation never modifies that argument;
the print name of the symbol is used.
\indexterm{print name}
In this respect the string-specific sequence operations are not
simply specializations of generic versions; the generic sequence
operations described in chapter \ref{KSEQUE} never accept symbols as sequences.
This slight inelegance is permitted in Common Lisp in the name of pragmatic utility.
One may get the effect of having a generic sequence function
operate on either symbols or strings by applying the coercion
function \cdf{string} to any argument whose data type is in doubt.

Любая функция, определенная в данной главе, имя которой имеет префикс
\cdf{string}, в качестве аргумента может принимать символ, при
условии, что операция не модифицирует этот аргумент;
использоваться будет выводимое имя символа.
Таким образом операции над последовательностями из строковый символов, не
являются специализированными версиями обобщенных функций; обобщенные операции
над последовательностями, описанные в главе~\ref{KSEQUE} не принимают символ в
качестве последовательности. Такая <<неизящность>> сделана в Common Lisp'е в
целях прагматичности. Для достижения унификации функций, предлагается использовать
функцию \cdf{string} применительно ко всем аргументам, тип которых не известен заранее.

\begin{new}
Note that this remark, predating the design of the Common Lisp Object System,
uses the term ``generic'' in a generic sense and not necessarily
in the technical sense used by CLOS
(see chapter \ref{DTYPES}).

Следует отметить, что здесь употребление 
\end{new}

Also, there is a slight non-parallelism in the names of string functions.
Where the suffixes \cdf{equalp} and \cdf{eql} would be more appropriate,
for historical compatibility the suffixes \cdf{equal} and \cd{=} are used instead
to indicate case-insensitive and case-sensitive character comparison,
respectively.

Any Lisp object may be tested for being a string by
the predicate \cdf{stringp}.

Note that strings, like all vectors, may have fill pointers
(though such strings are not necessarily \emph{simple}).
String operations generally operate only on the active portion of the string
(below the fill pointer).  See \cdf{fill-pointer} and related
functions.

\section{String Access}

The following functions access a single character element of a string.

\begin{defun}[Function]
char string index \\
schar simple-string index

The given \emph{index} must be a non-negative integer less than
the length of \emph{string}, which must be a
string.  The character at position \emph{index}
of the string is returned as a character object.
\begin{obsolete}
(This character will necessarily satisfy the predicate \cdf{string-char-p}.)
\end{obsolete}
\begin{newer}
X3J13 voted in March 1989 \issue{CHARACTER-PROPOSAL}
to eliminate \cdf{string-char-p}.
\end{newer}
As with all sequences in Common Lisp, indexing is zero-origin.
For example:
\begin{lisp}
(char "Floob-Boober-Bab-Boober-Bubs" 0) \EV\ \#{\Xbackslash}F \\
(char "Floob-Boober-Bab-Boober-Bubs" 1) \EV\ \#{\Xbackslash}l
\end{lisp}
See \cdf{aref} and \cdf{elt}.  In effect,
\begin{lisp}
(char s j) \EQ\ (aref (the string s) j)
\end{lisp}
\cdf{setf} may be used with \cdf{char} to destructively replace
a character within a string.

For \cdf{char}, the string may be any string;
for \cdf{schar}, it must be a simple string.
In some implementations of Common Lisp, the function \cdf{schar} may
be faster than \cdf{char} when it is applicable.
\end{defun}

\section{String Comparison}

The naming conventions for these functions and for their keyword
arguments generally follow the conventions for the generic sequence
functions (see chapter \ref{KSEQUE}).
\begin{new}
Note that this remark, predating the design of the Common Lisp Object System,
uses the term ``generic'' in a generic sense and not necessarily
in the technical sense used by CLOS
(see chapter \ref{DTYPES}).
\end{new}

\begin{defun}[Function]
string= string1 string2 &key :start1 :end1 :start2~:end2

\cd{string=} compares two strings and is true if
they are the same (corresponding characters are identical)
but is false if they are not.
The function \cdf{equal} calls \cd{string=} if
applied to two strings.

The keyword arguments \cd{:start1} and \cd{:start2} are the places
in the strings to start the comparison.
The arguments \cd{:end1} and \cd{:end2} are the
places in the strings to stop comparing; comparison stops just
\emph{before} the position specified by a limit.
The ``start'' arguments default to zero (beginning of string),
and the ``end'' arguments (if either omitted or {\false})
default to the lengths of the strings (end of string),
so that by default the entirety of each string is examined.
These arguments are provided so that substrings can be compared
efficiently.

\cd{string=} is necessarily false if the (sub)strings
being compared are of unequal length; that is, if
\begin{lisp}
(not (= (- end1 start1) (- end2 start2)))
\end{lisp}
is true, then \cd{string=} is false.

\begin{lisp}
(string= "foo" "foo") {\rm is true} \\
(string= "foo" "Foo") {\rm is false} \\
(string= "foo" "bar") {\rm is false} \\
(string= "together" "frog" :start1 1 :end1 3 :start2 2) \\
~~~{\rm is true}
\end{lisp}

\begin{newer}
X3J13 voted in June 1989 \issue{STRING-COERCION}
to clarify string coercion (see \cdf{string}).
\end{newer}

\beforenoterule
\begin{incompatibility}
\cd{string=} is called \cdf{strequal} in Interlisp.
\end{incompatibility}
\afternoterule
\end{defun}

\begin{defun}[Function]
string-equal string1 string2 &key :start1 :end1 :start2~:end2

\cdf{string-equal} is just like \cd{string=} except that differences
in case are ignored; two characters are considered to be the same
if \cdf{char-equal} is true of them.
For example:
\begin{lisp}
(string-equal "foo" "Foo") {\rm is true}
\end{lisp}
\begin{newer}
X3J13 voted in June 1989 \issue{STRING-COERCION}
to clarify string coercion (see \cdf{string}).
\end{newer}
\end{defun}

\begin{defun}[Function]
string< string1 string2 &key :start1 :end1 :start2~:end2 \\
string> string1 string2 &key :start1 :end1 :start2~:end2 \\
string<= string1 string2 &key :start1 :end1 :start2~:end2 \\
string>= string1 string2 &key :start1 :end1 :start2~:end2 \\
string/= string1 string2 &key :start1 :end1 :start2~:end2

These functions compare the two string arguments lexicographically,
and the result is {\false} unless \emph{string1} is respectively
less than, greater than,
less than or equal to, greater than or equal to, or not equal to \emph{string2}.
If the condition is satisfied, however, then
the result is the index within the strings of the first character
position at which the strings fail to match; put another way,
the result is the length of the longest common prefix of the strings.

A string \emph{a} is less than a string \emph{b} if
in the first position in which they differ the character of \emph{a}
is less than the corresponding character of \emph{b} according to
the function \cd{char<}, or
if string \emph{a} is a proper prefix of string \emph{b}
(of shorter length and matching in all the characters of \emph{a}).

The keyword arguments \cd{:start1} and \cd{:start2} are the places
in the strings to start the comparison.
The keyword arguments \cd{:end1} and \cd{:end2}
are the places in the strings to stop comparing; comparison stops just
\emph{before} the position specified by a limit.
The ``start'' arguments default to zero (beginning of string),
and the ``end'' arguments (if either omitted or {\false})
default to the lengths of the strings (end of string),
so that by default the entirety of each string is examined.
These arguments are provided so that substrings can be compared
efficiently.  The index returned in case of a mismatch
is an index into \emph{string1}.

\begin{newer}
X3J13 voted in June 1989 \issue{STRING-COERCION}
to clarify string coercion (see \cdf{string}).
\end{newer}
\end{defun}

\begin{defun}[Function]
string-lessp string1 string2 &key :start1 :end1 :start2~:end2 \\
string-greaterp string1 string2 &key :start1 :end1 :start2~:end2 \\
string-not-greaterp string1 string2 &key :start1 :end1 :start2~:end2 \\
string-not-lessp string1 string2 &key :start1 :end1 :start2~:end2 \\
string-not-equal string1 string2 &key :start1 :end1 :start2~:end2

These are exactly like \cd{string<}, \cd{string>}, \cd{string<=},
\cd{string>=}, and \cd{string/=}, respectively, except that distinctions between
uppercase and lowercase letters are ignored.  It is as if
\cdf{char-lessp} were used instead of \cd{char<}
for comparing characters.

\begin{newer}
X3J13 voted in June 1989 \issue{STRING-COERCION}
to clarify string coercion (see \cdf{string}).
\end{newer}
\end{defun}

\section{String Construction and Manipulation}

Most of the interesting operations on strings may be performed
with the generic sequence functions described in chapter \ref{KSEQUE}.
The following functions perform additional operations that are specific
to strings.
\begin{new}
Note that this remark, predating the design of the Common Lisp Object System,
uses the term ``generic'' in a generic sense and not necessarily
in the technical sense used by CLOS
(see chapter \ref{DTYPES}).
\end{new}

\begin{obsolete}
\begin{defun}[Function]
make-string size &key :initial-element

This returns a string (in fact a simple string)
of length \emph{size}, each of whose characters
has been initialized to the \cd{:initial-element} argument.
If an \cd{:initial-element} argument is not specified, then the string will
be initialized in an implementation-dependent way.

\beforenoterule
\begin{implementation}
It may be convenient to initialize the string
to null characters, or to spaces, or to garbage (``whatever was there'').
\end{implementation}
\afternoterule

A string is really just a one-dimensional array of ``string
characters'' (that is, those characters that are members of type
\cdf{string-char}).  More complex character arrays may be constructed using the
function \cdf{make-array}.
\end{defun}
\end{obsolete}

\begin{newer}
X3J13 voted in March 1989 \issue{CHARACTER-PROPOSAL}
to eliminate the type \cdf{string-char} and to add a keyword
argument \cd{:element-type} to \cdf{make-string}.  The new function
description is as follows.

\begin{defun}[Function]
make-string size &key :initial-element :element-type

This returns a simple string
of length \emph{size}, each of whose characters
has been initialized to the \cd{:initial-element} argument.
If an \cd{:initial-element} argument is not specified, then the string will
be initialized in an implementation-dependent way.

The \cd{:element-type} argument names the type of the elements
of the string; a string is constructed of the most specialized type
that can accommodate elements of the given type.  If \cd{:element-type}
is omitted, the type \cdf{character} is the default.

X3J13 voted in January 1989
\issue{ARGUMENTS-UNDERSPECIFIED}
to clarify that the \emph{size} argument
must be a non-negative integer less than the value of
\cdf{array-dimension-limit}.
\end{defun}
\end{newer}



\begin{defun}[Function]
string-trim character-bag string \\
string-left-trim character-bag string \\
string-right-trim character-bag string

\cdf{string-trim} returns a substring of \emph{string}, with all characters in
\emph{character-bag} stripped off the beginning and end.
The function \cdf{string-left-trim} is similar but strips characters
off only the beginning; \cdf{string-right-trim} strips off only the end.
The argument \emph{character-bag} may be any sequence containing
characters.
For example:
\begin{lisp}
(string-trim '(\#{\Xbackslash}Space \#{\Xbackslash}Tab \#{\Xbackslash}Newline) " garbanzo beans \\
~~~~~~~~") \EV\ "garbanzo beans" \\
(string-trim " (*)" " ( *three (silly) words* ) ") \\
~~~\EV\ "three (silly) words" \\
(string-left-trim " (*)" " ( *three (silly) words* ) ") \\
~~~\EV\ "three (silly) words* ) " \\
(string-right-trim " (*)" " ( *three (silly) words* ) ") \\
~~~\EV\ " ( *three (silly) words"
\end{lisp}
If no characters need to be trimmed from the \emph{string},
then either the argument \emph{string} itself or a copy of it may
be returned, at the discretion of the implementation.

\begin{newer}
X3J13 voted in June 1989 \issue{STRING-COERCION}
to clarify string coercion (see \cdf{string}).
\end{newer}
\end{defun}

\begin{defun}[Function]
string-upcase string &key :start :end \\
string-downcase string &key :start :end \\
string-capitalize string &key :start :end

\cdf{string-upcase} returns a string just like \emph{string} with all lowercase
characters replaced by the corresponding uppercase characters.  More
precisely, each character of the result string is produced by applying
the function \cdf{char-upcase} to the corresponding character of
\emph{string}.

\cdf{string-downcase} is similar, except that uppercase characters are
converted to lowercase characters (using \cdf{char-downcase}).

The keyword arguments \cd{:start} and \cd{:end} delimit the portion
of the string to be affected.  The result is always of the same length
as \emph{string}, however.

The argument is not destroyed.  However, if no characters in the argument
require conversion, the result may be either the argument or a copy of it,
at the implementation's discretion.
For example:
\begin{lisp}
(string-upcase "Dr. Livingstone, I presume?") \\
~~~\EV\ "DR. LIVINGSTONE, I PRESUME?" \\
(string-downcase "Dr. Livingstone, I presume?") \\
~~~\EV\ "dr. livingstone, i presume?" \\
(string-upcase "Dr. Livingstone, I presume?" \cd{:start} 6 \cd{:end} 10) \\
~~~\EV\ "Dr. LiVINGstone, I presume?"
\end{lisp}

\cdf{string-capitalize} produces a copy of \emph{string} such that,
for every word in the copy, the first character of the word,
if case-modifiable, is uppercase and
any other case-modifiable characters in the word are lowercase.
For the purposes of \cdf{string-capitalize},
a word is defined to be a
consecutive subsequence consisting of alphanumeric characters or digits,
delimited at each end either by a non-alphanumeric character
or by an end of the string.
For example:
\begin{lisp}
(string-capitalize " hello ") \EV\ " Hello " \\
(string-capitalize \\
~~~~\="occlUDeD cASEmenTs FOreSTAll iNADVertent DEFenestraTION") \\
\EV\>"Occluded Casements Forestall Inadvertent Defenestration" \\
(string-capitalize 'kludgy-hash-search) \EV\ "Kludgy-Hash-Search" \\
(string-capitalize "DON'T!") \EV\ "Don'T!"~~~~~;\emph{not} "Don't!" \\
(string-capitalize "pipe 13a, foo16c") \EV\ "Pipe 13a, Foo16c"
\end{lisp}

\begin{newer}
X3J13 voted in June 1989 \issue{STRING-COERCION}
to clarify string coercion (see \cdf{string}).
\end{newer}

\beforenoterule
\begin{incompatibility}
Some very approximate Interlisp equivalents to
\cdf{string-upcase}, \cdf{string-downcase}, and \cdf{string-capitalize}
are \cdf{u-case}, \cdf{l-case} with second argument {\nil},
and \cdf{l-case} with second argument {\true}.
\end{incompatibility}
\afternoterule
\end{defun}

\begin{defun}[Function]
nstring-upcase string &key :start :end \\
nstring-downcase string &key :start :end \\
nstring-capitalize string &key :start :end

These three functions are just like \cdf{string-upcase},
\cdf{string-downcase}, and \cdf{string-capitalize}
but destructively modify the argument \emph{string} by altering
case-modifiable characters as necessary.

The keyword arguments \cd{:start} and \cd{:end} delimit the portion
of the string to be affected.  The argument \emph{string} is returned as
the result.
\end{defun}

\begin{defun}[Function]
string x

Most of the string
functions effectively apply \cdf{string}
to such of their arguments as are supposed to be
strings.
If \emph{x} is a string, it is returned.
If \emph{x} is a symbol, its print name is returned.
\begin{obsolete}
If \emph{x} is a string character (a character of type \cdf{string-char}),
then a string containing that one character is returned.
\end{obsolete}
\begin{newer}
X3J13 voted in March 1989 \issue{CHARACTER-PROPOSAL}
to eliminate the type \cdf{string-char} and to redefine the type
\cdf{string} to be the union of one or more specialized vector
types, the types of whose elements are subtypes of the type \cdf{character}.
Presumably converting a character to a string always works according
to this vote.
\end{newer}
In any other situation, an error is signaled.

To convert a sequence of characters to a string, use \cdf{coerce}.
(Note that \cd{(coerce x 'string)} will not succeed if \cdf{x} is a symbol.
Conversely, \cdf{string} will not convert a list or other sequence
to be a string.)

To get the string representation of a number or any other Lisp
object, use \cd{prin1-to-string}, \cdf{princ-to-string},
or \cdf{format}.

\begin{newer}
X3J13 voted in June 1989 \issue{STRING-COERCION}
to specify that the following functions perform coercion
on their \emph{string} arguments identical to that performed
by the function \cdf{string}.

\begin{flushleft}
\cf
\begin{tabular*}{\textwidth}{@{}l@{\extracolsep{\fill}}ll@{}}
string= & string-equal & string-trim \\
string< & string-lessp &  string-left-trim \\
string> & string-greaterp &  string-right-trim \\
string<= & string-not-greaterp & string-upcase \\
string>= & string-not-lessp & string-downcase \\
string/= & string-not-equal & string-capitalize
\end{tabular*}
\end{flushleft}
Note that \cdf{nstring-upcase}, \cdf{nstring-downcase}, and
\cdf{nstring-capitalize} are absent from this list; because they modify destructively,
the argument must be a string.

As part of the same vote X3J13 specified that \cdf{string}
may perform additional implementation-dependent coercions
but the returned value must be of type \cdf{string}.
Only when no coercion is defined, whether standard or implementation-dependent,
is \cdf{string} required to signal an error, in which case the error condition
must be of type \cdf{type-error}.
\end{newer}
\end{defun}
      % Functions on strings
%Part{Struct, Root = "CLM.MSS"}
%%%Chapter of Common Lisp Manual.  Copyright 1984, 1988, 1989 Guy L. Steele Jr.

\clearpage\def\pagestatus{FINAL PROOF}

\ifx \rulang\Undef

\chapter{Structures}

Common Lisp provides a facility for creating named record structures
with named components.  In effect, the user can define a new data type;
every data structure of that type has components with specified names.
Constructor, access, and assignment constructs are automatically
defined when the data type is defined.

This chapter is divided into two parts.  The first part discusses
the basics of the structure facility, which is very simple and allows
the user to take advantage of the type-checking, modularity, and
convenience of user-defined record data types.  The second part,
beginning with section~\ref{Defstruct-Hairy-Stuff},
discusses a number of specialized features of the facility that
have advanced applications.  These features are completely optional,
and you needn't even know they exist in order to take
advantage of the basics.

\section{Introduction to Structures}
\label{DEFSTRUCT-INTRO-SECTION}

The structure facility is embodied in the \cdf{defstruct} macro,
which allows the user to create and use
aggregate data types with named elements.  These are like
``structures'' in {PL/I}, or ``records'' in Pascal.

As an example, assume you are writing a Lisp
program that deals with space ships in a two-dimensional plane.
In your program, you need to
represent a space ship by a Lisp object of some kind.  The interesting
things about a space ship, as far as your program is concerned, are
its position (represented as \emph{x} and \emph{y} coordinates),
velocity (represented as components along the \emph{x} and \emph{y} axes), and
mass.

A ship might therefore be represented as a record structure with five
components: \emph{x}-position, \emph{y}-position, \emph{x}-velocity, \emph{y}-velocity, and mass.
This structure could in turn be implemented as a Lisp object in a
number of ways.  It could be a list of five elements; the \emph{x}-position
could be the \emph{car}, the \emph{y}-position the \emph{cadr}, and so on.  Equally
well it could be a vector of five elements: the \emph{x}-position could be
element 0, the \emph{y}-position element 1, and so on.  The problem with either
of these representations is that the components occupy places in the
object that are quite arbitrary and hard to remember.  Someone looking at
\cd{(cadddr~ship1)} or \cd{(aref~ship1~3)} in a piece of code might
find it difficult to determine that this is accessing the \emph{y}-velocity
component of \cd{ship1}.  Moreover, if the representation of a ship should
have to be changed, it would be very difficult to find all the places
in the code to be changed to match (not all occurrences of \cdf{cadddr}
are intended to extract the \emph{y}-velocity from a ship).

Ideally components of record structures should have names.  One would
like to write something like
\cd{(ship-y-velocity ship1)} instead of \cd{(cadddr ship1)}.
One would also like a more mnemonic way to create a ship than this:
\begin{lisp}
(list 0 0 0 0 0)
\end{lisp}
Indeed, one would like \cdf{ship} to be a new data type, just like other
Lisp data types, that one could test with \cdf{typep}, for example.
The \cdf{defstruct} facility provides all of this.

\cdf{defstruct} itself is a macro that defines a structure.  For the
space ship example, one might define the structure by saying:
\begin{lisp}
(defstruct ship \\
~~x-position \\
~~y-position \\
~~x-velocity \\
~~y-velocity \\
~~mass)
\end{lisp}
This declares that every \cdf{ship} is an object with five named components.
The evaluation of this form does several things:

\begin{itemize}
\item
It defines \cdf{ship-x-position} to be a function
of one argument, a ship, that returns the \emph{x}-position
of the ship; \cdf{ship-y-position}
and the other components are given similar function definitions.
These functions are called the \emph{access functions}, as they
are used to access elements of the structure.

\item
The symbol \cdf{ship} becomes the name of a data type of which instances
of ships are elements.  This name becomes acceptable to \cdf{typep},
for example; \cd{(typep x 'ship)} is true if \cdf{x} is a ship
and false if \cdf{x} is any object other than a ship.

\item
A function named \cdf{ship-p} of one argument is defined; it is a predicate
that is true if its argument is a ship and is false otherwise.

\item
A function called \cdf{make-ship} is defined that, when invoked,
will create a data structure with five components, suitable for use with
the access functions.  Thus executing
\begin{lisp}
(setq ship2 (make-ship))
\end{lisp}
sets \cd{ship2} to a newly created \cdf{ship} object.
One can specify the initial values of any desired component in the call
to \cdf{make-ship} by using keyword arguments in this way:
\begin{lisp}
(setq ship2 (make-ship \cd{:mass} *default-ship-mass* \\
~~~~~~~~~~~~~~~~~~~~~~~\cd{:x-position} 0 \\
~~~~~~~~~~~~~~~~~~~~~~~\cd{:y-position} 0))
\end{lisp}
This constructs a new ship and initializes three of its components.
This function is called the \emph{constructor function}
because it constructs a new structure.

\item
The \cd{\#S} syntax can be used to read instances of \cdf{ship}
structures, and a printer function is provided for printing
out ship structures.  For example, the value of the
variable \cd{ship2} shown above might be printed as
\begin{lisp}
\#S(ship  x-position 0  y-position 0  x-velocity nil \\
~~~~~~~~~y-velocity nil  mass 170000.0)
\end{lisp}

\item
A function called \cdf{copy-ship} of one argument
is defined that, when given a \cdf{ship} object,
will create a new \cdf{ship} object that is a copy of the given one.
This function is called the \emph{copier function}.

\item
One may use \cdf{setf} to alter the components of a \cdf{ship}:
\begin{lisp}
(setf (ship-x-position ship2) 100)
\end{lisp}
This alters the \emph{x}-position of \emph{ship2} to be \cd{100}.
This works because \cdf{defstruct} behaves as if
it generates an appropriate \cdf{defsetf}
form for each access function.
\end{itemize}

This simple example illustrates the power of \cdf{defstruct} to provide
abstract record structures in a convenient manner.
\cdf{defstruct} has many other features as well for specialized purposes.

\section{How to Use Defstruct}

All structures are defined through the \cdf{defstruct} construct.
A call to \cdf{defstruct} defines a new data type whose instances
have named slots.

\begin{defmac}
defstruct name-and-options [doc-string] {slot-description}+

\begin{new}\noindent
X3J13 voted in June 1988
\issue{DEFSTRUCT-SLOTS-CONSTRAINTS-NUMBER}
to allow a \cdf{defstruct} definition
to have no \emph{slot-description} at all; in other words, the
occurrence of \Mplus{\emph{slot-description}} in the preceding
header line would be replaced by \Mstar{\emph{slot-description}}.

Such structure definitions are particularly useful if the
\cd{:include} option is used, perhaps with other options; for example,
one can have two structures that are exactly alike except that they
print differently (having different \cd{:print-function} options).

Implementors are encouraged to permit this simple extension as soon as
convenient.  Users, however, may wish to maximize portability of their
code by avoiding the use of this extension unless and until it is
adopted as part of the ANSI standard.
\end{new}

This defines a record-structure data type.
A general call to \cdf{defstruct} looks like the following example.
\begin{lisp}
(defstruct (\emph{name} \emph{option-1} \emph{option-2} ... \emph{option-m}) \\
~~~~~~~~~~~\emph{doc-string} \\
~~~~~~~~~~~\emph{slot-description-1} \\
~~~~~~~~~~~\emph{slot-description-2} \\
~~~~~~~~~~~... \\
~~~~~~~~~~~\emph{slot-description-n}) \\
\end{lisp}
The \emph{name} must be a symbol; it becomes the name of a new data type
consisting of all instances of the structure.
The function \cdf{typep} will accept and use this name
as appropriate.  The \emph{name} is returned as the value of the \emph{defstruct}
form.

Usually no options are needed at all.
If no options are specified, then one may write simply \emph{name} instead
of \cd{(\emph{name})} after the word \cdf{defstruct}.  The syntax of options
and the options provided are discussed in section~\ref{DEFSTRUCT-OPTIONS}.

If the optional documentation string \emph{doc-string} is present,
then it is attached to the \emph{name}
as a documentation string of type \cdf{structure}; see \cdf{documentation}.

Each \emph{slot-description-j} is of the form
\begin{lisp}
(\emph{slot-name} \emph{default-init} \\
~~~~~\emph{slot-option-name-1} \emph{slot-option-value-1} \\
~~~~~\emph{slot-option-name-2} \emph{slot-option-value-2} \\
~~~~~... \\
~~~~~\emph{slot-option-name-k${}_{j}$} \emph{slot-option-value-k${}_{j}$})
\end{lisp}
Each \emph{slot-name} must be a symbol; an access function is defined
for each slot. If no options and no \emph{default-init} are specified,
then one may write simply \emph{slot-name} instead of \cd{(\emph{slot-name})}
as the slot description.

\emph{default-init} form is evaluated only if the corresponding
argument is not supplied to the constructor function. 
The \emph{default-init} is a form that is
evaluated \emph{each time} its value
is to be used as the initial value of the slot.

If no \emph{default-init}
is specified, then the initial contents of the slot are undefined
and implementation-dependent.  The available slot-options are
described in section~\ref{Defstruct-Slot-Options}.

\begin{new}
X3J13 voted in January 1989
\issue{DEFSTRUCT-SLOTS-CONSTRAINTS-NAME}
to specify that it is an error for
two slots to have the same name; more precisely, no two slots may
have names for whose print names \cdf{string=} would be true.
Under this interpretation
\begin{lisp}
(defstruct lotsa-slots slot slot)
\end{lisp}
obviously is incorrect
but the following one is also in error, even assuming that the symbols
\cd{coin:slot} and \cd{blot:slot} really are distinct (non-\cdf{eql}) symbols:
\begin{lisp}
(defstruct no-dice coin:slot blot:slot)
\end{lisp}
To illustrate another case, the first \cdf{defstruct} form below is
correct, but the second one is in error.
\begin{lisp}
(defstruct one-slot slot) \\*
(defstruct (two-slots (:include one-slot)) slot)
\end{lisp}

\beforenoterule
\begin{rationale}
Print names are the criterion for slot-names being the same, rather
than the symbols themselves, because \cdf{defstruct} constructs names
of accessor functions from the print names and interns the resulting
new names in the current package.
\end{rationale}
\afternoterule

X3J13 recommended that expanding
a \cdf{defstruct} form violating this
restriction should signal an error and noted, with an eye to the Common Lisp
Object System
\issue{CLOS}, that the restriction applies only to the operation of the
\cdf{defstruct} macro as such and not to the \cdf{structure-class} or
structures defined with \cdf{defclass}.
\end{new}

\begin{newer}
X3J13 voted in March 1989 \issue{DEFINING-MACROS-NON-TOP-LEVEL}
to clarify that, while defining forms normally appear at top level,
it is meaningful to place them in non-top-level contexts;
\cdf{defstruct} must treat slot \emph{default-init} forms
and any\vadjust{\penalty-10000}
initialization forms within the specification of a by-position
constructor function as occurring
within the enclosing lexical environment, not within the global
environment.
\end{newer}

\cdf{defstruct} not only defines an access function for each slot, but also
arranges for \cdf{setf} to work properly on such access functions,
defines a predicate named \cd{\emph{name}-p},
defines a constructor function named \cd{make-\emph{name}},
and defines a copier function named \cd{copy-\emph{name}}.
All names of automatically created functions are interned
in whatever package is current at the time the \cdf{defstruct}
form is processed (see \cdf{*package*}).
Also, all such functions may be declared \cdf{inline}
at the discretion of the implementation to improve efficiency;
if you do not want some function declared \cdf{inline},
follow the \cdf{defstruct} form with a \cdf{notinline} declaration
to override any automatic \cdf{inline} declaration.

\begin{newer}
X3J13 voted in January 1989 \issue{DEFSTRUCT-REDEFINITION}
to specify that the results of redefining a \cdf{defstruct} structure
(that is, evaluating more than one \cdf{defstruct} structure
for the same name) are undefined.

The problem is that if instances have been created under the old definition
and then remain accessible after the new definition has been evaluated,
the accessors and other functions for the new definition may be incompatible
with the old instances.  Conversely, functions associated with the
old definition may have been declared \cdf{inline} and compiled
into code that remains accessible after the new definition has been
evaluated; such code may be incompatible with the new instances.

In practice this restriction affects the development
and debugging process rather than production runs of fully developed code.
The \cdf{defstruct} feature is intended to provide
``the most efficient'' structure class.
CLOS classes defined by \cdf{defclass}
allow much more flexible structures to be defined and redefined.

Programming environments are allowed and encouraged to permit \cdf{defstruct}
redefinition, perhaps with warning messages about possible interactions
with other parts of the programming environment or memory state.
It is beyond the scope of the Common Lisp
language standard to define those interactions except to note that they
are not portable.
\end{newer}
\end{defmac}

\section{Using the Automatically Defined Constructor Function}

After you have defined a new structure with \cdf{defstruct}, you can
create instances of this structure by using the constructor function.
By default, \cdf{defstruct} defines this function automatically.
For a structure named \cdf{foo}, the constructor function is normally
named \cdf{make-foo};
you can specify a different name
by giving it as the argument to the
\cd{:constructor} option, or specify that you don't
want a normal
constructor function at all by using {\false} as the argument
(in which case one or more ``by-position'' constructors should be
requested; see section~\ref{DEFSTRUCT-CONSTRUCTOR-SYNTAX}).

A call to a constructor function, in general, has the form
\begin{lisp}
(\emph{name-of-constructor-function} \\*
~~~~~~~~\emph{slot-keyword-1} \emph{form-1} \\*
~~~~~~~~\emph{slot-keyword-2} \emph{form-2} \\*
~~~~~~~~...)
\end{lisp}
All arguments are keyword arguments.  Each \emph{slot-keyword} should be a
keyword whose name matches the name of a slot of the structure
(\cdf{defstruct} determines the possible keywords simply by interning each
slot-name in the keyword package).  All the \emph{keywords} and \emph{forms}
are evaluated.  In short, it is just as if the constructor function
took all its arguments as \cd{\&key} parameters.  For example, the
\cdf{ship} structure shown in section~\ref{DEFSTRUCT-INTRO-SECTION}
has a constructor function that takes arguments roughly as if its definition
were
\begin{lisp}
(defun make-ship (\&key x-position y-position \\
~~~~~~~~~~~~~~~~~~~~~~~x-velocity y-velocity mass) \\
~~...)
\end{lisp}

\label{defstruct-initialization}
If \emph{slot-keyword-j} names a slot, then that element of
the created structure will be initialized to the value of \emph{form-j}.
If no pair \emph{slot-keyword-j} and \emph{form-j}
is present for a given slot, then the slot will be
initialized by evaluating the \emph{default-init} form specified
for that slot in the call to \cdf{defstruct}.
(In other words, the initialization specified in the \cdf{defstruct}
defers to any specified in a call to the constructor function.)
If the default initialization form is used, it is evaluated
at construction time, but
in the lexical environment of the \cdf{defstruct} form in which it appeared.
If the \cdf{defstruct} itself also did not
specify any initialization, the element's initial value is undefined.
You should always specify the initialization, either in the \cdf{defstruct}
or in the call to the constructor function,
if you care about the initial value of the slot.

Each initialization form specified for a \cdf{defstruct} component,
when used by the constructor function for an otherwise unspecified
component, is re-evaluated on every call to the
constructor function.  It is as if the initialization forms were
used as \emph{init} forms for the keyword parameters of the
constructor function.
For example, if the form \cd{(gensym)}
were used as an initialization form,
either in the constructor-function call or as the default initialization form
in the \cdf{defstruct} form,
then every call to the constructor
function would call \cdf{gensym} once to generate a new symbol.

\begin{newer}
X3J13 voted in October 1988 \issue{DEFSTRUCT-DEFAULT-VALUE-EVALUATION}
to clarify that the default value in a defstruct slot is not evaluated 
        unless it is needed in the creation of a particular structure
        instance.  If it is never needed, there can be no type-mismatch
        error, even if the type of the slot is specified, and no warning
        should be issued.


For example, in the following sequence only the last form is in error.
\begin{lisp}
(defstruct person (name .007 :type string)) \\*
\\*
(make-person :name "James") \\*
\\*
(make-person)~~~~~;\textrm{Error to give \cdf{name} the value \cd{.007}}
\end{lisp}
\end{newer}


\section{Defstruct Slot-Options}
\label{Defstruct-Slot-Options}

Each \emph{slot-description} in a \cdf{defstruct} form may specify one or more
slot-options.  A slot-option consists of a pair of a keyword and
a value (which is not a form to be evaluated, but the value itself).
For example:
\begin{lisp}
(defstruct ship \\
~~(x-position 0.0 \cd{:type} short-float) \\
~~(y-position 0.0 \cd{:type} short-float) \\
~~(x-velocity 0.0 \cd{:type} short-float) \\
~~(y-velocity 0.0 \cd{:type} short-float) \\
~~(mass *default-ship-mass* \cd{:type} short-float \cd{:read-only} t))
\end{lisp}
This specifies that each slot will always contain a
short-format floating-point number,
and that the last slot may not be altered once a ship is constructed.

The available slot-options are as follows.
\begin{flushdesc}
\item[\cd{:type}]
The option \cd{\cd{:type} \emph{type}} specifies that the contents of the
slot will always be of the specified data type.  This is entirely
analogous to the declaration of a variable or function; indeed, it
effectively declares the result type of the access function.  An
implementation may or may not choose to check the type of the new object
when initializing or assigning to a slot.
Note that the argument form \emph{type} is not evaluated;
it must be a valid type specifier.

\item[\cd{:read-only}]
The option \cd{\cd{:read-only} \emph{x}}, where \emph{x} is not {\false},
specifies that this slot may not be
altered; it will always contain the value specified at construction time.
\cdf{setf} will not accept the access function for this slot.
If \emph{x} is {\false}, this slot-option has no effect.
Note that the argument form \emph{x} is not evaluated.
\end{flushdesc}

Note that it is impossible to specify a slot-option unless
a default value is specified first.

\section{Defstruct Options}
\label{DEFSTRUCT-OPTIONS}
\label{Defstruct-Hairy-Stuff}

The preceding description of \cdf{defstruct} is all that the average
user will need (or want) to know in order to use structures.
The remainder of this chapter discusses more complex features of
the \cdf{defstruct} facility.

This section explains each of the options that can be given to \cdf{defstruct}.
A \cdf{defstruct} option may be either a keyword
or a list of a keyword and arguments for that keyword.
(Note that the syntax for \cdf{defstruct} options differs from
the pair syntax used for slot-options.  No part of any of these options
is evaluated.)

\begin{flushdesc}
\item[\cd{:conc-name}]
This provides for automatic prefixing of names of access functions.
It is conventional to begin the names of all the access functions of
a structure with a specific prefix,
the name of the structure followed by a hyphen.
This is the default behavior.

The argument to the \cd{:conc-name} option specifies an alternative
prefix to be used.  (If a hyphen is to be used as a separator,
it must be specified as part of the prefix.)
If {\false} is specified as an argument, then \emph{no} prefix is used;
then the names of the access functions
are the same as the slot-names, and it is up to the user
to name the slots reasonably.

Note that no matter what is specified for \cd{:conc-name},
with a constructor function one uses
slot keywords that match the slot-names, with no prefix attached.
On the other hand, one uses the access-function name
when using \cdf{setf}.  Here is an example:
\begin{lisp}
(defstruct door knob-color width material) \\
(setq my-door \\
~~~~~~(make-door :knob-color 'red :width 5.0)) \\
(door-width my-door) \EV\ 5.0 \\
(setf (door-width my-door) 43.7) \\
(door-width my-door) \EV\ 43.7 \\
(door-knob-color my-door) \EV\ red
\end{lisp}

\item[\cd{:constructor}]
This option takes one argument, a symbol,
which specifies the name of the constructor
function.  If the argument is not provided or if the option itself is not
provided, the name of the constructor is produced by concatenating the
string \cd{"MAKE-"} and the name of the structure, putting the name
in whatever package is current at the time the \cdf{defstruct}
form is processed (see \cdf{*package*}).
If the argument is
provided and is {\false}, no constructor function is defined.

This option actually has a more general syntax that is explained
in section~\ref{DEFSTRUCT-CONSTRUCTOR-SYNTAX}.

\item[\cd{:copier}]
This option takes one argument, a symbol,
which specifies the name of the copier
function.  If the argument is not provided or if the option itself is not
provided, the name of the copier is produced by concatenating the
string \cd{"COPY-"} and the name of the structure, putting the name
in whatever package is current at the time the \cdf{defstruct}
form is processed (see \cdf{*package*}).
If the argument is provided and is {\false}, no copier function is defined.

The automatically defined copier function simply makes a new structure
and transfers all components verbatim from the argument into the
newly created structure.  No attempt is made to make copies
of the components.  Corresponding components of the old and
new structures will therefore be \cdf{eql}.

\item[\cd{:predicate}]
This option takes one argument, which specifies the name of the type predicate.
If the argument is not provided or if the option itself is not
provided, the name of the predicate is made by concatenating the
name of the structure to the string \cd{"-P"}, putting the name
in whatever package is current at the time the \cdf{defstruct}
form is processed (see \cdf{*package*}).
If the argument is
provided and is {\false}, no predicate is defined.  A predicate can be defined
only if the structure is ``named'';
if the \cd{:type} option is specified
and the \cd{:named} option is
not specified, then the \cd{:predicate} option must either be unspecified
or have the value {\false}.

\item[\cd{:include}]
This option is used for building a new structure definition as
an extension of an old structure definition.  As an example,
suppose you have a structure called \cdf{person} that looks like this:
\begin{lisp}
(defstruct person name age sex)
\end{lisp}
Now suppose you want to make a new structure to represent an astronaut.
Since astronauts are people too, you would like them also to have the
attributes of name, age, and sex, and you would like Lisp functions
that operate on \cdf{person} structures to operate just as well on
\cdf{astronaut} structures.  You can do this by defining \cdf{astronaut}
with the \cd{:include} option, as follows:
\begin{lisp}
(defstruct (astronaut (:include person) \\
~~~~~~~~~~~~~~~~~~~~~~(:conc-name astro-)) \\
~~~helmet-size \\
~~~(favorite-beverage 'tang))
\end{lisp}

The \cd{:include} option causes the structure being defined
to have the same slots as the included structure.
This is done in such a way
that the access functions for the included
structure will also work on the structure being defined.
In this example, an
\cdf{astronaut} will therefore have five slots: the three defined in
\cdf{person} and the two defined in \cdf{astronaut}
itself.  The access functions defined by the \cdf{person} structure
can be applied to instances of the \cdf{astronaut} structure, and they
will work correctly.
Moreover, \cdf{astronaut} will have its own access functions for
components defined by the \cdf{person} structure.
The following examples illustrate how you can
use \cdf{astronaut} structures:
\begin{lisp}
(setq x (make-astronaut :name 'buzz \\
~~~~~~~~~~~~~~~~~~~~~~~~:age 45 \\
~~~~~~~~~~~~~~~~~~~~~~~~:sex t \\
~~~~~~~~~~~~~~~~~~~~~~~~:helmet-size 17.5)) \\
 \\
(person-name x) \EV\ buzz \\
(astro-name x) \EV\ buzz \\
\\
(astro-favorite-beverage x) \EV\ tang
\end{lisp}
The difference between the access functions \cdf{person-name} and \cdf{astro-name}
is that \cdf{person-name} may be correctly applied to any \cdf{person},
including an \cdf{astronaut}, while \cdf{astro-name} may be correctly
applied only to an \cdf{astronaut}.  (An implementation may or may not
check for incorrect use of access functions.)

At most one \cd{:include} option may be specified in a single
\cdf{defstruct} form.
The argument to the \cd{:include} option is required and must be the
name of some previously defined structure.  If the structure being
defined has no \cd{:type} option, then the included structure must
also have had no \cd{:type} option specified for it.
If the structure being defined has a \cd{:type} option,
then the included structure must have been declared with a \cd{:type}
option specifying the same representation type.

If no \cd{:type} option is involved, then
the structure name of the including structure definition
becomes the name of a data type, of course, and therefore
a valid type specifier recognizable by \cdf{typep}; moreover, it becomes
a subtype of the included structure.  In the above example,
\cdf{astronaut} is a subtype of \cdf{person}; hence
\begin{lisp}
(typep (make-astronaut) 'person)
\end{lisp}
is true, indicating that all operations on persons will also
work on astronauts.

The following is an advanced feature of the \cd{:include} option.
Sometimes, when one structure includes another, the default values or
slot-options for the slots that came from the included structure are not
what you want.  The new structure can specify default values or
slot-options for the included slots different from those the included
structure specifies, by giving the \cd{:include} option as
\begin{lisp}
(:include \emph{name} \emph{slot-description-1} \emph{slot-description-2} ...)
\end{lisp}
Each \emph{slot-description-j} must have a \emph{slot-name} or \emph{slot-keyword} that is the same
as that of some slot in the included structure.
If \emph{slot-description-j} has no \emph{default-init},
then in the new structure the slot will have no initial
value.  Otherwise its initial value form will be replaced by
the \emph{default-init} in \emph{slot-description-j}.
A normally writable slot may be made read-only.
If a slot is read-only in the included structure, then it
must also be so in the including structure.
If a type is specified for a slot, it must be the same as, or a subtype of, the
type specified in the included structure.  If it is a strict subtype,
the implementation may or may not choose to error-check assignments.

For example, if we had wanted to define \cdf{astronaut} so that the
default age for an astronaut is \cd{45}, then we could have said:
\begin{lisp}
(defstruct (astronaut (:include person (age 45))) \\
~~~helmet-size \\
~~~(favorite-beverage 'tang))
\end{lisp}

\begin{new}
X3J13 voted in June 1988
\issue{DATA-TYPES-HIERARCHY-UNDERSPECIFIED}
to require any structure type created by \cdf{defstruct}
(or \cdf{defclass}) to be disjoint from any of the types
\cdf{cons}, \cdf{symbol}, \cdf{array}, \cdf{number}, \cdf{character},
\cdf{hash-table}, \cdf{readtable}, \cdf{package}, \cdf{pathname},
\cdf{stream}, and \cdf{random-state}.  A consequence of this requirement
is that it is an error to specify any of these types, or any of their
subtypes, to the \cdf{defstruct} \cd{:include} option.
(The first edition said nothing explicitly about this.
Inasmuch as using such a type with the \cd{:include} option was
not defined to work, one might argue that such use was an error
in Common Lisp as defined by the first edition.)
\end{new}

\item[\cd{:print-function}]
This option may be used only if the \cd{:type}
option is not specified.
The argument to the \cd{:print-function} option
should be a function of three arguments,
in a form acceptable to the \cdf{function} special operator,
to be used to print structures of this type.
When a structure of this type is to be printed, the function
is called on three arguments:
the structure to be printed, a stream to print to,
and an integer indicating the current depth (to be compared against
\cdf{*print-level*}).
The printing function should observe the values of
such printer-control variables as \cdf{*print-escape*}
and \cdf{*print-pretty*}.

If the \cd{:print-function} option is not specified and the \cd{:type}
option also not specified, then a default printing function is
provided for the structure that will print out all its slots
using \cd{\#S} syntax (see section~\ref{SHARP-SIGN-MACRO-CHARACTER-SECTION}).

\begin{new}
X3J13 voted in January 1989
\issue{PRINT-CIRCLE-STRUCTURE}
to specify that user-defined printing functions for the \cdf{defstruct}
\cd{:print-function} option may print objects to the
supplied stream using \cdf{write}, \cd{print1}, \cdf{princ}, \cdf{format},
or \cdf{print-object} and expect circularities to be detected and printed
using \cd{\#\emph{n\/}\#} syntax (when \cdf{*print-circle*} is non-\cdf{nil}, of course).
See \cdf{*print-circle*}.
\end{new}


\begin{new}
X3J13 voted in January 1989
\issue{DEFSTRUCT-PRINT-FUNCTION-INHERITANCE}
to clarify that if the \cd{:print-function}
option is not specified but the \cd{:include} option \emph{is} specified,
then the print function is inherited from the included structure type.
Thus, for example, an \cdf{astronaut} will be printed by the same
printing function that is used for \cdf{person}.

X3J13 in the same vote extended the \cdf{print-function} option
as follows: If the \cdf{print-function} option is specified but with
no argument, then the standard default printing function (that uses
\cd{\#S} syntax) will be used.  This provides a means of overriding the
inheritance rule.  For example, if \cdf{person} and \cdf{astronaut}
had been defined as
\begin{lisp}
(defstruct (person \\*
~~~~~~~~~~~~~(:print-function~~~~~;\textrm{Special print function}\\*
~~~~~~~~~~~~~(lambda (p s k) \\*
~~~~~~~~~~~~~~~(format s "<{\Xtilde}A, age {\Xtilde}D>" \\*
~~~~~~~~~~~~~~~~~~~~~~~(person-name p) \\*
~~~~~~~~~~~~~~~~~~~~~~~(person-age p))))) \\*
~~name age sex) \\
\\
(defstruct (astronaut \\*
~~~~~~~~~~~~~(:include person) \\*
~~~~~~~~~~~~~(:conc-name astro-) \\*
~~~~~~~~~~~~~(:print-function))~~~~~;\textrm{Use default print function} \\*
~~~helmet-size \\*
~~~(favorite-beverage 'tang))
\end{lisp}
then an ordinary person would be printed as ``\cd{<Joe Schmoe, age 27>}''
but an astronaut would be printed as, for example,
\begin{lisp}
\#S(ASTRONAUT NAME BUZZ AGE 45 SEX T \\*
~~~HELMET-SIZE 17.5 FAVORITE-BEVERAGE TANG)
\end{lisp}
using the default \cd{\#S} syntax (yuk).

These changes make the behavior of \cdf{defstruct} with respect to the
\cd{:include} option a bit more like the behavior of classes in CLOS.
\end{new}

\item[\cd{:type}]
The \cd{:type} option explicitly specifies the representation to be used for
the structure.  It takes one argument, which must
be one of the types enumerated below.

Specifying this option has the effect of forcing
a specific representation and of forcing the components to be
stored in the order specified in the \cdf{defstruct} form
in corresponding successive elements of the specified representation.
It also \emph{prevents} the structure name from becoming a valid
type specifier recognizable by \cdf{typep}
(see section~\ref{EXPLICIT-TYPE-STRUCTURES}).

Normally this option is not specified, in which case the structure
is represented in an implementation-dependent manner.

\begin{quotation}    % Merely to advance the left margin
\begin{flushdesc}
\item[\cdf{vector}]
This produces the same result as specifying \cd{(vector t)}.
The structure is represented
as a general vector, storing components as vector elements.
The first component is vector
element 1 if the structure is \cd{:named}, and element 0 otherwise.

\item[\cd{(vector \emph{element-type})}]
The structure is represented as a (possibly specialized) vector, storing
components as vector elements.  Every component must be of a type that can be
stored in a vector of the type specified.  The first component is vector
element 1 if the structure is \cd{:named}, and element 0 otherwise.
The structure may be \cd{:named} only if the type \cdf{symbol} is a subtype of
the specified \cdf{element-type}.

\item[\cdf{list}]
The structure is represented as a list.
The first component is the \emph{cadr}
if the structure is \cd{:named}, and the \emph{car} if
it is \cd{:unnamed}.
\end{flushdesc}
\end{quotation}

\item[\cd{:named}]
The \cd{:named} option specifies that the structure is ``named''; this
option takes no argument.  If no \cd{:type} option is specified,
then the structure is always named; so this option is useful only in
conjunction with the \cd{:type} option.
See section~\ref{EXPLICIT-TYPE-STRUCTURES} for a further description of this
option.

\item[\cd{:initial-offset}]
This allows you to tell \cdf{defstruct} to skip over a certain
number of slots before it starts allocating the slots described in the
body.  This option requires an argument,
a non-negative integer,
which is the number of slots you want \cdf{defstruct} to skip.
The \cd{:initial-offset} option may be used only if the
\cd{:type} option is also specified.
See section~\ref{DEFSTRUCT-INITIAL-OFFSET} for a further description
of this option.
\end{flushdesc}

\section{By-Position Constructor Functions}
\label{DEFSTRUCT-CONSTRUCTOR-SYNTAX}

If the \cd{:constructor} option is given as
\cd{(\cd{:constructor} \emph{name} \emph{arglist})},
then instead of making a keyword-driven constructor function,
\cdf{defstruct} defines a ``positional'' constructor function,
taking arguments whose meaning is determined by the argument's position
rather than by a keyword.
The \emph{arglist} is used to describe what the arguments to the
constructor will be.  In the simplest case something like
\cd{(\cd{:constructor} make-foo (a b c))} defines \cdf{make-foo} to be
a three-argument constructor function whose arguments are used to initialize the
slots named \cdf{a}, \cdf{b}, and \cdf{c}.

In addition, the keywords \cd{\&optional}, \cd{\&rest}, and \cd{\&aux} are
recognized in the argument list.  They work in the way you might expect,
but there are a few fine points worthy of explanation.
Consider this example:
\begin{lisp}
(\cd{:constructor} create-foo \\
~~~~~~~~(a \&optional b (c 'sea) \&rest d \&aux e (f 'eff)))
\end{lisp}
This defines \cdf{create-foo} to be a constructor of one or more arguments.
The first argument is used to initialize the \cdf{a} slot.  The second
argument is used to initialize the \cdf{b} slot.  If there isn't any
second argument, then the default value given in the body of the
\cdf{defstruct} (if given) is used instead.  The third argument is used to
initialize the \cdf{c} slot.  If there isn't any third argument, then the
symbol \cdf{sea} is used instead.  Any arguments following the third
argument are collected into a list and used to initialize the \cdf{d}
slot.  If there are three or fewer arguments, then {\false} is placed in
the \cdf{d} slot.  The \cdf{e} slot \emph{is not initialized}; its initial
value is undefined.  Finally, the \cdf{f} slot is initialized to contain
the symbol \cdf{eff}.

The actions taken in the \cdf{b} and \cdf{e} cases were carefully
chosen to allow the user to specify all possible behaviors.  Note that
the \cd{\&aux} ``variables'' can be used to completely override the default
initializations given in the body.

With this definition, one can write
\begin{lisp}
(create-foo 1 2)
\end{lisp}
instead of
\begin{lisp}
(make-foo \cd{:a} 1 \cd{:b} 2)
\end{lisp}
and of course \cdf{create-foo} provides defaulting different
from that of \cdf{make-foo}.

It is permissible to use the
\cd{:constructor} option more than once, so that you can define several
different constructor functions, each taking different parameters.

Because a constructor of this type operates By Order of Arguments,
it is sometimes known as a BOA constructor.

\begin{new}
X3J13 voted in January 1989
\issue{DEFSTRUCT-CONSTRUCTOR-KEY-MIXTURE}
to allow \cd{\&key} and \cd{\&allow-other-keys}
in the
parameter list of a ``positional'' constructor.  The initialization of slots
corresponding to keyword parameters
is performed in the same manner as for \cd{\&optional} parameters.
A variant of the example shown above illustrates this:
\begin{lisp}
(\cd{:constructor} create-foo \\*
~~~~~~~~(a \&optional b (c 'sea) \\*
~~~~~~~~~\&key p (q 'cue) ((:why y)) ((:you u) 'ewe) \\*
~~~~~~~~~\&aux e (f 'eff)))
\end{lisp}
The treatment of slots \cdf{a}, \cdf{b}, \cdf{c}, \cdf{e}, and \cdf{f}
is the same as in the original example.  In addition,
if there is a \cd{:p} keyword argument, it is
used to initialize the \cdf{p} slot;  if there isn't any
\cd{:p} keyword argument, then the default value given in the body of the
\cdf{defstruct} (if given) is used instead.  Similarly,
if there is a \cd{:q} keyword argument, it is
used to initialize the \cdf{q} slot;  if there isn't any
\cd{:q} keyword argument, then
the symbol \cdf{cue} is used instead.

In order thoroughly to flog this presumably already dead horse,
we further observe that if there is a \cd{:why} keyword argument, it is
used to initialize the \cdf{y} slot; otherwise
the default value for slot \cdf{y} is used instead.  Similarly,
if there is a \cd{:you} keyword argument, it is
used to initialize the \cdf{u} slot;  otherwise
the symbol \cdf{ewe} is used instead.

If memory serves me correctly, \cdf{defstruct} was included in the original
design for Common Lisp some time before keyword arguments were approved.
The failure of positional constructors to accept keyword arguments may well
have been an oversight on my part; there is no logical reason to exclude
them.  I am grateful to X3J13 for rectifying this.

A remaining difficulty is that the possibility of keyword arguments
renders the term ``positional constructor'' a misnomer.  Worse yet,
it ruins the term ``BOA constructor.''  I suggest that
they continue to be called BOA constructors, as I refuse to abandon
a good pun.  (I regret appearing to have more compassion for puns than
for horses.)

As part of the same vote X3J13 also changed \cdf{defstruct}
to allow BOA constructors to have
parameters (including supplied-p parameters)
that do not correspond to any
slot.  Such parameters may be used in subsequent initialization forms in the
parameter list. Consider this example:
\begin{lisp}
(defstruct (ice-cream-factory \\*
~~~~~~~~~~~~~(:constructor fabricate-factory \\*
~~~~~~~~~~~~~~~(\&key (capacity 5) \\*
~~~~~~~~~~~~~~~~~~~~~~location \\
~~~~~~~~~~~~~~~~~~~~~~(local-flavors \\*
~~~~~~~~~~~~~~~~~~~~~~~~(case location \\*
~~~~~~~~~~~~~~~~~~~~~~~~~~((hawaii) '(pineapple macadamia guava)) \\*
~~~~~~~~~~~~~~~~~~~~~~~~~~((massachusetts) '(lobster baked-bean)) \\*
~~~~~~~~~~~~~~~~~~~~~~~~~~((california) '(ginger lotus avocado \\*
~~~~~~~~~~~~~~~~~~~~~~~~~~~~~~~~~~~~~~~~~~bean-sprout garlic)) \\*
~~~~~~~~~~~~~~~~~~~~~~~~~~((texas) '(jalapeno barbecue)))) \\
~~~~~~~~~~~~~~~~~~~~~~(flavors (subseq (append local-flavors \\*
~~~~~~~~~~~~~~~~~~~~~~~~~~~~~~~~~~~~~~~~~~~~~~~'(vanilla \\*
~~~~~~~~~~~~~~~~~~~~~~~~~~~~~~~~~~~~~~~~~~~~~~~~~chocolate \\*
~~~~~~~~~~~~~~~~~~~~~~~~~~~~~~~~~~~~~~~~~~~~~~~~~strawberry \\
~~~~~~~~~~~~~~~~~~~~~~~~~~~~~~~~~~~~~~~~~~~~~~~~~pistachio \\
~~~~~~~~~~~~~~~~~~~~~~~~~~~~~~~~~~~~~~~~~~~~~~~~~maple-walnut \\*
~~~~~~~~~~~~~~~~~~~~~~~~~~~~~~~~~~~~~~~~~~~~~~~~~peppermint)) \\*
~~~~~~~~~~~~~~~~~~~~~~~~~~~~~~~~~~~~~~~0 capacity))))) \\*
~~(capacity 3) \\*
~~(flavors '(vanilla chocolate strawberry mango)))
\end{lisp}

The structure type \cdf{ice-cream-factory} has two constructors.
The standard constructor, \cdf{make-ice-cream-factory},
takes two keyword arguments named \cd{:capacity} and \cd{:flavors}.
For this constructor, the default for the \cdf{capacity} slot is \cd{3}
and the default list of \cdf{flavors} is America's favorite threesome
and a dark horse (not a dead one).
The BOA constructor \cdf{fabricate-factory}
accepts four different keyword arguments.  The \cd{:capacity}
argument defaults to \cd{5}, and the \cd{:flavors} argument
defaults in a complicated manner based on the other three.
The \cd{:local-flavors} argument may be specified directly,
or may be allowed to default based on the \cd{:location} of the factory.
Here are examples of various factories:

\vskip0pt plus 2pt%manual
\begin{lisp}
(setq houston (fabricate-factory :capacity 4 :location 'texas)) \\*
(setq cambridge (fabricate-factory :location 'massachusetts)) \\
(setq seattle (fabricate-factory :local-flavors '(salmon))) \\
(setq wheaton (fabricate-factory :capacity 4 :location 'illinois)) \\*
(setq pittsburgh (fabricate-factory :capacity 4)) \\*
(setq cleveland (make-factory :capacity 4)) \\
 \\
(ice-cream-factory-flavors houston) \\*
~\EV~(jalapeno barbecue vanilla chocolate)
\end{lisp}
\newpage%manual
\begin{lisp}
(ice-cream-factory-flavors cambridge) \\*
~\EV~(lobster baked-bean vanilla chocolate strawberry) \\
\\
(ice-cream-factory-flavors seattle) \\*
~\EV~(salmon vanilla chocolate strawberry pistachio) \\
\\
(ice-cream-factory-flavors wheaton) \\*
~\EV~(vanilla chocolate strawberry pistachio) \\
\\
(ice-cream-factory-flavors pittsburgh) \\*
~\EV~(vanilla chocolate strawberry pistachio) \\
\\
(ice-cream-factory-flavors cleveland) \\*
~\EV~(vanilla chocolate strawberry mango)
\end{lisp}
\end{new}


\section{Structures of Explicitly Specified Representational Type}
\label{EXPLICIT-TYPE-STRUCTURES}

Sometimes it is important to have explicit control
over the representation of a structure.  The \cd{:type}
option allows one to specify that a structure must be implemented
in a particular way, using a list or a specific kind of vector,
and to specify the exact allocation of structure slots to
components of the representation.
A structure may also be ``unnamed'' or ``named,'' according to whether
the structure name is stored in (and thus recoverable from) the structure.

\subsection{Unnamed Structures}

Sometimes a particular data representation is imposed by external requirements,
and yet it is desirable to document the data format as a \cdf{defstruct}-style
structure.  For example, consider expressions built up from numbers,
symbols, and binary operations such as \cdf{+} and \cdf{*}.  An operation
might be represented as it is in Lisp, as a list of the operator
and the two operands.  This fact can be expressed succinctly with \cdf{defstruct}
in this manner:
е\begin{lisp}
(defstruct (binop (:type list)) \\
~~(operator '? :type symbol) \\
~~operand-1 \\
~~operand-2)
\end{lisp}
This will define a constructor function \cdf{make-binop} and three
selector functions, namely \cdf{binop-operator}, \cd{binop-operand-1},
and \cd{binop-operand-2}.  (It will \emph{not}, however, define a predicate
\cdf{binop-p}, for reasons explained below.)

The effect of \cdf{make-binop} is simply to construct a list of length 3:
\begin{lisp}
(make-binop :operator '+ :operand-1 'x :operand-2 5) \\*
~~~\EV\ (+ x 5) \\
\\
(make-binop :operand-2 4 :operator '*) \\*
~~~\EV\ (* {\nil} 4)
\end{lisp}
It is just like the function \cdf{list} except that it takes
keyword arguments and performs slot defaulting appropriate to the \cdf{binop}
conceptual data type.  Similarly, the selector functions
\cdf{binop-operator}, \cd{binop-operand-1},
and \cd{binop-operand-2} are essentially equivalent to \cdf{car},
\cdf{cadr}, and \cdf{caddr}, respectively.  (They might not be
completely equivalent because,
for example, an implementation would be justified in adding error-checking
code to ensure that the argument to each selector function is a length-3
list.)

We speak of \cdf{binop} as being a ``conceptual'' data type because \cdf{binop}
is not made a part of the Common Lisp type system.  The predicate
\cdf{typep} will not recognize \cdf{binop} as a type specifier, and \cdf{type-of}
will return \cdf{list} when given a \cdf{binop} structure.  Indeed, there is
no way to distinguish a data structure constructed by \cdf{make-binop}
from any other list that happens to have the correct structure.

There is not even any way to recover the structure name \cdf{binop} from
a structure created by \cdf{make-binop}.  This can be done, however,
if the structure is ``named.''

\subsection{Named Structures}

A ``named'' structure has the property that, given an instance of the
structure, the structure name (that names the type) can be reliably
recovered.  For structures defined
with no \cd{:type} option, the structure name actually becomes part
of the Common Lisp data-type system.  The function \cdf{type-of},
when applied to such a structure, will return the structure name
as the type of the object; the predicate \cdf{typep} will recognize
the structure name as a valid type specifier.

For structures defined with a \cd{:type} option, \cdf{type-of} will
return a type specifier such as \cdf{list} or \cd{(vector t)},
depending on the type specified to the \cd{:type} option.
The structure name does not become a valid type specifier.
However,
if the \cd{:named} option is also specified, then the first component
of the structure (as created by a \cdf{defstruct} constructor function)
will always contain the structure name.  This allows the structure name
to be recovered from an instance of the structure and allows a reasonable
predicate for the conceptual type to be defined:
the automatically defined
\cd{\emph{name}-p} predicate for the structure operates by first
checking that its argument is of the proper type (\cdf{list}, \cd{(vector t)},
or whatever) and then checking whether the first component contains
the appropriate type name.

Consider the \cdf{binop} example shown above, modified only to
include the \cd{:named} option:
\begin{lisp}
(defstruct (binop (:type list) :named) \\
~~(operator '? :type symbol) \\
~~operand-1 \\
~~operand-2)
\end{lisp}
As before, this will define a constructor function \cdf{make-binop} and three
selector functions \cdf{binop-operator}, \cd{binop-operand-1},
and \cd{binop-operand-2}.  It will also define a predicate \cdf{binop-p}.

The effect of \cdf{make-binop} is now to construct a list of length 4:
\begin{lisp}
(make-binop :operator '+ :operand-1 'x :operand-2 5) \\*
~~~\EV\ (binop + x 5) \\
\\
(make-binop :operand-2 4 :operator '*) \\*
~~~\EV\ (binop * {\nil} 4)
\end{lisp}
The structure has the same layout as before except that the structure name
\cdf{binop} is included as the first list element.
The selector functions
\cdf{binop-operator}, \cd{binop-operand-1},
and \cd{binop-operand-2} are essentially equivalent to \cdf{cadr},
\cdf{caddr}, and \cdf{cadddr}, respectively.
The predicate \cdf{binop-p} is more or less equivalent to the following
definition.
\begin{lisp}
(defun binop-p (x) \\
~~(and (consp x) (eq (car x) 'binop)))
\end{lisp}
The name \cdf{binop} is still not a valid type specifier recognizable
to \cdf{typep}, but at least there is a way of distinguishing \cdf{binop}
structures from other similarly defined structures.

\subsection{Other Aspects of Explicitly Specified Structures}
\label{DEFSTRUCT-INITIAL-OFFSET}

The \cd{:initial-offset} option allows one
to specify that slots be allocated beginning at a representational
element other than the first.  For example, the form
\begin{lisp}
(defstruct (binop (:type list) (:initial-offset 2)) \\*
~~(operator '? :type symbol) \\*
~~operand-1 \\*
~~operand-2)
\end{lisp}
would result in the following behavior for \cdf{make-binop}:
\begin{lisp}
(make-binop :operator '+ :operand-1 'x :operand-2 5) \\*
~~~\EV\ (nil nil + x 5) \\
\\
(make-binop :operand-2 4 :operator '*) \\*
~~~\EV\ (nil nil * {\nil} 4)
\end{lisp}
The selectors
\cdf{binop-operator}, \cd{binop-operand-1},
and \cd{binop-operand-2} would be essentially equivalent to \cdf{caddr},
\cdf{cadddr}, and \cdf{car} of \cdf{cddddr}, respectively.
Similarly, the form
\begin{lisp}
(defstruct (binop (:type list) :named (:initial-offset 2)) \\*
~~(operator '? :type symbol) \\*
~~operand-1 \\*
~~operand-2)
\end{lisp}
would result in the following behavior for \cdf{make-binop}:
\begin{lisp}
(make-binop :operator '+ :operand-1 'x :operand-2 5) \\*
~~~\EV\ (nil nil binop + x 5) \\
\\
(make-binop :operand-2 4 :operator '*) \\*
~~~\EV\ (nil nil binop * {\nil} 4)
\end{lisp}

If the \cd{:include} is used with the \cd{:type}
option, then the effect is first to skip over as many representation
elements as needed to represent the included structure, then to
skip over any additional elements specified by the \cd{:initial-offset}
option, and then to begin allocation of elements from that point.
For example:
\begin{lisp}
(defstruct (binop (:type list) :named (:initial-offset 2)) \\
~~(operator '? :type symbol) \\
~~operand-1 \\
~~operand-2) \\
 \\
(defstruct (annotated-binop (:type list) \\
~~~~~~~~~~~~~~~~~~~~~~~~~~~~(:initial-offset 3) \\
~~~~~~~~~~~~~~~~~~~~~~~~~~~~(:include binop)) \\
~~commutative associative identity) \\
 \\
(make-annotated-binop :operator '* \\
~~~~~~~~~~~~~~~~~~~~~~:operand-1 'x \\
~~~~~~~~~~~~~~~~~~~~~~:operand-2 5 \\
~~~~~~~~~~~~~~~~~~~~~~:commutative t \\
~~~~~~~~~~~~~~~~~~~~~~:associative t \\
~~~~~~~~~~~~~~~~~~~~~~:identity 1) \\
~~~\EV\ (nil nil binop * x 5 nil nil nil t t 1)
\end{lisp}
The first two {\nil} elements stem from the \cd{:initial-offset} of \cd{2}
in the definition of \cdf{binop}.  The next four elements contain the
structure name and three slots for \cdf{binop}.  The next three {\nil} elements
stem from the \cd{:initial-offset} of \cd{3} in the definition of
\cdf{annotated-binop}.  The last three list elements contain the additional
slots for an \cdf{annotated-binop}.

%RUSSIAN
\else

\chapter{Структуры}

Common Lisp предоставляет функциональность для создания структур (почти таких
же, как в других языках, с именем структуры, полями, и т.д.). Фактически,
пользователь может определить новый тип данных.Каждая структура данных
этого типа имеет компоненты с заданными именами.
При создании структуры автоматически создаются конструктор и конструкции доступа
и присваивания значений для полей.

Данная глава разделена на две части. Первая часть описывает основную
функциональность структур, которая очень проста и позволяет пользователю
воспользоваться проверкой типов, модульностью и удобством определённых им типов
данных. Вторая часть начинается с раздела~\ref{Defstruct-Hairy-Stuff},
описывающего специализированные возможности для сложных приложений. Эти
возможности совершенно необязательны к использованию, и вам даже не нужно о них
знать для того, чтобы делать хорошие программы.

\section{Введение в структуры}
\label{DEFSTRUCT-INTRO-SECTION}

Функциональность структур воплощена в макросе \cdf{defstruct}, который позволяет
пользователю создавать и использовать сгруппированные типы данных с именованными
элементами. Она похожа на функциональность <<структур (structures)>> в {PL/I}
или <<записей (records)>> в Pascal'е.

В качестве примера, предположим, что вы пишете программу на Lisp'е, которая
управляет космическими кораблями в двухмерном пространстве.
В вашей программе, вам необходимо представить космический корабль как некоторого
вида Lisp'овый объект. Интересующие вашу программу свойства корабля являются: его
позиция (представленная как \emph{x} и \emph{y} координаты), скорость
(представленная как отрезки по осям \emph{x} и \emph{y}) и масса.

Таким образом, корабль может быть представлен как запись структуры с пятью
компонентами: позиция-\emph{x}, позиция-\emph{y}, скорость-\emph{x},
скорость-\emph{y} и масса.
Эта структура может быть реализована как Lisp'овый объект несколькими способами.
Она может быть списком из пяти элементов; позиция-\emph{x} будет \emph{car}
элементом, позиция-\emph{y} будет \emph{cadr}, и так далее. Подобным образом
структура может быть вектором из пяти элементов: позиция-\emph{x} будет 0-ым
элементом, позиция-\emph{y} будет 1-ым, и так далее. Проблема данных
представлений состоит в том, что компоненты занимают совершенно случайные
позиции и их сложно запомнить. Кто-нибудь увидев где-то в коде
строки \cd{(cadddr~ship1)} или \cd{(aref~ship1~3)}, обнаружит сложность в
определении того, что это производится доступ к компоненту скорости-\emph{y}
структуры \cd{ship1}. Более того, если представление корабля должно быть
изменено, то будет очень сложно найти все места в коде для изменения в
соответствие с новым представлением (не все появления \cdf{cadddr} означают
доступ к скорости-\emph{y} корабля). 

Лучше было бы если бы записи структур имели имена. Можно было бы написать что-то
вроде \cd{(ship-y-velocity ship1)} вместо \cd{(cadddr ship1)}. Кроме того было
бы неплохо иметь более информативную запись для создания структур, чем эта:
\begin{lisp}
(list 0 0 0 0 0)
\end{lisp}
Несомненно, хочется, чтобы \cdf{ship} был новым типом данных, как и любой другой
тип Lisp'овых данных, чтобы, например, осуществить проверку с помощью
\cdf{typep}.
Функциональность \cdf{defstruct} предоставляет все, что выше было необходимо.

\cdf{defstruct} является макросом, который определяет структуру. Например, для
космического корабля, можно определить структуру так:
\begin{lisp}
(defstruct ship \\
~~x-position \\
~~y-position \\
~~x-velocity \\
~~y-velocity \\
~~mass)
\end{lisp}
Запись декларирует, что каждый объект \cdf{ship} является объектом с пятью
именованными компонентами. Вычисление этой формы делает несколько вещей.

\begin{itemize}

\item 
Она определяет \cdf{ship-x-position} как функцию одного аргумента, а именно,
корабля, которая возвращает позицию-\emph{x} корабля.
\cdf{ship-y-position} и другие компоненты получают такие же определения функций.
Эти функции называются \emph{функциями доступа}, так как используются для
доступа к компонентам структуры.

\item 
Символ \cdf{ship} становится именем типа данных, к которому принадлежат
экземпляры объектом кораблей. Например, это имя может использоваться в
\cdf{typep}.
\cd{(typep x 'ship)} истинен, если \cd{x} является кораблём, и ложен, если
\cdf{x} является любым другим объектом.

\item 
Определяется функция одного аргумента с именем \cdf{ship-p}. Она является
предикатом, который истинен, если аргумент является кораблём, и ложен в
противных случаях. 

\item 
Определяется функция с именем \cdf{make-ship}, при вызове которой создаётся
структура данных из пяти компонентов, готовая к использованию в \emph{функциях
  доступа}.
Так, выполнение
\begin{lisp}
(setq ship2 (make-ship))
\end{lisp}
устанавливает в ship2 свеже созданный объект \cdf{ship}.
Можно указать первоначальные значения для компонентов структуры используя
именованные параметры:
\begin{lisp}
(setq ship2 (make-ship \cd{:mass} *default-ship-mass* \\
~~~~~~~~~~~~~~~~~~~~~~~\cd{:x-position} 0 \\
~~~~~~~~~~~~~~~~~~~~~~~\cd{:y-position} 0))
\end{lisp}
Форма создаёт новый корабль и инициализирует три его компонента.
Эта функция называется \emph{функцией-конструктором}, потому что создаёт новую
структуру.

\item
Синтаксис \cd{\#S} может использоваться для чтения экземпляров структур
\cdf{ship}, и также предоставляется функция вывода для печати структур кораблей.
Например, значение ранее упомянутой переменной \cd{ship2} может быть выведено
как
\begin{lisp}
\#S(ship  x-position 0  y-position 0  x-velocity nil \\
~~~~~~~~~y-velocity nil  mass 170000.0)
\end{lisp}

\item 
Определяется функция одного аргумента с именем \cdf{copy-ship}, которая получая
объект \cdf{ship}, будет создавать новый объект \cdf{ship}, который является
копией исходного.
Эта функция называется \emph{функцией копирования}.

\item 
Можно использовать \cdf{setf} для изменения компонентов объекта \cdf{ship}:
\begin{lisp}
(setf (ship-x-position ship2) 100)
\end{lisp}
Данная запись изменяет позицию-\emph{x} переменной \emph{ship2} в \cd{100}.
Она работает, потому что \cdf{defstruct} ведёт себя так, будто создаёт
соответствующие \cdf{defsetf} формы для \emph{функций доступа}.
\end{itemize}

Этот простой пример отображает мощь и удобство \cdf{defstruct} для представления
записей структур.
\cdf{defstruct} имеет много других возможностей для специализированных целей.

\section{Как использовать defstruct}

Все структуры определяются с помощью конструкции \cdf{defstruct}.
Вызов \cdf{defstruct} определяет новый тип данных, экземпляры которого содержат
именованные слоты.

\begin{defmac}
defstruct name-and-options [doc-string] {slot-description}+

Этот макрос определяет структурный тип данных.
Обычно вызов \cdf{defstruct} выглядит как следующий пример.
\begin{lisp}
(defstruct (\emph{name} \emph{option-1} \emph{option-2} ... \emph{option-m}) \\
~~~~~~~~~~~\emph{doc-string} \\
~~~~~~~~~~~\emph{slot-description-1} \\
~~~~~~~~~~~\emph{slot-description-2} \\
~~~~~~~~~~~... \\
~~~~~~~~~~~\emph{slot-description-n}) \\
\end{lisp}
Имя структуры \emph{name} должно быть символом. Он становиться именем нового типа данных,
который включает в себя все экземпляры данной структуры.
Соответственно функция \cdf{typep} будет принимать и использовать это
имя. Имя структуры \emph{name} возвращается как значение формы \emph{defstruct}.

\begin{new}\noindent
X3J13 voted in June 1988
\issue{DEFSTRUCT-SLOTS-CONSTRAINTS-NUMBER}
to allow a \cdf{defstruct} definition
to have no \emph{slot-description} at all; in other words, the
occurrence of \Mplus{\emph{slot-description}} in the preceding
header line would be replaced by \Mstar{\emph{slot-description}}.

Such structure definitions are particularly useful if the
\cd{:include} option is used, perhaps with other options; for example,
one can have two structures that are exactly alike except that they
print differently (having different \cd{:print-function} options).

Implementors are encouraged to permit this simple extension as soon as
convenient.  Users, however, may wish to maximize portability of their
code by avoiding the use of this extension unless and until it is
adopted as part of the ANSI standard.
\end{new}

Обычно параметры не нужны. Если они не заданы, то после слова \cdf{defstruct} вместо
\cd{(\emph{name})} можно записать просто \emph{name} . Синтаксис параметров и их
смысл раскрываются в разделе~\ref{DEFSTRUCT-OPTIONS}.

Если присутствует необязательная строка документации \emph{doc-string}, тогда
она присоединяется к символу \emph{name}, как документация для типа
\cdf{structure}. Смотрите \cdf{documentation}.

Каждое описание слота \emph{slot-description-j} выглядит так:
\begin{lisp}
(\emph{slot-name} \emph{default-init} \\
~~~~~\emph{slot-option-name-1} \emph{slot-option-value-1} \\
~~~~~\emph{slot-option-name-2} \emph{slot-option-value-2} \\
~~~~~... \\
~~~~~\emph{slot-option-name-k${}_{j}$} \emph{slot-option-value-k${}_{j}$})
\end{lisp}
Каждое имя слота \emph{slot-name} должно быть символом. Для каждого слота
определяется функция доступа. Если параметры и первоначальные значения не указаны,
тогда в качестве описания слота вместо \cd{(slot-name)} можно записать просто
\emph{slot-name}.

Форма \emph{default-init} вычисляется, только если соответствующий аргумент не
указан при вызове функций-конструкторов.
\emph{default-init} вычисляется \emph{каждый раз}, когда необходимо получить
первоначальное значение для слота.

Если \emph{default-init} не указано, тогда первоначальное значение слота не
определено и зависит от реализации. Доступные параметры для слотов разобраны в
разделе~\ref{Defstruct-Slot-Options}.

\begin{new}
X3J13 voted in January 1989
\issue{DEFSTRUCT-SLOTS-CONSTRAINTS-NAME}
to specify that it is an error for
two slots to have the same name; more precisely, no two slots may
have names for whose print names \cdf{string=} would be true.
Under this interpretation
\begin{lisp}
(defstruct lotsa-slots slot slot)
\end{lisp}
obviously is incorrect
but the following one is also in error, even assuming that the symbols
\cd{coin:slot} and \cd{blot:slot} really are distinct (non-\cdf{eql}) symbols:
\begin{lisp}
(defstruct no-dice coin:slot blot:slot)
\end{lisp}
To illustrate another case, the first \cdf{defstruct} form below is
correct, but the second one is in error.
\begin{lisp}
(defstruct one-slot slot) \\*
(defstruct (two-slots (:include one-slot)) slot)
\end{lisp}

\beforenoterule
\begin{rationale}
Print names are the criterion for slot-names being the same, rather
than the symbols themselves, because \cdf{defstruct} constructs names
of accessor functions from the print names and interns the resulting
new names in the current package.
\end{rationale}
\afternoterule

X3J13 recommended that expanding
a \cdf{defstruct} form violating this
restriction should signal an error and noted, with an eye to the Common Lisp
Object System
\issue{CLOS}, that the restriction applies only to the operation of the
\cdf{defstruct} macro as such and not to the \cdf{structure-class} or
structures defined with \cdf{defclass}.
\end{new}

\begin{newer}
X3J13 voted in March 1989 \issue{DEFINING-MACROS-NON-TOP-LEVEL}
to clarify that, while defining forms normally appear at top level,
it is meaningful to place them in non-top-level contexts;
\cdf{defstruct} must treat slot \emph{default-init} forms
and any\vadjust{\penalty-10000}
initialization forms within the specification of a by-position
constructor function as occurring
within the enclosing lexical environment, not within the global
environment.
\end{newer}

\cdf{defstruct} не только определяет функции доступа к слотам, но также
выполняет интеграцию с \cdf{setf} для этих функций, определяет предикат с
именем \cd{\emph{name}-p}, определяет функцию-конструктор с именем
\cd{make-\emph{name}} и определяет функцию копирования с именем
\cd{copy-\emph{name}}.
Все имена автоматически создаваемых функций интернируются в текущий пакет на время
выполнения \cdf{defstruct} (смотрите \cdf{*package*}).
Кроме того, все эти функции могут быть задекларированы как \cdf{inline} на
усмотрение реализации для повышения производительности.
Если вы не хотите, чтобы функции были задекларированы как \cdf{inline}, укажите
декларацию \cdf{notinline} после формы \cdf{defstruct} для перезаписи декларации
\cdf{inline}.

\begin{newer}
X3J13 voted in January 1989 \issue{DEFSTRUCT-REDEFINITION}
to specify that the results of redefining a \cdf{defstruct} structure
(that is, evaluating more than one \cdf{defstruct} structure
for the same name) are undefined.

The problem is that if instances have been created under the old definition
and then remain accessible after the new definition has been evaluated,
the accessors and other functions for the new definition may be incompatible
with the old instances.  Conversely, functions associated with the
old definition may have been declared \cdf{inline} and compiled
into code that remains accessible after the new definition has been
evaluated; such code may be incompatible with the new instances.

In practice this restriction affects the development
and debugging process rather than production runs of fully developed code.
The \cdf{defstruct} feature is intended to provide
``the most efficient'' structure class.
CLOS classes defined by \cdf{defclass}
allow much more flexible structures to be defined and redefined.

Programming environments are allowed and encouraged to permit \cdf{defstruct}
redefinition, perhaps with warning messages about possible interactions
with other parts of the programming environment or memory state.
It is beyond the scope of the Common Lisp
language standard to define those interactions except to note that they
are not portable.
\end{newer}
\end{defmac}

\section{Использование автоматически определяемого конструктора}

После того, как вы определили новую структуру с помощью \cdf{defstruct}, вы
можете создавать экземпляры данной структуры с помощью функции-конструктора.
По-умолчанию \cdf{defstruct} автоматически определяет эту функцию.
Для структуры с именем \cdf{foo}, функция-конструктор обычно называется
\cdf{make-foo}.
Вы можете указать другое имя, передав его в качестве аргумента для параметра
\cd{:constructor}, или, если вы вообще не хотите обычную функцию-конструктор, указав в качества аргумента {\false}. 
В последнем случае должны быть запрошены один или более конструкторов
с <<позиционными>> аргументами FIXME, смотрите раздел~\ref{DEFSTRUCT-CONSTRUCTOR-SYNTAX}.

Общая форма вызова функции-конструктора выглядит так:
\begin{lisp}
(\emph{name-of-constructor-function} \\*
~~~~~~~~\emph{slot-keyword-1} \emph{form-1} \\*
~~~~~~~~\emph{slot-keyword-2} \emph{form-2} \\*
~~~~~~~~...)
\end{lisp}
Все аргументы являются именованными. Каждый \emph{slot-keyword} должен быть
ключевым символом, имя которого совпадает со именем слота структуры
(\cdf{defstruct} определяет возможные ключевые символы интернируя каждое имя
слота в пакет ключевых символов (keyword). Все ключевые символы \emph{keywords}
и формы \emph{forms} вычисляются. В целом, это выглядит как если функция-конструктор принимает все аргументы как \cd{\&key} параметры. Например,
структура \cd{ship} упомянутая ранее в разделе~\ref{DEFSTRUCT-INTRO-SECTION}
имеет функцию-конструктор, которая принимает аргументы в соотвествие со
следующим определением:
\begin{lisp}
(defun make-ship (\&key x-position y-position \\
~~~~~~~~~~~~~~~~~~~~~~~x-velocity y-velocity mass) \\
~~...)
\end{lisp}

\label{defstruct-initialization}
Если \emph{slot-keyword-j} задаёт имя слота, тогда элемент созданной структуры
будет инициализирован значением \emph{form-j}.
Если пара \emph{slot-keyword-j} и \emph{form-j} для слота не указана, тогда слот
будет инициализирован результатом вычисления указанной в вызове \cdf{defstruct}
для этого слота формой \emph{default-init}.
(Другими словами, инициализация указанная в \cdf{defstruct} замещается
инициализацией указанной в вызове функции-конструктора.)
Если используется форма инициализации \emph{default-init}, она вычисляется в
время создания экземпляра структуры, но в лексическом окружении формы
\cdf{defstruct}, в которой она используется.
Если эта форма инициализации не указана, первоначальное значение слота не
определено.
Если вам необходимо инициализировать слоты некоторыми значениями, вы должны
всегда указывать первоначальное значение или в \cdf{defstruct}, или в 
вызове функции-конструктора. 

Каждая форма инициализации, указанная в компоненте формы \cdf{defstruct}, когда
используется функцией-конструктором, перевычисляется при каждом вызове
функции. Это, как если бы формы инициализации использовались как формы
\emph{init} для именованных параметров функции-конструктора.
Например, если форма \cd{(gensym)} используется как форма инициализации или в
вызове функции-конструктора, или форме инициализации слота в форме
\cdf{defstruct}, тогда каждый вызов функции-конструктора вызывал бы \cdf{gensym}
для создания нового символа.

\begin{newer}
X3J13 voted in October 1988 \issue{DEFSTRUCT-DEFAULT-VALUE-EVALUATION}
to clarify that the default value in a defstruct slot is not evaluated 
        unless it is needed in the creation of a particular structure
        instance.  If it is never needed, there can be no type-mismatch
        error, even if the type of the slot is specified, and no warning
        should be issued.


For example, in the following sequence only the last form is in error.
\begin{lisp}
(defstruct person (name .007 :type string)) \\*
\\*
(make-person :name "James") \\*
\\*
(make-person)~~~~~;\textrm{Error to give \cdf{name} the value \cd{.007}}
\end{lisp}
\end{newer}

\section{Параметры слотов для defstruct}
\label{Defstruct-Slot-Options}

Каждая форма \emph{slot-description} в форме \cdf{defstruct} может указывать
один или более параметров слота. \emph{slot-option} является парой ключевого
символа и значения (которое не является формой для вычисления, а является просто
значением).  Например:
\begin{lisp}
(defstruct ship \\
~~(x-position 0.0 \cd{:type} short-float) \\
~~(y-position 0.0 \cd{:type} short-float) \\
~~(x-velocity 0.0 \cd{:type} short-float) \\
~~(y-velocity 0.0 \cd{:type} short-float) \\
~~(mass *default-ship-mass* \cd{:type} short-float \cd{:read-only} t))
\end{lisp}
Этот пример содержит определение, что каждый слот будет всегда содержать
короткое с плавающей точкой число, и что последний слот не может быть изменён
после создания корабля \emph{ship}.
Доступные параметры для слотов \emph{slot-options}:
\begin{flushdesc}
\item[\cd{:type}] Параметр \cd{\cd{:type} \emph{type}} указывает, что содержимое
  слота будет всегда принадлежать указанному типу данных. Он похож на
  аналогичную декларацию для переменной или функции. И конечно же, он также
  декларирует возвращаемый \emph{функцией доступа} тип. Реализация может
  проверять или не проверять тип нового объекта при инициализации или
  присваивании слота.  Следует отметить, что форма аргумента \emph{type} не
  вычисляется, и следовательно должна быть корректным спецификатором типа.

\item[\cd{:read-only}] Параметр \cd{\cd{:read-only} \emph{x}}, при \emph{x} не
  {\false}, указывает , что этот слот не может быть изменён. Он будет всегда
  содержать значение, указанное во время создания экземпляра структуры.
  \cdf{setf} не принимает \emph{функцию доступа} к данному слоту.  Если \emph{x}
  {\false}, этот параметр ничего не меняет.  Следует отметить, что форма
  аргумента \emph{x} не вычисляется.
\end{flushdesc}

Следует отметить, что невозможно определить параметр для слота без указания
значение по-умолчанию.

\section{Параметры defstruct}
\label{DEFSTRUCT-OPTIONS}
\label{Defstruct-Hairy-Stuff}

Предыдущего описание \cdf{defstruct} достаточно для среднестатистического
использования. Оставшуюся часть этой главы занимает описание более сложных
возможностей функционала \cdf{defstruct}.

Данный раздел объясняет каждый параметр, который может быть использован в
\cdf{defstruct}. Параметр для \cdf{defstruct} может быть или ключевым символом,
или списком из ключевого символа и аргумента для него.  (Следует отметить, что
синтаксис для параметров \cdf{defstruct} отличается от синтаксиса пар,
используемых для параметров слота. Никакая часть этих параметров не вычисляется.)

\begin{flushdesc}
\item[\cd{:conc-name}] Данный параметр предоставляет префикс для имён функций
  доступа.  По соглашению, имена всех функций доступа к слотам структуры
  начинаются с префикса --- имени структуры с последующим дефисом. Это поведение
  по-умолчанию.

  Аргумент \cd{:conc-name} указывает альтернативный префикс. (Если в качестве
  разделителя используется дефис, он указывается как часть префикса.)  Если в
  качестве аргумента указано {\false}, тогда префикс не устанавливается вообще.
  Тогда имена функций доступа совпадают с именами слотов, и это повод давать
  слотам информативные имена.

  Следует отметить, что не зависимо от того, что указано в \cd{:conc-name}, в
  функции-конструкторе используются ключевые символы, совпадающие с именами
  слотов без присоединяемого префикса.  С другой стороны префикс используется в
  именах функций доступа. Например:
  \begin{lisp}
    (defstruct door knob-color width material) \\
    (setq my-door \\
    ~~~~~~(make-door :knob-color 'red :width 5.0)) \\
    (door-width my-door) \EV\ 5.0 \\
    (setf (door-width my-door) 43.7) \\
    (door-width my-door) \EV\ 43.7 \\
    (door-knob-color my-door) \EV\ red
  \end{lisp}

\item[\cd{:constructor}] Данный параметр принимает один аргумент, символ,
  который указывает имя функции-конструктора. Если аргумент не указан, или если
  не указан параметр, имя конструктора создаётся соединением строки \cd{"MAKE-"}
  и имени структуры, и помещается в текущий пакет во время выполнения формы
  \cdf{defstruct} (смотрите \cdf{*package*}).  Если аргумент указан и равен
  {\false}, то функция конструктор не создаётся.

Этот параметр имеет более сложный синтаксис, описываемый в
разделе~\ref{DEFSTRUCT-CONSTRUCTOR-SYNTAX}.

\item[\cd{:copier}] Данный параметр принимает один аргумент, символ, который
  указывает имя функции копирования. Если аргумент не указан, или если не указан
  весь параметр, имя функции копирования создаётся соединением строки
  \cd{"COPY-"} и имени структуры, и помещается в текущий пакет в время
  выполнения формы \cdf{defstruct} (смотрите \cdf{*package*}).  Если аргумент
  указан и равен {\false}, то функция копирования не создаётся.

  Автоматически создаваемая функция копирования просто создаёт новую структуру и
  переносит все компоненты из структуры аргумента в свеже создаваемую
  структуру. Копирование самих компонентов структуры не производится.
  Соответствующие элементы старой и новой структуры равны \cdf{eql} между собой.

\item[\cd{:predicate}] Этот параметр принимает один аргумент, который задаёт имя
  предиката типа.  Если аргумент не указан, или если не указан весь параметр, то
  имя предиката создаётся соединением имени структуры и строки \cd{"-P"},
  помещая имя в текущий пакет на момент вычисления формы \cdf{defstruct}
  (смотрите \cdf{*package*}).  Если указанный аргумент равен {\false}, предикат
  не создаётся.  Предикат может быть определён, только если структура имеет
  <<имя>>.  Если указан параметр \cd{:type} и не указан \cd{:named}, тогда
  параметр \cd{:predicate} не должен использоваться или должна иметь значение
  {\false}.

\item[\cd{:include}] Этот параметр используется для создания нового определения
  структуры как расширения для старого определения структуры. В качестве
  примера, предположим у вас есть структура, называемая \cdf{person}, которая
  выглядит так:
  \begin{lisp}
    (defstruct person name age sex)
  \end{lisp}
  Теперь, предположим, вы хотите создать новую структуру для представления
  астронавта.  Так как астронавт также человек, вы хотите, чтобы он также имел
  свойства имя, возраст и пол, и вы хотите чтобы Lisp'овые функции работали со
  структурами \cdf{astronaut} также как и с структурами \cdf{person}. Вы можете
  сделать это определив структуру \cdf{astronaut} с параметром \cd{:include},
  так:
  \begin{lisp}
    (defstruct (astronaut (:include person) \\
    ~~~~~~~~~~~~~~~~~~~~~~(:conc-name astro-)) \\
    ~~~helmet-size \\
    ~~~(favorite-beverage 'tang))
  \end{lisp}

  Параметр \cd{:include} заставляет структуру, будучи определённой, иметь те же
  слоты, что и включаемая в параметре структура.  Это реализуется с помощью
  того, что функции доступа включаемой структуры будут также работать с
  определяемой структурой.  Таким образом, в этом примере, \cdf{astronaut} будет
  иметь пять слотов: три определены в \cdf{person} и два в самом
  \cdf{astronaut}.  Функции доступа, определённые с помощью структуры
  \cdf{person}, могут применяться к экземплярам структуры \cdf{astronaut}, и
  будут корректно работать.  Более того, \cdf{astronaut} будет иметь свои
  функции доступа для компонентов унаследованных от структуры \cdf{person}.
  Следующий пример иллюстрирует то, как вы можете использовать структуры
  \cd{astronaut}:
  \begin{lisp}
    (setq x (make-astronaut :name 'buzz \\
    ~~~~~~~~~~~~~~~~~~~~~~~~:age 45 \\
    ~~~~~~~~~~~~~~~~~~~~~~~~:sex t \\
    ~~~~~~~~~~~~~~~~~~~~~~~~:helmet-size 17.5)) \\
    \\
    (person-name x) \EV\ buzz \\
    (astro-name x) \EV\ buzz \\
    \\
    (astro-favorite-beverage x) \EV\ tang
  \end{lisp}
  Различие между функциями доступа \cdf{person-name} и \cdf{astro-name} в том,
  что \cdf{person-name} может быть корректно применена к любому экземпляру
  \cdf{person}, включая \cdf{astronaut}, тогда как \cdf{astro-name} может
  работать только с \cdf{astronaut}. (Реализация может проверять или не
  проверять корректное использоваться таких функций доступа.)

  В одной форме \cdf{defstruct} не может использовать более одного параметра
  \cdf{:include}.  Аргумент для параметра \cd{:include} является обязательным и
  должен быть именем определённой ранее структуры. Если структура, будучи
  определённой, не содержала параметра \cd{:type}, тогда наследуемая структура
  также не должна содержать этот параметр.  Если структура, будучи определённой,
  содержала параметр \cd{:type}, тогда наследуемая структура должна содержать
  этот параметр с тем же типом.

  Если параметр \cd{:type} не указан, тогда имя структуры становиться именем
  типа данных. Более того, тип будет является подтипом типа структуры, от
  которой произошло наследование. В вышеприведённом примере, \cdf{astronaut}
  является подтипом \cdf{person}. Так,
  \begin{lisp}
    (typep (make-astronaut) 'person)
  \end{lisp}
  но и указывает, что все операции над \cdf{person} будут также работать для
  \cdf{astronaut}.

  Далее рассказывается чуть более сложные возможности параметра \cd{:include}.
  Иногда, когда одна структура включает другую, необходимо, чтобы значения
  по-умолчанию или параметры слотов из родительской структуры при наследовании
  стали слегка другими.  Новая структура может задать значения по-умолчанию или
  параметры для наследуемых слотов отличными от родительских, с помощью
  конструкции:
  \begin{lisp}
    (:include \emph{name} \emph{slot-description-1} \emph{slot-description-2}
    ...)
  \end{lisp}
  Каждая форма \emph{slot-description-j} должна иметь имя \emph{slot-name} или
  \emph{slot-keyword}, такое же как в родительской структуре.  Если
  \emph{slot-description-j} не имеет формы инициализации \emph{default-init},
  тогда в новой структуре слот также не будет иметь первоначального значения. В
  противном случае его первоначальное значение будет заменено формой
  \emph{default-init} из \emph{slot-description-j}.  Доступный для записи слот
  может быть переделан в слот только для чтения.  Если слот только для чтения в
  родительской структуре, тогда он также должен быть только для чтения в
  дочерней.  Если для слота указан тип, он должен быть таким же или подтипом в
  дочерней структуре. Если это строгий подтип, то реализация может проверять или
  не проверять ошибки несовпадения типов при присваивании значений слотам.

  Например, если мы хотели бы определить \cdf{astronaut} так, чтобы по-умолчанию
  возрастом астронавта было 45 лет, то мы могли бы сказать:
  \begin{lisp}
    (defstruct (astronaut (:include person (age 45))) \\
    ~~~helmet-size \\
    ~~~(favorite-beverage 'tang))
  \end{lisp}

  \begin{new}
    X3J13 voted in June 1988 \issue{DATA-TYPES-HIERARCHY-UNDERSPECIFIED} to
    require any structure type created by \cdf{defstruct} (or \cdf{defclass}) to
    be disjoint from any of the types \cdf{cons}, \cdf{symbol}, \cdf{array},
    \cdf{number}, \cdf{character}, \cdf{hash-table}, \cdf{readtable},
    \cdf{package}, \cdf{pathname}, \cdf{stream}, and \cdf{random-state}.  A
    consequence of this requirement is that it is an error to specify any of
    these types, or any of their subtypes, to the \cdf{defstruct} \cd{:include}
    option.  (The first edition said nothing explicitly about this.  Inasmuch as
    using such a type with the \cd{:include} option was not defined to work, one
    might argue that such use was an error in Common Lisp as defined by the
    first edition.)
  \end{new}

\item[\cd{:print-function}] Этот параметр может использоваться, только если не
  параметр \cd{:type} не была указан.  Аргумент для параметра
  \cd{:print-function} должен быть функцией трёх аргументов, в форме принимаемой
  специальной формой \cdf{function}. Эта функция используется для вывода
  рассматриваемой структуры.  Когда структура выводится на консоль, данная
  функция вызывается с тремя аргументами: структура для печати, поток, в который
  производить вывод, и целое число, отображающее текущую глубину (можно сравнить
  с \cdf{*print-level*}).  Функция вывода должна следить за значениями таких
  переменных настройки вывода, как \cdf{*print-escape*} и \cdf{*print-pretty*}.

  Если параметры \cd{:print-function} и \cd{:type} не указаны, тогда функция
  вывода по-умолчанию выводит все слоты используя синтаксис \cd{\#S} (смотрите
  раздел~\ref{SHARP-SIGN-MACRO-CHARACTER-SECTION}).

\begin{new}
  X3J13 voted in January 1989 \issue{PRINT-CIRCLE-STRUCTURE} to specify that
  user-defined printing functions for the \cdf{defstruct} \cd{:print-function}
  option may print objects to the supplied stream using \cdf{write},
  \cd{print1}, \cdf{princ}, \cdf{format}, or \cdf{print-object} and expect
  circularities to be detected and printed using \cd{\#\emph{n\/}\#} syntax
  (when \cdf{*print-circle*} is non-\cdf{nil}, of course).  See
  \cdf{*print-circle*}.
\end{new}


\begin{new}
  X3J13 voted in January 1989 \issue{DEFSTRUCT-PRINT-FUNCTION-INHERITANCE} to
  clarify that if the \cd{:print-function} option is not specified but the
  \cd{:include} option \emph{is} specified, then the print function is inherited
  from the included structure type.  Thus, for example, an \cdf{astronaut} will
  be printed by the same printing function that is used for \cdf{person}.

  X3J13 in the same vote extended the \cdf{print-function} option as follows: If
  the \cdf{print-function} option is specified but with no argument, then the
  standard default printing function (that uses \cd{\#S} syntax) will be used.
  This provides a means of overriding the inheritance rule.  For example, if
  \cdf{person} and \cdf{astronaut} had been defined as
\begin{lisp}
(defstruct (person \\*
~~~~~~~~~~~~~(:print-function~~~~~;\textrm{Special print function}\\*
~~~~~~~~~~~~~(lambda (p s k) \\*
~~~~~~~~~~~~~~~(format s "<{\Xtilde}A, age {\Xtilde}D>" \\*
~~~~~~~~~~~~~~~~~~~~~~~(person-name p) \\*
~~~~~~~~~~~~~~~~~~~~~~~(person-age p))))) \\*
~~name age sex) \\
\\
(defstruct (astronaut \\*
~~~~~~~~~~~~~(:include person) \\*
~~~~~~~~~~~~~(:conc-name astro-) \\*
~~~~~~~~~~~~~(:print-function))~~~~~;\textrm{Use default print function} \\*
~~~helmet-size \\*
~~~(favorite-beverage 'tang))
\end{lisp}
then an ordinary person would be printed as ``\cd{<Joe Schmoe, age 27>}''
but an astronaut would be printed as, for example,
\begin{lisp}
\#S(ASTRONAUT NAME BUZZ AGE 45 SEX T \\*
~~~HELMET-SIZE 17.5 FAVORITE-BEVERAGE TANG)
\end{lisp}
using the default \cd{\#S} syntax (yuk).

These changes make the behavior of \cdf{defstruct} with respect to the
\cd{:include} option a bit more like the behavior of classes in CLOS.
\end{new}

\item[\cd{:type}] Параметр \cd{:type} явно задаёт представление используемое для
  структуры. Он принимает один аргумент, который должен быть одним из
  перечисленных ниже.

  Этот параметр заставляет использовать указанное представление и заставляет
  компоненты быть размещёнными в порядке, предусмотренном в \cdf{defstruct}
  форме, в соответствующие последовательные элементы указанного представления.
  Параметр также отключает возможность имени структуры стать именем типа.
  (смотрите раздел~\ref{EXPLICIT-TYPE-STRUCTURES}).

  Обычно этот параметр не используется, и тогда структура представляется так,
  как предусмотрено в реализации.

  \begin{quotation} % Merely to advance the left margin
    \begin{flushdesc}

    \item[\cdf{vector}] Структура представляется как вектор \cd{(vector
        t)}. размещая компоненты как элементы вектора. Первый компонент
      находится в первом элементе вектора, если структура содержит параметр
      \cd{:named}, и в нулевом, если структура содержит \cd{:unnamed}.

    \item[\cd{(vector \emph{element-type})}] Структура представляется как
      (возможно специализированный) вектор, размещая компоненты как элементы
      вектора. Каждый компонент должен принадлежать типу, который мог быть
      указан для вектора. Первый компонент находится в первом элементе вектора,
      если структура содержит параметр \cd{:named}, и в нулевом, если структура
      содержит \cd{:unnamed}.  Структура может быть \cd{:named}, только если тип
      \cdf{symbol} является подтипом указанного в \cdf{element-type}.

    \item[\cdf{list}] Структура представляется как список.  Первый компонент
      является \emph{cadr} элементом, если структура содержит \cd{:named}, и
      является \emph{car} элементом, если структура содержит \cd{:unnamed}.
    \end{flushdesc}
  \end{quotation}

\item[\cd{:named}] Параметр \cd{:named} указывает, что структура будет иметь
  <<имя>>. Этот параметр не принимает аргументов. Если параметр \cd{:type} не
  указан, тогда структура имеет имя. Таким образом этот параметр полезен только
  при использовании вместе с \cd{:type}.  Смотрите
  раздел~\ref{EXPLICIT-TYPE-STRUCTURES} для подробного описания.

\item[\cd{:initial-offset}] Этот параметр позволяет вам указать \cdf{defstruct}
  пропустить указанное количество слотов, перед тем как начинать размещать
  указанные в теле слоты. Эта параметр требует аргумент, а именно,
  неотрицательное целое, обозначающее количество пропускаемых слотов. Параметр
  \cd{:initial-offset} может использоваться, только если также указан параметр
  \cd{:type}.  Смотрите раздел~\ref{DEFSTRUCTU-INIITAL-OFFSET} для подробного
  описания.
\end{flushdesc}

\section{Функции-конструкторы с позиционными аргументами}
\label{DEFSTRUCT-CONSTRUCTOR-SYNTAX}

Если параметр \cd{:constructor} указан: так \cd{(\cd{:constructor} \emph{name}
  \emph{arglist})}, тогда вместо создания конструктора с именованными
параметрами, \cdf{defstruct} определит конструктор с позиционными аргументами.
Форма \emph{arglist} используется для описания того, какие аргументы будет
принимать конструктор. В простейшем случае что-то вроде \cd{(\cd{:constructor}
  make-foo (a b c))} определяет функцию \cdf{make-foo} с тремя аргументами,
которые используются для инициализации слотов \cdf{a}, \cdf{b} и \cdf{c}.

Кроме того в списке аргументов могут использоваться \cd{\&optional}, \cd{\&rest}
и \cd{\&aux}. Они работают так, как и ожидается, но есть несколько тонкостей
требующих объяснения. Рассмотрим пример:
\begin{lisp}
(\cd{:constructor} create-foo \\
~~~~~~~~(a \&optional b (c 'sea) \&rest d \&aux e (f 'eff)))
\end{lisp}
Эта конструкция определяет конструктор \cdf{create-foo} для использования с
одним или более аргументами. Первый аргумент используется для инициализации
слота \cdf{a}. Если второй параметр не указан, используется значение (если
указано) по-умолчанию из тела \cdf{defstruct}. Третий аргумент используется для
инициализации слота \cdf{c}. Если третий параметр не указан, тогда используется
символ \cdf{sea}. Все параметры после третьего собираются в список и
используются для инициализации слота \cdf{d}. Если указано три и более
параметров, тогда в слот \cdf{d} помещается значение {\false}. Слот \cdf{e}
\emph{не инициализируется}. Его первоначальное значение не определено. Наконец,
слот \cdf{f} инициализируется символом \cdf{eff}.

Действия со слотами \cdf{b} и \cdf{e} выбраны не случайно, а для того чтобы
показать все возможные случаи использования аргументов.
Следует отметить, что \cd{\&aux} (вспомогательные) <<переменные>> могут
использоваться для перекрытия форм инициализации из тела \cdf{defstruct}.

Следуя этому определению можно записать
\begin{lisp}
(create-foo 1 2)
\end{lisp}
вместо
\begin{lisp}
(make-foo \cd{:a} 1 \cd{:b} 2)
\end{lisp}
и, конечно, \cdf{create-foo} предоставляет инициализацию отличную от
\cdf{make-foo}.

Использовать \cd{:constructor} можно более одного раза. Таким образом вы можете
определить несколько различных функций-конструкторов с различными аргументами.

\begin{new}
X3J13 voted in January 1989
\issue{DEFSTRUCT-CONSTRUCTOR-KEY-MIXTURE}
to allow \cd{\&key} and \cd{\&allow-other-keys}
in the
parameter list of a ``positional'' constructor.  The initialization of slots
corresponding to keyword parameters
is performed in the same manner as for \cd{\&optional} parameters.
A variant of the example shown above illustrates this:
\begin{lisp}
(\cd{:constructor} create-foo \\*
~~~~~~~~(a \&optional b (c 'sea) \\*
~~~~~~~~~\&key p (q 'cue) ((:why y)) ((:you u) 'ewe) \\*
~~~~~~~~~\&aux e (f 'eff)))
\end{lisp}
The treatment of slots \cdf{a}, \cdf{b}, \cdf{c}, \cdf{e}, and \cdf{f}
is the same as in the original example.  In addition,
if there is a \cd{:p} keyword argument, it is
used to initialize the \cdf{p} slot;  if there isn't any
\cd{:p} keyword argument, then the default value given in the body of the
\cdf{defstruct} (if given) is used instead.  Similarly,
if there is a \cd{:q} keyword argument, it is
used to initialize the \cdf{q} slot;  if there isn't any
\cd{:q} keyword argument, then
the symbol \cdf{cue} is used instead.

In order thoroughly to flog this presumably already dead horse,
we further observe that if there is a \cd{:why} keyword argument, it is
used to initialize the \cdf{y} slot; otherwise
the default value for slot \cdf{y} is used instead.  Similarly,
if there is a \cd{:you} keyword argument, it is
used to initialize the \cdf{u} slot;  otherwise
the symbol \cdf{ewe} is used instead.

If memory serves me correctly, \cdf{defstruct} was included in the original
design for Common Lisp some time before keyword arguments were approved.
The failure of positional constructors to accept keyword arguments may well
have been an oversight on my part; there is no logical reason to exclude
them.  I am grateful to X3J13 for rectifying this.

A remaining difficulty is that the possibility of keyword arguments
renders the term ``positional constructor'' a misnomer.  Worse yet,
it ruins the term ``BOA constructor.''  I suggest that
they continue to be called BOA constructors, as I refuse to abandon
a good pun.  (I regret appearing to have more compassion for puns than
for horses.)

As part of the same vote X3J13 also changed \cdf{defstruct}
to allow BOA constructors to have
parameters (including supplied-p parameters)
that do not correspond to any
slot.  Such parameters may be used in subsequent initialization forms in the
parameter list. Consider this example:
\begin{lisp}
(defstruct (ice-cream-factory \\*
~~~~~~~~~~~~~(:constructor fabricate-factory \\*
~~~~~~~~~~~~~~~(\&key (capacity 5) \\*
~~~~~~~~~~~~~~~~~~~~~~location \\
~~~~~~~~~~~~~~~~~~~~~~(local-flavors \\*
~~~~~~~~~~~~~~~~~~~~~~~~(case location \\*
~~~~~~~~~~~~~~~~~~~~~~~~~~((hawaii) '(pineapple macadamia guava)) \\*
~~~~~~~~~~~~~~~~~~~~~~~~~~((massachusetts) '(lobster baked-bean)) \\*
~~~~~~~~~~~~~~~~~~~~~~~~~~((california) '(ginger lotus avocado \\*
~~~~~~~~~~~~~~~~~~~~~~~~~~~~~~~~~~~~~~~~~~bean-sprout garlic)) \\*
~~~~~~~~~~~~~~~~~~~~~~~~~~((texas) '(jalapeno barbecue)))) \\
~~~~~~~~~~~~~~~~~~~~~~(flavors (subseq (append local-flavors \\*
~~~~~~~~~~~~~~~~~~~~~~~~~~~~~~~~~~~~~~~~~~~~~~~'(vanilla \\*
~~~~~~~~~~~~~~~~~~~~~~~~~~~~~~~~~~~~~~~~~~~~~~~~~chocolate \\*
~~~~~~~~~~~~~~~~~~~~~~~~~~~~~~~~~~~~~~~~~~~~~~~~~strawberry \\
~~~~~~~~~~~~~~~~~~~~~~~~~~~~~~~~~~~~~~~~~~~~~~~~~pistachio \\
~~~~~~~~~~~~~~~~~~~~~~~~~~~~~~~~~~~~~~~~~~~~~~~~~maple-walnut \\*
~~~~~~~~~~~~~~~~~~~~~~~~~~~~~~~~~~~~~~~~~~~~~~~~~peppermint)) \\*
~~~~~~~~~~~~~~~~~~~~~~~~~~~~~~~~~~~~~~~0 capacity))))) \\*
~~(capacity 3) \\*
~~(flavors '(vanilla chocolate strawberry mango)))
\end{lisp}

The structure type \cdf{ice-cream-factory} has two constructors.
The standard constructor, \cdf{make-ice-cream-factory},
takes two keyword arguments named \cd{:capacity} and \cd{:flavors}.
For this constructor, the default for the \cdf{capacity} slot is \cd{3}
and the default list of \cdf{flavors} is America's favorite threesome
and a dark horse (not a dead one).
The BOA constructor \cdf{fabricate-factory}
accepts four different keyword arguments.  The \cd{:capacity}
argument defaults to \cd{5}, and the \cd{:flavors} argument
defaults in a complicated manner based on the other three.
The \cd{:local-flavors} argument may be specified directly,
or may be allowed to default based on the \cd{:location} of the factory.
Here are examples of various factories:

\vskip0pt plus 2pt%manual
\begin{lisp}
(setq houston (fabricate-factory :capacity 4 :location 'texas)) \\*
(setq cambridge (fabricate-factory :location 'massachusetts)) \\
(setq seattle (fabricate-factory :local-flavors '(salmon))) \\
(setq wheaton (fabricate-factory :capacity 4 :location 'illinois)) \\*
(setq pittsburgh (fabricate-factory :capacity 4)) \\*
(setq cleveland (make-factory :capacity 4)) \\
 \\
(ice-cream-factory-flavors houston) \\*
~\EV~(jalapeno barbecue vanilla chocolate)
\end{lisp}
\newpage%manual
\begin{lisp}
(ice-cream-factory-flavors cambridge) \\*
~\EV~(lobster baked-bean vanilla chocolate strawberry) \\
\\
(ice-cream-factory-flavors seattle) \\*
~\EV~(salmon vanilla chocolate strawberry pistachio) \\
\\
(ice-cream-factory-flavors wheaton) \\*
~\EV~(vanilla chocolate strawberry pistachio) \\
\\
(ice-cream-factory-flavors pittsburgh) \\*
~\EV~(vanilla chocolate strawberry pistachio) \\
\\
(ice-cream-factory-flavors cleveland) \\*
~\EV~(vanilla chocolate strawberry mango)
\end{lisp}
\end{new}

\section{Структуры с явно заданным типом представления}
\label{EXPLICIT-TYPE-STRUCTURES}

Иногда необходимо явно контролировать представление структуры. Параметр
\cd{:type} позволяет выбрать представление между списком или некоторым видом
вектора и указать соотвествие для размещения слотов в выбранном представлении.
Структура также может быть <<безымянной>> или <<именованной>>. Это означает
может ли имя структуры быть сохранено в ней самой (и соответственно, прочитано
из неё).

\subsection{Безымянные структуры}

Иногда конкретное представление данных навязывается внешними требованиями и,
кроме того, формат данных прекрасно ложится в структурный стиль хранения.
Например, рассмотрим выражение созданное из чисел, символов и таких операций как
\cdf{+} и \cdf{*}. Операция может быть представлена, как в Lisp'е, списком из
оператора и двух операндов. Этот факт может быть выражен кратко в терминах
\cdf{defstruct}:
е\begin{lisp}
(defstruct (binop (:type list)) \\
~~(operator '? :type symbol) \\
~~operand-1 \\
~~operand-2)
\end{lisp}

Результатом выполнения \cdf{make-binop} является 3-ёх элементный список:
\begin{lisp}
(make-binop :operator '+ :operand-1 'x :operand-2 5) \\*
~~~\EV\ (+ x 5) \\
\\
(make-binop :operand-2 4 :operator '*) \\*
~~~\EV\ (* {\nil} 4)
\end{lisp}
Выглядит как функция \cdf{list} за исключением того, что принимает именованные
параметры и выполняет инициализацию слотов соответствующую концептуальному типу
данных \cdf{binop}.
Таким же образом, селекторы \cdf{binop-operator}, \cdf{binop-operand-1} и
\cdf{binop-operand-2} эквивалентны соответственно \cdf{car}, \cdf{cadr} и
\cdf{caddr}. (Они, конечно, не полностью эквивалентны, так как реализация
может осуществлять проверки типов элементов, длины массивов при использовании
селекторов слотов структур.)

Мы говорим о \cdf{binop} как о <<концептуальном>> типе данных, потому что
\cdf{binop} не принадлежит Common Lisp'овой системе типов. Предикат \cdf{typep}
не может использовать \cdf{binop} как спецификатор типа, и \cdf{type-of} будет
возвращать \cdf{list} для заданной \cdf{binop} структуры. Несомненно, различий
между структурой данных, созданной с помощью \cdf{make-binop}, и  простым
списком нет.

Невозможно даже получить имя структуры для объекта, созданного с помощью
\cdf{make-binop}. Однако имя может быть сохранено и получено, если структура
содержит <<имя>>.

\subsection{Именованные структуры}

Структура с <<именем>> имеет свойство, которое заключается в том, что для любого
экземпляра структуры можно получить имя этой структуры. Для структур
определённых без указания параметра \cd{:type}, имя структуры фактически становится
частью Common Lisp'овой системы типов. Функция \cdf{type-of} при применении к
экземпляру такой структуры будет возвращать имя структуры. Предикат \cdf{typep}
будет рассматривать имя структуры, как корректный спецификатор типа.

Для структур определённых с параметром \cd{:type}, \cdf{type-of} будет
возвращать спецификатор типа один из \cdf{list} или \cdf{(vector t)}, в
зависимости от указанного аргумента параметра \cd{:type}.  Имя структуры не
становится корректным спецификатором типа. Однако если также указан параметр
\cd{:named}, тогда первый компонент структуры всегда содержит её имя. Это
позволяет получить это имя, имея только экземпляр структуры. Это также позволяет
автоматически определить предикат для концептуального типа.  Предикат
\cd{\emph{name}-p} для структуры принимает в качестве первого аргумента объект и
истинен, если объект является экземпляром структуры, иначе ложен.

Рассмотрим вышеупомянутый пример \cdf{binop} и модифицируем его, добавив
параметр \cd{:named}:
\begin{lisp}
(defstruct (binop (:type list) :named) \\
~~(operator '? :type symbol) \\
~~operand-1 \\
~~operand-2)
\end{lisp}
Как и раньше, конструкция определить функцию-конструктор \cdf{make-binop} и три
функции-селектора  \cdf{binop-operator}, \cd{binop-operand-1} и
\cd{binop-operand-2}. Она также определит предикат \cdf{binop-p}.

Результатом \cdf{make-binop} теперь является список с 4-мя элементами:
\begin{lisp}
(make-binop :operator '+ :operand-1 'x :operand-2 5) \\*
~~~\EV\ (binop + x 5) \\
\\
(make-binop :operand-2 4 :operator '*) \\*
~~~\EV\ (binop * {\nil} 4)
\end{lisp}
Структура имеет такую же разметку как и раньше за исключением имени структуры в
первом элементе.
Функции-селекторы
\cdf{binop-operator}, \cd{binop-operand-1}и \cd{binop-operand-2} эквивалентны
соответственно \cdf{cadr}, \cdf{caddr} и \cdf{cadddr}.
Предикат \cdf{binop-p} примерно соответствует следующему определению.
\begin{lisp}
(defun binop-p (x) \\
~~(and (consp x) (eq (car x) 'binop)))
\end{lisp}
Имя \cdf{binop} не является корректным спецификатором типа, и не может
использоваться в \cdf{typep}. Но с помощью предиката структура теперь отличима 
от других структур. 

\subsection{Другие аспекты явно определённых типов для представления структур}
\label{DEFSTRUCT-INITIAL-OFFSET}

Параметр \cd{:initial-offset} позволяет указывать начало размещения слотов в
представлении структуры. Например, форма
\begin{lisp}
(defstruct (binop (:type list) (:initial-offset 2)) \\*
~~(operator '? :type symbol) \\*
~~operand-1 \\*
~~operand-2)
\end{lisp}
создаст конструктор \cdf{make-binop} со следующим поведением:
\begin{lisp}
(make-binop :operator '+ :operand-1 'x :operand-2 5) \\*
~~~\EV\ (nil nil + x 5) \\
\\
(make-binop :operand-2 4 :operator '*) \\*
~~~\EV\ (nil nil * {\nil} 4)
\end{lisp}
Селекторы
\cdf{binop-operator}, \cd{binop-operand-1} и \cd{binop-operand-2} 
будут эквивалентны соответственно \cdf{caddr},
\cdf{cadddr}, и \cdf{car} от \cdf{cddddr}.
Таким же образом, форма
\begin{lisp}
(defstruct (binop (:type list) :named (:initial-offset 2)) \\*
~~(operator '? :type symbol) \\*
~~operand-1 \\*
~~operand-2)
\end{lisp}
создаст конструктор \cdf{make-binop} со следующим поведением:
\begin{lisp}
(make-binop :operator '+ :operand-1 'x :operand-2 5) \\*
~~~\EV\ (nil nil binop + x 5) \\
\\
(make-binop :operand-2 4 :operator '*) \\*
~~~\EV\ (nil nil binop * {\nil} 4)
\end{lisp}

Если вместе с параметром \cd{:type} используется \cd{:include}, тогда в
представлении выделяется столько места, сколько необходимо для родительской
структуры, затем пропускается столько места, сколько указано в параметре
\cd{:initial-offset}, и затем начинается расположение элементов определяемой
структуры.  Например:
\begin{lisp}
(defstruct (binop (:type list) :named (:initial-offset 2)) \\
~~(operator '? :type symbol) \\
~~operand-1 \\
~~operand-2) \\
 \\
(defstruct (annotated-binop (:type list) \\
~~~~~~~~~~~~~~~~~~~~~~~~~~~~(:initial-offset 3) \\
~~~~~~~~~~~~~~~~~~~~~~~~~~~~(:include binop)) \\
~~commutative associative identity) \\
 \\
(make-annotated-binop :operator '* \\
~~~~~~~~~~~~~~~~~~~~~~:operand-1 'x \\
~~~~~~~~~~~~~~~~~~~~~~:operand-2 5 \\
~~~~~~~~~~~~~~~~~~~~~~:commutative t \\
~~~~~~~~~~~~~~~~~~~~~~:associative t \\
~~~~~~~~~~~~~~~~~~~~~~:identity 1) \\
~~~\EV\ (nil nil binop * x 5 nil nil nil t t 1)
\end{lisp}
Первые два {\nil} элемента пропущены по причине параметра \cd{:initial-offset}
со значением \cd{2} в определении \cdf{binop}. Следующие четыре элемента
содержат имя и три слота структуры \cdf{binop}. Следующие три {\nil} элементы
пропущено по причине параметра \cd{:initial-offset} за значением \cd{3} в
определении структуры \cdf{annotated-binop}. Последние три элемента содержат три
слота, определённых в структуре \cdf{annotated-binop}.

\fi      % Structure package
%Part{Eval, Root = "CLM.MSS"}
%%%Chapter of Spice Lisp Manual.  Copyright 1984, 1988, 1989 Guy L. Steele Jr.

\clearpage\def\pagestatus{FINAL PROOF}

\ifx \rulang\Undef

\chapter{Evaluator}

The mechanism that executes Lisp programs is called the evaluator.
More precisely, the evaluator accepts a form and performs the
computation specified by the form.  This mechanism is made available
to the user through the function \cdf{eval}.

The evaluator is typically implemented as an interpreter
that traverses the given form recursively, performing each step
of the computation as it goes.  An interpretive implementation is not
required, however.  A permissible alternative approach is
for the evaluator first to completely compile the form into
machine-executable code and then invoke the resulting code.
This technique virtually eliminates incompatibilities
between interpreted and compiled code but also renders the \cdf{evalhook}
mechanism relatively useless.
Various mixed strategies are also possible.  All of these approaches
should produce the same results when executing a correct program
but may produce different results for incorrect programs.
For example, the approaches may differ as to when macro calls
are expanded; macro definitions should not depend on the time
at which they are expanded.  Implementors should
document the evaluation strategy for each implementation.

\section{Run-Time Evaluation of Forms}

The function \cdf{eval} is the main user interface to the evaluator.
Hooks are provided for user-supplied debugging routines
to obtain control during the execution of an interpretive evaluator.
The functions \cdf{evalhook} and \cdf{applyhook} provide alternative
interfaces to the evaluator mechanism for use by these debugging routines.

\begin{defun}[Function]
eval form

The \emph{form} is evaluated in the current dynamic environment and
a null lexical environment.  Whatever results from the evaluation
is returned from the call to \cdf{eval}.

Note that when you write a call to \cdf{eval} \emph{two} levels
of evaluation occur on the argument form you write.
First the argument form is evaluated, as for arguments to any function,
by the usual argument evaluation mechanism
(which involves an implicit use of \cdf{eval}).  Then the argument
is passed to the \cdf{eval} function, where another evaluation occurs.
For example:
\begin{lisp}
(eval (list 'cdr (car '((quote (a . b)) c)))) \EV\ b
\end{lisp}
The argument form \cd{(list 'cdr (car '((quote (a . b)) c)))} is evaluated
in the usual way to produce the argument \cd{(cdr (quote (a . b)))};
this is then given to \cdf{eval} because \cdf{eval} is being called explicitly,
and \cdf{eval} evaluates its argument \cd{(cdr (quote (a . b)))} to produce \cdf{b}.

If all that is required for some application is
to obtain the current dynamic value of a given symbol, the function
\cdf{symbol-value} may be more efficient than \cdf{eval}.

\begin{new}
X3J13 voted in January 1989
\issue{MAPPING-DESTRUCTIVE-INTERACTION}
to restrict user side effects; see section~\ref{STRUCTURE-TRAVERSAL-SECTION}.
\end{new}
\end{defun}

\begin{defun}[Variable]
*evalhook* \\
*applyhook*

If the value of \cdf{*evalhook*} is not {\false}, then \cdf{eval} behaves
in a special way.  The non-{\false} value of \cdf{*evalhook*} should be a function
that takes two arguments, a form and an environment;
this is called the \emph{eval hook function}.
When a form is to be evaluated (any form at all, even a number or a symbol),
whether implicitly or via an explicit call to \cdf{eval}, no attempt
is made to evaluate the form.
Instead, the hook function is invoked and is passed the form to be evaluated
as its first argument.  The hook function is then responsible for
evaluating the form; whatever is returned by the hook function is assumed
to be the result of evaluating the form.

The variable \cdf{*applyhook*} is similar to \cdf{*evalhook*} but is used
when a function is about to be applied to arguments.
If the value of \cdf{*applyhook*} is not {\false}, then \cdf{eval} behaves
in a special way.
\begin{obsolete}
The non-{\false} value of \cdf{*applyhook*} should be a function
that takes three arguments: a function, a list of arguments,
and an environment;
this is called the \emph{apply hook function}.
\end{obsolete}

\begin{new}
X3J13 voted in January 1989
\issue{APPLYHOOK-ENVIROMENT}
to revise the definition of \cdf{*applyhook*}.
Its value should be a function of \emph{two} arguments,
a function and a list of arguments; no environment information is passed
to an apply hook function.

This was simply a flaw in the first edition.  Sorry about that.
\end{new}

When a function is about to be applied to a list of arguments,
no attempt is made to apply the function.
Instead, the hook function is invoked and is passed the function and the list
of arguments
as its first and second arguments.  The hook function is then responsible for
evaluating the form; whatever is returned by the hook function is assumed
to be the result of evaluating the form.
The apply hook function is used only for application of ordinary functions
within \cdf{eval}.  It is not used for applications via \cdf{apply} or
\cdf{funcall}, for applications by such functions as \cdf{map} or
\cdf{reduce}, or for invocation of macro-expansion functions
by either \cdf{eval} or \cdf{macroexpand}.
\begin{newer}
X3J13 voted in June 1988 \issue{FUNCTION-TYPE} to specify
that the value of \cdf{*macroexpand-hook*} is first coerced to a
function before being called as the expansion interface hook.
This vote made no mention of \cdf{*evalhook*} or \cdf{*applyhook*},
but this may have been an oversight.

A proposal was submitted to X3J13 in September 1989 to specify
that the value of \cdf{*evalhook*} or \cdf{*applyhook*} is first coerced to a
function before being called.  If this proposal is accepted,
the value of either variable may be \cdf{nil}, any other symbol,
a lambda-expression, or any object of type \cdf{function}.
\end{newer}

The last argument passed to either kind of hook function contains information
about the lexical environment in an implementation-dependent format.
These arguments are suitable for the functions
\cdf{evalhook}, \cdf{applyhook}, and \cdf{macroexpand}.

When either kind of hook function is invoked, both of the variables
\cdf{*evalhook*}
and \cdf{*applyhook*} are rebound to the value {\nil} around the invocation
of the hook function.  This is so that the hook function will not be
invoked recursively on evaluations and applications that occur
in the course of executing the code of the hook function.
The functions \cdf{evalhook}
and \cdf{applyhook} are useful for performing recursive evaluations
and applications within the hook function.

The hook feature is provided as an aid to debugging.
The \cdf{step} facility is implemented using this hook.

If a non-local exit causes a throw back to the top level
of Lisp, perhaps because an error could not
be corrected, then \cdf{*evalhook*} and \cdf{*applyhook*} are
automatically reset to {\false} as a safety feature.
\end{defun}

\begin{defun}[Function]
evalhook form evalhookfn applyhookfn &optional env \\
applyhook function args evalhookfn applyhookfn &optional env

The functions \cdf{evalhook} and \cdf{applyhook} are provided to make it
easier to exploit the hook feature.

In the case of \cdf{evalhook}, the \emph{form} is evaluated.
In the case of \cdf{applyhook}, the \emph{function} is applied to the
list of arguments \emph{args}.  In either case,
for the duration of the operation
the variable \cdf{*evalhook*} is bound to \emph{evalhookfn}, and
\cdf{*applyhook*} is bound to \emph{applyhookfn}.
Furthermore, the \emph{env} argument
is used as the lexical environment for the operation;
\emph{env} defaults to the null environment.
The check for a hook function is \emph{bypassed} for the evaluation
of the \emph{form} itself (for \cdf{evalhook}) or for the application
of the \emph{function} to the \emph{args} itself (for \cdf{applyhook}),
but not for subsidiary evaluations and applications
such as evaluations of subforms.  It is this one-shot bypass that makes
\cdf{evalhook} and \cdf{applyhook} so useful.

\begin{new}
X3J13 voted in January 1989
\issue{APPLYHOOK-ENVIROMENT}
to eliminate the optional \emph{env}
parameter to \cdf{applyhook}, because it is not (and cannot)
be useful.  Any function that can be applied carries its own
environment and does not need another environment to be specified
separately.
This was a flaw in the first edition.
\end{new}

Here is an example of a very simple tracing routine that uses just the
\cdf{evalhook} feature.
\begin{lisp}
(defvar *hooklevel* 0) \\
 \\
(defun hook (x) \\
~~(let ((*evalhook* 'eval-hook-function)) \\
~~~~(eval x))) \\
 \\
(defun eval-hook-function (form \cd{\&rest} env) \\
~~(let ((*hooklevel* (+ *hooklevel* 1))) \\
~~~~(format *trace-output* "{\Xtilde}\%{\Xtilde}V{\Xatsign}TForm:~~{\Xtilde}S" \\
~~~~~~~~~~~~(* *hooklevel* 2) form) \\
~~~~(let ((values (multiple-value-list \\
~~~~~~~~~~~~~~~~~~~~~(evalhook form \\
~~~~~~~~~~~~~~~~~~~~~~~~~~~~~~~\#'eval-hook-function \\
~~~~~~~~~~~~~~~~~~~~~~~~~~~~~~~{\nil} \\
~~~~~~~~~~~~~~~~~~~~~~~~~~~~~~~env)))) \\
~~~~~~(format *trace-output* "{\Xtilde}\%{\Xtilde}V{\Xatsign}TValue:{\Xtilde}{\Xlbrace} {\Xtilde}S{\Xtilde}{\Xrbrace}" \\
~~~~~~~~~~~~~~(* *hooklevel* 2) values) \\
~~~~~~(values-list values))))
\end{lisp}
Using these routines, one might see the following interaction:
\begin{lisp}
(hook '(cons (floor *print-base* 2) 'b)) \\
~~Form:  (CONS (FLOOR *PRINT-BASE* 2) (QUOTE B)) \\
~~~~Form:  (FLOOR *PRINT-BASE* 3) \\
~~~~~~Form:  *PRINT-BASE* \\
~~~~~~Value: 10 \\
~~~~~~Form:  3 \\
~~~~~~Value: 3 \\
~~~~Value: 3 1 \\
~~~~Form:  (QUOTE B) \\
~~~~Value: B \\
~~Value: (3 . B) \\
(3 . B)
\end{lisp}
\end{defun}

\begin{defun}[Function]
constantp object

If the predicate \cdf{constantp} is true of an object,
then that object, when considered as a form to
be evaluated, always evaluates to the same thing;
it is a constant.
This includes self-evaluating objects such as numbers, characters,
strings, bit-vectors, and keywords, as well as all constant symbols
declared by \cdf{defconstant},
such as \cdf{nil}, \cdf{t}, and \cdf{pi}.
In addition, a list whose \emph{car} is \cdf{quote},
such as \cd{(quote foo)}, is considered to be a constant.

If \cdf{constantp} is false of an object, then
that object, considered as a form,
might or might not always evaluate to the same thing.
\end{defun}

\section{The Top-Level Loop}

Normally one interacts with Lisp through a ``top-level
\cdf{read}-\cdf{eval}-\cdf{print} loop,'' so called because
it is the highest level of control and consists of an endless
loop that reads an expression, evaluates it, and prints the
results.  One has an effect on the state of the Lisp system
only by invoking actions that have side effects.

The precise nature of the top-level loop for Common Lisp
is purposely not rigorously specified here so that implementors can
experiment to improve the user interface.
For example, an implementor may choose to require line-at-a-time
input, or may provide a fancy editor or complex graphics-display
interface.  An implementor may choose to provide
explicit prompts for input,
or may choose (as MacLisp does) not to clutter up the transcript
with prompts.

The top-level loop is required to trap all throws and recover gracefully.
It is also required to print all values resulting from evaluation of a form,
perhaps on separate lines.  If a form returns zero values, as little
as possible should be printed.

The following variables are maintained by the top-level loop
as a limited safety net, in case the user forgets to save an interesting input
expression or output value.  (Note that the names of some of these variables
violate the convention that names of global variables begin and end with
an asterisk.)  These are intended primarily for user interaction, which is why
they have short names.  Use of these variables should be avoided in programs.

\begin{defun}[Variable]
+ \\
++ \\
+++ 

While a form is being evaluated by the top-level loop,
the variable \cdf{+} is bound to the previous form read by the loop.
The variable \cdf{++} holds the previous value of \cdf{+} (that is, the form
evaluated two interactions ago), and \cdf{+++} holds the previous value
of \cdf{++}.
\end{defun}

\begin{defun}[Variable]
-

While a form is being evaluated by the top-level loop,
the variable \cdf{-} is bound to the form itself; that is,
it is the value about to be given to \cdf{+} once this interaction
is done.
\begin{new}%CORR
\emph{Notice of correction.}
In the first edition, the name of the variable \cdf{-} was
inadvertently omitted.
\end{new}
\end{defun}

\begin{defun}[Variable]
* \\
** \\
***

While a form is being evaluated by the top-level loop,
the variable \cdf{*} is bound to the result printed at the
end of the last time through the loop; that is, it is the value
produced by evaluating the form in \cdf{+}.  If several values were produced,
\cdf{*} contains the first value only; \cdf{*} contains {\nil} if zero values
were produced.
The variable \cdf{**} holds the previous value of \cdf{*} (that is, the result
printed two interactions ago), and \cdf{***} holds the previous value
of \cdf{**}.

If the evaluation of \cdf{+} is aborted for some reason,
then the values associated with \cdf{*}, \cdf{**}, and \cdf{***} are not updated;
they are updated only if the printing of values is at least begun (though not
necessarily completed).
\end{defun}

\begin{defun}[Variable]
/ \\
// \\
///

While a form is being evaluated by the top-level loop,
the variable \cdf{/} is bound to a list of the results printed at the
end of the last time through the loop; that is, it is a list of all values
produced by evaluating the form in \cdf{+}.  The value of \cdf{*} should
always be the same as the \emph{car} of the value of \cdf{/}.
The variable \cdf{//} holds the previous value of \cdf{/} (that is, the results
printed two interactions ago), and \cdf{///} holds the previous value
of \cdf{//}.  Therefore the value of \cdf{**} should always be the same
as the \emph{car} of \cdf{//}, and similarly for \cdf{***} and \cdf{///}.

If the evaluation of \cdf{+} is aborted for some reason,
then the values associated with \cdf{/}, \cdf{//}, and \cdf{///} are not updated;
they are updated only if the printing of values is at least begun (though not
necessarily completed).
\end{defun}

As an example of the processing of these variables, consider the
following possible transcript, where \cdf{>} is a prompt by
the top-level loop for user input:
\begin{lisp}
\hskip 12pc\=\kill
>(cons - -)\>;\textrm{Interaction 1} \\
((CONS - -) CONS - -)\>;\textrm{Cute, huh?} \\
 \\
>(values)\>;\textrm{Interaction 2} \\
~~~~~~~~~~~~~~~~~~~~~~~~~~~~~~~~;\textrm{Nothing to print} \\
>(cons 'a 'b)\>;\textrm{Interaction 3} \\
(A . B)\>;\textrm{There is a single value} \\
 \\
>(hairy-loop){\Xcircumflex}G\>;\textrm{Interaction 4} \\
\#\#\# QUIT to top level.\>;\textrm{(User aborts the computation.)} \\
 \\
>(floor 13 4)\>;\textrm{Interaction 5} \\
3\>;\textrm{There are two values} \\
1
\end{lisp}
At this point we have:
\begin{lisp}
\begin{tabular*}{\textwidth}{@{}l@{\extracolsep{\fill}}ll@{}}
+++ \EV\ (cons 'a 'b)&*** \EV\ NIL    &/// \EV\ () \\
++  \EV\ (hairy-loop)&**  \EV\ (A . B)&//  \EV\ ((A . B)) \\
+   \EV\ (floor 13 4)&*   \EV\ 3      &/   \EV\ (3 1)
\end{tabular*}
\end{lisp}

%RUSSIAN
\else

\chapter{Вычислитель}

Механизм, который выполняет Lisp'овые программы, называется вычислитель.
А точнее, вычислитель принимает форму и выполняет расчёты указанные формой.
Этот механизм доступен пользователю через функцию \cdf{eval}.

Вычислитель, как правило, реализован как интерпретатор, который рекурсивно
проходит по заданной форме, выполняя каждый шаг вычисления. Однако такая
реализация не обязательна. Возможно альтернативное поведение, когда вычислитель
сначала полностью компилирует форму в машинновыполняемый код, и затем его
вызывает.
Этот метод практически исключает несовместимости между интерпретируемым и
компилируемым кодом, но и делает механизм \cdf{evalhook} относительно
бесполезным.
Допустимы также разные смешанные стратегии. Все эти методы должны возвращать
одинаковые результаты для правильного кода, и могут возвращать разные ошибки для
неправильного кода.
Например, поведение может отличаться в том, когда раскрывается
макровызов. Поэтому определение макроса не должно зависеть от времени, когда он
раскрывается. Для каждой реализации разработчики должны документировать
стратегию вычисления. 

\section{Вычисление форм}

Функция \cdf{eval} является главным пользовательским интерфейсом к вычислителю.
Для пользовательских отладочных функций в интерпретаторе предусмотрены ловушки. 
Функции \cdf{evalhook} и \cdf{applyhook} предоставляют альтернативные интерфейсы
к механизму вычислителя для использования этих отладочных возможностей.

\begin{defun}[Функция]
eval form

Форма \emph{form} вычисляется в текущем динамическом окружении и нулевом
лексическом. Результатом функции является вычисленное значение для переданной
формы.

Следует отметить, что когда вы записываете вызов к \cdf{eval}, то для переданной
формы происходят \emph{два} уровня вычислений.
Сначала происходит вычисление формы аргумента, как и любого аргумента для любой
функции. Данное вычисление в свою очередь неявно вызывает \cdf{eval}.
Затем происходит вычисление значения аргумента переданного в функцию \cdf{eval}.
Например:
\begin{lisp}
(eval (list 'cdr (car '((quote (a . b)) c)))) \EV\ b
\end{lisp}
Форма аргумента \cd{(list 'cdr (car '((quote (a . b)) c)))} вычисляется в
\cd{(cdr (quote (a . b)))}.
Затем \cdf{eval} вычисляет полученный аргумент и возвращает \cd{b}.

Если необходимо получить динамическое значения для символа, то удобнее
использовать функцию \cdf{symbol-value}.
\end{defun}

\begin{defun}[Переменная]
*evalhook* \\
*applyhook*

Если значение \cdf{*evalhook*} не является {\false}, тогда \cdf{eval} ведёт себя
специальным образом. Не-{/false} значение \cdf{*evalhook*} должно быть функцией,
которая принимает два аргумента, форму и окружение.
Эта функция называется <<функцией-ловушкой для вычислителя>>.
Когда форма должна быть вычислена (любая, даже просто число или символ) неявно или явно с
помощью \cdf{eval}, то вместо вычисления вызывается данная функция с формой в
первом аргументе.
Тогда функция-ловушка несёт ответственность за вычисление формы, и все что она
вернёт будет расценено как результат вычисления этой формы.

Переменная \cdf{*applyhook*} похожа на \cdf{*evalhook*}, но используется, когда
функция должна быть применена к аргументам.
Если значение \cdf{*applyhook*} не {\false}, тогда \cdf{eval} ведёт себя
специальным образом.

Когда функция должна примениться к списку аргументов, то вызывается
функция-ловушка с данной функцией и списком аргументов в качестве параметров.
Выполнение формы доверяется функции-ловушке. То, что она вернёт, будет расценено
как результат вычисления формы.
Функция-ловушка используется для применения обычных функций внутри
\cdf{eval}. Она не используется для вызовов \cdf{apply} или
\cdf{funcall}, таких функций, как \cdf{map} или \cdf{reduce}, или
вызовов функций раскрытия макросов, таких как \cdf{eval} или \cdf{macroexpand}.

Последний аргумент помещаемый в функции-ловушки содержит информацию о
лексическом окружении в формате, который зависит от реализации.
Эти аргументы одинаковы для функций \cdf{evalhook}, \cdf{applyhook} и
\cdf{macroexpand}.

Когда вызывается одна из функций-ловушек, то обе переменные \cdf{*evalhook*} и
\cdf{*applyhook} связываются со значениями {\nil} на время выполнения данных
функций. Это сделано для того, чтобы функция-ловушка не зациклилась.
Функции \cdf{evalhook} и \cdf{applyhook} полезны для выполнения рекурсивных
вычислений и применений (функции) с функцией ловушкой.

Функциональность ловушки предоставляется для облегчения отладки.
Функциональность \cdf{step} реализована с помощью такой ловушки.

Если случается нелокальный выход на верхний уровень Lisp'а, возможно потому, что
ошибка не может быть исправлена, тогда \cdf{*evalhook*} и \cdf{*applyhook*}
в целях безопасности автоматически сбрасываются в {\false}.
\end{defun}

\begin{defun}[Функция]
evalhook form evalhookfn applyhookfn &optional env \\
applyhook function args evalhookfn applyhookfn &optional env

Функции \cdf{evalhook} и \cdf{applyhook} представлены для облегчения
использования функциональности ловушек.

В случае \cdf{evalhook} вычисляется форма \emph{form}.
В случае \cdf{applyhook} функция \emph{function} применяется к списку аргументов
\emph{args}.
В обоих случаях, в процессе выполнения операции переменная \cdf{*evalhook*}
связана с \emph{evalhookfn}, и \cdf{*applyhook*} с \emph{applyhookfn}.
Кроме того, аргумент \emph{env} используется для установки лексического
окружения.
По-умолчанию \emph{env} установлен в нулевое окружение.
The check for a hook function is \emph{bypassed} for the evaluation
of the \emph{form} itself (for \cdf{evalhook}) or for the application
of the \emph{function} to the \emph{args} itself (for \cdf{applyhook}),
but not for subsidiary evaluations and applications
such as evaluations of subforms.  It is this one-shot bypass that makes
\cdf{evalhook} and \cdf{applyhook} so useful. FIXME

Вот пример, очень простой функции трассировки, которая использует возможности
\cdf{evalhook}.
\begin{lisp}
(defvar *hooklevel* 0) \\
 \\
(defun hook (x) \\
~~(let ((*evalhook* 'eval-hook-function)) \\
~~~~(eval x))) \\
 \\
(defun eval-hook-function (form \cd{\&rest} env) \\
~~(let ((*hooklevel* (+ *hooklevel* 1))) \\
~~~~(format *trace-output* "{\Xtilde}\%{\Xtilde}V{\Xatsign}TForm:~~{\Xtilde}S" \\
~~~~~~~~~~~~(* *hooklevel* 2) form) \\
~~~~(let ((values (multiple-value-list \\
~~~~~~~~~~~~~~~~~~~~~(evalhook form \\
~~~~~~~~~~~~~~~~~~~~~~~~~~~~~~~\#'eval-hook-function \\
~~~~~~~~~~~~~~~~~~~~~~~~~~~~~~~{\nil} \\
~~~~~~~~~~~~~~~~~~~~~~~~~~~~~~~env)))) \\
~~~~~~(format *trace-output* "{\Xtilde}\%{\Xtilde}V{\Xatsign}TValue:{\Xtilde}{\Xlbrace} {\Xtilde}S{\Xtilde}{\Xrbrace}" \\
~~~~~~~~~~~~~~(* *hooklevel* 2) values) \\
~~~~~~(values-list values))))
\end{lisp}
Используя этот функционал можно, например, увидеть такую последовательность:
\begin{lisp}
(hook '(cons (floor *print-base* 2) 'b)) \\
~~Form:  (CONS (FLOOR *PRINT-BASE* 2) (QUOTE B)) \\
~~~~Form:  (FLOOR *PRINT-BASE* 3) \\
~~~~~~Form:  *PRINT-BASE* \\
~~~~~~Value: 10 \\
~~~~~~Form:  3 \\
~~~~~~Value: 3 \\
~~~~Value: 3 1 \\
~~~~Form:  (QUOTE B) \\
~~~~Value: B \\
~~Value: (3 . B) \\
(3 . B)
\end{lisp}
\end{defun}

\begin{defun}[Функция]
constantp object

Если предикат \cdf{constantp} для объекта \emph{object} истинен, то данный
объект, когда рассматривается как вычисляемая форма, всегда вычисляется в одно и
то же значение.
Константные объекты включают самовычисляемые объекты, такие как числа, строковые
символы, строки, битовые вектора, ключевые символы, а также символы констант,
определённых с помощью \cdf{defconstant}, \cdf{nil}, \cdf{t} и \cdf{pi}.
В дополнение, список, у которого \emph{car} элемент равен \cdf{quote}, например
\cd{(quote foo)}, также является константным объектом.

Если \cdf{constantp} для объекта \emph{object} ложен, то этот объект,
рассматриваемый как форма, может не всегда вычисляться в одно и то же
значение.
\end{defun}

\section{Цикл взаимодействия с пользователем}

С Lisp'ом можно работать через специальный цикл вида
<<\cdf{read}-\cdf{eval}-\cdf{print} и сначала>>.
Изменять состояние Lisp системы можно вызовом в этом цикле действий, имеющих
побочные эффекты.  

Точное определение такого цикла для Common Lisp'а не указывается здесь
специально, оставляя разработчикам поле для творчества.
Например, они могут сделать командную строку, или простой текстовый редактор,
или сложный графический интерфейс. Разработчики могут предоставлять явный запрос
ввода, или (как в MacLisp'е) не загромождать экран подсказками.

Цикл взаимодействия с пользователем должен ловить все исключения и изящно их
обрабатывать. Он должен также выводить все результаты вычисления форм, возможно
в отдельных строках. Если форма вернула ноль значений, это также должно быть
отображено.

Следующие переменные управляются циклом взаимодействия с пользователем, для
удобства работы, например в случае, если пользователь забыл сохранить
интересное введённое выражение или выведенное значение. (Следует отметить, что
имена некоторых этих переменных нарушают нотацию глобальных переменных в
использовании звёздочек в начале и конце имени.) Эти переменные в основном
используются для взаимодействия с пользователем, поэтому имеют короткие
имена. Использование этих переменных в программах следует избегать.

\begin{defun}[Переменная]
+ \\
++ \\
+++ 

Когда форма вычисляется в цикле взаимодействия, переменная \cdf{+} связывается с
предыдущим прочитанным выражением.
Переменная \cdf{++} хранит предыдущее относительно значения \cdf{+} (то есть,
форма вычисленная два шага назад), и \cdf{+++} хранит предыдущее значение
относительно \cdf{++}.
\end{defun}

\begin{defun}[Переменная]
-

Когда формы выполняется в цикле взаимодействия, переменная \cdf{-} связана с
этой формой. Это значение, которое получит переменная \cdf{+} на следующей
итерации цикла.
\end{defun}

\begin{defun}[Переменная]
* \\
** \\
***

Во время вычисления формы в цикле взаимодействия, переменная \cdf{*} связана с
результатом выполнения предыдущей формы, то есть значения формы, которая
хранится в \cdf{+}.
Если результат той формы содержал несколько значений, в \cdf{*} будет только
первое. Если было возвращено ноль значений, \cdf{*} будет содержать {\nil}.
Переменная \cdf{**} хранит предыдущее значение относительно \cdf{*}. То есть,
результат вычисления формы из \cdf{**}. \cdf{***} хранит предыдущее значение
относительно \cdf{***}.

Если выполнение \cdf{+} было по каким-то причинам прервано, тогда значения
\cdf{*}, \cdf{**} и \cdf{***} не меняются.
Они изменяются, если вывод значений как минимум начался (необязательно, чтобы он
закончился).
\end{defun}

\begin{defun}[Переменная]
/ \\
// \\
///

Во время вычисления формы в цикле взаимодействия,
переменная \cdf{/} связана со списком результатов выведенных на предыдущей
итерации. То есть это список всех значений выполнения формы \cdf{+}.
Значение \cdf{*} должно всегда совпадать со значением \emph{car} элемента
значения \cdf{/}.
Переменная \cdf{//} хранит предыдущий список значений относительно \cdf{/} (то
есть, результат вычисленный две итерации назад), и \cdf{///} содержит предыдущий
список значений относительно \cdf{///}. Таким образом, значение \cdf{**} должно
всегда совпадать со значением \emph{car} элемента списка \cdf{//}, а \cdf{***} с
с \emph{car} элементом \cdf{///}. 

Если выполнение \cdf{+} было по каким-то причинам прервано, тогда значения
\cdf{/}, \cdf{//} и \cdf{///} не меняются.
Они изменяются, если вывод значений как минимум начался (необязательно, чтобы он
закончился).
\end{defun}

В качестве примера работы с этими переменным, рассмотрим следующую возможную
работу с системой, где \cdf{>} приглашение ввода:
\begin{lisp}
\hskip 12pc\=\kill
>(cons - -)\>;\textrm{Итерация 1} \\
((CONS - -) CONS - -)\>;\textrm{Очаровашка?} \\
 \\
>(values)\>;\textrm{Итерация 2} \\
~~~~~~~~~~~~~~~~~~~~~~~~~~~~~~~~;\textrm{Ничего не выводится} \\
>(cons 'a 'b)\>;\textrm{Итерация 3} \\
(A . B)\>;\textrm{Это одно значение} \\
 \\
>(hairy-loop){\Xcircumflex}G\>;\textrm{Итерация 4} \\
\#\#\# QUIT to top level.\>;\textrm{(Пользователь прервал вычисления.)} \\
 \\
>(floor 13 4)\>;\textrm{Итерация 5} \\
3\>;\textrm{Вернулось два значения} \\
1
\end{lisp}
В этой точке мы имеем:
\begin{lisp}
\begin{tabular*}{\textwidth}{@{}l@{\extracolsep{\fill}}ll@{}}
+++ \EV\ (cons 'a 'b)&*** \EV\ NIL    &/// \EV\ () \\
++  \EV\ (hairy-loop)&**  \EV\ (A . B)&//  \EV\ ((A . B)) \\
+   \EV\ (floor 13 4)&*   \EV\ 3      &/   \EV\ (3 1)
\end{tabular*}
\end{lisp}

\fi        % Includes stack-crawling stuff
%Part{Stream, Root = "CLM.MSS"}
%%%Chapter of Common Lisp Manual.  Copyright 1984, 1988, 1989 Guy L. Steele Jr.

\clearpage\def\pagestatus{FINAL PROOF}

\ifx \rulang\Undef

\chapter{Streams}
\label{STREAM}

Streams are objects that serve as sources or sinks of data.
Character streams produce or absorb characters;
binary streams produce or absorb integers.
The normal action of a Common Lisp system is to read characters from
a character input stream, parse the characters as representations
of Common Lisp data objects, evaluate each object (as a form) as it is read, and
print representations of the results of
evaluation to an output character stream.

Typically streams are connected to files or to an interactive terminal.
Streams, being Lisp objects, serve as the ambassadors of external
devices by which input/output is accomplished.

A stream, whether a character stream or a binary
stream, may be input-only, output-only, or bidirectional.
What operations may be performed on a stream depends on which of
the six types of stream it is.

\section {Standard Streams}

There are several variables whose values are streams used by many
functions in the Lisp system.  These variables and their uses are
listed here.  By convention, variables that are expected to hold a
stream capable of input have names ending with \cdf{-input}, and
variables that are expected to hold a
stream capable of output have names ending with \cdf{-output}.
Variables expected to hold a
bidirectional stream have names ending with \cdf{-io}.

\begin{defun}[Variable]
*standard-input*

In the normal Lisp top-level loop, input is read from
\cdf{*standard-input*} (that is, whatever stream is the value of the global
variable \cdf{*standard-input*}).  Many input functions, including
\cdf{read} and \cdf{read-char}, take a stream argument that defaults to
\cdf{*standard-input*}.
\end{defun}

\begin{defun}[Variable]
*standard-output*

In the normal Lisp top-level loop, output is sent to
\cdf{*standard-output*} (that is, whatever stream is the value of the global
variable \cdf{*standard-output*}).  Many output functions, including
\cdf{print} and \cdf{write-char}, take a stream argument that defaults
to \cdf{*standard-output*}.
\end{defun}

\begin{defun}[Variable]
*error-output*

The value of \cdf{*error-output*} is a stream to which error messages
should be sent.  Normally this is the same as \cdf{*standard-output*},
but \cdf{*standard-output*} might be bound to a file and \cdf{*error-output*}
left going to the terminal or to a separate file of error messages.
\end{defun}

\begin{defun}[Variable]
*query-io*

The value of \cdf{*query-io*} is a stream to be used when
asking questions of the user.  The question should be output to this
stream, and the answer read from it.  When
the normal input to a program may be coming from a file, questions such
as ``Do you really want to delete all of the files in your directory?'' should
nevertheless
be sent directly to the user; and the answer should come from the user,
not from the data file.  For such purposes \cdf{*query-io*} should be
used instead of \cdf{*standard-input*} and \cdf{*standard-output*}.
\cdf{*query-io*} is used by such functions
as \cdf{yes-or-no-p}.
\end{defun}

\begin{defun}[Variable]
*debug-io*

The value of \cdf{*debug-io*} is a stream to be used for interactive
debugging purposes.  This is often the same as the value of \cdf{*query-io*},
but need not be.
\end{defun}

\begin{defun}[Variable]
*terminal-io*

The value of \cdf{*terminal-io*} is ordinarily
the stream that connects to the user's console.
Typically, writing to this stream would cause the output to appear
on a display screen, for example, and reading from the stream would
accept input from a keyboard.

It is intended
that standard input functions such as \cdf{read} and \cdf{read-char},
when used with this stream, would cause ``echoing'' of the input
into the output side of the stream.  (The means by which this is
accomplished are of course highly implementation-dependent.)
\end{defun}

\begin{defun}[Variable]
*trace-output*

The value of \cdf{*trace-output*} is the stream on which the \cdf{trace}
function prints its output.
\end{defun}

The variables
\cdf{*standard-input*}, \cdf{*standard-output*},
\cdf{*error-output*},
\cdf{*trace-output*},
\cdf{*query-io*}, and \cdf{*debug-io*}
are initially bound to synonym streams that pass all
operations on to the stream that is the value of \cdf{*terminal-io*}.
(See \cdf{make-synonym-stream}.)
Thus any operations performed on those streams will go to the terminal.

\begin{new}
X3J13 voted in January 1989
\issue{STANDARD-INPUT-INITIAL-BINDING}
to replace the requirements of the preceding
paragraph with the following new requirements:

The seven standard stream variables,
\cdf{*standard-input*}, \cdf{*standard-output*}, \cdf{*query-io*},
\cdf{*debug-io*}, \cdf{*terminal-io*},
\cdf{*error-output*}, and
\cdf{*trace-output*},
are initially bound to open streams.  (These will be called
\emph{the standard initial streams}.)

The streams that are the initial values of
\cdf{*standard-input*}, \cdf{*query-io*}, \cdf{*debug-io*}, and \cdf{*terminal-io*}
must support input.

The streams that are the initial values of
\cdf{*standard-output*},
\cdf{*error-output*},
\cdf{*trace-output*}, \cdf{*query-io*}, \cdf{*debug-io*}, and \cdf{*terminal-io*}
must support output.

None of the standard initial streams (including the one to which
\cdf{*terminal-io*} is initially bound) may be a synonym, either directly
or indirectly, for any of the standard stream variables
except \cdf{*terminal-io*}.  For example, the initial value of
\cdf{*trace-output*} may be a synonym stream for \cdf{*terminal-io*}
but not a synonym stream for \cdf{*standard-output*} or \cdf{*query-io*}.
(These are examples of direct synonyms.)  As another example,
\cdf{*query-io*} may be a two-way stream or echo stream whose
input component is a synonym for \cdf{*terminal-io*},
but its input component may not be a synonym for \cdf{*standard-input*}
or \cdf{*debug-io*}.  (These are examples of indirect synonyms.)

Any or all of the standard initial streams may be direct or indirect
synonyms for one or more common implementation-dependent streams.
For example, the standard initial streams might all be synonym streams
(or two-way or echo streams whose components are synonym streams)
to a pair of hidden terminal input and output streams maintained by
the implementation.

Part of the intent of these rules is to ensure that it is always safe
to bind any standard stream variable to the value of any other
standard stream variable (that is, unworkable circularities are
avoided) without unduly restricting implementation flexibility.
\end{new}

No user program should ever change the value of \cdf{*terminal-io*}.  A program
that wants (for example) to divert output to a file should do so by binding
the value of \cdf{*standard-output*}; that way error messages sent to
\cdf{*error-output*} can still get to the user by going through \cdf{*terminal-io*},
which is usually what is desired.

\section {Creating New Streams}

Perhaps the most important constructs for creating new streams
are those that open files; see \cdf{with-open-file} and \cdf{open}.
The following functions construct streams without reference to a file system.

\begin{defun}[Function]
make-synonym-stream symbol

\cdf{make-synonym-stream} creates and returns
a synonym stream.
Any operations on the new stream will be performed on the stream
that is then the value of the dynamic variable named by the \emph{symbol}.
If the value of the variable should change or be bound,
then the synonym stream will operate on the new stream.

\begin{new}
X3J13 voted in January 1989
\issue{STREAM-ACCESS}
to specify that the result of
\cd{make-\discretionary{}{}{}synonym-stream} is always a stream of type \cdf{synonym-stream}.
Note that the type of a synonym stream is \emph{always} \cdf{synonym-stream},
regardless of the type of the stream for which it is a synonym.
\end{new}
\end{defun}

\begin{defun}[Function]
make-broadcast-stream &rest streams

This returns a stream that works only in the output direction.  Any output
sent to this stream will be sent to all of the \emph{streams} given.
The set of
operations that may be performed on the new stream is the intersection
of those for the given streams.  The results returned by a stream
operation are the values resulting from
performing the operation on the last stream in \emph{streams}; the
results of performing the operation on all preceding streams are
discarded.
If no \emph{streams} are given as arguments, then the result
is a ``bit sink''; all output to the resulting stream is discarded.
\begin{new}
X3J13 voted in January 1989
\issue{STREAM-ACCESS}
to specify that the result of
\cd{make-broadcast-stream} is always a stream of type \cdf{broadcast-stream}.
\end{new}
\end{defun}

\begin{defun}[Function]
make-concatenated-stream &rest streams

This returns a stream that works only in the input direction.
Input is taken from the first of the \emph{streams} until it reaches
end-of-file; then that stream is discarded, and input is taken
from the next of the \emph{streams}, and so on.  If no arguments
are given, the result is a stream with no content; any input attempt
will result in end-of-file.
\begin{new}
X3J13 voted in January 1989
\issue{STREAM-ACCESS}
to specify that the result of
\cd{make-concatenated-stream} is always a stream of type \cdf{concatenated-stream}.
\end{new}
\end{defun}

\begin{defun}[Function]
make-two-way-stream input-stream output-stream

This returns a bidirectional stream that gets its input from \emph{input-stream}
and sends its output to \emph{output-stream}.
\begin{new}
X3J13 voted in January 1989
\issue{STREAM-ACCESS}
to specify that the result of
\cd{make-two-way-stream} is always a stream of type \cdf{two-way-stream}.
\end{new}
\end{defun}

\begin{defun}[Function]
make-echo-stream input-stream output-stream

This returns a bidirectional stream that gets its input from \emph{input-stream}
and sends its output to \emph{output-stream}.  In addition, all
input taken from \emph{input-stream} is echoed to \emph{output-stream}.
\begin{new}
X3J13 voted in January 1989
\issue{STREAM-ACCESS}
to specify that the result of
\cd{make-echo-stream} is always a stream of type \cdf{echo-stream}.
\end{new}

\begin{new}
X3J13 voted in January 1989
\issue{PEEK-CHAR-READ-CHAR-ECHO}
to clarify the interaction of
\cdf{read-char}, \cdf{unread-char}, and \cdf{peek-char} with echo streams.
(See the descriptions of those functions for details.)

X3J13 explicitly noted that the bidirectional streams that are the initial
values of \cdf{*query-io*}, \cdf{*debug-io*}, and \cdf{*terminal-io*},
even though they may have some echoing behavior, conceptually
are not necessarily the products of calls to \cdf{make-echo-stream}
and therefore are not subject to the new rules about echoing on echo
streams.  Instead, these initial interactive streams may have
implementation-dependent echoing behavior.
\end{new}
\end{defun}

\begin{defun}[Function]
make-string-input-stream string &optional start end

This returns an input stream.
The input stream will supply, in order, the characters in the substring
of \emph{string} delimited by \emph{start} and \emph{end}; after the last
character has been supplied, the stream will then be at end-of-file.

\begin{new}
X3J13 voted in January 1989
\issue{STREAM-ACCESS}
to specify that the result of
\cd{make-string-input-stream} is always a stream of type \cdf{string-stream}.
\end{new}
\end{defun}

\begin{newer}
X3J13 voted in June 1989 \issue{MORE-CHARACTER-PROPOSAL}
to let \cdf{make-string-output-stream} take an \cd{:element-type} argument.

\begin{defun}[Function]
make-string-output-stream &key :element-type

This returns an output stream that will 
accumulate all output given it for the benefit of the function
\cdf{get-output-stream-string}.

The \cd{:element-type} argument specifies what characters
must be accepted by the created stream.  If the \cd{:element-type} argument
is omitted, the created stream must accept all characters.

X3J13 voted in January 1989
\issue{STREAM-ACCESS}
to specify that the result of
\cdf{make-string-output-stream} is always a stream of type \cdf{string-stream}.
\end{defun}
\end{newer}


\begin{defun}[Function]
get-output-stream-string string-output-stream

Given a stream produced by \cdf{make-string-output-stream}, this
returns a string containing all the characters output to the stream so far.
The stream is then reset; thus each call to \cdf{get-output-stream-string}
gets only the characters since the last such call (or the creation
of the stream, if no such previous call has been made).
\end{defun}

\begin{defmac}
with-open-stream (var stream) {declaration}* {\,form}*

The form \emph{stream} is evaluated and must produce a stream.
The variable \emph{var} is bound with the stream as its value,
and then the forms of the body are executed
as an implicit \cdf{progn}; the results of evaluating
the last form are returned as the value of the \cdf{with-open-stream} form.
The stream
is automatically closed on exit from the \cdf{with-open-stream} form,
no matter whether the exit is normal or abnormal;
see \cdf{close}.
The stream should be regarded as having dynamic extent.
\begin{new}
X3J13 voted in January 1989
\issue{STREAM-ACCESS}
to specify that the stream created by
\cdf{with-open-stream} is always of type \cdf{file-stream}.
\end{new}
\end{defmac}

\begin{defmac}
with-input-from-string (var string {keyword value}*)
      {declaration}* {\,form}*

The body is executed as an implicit \cdf{progn} with the variable \emph{var}
bound to a character input stream that supplies successive characters from
the value of the form \emph{string}.  \cdf{with-input-from-string}
returns the results from the last \emph{form} of the body.

The input stream is automatically closed on exit from
the \cd{with-input-from-string} form,
no matter whether the exit is normal or abnormal.
The stream should be regarded as having dynamic extent.

\begin{new}
X3J13 voted in January 1989
\issue{STREAM-ACCESS}
to specify that the stream created by
\cdf{with-input-from-string} is always of type \cdf{string-stream}.
\end{new}

The following keyword options may be used:
\begin{quotation}
\begin{flushdesc}
\item[\cd{:index}]
The form after the \cd{:index} keyword should be a \emph{place}
acceptable to \cdf{setf}.  If the \cdf{with-input-from-string} form
is exited normally, then the \emph{place} will have stored into it the
index into the \emph{string} indicating the first character not read
(the length of the string if all characters were used).
The \emph{place} is not updated as reading progresses, but only at the
end of the operation. 

\item[\cd{:start}]
The \cd{:start} keyword takes an argument indicating, in the manner
usual for sequence functions, the beginning of
a substring of \emph{string} to be used.

\item[\cd{:end}]
The \cd{:end} keyword takes an argument indicating, in the manner
usual for sequence functions, the end of
a substring of \emph{string} to be used.
\end{flushdesc}
\end{quotation}

Here is an example of the use of \cdf{with-input-from-string}:
\begin{lisp}
(with-input-from-string (s "Animal Crackers" :index j :start 6) \\
~~(read s)) \EV\ crackers
\end{lisp}
As a side effect, the variable \cdf{j} is set to \cd{15}.

\begin{new}
X3J13 voted in January 1989
\issue{MAPPING-DESTRUCTIVE-INTERACTION}
to restrict user side effects; see section \ref{STRUCTURE-TRAVERSAL-SECTION}.
\end{new}
\end{defmac}

\begin{newer}
X3J13 voted in June 1989 \issue{MORE-CHARACTER-PROPOSAL}
to let \cdf{with-output-to-string} take an \cd{:element-type} argument.

\begin{defmac}
with-output-to-string (var [string [\!:element-type! type]])
                      {declaration}* {\,form}*

One may specify \cdf{nil} instead of a string as the \emph{string}
and use the \cd{:element-type} argument to specify what characters
must be accepted by the created stream.  If no \emph{string} argument
is provided, or if it is \cdf{nil} and no \cd{:element-type} is specified,
the created stream must accept all characters.

X3J13 voted in October 1988
\issue{WITH-OUTPUT-TO-STRING-APPEND-STYLE}
to specify that
if \emph{string} is specified, it must be a string with a fill pointer;
the output is incrementally appended to the string (as if by use of
\cdf{vector-push-extend}).

In this way output cannot be accidentally lost.  This change makes
\cd{with-output-to-string} behave in the same way that \cdf{format} does
when given a string as its first argument.

X3J13 voted in January 1989
\issue{STREAM-ACCESS}
to specify that the stream created by
\cdf{with-output-to-string} is always of type \cdf{string-stream}.

X3J13 voted in January 1989
\issue{MAPPING-DESTRUCTIVE-INTERACTION}
to restrict user side effects; see section \ref{STRUCTURE-TRAVERSAL-SECTION}.
\end{defmac}
\end{newer}

\section {Operations on Streams}

This section contains discussion of only those operations that
are common to all streams.  Input and output is rather complicated
and is discussed separately in chapter~\ref{IO}.
The interface between streams and the file system is discussed
in chapter~\ref{FILES}.

\begin{defun}[Function]
streamp object

\cdf{streamp} is true if its argument is a stream,
and otherwise is false.
\begin{lisp}
(streamp x) \EQ\ (typep x 'stream)
\end{lisp}

\begin{new}
X3J13 voted in January 1989
\issue{CLOSED-STREAM-OPERATIONS}
to specify that \cdf{streamp} is unaffected
by whether its argument, if a stream, is open or closed.  In either case
it returns true.
\end{new}
\end{defun}

\begin{newer}
\begin{defun}[Function]
open-stream-p stream

X3J13 voted in January 1989 \issue{STREAM-ACCESS}
to add the predicate \cdf{open-stream-p}.
It is true if its argument (which must be a stream)
is open, and otherwise is false.

A stream is always created open; it remains open until closed
with the \cdf{close} function.  The macros \cdf{with-open-stream},
\cdf{with-input-from-string}, \cdf{with-output-to-string},
and \cdf{with-open-file} automatically close the created stream
as control leaves their bodies, in effect imposing dynamic extent
on the openness of the stream.
\end{defun}
\end{newer}

\begin{defun}[Function]
input-stream-p stream

This predicate is true if its argument (which must be a stream) can handle
input operations, and otherwise is false.
\end{defun}

\begin{defun}[Function]
output-stream-p stream

This predicate is true if its argument (which must be a stream) can handle
output operations, and otherwise is false.
\end{defun}

\begin{defun}[Function]
stream-element-type stream

A type specifier is returned to indicate what objects
may be read from or written to the argument \emph{stream}, which must be a stream.
Streams created by \cdf{open} will have an element type
restricted to a subset of \cdf{character} or \cdf{integer},
but in principle a stream may conduct transactions using any
Lisp objects.
\end{defun}

\begin{defun}[Function]
close stream &key :abort

The argument must be a stream.
The stream is closed.  No further input/output operations may be performed
on it.  However, certain inquiry operations may still be performed,
and it is permissible to close an already closed stream.

\begin{newer}
X3J13 voted in January 1989
\issue{CLOSED-STREAM-OPERATIONS}
and revised the vote in March 1989
to specify that if \cdf{close} is called
on an open stream, the stream is closed and \cdf{t} is returned;
but if \cdf{close} is called on a closed stream, it succeeds without
error and returns an unspecified value.
(The rationale for not specifying the value returned for a closed stream
is that in some implementations closing certain streams does not really
have an effect on them---for example, closing the \cdf{*terminal-io*}
stream might not ``really'' close it---and it is not desirable to force
such implementations to keep otherwise unnecessary state.  Portable programs
will of course not rely on such behavior.)


X3J13 also voted in January 1989 to specify exactly which inquiry
functions may be applied to closed streams:
\begin{tabbing}
\begin{tabular*}{\textwidth}{@{\extracolsep{\fill}}lll@{}}
\cdf{streamp} & \cdf{pathname-host} & \cdf{namestring} \\
\cdf{pathname} & \cdf{pathname-device} & \cdf{file-namestring} \\
\cdf{truename} & \cdf{pathname-directory} & \cdf{directory-namestring} \\
\cdf{merge-pathnames} & \cdf{pathname-name} & \cdf{host-namestring} \\
\cdf{open} & \cdf{pathname-type} & \cdf{enough-namestring} \\
\cdf{probe-file} & \cdf{pathname-version} & \cdf{directory} \\
\end{tabular*}
\end{tabbing}
See the individual descriptions of these functions for more information
on how they operate on closed streams.
\end{newer}

\begin{new}
X3J13 voted in January 1989
\issue{CLOSE-CONSTRUCTED-STREAM}
to clarify the effect of closing various
kinds of streams.  First some terminology:
\begin{itemize}
\item
A \emph{composite} stream is one that was returned by a call to
\cd{make-synonym-stream},
\cdf{make-broadcast-stream},
\cdf{make-concatenated-stream},
\cd{make-two-way-stream},
or \cd{make-echo-stream}.

\item
The \emph{constituents} of a composite stream are the streams that were given
as arguments to the function that constructed it or, in the case of
\cd{make-synonym-stream}, the stream that is the \cdf{symbol-value} of
the symbol that was given as an argument.  (The constituent of
a synonym stream may therefore vary over time.)

\item
A \emph{constructed} stream is either a composite stream or one returned
by a call to \cdf{make-string-input-stream}, \cdf{make-string-output-stream},
\cd{with-input-from-string}, or
\cdf{with-output-to-string}.
\end{itemize}

The effect of applying \cdf{close} to a constructed stream is to close
that stream only.  No input/output operations are permitted on the
constructed stream once it has been closed (though certain inquiry
functions are still permitted, as described above).

Closing a composite stream has no effect on its constituents;
any constituents that are open remain open.

If a stream created by \cdf{make-string-output-stream} is closed,
the result of then applying \cdf{get-output-stream-string} to the
stream is unspecified.
\end{new}

If the \cd{:abort} parameter is not {\false} (it defaults to {\false}), it
indicates an abnormal termination of the use of the stream.  An attempt
is made to clean up any side effects of having created the stream in the
first place.  For example, if the stream performs output to a file
that was newly created when the stream was created, then if possible the
file is deleted and any previously existing file is not superseded.
\end{defun}

\begin{new}
X3J13 voted in January 1989
\issue{STREAM-ACCESS}
to add the following accessor functions
for obtaining information about streams.


\begin{defun}[Function]
broadcast-stream-streams broadcast-stream

The argument must be of type \cdf{broadcast-stream}.
A list of the constituent output streams (whether open or not) is returned.

\end{defun}


\begin{defun}[Function]
concatenated-stream-streams concatenated-stream

The argument must be of type \cdf{concatenated-stream}.
A list of constituent streams (whether open or not) is returned.
This list represents the ordered set of input streams from which
the concatenated stream may yet read; the stream from which it is
currently reading is first in the list.  The list may be empty
if no more streams remain to be read.
\end{defun}


\begin{defun}[Function]
echo-stream-input-stream echo-stream \\
echo-stream-output-stream echo-stream

The argument must be of type \cdf{echo-stream}.
The function \cdf{echo-stream-input-stream} returns the constituent
input stream; \cdf{echo-stream-output-stream} returns the constituent
output stream.
\end{defun}


\begin{defun}[Function]
synonym-stream-symbol synonym-stream

The argument must be of type \cdf{synonym-stream}.  This function returns
the symbol for whose value the \emph{synonym-stream} is a synonym.
\end{defun}


\begin{defun}[Function]
two-way-stream-input-stream two-way-stream \\
two-way-stream-output-stream two-way-stream

The argument must be of type \cdf{two-way-stream}.
The function \cdf{two-way-stream-input-stream} returns the constituent
input stream; \cdf{two-way-stream-output-stream} returns the constituent
output stream.
\end{defun}
\end{new}

\begin{newer}
\begin{defun}[Function]
interactive-stream-p stream

X3J13 voted in June 1989 \issue{STREAM-CAPABILITIES} to add the
predicate \cdf{interactive-stream-p}, which returns \cdf{t}
if the \emph{stream\/} is interactive and otherwise returns \cdf{nil}.
A \cd{type-error} error is signalled if the argument is not of type \cdf{stream}.

The precise meaning of \cdf{interactive-stream-p} is implementation-dependent
and may depend on the underlying operating system.
The intent is to distinguish between interactive and batch (background,
command-file) operations.  Some characteristics that might
distinguish a stream as interactive:
\begin{itemize}
\item The stream is connected to a person (or the equivalent)
in such a way that the program can prompt for information and
expect to receive input that might depend on the prompt.
\item The program is expected to prompt for input and to support
``normal input editing protocol'' for that operating environment.
\item A call to \cdf{read-char} might hang waiting for the user to type something
rather than quickly returning a character or an end-of-file
indication.
\end{itemize}
The value of \cdf{*terminal-io*} might or might not be interactive.
\end{defun}
\end{newer}

\begin{newer}
\begin{defun}[Function]
stream-external-format stream

X3J13 voted in June 1989 \issue{MORE-CHARACTER-PROPOSAL} to add the
function \cdf{stream-external-format}, which returns a
specifier for the implementation-recognized scheme used for
representing characters in the argument \emph{stream}.
See the \cd{:external-format} argument to \cdf{open}.
\end{defun}
\end{newer}

%RUSSIAN
\else

\chapter{Потоки}
\label{STREAM}

Потоки является объектами, которые служат в качестве источников или получателей
данных.
Потоки символов (или символьные потоки) возвращают или принимают строковые
символы.
Бинарные потоки возвращают или принимают целые числа.
Обычное действие Common Lisp системы заключается в чтении символов из
символьного входного потока, распознавании символов как представлений Common
Lisp'овых объектов данных, вычислении каждого объекта (как формы) и выводе
результата в выходной символьный поток.

Обычно потоки соединены с файлами или интерактивными терминалами. 
Потоки, будучи Lisp'овыми объектами, служат соединителями со внешними
устройствами, с помощью которых осуществляется ввод/вывод информации.

Потоки, символьные или бинарные, могут быть только для чтения, только для
записи, или для чтения и записи.
Какие действия могут производиться над потоком зависит от того, к какому из шести
типов он принадлежит.

\section{Стандартные потоки}

В Lisp системе есть несколько переменных, значения которых является потоками,
используемыми большим количеством функций. Эти переменные и их использование
описаны ниже. По соглашению, переменные, которые содержат поток для чтения,
имеют имена заканчивающиеся на \cd{-input}, и переменные, которые содержат поток
для записи, имеют имена, заканчивающиеся на \cd{-output}.
Имена переменных, содержащих потоки и для чтения, и для записи, заканчиваются на
\cd{-io}.

\begin{defun}[Переменная]
*standard-input*

В обычном Lisp'овом цикле взаимодействия с пользователем, входные данные
читаются из \cdf{*standard-input*} (то есть, из потока, который является
значением глобальной переменной \cdf{*standard-input*}). Большинство функций,
включая \cdf{read} и \cdf{read-char}, принимают аргумент --- поток, который
по-умолчанию \cdf{*standard-input*}. 
\end{defun}

\begin{defun}[Переменная]
*standard-output*

В обычном Lisp'овом цикле взаимодействия с пользователем, выходные данные
посылаются в \cdf{*standard-output*} (то есть, в поток, который является
значением глобальной переменной \cdf{*standard-output*}). Большинство функций,
включая \cdf{print} и \cdf{write-char}, принимают аргумент -- поток, который
по-умолчанию \cdf{*standard-output*}.
\end{defun}

\begin{defun}[Переменная]
*error-output*

Значение \cdf{*error-output*} является потоком, в который должны посылаться
сообщения об ошибках. Обычно значение совпадает с \cdf{*standard-output*}, но
\cdf{*standard-output*} может быть связан с файлов и \cdf{*error-output*}
остаётся направленной на терминал или отдельный файл для сообщений об ошибках.
\end{defun}

\begin{defun}[Переменная]
*query-io*

Значение \cdf{*query-io*} является потоком, используемым, когда необходимо
получить от пользователя ответ на некоторый вопрос. Вопрос должен быть выведен в
этот поток, и ответ из него прочитан. Когда входной поток для программы может
производится из файла, вопрос <<Вы действительно хотите удалить все файлы в
вашей директории?>> никогда не должен посылаться напрямую к пользователю. И
ответ должен прийти от пользователя, а не из данных файла.
Поэтому в этих целях, вместо \cdf{*standard-input*} и \cdf{*standard-output*},
должен использоваться \cdf{*query-io*} с помощью функции \cdf{yes-or-no-p}.
\end{defun}

\begin{defun}[Переменная]
*debug-io*

Значение \cdf{*debug-io*} является потоком, используемым для интерактивной
отладки. Часто может совпадать с \cdf{*query-io*}, но это необязательно.
\end{defun}

\begin{defun}[Переменная]
*terminal-io*

Значение \cdf{*terminal-io*} является потоком, который соединён с
пользовательской консолью. Обычно, запись в этот поток выводит данные на экран,
например, а чтение из потока осуществляет чтение ввода с клавиатуры.

Когда стандартные функции, такие как \cdf{read} и \cdf{read-char} используются с
этим потоком, то происходит копирование входных данных обратно в поток
или <<эхо>>. (Способ, с помощью которого это происходит, зависит от реализации.)
\end{defun}

\begin{defun}[Переменная]
*trace-output*

Значение \cdf{*trace-output*} является потоком, в который функция \cdf{trace}
выводит информацию.
\end{defun}

Переменные \cdf{*standard-input*}, \cdf{*standard-output*},
\cdf{*error-output*},
\cdf{*trace-output*},
\cdf{*query-io*} и \cdf{*debug-io*}
первоначально связаны с потоками-синонимами, которые направляют все операции в
поток \cdf{*terminal-io*}.
(Смотрите \cdf{make-synonym-stream}.)
Таким образом все проделанные операции на этих потоках отобразятся на терминале.

Пользовательская программа не должна изменять значение
\cdf{*terminal-io*}. Программа, которая, например, хочет перенаправить вывод в файл,
должна изменить значение переменной \cdf{*standard-output*}. В таком случае,
сообщения об ошибках будут продолжать посылаться в \cdf{*error-output*}, а
следовательно в \cdf{*terminal-io*}, и пользователи сможет их увидеть.

\section {Создание новых потоков}

Пожалуй самые важные конструкции для создания новых потоков это то, которые
открывают файлы. Смотрите \cdf{with-open-file} и \cdf{open}.
Следующие функции создают потоки без ссылок на файловую систему.

\begin{defun}[Функция]
make-synonym-stream symbol

\cdf{make-synonym-stream} создаёт и возвращает поток-синоним.
Любые операции на новом потоке будут проделаны на потоке, являющемся значением
динамической переменной с именем \emph{symbol}.
Если значение этой переменной изменится или будет пересвязано, то поток-синоним
будет воздействовать на новый установленный поток.
\end{defun}

\begin{defun}[Функция]
make-broadcast-stream &rest streams

Эта функция возвращает поток, который работает только для записи. Любая выходная
информация, посланная в этот поток, будет отослана в все указанные потоки
\emph{streams}.
Множество операций, которые могут быть выполнены на новом потоке, является
пересечением множеств операций для указанных потоков. Результаты, возвращаемые
операциями над новым потоком, являются результатами возвращёнными операциями на
последнем потоке из списка \emph{streams}.
Результаты полученные в ходе выполнения функции над всеми, кроме последнего,
потоками игнорируются.
Если не было передано ни одного потока в аргументе \emph{streams}, тогда
результат является <<кусочком клоаки>>. Вся выводимая информация будет
игнорироваться.
\end{defun}

\begin{defun}[Функция]
make-concatenated-stream &rest streams

Данная функция возвращает поток, который работает только для чтения.
Входная информация берётся из первого потоки из списка \emph{streams} пока
указатель не достигнет конца-файла end-of-file, затем данный поток
откладывается, и входная информация берётся из следующего, и так далее. Если
список потоков \emph{stream} был пуст, то возвращается поток без
содержимого. Любая попытка чтения будет возвращать конец-файла end-of-file. 
\end{defun}

\begin{defun}[Функция]
make-two-way-stream input-stream output-stream

Данная функция возвращает поток для чтения и записи, который входную информацию
получает из \emph{input-stream} и посылает выходную информацию в \emph{output-stream}.
\end{defun}

\begin{defun}[Функция]
make-echo-stream input-stream output-stream

Данная функция возвращает поток для чтения и записи, который получает входную
информацию из \emph{input-stream} и отсылает выходную в \emph{output-stream}. В
дополнение, входная информация посылается в \emph{output-stream} (эхо).
\end{defun}

\begin{defun}[Функция]
make-string-input-stream string &optional start end

Данная функция возвращает поток для чтения.
Данный поток последовательно будет сохранять строковые символы в подстроке в
строке \emph{string} ограниченной с помощью \emph{start} и \emph{end}. После
того, как будет достигнут последний символ, поток вернёт конец-файла.
\end{defun}

\begin{defun}[Функция]
make-string-output-stream &key :element-type

Данная функция возвращает поток для записи, который будет аккумулировать всю
полученную информацию в строку, которая может быть получена с помощью функции
\cdf{get-output-stream-string}.

Аргумент \cd{:element-type} указывает, какие символы могут приниматься
потоком. Если аргумент \cd{:element-type} опущен, созданный поток должен
принимать все символы.

Результатом \cdf{make-string-output-stream} всегда является поток типа
\cdf{string-stream}.
\end{defun}

\begin{defun}[Функция]
get-output-stream-string string-output-stream

Данная функция возвращает строку, для потока, возвращённого функцией
\cdf{make-string-output-stream}, которая содержит все записанную в данный поток
информацию. После этого поток сбрасывается. Таким образом каждый вызов
\cdf{get-output-stream-string} возвращает только те символы, которые были
записаны с момента предыдущего вызова этой функции (или создания потока, если
предыдущего вызова ещё не было).
\end{defun}

\begin{defmac}
with-open-stream (var stream) {declaration}* {\,form}*

Форма \emph{stream} вычисляется и должна вернуть поток.
Переменная \emph{var} связывается с этим потоком, и затем выполняются формы тела
как неявный \cdf{progn}. Результатом выполнения \cdf{with-open-stream} является
значение последней формы.
Поток автоматически закрывается при выходе из формы \cdf{with-open-stream}, вне
зависимости от типа выхода. Смотрите \cdf{close}.
Поток следует рассматривать, как имеющий динамическую продолжительность
видимости.
\end{defmac}

\begin{defmac}
with-input-from-string (var string {keyword value}*)
      {declaration}* {\,form}*

Тело выполняется как неявный \cdf{progn} с переменной \emph{var} связанной с
потоком символов для чтения, который последовательно предоставляет символы из
значения формы \emph{string}. \cdf{with-input-from-string} возвращает результат
выполнения последней формы \emph{form} тела.

В параметрах могут использоваться следующие имена:
\begin{quotation}
\begin{flushdesc}
\item[\cd{:index}]
Форма после \cd{:index} должна быть \emph{местом}, в которое можно осуществить
запись с помощью \cdf{setf}. Если форма \cdf{with-input-from-string} завершается
нормально, то \emph{место} будет содержать позицию первого не прочитанного
символа из строки \emph{string} (или длину строки, если все символы были
прочитаны).
\emph{Место} не изменяется в процессе чтения, а только во время выхода.

\item[\cd{:start}]
\cd{:start} принимает аргумент, указывающий позицию с которой
необходимо начинать чтение символов из строки \emph{string}.

\item[\cd{:end}]
The \cd{:end} keyword takes an argument indicating, in the manner
usual for sequence functions, the end of
a substring of \emph{string} to be used.
\cd{:end} принимает аргумент, указывающий на позицию на которой необходимо
завершить чтение символов из строки \emph{string}

\end{flushdesc}
\end{quotation}

The \cd{:start} and \cd{:index} keywords may both specify
the same variable, which is a pointer within the string to be advanced,
perhaps repeatedly by some containing loop.

Вот простой пример использования \cdf{with-input-from-string}:
\begin{lisp}
(with-input-from-string (s "Animal Crackers" :index j :start 6) \\
~~(read s)) \EV\ crackers
\end{lisp}
В качестве побочного эффекта переменная \cd{j} будет установлена в \cd{15}.

\cd{:start} и \cd{:index} могут оба содержать одну переменную, указывающую
позицию в строке, возможно, внутри цикла.
\end{defmac}

\begin{newer}
\begin{defmac}
with-output-to-string (var [string [\!:element-type! type]])
                      {declaration}* {\,form}*

Можно указать \cdf{nil} вместо строки \emph{string} и использовать аргумент
\cd{:element-type} для указания, какие символы должны приниматься созданным
потоком. Если аргумент \emph{string} не указан или он \cdf{nil} и не указан
\cd{:element-type}, то созданный поток должен принимать все символы.
\end{defmac}
\end{newer}

\section {Операции над потоками}

В этом разделе описаны только те функции, которые работают со всеми
потоки. Ввод и вывод информации слегка сложнее и описаны отдельно в
главе~\ref{IO}.
Интерфейс между потоками и файловой системой описан в главе~\ref{FILES}

\begin{defun}[Функция]
streamp object

\cdf{streamp} истинен, если его аргумент является потоком, иначе ложен.
\begin{lisp}
(streamp x) \EQ\ (typep x 'stream)
\end{lisp}
\end{defun}

\begin{newer}
\begin{defun}[Функция]
open-stream-p stream

X3J13 проголосовал в январе 1989
 \issue{STREAM-ACCESS}
добавить предикат \cdf{open-stream-p}.
Если аргумент, который должен быть потоком, открыт, предикат истинен, иначе
ложен.

Поток всегда создаётся открытым. Он продолжает быть открытым, пока не будет
закрыт с помощью функции \cdf{close}. Макросы \cdf{with-open-stream},
\cdf{with-input-from-string}, \cdf{with-output-to-string} и \cdf{with-open-file}
автоматически закрывают созданный поток, когда управление выходит из их тел, по
сути открытость совпадает с динамической продолжительностью видимости потока. 
\end{defun}
\end{newer}

\begin{defun}[Функция]
input-stream-p stream

Если аргумент, который должен быть потоком, может работать для чтения, предикат
истинен, иначе ложен.
\end{defun}

\begin{defun}[Функция]
output-stream-p stream

Если аргумент, который должен быть потоком, может работать для записи, предикат
истинен, иначе ложен.
\end{defun}

\begin{defun}[Функция]
stream-element-type stream

Функция возвращает спецификатор типа, который указывает на то, какие объекты
могут быть прочитаны или записаны из/в поток \emph{stream}.
Потоки созданные с помощью \cdf{open} будут иметь тип элементов, ограниченный
подмножеством \cdf{character} или \cdf{integer}. Но в принципе поток может
проводить операции используя любые Lisp'овые объекты.
\end{defun}

\begin{defun}[Функция]
close stream &key :abort

Аргумент должен быть потоком.
Функцией этот поток закрывается. После чего операции чтения и записи выполняться
над ним не могут. Однако, конечно, некоторые операции все ещё могут
выполняться. Допускается повторное закрытие уже закрытого потока.

Если параметр \cd{:abort} не-{\false} (а по-умолчанию он {\false}), то он
указывает на ненормальное завершение использования потока. Осуществляется
попытка убрать все побочные эффекты, созданные потоком. Например, если поток
выполнял вывод в файл, который был создан вместе с потоком, тогда, если
возможно, файл удаляется и любой ранее существовавший файл не заменяется.
\end{defun}

\begin{new}

\begin{defun}[Функция]
broadcast-stream-streams broadcast-stream

Аргумент должен быть типа \cdf{broadcast-stream}.
Функцией возвращается список потоков для записи (и открытых, и нет).
\end{defun}

\begin{defun}[Функция]
concatenated-stream-streams concatenated-stream

Аргумент должен быть типа \cdf{concatenated-stream}.
Функцией возвращается список потоков (и открытых, и нет).
Этот список отображает упорядоченное множество потоков для чтения, из которых
поток \emph{concatenated-stream} все ещё может получать данные. Поток, из
которого в данный момент читались данные, находится в начале списка.
Если потоков для чтения нет, список может быть пустым.
\end{defun}


\begin{defun}[Функция]
echo-stream-input-stream echo-stream \\
echo-stream-output-stream echo-stream

Аргумент должен быть типа \cdf{echo-stream}.
Функция \cdf{echo-stream-input-stream} возвращает список потоков для чтения.
\cdf{echo-stream-output-stream} возвращает список потоков для записи.
\end{defun}


\begin{defun}[Функция]
synonym-stream-symbol synonym-stream

Аргумент должен быть типа \cdf{synonym-stream}. Эта функция возвращает символ,
значение которого является потоком для потока-синонима \emph{synonym-stream}.
\end{defun}


\begin{defun}[Функция]
two-way-stream-input-stream two-way-stream \\
two-way-stream-output-stream two-way-stream

Аргумент должен быть типа \cdf{two-way-stream}.
Функция \cdf{two-way-stream-input-stream} возвращает список потоков для чтения. 
\cdf{two-way-stream-output-stream} возвращает список потоков для записи.
\end{defun}

\end{new}

\fi      % Stream functions
%Part{Io, Root = "CLM.MSS"}
%%%Chapter of Common Lisp Manual.  Copyright 1984, 1988, 1989 Guy L. Steele Jr.

\clearpage\def\pagestatus{FINAL PROOF}


\chapter{Input/Output Ввод/Вывод}
\label{IO}

Common Lisp provides a rich set of facilities for performing input/output.
All input/output operations are performed on streams of various kinds.
This chapter is devoted to stream data transfer operations.
Streams are discussed in chapter~\ref{STREAM}, and
ways of manipulating files through streams are discussed in
chapter~\ref{FILES}.

While there is provision for reading and writing binary data,
most of the I/O operations in Common Lisp read or write characters.
There are simple primitives for reading and writing single characters
or lines of data.  The \cdf{format} function can perform complex
formatting of output data, directed by a control string
in manner similar to a Fortran \cdf{FORMAT} statement
or a \cd{PL/I} \cd{PUT EDIT} statement.  The most useful I/O operations,
however, read and write printed representations of arbitrary
Lisp objects.

Common Lisp содержит богатый функционал для выполнения операций ввода/вывода.
Все эти операции производятся на различного вида потоках.
Данная глава посвящена тому, как оперировать данными в потоках.
Потоки обсуждаются в главе~\ref{STREAM}, а способы работы с файлами
через потоки в главе~\ref{FILES}.

Большинство операций ввода/вывода в Common Lisp'е читают и записывают строковые
символы, но также есть функции и для бинарных данных.
Есть простые примитивы для чтения и записи одного символа
или строк данных. Функция \cdf{format} функции может выполнять сложное
форматирование выходных данных, направленных управляющей строкой как в выражении
\cdf{FORMAT} в Fortran'е или в \cd{PUT EDIT} в \cd{PL/I}.
Однако, самые полезные операции ввода/вывода читают и записывает выводимые
представления произвольных Lisp'овых объектов.

\section{Printed Representation of Lisp Objects}

Lisp objects in general are not text strings but complex data structures.
They have very different properties from text strings as a consequence of
their internal representation.  However, to make it possible to get at
and talk about Lisp objects, Lisp provides a representation of
most objects in the form of printed text; this is called the \emph{printed
representation}, which is used for input/output purposes and in the
examples throughout this book.  Functions such as \cdf{print} take a
Lisp object and send the characters of its printed representation to a
stream.  The collection of routines that does this is known as the
(Lisp) \emph{printer}.  The \cdf{read} function takes characters from a
stream, interprets them as a printed representation of a Lisp object,
builds that object, and returns it; the collection of routines
that does this is called the (Lisp) \emph{reader}.
\indexterm{printer}
\indexterm{printed representation}
\indexterm{reader}

Ideally, one could print a Lisp object and then read the printed
representation back in, and so obtain the same identical object.
In practice this is difficult and for some purposes not even desirable.
Instead, reading a printed representation produces an object
that is (with obscure technical exceptions)
\cdf{equal} to the originally printed object.

Most Lisp objects have more than one possible printed representation.
For example, the integer twenty-seven can be written in any of these ways:
\begin{lisp}
27~~~~27.~~~~\#o33~~~~\#x1B~~~~\#b11011~~~~\#.(* 3 3 3)~~~~81/3
\end{lisp}
A list of two symbols \cdf{A} and \cdf{B} can be printed in many ways:
\begin{lisp}
~~~~(A B)~~~~(a b)~~~~(~~a~~b~)~~~~({\Xbackslash}A |B|) \\
~~~~(|{\Xbackslash}A| \\
~~B \\
)
\end{lisp}
The last example, which is spread over three lines, may be ugly, but it
is legitimate.  In general, wherever whitespace is permissible in a printed
representation, any number of spaces and newlines may appear.

When \cdf{print} produces a printed representation, it must choose arbitrarily
from among many possible printed representations.  It attempts to choose
one that is readable.  There are a number of global variables that can
be used to control the actions of \cdf{print}, and a number of different
printing functions.

This section describes in detail what is the standard printed
representation for any Lisp object and also describes how \cdf{read} operates.

\subsection{What the Read Function Accepts}
\label{READER}
\indexterm{reader}
The purpose of the Lisp reader is to accept characters, interpret them
as the printed representation of a Lisp object, and construct and
return such an object.  The reader cannot accept everything that the
printer produces; for example, the printed representations of compiled
code objects cannot be read in.  However, the reader has
many features that are not used by the output of the printer at all,
such as comments, alternative representations, and convenient
abbreviations for frequently used but unwieldy constructs.  The reader is
also parameterized in such a way that it can be used as a lexical
analyzer for a more general user-written parser.

The reader is organized as a recursive-descent parser.
Broadly speaking,
the reader operates by reading a character from
the input stream and treating it in one of three ways.
Whitespace characters serve as separators but are otherwise
ignored.  Constituent and escape characters are accumulated
to make a \emph{token}, which is then interpreted as a number or symbol.
Macro characters trigger the invocation of functions (possibly
user-supplied) that can perform arbitrary parsing actions,
including recursive invocation of the reader.

More precisely,
when the reader is invoked, it reads a single character from the input stream
and dispatches according to the syntactic type of that character.
Every character that can appear in the input stream
must be of exactly one of the following kinds:
\emph{illegal},
\emph{whitespace},
\emph{constituent},
\emph{single escape},
\emph{multiple escape}, or
\emph{macro}.
Macro characters are further divided
into the types \emph{terminating} and \emph{non-terminating} (of tokens).
(Note that macro characters have nothing whatever to do with macros
in their operation.  There is a superficial similarity in that macros allow
the user to extend the syntax of Common Lisp at the level of forms,
while macro characters allow the user to extend the syntax at the
level of characters.)
Constituents additionally have one or more attributes,
the most important of which is \emph{alphabetic}; these attributes are discussed
further in section~\ref{PARSE-TOKENS-SECTION}.

The parsing of Common Lisp expressions is discussed in terms of these
syntactic character types because the types of individual characters
are not fixed
but may be altered by the user (see \cdf{set-syntax-from-char}
and \cdf{set-macro-character}).
The characters of the standard character set initially have the
syntactic types shown in table~\ref{Standard-Character-Syntax-Table}.
Note that
the brackets, braces, question mark, and exclamation point
(that is, \cd{{\Xlbracket}}, \cd{{\Xrbracket}}, \cd{{\Xlbrace}},
\cd{{\Xrbrace}}, \cd{?},
and \cd{!}) are normally defined to be constituents, but they
are not used for any purpose in standard Common Lisp syntax and do not occur
in the names of built-in Common Lisp functions or variables.
These characters are explicitly reserved to the user.
The primary intent
is that they be used as macro characters; but a user might choose,
for example, to make \cd{!} be a \emph{single escape} character
(as it is in Portable Standard Lisp).

\begin{table}
\caption{Standard Character Syntax Types}
\label{Standard-Character-Syntax-Table}

\begin{tabular*}{\textwidth}{@{}l@{\extracolsep{\fill}}ll@{}}
$\langle$tab$\rangle$\cd{~~}\emph{whitespace}&$\langle$page$\rangle$\cd{~~}\emph{whitespace}&$\langle$newline$\rangle$\cd{~~}\emph{whitespace} \\
$\langle$space$\rangle$\cd{~~}\emph{whitespace}&\cd{{\Xatsign}~~}\emph{constituent}&\cd{{\Xbq}~~}\emph{terminating macro} \\
\cd{!~~}\emph{constituent} *&\cd{A~~}\emph{constituent}&\cd{a~~}\emph{constituent} \\
\cd{"~~}\emph{terminating macro}&\cd{B~~}\emph{constituent}&\cd{b~~}\emph{constituent} \\
\cd{\#~~}\emph{non-terminating macro}&\cd{C~~}\emph{constituent}&\cd{c~~}\emph{constituent} \\
\cd{\$~~}\emph{constituent}&\cd{D~~}\emph{constituent}&\cd{d~~}\emph{constituent} \\
\cd{\%~~}\emph{constituent}&\cd{E~~}\emph{constituent}&\cd{e~~}\emph{constituent} \\
\cd{\&~~}\emph{constituent}&\cd{F~~}\emph{constituent}&\cd{f~~}\emph{constituent} \\
\cd{'~~}\emph{terminating macro}&\cd{G~~}\emph{constituent}&\cd{g~~}\emph{constituent} \\
\cd{(~~}\emph{terminating macro}&\cd{H~~}\emph{constituent}&\cd{h~~}\emph{constituent} \\
\cd{)~~}\emph{terminating macro}&\cd{I~~}\emph{constituent}&\cd{i~~}\emph{constituent} \\
\cd{*~~}\emph{constituent}&\cd{J~~}\emph{constituent}&\cd{j~~}\emph{constituent} \\
\cd{+~~}\emph{constituent}&\cd{K~~}\emph{constituent}&\cd{k~~}\emph{constituent} \\
\cd{,~~}\emph{terminating macro}&\cd{L~~}\emph{constituent}&\cd{l~~}\emph{constituent} \\
\cd{-~~}\emph{constituent}&\cd{M~~}\emph{constituent}&\cd{m~~}\emph{constituent} \\
\cd{.~~}\emph{constituent}&\cd{N~~}\emph{constituent}&\cd{n~~}\emph{constituent} \\
\cd{/~~}\emph{constituent}&\cd{O~~}\emph{constituent}&\cd{o~~}\emph{constituent} \\
\cd{0~~}\emph{constituent}&\cd{P~~}\emph{constituent}&\cd{p~~}\emph{constituent} \\
\cd{1~~}\emph{constituent}&\cd{Q~~}\emph{constituent}&\cd{q~~}\emph{constituent} \\
\cd{2~~}\emph{constituent}&\cd{R~~}\emph{constituent}&\cd{r~~}\emph{constituent} \\
\cd{3~~}\emph{constituent}&\cd{S~~}\emph{constituent}&\cd{s~~}\emph{constituent} \\
\cd{4~~}\emph{constituent}&\cd{T~~}\emph{constituent}&\cd{t~~}\emph{constituent} \\
\cd{5~~}\emph{constituent}&\cd{U~~}\emph{constituent}&\cd{u~~}\emph{constituent} \\
\cd{6~~}\emph{constituent}&\cd{V~~}\emph{constituent}&\cd{v~~}\emph{constituent} \\
\cd{7~~}\emph{constituent}&\cd{W~~}\emph{constituent}&\cd{w~~}\emph{constituent} \\
\cd{8~~}\emph{constituent}&\cd{X~~}\emph{constituent}&\cd{x~~}\emph{constituent} \\
\cd{9~~}\emph{constituent}&\cd{Y~~}\emph{constituent}&\cd{y~~}\emph{constituent} \\
\cd{:~~}\emph{constituent}&\cd{Z~~}\emph{constituent}&\cd{z~~}\emph{constituent} \\
\cd{;~~}\emph{terminating macro}&\cd{{\Xlbracket}~~}\emph{constituent} *&\cd{{\Xlbrace}~~}\emph{constituent} * \\
\cd{<~~}\emph{constituent}&\cd{{\Xbackslash}~~}\emph{single escape}&\cd{|~~}\emph{multiple escape} \\
\cd{=~~}\emph{constituent}&\cd{{\Xrbracket}~~}\emph{constituent} *&\cd{{\Xrbrace}~~}\emph{constituent} * \\
\cd{>~~}\emph{constituent}&\cd{{\Xcircumflex}~~}\emph{constituent}&\cd{{\Xtilde}~~}\emph{constituent} \\
\cd{?~~}\emph{constituent} *&\cd{{\Xunderscore}~~}\emph{constituent}&$\langle$rubout$\rangle$\cd{~~}\emph{constituent} \\
$\langle$backspace$\rangle$\cd{~~}\emph{constituent}&$\langle$return$\rangle$\cd{~~}\emph{whitespace}&$\langle$linefeed$\rangle$\cd{~~}\emph{whitespace}
\end{tabular*}

\vfill
\begin{small}
\noindent
The characters marked with an asterisk are initially constituents
but are reserved to the user for use as macro characters or for
any other desired purpose.
\end{small}
\end{table}

\begin{table}
\caption{Стандартные типы символьного синтакса}
\label{Standard-Character-Syntax-Table}

\begin{tabular*}{\textwidth}{@{}l@{\extracolsep{\fill}}ll@{}}
$\langle$tab$\rangle$\cd{~~}\emph{пробел}&$\langle$page$\rangle$\cd{~~}\emph{пробел}&$\langle$newline$\rangle$\cd{~~}\emph{пробел} \\
$\langle$space$\rangle$\cd{~~}\emph{пробел}&\cd{{\Xatsign}~~}\emph{составная часть}&\cd{{\Xbq}~~}\emph{терминальный макрос} \\
\cd{!~~}\emph{составная часть} *&\cd{A~~}\emph{составная часть}&\cd{a~~}\emph{составная часть} \\
\cd{"~~}\emph{терминальный макрос}&\cd{B~~}\emph{составная часть}&\cd{b~~}\emph{составная часть} \\
\cd{\#~~}\emph{не-терминальный макрос}&\cd{C~~}\emph{составная часть}&\cd{c~~}\emph{составная часть} \\
\cd{\$~~}\emph{составная часть}&\cd{D~~}\emph{составная часть}&\cd{d~~}\emph{составная часть} \\
\cd{\%~~}\emph{составная часть}&\cd{E~~}\emph{составная часть}&\cd{e~~}\emph{составная часть} \\
\cd{\&~~}\emph{составная часть}&\cd{F~~}\emph{составная часть}&\cd{f~~}\emph{составная часть} \\
\cd{'~~}\emph{терминальный макрос}&\cd{G~~}\emph{составная часть}&\cd{g~~}\emph{составная часть} \\
\cd{(~~}\emph{терминальный макрос}&\cd{H~~}\emph{составная часть}&\cd{h~~}\emph{составная часть} \\
\cd{)~~}\emph{терминальный макрос}&\cd{I~~}\emph{составная часть}&\cd{i~~}\emph{составная часть} \\
\cd{*~~}\emph{составная часть}&\cd{J~~}\emph{составная часть}&\cd{j~~}\emph{составная часть} \\
\cd{+~~}\emph{составная часть}&\cd{K~~}\emph{составная часть}&\cd{k~~}\emph{составная часть} \\
\cd{,~~}\emph{терминальный макрос}&\cd{L~~}\emph{составная часть}&\cd{l~~}\emph{составная часть} \\
\cd{-~~}\emph{составная часть}&\cd{M~~}\emph{составная часть}&\cd{m~~}\emph{составная часть} \\
\cd{.~~}\emph{составная часть}&\cd{N~~}\emph{составная часть}&\cd{n~~}\emph{составная часть} \\
\cd{/~~}\emph{составная часть}&\cd{O~~}\emph{составная часть}&\cd{o~~}\emph{составная часть} \\
\cd{0~~}\emph{составная часть}&\cd{P~~}\emph{составная часть}&\cd{p~~}\emph{составная часть} \\
\cd{1~~}\emph{составная часть}&\cd{Q~~}\emph{составная часть}&\cd{q~~}\emph{составная часть} \\
\cd{2~~}\emph{составная часть}&\cd{R~~}\emph{составная часть}&\cd{r~~}\emph{составная часть} \\
\cd{3~~}\emph{составная часть}&\cd{S~~}\emph{составная часть}&\cd{s~~}\emph{составная часть} \\
\cd{4~~}\emph{составная часть}&\cd{T~~}\emph{составная часть}&\cd{t~~}\emph{составная часть} \\
\cd{5~~}\emph{составная часть}&\cd{U~~}\emph{составная часть}&\cd{u~~}\emph{составная часть} \\
\cd{6~~}\emph{составная часть}&\cd{V~~}\emph{составная часть}&\cd{v~~}\emph{составная часть} \\
\cd{7~~}\emph{составная часть}&\cd{W~~}\emph{составная часть}&\cd{w~~}\emph{составная часть} \\
\cd{8~~}\emph{составная часть}&\cd{X~~}\emph{составная часть}&\cd{x~~}\emph{составная часть} \\
\cd{9~~}\emph{составная часть}&\cd{Y~~}\emph{составная часть}&\cd{y~~}\emph{составная часть} \\
\cd{:~~}\emph{составная часть}&\cd{Z~~}\emph{составная часть}&\cd{z~~}\emph{составная часть} \\
\cd{;~~}\emph{терминальный макрос}&\cd{{\Xlbracket}~~}\emph{составная часть} *&\cd{{\Xlbrace}~~}\emph{составная часть} * \\
\cd{<~~}\emph{составная часть}&\cd{{\Xbackslash}~~}\emph{экранирующий один}&\cd{|~~}\emph{экранирующий много} \\
\cd{=~~}\emph{составная часть}&\cd{{\Xrbracket}~~}\emph{составная часть} *&\cd{{\Xrbrace}~~}\emph{составная часть} * \\
\cd{>~~}\emph{составная часть}&\cd{{\Xcircumflex}~~}\emph{составная часть}&\cd{{\Xtilde}~~}\emph{составная часть} \\
\cd{?~~}\emph{составная часть} *&\cd{{\Xunderscore}~~}\emph{составная часть}&$\langle$rubout$\rangle$\cd{~~}\emph{составная часть} \\
$\langle$backspace$\rangle$\cd{~~}\emph{составная часть}&$\langle$return$\rangle$\cd{~~}\emph{пробел}&$\langle$linefeed$\rangle$\cd{~~}\emph{пробел}
\end{tabular*}

\vfill
\begin{small}
\noindent
Символы помеченный звездочкой первоначально являются составной частью, но
зарезервированы для пользователя в качестве использования макросимволов или для
других целей.
\end{small}
\end{table}

The algorithm performed by the Common Lisp reader is roughly as follows:
\begingroup\leftmargini 1.5em
\begin{enumerate}
\item
If at end of file, perform end-of-file processing (as specified
by the caller of the \cdf{read} function).
Otherwise,
read one character from the input stream, call it \emph{x}, and
dispatch according to the syntactic type of \emph{x} to one
of steps~\ref{READER-ILLEGAL} to~\ref{READER-CONSTITUENT}.
\label{READER-START}

\item
If \emph{x} is an \emph{illegal} character, signal an error.
\label{READER-ILLEGAL}

\item
If \emph{x} is a \emph{whitespace} character,
then discard it and go back to step~\ref{READER-START}.
\label{READER-WHITESPACE}

\item
If \emph{x} is a \emph{macro} character (at this point the
distinction between \emph{terminating} and \emph{non-terminating} macro characters
does not matter), then execute the function associated
with that character.  The function may return zero values or one value
(see \cdf{values}).

The macro-character function may of course read characters from the input
stream; if it does, it will see those characters following the macro
character.  The function may even invoke the reader recursively.
This is how the macro character \cd{(} constructs a list:
by invoking the reader recursively to read the elements of the list.

If one value is returned, then return that value as the result of the
read operation; the algorithm is done.
If zero values are returned, then go back to step~\ref{READER-START}.

\item
If \emph{x} is a \emph{single escape} character (normally \cd{{\Xbackslash}}),
then read the next character and call it \emph{y}
(but if at end of file, signal an error instead).
Ignore the usual syntax of \emph{y}
and pretend it is a \emph{constituent} whose only attribute is
\emph{alphabetic}.
\begin{obsolete}
(If \emph{y} is a lowercase character, leave it alone;
do not replace it with the corresponding uppercase character.)
\end{obsolete}
\begin{newer}
For the purposes of \cdf{readtable-case}, \emph{y} is not replaceable.
\end{newer}
Use \emph{y} to begin a token, and go to step~\ref{READER-PLAIN-TOKEN}.

\item
If \emph{x} is a \emph{multiple escape} character (normally \cd{|}),
then begin a token (initially
containing no characters) and go to step~\ref{READER-MULTI-TOKEN}.

\item
If \emph{x} is a \emph{constituent} character, then it begins an extended token.
\label{READER-CONSTITUENT}\relax
After the entire token is read in, it will be interpreted
either as representing a Lisp object such as a symbol or number
(in which case that object is returned as the result of the read operation),
or as being of illegal syntax (in which case an error is signaled).
\begin{obsolete}
If \emph{x} is a lowercase character, replace it with the
corresponding uppercase character.
\end{obsolete}
\begin{newer}
X3J13 voted in June 1989 \issue{READ-CASE-SENSITIVITY} to introduce
\cdf{readtable-case}.  Consequently, the preceding sentence
should be ignored.  The case of \emph{x\/} should not be altered; instead,
\emph{x} should be regarded as replaceable.
\end{newer}
Use \emph{x} to begin a token, and go on to step~\ref{READER-PLAIN-TOKEN}.

\item
(At this point a token is being accumulated, and an even number
of \emph{multiple escape} characters have been encountered.)
If at end of file, go to step~\ref{READER-TOKEN-END}.
Otherwise, read a character (call it \emph{y}), and
perform one of the following actions according to its syntactic type:
\label{READER-PLAIN-TOKEN}
\begin{itemize}
\item
If \emph{y} is a \emph{constituent} or \emph{non-terminating macro},
then do the following.
\begin{obsolete}
If \emph{y} is a lowercase character, replace it with the
corresponding uppercase character.
\end{obsolete}
\begin{newer}
X3J13 voted in June 1989 \issue{READ-CASE-SENSITIVITY} to introduce
\cdf{readtable-case}.  Consequently, the preceding sentence
should be ignored.  The case of \emph{y\/} should not be altered; instead,
\emph{y} should be regarded as replaceable.
\end{newer}
Append \emph{y} to the token being built,
and repeat step~\ref{READER-PLAIN-TOKEN}.

\item
If \emph{y} is a \emph{single escape} character, then read the next character
and call it \emph{z}
(but if at end of file, signal an error instead).
Ignore the usual syntax of \emph{z}
and pretend it is a \emph{constituent} whose only attribute is
\emph{alphabetic}.
\begin{obsolete}
(If \emph{z} is a lowercase character, leave it alone;
do not replace it with the corresponding uppercase character.)
\end{obsolete}
\begin{newer}
For the purposes of \cdf{readtable-case}, \emph{z} is not replaceable.
\end{newer}
Append \emph{z} to the token being built,
and repeat step~\ref{READER-PLAIN-TOKEN}.

\item
If \emph{y} is a \emph{multiple escape} character,
then go to step~\ref{READER-MULTI-TOKEN}.

\item
If \emph{y} is an \emph{illegal} character, signal an error.

\item
If \emph{y} is a \emph{terminating macro} character, it terminates
the token.  First ``unread'' the character \emph{y}
(see \cdf{unread-char}), then go to step~\ref{READER-TOKEN-END}.

\item
If \emph{y} is a \emph{whitespace} character, it terminates
the token.  First ``unread'' \emph{y}
if appropriate (see \cdf{read-preserving-whitespace}),
then go to step~\ref{READER-TOKEN-END}.
\end{itemize}

\item
(At this point a token is being accumulated, and an odd number
of \emph{multiple escape} characters have been encountered.)
If at end of file, signal an error.
Otherwise, read a character (call it \emph{y}), and
perform one of the following actions according to its syntactic type:
\label{READER-MULTI-TOKEN}
\begin{itemize}
\item
If \emph{y} is a \emph{constituent}, \emph{macro}, or \emph{whitespace}
character, then ignore the usual syntax of that character
and pretend it is a \emph{constituent} whose only attribute is
\emph{alphabetic}.
\begin{obsolete}
(If \emph{y} is a lowercase character, leave it alone;
do not replace it with the corresponding uppercase character.)
\end{obsolete}
\begin{newer}
For the purposes of \cdf{readtable-case}, \emph{y} is not replaceable.
\end{newer}
Append \emph{y} to the token being built,
and repeat step~\ref{READER-MULTI-TOKEN}.

\item
If \emph{y} is a \emph{single escape} character, then read the next character
and call it \emph{z}
(but if at end of file, signal an error instead).
Ignore the usual syntax of \emph{z}
and pretend it is a \emph{constituent} whose only attribute is
\emph{alphabetic}.
\begin{obsolete}
(If \emph{z} is a lowercase character, leave it alone;
do not replace it with the corresponding uppercase character.)
\end{obsolete}
\begin{newer}
For the purposes of \cdf{readtable-case}, \emph{z} is not replaceable.
\end{newer}
Append \emph{z} to the token being built,
and repeat step~\ref{READER-MULTI-TOKEN}.

\item
If \emph{y} is a \emph{multiple escape} character,
then go to step~\ref{READER-PLAIN-TOKEN}.

\item
If \emph{y} is an \emph{illegal} character, signal an error.
\end{itemize}

\item
An entire token has been accumulated.
\begin{newer}
X3J13 voted in June 1989 \issue{READ-CASE-SENSITIVITY} to introduce
\cdf{readtable-case}.  If the accumulated token
is to be interpreted as a symbol, any case conversion of replaceable
characters should be performed at this point according to the value
of the \cdf{readtable-case} slot of the current readtable (the value
of \cd{*readtable*}).
\end{newer}
Interpret the token as representing
a Lisp object and return that object as the result
of the read operation, or signal an error if the token
is not of legal syntax.
\begin{newer}
X3J13 voted in March 1989 \issue{CHARACTER-PROPOSAL}
to specify that implementation-defined
attributes may be removed from the characters of a symbol token
when constructing the print name.
It is implementation-dependent which attributes are removed.
\end{newer}
\label{READER-TOKEN-END}
\end{enumerate}
\endgroup

As a rule, a \emph{single escape} character never stands for itself but always
serves to cause the following character to be treated as a simple alphabetic
character.  A \emph{single escape} character can be included in a token only
if preceded by another \emph{single escape} character.

A \emph{multiple escape} character also never stands for itself.  The characters
between a pair of \emph{multiple escape} characters are all treated as
simple alphabetic characters, except that \emph{single escape} and
\emph{multiple escape} characters must nevertheless be preceded by
a \emph{single escape} character to be included.

\subsection{Parsing of Numbers and Symbols}
\label{PARSE-TOKENS-SECTION}

When an extended token is read, it is interpreted as a number or symbol.
In general, the token is interpreted as a number if it satisfies
the syntax for numbers specified in table~\ref{NUMBER-SYNTAX-TABLE};
this is discussed in more detail below.

The characters of the extended token may serve various syntactic
functions as shown
in table~\ref{Standard-Readtable-Attributes-Table}, but it must be
remembered that any character included in a token under the control
of an escape character is treated as \emph{alphabetic} rather than
according to the attributes shown in the table.
One consequence of this rule is that a whitespace, macro, or escape
character will always be treated as alphabetic within an extended token
because such a character cannot be included in an extended
token except under the control of an escape character.

To allow for extensions to the syntax of numbers, a
syntax for \emph{potential numbers} is defined in Common Lisp that is
more general than the actual syntax for numbers.
Any token that is not a potential number and does not consist
entirely of dots will always be taken to be a symbol,
now and in the future; programs may rely on this fact.
Any token that is a potential number but does not fit the
actual number syntax defined below is a \emph{reserved token} and
has an implementation-dependent interpretation;
an implementation may signal an error, quietly treat the token
as a symbol, or take some other action.  Programmers should avoid
the use of such reserved tokens.  (A symbol whose name looks like a reserved
token can always be written using one or more escape characters.)

\begin{new}
Just as \emph{bignum} is the standard term used by Lisp implementors for
very large integers, and \emph{flonum} (rhymes with ``low hum'') refers
to a floating-point number, the term \emph{potnum} has been used widely
as an abbreviation for ``potential number.''  ``Potnum'' rhymes with ``hot rum.''
\end{new}

\goodbreak

A token is a potential number if it satisfies the following
requirements:


\begin{table}[t]
\caption{Actual Syntax of Numbers}
\label{NUMBER-SYNTAX-TABLE}
\tabbingsep=0pt
\normalsize
\begin{tabbing}
\emph{number} ::= \emph{integer} {\Mor} \emph{ratio} {\Mor} \emph{floating-point-number} \\
\emph{integer} ::= \Mopt{\emph{sign}} \Mplus{\emph{digit}} \Mopt{\emph{decimal-point}} \\
\emph{ratio} ::= \Mopt{\emph{sign}} \Mplus{\emph{digit}} \cdf{/} \Mplus{\emph{digit}} \\
\emph{floating-point-number} ::=\= \Mopt{\emph{sign}} \Mstar{\emph{digit}} \emph{decimal-point} \Mplus{\emph{digit}} \Mopt{\emph{exponent}} \\
\>{\Mor} \'\Mopt{\emph{sign}} \Mplus{\emph{digit}} \Mopt{\emph{decimal-point}
\Mstar{\emph{digit}}} \emph{exponent} \\ \emph{sign} ::= \cdf{+} {\Mor} \cdf{-} \\
\emph{decimal-point} ::= \cd{.} \\
\emph{digit} ::= \cd{0} {\Mor} \cd{1} {\Mor} \cd{2} {\Mor} \cd{3} {\Mor} \cd{4}
         {\Mor} \cd{5} {\Mor} \cd{6} {\Mor} \cd{7} {\Mor} \cd{8} {\Mor} \cd{9} \\
\emph{exponent} ::= \emph{exponent-marker} \Mopt{\emph{sign}} \Mplus{\emph{digit}} \\
\emph{exponent-marker} ::= \cdf{e} {\Mor} \cdf{s} {\Mor} \cdf{f} {\Mor} \cdf{d} {\Mor} \cdf{l}
                   {\Mor} \cdf{E} {\Mor} \cdf{S} {\Mor} \cdf{F} {\Mor} \cdf{D} {\Mor} \cdf{L}
\end{tabbing}
\end{table}

\begin{table}[t]
\caption{Синтаксис чисел}
\label{NUMBER-SYNTAX-TABLE}
\tabbingsep=0pt
\normalsize
\begin{tabbing}
\emph{number} ::= \emph{integer} {\Mor} \emph{ratio} {\Mor} \emph{floating-point-number} \\
\emph{integer} ::= \Mopt{\emph{sign}} \Mplus{\emph{digit}} \Mopt{\emph{decimal-point}} \\
\emph{ratio} ::= \Mopt{\emph{sign}} \Mplus{\emph{digit}} \cdf{/} \Mplus{\emph{digit}} \\
\emph{floating-point-number} ::=\= \Mopt{\emph{sign}} \Mstar{\emph{digit}} \emph{decimal-point} \Mplus{\emph{digit}} \Mopt{\emph{exponent}} \\
\>{\Mor} \'\Mopt{\emph{sign}} \Mplus{\emph{digit}} \Mopt{\emph{decimal-point}
\Mstar{\emph{digit}}} \emph{exponent} \\ \emph{sign} ::= \cdf{+} {\Mor} \cdf{-} \\
\emph{decimal-point} ::= \cd{.} \\
\emph{digit} ::= \cd{0} {\Mor} \cd{1} {\Mor} \cd{2} {\Mor} \cd{3} {\Mor} \cd{4}
         {\Mor} \cd{5} {\Mor} \cd{6} {\Mor} \cd{7} {\Mor} \cd{8} {\Mor} \cd{9} \\
\emph{exponent} ::= \emph{exponent-marker} \Mopt{\emph{sign}} \Mplus{\emph{digit}} \\
\emph{exponent-marker} ::= \cdf{e} {\Mor} \cdf{s} {\Mor} \cdf{f} {\Mor} \cdf{d} {\Mor} \cdf{l}
                   {\Mor} \cdf{E} {\Mor} \cdf{S} {\Mor} \cdf{F} {\Mor} \cdf{D} {\Mor} \cdf{L}
\end{tabbing}
\end{table}

\begin{itemize}
\item
It consists entirely of digits, signs (\cdf{+} or \cdf{-}),
ratio markers (\cdf{/}), decimal points (\cd{.}), extension characters
(\cd{{\Xcircumflex}} or \cd{{\Xunderscore}}), and number markers.  (A number marker is
a letter.  Whether a letter may be treated as a number marker depends
on context, but no letter that is adjacent to another letter may ever be
treated as a number marker.  Floating-point exponent markers are instances
of number markers.)

\item
It contains at least one digit.  (Letters may be considered to be
digits, depending on the value of \cd{*read-base*}, but only
in tokens containing no decimal points.)

\item
It begins with a digit, sign, decimal point, or extension character.

\item
It does not end with a sign.
\end{itemize}
As examples, the following tokens are potential numbers,
but they are \emph{not} actually numbers as defined below, and so are
reserved tokens.  (They do indicate some interesting possibilities
for future extensions.)

\begin{table}
\caption{Standard Constituent Character Attributes}
\label{Standard-Readtable-Attributes-Table}

\begin{tabular*}{\textwidth}{@{\extracolsep{\fill}}l@{\extracolsep{\fill}}lllll@{}}
\cd{!}&\emph{alphabetic}&$\langle$page$\rangle$&\emph{illegal}&$\langle$backspace$\rangle$&\emph{illegal} \\
\cd{"}&\emph{alphabetic} *&$\langle$return$\rangle$&\emph{illegal} *&$\langle$tab$\rangle$&\emph{illegal} * \\
\cd{\#}&\emph{alphabetic} *&$\langle$space$\rangle$&\emph{illegal} *&$\langle$newline$\rangle$&\emph{illegal} * \\
\cd{\$}&\emph{alphabetic}&$\langle$rubout$\rangle$&\emph{illegal}&$\langle$linefeed$\rangle$&\emph{illegal} * \\
\cd{\%}&\emph{alphabetic}&\cd{.}&\multicolumn{3}{l}{\emph{alphabetic}, \emph{dot}, \emph{decimal point}}\\
\cd{\&}&\emph{alphabetic}&\cdf{+}&\multicolumn{3}{l}{\emph{alphabetic}, \emph{plus sign}} \\
\cd{'}&\emph{alphabetic} *&\cdf{-}&\multicolumn{3}{l}{\emph{alphabetic}, \emph{minus sign}} \\
\cd{(}&\emph{alphabetic} *&\cdf{*}&\emph{alphabetic} \\
\cd{)}&\emph{alphabetic} *&\cdf{/}&\multicolumn{3}{l}{\emph{alphabetic}, \emph{ratio marker}} \\
\cd{,}&\emph{alphabetic} *&\cd{{\Xatsign}}&\emph{alphabetic} \\
\cd{0}&\emph{alphadigit}&\cdf{A}, \cdf{a}&\emph{alphadigit} \\
\cd{1}&\emph{alphadigit}&\cdf{B}, \cdf{b}&\emph{alphadigit} \\
\cd{2}&\emph{alphadigit}&\cdf{C}, \cdf{c}&\emph{alphadigit} \\
\cd{3}&\emph{alphadigit}&\cdf{D}, \cdf{d}&\multicolumn{3}{l}{\emph{alphadigit}, \emph{double-float exponent marker}} \\
\cd{4}&\emph{alphadigit}&\cdf{E}, \cdf{e}&\multicolumn{3}{l}{\emph{alphadigit}, \emph{float exponent marker}} \\
\cd{5}&\emph{alphadigit}&\cdf{F}, \cdf{f}&\multicolumn{3}{l}{\emph{alphadigit}, \emph{single-float exponent marker}} \\
\cd{6}&\emph{alphadigit}&\cdf{G}, \cdf{g}&\emph{alphadigit} \\
\cd{7}&\emph{alphadigit}&\cdf{H}, \cdf{h}&\emph{alphadigit} \\
\cd{8}&\emph{alphadigit}&\cdf{I}, \cdf{i}&\emph{alphadigit} \\
\cd{9}&\emph{alphadigit}&\cdf{J}, \cdf{j}&\emph{alphadigit} \\
\cd{:}&\emph{package marker}~~~~~~&\cdf{K}, \cdf{k}&\emph{alphadigit} \\
\cd{;}&\emph{alphabetic} *&\cdf{L}, \cdf{l}&\multicolumn{3}{l}{\emph{alphadigit}, \emph{long-float exponent marker}} \\
\cdf{<}&\emph{alphabetic}&\cdf{M}, \cdf{m}&\emph{alphadigit} \\
\cdf{=}&\emph{alphabetic}&\cdf{N}, \cdf{n}&\emph{alphadigit} \\
\cdf{>}&\emph{alphabetic}&\cdf{O}, \cdf{o}&\emph{alphadigit} \\
\cd{?}&\emph{alphabetic}&\cdf{P}, \cdf{p}&\emph{alphadigit} \\
\cd{{\Xlbracket}}&\emph{alphabetic}&\cdf{Q}, \cdf{q}&\emph{alphadigit} \\
\cd{{\Xbackslash}}&\emph{alphabetic} *&\cdf{R}, \cdf{r}&\emph{alphadigit} \\
\cd{{\Xrbracket}}&\emph{alphabetic}&\cdf{S}, \cdf{s}&\multicolumn{3}{l}{\emph{alphadigit}, \emph{short-float exponent marker}} \\
\cd{{\Xcircumflex}}&\emph{alphabetic}&\cdf{T}, \cdf{t}&\emph{alphadigit} \\
\cd{{\Xunderscore}}&\emph{alphabetic}&\cdf{U}, \cdf{u}&\emph{alphadigit} \\
\cd{{\Xbq}}&\emph{alphabetic} *&\cdf{V}, \cdf{v}&\emph{alphadigit} \\
\cd{{\Xlbrace}}&\emph{alphabetic}&\cdf{W}, \cdf{w}&\emph{alphadigit} \\
\cd{|}&\emph{alphabetic} *&\cdf{X}, \cdf{x}&\emph{alphadigit} \\
\cd{{\Xrbrace}}&\emph{alphabetic}&\cdf{Y}, \cdf{y}&\emph{alphadigit} \\
\cd{{\Xtilde}}&\emph{alphabetic}&\cdf{Z}, \cdf{z}&\emph{alphadigit} \\
\end{tabular*}

\vfill
\begin{footnotesize}
\noindent
These interpretations apply only to characters whose
syntactic type is \emph{constituent}.  Entries marked
with an asterisk are normally shadowed because the characters
are of syntactic type
\emph{whitespace}, \emph{macro}, \emph{single escape}, or \emph{multiple escape}.
An \emph{alphadigit} character is interpreted as a
digit if it is a valid digit in the radix specified by {\small \cd{*read-base*}};
otherwise it is alphabetic.
Characters with an \emph{illegal} attribute can never appear in
a token except under the control of an escape character.
\end{footnotesize}
\end{table}

\begin{table}
\caption{Свойства стандартных символов}
\label{Standard-Readtable-Attributes-Table}

\begin{tabular*}{\textwidth}{@{\extracolsep{\fill}}l@{\extracolsep{\fill}}lllll@{}}
\cd{!}&\emph{алфавитный}&$\langle$page$\rangle$&\emph{недопустимый}&$\langle$backspace$\rangle$&\emph{недопустимый} \\
\cd{"}&\emph{алфавитный} *&$\langle$return$\rangle$&\emph{недопустимый} *&$\langle$tab$\rangle$&\emph{недопустимый} * \\
\cd{\#}&\emph{алфавитный} *&$\langle$space$\rangle$&\emph{недопустимый} *&$\langle$newline$\rangle$&\emph{недопустимый} * \\
\cd{\$}&\emph{алфавитный}&$\langle$rubout$\rangle$&\emph{недопустимый}&$\langle$linefeed$\rangle$&\emph{недопустимый} * \\
\cd{\%}&\emph{алфавитный}&\cd{.}&\multicolumn{3}{l}{\emph{алфавитный},
  \emph{точка}, \emph{разделитель десятичной части}}\\
\cd{\&}&\emph{алфавитный}&\cdf{+}&\multicolumn{3}{l}{\emph{алфавитный},
  \emph{знак плюс}} \\
\cd{'}&\emph{алфавитный} *&\cdf{-}&\multicolumn{3}{l}{\emph{алфавитный},
  \emph{знак минус}} \\
\cd{(}&\emph{алфавитный} *&\cdf{*}&\emph{алфавитный} \\
\cd{)}&\emph{алфавитный} *&\cdf{/}&\multicolumn{3}{l}{\emph{алфавитный},
  \emph{маркер дроби}} \\
\cd{,}&\emph{алфавитный} *&\cd{{\Xatsign}}&\emph{алфавитный} \\
\cd{0}&\emph{алфавитно-цифровой}&\cdf{A}, \cdf{a}&\emph{алфавитно-цифровой} \\
\cd{1}&\emph{алфавитно-цифровой}&\cdf{B}, \cdf{b}&\emph{алфавитно-цифровой} \\
\cd{2}&\emph{алфавитно-цифровой}&\cdf{C}, \cdf{c}&\emph{алфавитно-цифровой} \\
\cd{3}&\emph{алфавитно-цифровой}&\cdf{D}, \cdf{d}&\multicolumn{3}{l}{\emph{алфавитно-цифровой}, \emph{маркер экспоненты
    для двойного с плавающей точкой}} \\
\cd{4}&\emph{алфавитно-цифровой}&\cdf{E}, \cdf{e}&\multicolumn{3}{l}{\emph{алфавитно-цифровой}, \emph{маркер экспоненты
    для числа с плавающей точкой}} \\
\cd{5}&\emph{алфавитно-цифровой}&\cdf{F}, \cdf{f}&\multicolumn{3}{l}{\emph{алфавитно-цифровой}, \emph{маркер экспоненты
    для одинарного с плавающей точкой}} \\
\cd{6}&\emph{алфавитно-цифровой}&\cdf{G}, \cdf{g}&\emph{алфавитно-цифровой} \\
\cd{7}&\emph{алфавитно-цифровой}&\cdf{H}, \cdf{h}&\emph{алфавитно-цифровой} \\
\cd{8}&\emph{алфавитно-цифровой}&\cdf{I}, \cdf{i}&\emph{алфавитно-цифровой} \\
\cd{9}&\emph{алфавитно-цифровой}&\cdf{J}, \cdf{j}&\emph{алфавитно-цифровой} \\
\cd{:}&\emph{package marker}~~~~~~&\cdf{K}, \cdf{k}&\emph{алфавитно-цифровой} \\
\cd{;}&\emph{алфавитный} *&\cdf{L}, \cdf{l}&\multicolumn{3}{l}{\emph{алфавитно-цифровой}, \emph{маркер экспоненты
    для длинного с плавающей точкой}} \\
\cdf{<}&\emph{алфавитный}&\cdf{M}, \cdf{m}&\emph{алфавитно-цифровой} \\
\cdf{=}&\emph{алфавитный}&\cdf{N}, \cdf{n}&\emph{алфавитно-цифровой} \\
\cdf{>}&\emph{алфавитный}&\cdf{O}, \cdf{o}&\emph{алфавитно-цифровой} \\
\cd{?}&\emph{алфавитный}&\cdf{P}, \cdf{p}&\emph{алфавитно-цифровой} \\
\cd{{\Xlbracket}}&\emph{алфавитный}&\cdf{Q}, \cdf{q}&\emph{алфавитно-цифровой} \\
\cd{{\Xbackslash}}&\emph{алфавитный} *&\cdf{R}, \cdf{r}&\emph{алфавитно-цифровой} \\
\cd{{\Xrbracket}}&\emph{алфавитный}&\cdf{S},
\cdf{s}&\multicolumn{3}{l}{\emph{алфавитно-цифровой}, \emph{маркер экспоненты
    для короткого с плавающей точкой}} \\
\cd{{\Xcircumflex}}&\emph{алфавитный}&\cdf{T}, \cdf{t}&\emph{алфавитно-цифровой} \\
\cd{{\Xunderscore}}&\emph{алфавитный}&\cdf{U}, \cdf{u}&\emph{алфавитно-цифровой} \\
\cd{{\Xbq}}&\emph{алфавитный} *&\cdf{V}, \cdf{v}&\emph{алфавитно-цифровой} \\
\cd{{\Xlbrace}}&\emph{алфавитный}&\cdf{W}, \cdf{w}&\emph{алфавитно-цифровой} \\
\cd{|}&\emph{алфавитный} *&\cdf{X}, \cdf{x}&\emph{алфавитно-цифровой} \\
\cd{{\Xrbrace}}&\emph{алфавитный}&\cdf{Y}, \cdf{y}&\emph{алфавитно-цифровой} \\
\cd{{\Xtilde}}&\emph{алфавитный}&\cdf{Z}, \cdf{z}&\emph{алфавитно-цифровой} \\
\end{tabular*}

\vfill
\begin{footnotesize}
\noindent
These interpretations apply only to characters whose
syntactic type is \emph{constituent}.  Entries marked
with an asterisk are normally shadowed because the characters
are of syntactic type
\emph{whitespace}, \emph{macro}, \emph{single escape}, or \emph{multiple escape}.
An \emph{alphadigit} character is interpreted as a
digit if it is a valid digit in the radix specified by {\small \cd{*read-base*}};
otherwise it is alphabetic.
Characters with an \emph{illegal} attribute can never appear in
a token except under the control of an escape character.
\end{footnotesize}
\end{table}

\begin{lisp}
\hskip 0.2\linewidth\=\hskip 0.2\linewidth\=\hskip 0.2\linewidth\=\hskip 0.2\linewidth\=\kill
1b5000\>777777q\>1.7J\>-3/4+6.7J\>12/25/83 \\
27{\Xcircumflex}19\>3{\Xcircumflex}4/5\>6//7\>3.1.2.6\>{\Xcircumflex}-43{\Xcircumflex} \\
3.141{\Xunderscore}592{\Xunderscore}653{\Xunderscore}589{\Xunderscore}793{\Xunderscore}238{\Xunderscore}4\>\>\>-3.7+2.6i-6.17j+19.6k
\end{lisp}
The following tokens are \emph{not} potential numbers but are always
treated as symbols:
\begin{lisp}
\hskip 0.2\linewidth\=\hskip 0.2\linewidth\=\hskip 0.2\linewidth\=\hskip 0.2\linewidth\=\kill
/\>/5\>+\>1+\>1- \\*
foo+\>ab.cd\>{\Xunderscore}\>{\Xcircumflex}\>{\Xcircumflex}/-
\end{lisp}
The following tokens are potential numbers if the value of
\cd{*read-base*} is \cd{16} (an abnormal situation), but they are
always treated as symbols if the value of \cd{*read-base*}
is \cd{10} (the usual value):
\begin{lisp}
\hskip 0.2\linewidth\=\hskip 0.2\linewidth\=\hskip 0.2\linewidth\=\hskip 0.2\linewidth\=\kill
bad-face\>25-dec-83\>a/b\>fad{\Xunderscore}cafe\>f{\Xcircumflex}
\end{lisp}
It is possible for there to be an ambiguity as to whether
a letter should be treated as a digit or as a number marker.
In such a case, the letter is always treated as a digit
rather than as a number marker.

Note that the printed representation for a potential
number may not contain any escape characters.
An escape character robs the following character of all syntactic
qualities, forcing it to be strictly alphabetic and therefore unsuitable
for use in a potential number.  For example,
all of the following representations are interpreted as symbols, not numbers:
\begin{lisp}
{\Xbackslash}256~~~25{\Xbackslash}64~~~1.0{\Xbackslash}E6~~~|100|~~~3{\Xbackslash}.14159~~~|3/4|~~~3{\Xbackslash}/4~~~5||
\end{lisp}
In each case, removing the escape character(s) would allow the token
to be treated as a number.

If a potential number can in fact
be interpreted as a number according to the BNF
syntax in table~\ref{NUMBER-SYNTAX-TABLE}, then a number object of the
appropriate type is constructed and returned.  It should be noted that in
a given implementation it may be that not all tokens conforming to the
actual syntax for numbers can actually be converted into number objects.
For example, specifying too large or too small an exponent for a floating-point
number may make the number impossible to represent in the implementation.
Similarly, a ratio with denominator zero (such as \cd{-35/000})
cannot be represented in \emph{any} implementation.
In any such circumstance where
a token with the syntax of a number cannot be converted to an internal
number object, an error is signaled.  (On the other hand, an error
must not be signaled for specifying too many significant digits
for a floating-point number; an appropriately truncated or rounded
value should be produced.)

There is an omission in the syntax of numbers
as described in table~\ref{NUMBER-SYNTAX-TABLE},
in that the syntax does not account for the possible
use of letters as digits.
The radix used for reading integers and ratios is normally decimal.
However, this radix is actually determined by the value of
the variable \cd{*read-base*}, whose initial value is \cd{10}.
\cd{*read-base*} may take on any integral value between \cd{2} and \cd{36};
let this value be \emph{n}.  Then a token \emph{x} is interpreted as
an integer or ratio in base \emph{n} if it could be properly
so interpreted in the syntax \cd{\#\emph{n}R\emph{x}}
(see section~\ref{SHARP-SIGN-MACRO-CHARACTER-SECTION}).
So, for example, if the value of \cd{*read-base*} is \cd{16},
then the printed representation
\begin{lisp}
(a small face in a bad place)
\end{lisp}
would be interpreted as if the following representation had
been read with \cd{*read-base*} set to \cd{10}:
\begin{lisp}
(10 small 64206 in 10 2989 place)
\end{lisp}
because four of the seven tokens in the list can be interpreted
as hexadecimal numbers.  This facility is intended to be used
in reading files of data that for some reason contain numbers
not in decimal radix; it may also be used for reading programs
written in Lisp dialects (such as MacLisp) whose default number radix is not
decimal.  Non-decimal constants in Common Lisp programs
or portable Common Lisp data files should be written using
\cd{\#O}, \cd{\#X}, \cd{\#B}, or \cd{\#\emph{n}R} syntax.

When \cd{*read-base*} has a value greater than \cd{10}, an ambiguity
is introduced into the actual syntax for numbers because a letter can serve
as either a digit or an exponent marker; a simple example is \cd{1E0}
when the value of \cd{*read-base*} is \cd{16}.  The ambiguity is resolved
in accordance with the general principle that interpretation
as a digit is preferred to interpretation as a number marker.
The consequence in this case is that
if a token can be interpreted as either an integer or a floating-point
number, then it is taken to be an integer.

If a token consists solely of dots (with no escape characters), then an
error is signaled, except in one circumstance: if the token is a single
dot and occurs in a situation appropriate to ``dotted list'' syntax,
then it is accepted as a part of such syntax.  Signaling an error
catches not only misplaced dots in dotted list syntax but also
lists that were truncated by \cd{*print-length*} cutoff,
because such lists end with a three-dot sequence (\cd{...}).
Examples:
\begin{lisp}
~~~~~~~~~~~~~~~~\=\kill
(a . b)\>;\textrm{A dotted pair of \cdf{a} and \cdf{b}} \\
(a.b)\>;\textrm{A list of one element, the symbol named \cd{a.b}} \\
(a. b)\>;\textrm{A list of two elements \cd{a.} and \cdf{b}} \\
(a .b)\>;\textrm{A list of two elements \cdf{a} and \cd{.b}} \\
(a {\Xbackslash}. b)\>;\textrm{A list of three elements \cdf{a}, \cd{.}, and \cdf{b}} \\
(a |.| b)\>;\textrm{A list of three elements \cdf{a}, \cd{.}, and \cdf{b}} \\
(a {\Xbackslash}... b)\>;\textrm{A list of three elements \cdf{a}, \cd{...}, and \cdf{b}} \\
(a |...| b)\>;\textrm{A list of three elements \cdf{a}, \cd{...}, and \cdf{b}} \\
(a b . c)\>;\textrm{A dotted list of \cdf{a} and \cdf{b} with \cdf{c} at the end} \\
.iot\>;\textrm{The symbol whose name is \cd{.iot}} \\
(. b)\>;\textrm{Illegal; an error is signaled} \\
(a .)\>;\textrm{Illegal; an error is signaled} \\
(a .. b)\>;\textrm{Illegal; an error is signaled} \\
(a . . b)\>;\textrm{Illegal; an error is signaled} \\
(a b c ...)\>;\textrm{Illegal; an error is signaled}
\end{lisp}

In all other cases, the token is construed to be the name of a symbol.
If there are any package markers
(colons) in the token, they divide the token into pieces used to
control the lookup and creation of the symbol.

\begin{obsolete}
If there is a single package marker, and it occurs at the beginning of the
token, then the token is interpreted as a keyword, that is, a symbol in the
\cdf{keyword} package.  The part of the token after the package marker must
not have the syntax of a number.

If there is a single package marker not at the beginning or end of the
token, then it divides the token into two parts.  The first part
specifies a package; the second part is the name of an external symbol
available in that package.  Neither of the two parts may have the syntax
of a number.

If there are two adjacent package markers not at the beginning or end of the
token, then they divide the token into two parts.  The first part
specifies a package; the second part is the name of a symbol within
that package (possibly an internal symbol).
Neither of the two parts may have the syntax of a number.
\end{obsolete}

\begin{new}
X3J13 voted in March 1988
\issue{COLON-NUMBER}
to clarify that, in the situations described in the
preceding three paragraphs, the restriction on the syntax of the parts should
be strengthened:  none of the parts may have the syntax of even
a \emph{potential} number.  Tokens such as \cd{:3600}, \cd{:1/2},
and \cd{editor:3.14159} were already ruled out; this clarification further
declares that such tokens as \cd{:2\Xcircumflex 3}, \cd{compiler:1.7J},
and \cd{Christmas:12/25/83} are also in error and therefore should not be used
in portable programs.  Implementations may differ in their treatment of
such package-marked potential numbers.
\end{new}

If a symbol token contains no package markers, then the entire token
is the name of the symbol.  The symbol is looked up in
the default package, which is the value
of the variable \cdf{*package*}.

All other patterns of package markers,
including the cases where there are more than two
package markers or where a package marker appears at the end of the token,
at present do not mean anything in Common Lisp (see chapter~\ref{XPACK}).
It is therefore currently an error to use such patterns in a Common Lisp program.
The valid patterns for tokens may be summarized as follows:
\begin{tabbing}
\hskip 8pc\=\kill
\emph{nnnnn}\>a number \\
\emph{xxxxx}\>a symbol in the current package \\
\cd{:\emph{xxxxx}}\>a symbol in the keyword package \\
\cd{\emph{ppppp}:\emph{xxxxx}}\>an external symbol in the \emph{ppppp} package \\
\cd{\emph{ppppp}::\emph{xxxxx}}\>a (possibly internal) symbol in the \emph{ppppp} package
\end{tabbing}
where \emph{nnnnn} has the syntax of a number, and \emph{xxxxx} and \emph{ppppp} do
not have the syntax of a number.

\begin{new}
In accordance with the X3J13 decision noted above
\issue{COLON-NUMBER}, \emph{xxxxx} and \emph{ppppp} may not have the syntax of even
a potential number.
\end{new}

\begin{defun}[Variable]
*read-base*

The value of \cd{*read-base*} controls the interpretation of tokens
by \cdf{read} as being integers or ratios.  Its value is the radix
in which integers and ratios are to be read; the value may be any integer
from \cd{2} to \cd{36} (inclusive) and is normally \cd{10} (decimal radix).
Its value affects only the reading of integers and ratios.
In particular, floating-point numbers are always read in decimal radix.
The value of \cd{*read-base*} does not affect the radix for rational numbers
whose radix is explicitly indicated by
\cd{\#O}, \cd{\#X}, \cd{\#B}, or \cd{\#\emph{n}R} syntax
or by a trailing decimal point.

Care should be taken when setting \cd{*read-base*} to a value larger
than \cd{10}, because tokens that would normally be interpreted as
symbols may be interpreted as numbers instead.  For example,
with \cd{*read-base*} set to \cd{16} (hexadecimal radix), variables
with names such as \cdf{a}, \cdf{b}, \cdf{f}, \cdf{bad}, and \cdf{face}
will be treated by the reader as numbers (with decimal values
10, 11, 15, 2989, and 64206, respectively).  The ability to alter
the input radix is provided in Common Lisp primarily for the purpose of
reading data files in special operatorats, rather than for the purpose of altering
the default radix in which to read programs.  The user is strongly
encouraged to use \cd{\#O}, \cd{\#X}, \cd{\#B}, or \cd{\#\emph{n}R} syntax when
notating non-decimal constants in programs.
\end{defun}

\begin{defun}[Variable]
*read-suppress*

When the value of \cd{*read-suppress*} is {\nil}, the Lisp reader
operates normally.  When it is not {\nil}, then most of the interesting
operations of the reader are suppressed; input characters are parsed,
but much of what is read is not interpreted.

The primary purpose of \cd{*read-suppress*} is to support the operation of
the read-time conditional constructs \cd{\#+} and \cd{\#-}
(see section~\ref{SHARP-SIGN-MACRO-CHARACTER-SECTION}).  It is important for these
constructs to be able to skip over the printed representation of a Lisp
expression despite the possibility that the syntax of the skipped
expression may not be entirely legal for the current implementation; this
is because a primary application of \cd{\#+} and \cd{\#-} is to allow the
same program to be shared among several Lisp implementations despite
small incompatibilities of syntax.

A non-{\nil} value of \cd{*read-suppress*} has the following specific
effects on the Common Lisp reader:
\begin{itemize}
\item
All extended tokens are completely uninterpreted.  It matters not
whether the token looks like a number, much less like a valid number;
the pattern of package markers also does not matter.  An extended token
is simply discarded and treated as if it were {\nil}; that is, reading
an extended token when \cd{*read-suppress*} is non-{\nil} simply returns {\nil}.
(One consequence of this is that the error concerning improper
dotted-list syntax will not be signaled.)

\item
Any standard
\cd{\#} macro-character construction that requires, permits, or disallows
an infix numerical argument, such as \cd{\#\emph{n}R}, will not enforce
any constraint on the presence, absence, or value of such an argument.

\item
The \cd{\#{\Xbackslash}} construction always produces the value {\nil}.
It will not signal an error even if an unknown character name is seen.

\item
Each of the \cd{\#B}, \cd{\#O}, \cd{\#X}, and \cd{\#R}
constructions always scans over a following token and produces the value {\nil}.
It will not signal an error even if the token does not have the syntax
of a rational number.

\item
The \cd{\#*} construction always scans over a following token and
produces the value {\nil}.
It will not signal an error even if the token does not consist solely
of the characters \cd{0} and \cd{1}.
\end{itemize}

\begin{itemize}
\item
The \cd{\#.} construction reads the following
form (in suppressed mode, of course) but does not evaluate it.
The form is discarded and {\nil} is produced.
\end{itemize}


\begin{itemize}
\item
Each of the \cd{\#A}, \cd{\#S}, and \cd{\#:}
constructions reads the following
form (in suppressed mode, of course) but does not interpret it in any way;
it need not even be a list in the case of \cd{\#S}, or a symbol
in the case of \cd{\#:}.  The form is discarded and {\nil} is produced.

\item
The \cd{\#=} construction is totally ignored.  It does not read
a following form.  It produces no object, but is treated as whitespace.

\item
The \cd{\#\#} construction always produces {\nil}.
\end{itemize}
Note that, no matter what the value of \cd{*read-suppress*},
parentheses still continue to delimit (and construct) lists;
the \cd{\#(} construction continues to delimit vectors;
and comments, strings, and the quote and backquote constructions continue to be
interpreted properly.  Furthermore, such situations as
\cd{')},
\cd{\#<}, \cd{\#)}, and \cd{\#$\langle$\textrm{space}$\rangle$} continue to signal errors.

In some cases, it may be appropriate for a user-written macro-character
definition to check the value of \cd{*read-suppress*} and to avoid certain
computations or side effects if its value is not {\nil}.
\end{defun}

\begin{defun}[Variable]
*read-eval*

Default value of \cd{*read-eval*} is \cdf{t}.
If \cd{*read-eval*} is false, the \cd{\#.} reader macro signals an error.

Printing is also affected.  If
  \cd{*read-eval*} is false and \cd{*print-readably*} is true, any \cdf{print-object}
  method that would otherwise output a \cd{\#.} reader macro must either output something
  different or signal an error of type \cdf{print-not-readable}.

Binding \cd{*read-eval*} to \cdf{nil} is useful when reading data that came from
  an untrusted source, such as a network or a user-supplied data file; it
  prevents the \cd{\#.} reader macro from being exploited as a ``Trojan horse'' to
  cause arbitrary forms to be evaluated.
\end{defun}

\subsection{Macro Characters}
\label{MACRO-CHARACTERS-SECTION}
\indexterm{parsing}
\indexterm{macro character}
If the reader encounters a macro
character, then the function associated with that macro character is
invoked and may produce an object to be returned.  This function may read
following characters in the stream in whatever syntax it likes (it may
even call \cdf{read} recursively) and return the object represented by
that syntax.  Macro characters may or may not be recognized, of course,
when read as part of other special syntaxes (such as for strings).

The reader is therefore organized into two parts: the basic dispatch loop,
which also distinguishes symbols and numbers, and the collection
of macro characters.  Any character can be reprogrammed as a macro character;
this is a means by which the reader can be extended.
The macro characters normally defined are as follows:
\begin{flushdesc}
\item[\cd{(}]
The left-parenthesis character initiates reading of a pair or list.
The function~\cdf{read} is called recursively to read successive objects
until a right parenthesis is found to be next in the input stream.
A list of the objects read is returned.  Thus the input sequence
\begin{lisp}
(a b c)
\end{lisp}
is read as a list of three objects (the symbols \cdf{a}, \cdf{b}, and \cdf{c}).
The right parenthesis need not immediately follow the printed representation of
the last object; whitespace
characters and comments may precede it.
This can be useful for putting one object
on each line and making it easy to add new objects:
\begin{lisp}
(defun traffic-light (color) \\
~~(case color \\
~~~~(green) \\
~~~~(red (stop)) \\
~~~~(amber (accelerate))~~~~~;\textrm{Insert more colors after this line} \\
~~~~))
\end{lisp}

It may be that \emph{no} objects precede the right parenthesis, as in \cd{()}
or \cd{(~)}; this reads as a list of zero objects (the empty list).

If a token that is just a dot,
not preceded by an escape character,
is read after some object,
then exactly one more object must follow the dot,
possibly followed by whitespace,
followed by the right parenthesis:
\begin{lisp}
(a b c . d)
\end{lisp}
This means that the \emph{cdr} of the last pair in the list is not {\nil},
but rather the object whose representation followed the dot.
The above example might have been the result of evaluating
\begin{lisp}
(cons 'a (cons 'b (cons 'c 'd))) \EV\ (a b c . d)
\end{lisp}
Similarly, we have
\begin{lisp}
(cons 'znets 'wolq-zorbitan) \EV\ (znets . wolq-zorbitan)
\end{lisp}
It is permissible for the object following the dot to be a list:
\begin{lisp}
(a b c d . (e f . (g)))
\end{lisp}
is the same as
\begin{lisp}
(a b c d e f g)
\end{lisp}
but a list following a dot
is a non-standard form that \cdf{print} will never produce.

\item[\cd{)}]
The right-parenthesis character is part of various constructs
(such as the syntax for lists) using the left-parenthesis character and
is invalid except when used in such a construct.
\indexterm{\cd{) }macro character}

\newpage%required

\item[\cd{{\Xquote}}]
The single-quote (accent acute) character provides
an abbreviation to make it easier to put constants in programs.
The form
\cd{'\emph{foo}} reads the same as \cd{(quote \emph{foo})}: a list of the symbol
\cdf{quote} and \emph{foo}.

\item[\cd{;}]
Semicolon is used to write comments.
\indexterm{\cd{; }macro character}
\indexterm{comments}
The semicolon and all characters
up to and including the next newline are ignored.
Thus a comment can be put at
the end of any line without affecting the reader.
(A comment will terminate a token, but a newline would terminate the
token anyway.)

There is no functional difference between using one semicolon and using
more than one, but the conventions shown here are in common use.


\begin{lisp}
;;;; COMMENT-EXAMPLE function. \\
;;; This function is useless except to demonstrate comments. \\
;;; (Actually, this example is much too cluttered with them.) \\
 \\
(defun comment-example (x y)~~~~~~;X is anything; Y is an a-list. \\
~~(cond ((listp x) x)~~~~~~~~~~~~~;If X is a list, use that. \\
~~~~~~~~;; X is now not a list.  There are two other cases. \\
~~~~~~~~((symbolp x) \\
~~~~~~~~~;; Look up a symbol in the a-list. \\
~~~~~~~~~(cdr (assoc x y)))~~~~~~~;Remember, (cdr {\false}) is {\false}. \\
~~~~~~~~;; Do this when all else fails: \\
~~~~~~~~(t (cons x~~~~~~~~~~~~~~~~;Add x to a default list. \\
~~~~~~~~~~~~~~~~~'((lisp t)~~~~~~~;LISP is okay. \\
~~~~~~~~~~~~~~~~~~~(fortran nil)~~;FORTRAN is not. \\
~~~~~~~~~~~~~~~~~~~(pl/i -500)~~~~;Note that you can put comments in \\
~~~~~~~~~~~~~~~~~~~(ada .001)~~~~~; "data" as well as in "programs". \\
~~~~~~~~~~~~~~~~~~~;; COBOL?? \\
~~~~~~~~~~~~~~~~~~~(teco -1.0e9))))))
\end{lisp}
In this example, comments may begin with one to four semicolons.
\begin{itemize}
\item
Single-semicolon comments are all aligned to the same column at
the right; usually each comment concerns only the code it is next to.
Occasionally a comment is long enough to occupy two or three
lines; in this case, it is conventional to indent the
continued lines of the comment one space (after the
semicolon).

\item
Double-semicolon comments are aligned to the level of indentation
of the code.  A space conventionally follows the two semicolons.
Such comments usually describe the state of the program at that point
or the code section that follows the comment.

\item
Triple-semicolon comments are aligned to the left margin.
They usually document whole programs or large code blocks.

\item
Quadruple-semicolon comments usually indicate titles of whole programs or large code blocks.
\end{itemize}

\item[\cd{"}]
The double quote character begins the printed representation
of a string.
Successive characters are read from the input stream and accumulated until
another double quote is encountered.
An exception to this occurs if a \emph{single escape} character
is seen; the escape character is discarded,
the next character is accumulated, and accumulation
continues.  When a matching double quote is seen, all the accumulated
characters up to but not including the matching double quote are
made into a simple string and returned.

\item[\cd{{\Xbq}}]
The backquote (accent grave) character
makes it easier to write programs to construct complex data structures by
using a template.
\label{BACKQUOTE}
\begin{new}%CORR
\emph{Notice of correction.}
In the first edition, the backquote character $\langle$\cd{\Xbq}$\rangle$
appearing at the left margin above was inadvertently omitted.
\end{new}
As an example, writing
\begin{lisp}
{\Xbq}(cond ((numberp ,x) ,{\Xatsign}y) (t (print ,x) ,{\Xatsign}y))
\end{lisp}
is roughly equivalent to writing
\begin{lisp}
(list 'cond  \\
~~~~~~(cons (list 'numberp x) y)  \\
~~~~~~(list* 't (list 'print x) y))
\end{lisp}
The general idea is that the backquote is followed by a template,
a picture of a data structure to be built.  This template is copied,
except that within the template commas can appear.  Where a comma
occurs, the form following the comma is to be evaluated to produce an object to
be inserted at that point.  Assume \cdf{b} has the value \cd{3}; then
evaluating the form denoted by \cd{{\Xbq}(a b ,b ,(+ b 1) b)} produces
the result \cd{(a b 3 4 b)}.

If a comma is immediately followed by an at-sign (\cd{{\Xatsign}}), then the
form following the at-sign is evaluated to produce a \emph{list} of objects.
These objects are then ``spliced'' into place in the template.  For
example, if \cdf{x} has the value \cd{(a b c)}, then
\begin{lisp}
{\Xbq}(x ,x ,{\Xatsign}x foo ,(cadr x) bar ,(cdr x) baz ,{\Xatsign}(cdr x)) \\
~~~\EV\ (x (a b c) a b c foo b bar (b c) baz b c)
\end{lisp}

The backquote syntax can be summarized formally as follows.
For each of several situations in which backquote can be used,
a possible interpretation of that situation as an equivalent form
is given.  Note that the form is equivalent only
in the sense that when it is evaluated it will calculate the
correct result.
An implementation is quite free to interpret backquote in any way
such that a backquoted form, when evaluated, will produce a result
\cdf{equal} to that produced by the interpretation shown here.
\begin{itemize}
\item
\cd{{\Xbq}\emph{basic}} is the same as \cd{'\emph{basic}},
that is, \cd{(quote \emph{basic})}, for any form \emph{basic} that is not a
list or a general vector.

\item
\cd{{\Xbq},\emph{form}} is the same as \emph{form}, for any \emph{form}, provided
that the representation of \emph{form} does not begin with ``\cd{{\Xatsign}}''
or ``\cd{.}''.  (A similar caveat holds for all occurrences of a form
after a comma.)

\item
\cd{{\Xbq},{\Xatsign}\emph{form}} is an error.

\item
\cd{{\Xbq}(\emph{x1} \emph{x2} \emph{x3} ... \emph{xn} . \emph{atom})} may be interpreted to mean
\begin{lisp}
(append \textrm{[}\emph{x1}\textrm{]} \textrm{[}\emph{x2}\textrm{]}
    \textrm{[}\emph{x3}\textrm{]} ... \textrm{[}\emph{xn}\textrm{]} (quote \emph{atom}))
\end{lisp}
where the brackets are used to indicate
a transformation of an \emph{xj} as follows:
\begin{itemize}
\item
\textrm{[}\emph{form}\textrm{]} is interpreted as \cd{(list {\Xbq}\emph{form})}, which
contains a backquoted form that must then be further interpreted.

\item
\textrm{[}\cd{,\emph{form}}\textrm{]} is interpreted as \cd{(list \emph{form})}.

\item
\textrm{[}\cd{,{\Xatsign}\emph{form}}\textrm{]} is interpreted simply as \emph{form}.
\end{itemize}

\item
\cd{{\Xbq}(\emph{x1} \emph{x2} \emph{x3} ... \emph{xn})} may be interpreted to mean
the same as the backquoted form
\cd{{\Xbq}(\emph{x1} \emph{x2} \emph{x3} ... \emph{xn} . {\nil})},
thereby reducing it to the previous case.

\item
\cd{{\Xbq}(\emph{x1} \emph{x2} \emph{x3} ... \emph{xn} . ,\emph{form})} may be interpreted to mean
\begin{lisp}
(append \textrm{[}\emph{x1}\textrm{]} \textrm{[}\emph{x2}\textrm{]}
    \textrm{[}\emph{x3}\textrm{]} ... \textrm{[}\emph{xn}\textrm{]} \emph{form})
\end{lisp}
where the brackets indicate a transformation of an \emph{xj} as described above.

\item
\cd{{\Xbq}(\emph{x1} \emph{x2} \emph{x3} ... \emph{xn} . ,{\Xatsign}\emph{form})} is an error.

\item
\cd{{\Xbq}\#(\emph{x1} \emph{x2} \emph{x3} ... \emph{xn})} may be interpreted to mean
\begin{lisp}
(apply \#'vector {\Xbq}(\emph{x1} \emph{x2} \emph{x3} ... \emph{xn}))
\end{lisp}
\end{itemize}

No other uses of comma are permitted; in particular, it may not appear within
the \cd{\#A} or \cd{\#S} syntax.

Anywhere ``\cd{,{\Xatsign}}'' may be used, the syntax ``\cd{,.}'' may be used instead
to indicate that it is permissible to destroy the list produced by the form
following the ``\cd{,.}''; this may permit more efficient code, using
\cdf{nconc} instead of \cdf{append}, for example.

If the backquote syntax is nested, the innermost backquoted form
should be expanded first.  This means that if several commas occur
in a row, the leftmost one belongs to the innermost backquote.

Once again, it is emphasized that an implementation is free to interpret
a backquoted form as any form that, when evaluated, will produce a result
that is \cdf{equal} to the result implied by the above definition.
In particular, no guarantees are made as to whether the constructed
copy of the template will or will not share list structure with the
template itself.  As an example, the above definition implies that
\begin{lisp}
{\Xbq}((,a b) ,c ,{\Xatsign}d)
\end{lisp}
will be interpreted as if it were
\begin{lisp}
(append (list (append (list a) (list 'b) '{\nil})) (list c) d '{\nil})
\end{lisp}
but it could also be legitimately interpreted to mean any of the following.
\begin{lisp}
(append (list (append (list a) (list 'b))) (list c) d) \\
(append (list (append (list a) '(b))) (list c) d) \\
(append (list (cons a '(b))) (list c) d) \\
(list* (cons a '(b)) c d) \\
(list* (cons a (list 'b)) c d) \\
(list* (cons a '(b)) c (copy-list d))
\end{lisp}
(There is no good reason why \cdf{copy-list} should be performed, but
it is not prohibited.)

\penalty-10000%required

\begin{new}
Some users complain that backquote syntax is difficult to read,
especially when it is nested.  I agree that it can get complicated,
but in some situations (such as writing macros that expand into
definitions for other macros) such complexity is to be expected,
and the alternative is much worse.

After I gained some experience in writing nested backquote forms,
I found that I was not stopping to analyze the various patterns
of nested backquotes and interleaved commas and quotes; instead, I was
recognizing standard idioms wholesale, in the same manner that I recognize \cdf{cadar}
as the primitive for ``extract the lambda-list from the
form \cd{((lambda ...)~...)})'' without stopping to analyze it into
``\cdf{car} of \cdf{cdr} of \cdf{car}.''  For example, \cd{,x} within
a doubly-nested backquote form means ``the value of \cdf{x} available
during the second evaluation
will appear here once the form has been twice
evaluated,'' whereas \cd{,',x} means ``the value of \cdf{x} available during the
first evaluation will appear here once the form has been twice
evaluated'' and \cd{,,x} means ``the value of the value of \cdf{x} will appear here.''

See appendix~\ref{BACKQUOTE-SIMULATOR} for a systematic set of examples
of the use of nested backquotes.
\end{new}

\item[\cd{,}]
The comma character is part of the backquote syntax
and is invalid if used other than inside the body of a backquote
construction as described above.
\indexterm{\cd{, }macro character}

\item[\cd{\#}]
This is a \emph{dispatching} macro character.
It reads an optional digit string and then one more character,
and uses that character to select a function to run as a macro-character
function.

The \cd{\#} character also happens to be a non-terminating
macro character.  This is completely independent of the fact that
it is a dispatching macro character; it is a coincidence that
the only standard dispatching macro character in Common Lisp is
also the only standard non-terminating macro character.

See the next section for predefined \cd{\#} macro-character constructions.
\end{flushdesc}

\subsection{Standard Dispatching Macro Character Syntax}
\label{SHARP-SIGN-MACRO-CHARACTER-SECTION}
\indexterm{\cd{\# }macro character}

The standard syntax includes forms introduced by the \cd{\#} character.
These take the general form of a \cd{\#},
a second character that identifies the syntax,
and following arguments in some form.
If the second character is a letter, then case is not important;
\cd{\#O} and \cd{\#o} are considered to be equivalent, for example.

Certain \cd{\#} forms allow an unsigned decimal number to appear
between the \cd{\#} and the second character; some other
forms even require it.  Those forms that do not explicitly permit
such a number to appear forbid it.

\begin{table}
\caption{Standard \# Macro Character Syntax}
\label{Standard-Sharp-Macro-Definitions-Table}

\begin{tabular*}{\textwidth}{@{\extracolsep{\fill}}l@{\extracolsep{\fill}}lll@{}}
\cd{\#!}&\emph{undefined} *&\cd{\#}$\langle$backspace$\rangle$&\emph{signals error} \\
\cd{\#"}&\emph{undefined}&\cd{\#}$\langle$tab$\rangle$&\emph{signals error} \\
\cd{\#\#}&\emph{reference to \cd{\#=} label}&\cd{\#}$\langle$newline$\rangle$&\emph{signals error} \\
\cd{\#\$}&\emph{undefined}&\cd{\#}$\langle$linefeed$\rangle$&\emph{signals error} \\
\cd{\#\%}&\emph{undefined}&\cd{\#}$\langle$page$\rangle$&\emph{signals error} \\
\cd{\#\&}&\emph{undefined}&\cd{\#}$\langle$return$\rangle$&\emph{signals error} \\
\cd{\#'}&\emph{\cdf{function} abbreviation}&\cd{\#}$\langle$space$\rangle$&\emph{signals error} \\
\cd{\#(}&\emph{simple vector}&\cd{\#+}&\emph{read-time conditional} \\
\cd{\#)}&\emph{signals error}&\cd{\#-}&\emph{read-time conditional} \\
\cd{\#*}&\emph{bit-vector}&\cd{\#.}&\emph{read-time evaluation} \\
\cd{\#,}&\emph{load-time evaluation}&\cd{\#/}&\emph{undefined} \\
\cd{\#0}&\emph{used for infix arguments}~~~~~~&\cd{\#A}, \cd{\#a}&\emph{array} \\
\cd{\#1}&\emph{used for infix arguments}&\cd{\#B}, \cd{\#b}&\emph{binary rational} \\
\cd{\#2}&\emph{used for infix arguments}&\cd{\#C}, \cd{\#c}&\emph{complex number} \\
\cd{\#3}&\emph{used for infix arguments}&\cd{\#D}, \cd{\#d}&\emph{undefined} \\
\cd{\#4}&\emph{used for infix arguments}&\cd{\#E}, \cd{\#e}&\emph{undefined} \\
\cd{\#5}&\emph{used for infix arguments}&\cd{\#F}, \cd{\#f}&\emph{undefined} \\
\cd{\#6}&\emph{used for infix arguments}&\cd{\#G}, \cd{\#g}&\emph{undefined} \\
\cd{\#7}&\emph{used for infix arguments}&\cd{\#H}, \cd{\#h}&\emph{undefined} \\
\cd{\#8}&\emph{used for infix arguments}&\cd{\#I}, \cd{\#i}&\emph{undefined} \\
\cd{\#9}&\emph{used for infix arguments}&\cd{\#J}, \cd{\#j}&\emph{undefined} \\
\cd{\#:}&\emph{uninterned symbol}&\cd{\#K}, \cd{\#k}&\emph{undefined} \\
\cd{\#;}&\emph{undefined}&\cd{\#L}, \cd{\#l}&\emph{undefined} \\
\cd{\#<}&\emph{signals error}&\cd{\#M}, \cd{\#m}&\emph{undefined} \\
\cd{\#=}&\emph{label following object}&\cd{\#N}, \cd{\#n}&\emph{undefined} \\
\cd{\#>}&\emph{undefined}&\cd{\#O}, \cd{\#o}&\emph{octal rational} \\
\cd{\#?}&\emph{undefined} *&\cd{\#P}, \cd{\#p}&\emph{pathname} \\
\cd{\#{\Xatsign}}&\emph{undefined}&\cd{\#Q}, \cd{\#q}&\emph{undefined} \\
\cd{\#{\Xlbracket}}&\emph{undefined} *&\cd{\#R}, \cd{\#r}&\emph{radix-n rational} \\
\cd{\#{\Xbackslash}}&\emph{character object}&\cd{\#S}, \cd{\#s}&\emph{structure} \\
\cd{\#{\Xrbracket}}&\emph{undefined} *&\cd{\#T}, \cd{\#t}&\emph{undefined} \\
\cd{\#{\Xcircumflex}}&\emph{undefined}&\cd{\#U}, \cd{\#u}&\emph{undefined} \\
\cd{\#{\Xunderscore}}&\emph{undefined}&\cd{\#V}, \cd{\#v}&\emph{undefined} \\
\cd{\#{\Xbq}}&\emph{undefined}&\cd{\#W}, \cd{\#w}&\emph{undefined} \\
\cd{\#{\Xlbrace}}&\emph{undefined} *&\cd{\#X}, \cd{\#x}&\emph{hexadecimal rational} \\
\cd{\#|}&\emph{balanced comment}&\cd{\#Y}, \cd{\#y}&\emph{undefined} \\
\cd{\#{\Xrbrace}}&\emph{undefined} *&\cd{\#Z}, \cd{\#z}&\emph{undefined} \\
\cd{\#{\Xtilde}}&\emph{undefined}&\cd{\#}$\langle$rubout$\rangle$&\emph{undefined}
\end{tabular*}

\vfill
\begin{small}
\noindent
The combinations marked by an asterisk are explicitly reserved to the user
and will never be defined by Common Lisp.

\begin{new}
X3J13 voted in June 1989 \issue{PATHNAME-PRINT-READ} to
specify \cd{\#P} and \cd{\#p} (\emph{undefined}
in the first edition).
\end{new}
\end{small}
\end{table}


\begin{table}
\caption{Стандартный синтаксис для макросимвола \#}
\label{Standard-Sharp-Macro-Definitions-Table}

\begin{tabular*}{\textwidth}{@{\extracolsep{\fill}}l@{\extracolsep{\fill}}lll@{}}
\cd{\#!}&\emph{неопределен} *&\cd{\#}$\langle$backspace$\rangle$&\emph{сигнализирует ошибку} \\
\cd{\#"}&\emph{неопределен}&\cd{\#}$\langle$tab$\rangle$&\emph{сигнализирует ошибку} \\
\cd{\#\#}&\emph{ссылка на \cd{\#=} метку}&\cd{\#}$\langle$newline$\rangle$&\emph{сигнализирует ошибку} \\
\cd{\#\$}&\emph{неопределен}&\cd{\#}$\langle$linefeed$\rangle$&\emph{сигнализирует ошибку} \\
\cd{\#\%}&\emph{неопределен}&\cd{\#}$\langle$page$\rangle$&\emph{сигнализирует ошибку} \\
\cd{\#\&}&\emph{неопределен}&\cd{\#}$\langle$return$\rangle$&\emph{сигнализирует ошибку} \\
\cd{\#'}&\emph{аббревиатуры для \cdf{function}}&\cd{\#}$\langle$space$\rangle$&\emph{сигнализирует ошибку} \\
\cd{\#(}&\emph{простой вектор}&\cd{\#+}&\emph{условное вычисление во время чтения} \\
\cd{\#)}&\emph{сигнализирует ошибку}&\cd{\#-}&\emph{условное вычисление во время чтения} \\
\cd{\#*}&\emph{битовый вектор}&\cd{\#.}&\emph{вычисление во время чтения} \\
\cd{\#,}&\emph{вычисление во время загрузки}&\cd{\#/}&\emph{неопределен} \\
\cd{\#0}&\emph{используется для инфиксных аргументов}~~~~~~&\cd{\#A}, \cd{\#a}&\emph{массив} \\
\cd{\#1}&\emph{используется для инфиксных аргументов}&\cd{\#B}, \cd{\#b}&\emph{двоичное число} \\
\cd{\#2}&\emph{используется для инфиксных аргументов}&\cd{\#C}, \cd{\#c}&\emph{комплексное число} \\
\cd{\#3}&\emph{используется для инфиксных аргументов}&\cd{\#D}, \cd{\#d}&\emph{неопределен} \\
\cd{\#4}&\emph{используется для инфиксных аргументов}&\cd{\#E}, \cd{\#e}&\emph{неопределен} \\
\cd{\#5}&\emph{используется для инфиксных аргументов}&\cd{\#F}, \cd{\#f}&\emph{неопределен} \\
\cd{\#6}&\emph{используется для инфиксных аргументов}&\cd{\#G}, \cd{\#g}&\emph{неопределен} \\
\cd{\#7}&\emph{используется для инфиксных аргументов}&\cd{\#H}, \cd{\#h}&\emph{неопределен} \\
\cd{\#8}&\emph{используется для инфиксных аргументов}&\cd{\#I}, \cd{\#i}&\emph{неопределен} \\
\cd{\#9}&\emph{используется для инфиксных аргументов}&\cd{\#J}, \cd{\#j}&\emph{неопределен} \\
\cd{\#:}&\emph{uninterned symbol}&\cd{\#K}, \cd{\#k}&\emph{неопределен} \\
\cd{\#;}&\emph{неопределен}&\cd{\#L}, \cd{\#l}&\emph{неопределен} \\
\cd{\#<}&\emph{сигнализирует ошибку}&\cd{\#M}, \cd{\#m}&\emph{неопределен} \\
\cd{\#=}&\emph{метка следующего объекта}&\cd{\#N}, \cd{\#n}&\emph{неопределен} \\
\cd{\#>}&\emph{неопределен}&\cd{\#O}, \cd{\#o}&\emph{восьмеричное число} \\
\cd{\#?}&\emph{неопределен} *&\cd{\#P}, \cd{\#p}&\emph{имя файла} \\
\cd{\#{\Xatsign}}&\emph{неопределен}&\cd{\#Q}, \cd{\#q}&\emph{неопределен} \\
\cd{\#{\Xlbracket}}&\emph{неопределен} *&\cd{\#R}, \cd{\#r}&\emph{число с основанием} \\
\cd{\#{\Xbackslash}}&\emph{символьный объект}&\cd{\#S}, \cd{\#s}&\emph{структура} \\
\cd{\#{\Xrbracket}}&\emph{неопределен} *&\cd{\#T}, \cd{\#t}&\emph{неопределен} \\
\cd{\#{\Xcircumflex}}&\emph{неопределен}&\cd{\#U}, \cd{\#u}&\emph{неопределен} \\
\cd{\#{\Xunderscore}}&\emph{неопределен}&\cd{\#V}, \cd{\#v}&\emph{неопределен} \\
\cd{\#{\Xbq}}&\emph{неопределен}&\cd{\#W}, \cd{\#w}&\emph{неопределен} \\
\cd{\#{\Xlbrace}}&\emph{неопределен} *&\cd{\#X},
\cd{\#x}&\emph{шестнадцатиричное число} \\
\cd{\#|}&\emph{многострочный комментарий}&\cd{\#Y}, \cd{\#y}&\emph{неопределен} \\
\cd{\#{\Xrbrace}}&\emph{неопределен} *&\cd{\#Z}, \cd{\#z}&\emph{неопределен} \\
\cd{\#{\Xtilde}}&\emph{неопределен}&\cd{\#}$\langle$rubout$\rangle$&\emph{неопределен}
\end{tabular*}

\vfill
\begin{small}
\noindent
Комбинации обозначенные звёздочкой зарезервированы для использования
пользователем. Такие комбинации никогда не будут использованы стандартом Common
Lisp'а. 
\end{small}
\end{table}

The currently defined \cd{\#} constructs are described below
and summarized in table~\ref{Standard-Sharp-Macro-Definitions-Table};
more are likely to be added in the future.  However, the constructs
\cd{\#!}, \cd{\#?}, \cd{\#{\Xlbracket}}, \cd{\#{\Xrbracket}},
\cd{\#{\Xlbrace}}, and \cd{\#{\Xrbrace}}
are explicitly reserved for the user and will never be defined by the
Common Lisp standard.
\begin{flushdesc}
\item[\cd{\#{\Xbackslash}}]
\cd{\#{\Xbackslash}\emph{x}} reads in as a character object that represents the
character \emph{x}.  Also, \cd{\#{\Xbackslash}\emph{name}} reads in as the character object
whose name is \emph{name}.
\indexterm{character syntax}
Note that the backslash \cd{{\Xbackslash}} allows this
construct to be parsed easily by {EMACS}-like editors.

In the single-character case, the character \emph{x} must be followed
by a non-constituent character, lest a \emph{name} appear to follow the
\cd{\#{\Xbackslash}}.  A good model of what happens is that after \cd{\#{\Xbackslash}} is read,
the reader backs up over the \cd{{\Xbackslash}} and then reads an extended token,
treating the initial \cd{{\Xbackslash}} as an escape character (whether it really
is or not in the current readtable).

Uppercase and lowercase letters are distinguished after \cd{\#{\Xbackslash}};
\cd{\#{\Xbackslash}A} and \cd{\#{\Xbackslash}a} denote different character objects.  Any
character works after \cd{\#{\Xbackslash}}, even those that are normally special to
\cdf{read}, such as parentheses.  Non-printing characters may be used
after \cd{\#{\Xbackslash}}, although for them names are generally preferred.

\cd{\#{\Xbackslash}\emph{name}} reads in as a character object whose name is \emph{name}
(actually, whose name is \cd{(string-upcase \emph{name})};
therefore the syntax is case-insensitive).
The \emph{name} should have the syntax of a symbol.
The following names are standard across all implementations:
\begin{tabbing}
\hskip 7pc\=\kill
\cdf{newline}\>The character that represents the division between lines \\
\cdf{space}\>The space or blank character
\end{tabbing}
The following names are semi-standard; if an implementation supports
them, they should be used for the described characters and no others.
\begin{tabbing}
\hskip 7pc\=\kill
\cdf{rubout}\>The rubout or delete character.\\
\cdf{page}\>The form-feed or page-separator character \\
\cdf{tab}\>The tabulate character \\
\cdf{backspace}\>The backspace character \\
\cdf{return}\>The carriage return character \\
\cdf{linefeed}\>The line-feed character
\end{tabbing}
In some implementations, one or more of these characters might be
a synonym for a standard character; the \cd{\#{\Xbackslash}Linefeed} character
might be the same as \cd{\#{\Xbackslash}Newline}, for example.

When the Lisp printer types out the name of a special character, it uses the
same table as the \cd{\#{\Xbackslash}} reader; therefore any character name you see typed out
is acceptable as input (in that implementation).  Standard names are always
preferred over non-standard names for printing.

The following convention is used in implementations that support
non-zero bits attributes for character objects.
If a name after \cd{\#{\Xbackslash}} is longer than one character and has a hyphen in it,
then it may be split into the two parts preceding
and following the first hyphen; the first part (actually, \cdf{string-upcase}
of the first part)
may then be interpreted as
the name or initial of a bit, and the second part as the name of the character
(which may in turn contain a hyphen and be subject to further splitting).
For example:
\begin{lisp}
\hskip10pc\=\kill
\#{\Xbackslash}Control-Space			\>\#{\Xbackslash}Control-Meta-Tab \\
\#{\Xbackslash}C-M-Return			\>\#{\Xbackslash}H-S-M-C-Rubout
\end{lisp}
If the character name consists of a single character, then that character
is used.  Another \cd{{\Xbackslash}} may be necessary to quote the character.
\begin{lisp}
\hskip10pc\=\kill
\#{\Xbackslash}Control-\%			\>\#{\Xbackslash}Control-Meta-{\Xbackslash}" \\
\#{\Xbackslash}Control-{\Xbackslash}a			\>\#{\Xbackslash}Meta->
\end{lisp}

\begin{obsolete}
If an unsigned decimal integer appears between the \cd{\#} and \cd{{\Xbackslash}},
it is interpreted as a font number, to become the font attribute
of the character object (see \cdf{char-font}).
\end{obsolete}

\begin{newer}
X3J13 voted in March 1989 \issue{CHARACTER-PROPOSAL}
to replace the notion of bits and font attributes with
that of implementation-defined attributes.  Presumably
this eliminates the portable use of this syntax for font information,
although the vote did not address this question directly.
\end{newer}

\item[\cd{\#'}]
\cd{\#'\emph{foo}} is an abbreviation for \cd{(function \emph{foo})}.
\emph{foo} may be the printed representation of any Lisp object.
This abbreviation may be remembered by analogy with the \cd{'}
macro character, since the \cdf{function} and \cdf{quote} special operators are
similar in form.

\item[\cd{\#(}]
A series of representations of objects enclosed by \cd{\#(} and \cd{)}
is read as a simple vector of those objects.  This is analogous to
the notation for lists.

If an unsigned decimal integer appears between the \cd{\#} and \cd{(},
it specifies explicitly the length of the vector.  In that case,
it is an error if too many objects are specified before the closing \cd{)},
and if too few are specified, the last object
(it is an error if there are none in this case)
is used to fill all
remaining elements of the vector.
For example,
\begin{lisp}
\#(a b c c c c)~~~~~\#6(a b c c c c)~~~~~\#6(a b c)~~~~~\#6(a b c c)
\end{lisp}
all mean the same thing: a vector of length 6 with elements \cdf{a}, \cdf{b},
and four instances of \cdf{c}.  
The notation \cd{\#()} denotes an empty vector, as does \cd{\#0()}
(which is legitimate because it is not the case that too few elements
are specified).

\item[\cd{\#*}]
A series of binary digits (\cd{0} and \cd{1}) preceded by \cd{\#*} is
read as a simple bit-vector containing those bits, the leftmost bit
in the series being bit 0 of the bit-vector.

If an unsigned decimal integer appears between the \cd{\#} and \cdf{*},
it specifies explicitly the length of the vector.  In that case,
it is an error if too many bits are specified,
and if too few are specified the last one
(it is an error if there are none in this case)
is used to fill all remaining elements of the bit-vector.
For example,
\begin{lisp}
\#*101111~~~~~\#6*101111~~~~~\#6*101~~~~~\#6*1011
\end{lisp}
all mean the same thing: a vector of length 6 with elements \cd{1}, \cd{0},
\cd{1}, \cd{1}, \cd{1}, and \cd{1}.
The notation \cd{\#*} denotes an empty bit-vector, as does \cd{\#0*}
(which is legitimate because it is not the case that too few elements
are specified).
\begin{new}
Compare this to \cd{\#B}, used for expressing integers in binary notation.
\end{new}

\item[\cd{\#:}]
\cd{\#:\emph{foo}} requires \emph{foo} to have the syntax of an unqualified
symbol name (no embedded colons).  It denotes an \emph{uninterned} symbol
whose name is \emph{foo}.  Every time this syntax is encountered, a different
uninterned symbol is created.  If it is necessary to refer to the
same uninterned symbol more than once in the same expression,
the \cd{\#=} syntax may be useful.

\item[\cd{\#.}]
\cd{\#.\emph{foo}} is read as the object resulting from the evaluation
of the Lisp object represented by \emph{foo},
which may be the printed representation of any Lisp object.
The evaluation is done during the \cdf{read} process, when the \cd{\#.}
construct is encountered.

\begin{newer}
X3J13 voted in June 1989 \issue{DATA-IO} to add a new reader control variable,
\cd{*read-eval*}.  If it is true,
the \cd{\#.} reader macro behaves as described above;
if it is false, the \cd{\#.} reader macro signals an error.
\end{newer}

The \cd{\#.} syntax therefore performs a read-time evaluation of \emph{foo}.
By contrast, \cd{\#,} (see below) performs a load-time evaluation.

Both \cd{\#.} and \cd{\#,} allow you to include, in an expression
being read, an object that does not have a convenient printed
representation; instead of writing a representation for the object,
you write an expression that will \emph{compute} the object.
\end{flushdesc}

\begin{flushdesc}
\item[\cd{\#B}]
\cd{\#b\emph{rational}} reads \emph{rational} in binary (radix 2).
For example, \cd{\#B1101} \EQ\ \cd{13}, and \cd{\#b101/11} \EQ\ \cd{5/3}.
\begin{new}
Compare this to \cd{\#*}, used for expressing bit-vectors in binary notation.
\end{new}

\item[\cd{\#O}]
\cd{\#o\emph{rational}} reads \emph{rational} in octal (radix 8).
For example, \cd{\#o37/15} \EQ\ \cd{31/13}, and \cd{\#o777} \EQ\ \cd{511}.

\item[\cd{\#X}]
\cd{\#x\emph{rational}} reads \emph{rational} in hexadecimal (radix 16).
The digits above \cd{9} are the letters \cdf{A} through \cdf{F} (the lowercase
letters \cdf{a} through \cdf{f} are also acceptable).  For example,
\cd{\#xF00} \EQ\ \cd{3840}.

\item[\cd{\#\emph{n}R}]
\cd{\#\emph{radix}r\emph{rational}} reads \emph{rational} in radix \emph{radix}.
\emph{radix} must consist of only digits, and
it is read in decimal; its value must be between 2 and 36 (inclusive).

\penalty-10000 %required

For example, \cd{\#3r102} is another way of writing \cd{11}, and \cd{\#11R32}
is another way of writing \cd{35}.  For radices larger than 10, letters of
the alphabet are used in order for the digits after \cd{9}.

\item[\cd{\#\emph{n}A}]
The syntax \cd{\#\emph{n}A\emph{object}} constructs an \emph{n}-dimensional array,
using \emph{object} as the value of the \cd{:initial-contents} argument
to \cdf{make-array}.

The value of \emph{n} makes a difference:
\cd{\#2A((0 1 5) (foo 2 (hot dog)))}, for example, represents a 2-by-3 matrix:
\begin{lisp}
0~~~~~~~1~~~~~~~5 \\
foo~~~~~2~~~~~~~(hot dog)
\end{lisp}
In contrast, \cd{\#1A((0 1 5) (foo 2 (hot dog)))} represents a length-2
array whose elements are lists:
\begin{lisp}
(0 1 5)~~~~(foo 2 (hot dog))
\end{lisp}
Furthermore, \cd{\#0A((0 1 5) (foo 2 (hot dog)))} represents a zero-dimensional
array whose sole element is a list:
\begin{lisp}
((0 1 5) (foo 2 (hot dog)))
\end{lisp}
Similarly, \cd{\#0Afoo} (or, more readably, \cd{\#0A foo}) represents
a zero-dimensional array whose sole element is the symbol \cdf{foo}.
The expression \cd{\#1Afoo} would not be legal because \cdf{foo} is
not a sequence.

\item[\cd{\#S}]
The syntax \cd{\#s(\emph{name} \emph{slot1} \emph{value1} \emph{slot2} \emph{value2} ...)}
denotes a structure.  This is legal only if \emph{name} is the name
of a structure already defined by \cdf{defstruct} and if the
structure has a standard constructor macro, which it normally will.
Let \emph{cm} stand for the name of this constructor macro;
then this syntax is equivalent to
\begin{lisp}
\#.(\emph{cm} \emph{keyword1} '\emph{value1} \emph{keyword2} '\emph{value2} ...)
\end{lisp}
where each \emph{keywordj} is the result of computing
\begin{lisp}
(intern (string \emph{slotj}) 'keyword)
\end{lisp}
(This computation is made so that one need not write a colon in
front of every slot name.)
The net effect is that the constructor macro is called
with the specified slots
having the specified values (note that one does not write quote marks
in the \cd{\#S} syntax).  Whatever object the constructor macro returns
is returned by the \cd{\#S} syntax.
\end{flushdesc}

\begin{newer}
\begin{flushdesc}
\item[\cd{\#P}]
X3J13 voted in June 1989 \issue{PATHNAME-PRINT-READ}
to define the reader syntax \cd{\#p"..."} to be equivalent to
\cd{\#.(parse-namestring "...")}.  Presumably this was meant to
be taken descriptively and not literally.  I would think, for example,
that the committee did not wish to quibble over the package in which
the name \cdf{parse-namestring} was to be read.  Similarly, I would
presume that the \cd{\#p} syntax operates normally rather than signaling
an error when \cd{*read-eval*} is false.  I interpret the intent
of the vote to be that \cd{\#p} reads a following form, which should be
a string, that is then converted to a pathname as if by application
of the standard function \cdf{parse-namestring}.
\end{flushdesc}
\end{newer}

\begin{flushdesc}
\item[\cd{\#\emph{n}{\Xequal}}]
The syntax \cd{\#\emph{n}=\emph{object}} reads as whatever Lisp object
has \emph{object} as its printed representation.  However, that object
is labelled by \emph{n}, a required unsigned decimal integer, for
possible reference by the syntax \cd{\#\emph{n}\#} (below).
The scope of the label is the expression being read by the outermost
call to \cdf{read}.  Within this expression
the same label may not appear twice.

\item[\cd{\#\emph{n}\#}]
The syntax \cd{\#\emph{n}\#}, where \emph{n} is a required unsigned decimal integer,
serves as a reference to some object labelled by \cd{\#\emph{n}=};
that is, \cd{\#\emph{n}\#} represents a pointer to the same identical
(\cdf{eq}) object labelled by \cd{\#\emph{n}=}.
This permits notation of structures with shared or circular substructure.
For example, a structure created in the variable \cdf{y} by this code:
\begin{lisp}
(setq x (list 'p 'q)) \\
(setq y (list (list 'a 'b) x 'foo x)) \\
(rplacd (last y) (cdr y))
\end{lisp}
could be represented in this way:
\begin{lisp}
((a b) . \#1=(\#2=(p q) foo \#2\# . \#1\#))
\end{lisp}
Without this notation, but with \cd{*print-length*} set to \cd{10},
the structure would print in this way:
\begin{lisp}
((a b) (p q) foo (p q) (p q) foo (p q) (p q) foo (p q) ...)
\end{lisp}
A reference \cd{\#\emph{n}\#} may occur only after a label \cd{\#\emph{n}=};
forward references are not permitted.  In addition, the reference
may not appear as the labelled object itself (that is,
one may not write \cd{\#\emph{n}= \#\emph{n}\#}), because the object
labelled by \cd{\#\emph{n}=} is not well defined in this case.

\item[\cd{\#+}]
\label{READ-TIME-CONDITIONAL}
The \cd{\#+} syntax provides a read-time conditionalization facility;
the syntax is
\begin{lisp}
\#+\emph{feature} \emph{form}
\end{lisp}
If \emph{feature} is ``true,'' then this syntax represents a Lisp object
whose printed representation is \emph{form}.  If \emph{feature} is ``false,''
then this syntax is effectively whitespace; it is as if it did not appear.

The \emph{feature} should be the printed representation of a symbol or list.
If \emph{feature} is a symbol, then 
it is true if and only if it is a member of the list that is the value of
the global variable \cdf{*features*}.

Otherwise,
\emph{feature} should be a Boolean expression composed of \cdf{and}, \cdf{or}, and
\cdf{not} operators on (recursive) \emph{feature} expressions.

For example, suppose that in implementation A the features \cdf{spice} and
\cdf{perq} are true, and in implementation B the feature \cdf{lispm} is true.
Then the expressions on the left below are read the same as those on the
right in implementation A:
\begin{lisp}
(cons \#+spice "Spice" \#+lispm "Lispm" x)~~~~~~(cons "Spice" x) \\*
(setq a '(1 2 \#+perq 43 \#+(not perq) 27))~~~~~(setq a '(1 2 43)) \\*
(let ((a 3) \#+(or spice lispm) (b 3))~~~~~~~~~(let ((a 3) (b 3)) \\*
~~(foo a))~~~~~~~~~~~~~~~~~~~~~~~~~~~~~~~~~~~~~~(foo a))\\*
(cons a \#+perq \#-perq b c)~~~~~~~~~~~~~~~~~~~~(cons a c)
\end{lisp}
In implementation B, however, they are read in this way:
\begin{lisp}
(cons \#+spice "Spice" \#+lispm "Lispm" x)~~~~~~(cons "Lispm" x) \\*
(setq a '(1 2 \#+perq 43 \#+(not perq) 27))~~~~~(setq a '(1 2 27)) \\*
(let ((a 3) \#+(or spice lispm) (b 3))~~~~~~~~~(let ((a 3) (b 3)) \\*
~~(foo a))~~~~~~~~~~~~~~~~~~~~~~~~~~~~~~~~~~~~~~(foo a))\\*
(cons a \#+perq \#-perq b c)~~~~~~~~~~~~~~~~~~~~(cons a c)
\end{lisp}

\newpage%manual

The \cd{\#+} construction must be used judiciously if unreadable code is
not to result.  The user should make a careful choice between read-time
conditionalization and run-time conditionalization.

The \cd{\#+} syntax operates by first reading the \emph{feature} specification
and then skipping over the \emph{form} if the \emph{feature} is ``false.''
This skipping of a form is a bit tricky because of the possibility of
user-defined macro characters and side effects caused by the \cd{\#.}
construction.  It is accomplished by binding the variable 
\cd{*read-suppress*} to a non-{\nil} value and then calling the \cdf{read}
function.  See the description of \cd{*read-suppress*} for the details
of this operation.

\begin{newer}
X3J13 voted in March 1988 \issue{SHARPSIGN-PLUS-MINUS-PACKAGE}
to specify that the \cdf{keyword} package is the default package during
the reading of a feature specification.  Thus \cd{\#+spice} means the
same thing as \cd{\#+:spice}, and
\cd{\#+(or~spice~lispm)} means the same thing as \cd{\#+(or~:spice~:lispm)}.
Symbols in other packages
may be used as feature names, but one must use an explicit package prefix
to cite one after \cd{\#+}.
\end{newer}

\item[\cd{\#-}]
\cd{\#-\emph{feature} \emph{form}} is equivalent to \cd{\#+(not \emph{feature}) \emph{form}}.

\item[\cd{\#|}]
\cd{\#|...|\#} is treated as a comment by the reader, just as everything
from a semicolon to the next newline is treated as a comment.
Anything may appear in the comment, except that it must be balanced
with respect to other occurrences of \cd{\#|} and \cd{|\#}.
Except for this nesting rule, the comment may contain any characters
whatsoever.

The main purpose of this construct is to allow ``commenting out''
of blocks of code or data.  The balancing rule allows such blocks
to contain pieces already so commented out.  In this respect
the \cd{\#|...|\#} syntax of Common Lisp differs from the \cd{/*...*/} comment syntax
used by {PL/I} and C.

\item[\cd{\#<}]
This is not legal reader syntax.
It is conventionally used in the printed representation of objects that cannot
be read back in.  Attempting to read a \cd{\#<} will cause an error.
(More precisely, it \emph{is} legal syntax, but the macro-character
function for \cd{\#<} signals an error.)

\begin{newer}
The usual convention for printing unreadable data objects is to print some identifying
information (the internal machine address of the object, if nothing else)
preceded by \cd{\#<} and followed by \cdf{>}.

X3J13 voted in June 1989 \issue{DATA-IO} to add
\cdf{print-unreadable-object}, a macro that prints an object using \cd{\#<...>}
syntax and also takes care of checking the variable \cd{*print-readably*}.
\end{newer}

\item[\cd{\#}$\langle$space$\rangle$, \cd{ \#}$\langle$tab$\rangle$,
      \cd{ \#}$\langle$newline$\rangle$,
      \cd{ \#}$\langle$page$\rangle$, \cd{ \#}$\langle$return$\rangle$]
A \cd{\#} followed by a whitespace character is not legal reader syntax.
This prevents abbreviated forms produced via \cd{*print-level*} cutoff
from reading in again, as a safeguard against losing
information.
(More precisely, this \emph{is} legal syntax, but the macro-character
function for it signals an error.)

\item[\cd{\#)}]
This is not legal reader syntax.
This prevents abbreviated forms produced via \cd{*print-level*} cutoff
from reading in again, as a safeguard against losing information.
(More precisely, this \emph{is} legal syntax, but the macro-character
function for it signals an error.)
\end{flushdesc}

\subsection{The Readtable}
\label{READTABLE-SECTION}
\indexterm{readtable}

Previous sections describe the standard syntax accepted by the \cdf{read}
function.  This section discusses the advanced topic of altering the
standard syntax either to provide extended syntax for Lisp objects
or to aid the writing of other parsers.

There is a data structure called the \emph{readtable} that is used to
control the reader.  It contains information about the syntax of each
character equivalent to that in table~\ref{Standard-Character-Syntax-Table}.
It is set up exactly as in
table~\ref{Standard-Character-Syntax-Table} to give the standard Common Lisp
meanings to all the characters, but the user can change the meanings of
characters to alter and customize the syntax of characters.  It is also
possible to have several readtables describing different syntaxes and to
switch from one to another by binding the variable \cd{*readtable*}.

\begin{defun}[Variable]
*readtable*

The value of \cd{*readtable*} is the current readtable.  The initial
value of this is a readtable set up for standard Common Lisp syntax.
You can bind this variable to temporarily change the readtable being used.
\end{defun}

To program the reader for a different syntax, a set of functions are
provided for manipulating readtables.  Normally, you should begin
with a copy of the standard Common Lisp readtable and then customize
the individual characters within that copy.

\begin{defun}[Function]
copy-readtable &optional from-readtable to-readtable

A copy is made of \emph{from-readtable}, which defaults to the current readtable
(the value of the global variable \cd{*readtable*}).  If \emph{from-readtable}
is {\false}, then a copy of a standard Common Lisp readtable is made.
For example,
\begin{lisp}
(setq *readtable* (copy-readtable nil))
\end{lisp}
will restore the input syntax to standard Common Lisp syntax, even if
the original readtable has been clobbered (assuming it is not so
badly clobbered that you cannot type in the above expression!).
On the other hand,
\begin{lisp}
(setq *readtable* (copy-readtable))
\end{lisp}
will merely replace the current readtable with a copy of itself.

If \emph{to-readtable} is unsupplied or {\false}, a fresh copy is made.  Otherwise,
\emph{to-readtable} must be a readtable, which is destructively copied into.
\end{defun}

\begin{defun}[Function]
readtablep object

\cdf{readtablep} is true if its argument is a readtable,
and otherwise is false.
\begin{lisp}
(readtablep x) \EQ\ (typep x 'readtable)
\end{lisp}
\end{defun}

\begin{defun}[Function]
set-syntax-from-char to-char from-char &optional to-readtable from-readtable

This makes the syntax of \emph{to-char} in \emph{to-readtable} be the same as the
syntax of \emph{from-char} in \emph{from-readtable}.  The \emph{to-readtable}
defaults to the current readtable (the value of the global variable
\cd{*readtable*}), and \emph{from-readtable} defaults to {\false}, meaning
to use the syntaxes from the standard Lisp readtable.
\begin{new}
X3J13 voted in January 1989
\issue{ARGUMENTS-UNDERSPECIFIED}
to clarify that the \emph{to-char} and \emph{from-char}
must each be a character.
\end{new}

Only attributes as shown in
table~\ref{Standard-Character-Syntax-Table} are copied; moreover,
if a \emph{macro character} is copied, the macro definition function
is copied also.
However, attributes as shown
in table~\ref{Standard-Readtable-Attributes-Table} are not copied;
they are ``hard-wired'' into the extended-token parser.
For example, if the definition of \cdf{S} is copied to \cdf{*},
then \cdf{*} will become a \emph{constituent} that is
\emph{alphabetic} but cannot be used
as an exponent indicator for short-format floating-point number syntax.

It works to copy a macro definition from a character such as
\cd{"} to another character; the standard definition for \cd{"}
looks for another character that is the same as the character that
invoked it.  It doesn't work to copy the definition of \cd{(}
to \cd{{\Xlbrace}}, for example; it can be done, but it lets one write lists
in the form \cd{{\Xlbrace}a b c)}, not \cd{{\Xlbrace}a b c{\Xrbrace}},
because the definition
always looks for a closing parenthesis, not a closing brace.  See the function
\cdf{read-delimited-list}, which is useful in this connection.

The \cdf{set-syntax-from-char} function returns \cdf{t}.
\end{defun}

\begin{defun}[Function]
set-macro-character char function &optional non-terminating-p readtable \\
get-macro-character char &optional readtable

\cdf{set-macro-character} causes \emph{char} to be a macro character that
when seen by \cdf{read} causes \emph{function} to be called.
If \emph{non-terminating-p} is not {\false} (it defaults to {\false}),
then it will be a non-terminating macro character: it may be embedded
within extended tokens.
\cdf{set-macro-character} returns {\true}.

\cdf{get-macro-character} returns the function associated with \emph{char}
and, as a second value, returns the \emph{non-terminating-p} flag; it returns
{\false} if \emph{char} does not have macro-character syntax.  In each case,
\emph{readtable} defaults to the current readtable.

If \cdf{nil} is explicitly passed as the
second argument to \cdf{get-macro-character}, then the standard readtable is used.
This is consistent with the behavior of \cdf{copy-readtable}.

The \emph{function} is called with two arguments, \emph{stream}
and \emph{char}.  The \emph{stream} is the input stream, and \emph{char} is the
macro character itself.
In the simplest case, \emph{function} may return a Lisp object.
This object is taken to be that whose printed representation
was the macro character and any following characters read
by the \emph{function}.
As an example, a plausible definition of the standard single quote
character is:
\begin{lisp}
(defun single-quote-reader (stream char) \\
~~(declare (ignore char)) \\
~~(list 'quote (read stream t nil t))) \\
 \\
(set-macro-character \#{\Xbackslash}' \#'single-quote-reader)
\end{lisp}
(Note that {\true} is specified for the \emph{recursive-p} argument
to \cdf{read}; see section~\ref{CHARACTER-INPUT-SECTION}.)
The function reads an object following the single-quote and returns
a list of the symbol \cdf{quote} and that object.
The \emph{char} argument is ignored.

\penalty-10000 %required

The function may choose instead to return \emph{zero} values
(for example, by using \cd{(values)} as the return expression).
In this case, the macro character and whatever it may have read
contribute nothing to the object being read.
As an example, here is a plausible definition for the standard semicolon
(comment) character:
\begin{lisp}
(defun semicolon-reader (stream char) \\
~~(declare (ignore char)) \\
~~;; First swallow the rest of the current input line. \\
~~;; End-of-file is acceptable for terminating the comment. \\
~~(do () ((char= (read-char stream nil \#{\Xbackslash}Newline t) \#{\Xbackslash}Newline))) \\
~~;; Return zero values. \\
~~(values)) \\
 \\
(set-macro-character \#{\Xbackslash}; \#'semicolon-reader)
\end{lisp}
(Note that {\true} is specified for the \emph{recursive-p} argument
to \cdf{read-char}; see section~\ref{CHARACTER-INPUT-SECTION}.)

The \emph{function} should not have any side effects other than on the
\emph{stream}.
Because of backtracking and restarting of the \cdf{read} operation,
front ends (such as editors and
rubout handlers) to the reader may cause
\emph{function} to be called repeatedly during the
reading of a single expression in which the macro character only appears
once.

\begin{new}
Here is an example of a more elaborate set of read-macro characters
that I used in the implementation of the original
simulator for Connection Machine Lisp
\cite{CONNECTION-MACHINE-LISP,CMLISP-IMPLEMENTATION},
a parallel dialect of Common Lisp.  This simulator was used to gain experience with
the language before freezing its design for full-scale implementation on a
Connection Machine computer system.  This example illustrates the typical manner
in which a language designer can embed a new language within the syntactic and
semantic framework of Lisp, saving the effort of designing an implementation
from scratch.

Connection Machine Lisp introduces a new data type called a \emph{xapping},
which is simply an unordered set of ordered pairs of Lisp objects.
The first element of each pair is called the \emph{index} and the second element
the \emph{value}.  We say that the xapping maps each index to its corresponding value.
No two pairs of the same xapping may have the same (that is, \cdf{eql}) index.
Xappings may be finite or infinite sets of pairs; only certain kinds
of infinite xappings are required, and special representations are used for them.

A finite xapping is notated by writing the pairs between braces, separated by whitespace.
A pair is notated by writing the index and the value, separated by a right arrow
(or an exclamation point if the host Common Lisp has no right-arrow character).

\beforenoterule
\begin{sideremark}
The original language design used the right arrow; the exclamation point was
chosen to replace it on {ASCII}-only terminals because it is one of
the six characters \cd{[~] \{~\} !~?} reserved by Common Lisp to the user.

While preparing the \TeX\ manuscript for this book I made a mistake
in font selection and discovered that by an absolutely incredible coincidence
the right arrow has the same numerical code (octal 41) within \TeX\ fonts
as the {ASCII} exclamation point.
The result was that although the manuscript called for right arrows,
exclamation points came out in the printed copy.  Imagine my astonishment!
\end{sideremark}
\afternoterule

Here is an example of a xapping that maps three symbols to strings:
% Stooges
\begin{lisp}
\{moe\Xarrowright "Oh, a wise guy, eh?" larry\Xarrowright "Hey, what's the idea?" \\*
 ~curly\Xarrowright "Nyuk, nyuk, nyuk!"\}
\end{lisp}
For convenience there are certain abbreviated notations.
If the index and value for a pair are the same object \emph{x},
then instead of having to write ``\emph{x}\Xarrowright\,\emph{x}''
(or, worse yet, ``\cd{\#43=\emph{x}\Xarrowright \#43\#}'')
we may write simply \emph{x} for the pair.
If all pairs of a xapping are of this form, we call the xapping a \emph{xet}.
For example, the notation
\begin{lisp}
\{baseball chess cricket curling bocce 43-man-squamish\}
\end{lisp}
is entirely equivalent in meaning to
\begin{lisp}
\{baseball\Xarrowright baseball curling\Xarrowright curling cricket\Xarrowright cricket \\*
~chess\Xarrowright chess bocce\Xarrowright bocce 43-man-squamish\Xarrowright 43-man-squamish\}
\end{lisp}
namely a xet of symbols naming six sports.

Another useful abbreviation covers the situation where the \emph{n} pairs of a finite
xapping are integers, collectively covering a range from zero to $\emph{n}-1$.
This kind of xapping is called a \emph{xector} and may be notated by writing
the values between brackets in ascending order of their indices.
Thus
\begin{lisp}
[tinker evers chance]
\end{lisp}
is merely an abbreviation for
\begin{lisp}
\{tinker\Xarrowright 0 evers\Xarrowright 1 chance\Xarrowright 2\}
\end{lisp}

There are two kinds of infinite xapping: constant and universal.
A constant xapping \cd{\{\Xarrowright \emph{z}\}} maps every object to the same value \emph{z}.
The universal xapping \cd{\{\Xarrowright\}} maps every object to itself and is therefore the xet
of all Lisp objects, sometimes called simply the universe.
Both kinds of infinite xet may be modified by explicitly writing exceptions.
One kind of exception is simply a pair, which specifies the value for a particular index;
the other kind of exception is simply \,\emph{k}\Xarrowright\, indicating that the xapping does
\emph{not} have a pair with index \emph{k} after all.  Thus the notation
\begin{lisp}
\{sky\Xarrowright blue grass\Xarrowright green idea\Xarrowright{} glass\Xarrowright{} \Xarrowright red\}
\end{lisp}
indicates a xapping that maps \cdf{sky} to \cdf{blue}, \cdf{grass} to \cdf{green},
and every other object except \cdf{idea} and \cdf{glass} to \cdf{red}.
Note well that the presence or absence of whitespace on either side
of an arrow is crucial to the correct interpretation
of the notation.

Here is the representation of a xapping as a structure:
\begin{lisp}
(defstruct \\*
~~(xapping (:print-function print-xapping) \\*
~~~~~~~~~~~(:constructor xap \\*
~~~~~~~~~~~~~(domain range \&optional \\*
~~~~~~~~~~~~~~(default ':unknown defaultp) \\*
~~~~~~~~~~~~~~(infinite (and defaultp :constant)) \\*
~~~~~~~~~~~~~~(exceptions '())))) \\
~~domain \\*
~~range \\*
~~default \\*
~~(infinite nil :type (member nil :constant :universal) \\*
~~exceptions)
\end{lisp}
The explicit pairs are represented as two parallel lists, one of indexes (\cdf{domain})
and one of values (\cdf{range}).  The \cdf{default} slot is the default value, relevant
only if the \cdf{infinite} slot is \cd{:constant}.
The \cdf{exceptions} slot is a list of indices for which there are no values.
(See the end of section~\ref{FORMAT-SECTION} for the definition of \cdf{print-xapping}.)

Here, then, is the code for reading xectors in bracket notation:
\begin{lisp}
(defun open-bracket-macro-char (stream macro-char) \\*
~~(declare (ignore macro-char)) \\*
~~(let ((range (read-delimited-list \#{\Xbackslash}] stream t))) \\*
~~~~(xap (iota-list (length range)) range))) \\
 \\
(set-macro-character \#{\Xbackslash}[ \#'open-bracket-macro-char) \\*
(set-macro-character \#{\Xbackslash}] (get-macro-character \#{\Xbackslash}) )) \\
 \\
(defun iota-list (n)~~~~~;\textrm{Return list of integers from $0$ to $\emph{n}-1$}\\*
~~(do ((j (- n 1) (- j 1)) \\*
~~~~~~~(z '() (cons j z))) \\*
~~~~~~((< j 0) z)))
\end{lisp}
The code for reading xappings in the more general brace notation, with all the
possibilities for xets (or individual xet pairs), infinite xappings, and exceptions,
is a bit more complicated; it is shown in table~\ref{XAPPING-MACRO-CHAR-TABLE}.
That code is used in conjunction with the initializations
\begin{lisp}
(set-macro-character \#{\Xbackslash}{\Xlbrace} \#'open-brace-macro-char) \\*
(set-macro-character \#{\Xbackslash}{\Xrbrace} (get-macro-character \#{\Xbackslash}) ))
\end{lisp}
\end{new}
\end{defun}


\begin{table}
\begin{new}
\caption{Macro Character Definition for Xapping Syntax}
\label{XAPPING-MACRO-CHAR-TABLE}
\begin{lisp}
(defun open-brace-macro-char (s macro-char) \\
~~(declare (ignore macro-char)) \\
~~(do ((ch (peek-char t s t nil t) (peek-char t s t nil t)) \\
~~~~~~~(domain '())~~(range '())~~(exceptions '())) \\
~~~~~~((char= ch \#{\Xbackslash}{\Xrbrace}) \\
~~~~~~~(read-char s t nil t) \\
~~~~~~~(construct-xapping (reverse domain) (reverse range))) \\
~~~~(cond ((char= ch \#{\Xbackslash}{\Xarrowright}) \\
~~~~~~~~~~~(read-char s t nil t) \\
~~~~~~~~~~~(let ((nextch (peek-char nil s t nil t))) \\
~~~~~~~~~~~~~(cond ((char= nextch \#{\Xbackslash}{\Xrbrace}) \\
~~~~~~~~~~~~~~~~~~~~(read-char s t nil t) \\
~~~~~~~~~~~~~~~~~~~~(return (xap (reverse domain) \\
~~~~~~~~~~~~~~~~~~~~~~~~~~~~~~~~~(reverse range) \\
~~~~~~~~~~~~~~~~~~~~~~~~~~~~~~~~~nil :universal exceptions))) \\
~~~~~~~~~~~~~~~~~~~(t (let ((item (read s t nil t))) \\
~~~~~~~~~~~~~~~~~~~~~~~~(cond ((char= (peek-char t s t nil t) \#{\Xbackslash}{\Xrbrace}) \\
~~~~~~~~~~~~~~~~~~~~~~~~~~~~~~~(read-char s t nil t) \\
~~~~~~~~~~~~~~~~~~~~~~~~~~~~~~~(return (xap (reverse domain) \\
~~~~~~~~~~~~~~~~~~~~~~~~~~~~~~~~~~~~~~~~~~~~(reverse range) \\
~~~~~~~~~~~~~~~~~~~~~~~~~~~~~~~~~~~~~~~~~~~~item :constant \\
~~~~~~~~~~~~~~~~~~~~~~~~~~~~~~~~~~~~~~~~~~~~exceptions))) \\
~~~~~~~~~~~~~~~~~~~~~~~~~~~~~~(t (reader-error s \\
~~~~~~~~~~~~~~~~~~~~~~~~~~~~~~~~~~~"Default~{\Xarrowright} item must be last")))))))) \\
~~~~~~~~~~(t (let ((item (read-preserving-whitespace s t nil t)) \\
~~~~~~~~~~~~~~~~~~~(nextch (peek-char nil s t nil t))) \\
~~~~~~~~~~~~~~~(cond ((char= nextch \#{\Xbackslash}{\Xarrowright}) \\
~~~~~~~~~~~~~~~~~~~~~~(read-char s t nil t) \\
~~~~~~~~~~~~~~~~~~~~~~(cond ((member (peek-char nil s t nil t) \\
~~~~~~~~~~~~~~~~~~~~~~~~~~~~~~~~~~~~~'(\#{\Xbackslash}Space \#{\Xbackslash}Tab \#{\Xbackslash}Newline)) \\
~~~~~~~~~~~~~~~~~~~~~~~~~~~~~(push item exceptions)) \\
~~~~~~~~~~~~~~~~~~~~~~~~~~~~(t (push item domain) \\
~~~~~~~~~~~~~~~~~~~~~~~~~~~~~~~(push (read s t nil t) range)))) \\
~~~~~~~~~~~~~~~~~~~~~((char= nch \#{\Xbackslash}{\Xrbrace}) \\
~~~~~~~~~~~~~~~~~~~~~~(read-char s t nil t) \\
~~~~~~~~~~~~~~~~~~~~~~(push item domain) \\
~~~~~~~~~~~~~~~~~~~~~~(push item range) \\
~~~~~~~~~~~~~~~~~~~~~~(return (xap (reverse domain) (reverse range)))) \\
~~~~~~~~~~~~~~~~~~~~~(t (push item domain) \\
~~~~~~~~~~~~~~~~~~~~~~~~(push item range))))))))
\end{lisp}
\end{new}
\end{table}

\begin{defun}[Function]
make-dispatch-macro-character char &optional non-terminating-p readtable

This causes the character \emph{char} to be a dispatching macro character
in \emph{readtable} (which defaults to the current readtable).
If \emph{non-terminating-p} is not {\false} (it defaults to {\false}),
then it will be a non-terminating macro character: it may be embedded
within extended tokens.
\cdf{make-dispatch-macro-character} returns {\true}.

Initially every character in the dispatch table has a character-macro
function that signals an error.  Use \cdf{set-dispatch-macro-character}
to define entries in the dispatch table.
\begin{new}
X3J13 voted in January 1989
\issue{ARGUMENTS-UNDERSPECIFIED}
to clarify that \emph{char} must be a character.
\end{new}

\end{defun}

\begin{defun}[Function]
set-dispatch-macro-character disp-char sub-char function &optional readtable \\
get-dispatch-macro-character disp-char sub-char ~~~~ &optional readtable

\cdf{set-dispatch-macro-character}
causes \emph{function} to be called when the \emph{disp-char}
followed by \emph{sub-char} is read.  The \emph{readtable} defaults to the
current readtable.  The arguments and return values for \emph{function} are
the same as for normal macro characters
except that \emph{function} gets \emph{sub-char}, not \emph{disp-char},
as its second argument and also receives a third
argument that is the non-negative integer whose decimal
representation appeared between
\emph{disp-char} and \emph{sub-char}, or {\false} if no decimal integer appeared
there.

The \emph{sub-char} may not be one of the ten decimal digits;
they are always reserved for specifying an infix integer argument.
Moreover, if \emph{sub-char} is a lowercase character
(see \cdf{lower-case-p}), its uppercase equivalent is used instead.
(This is how the rule is enforced that the case of a dispatch sub-character
doesn't matter.)

\cdf{set-dispatch-macro-character} returns {\true}.

\cdf{get-dispatch-macro-character} returns the macro-character function
for \emph{sub-char} under \emph{disp-char}, or {\nil} if there is no
function associated with \emph{sub-char}.

If the \emph{sub-char} is one of the ten decimal digits \cd{0 1 2 3 4 5 6 7 8 9},
\cdf{get-dispatch-macro-character} always returns {\nil}.
If \emph{sub-char} is a lowercase character,
its uppercase equivalent is used instead.

\begin{new}
X3J13 voted in January 1989
\issue{GET-MACRO-CHARACTER-READTABLE}
to specify that if \cdf{nil} is explicitly passed as the
second argument to \cdf{get-dispatch-macro-character}, then the standard readtable is used.
This is consistent with the behavior of \cdf{copy-readtable}.
\end{new}

For either function,
an error is signaled if the specified \emph{disp-char} is not in fact
a dispatch character in the specified readtable.  It is necessary
to use \cdf{make-dispatch-macro-character} to set up the
dispatch character before specifying its sub-characters.

As an example, suppose one would like \cd{\#\$\emph{foo}} to be read
as if it were \cd{(dollars \emph{foo})}.  One might say:
\begin{lisp}
(defun |\#\$-reader| (stream subchar arg) \\
~~(declare (ignore subchar arg)) \\
~~(list 'dollars (read stream t nil t))) \\
 \\
(set-dispatch-macro-character \#{\Xbackslash}\# \#{\Xbackslash}\$ \#'|\#\$-reader|)
\end{lisp}
\end{defun}

\begin{newer}
\begin{defun}[Function]
readtable-case readtable

X3J13 voted in June 1989 \issue{READ-CASE-SENSITIVITY}
to introduce the function \cdf{readtable-case} to
control the reader's interpretation of case.
It provides access to a slot in a readtable, and may be used with \cdf{setf}
to alter the state of that slot.
The possible values for the slot are \cd{:upcase}, \cd{:downcase},
\cd{:preserve}, and \cd{:invert}; the \cdf{readtable-case} for the standard readtable
is \cd{:upcase}.
Note that \cdf{copy-readtable} is required to copy the \cdf{readtable-case} slot
along with all other readtable information.

Once the reader has accumulated a token as described in section~\ref{READER},
if the token is a symbol, ``replaceable'' characters (unescaped uppercase or
lowercase constituent characters)
may be modified under the control of the \cdf{readtable-case} of the current readtable:
\begin{itemize}
\item For \cd{:upcase}, replaceable characters are converted to uppercase.
(This was the behavior specified by the first edition.)

\item For \cd{:downcase}, replaceable characters are converted to lowercase.

\item For \cd{:preserve}, the cases of all characters remain unchanged.

\item For \cd{:invert}, if all of the replaceable letters
in the extended token are of the same case, they are all converted to the opposite case;
otherwise the cases of all characters in that token remain unchanged.
\end{itemize}
As an illustration, consider the following code.
\begin{lisp}
(let ((*readtable* (copy-readtable nil))) \\*
~~(format t "READTABLE-CASE~~Input~~~Symbol-name{\Xtilde} \\*
~~~~~~~~~~~{\Xtilde}\%\hbox to 0pt{------------------\hss}~~~~~~~~~~~~~~~~~~-----------------{\Xtilde} \\*
~~~~~~~~~~~{\Xtilde}\%") \\
~~(dolist (readtable-case '(:upcase :downcase :preserve :invert)) \\*
~~~~(setf (readtable-case *readtable*) readtable-case) \\*
~~~~(dolist (input '("ZEBRA" "Zebra" "zebra")) \\
~~~~~~(format t ":{\Xtilde}A{\Xtilde}16T{\Xtilde}A{\Xtilde}24T{\Xtilde}A{\Xtilde}\%" \\*
~~~~~~~~~~~~~~~~(string-upcase readtable-case) \\*
~~~~~~~~~~~~~~~~input \\*
~~~~~~~~~~~~~~~~(symbol-name (read-from-string input)))))))
\end{lisp}
The output from this test code should be
\begin{lisp}
READTABLE-CASE~~Input~~~Symbol-name \\*
----------------------------------- \\*
:UPCASE~~~~~~~~~ZEBRA~~~ZEBRA \\*
:UPCASE~~~~~~~~~Zebra~~~ZEBRA \\*
:UPCASE~~~~~~~~~zebra~~~ZEBRA \\
:DOWNCASE~~~~~~~ZEBRA~~~zebra \\
:DOWNCASE~~~~~~~Zebra~~~zebra \\
:DOWNCASE~~~~~~~zebra~~~zebra \\
:PRESERVE~~~~~~~ZEBRA~~~ZEBRA \\
:PRESERVE~~~~~~~Zebra~~~Zebra \\
:PRESERVE~~~~~~~zebra~~~zebra \\
:INVERT~~~~~~~~~ZEBRA~~~zebra \\*
:INVERT~~~~~~~~~Zebra~~~Zebra \\*
:INVERT~~~~~~~~~zebra~~~ZEBRA
\end{lisp}
\end{defun}
The \cdf{readtable-case} of the current readtable also affects the printing
of symbols (see \cdf{*print-case*} and \cdf{*print-escape*}).
\end{newer}

\subsection{What the Print Function Produces}
\label{PRINTER}

The Common Lisp printer is controlled by a number of special variables.
These are referred to in the following discussion
and are fully documented at the end of this section.

How an expression is printed depends on its data type, as described
in the following paragraphs.

\begin{flushdesc}
\item[\emph{Integers}]
If appropriate, a radix specifier may be printed; see the
variable \cd{*print-radix*}.
If an integer is negative, a minus sign is printed and then the
absolute value of the integer is printed.
Integers are printed in the radix specified by the
variable \cd{*print-base*}
in the usual positional notation, most significant digit first.
The number zero is represented
by the single digit \cd{0} and never has a sign.
A decimal point may then be printed, depending on the value
of \cd{*print-radix*}.

\item[\emph{Ratios}]
If appropriate, a radix specifier may be printed; see the variable
\cd{*print-radix*}.
If the ratio is negative, a minus sign is printed.
Then the absolute value of the numerator is printed, as for an integer;
then a \cdf{/}; then the denominator.  The numerator and denominator are
both printed in the radix specified by the variable \cd{*print-base*}; they are obtained as if by
the \cdf{numerator} and \cdf{denominator} functions, and so ratios
are always printed in reduced form (lowest terms).

\item[\emph{Floating-point numbers}]
If the sign of the number (as determined by the function \cdf{float-sign})
is negative, then a minus sign is printed.  Then the magnitude is
printed in one of two ways.
If the magnitude of the
floating-point number is either zero or between $10^{-3}$ (inclusive)
and $10^7$ (exclusive), it may be printed as
the integer part of the number, then a decimal point,
followed by the fractional part of the number; there is always at least one
digit on each side of the decimal point.
If the format of the number does not match that specified by the variable
\cdf{*read-default-float-format*}, then the exponent marker for
that format and the digit \cd{0} are also printed.
For example, the base of the natural logarithms as a short-format
floating-point number might be printed as \cd{2.71828S0}.

For non-zero magnitudes
outside of the range $10^{-3}$
to $10^7$, a floating-point number
will be printed in ``computerized scientific
notation.''  The representation of the number is scaled to be between
1 (inclusive) and 10 (exclusive) and then printed, with one digit
before the decimal point and at least one digit after the decimal point.
Next the exponent marker for the format is printed,
except that
if the format of the number matches that specified by the variable
\cdf{*read-default-float-format*}, then the exponent marker \cdf{E}
is used.
Finally, the power of 10 by which the fraction must be multiplied
to equal the original number is printed as a decimal integer.
For example, Avogadro's number as a short-format floating-point number
might be printed as \cd{6.02S23}.

\item[\emph{Complex numbers}]
A complex number is printed as \cd{\#C},
an open parenthesis,
the printed representation of its real part, a space,
the printed representation of its imaginary part, and finally
a close parenthesis.


\begin{obsolete}
\item[\emph{Characters}]
When \cdf{*print-escape*} is {\false}, a character prints as itself;
it is sent directly to the output stream.
When \cdf{*print-escape*} is not {\false}, then \cd{\#{\Xbackslash}} syntax is used.
For example, the
printed representation of the character \cd{\#{\Xbackslash}A} with control and meta
bits on would be \cd{\#{\Xbackslash}CONTROL-META-A}, and that of \cd{\#{\Xbackslash}a} with
control and meta bits on would be \cd{\#{\Xbackslash}CONTROL-META-{\Xbackslash}a}.
\end{obsolete}
\begin{newer}
X3J13 voted in June 1989 \issue{DATA-IO} to specify that if \cd{*print-readably*}
is not {\false} then every object must be printed in a readable form,
regardless of other printer control variables.  For characters, the simplest approach
is always to use \cd{\#{\Xbackslash}} syntax when \cd{*print-readably*}
is not {\false}, regardless of the value of \cdf{*print-escape*}.
\end{newer}


\begin{obsolete}
\item[\emph{Symbols}]

When \cdf{*print-escape*} is {\false}, only the characters of the print name
of the symbol are output (but the case in which to print any
uppercase characters in the print name is controlled by the
variable \cdf{*print-case*}).
\end{obsolete}

\begin{newer}
X3J13 voted in June 1989 \issue{READ-CASE-SENSITIVITY}
to specify that the new \cdf{readtable-case} slot of the current readtable
also controls the case in which letters (whether uppercase or lowercase)
in the print name of a symbol are output, no matter what the value of \cdf{*print-escape*}.
\end{newer}

\begin{obsolete}
The remaining paragraphs describing the printing of symbols cover
the situation when \cdf{*print-escape*} is not {\false}.
\end{obsolete}

\begin{newer}
X3J13 voted in June 1989 \issue{DATA-IO} to specify that if \cd{*print-readably*}
is not {\false} then every object must be printed in a readable form,
regardless of other printer control variables.
For symbols, the simplest approach
is to print them, when \cd{*print-readably*} is not {\false}, as if \cdf{*print-escape*}
were not {\false}, regardless of the actual value of \cdf{*print-escape*}.
\end{newer}

Backslashes \cd{{\Xbackslash}} and
vertical bars \cd{|} are included as required.  In particular,
backslash or vertical-bar syntax is used when the name of
the symbol would be otherwise treated by the reader as a potential number
(see section~\ref{PARSE-TOKENS-SECTION}).
In making this decision, it is assumed that the value of \cd{*print-base*}
being used for printing would be used as the value of \cd{*read-base*}
used for reading; the value of \cd{*read-base*} at the time of printing
is irrelevant.  For example, if the value of \cd{*print-base*} were \cd{16}
when printing the symbol \cdf{face}, it would have to be printed as
\cd{{\Xbackslash}FACE} or \cd{{\Xbackslash}Face} or \cd{|FACE|}, because the token
\cdf{face} would be read as a hexadecimal
number (decimal value 64206) if \cd{*read-base*} were \cd{16}.

\begin{obsolete}
The case in which to print any
uppercase characters in the print name is controlled by the
variable \cdf{*print-case*}.
\end{obsolete}
\begin{newer}
X3J13 voted in June 1989 \issue{PRINT-CASE-PRINT-ESCAPE-INTERACTION}
to clarify the interaction of \cdf{*print-case*} with \cdf{*print-escape*};
see \cdf{*print-case*}.
\end{newer}
As a special case [no pun intended], {\nil} may sometimes be printed as \cd{()} instead,
when \cdf{*print-escape*} and \cdf{*print-pretty*} are both not {\false}.

Package prefixes may be printed (using colon syntax)
if necessary.
The rules for package qualifiers are as follows.
When the symbol is printed, if it is in the
keyword package, then it is printed with a preceding colon; otherwise, if
it is accessible in the current package, it is printed without any
qualification; otherwise, it is printed with qualification.
See chapter~\ref{XPACK}.

\begin{obsolete}
A symbol that is uninterned (has no home package) is printed
preceded by \cd{\#:} if the variables \cd{*print-gensym*}
and \cdf{*print-escape*} are both non-{\nil};
if either is {\nil}, then the symbol is printed without
a prefix, as if it were in the current package.
\end{obsolete}
\begin{newer}
X3J13 voted in June 1989 \issue{DATA-IO} to specify that if \cd{*print-readably*}
is not {\false} then every object must be printed in a readable form,
regardless of other printer control variables.  For uninterned symbols, the simplest approach
is to print them, when \cd{*print-readably*} is not {\false}, as if \cdf{*print-escape*}
and \cd{*print-gensym*}
were not {\false}, regardless of their actual values.
\end{newer}

\beforenoterule
\begin{implementation}
Because the \cd{\#:} syntax does not intern the
following symbol, it is necessary to use circular-list syntax
if \cdf{*print-circle*} is not {\false} and
the same uninterned symbol appears several times in an expression
to be printed.  For example, the result of
\begin{lisp}
(let ((x (make-symbol "FOO"))) (list x x))
\end{lisp}
would be printed as
\begin{lisp}
\cd{(\#:foo \#:foo)}
\end{lisp}
if \cdf{*print-circle*}
were {\false}, but as
\begin{lisp}
\cd{(\#1{\Xequal}\#:foo \#1\#)}
\end{lisp}
if \cdf{*print-circle*}
were not {\false}.
\end{implementation}
\afternoterule

\begin{obsolete}
The case in which symbols are to be printed is controlled by the variable
\cdf{*print-case*}.
\end{obsolete}
\begin{newer}
It is also controlled by \cdf{*print-escape*} and the \cdf{readtable-case} slot of the current
readtable (the value of \cd{*readtable*}).
\end{newer}
\begin{obsolete}
\item[\emph{Strings}]
The characters of the string are output in order.
If \cdf{*print-escape*} is not {\false}, a double quote
is output before and after, and all
double quotes and single escape characters are preceded by backslash.
The printing of strings is not affected by \cd{*print-array*}.
If the string has a fill pointer, then only those characters below
the fill pointer are printed.
\end{obsolete}

\begin{newer}
X3J13 voted in June 1989 \issue{DATA-IO} to specify that if \cd{*print-readably*}
is not {\false} then every object must be printed in a readable form,
regardless of other printer control variables.  For strings, the simplest approach
is to print them, when \cd{*print-readably*} is not {\false}, as if \cdf{*print-escape*}
were not {\false}, regardless of the actual value of \cdf{*print-escape*}.
\end{newer}

\item[\emph{Conses}]
Wherever possible, list notation is preferred over dot
notation.  Therefore the following algorithm is used:
\begin{enumerate}
\item Print an open parenthesis, \cd{(}.
\item Print the \emph{car} of the cons.
\item If the \emph{cdr} is a cons, make it the current cons, print a space, and go to step 2.
\item If the \emph{cdr} is not null, print a space, a dot, a space, and the \emph{cdr}.
\item Print a close parenthesis, \cd{)}.
\end{enumerate}

This form of printing is clearer than showing each individual cons
cell.  Although the two expressions below are equivalent,
and the reader will accept
either one and produce the same data structure, the printer will
always print such a data structure in the second form.
\begin{lisp}
(a . (b . ((c . (d . {\nil})) . (e . {\nil})))) \\
\\
(a b (c d) e)
\end{lisp}
\begin{obsolete}
The printing of conses is affected by the variables \cd{*print-level*}
and \cd{*print-length*}.
\end{obsolete}

\begin{newer}
X3J13 voted in June 1989 \issue{DATA-IO} to specify that if \cd{*print-readably*}
is not {\false} then every object must be printed in a readable form,
regardless of other printer control variables.  For conses, the simplest approach
is to print them, when \cd{*print-readably*} is not {\false}, as if \cd{*print-level*}
and \cd{*print-length*} were {\false}, regardless of their actual values.
\end{newer}

\begin{obsolete}
\item[\emph{Bit-vectors}]
A bit-vector is printed as \cd{\#*} followed by the bits of the bit-vector
in order.  If \cd{*print-array*} is {\false}, however, then the bit-vector is
printed in a format (using \cd{\#<}) that is concise but not readable.
If the bit-vector has a fill pointer, then only those bits below
the fill pointer are printed.
\end{obsolete}
\begin{newer}
X3J13 voted in June 1989 \issue{DATA-IO} to specify that if \cd{*print-readably*}
is not {\false} then every object must be printed in a readable form,
regardless of other printer control variables.  For bit-vectors, the simplest approach
is to print them, when \cd{*print-readably*} is not {\false}, as if \cd{*print-array*}
were not {\false}, regardless of the actual value of \cd{*print-array*}.
\end{newer}

\item[\emph{Vectors}]
Any vector other than a string or bit-vector is printed using
general-vector syntax; this means that information
about specialized vector representations will be lost.
The printed representation of a zero-length vector is \cd{\#()}.  The
printed representation of a non-zero-length vector begins with \cd{\#(}.
Following that, the first element of the vector is printed.  If
there are any other elements, they are printed in turn, with a space
printed before each additional element.  A close parenthesis
after the
last element terminates the printed representation of the vector.
\begin{obsolete}
The
printing of vectors is affected by the variables \cd{*print-level*} and
\cd{*print-length*}.
If the vector has a fill pointer, then only those elements below
the fill pointer are printed.

If \cd{*print-array*} is {\false}, however, then the vector is not printed
as described above, but
in a format (using \cd{\#<}) that is concise but not readable.
\end{obsolete}
\begin{newer}
X3J13 voted in June 1989 \issue{DATA-IO} to specify that if \cd{*print-readably*}
is not {\false} then every object must be printed in a readable form,
regardless of other printer control variables.  For vectors, the simplest approach
is to print them, when \cd{*print-readably*} is not {\false}, as if \cd{*print-level*}
and \cd{*print-length*} were {\false} and \cd{*print-array*} were not {\false},
regardless of their actual values.
\end{newer}

\item[\emph{Arrays}]
Normally any array other than a vector is printed
using \cd{\#\emph{n}A} format.  Let \emph{n} be the rank of the array.
Then \cd{\#} is printed, then \emph{n} as a decimal integer,
then \cdf{A}, then \emph{n} open parentheses.  Next the elements
are scanned in row-major order.  Imagine the array indices being
enumerated in odometer fashion, recalling that the dimensions
are numbered from 0 to $\emph{n}-1$.  Every time the index for
dimension \emph{j} is incremented, the following actions are taken:
\begin{enumerate}
\item
If $\emph{j}<\emph{n}-1$, then print a close parenthesis.

\item
If incrementing the index for dimension \emph{j} caused it to equal
dimension \emph{j}, reset that index to zero and increment dimension
$\emph{j}-1$ (thereby performing these three steps recursively),
unless $\emph{j}=0$, in which case simply terminate the entire algorithm.
If incrementing the index for dimension \emph{j} did not cause it to
equal dimension \emph{j}, then print a space.

\item
If $\emph{j}<\emph{n}-1$, then print an open parenthesis.
\end{enumerate}
This causes the contents to be printed in a format suitable for
use as the \cd{:initial-\discretionary{}{}{}contents} argument to \cdf{make-array}.
\begin{obsolete}
The lists effectively printed by this procedure are subject to
truncation by \cd{*print-level*} and \cd{*print-length*}.
\end{obsolete}

If the array is of a specialized type, containing bits or string-characters,
then the innermost lists generated by the algorithm given above may instead
be printed using bit-vector or string syntax, provided that these innermost
lists would not be subject to truncation by \cd{*print-length*}.  For example,
a 3-by-2-by-4 array of string-characters that would ordinarily be printed as
\begin{lisp}
\#3A(((\#{\Xbackslash}s \#{\Xbackslash}t \#{\Xbackslash}o \#{\Xbackslash}p) (\#{\Xbackslash}s \#{\Xbackslash}p \#{\Xbackslash}o \#{\Xbackslash}t)) \\
~~~~((\#{\Xbackslash}p \#{\Xbackslash}o \#{\Xbackslash}s \#{\Xbackslash}t) (\#{\Xbackslash}p \#{\Xbackslash}o \#{\Xbackslash}t \#{\Xbackslash}s)) \\
~~~~((\#{\Xbackslash}t \#{\Xbackslash}o \#{\Xbackslash}p \#{\Xbackslash}s) (\#{\Xbackslash}o \#{\Xbackslash}p \#{\Xbackslash}t \#{\Xbackslash}s)))
\end{lisp}
may instead be printed more concisely as
\begin{lisp}
\#3A(("stop" "spot") ("post" "pots") ("tops" "opts"))
\end{lisp}

\begin{obsolete}
If \cd{*print-array*} is {\false}, then the array is printed
in a format (using \cd{\#<}) that is concise but not readable.
\end{obsolete}
\begin{newer}
X3J13 voted in June 1989 \issue{DATA-IO} to specify that if \cd{*print-readably*}
is not {\false} then every object must be printed in a readable form,
regardless of other printer control variables.  For arrays, the simplest approach
is to print them, when \cd{*print-readably*} is not {\false}, as if \cd{*print-level*}
and \cd{*print-length*} were {\false} and \cd{*print-array*} were not {\false},
regardless of their actual values.
\end{newer}

\item[\emph{Random-states}]
Common Lisp does not specify a specific syntax
for printing objects of type \cdf{random-state}.  However, every implementation
must arrange to print a random-state object in such a way that,
within the same implementation of Common Lisp, the function \cdf{read}
can construct from the printed representation a copy of the random-state
object as if the copy had been made by \cdf{make-random-state}.

\item[\emph{Pathnames}]
If \cdf{*print-escape*} is true, a pathname
should be printed by \cdf{write} as \cd{\#P"..."} where \cd{"..."} is the
namestring representation of the pathname.  If \cdf{*print-escape*}
is false, \cdf{write} prints a pathname by printing its namestring
(presumably without escape characters or surrounding double quotes).

If \cd{*print-readably*}
is not {\false} then every object must be printed in a readable form,
regardless of other printer control variables.  For pathnames, the simplest approach
is to print them, when \cd{*print-readably*} is not {\false}, as if \cdf{*print-escape*}
were {\false},
regardless of its actual value.
\end{flushdesc}

Structures defined by \cdf{defstruct} are printed under the
control of the user-specified \cd{:print-function} option to \cdf{defstruct}.
If the user does not provide a printing function explicitly,
then a default printing function is supplied that prints the structure
using \cd{\#S} syntax (see section~\ref{SHARP-SIGN-MACRO-CHARACTER-SECTION}).

If \cd{*print-readably*}
is not {\false} then every object must be printed in a readable form,
regardless of the values of other printer control variables; if this is not possible,
then an error of type \cdf{print-not-readable} must be signaled to avoid
printing an unreadable syntax such as \cd{\#<...>}.

Macro \cdf{print-unreadable-object} prints an object using \cd{\#<...>}
syntax and also takes care of checking the variable \cd{*print-readably*}.

When debugging or when frequently dealing with large or
deep objects at top level, the user may wish to restrict the printer
from printing large amounts of information.  The variables
\cd{*print-level*} and \cd{*print-length*} allow the user to control how deep
the printer will print and how many elements at a given level the
printer will print.  Thus the user can see enough of the object to
identify it without having to wade through the entire expression.

\begin{defun}[Variable]
*print-readably*

The default value of \cd{*print-readably*} is
  \cdf{nil}.  If \cd{*print-readably*} is true, then printing any object must either
  produce a
  printed representation that the reader will accept or signal an error.
  If printing is successful, the reader will, on reading the printed representation,
  produce an object that is ``similar as a constant''
  (see section~\ref{SIMILAR-AS-A-CONSTANT-SECTION}) to the object that was printed.

  If \cd{*print-readably*} is true and printing a readable printed
  representation is not possible, the printer signals an error of type
  \cdf{print-not-readable} rather than using an unreadable syntax such as \cd{\#<}.
  The printed representation produced when \cd{*print-readably*} is true might
  or might not be the same as the printed representation produced when
  \cd{*print-readably*} is false.

  If \cd{*print-readably*} is true and another printer control variable
  (such as \cd{*print-length*}, \cd{*print-level*}, \cdf{*print-escape*}, \cd{*print-gensym*},
  \cd{*print-\discretionary{}{}{}array*}, or an implementation-defined printer control variable)
  would cause the preceding requirements to be violated, that other
  printer control variable is ignored.

  The printing of interned symbols is not affected by \cd{*print-readably*}.

  Note that the ``similar as a constant'' rule for readable printing
  implies that \cd{\#A} or \cd{\#(} syntax cannot be used for arrays of element-type
  other than \cdf{t}.  An implementation will have to use another syntax or
  signal a \cdf{print-not-readable} error.  A \cdf{print-not-readable} error will not
  be signaled for strings or bit-vectors.

  All methods for \cdf{print-object} must obey \cd{*print-readably*}.  This
  rule applies to both user-defined methods and implementation-defined methods.

  The reader control variable \cd{*read-eval*} also affects printing.
  If \cd{*read-eval*} is false and \cd{*print-readably*} is true, any \cdf{print-object}
  method that would otherwise output a \cd{\#.} reader macro must either output something
  different or signal an error of type \cdf{print-not-readable}.

  Readable printing of structures and objects of type \cdf{standard-object}
  is controlled
  by their \cdf{print-object} methods, not by their \cdf{make-load-form} methods.
  ``Similarity as a constant'' for these objects is application-dependent
  and hence is defined to be whatever these methods do.

  \cd{*print-readably*} allows errors involving data with no
  readable printed representation to be detected when writing the file rather than
  later on when the file is read.

  \cd{*print-readably*} is more rigorous than \cdf{*print-escape*}; output printed
  with escapes must be merely generally recognizable by humans, with a good chance
  of being recognizable by computers, whereas
  output printed readably must be reliably recognizable by computers.
\end{defun}

\begin{defun}[Variable]
*print-escape*

When this flag is {\false}, then escape characters are not output
when an expression is printed.  In particular, a symbol is printed
by simply printing the characters of its print name.
The function \cdf{princ} effectively binds \cdf{*print-escape*} to {\false}.

When this flag is not {\false}, then an attempt is made to print an
expression in such a way that it can be read again to produce an
\cdf{equal} structure.
The function \cd{prin1} effectively binds \cdf{*print-escape*} to {\true}.
The initial value of this variable is {\true}.
\end{defun}

\begin{defun}[Variable]
*print-pretty*

When this flag is {\false}, then only a small amount of whitespace is
output when printing an expression.

When this flag is not {\false}, then the printer will endeavor to insert
extra whitespace where appropriate to make the expression more readable.
A few other simple changes may be made, such as printing \cd{'foo}
instead of \cd{(quote foo)}.

The initial value of \cdf{*print-pretty*} is implementation-dependent.

\begin{new}
X3J13 voted in January 1989
\issue{PRETTY-PRINT-INTERFACE}
to adopt a facility for user-controlled pretty printing
in Common Lisp
(see chapter~\ref{PPRINT}).
\end{new}
\end{defun}

\begin{defun}[Variable]
*print-circle*

When this flag is {\false} (the default), then the printing process proceeds
by recursive descent; an attempt to print a circular structure may lead
to looping behavior and failure to terminate.

If \cdf{*print-circle*} is true, the printer is required
to detect not only cycles but shared substructure, indicating both through the
use of \cd{\#\emph{n\/}=} and \cd{\#\emph{n\/}\#} syntax.
As an example, under the specification of the first edition
\begin{lisp}
(print '(\#1=(a \#1\#) \#1\#))
\end{lisp}
might legitimately print \cd{(\#1=(A \#1\#) \#1\#)} or
\cd{(\#1=(A \#1\#) \#2=(A \#2\#))}; the vote specifies that the first
form is required.

User-defined printing functions for the \cdf{defstruct}
\cd{:print-function} option, as well as user-defined methods for the
CLOS generic function \cdf{print-object}, may print objects to the
supplied stream using \cdf{write}, \cd{print1}, \cdf{princ}, \cdf{format},
or \cdf{print-object} and expect circularities to be detected and printed
using \cd{\#\emph{n\/}\#} syntax (when \cdf{*print-circle*} is non-\cdf{nil}, of course).

It seems to me that the same ought to apply to abbreviation
as controlled by \cd{*print-level*} and \cd{*print-length*}, but that
was not addressed by this vote.
\end{defun}

\begin{defun}[Variable]
*print-base*

The value of \cd{*print-base*} determines in what radix the printer will print
rationals.  This may be any integer from \cd{2} to \cd{36}, inclusive;
the default value is \cd{10} (decimal radix).
For radices above \cd{10}, letters of the alphabet are used to represent
digits above \cd{9}.
\end{defun}

\begin{defun}[Variable]
*print-radix*

If the variable \cd{*print-radix*} is non-{\false}, the printer will print a
radix specifier to indicate the radix in which it is printing a rational
number.  To prevent confusion of the letter \cdf{O} with the digit \cd{0},
and of the letter \cdf{B} with the digit \cd{8}, the radix specifier
is always printed using lowercase letters.
For example, if the current base is twenty-four (decimal), the
decimal integer twenty-three would print as \cd{\#24rN}.  If \cd{*print-base*}
is \cd{2}, \cd{8}, or \cd{16}, then the radix specifier used is \cd{\#b},
\cd{\#o}, or \cd{\#x}.  For integers, base ten is indicated by a trailing
decimal point instead of a leading radix specifier; for ratios, however,
\cd{\#10r} is used.  The default value of \cd{*print-radix*} is {\false}.
\end{defun}

\begin{defun}[Variable]
*print-case*

The \cdf{read} function normally converts lowercase characters
appearing in symbols to corresponding uppercase characters,
so that internally print
names normally contain only uppercase characters.
However, users may prefer to see output using lowercase letters
or letters of mixed case.
This variable controls the case (upper, lower, or mixed) in which to print
any uppercase characters in the names of symbols
when vertical-bar syntax is not used.
The value of \cdf{*print-case*} should be one of the keywords
\cd{:upcase}, \cd{:downcase}, or \cd{:capitalize};
the initial value is \cd{:upcase}.

Lowercase characters in the internal print name
are always printed in lowercase, and are
preceded by a single escape character or enclosed by multiple
escape characters.
Uppercase characters in the internal print name
are printed in uppercase, in lowercase, or in mixed case
so as to capitalize words, according to the value of
\cdf{*print-case*}.  The convention for what constitutes
a ``word'' is the same as for the function \cdf{string-capitalize}.

\begin{newer}
X3J13 voted in June 1989 \issue{PRINT-CASE-PRINT-ESCAPE-INTERACTION}
to clarify the interaction of \cdf{*print-case*} with \cdf{*print-escape*}.
When \cdf{*print-escape*} is \cdf{nil}, \cdf{*print-case*} determines the
case in which to print all uppercase characters in the print name of the symbol.
When \cdf{*print-escape*} is not \cdf{nil}, the implementation has some freedom
as to which characters will be printed so as to appear in an ``escape context''
(after an escape character, typically~\cd{\Xbackslash}, or between multiple escape characters,
typically~\cd{|}); \cdf{*print-case*} determines the
case in which to print all uppercase characters that will not appear in an escape context.
For example, when the value of \cdf{*print-case*} is \cd{:upcase},
an implementation might choose to print the symbol whose print name is \cd{"(S)HE"}
as \cd{{\Xbackslash}(S{\Xbackslash})HE} or as \cd{|(S)HE|}, among other possibilities.
When the value of \cdf{*print-case*} is \cd{:downcase}, the corresponding output
should be \cd{{\Xbackslash}(s{\Xbackslash})he} or \cd{|(S)HE|}, respectively.

Consider the following test code.  (For the sake of this example
assume that \cdf{readtable-case} is \cd{:upcase} in the current readtable; this is
discussed further below.)
\begin{lisp}
(let ((tabwidth 11)) \\*
~~(dolist (sym '(|x| |FoObAr| |fOo|)) \\*
~~~~(let ((tabstop -1)) \\
~~~~~~(format t "{\Xtilde}\&") \\*
~~~~~~(dolist (escape '(t nil)) \\
~~~~~~~~(dolist (case '(:upcase :downcase :capitalize)) \\
~~~~~~~~~~(format t "{\Xtilde}VT" (* (incf tabstop) tabwidth)) \\*
~~~~~~~~~~(write sym :escape escape :case case))))) \\*
~~(format t "~\%"))
\end{lisp}
An implementation that leans heavily on multiple-escape characters (vertical bars)
might produce the following output:
\begin{lisp}
|x|~~~~~~~~|x|~~~~~~~~|x|~~~~~~~~x~~~~~~~~~~x~~~~~~~~~~x \\*
|FoObAr|~~~|FoObAr|~~~|FoObAr|~~~FoObAr~~~~~foobar~~~~~Foobar \\*
|fOo|~~~~~~|fOo|~~~~~~|fOo|~~~~~~fOo~~~~~~~~foo~~~~~~~~foo
\end{lisp}
An implementation that leans heavily on single-escape characters (backslashes)
might produce the following output:
\begin{lisp}
{\Xbackslash}x~~~~~~~~~{\Xbackslash}x~~~~~~~~~{\Xbackslash}x~~~~~~~~~x~~~~~~~~~~x~~~~~~~~~~x \\*
F{\Xbackslash}oO{\Xbackslash}bA{\Xbackslash}r~~f{\Xbackslash}oo{\Xbackslash}ba{\Xbackslash}r~~F{\Xbackslash}oo{\Xbackslash}ba{\Xbackslash}r~~FoObAr~~~~~foobar~~~~~Foobar \\*
{\Xbackslash}fO{\Xbackslash}o~~~~~~{\Xbackslash}fo{\Xbackslash}o~~~~~~{\Xbackslash}fo{\Xbackslash}o~~~~~~fOo~~~~~~~~foo~~~~~~~~foo
\end{lisp}
These examples are not exhaustive; output using both kinds of escape characters
(for example, \cd{|FoO|{\Xbackslash}bA{\Xbackslash}r}) is permissible (though ugly).

X3J13 voted in June 1989 \issue{READ-CASE-SENSITIVITY}
to add a new \cdf{readtable-case} slot to readtables to control
automatic case conversion during the reading of symbols.
The value of \cdf{readtable-case} in the current readtable also affects the printing
of unescaped letters (letters appearing in an escape context are always
printed in their own case).
\begin{itemize}
\item If \cdf{readtable-case} is \cd{:upcase}, unescaped uppercase letters are printed
    in the case specified by \cdf{*print-case*} and unescaped lowercase letters
    are printed in their own case.  (If \cdf{*print-escape*} is non-{\false},
    all lowercase letters will necessarily be escaped.)

\item If \cdf{readtable-case} is \cd{:downcase}, unescaped lowercase letters are printed
    in the case specified by \cdf{*print-case*} and unescaped uppercase letters
    are printed in their own case.  (If \cdf{*print-escape*} is non-{\false},
    all uppercase letters will necessarily be escaped.)

\item  If \cdf{readtable-case} is \cd{:preserve},
    all unescaped letters are printed in their own case,
    regardless of the value of \cdf{*print-case*}.  There is no
    need to escape any letters, even if \cdf{*print-escape*} is non-{\false},
    though the X3J13 vote did not prohibit escaping letters in this situation.

\item If \cdf{readtable-case} is \cd{:invert},
    and if all unescaped letters are of the same
    case, then the case of all the unescaped letters is inverted; but if the unescaped
    letters are not all of the same case then each is printed in its own case.
    (Thus \cd{:invert} does not always invert the case; the inversion is conditional.)
    There is no
    need to escape any letters, even if \cdf{*print-escape*} is non-{\false},
    though the X3J13 vote did not prohibit escaping letters in this situation.
\end{itemize}
Consider the following code.
\begin{lisp}
;;; Generate a table illustrating READTABLE-CASE and *PRINT-CASE*. \\*
\\*
(let ((*readtable* (copy-readtable nil)) \\*
~~~~~~(*print-case* *print-case*)) \\*
~~(format t "READTABLE-CASE~*PRINT-CASE*~~Symbol-name~~Output{\Xtilde} \\*
~~~~~~~~~~~{\Xtilde}\%\hbox to 0pt{-------------------------\hss}~~~~~~~~~~~~~~~~~~~~~~~~~-------------------------{\Xtilde} \\*
~~~~~~~~~~~{\Xtilde}\%") \\
~~(dolist (readtable-case '(:upcase :downcase :preserve :invert)) \\*
~~~~(setf (readtable-case *readtable*) readtable-case) \\*
~~~~(dolist (print-case '(:upcase :downcase :capitalize)) \\*
~~~~~~(dolist (sym '(|ZEBRA| |Zebra| |zebra|)) \\*
~~~~~~~~(setq *print-case* print-case) \\
~~~~~~~~(format t ":{\Xtilde}A{\Xtilde}15T:{\Xtilde}A{\Xtilde}29T{\Xtilde}A{\Xtilde}42T{\Xtilde}A{\Xtilde}\%" \\*
~~~~~~~~~~~~~~~~~~(string-upcase readtable-case) \\*
~~~~~~~~~~~~~~~~~~(string-upcase print-case) \\*
~~~~~~~~~~~~~~~~~~(symbol-name sym) \\*
~~~~~~~~~~~~~~~~~~(prin1-to-string sym)))))))
\end{lisp}

\vskip 0pt plus 3pt
\noindent
Note that the call to \cd{prin1-to-string}
(the last argument in the call to \cdf{format} that is within the nested loops)
effectively uses a non-{\false} value for \cdf{*print-escape*}.

Assuming an implementation that uses vertical bars around a symbol name
if any characters need escaping,
the output from this test code should be
\def\foo{-6pt}
\begin{lisp}
READTABLE-CASE *PRINT-CASE*~~Symbol-name~~Output \\*
-------------------------------------------------- \\*
:UPCASE~~~~~~~~:UPCASE~~~~~~~ZEBRA~~~~~~~~ZEBRA \\*
:UPCASE~~~~~~~~:UPCASE~~~~~~~Zebra~~~~~~~~|Zebra| \\*
:UPCASE~~~~~~~~:UPCASE~~~~~~~zebra~~~~~~~~|zebra| \\*
:UPCASE~~~~~~~~:DOWNCASE~~~~~ZEBRA~~~~~~~~zebra \\*
:UPCASE~~~~~~~~:DOWNCASE~~~~~Zebra~~~~~~~~|Zebra| \\*
:UPCASE~~~~~~~~:DOWNCASE~~~~~zebra~~~~~~~~|zebra| \\
:UPCASE~~~~~~~~:CAPITALIZE~~~ZEBRA~~~~~~~~Zebra \\*
:UPCASE~~~~~~~~:CAPITALIZE~~~Zebra~~~~~~~~|Zebra| \\*
:UPCASE~~~~~~~~:CAPITALIZE~~~zebra~~~~~~~~|zebra| \\
:DOWNCASE~~~~~~:UPCASE~~~~~~~ZEBRA~~~~~~~~|ZEBRA| \\*
:DOWNCASE~~~~~~:UPCASE~~~~~~~Zebra~~~~~~~~|Zebra| \\*
:DOWNCASE~~~~~~:UPCASE~~~~~~~zebra~~~~~~~~ZEBRA \\
:DOWNCASE~~~~~~:DOWNCASE~~~~~ZEBRA~~~~~~~~|ZEBRA| \\*
:DOWNCASE~~~~~~:DOWNCASE~~~~~Zebra~~~~~~~~|Zebra| \\*
:DOWNCASE~~~~~~:DOWNCASE~~~~~zebra~~~~~~~~zebra \\
:DOWNCASE~~~~~~:CAPITALIZE~~~ZEBRA~~~~~~~~|ZEBRA| \\*
:DOWNCASE~~~~~~:CAPITALIZE~~~Zebra~~~~~~~~|Zebra| \\*
:DOWNCASE~~~~~~:CAPITALIZE~~~zebra~~~~~~~~Zebra \\
:PRESERVE~~~~~~:UPCASE~~~~~~~ZEBRA~~~~~~~~ZEBRA \\*
:PRESERVE~~~~~~:UPCASE~~~~~~~Zebra~~~~~~~~Zebra \\*
:PRESERVE~~~~~~:UPCASE~~~~~~~zebra~~~~~~~~zebra \\
:PRESERVE~~~~~~:DOWNCASE~~~~~ZEBRA~~~~~~~~ZEBRA \\*
:PRESERVE~~~~~~:DOWNCASE~~~~~Zebra~~~~~~~~Zebra \\*
:PRESERVE~~~~~~:DOWNCASE~~~~~zebra~~~~~~~~zebra \\
:PRESERVE~~~~~~:CAPITALIZE~~~ZEBRA~~~~~~~~ZEBRA \\*
:PRESERVE~~~~~~:CAPITALIZE~~~Zebra~~~~~~~~Zebra \\*
:PRESERVE~~~~~~:CAPITALIZE~~~zebra~~~~~~~~zebra \\
:INVERT~~~~~~~~:UPCASE~~~~~~~ZEBRA~~~~~~~~zebra \\*
:INVERT~~~~~~~~:UPCASE~~~~~~~Zebra~~~~~~~~Zebra \\*
:INVERT~~~~~~~~:UPCASE~~~~~~~zebra~~~~~~~~ZEBRA \\
:INVERT~~~~~~~~:DOWNCASE~~~~~ZEBRA~~~~~~~~zebra \\*
:INVERT~~~~~~~~:DOWNCASE~~~~~Zebra~~~~~~~~Zebra \\*
:INVERT~~~~~~~~:DOWNCASE~~~~~zebra~~~~~~~~ZEBRA \\
:INVERT~~~~~~~~:CAPITALIZE~~~ZEBRA~~~~~~~~zebra \\*
:INVERT~~~~~~~~:CAPITALIZE~~~Zebra~~~~~~~~Zebra \\*
:INVERT~~~~~~~~:CAPITALIZE~~~zebra~~~~~~~~ZEBRA \\[\foo]
\end{lisp}

This illustrates all combinations for
\cdf{readtable-case} and \cdf{*print-case*}.
\end{newer}
\end{defun}

\penalty-10000 %required

\begin{defun}[Variable]
*print-gensym*

The \cd{*print-gensym*} variable controls whether the prefix \cd{\#:}
is printed before symbols that have no home package.
The prefix is printed if the variable is not {\false}.
The initial value of \cd{*print-gensym*} is {\true}.
\end{defun}

\begin{table}[t]
\caption{Examples of Print Level and Print Length Abbreviation}
\label{LEVEL-LENGTH-TABLE}
\begin{lisp}
\begin{tabular*}{\textwidth}{@{}l@{\extracolsep{\fill}}@{}ll@{}}
\emph{v}&\emph{n}&Output \\
\hlinesp
0&1&\# \\
1&1&(if ...) \\
1&2&(if \# ...) \\
1&3&(if \# \# ...) \\
1&4&(if \# \# \#) \\
2&1&(if ...) \\
2&2&(if (member x ...) ...) \\
2&3&(if (member x y) (+ \# 3) ...) \\
3&2&(if (member x ...) ...) \\
3&3&(if (member x y) (+ (car x) 3) ...) \\
3&4&(if (member x y) (+ (car x) 3) '(foo . \#(a b c d ...))) \\
3&5&(if (member x y) (+ (car x) 3) '(foo . \#(a b c d "Baz")))
\end{tabular*}
\end{lisp}
\end{table}

\begin{defun}[Variable]
*print-level* \\
*print-length*

The \cd{*print-level*} variable controls how many levels deep a nested
data object will print.
If \cd{*print-level*} is {\false} (the initial value), then no control is exercised.
Otherwise, the value should be an integer, indicating the maximum level to
be printed.  An object to be printed is at level \cd{0};
its components (as of a list or vector) are at level \cd{1}; and so on.
If an object to be recursively printed has components and is at a level
equal to or greater than the value of \cd{*print-level*}, then the object
is printed as simply \cd{\#}.

The \cd{*print-length*} variable controls how many elements at a given level
are printed.  A value of {\false} (the initial value) indicates that there
be no limit to the number of components printed.  Otherwise, the value of
\cd{*print-length*} should be an integer.  Should the number of elements of a
data object exceed the value \cd{*print-length*}, the printer will print three
dots, \cd{...}, in place of those elements beyond the number specified
by \cd{*print-length*}.  (In the case of a dotted list, if the list contains
exactly as many elements as the value of \cd{*print-length*}, and in addition
has the non-null atom terminating it, that terminating atom is printed
rather than the three dots.)

\cd{*print-level*} and \cd{*print-length*} affect the printing not only of lists
but also of vectors, arrays, and any other object printed with
a list-like syntax.  They do not affect the printing of symbols,
strings, and bit-vectors.

The Lisp reader will normally signal an error when reading
an expression that has been abbreviated because of level or length limits.
This signal is given because the \cd{\#} dispatch character normally signals
an error when followed by whitespace or \cd{)}, and because \cd{...}
is defined to be an illegal token, as are all tokens consisting
entirely of periods (other than the single dot used in dot notation).

As an example, table~\ref{LEVEL-LENGTH-TABLE} shows the ways the object
\begin{lisp}
(if (member x y) (+ (car x) 3) '(foo . \#(a b c d "Baz")))
\end{lisp}
would be printed for various values of \cd{*print-level*}
(in the column labeled \emph{v}) and \cd{*print-length*} (in the column labeled \emph{n}).
\end{defun}

\begin{defun}[Variable]
*print-array*

If \cd{*print-array*} is {\false}, then the contents of arrays other than strings
are never printed.  Instead, arrays are printed in a concise form (using
\cd{\#<}) that gives enough information for the user to be able to
identify the array but does not include the entire array contents.
If \cd{*print-array*} is not {\false}, non-string arrays are printed using
\cd{\#(}, \cd{\#*}, or \cd{\#\emph{n}A} syntax.
\begin{new}%CORR
\emph{Notice of correction.}
In the first edition, the preceding paragraph mentioned the nonexistent
variable \cdf{print-array} instead of \cd{*print-array*}.
\end{new}
The initial value of \cd{*print-array*} is implementation-dependent.
\end{defun}

\begin{defmac}
with-standard-io-syntax {declaration}* {form}*
 
Within the dynamic extent of the body, all reader/printer control
    variables, including any implementation-defined ones not specified by
    Common Lisp, are bound to values that produce standard read/print
    behavior.  Table~\ref{WITH-STANDARD-IO-SYNTAX-TABLE} shows
    the values to which standard Common Lisp variables are bound.

\begin{table}[t]
\caption{Standard Bindings for I/O Control Variables}
\label{WITH-STANDARD-IO-SYNTAX-TABLE}
\begin{flushleft}
\cf
\begin{tabular}{@{}ll@{}}
\textrm{Variable}&\textrm{Value} \\
\hlinesp
      {*package*}                      &     \textrm{the \cdf{common-lisp-user} package} \\
      {*print-array*}                  &     t \\
      {*print-base*}                   &     10 \\
      {*print-case*}                   &     :upcase \\
      {*print-circle*}                 &     nil \\
      {*print-escape*}                 &     t \\
      {*print-gensym*}                 &     t \\
      {*print-length*}                 &     nil \\
      {*print-level*}                  &     nil \\
      {*print-lines*}                  &     nil \textrm{*} \\
      {*print-miser-width*}            &     nil \textrm{*} \\
      {*print-pprint-dispatch*}        &     nil \textrm{*} \\
      {*print-pretty*}                 &     nil \\
      {*print-radix*}                  &     nil \\
      {*print-readably*}               &     t \\
      {*print-right-margin*}           &     nil \textrm{*} \\
      {*read-base*}                    &     10 \\
      {*read-default-float-format*}    &     single-float \\
      {*read-eval*}                    &     t \\
      {*read-suppress*}                &     nil \\
      {*readtable*}                    &     \textrm{the standard readtable}
\end{tabular}
\end{flushleft}
* X3J13 voted in June 1989 \issue{PRETTY-PRINT-INTERFACE}
to introduce the printer control variables
\cdf{*print-right-margin*},
\cdf{*print-miser-width*},
\cdf{*print-lines*},
and \cdf{*print-pprint-dispatch*}
(see section~\ref{PPRINT-VARIABLES-SECTION})
but did not specify the values to which \cdf{with-standard-io-syntax}
should bind them.  I recommend that all four should be bound to \cdf{nil}.
\end{table}


\begin{table}[t]
\caption{Основные связывания для переменных для ввода/вывода}
\label{WITH-STANDARD-IO-SYNTAX-TABLE}
\begin{flushleft}
\cf
\begin{tabular}{@{}ll@{}}
\textrm{Переменная}&\textrm{Значение} \\
\hlinesp
      {*package*}                      &     \textrm{пакет \cdf{common-lisp-user}} \\
      {*print-array*}                  &     t \\
      {*print-base*}                   &     10 \\
      {*print-case*}                   &     :upcase \\
      {*print-circle*}                 &     nil \\
      {*print-escape*}                 &     t \\
      {*print-gensym*}                 &     t \\
      {*print-length*}                 &     nil \\
      {*print-level*}                  &     nil \\
      {*print-lines*}                  &     nil \textrm{*} \\
      {*print-miser-width*}            &     nil \textrm{*} \\
      {*print-pprint-dispatch*}        &     nil \textrm{*} \\
      {*print-pretty*}                 &     nil \\
      {*print-radix*}                  &     nil \\
      {*print-readably*}               &     t \\
      {*print-right-margin*}           &     nil \textrm{*} \\
      {*read-base*}                    &     10 \\
      {*read-default-float-format*}    &     single-float \\
      {*read-eval*}                    &     t \\
      {*read-suppress*}                &     nil \\
      {*readtable*}                    &     \textrm{стандартная таблица с макросимволами}
\end{tabular}
\end{flushleft}
* X3J13 voted in June 1989 \issue{PRETTY-PRINT-INTERFACE}
to introduce the printer control variables
\cdf{*print-right-margin*},
\cdf{*print-miser-width*},
\cdf{*print-lines*},
and \cdf{*print-pprint-dispatch*}
(see section~\ref{PPRINT-VARIABLES-SECTION})
but did not specify the values to which \cdf{with-standard-io-syntax}
should bind them.  I recommend that all four should be bound to \cdf{nil}.
\end{table}

    The values returned by \cdf{with-standard-io-syntax} are the values
    of the last body \emph{form}, or \cdf{nil} if there are no body forms.

The intent is that a pair of executions, as shown in the following example,
should provide reasonable reliable communication of data from
one Lisp process to another:
\begin{lisp}
;;; Write DATA to a file. \\*
(with-open-file (file pathname :direction :output) \\*
~~(with-standard-io-syntax \\*
~~~~(print data file))) \\
\\
;;; ...~~Later, in another Lisp: \\*
(with-open-file (file pathname :direction :input) \\*
~~(with-standard-io-syntax \\*
~~~~(setq data (read file))))
\end{lisp}

Using \cdf{with-standard-io-syntax} to bind all the variables,
  instead of using \cdf{let} and explicit bindings,
  ensures that nothing is overlooked and avoids problems with
  implementation-defined reader/printer control variables.
  If the user wishes to use a non-standard value for some variable, such as
  \cdf{*package*} or \cd{*read-eval*}, it can be bound by \cdf{let} inside the body of
  \cdf{with-standard-io-syntax}.  For example:
\begin{lisp}
;;; Write DATA to a file.  Forbid use of \#. syntax. \\*
(with-open-file (file pathname :direction :output) \\*
~~(let ((*read-eval* nil)) \\*
~~~~(with-standard-io-syntax \\*
~~~~~~(print data file)))) \\
\\
;;; Read DATA from a file.  Forbid use of \#. syntax. \\*
(with-open-file (file pathname :direction :input) \\*
~~(let ((*read-eval* nil)) \\*
~~~~(with-standard-io-syntax \\*
~~~~~~(setq data (read file)))))
\end{lisp}
Similarly, a user who dislikes the
  arbitrary choice of values for \cd{*print-\discretionary{}{}{}circle*} and \cdf{*print-pretty*}
  can bind these variables to other values inside the body.

The X3J13 vote left it unclear whether \cdf{with-standard-io-syntax}
permits declarations to appear before the body of the macro call.
I believe that was the intent, and this is reflected in the syntax shown above;
but this is only my interpretation.
\end{defmac}

\section{Input Functions}

The input functions are divided into two groups: those that
operate on streams of characters and those that operate on
streams of binary data.

\subsection{Input from Character Streams}
\label{CHARACTER-INPUT-SECTION}

Many character
input functions take optional arguments called \emph{input-stream},
\emph{eof-error-p}, and \emph{eof-value}.  The \emph{input-stream} argument is the
stream from
which to obtain input; if unsupplied or {\false} it defaults to the value of
the special variable \cdf{*standard-input*}.  One may also specify {\true}
as a stream, meaning the value of the special variable
\cdf{*terminal-io*}.

The \emph{eof-error-p} argument
controls what happens if input is from a file (or any other
input source that has a definite end) and the end of the file is reached.
If \emph{eof-error-p} is true (the default), an error will be signaled
at end of file.  If it is false, then no error is signaled, and instead
the function returns \emph{eof-value}.

An \emph{eof-value} argument
may be any Lisp datum whatsoever.

Functions such as \cdf{read} that read the representation
of an object rather than a single
character will always signal an error, regardless of \emph{eof-error-p}, if
the file ends in the middle of an object representation.
For example, if a file does
not contain enough right parentheses to balance the left parentheses in
it, \cdf{read} will complain.  If a file ends in a symbol or a number
immediately followed by end-of-file, \cdf{read} will read the symbol or
number successfully and when called again will see the end-of-file and
only then act according to \emph{eof-error-p}.
Similarly, the function \cdf{read-line}
will successfully read the last line of a file even if that line
is terminated by end-of-file rather than the newline character.
If a file contains ignorable text at the end, such
as blank lines and comments, \cdf{read} will not consider it to end in the
middle of an object.
Thus an \emph{eof-error-p} argument controls what happens
when the file ends \emph{between} objects.

Many input functions also take an argument called \emph{recursive-p}.
If specified and not {\nil}, this argument specifies that
this call is not a ``top-level'' call to \cdf{read} but an imbedded call,
typically from the function for a macro character.
It is important to distinguish such recursive calls for three reasons.

First, a top-level call establishes the context within which the
\cd{\#\emph{n}=} and \cd{\#\emph{n}\#} syntax is scoped.  Consider, for example,
the expression
\begin{lisp}
(cons '\#3=(p q r) '(x y . \#3\#))
\end{lisp}
If the single-quote macro character were defined in this way:
\begin{lisp}
(set-macro-character \#{\Xbackslash}' \\
~~~~~~~~~~~~~~~~~~~~~\#'(lambda (stream char) \\
~~~~~~~~~~~~~~~~~~~~~~~~~(declare (ignore char)) \\
~~~~~~~~~~~~~~~~~~~~~~~~~(list 'quote (read stream))))
\end{lisp}
then the expression could not be read properly, because there would be no way
to know when \cdf{read} is called recursively by the first
occurrence of \cd{'} that the label \cd{\#3=} would be referred to
later in the containing expression.
There would be no way to know because \cdf{read}
could not determine that it was called by a macro-character function
rather than from ``top level.''  The correct way to define the single quote
macro character uses the \emph{recursive-p} argument:
\begin{lisp}
(set-macro-character \#{\Xbackslash}' \\
~~~~~~~~~~~~~~~~~~~~~\#'(lambda (stream char) \\
~~~~~~~~~~~~~~~~~~~~~~~~~(declare (ignore char)) \\
~~~~~~~~~~~~~~~~~~~~~~~~~(list 'quote (read stream t nil t))))
\end{lisp}

Second, a recursive call does not alter whether the reading process
is to preserve whitespace or not (as determined by whether the
top-level call was to \cdf{read} or \cdf{read-preserving-whitespace}).
Suppose again that the single quote had the first, incorrect, macro-character
definition shown above.  Then a call to \cdf{read-preserving-whitespace}
that read the expression \cd{'foo } would fail to preserve the space
character following the symbol \cdf{foo} because the single-quote
macro-character function calls \cdf{read}, not \cdf{read-preserving-whitespace},
to read the following expression (in this case \cdf{foo}).
The correct definition, which passes the value {\true} for the \emph{recursive-p}
argument to \cdf{read}, allows the top-level call to determine
whether whitespace is preserved.

Third, when end-of-file is encountered and the \emph{eof-error-p} argument
is not {\nil}, the kind of error that is signaled may depend on the value
of \emph{recursive-p}.  If \emph{recursive-p} is not {\nil}, then the end-of-file
is deemed to have occurred within the middle of a printed representation;
if \emph{recursive-p} is {\nil}, then the end-of-file may be deemed to have
occurred between objects rather than within the middle of one.


\begin{defun}[Function]
read &optional input-stream eof-error-p eof-value recursive-p

\cdf{read} reads in the printed representation of a Lisp object
from \emph{input-stream}, builds a corresponding Lisp object, and returns
the object.

Note that when the variable \cd{*read-suppress*} is not {\nil},
then \cdf{read} reads in a printed representation as best it can,
but most of the work of interpreting the representation is avoided
(the intent being that the result is to be discarded anyway).
For example, all extended tokens produce the result {\nil} regardless
of their syntax.
\end{defun}

\begin{defun}[Variable]
*read-default-float-format*

The value of this variable must be a type specifier symbol for
a specific floating-point format; these include
\cdf{short-float}, \cdf{single-float},
\cdf{double-float}, and \cdf{long-float} and may include implementation-specific
types as well.  The default value is \cdf{single-float}.

\cdf{*read-default-float-format*}
indicates the floating-point format to be used
for reading floating-point numbers that have no exponent marker or have
\cdf{e} or \cdf{E} for an exponent marker.  (Other exponent markers
explicitly prescribe the floating-point format to be used.)
The printer also uses this variable to guide the choice of exponent
markers when printing floating-point numbers.
\end{defun}

\begin{defun}[Function]
read-preserving-whitespace &optional in-stream eof-error-p eof-value recursive-p

Certain printed representations given to \cdf{read}, notably those of symbols
and numbers, require a delimiting character after them.  (Lists do not, because
the close parenthesis marks the end of the list.)
Normally \cdf{read} will throw away the delimiting character if it is a
whitespace character;
but \cdf{read} will preserve the character (using \cdf{unread-char}) if it is
syntactically meaningful, because it may be the start of the next expression.

\begin{new}
X3J13 voted in January 1989
\issue{PEEK-CHAR-READ-CHAR-ECHO}
to clarify the interaction of \cdf{unread-char}
with echo streams.  These changes indirectly affect the echoing behavior
of \cdf{read-preserving-whitespace}.
\end{new}

The function \cdf{read-preserving-whitespace} is provided for some specialized
situations where it is desirable to determine precisely what character
terminated the extended token.

As an example, consider this macro-character definition:
\begin{lisp}
(defun slash-reader (stream char) \\
~~(declare (ignore char)) \\
~~(do ((path (list (read-preserving-whitespace stream)) \\
~~~~~~~~~~~~~(cons (progn (read-char stream nil nil t) \\
~~~~~~~~~~~~~~~~~~~~~~~~~~(read-preserving-whitespace \\
~~~~~~~~~~~~~~~~~~~~~~~~~~~~~stream)) \\
~~~~~~~~~~~~~~~~~~~path))) \\
~~~~~~((not (char= (peek-char nil stream nil nil t) \#{\Xbackslash}/)) \\
~~~~~~~(cons 'path (nreverse path))))) \\
(set-macro-character \#{\Xbackslash}/ \#'slash-reader)
\end{lisp}
(This is actually a rather dangerous definition to make because
expressions such as \cd{(/ x 3)} will no longer be read properly.
The ability to reprogram the reader syntax is very powerful and
must be used with caution.  This redefinition of \cdf{/} is shown
here purely for the sake of example.)

Consider now calling \cdf{read} on this expression:
\begin{lisp}
(zyedh /usr/games/zork /usr/games/boggle)
\end{lisp}
The \cdf{/} macro reads objects separated by more \cdf{/} characters;
thus \cd{/usr/games/zork} is intended to be read as \cd{(path usr games zork)}.
The entire example expression should therefore be read as
\begin{lisp}
(zyedh (path usr games zork) (path usr games boggle))
\end{lisp}
However, if \cdf{read} had been used instead of
\cdf{read-preserving-whitespace}, then after the reading of the symbol
\cdf{zork}, the following space would be discarded; the next call
to \cdf{peek-char} would see the following \cdf{/}, and the loop would
continue, producing this interpretation:
\begin{lisp}
(zyedh (path usr games zork usr games boggle))
\end{lisp}
On the other hand, there are times when whitespace \emph{should} be discarded.
If a command interpreter takes single-character commands,
but occasionally reads a Lisp object, then if the whitespace
after a symbol is not discarded it might be interpreted as a command
some time later after the symbol had been read.

Note that \cdf{read-preserving-whitespace} behaves \emph{exactly} like \cdf{read}
when the \emph{recursive-p} argument is not {\nil}.  The distinction is established
only by calls with \emph{recursive-p} equal to {\nil} or omitted.
\end{defun}

\begin{defun}[Function]
read-delimited-list char &optional input-stream recursive-p

This reads objects from \emph{stream} until the next character after an object's
representation (ignoring whitespace characters and comments) is \emph{char}.
(The \emph{char} should not have whitespace syntax in the current
readtable.)
A list of the objects read is returned.

To be more precise, \cdf{read-delimited-list} looks ahead at each step
for the next non-whitespace character and peeks at it
as if with \cdf{peek-char}.  If it is \emph{char}, then the
character is consumed and the list of objects is returned.
If it is a constituent or escape character, then \cdf{read} is used
to read an object, which is added to the end of the list.
If it is a macro character, the associated macro function
is called; if the function returns a value, that value is added
to the list.  The peek-ahead process is then repeated.

\begin{new}
X3J13 voted in January 1989
\issue{PEEK-CHAR-READ-CHAR-ECHO}
to clarify the interaction of \cdf{peek-char}
with echo streams.  These changes indirectly affect the echoing behavior
of the function \cdf{read-delimited-list}.
\end{new}

This function is particularly useful for defining new macro characters.
Usually it is desirable for the terminating character \emph{char} to be a
terminating macro character so that it may be used to delimit tokens;
however, \cdf{read-delimited-list} makes no attempt to alter the syntax
specified for \emph{char} by the current readtable.  The user must make any
necessary changes to the readtable syntax explicitly.  The following
example illustrates this.

Suppose you wanted \cd{\#{\Xlbrace}\emph{a} \emph{b} \emph{c} ... \emph{z}{\Xrbrace}}
to be read as a list of all pairs of the elements \emph{a}, \emph{b}, \emph{c}, \cd{...},
\emph{z}; for example:
\begin{lisp}
\#{\Xlbrace}p q z a{\Xrbrace}~~\textrm{reads as}~~((p q) (p z) (p a) (q z) (q a) (z a))
\end{lisp}
This can be done by specifying a macro-character definition for \cd{\#{\Xlbrace}}
that does two things: read in all the items up to the \cd{{\Xrbrace}},
and construct the pairs.  \cdf{read-delimited-list} performs
the first task.

\begin{new}
Note that \cdf{mapcon} allows the mapped function to examine
the items of the list after the current one, and that
\cdf{mapcon} uses \cdf{nconc}, which is all right because \cdf{mapcar}
will produce fresh lists.
\end{new}

\penalty-10000 %required

\begin{lisp}
(defun |\#{\Xlbrace}-reader| (stream char arg) \\
~~(declare (ignore char arg)) \\
~~(mapcon \#'(lambda (x) \\
~~~~~~~~~~~~~~(mapcar \#'(lambda (y) (list (car x) y)) (cdr x))) \\
~~~~~~~~~~(read-delimited-list \#{\Xbackslash}{\Xrbrace} stream t))) \\
 \\
(set-dispatch-macro-character \#{\Xbackslash}\# \#{\Xbackslash}{\Xlbrace} \#'|\#{\Xlbrace}-reader|) \\
 \\
(set-macro-character \#{\Xbackslash}{\Xrbrace} (get-macro-character \#{\Xbackslash}) {\nil}))
\end{lisp}
(Note that {\true} is specified for the \emph{recursive-p} argument.)

It is necessary here to give a definition to the character \cd{{\Xrbrace}} as
well to prevent it from being a constituent.
If the line
\begin{lisp}
(set-macro-character \#{\Xbackslash}{\Xrbrace} (get-macro-character \#{\Xbackslash}) {\nil}))
\end{lisp}
shown above were not included, then the \cd{{\Xrbrace}} in
\begin{lisp}
\#{\Xlbrace}p q z a{\Xrbrace}
\end{lisp}
would be considered a constituent character, part of the symbol named
\cd{a{\Xrbrace}}.  One could correct for this by putting a space before
the \cd{{\Xrbrace}}, but it is better simply to use the call to
\cdf{set-macro-character}.

Giving \cd{{\Xrbrace}} the same
definition as the standard definition of the character \cd{)} has the
twin benefit of making it terminate tokens for use with \cdf{read-delimited-list}
and also making it illegal for use in any other context (that is, attempting to
read a stray \cd{{\Xrbrace}} will signal an error).

Note that \cdf{read-delimited-list} does not take an \emph{eof-error-p}
(or \emph{eof-value})
argument.  The reason is that it is always an error
to hit end-of-file during the operation of \cdf{read-delimited-list}.
\end{defun}

\begin{defun}[Function]
read-line &optional input-stream eof-error-p eof-value recursive-p

\cdf{read-line} reads in a line of text terminated by a newline.
It returns the line as a character string (\emph{without} the newline character).
This function is usually used to get a line of input from the user.
A second returned value is a flag that is false if the line was
terminated normally, or true if end-of-file terminated the (non-empty) line.
If end-of-file is encountered immediately (that is, appears to terminate
an empty line), then end-of-file processing is controlled in the
usual way by the \emph{eof-error-p}, \emph{eof-value}, and \emph{recursive-p} arguments.

The corresponding output function is \cdf{write-line}.
\end{defun}

\begin{defun}[Function]
read-char &optional input-stream eof-error-p eof-value recursive-p

\cdf{read-char} inputs one character from \emph{input-stream} and returns it
as a character object.

The corresponding output function is \cdf{write-char}.

\begin{new}
X3J13 voted in January 1989
\issue{PEEK-CHAR-READ-CHAR-ECHO}
to clarify the interaction of \cdf{read-char} with echo streams
(as created by \cdf{make-echo-stream}).  A character is echoed from the input stream
to the associated output stream the first time it is seen.
If a character is read again because of an intervening \cdf{unread-char} operation,
the character is not echoed again when read for the second time or any subsequent time.
\end{new}
\end{defun}

\begin{defun}[Function]
unread-char character &optional input-stream

\cdf{unread-char} puts the \emph{character} onto the front of \emph{input-stream}.
The \emph{character} must be the same character that was most recently read
from the \emph{input-stream}.  The \emph{input-stream} ``backs up'' over this
character; when a character is next read from \emph{input-stream}, it will
be the specified character followed by the previous contents of
\emph{input-stream}.  \cdf{unread-char} returns {\false}.

One may apply \cdf{unread-char} only to the character most recently
read from \emph{input-stream}.  Moreover, one may not invoke \cdf{unread-char}
twice consecutively without an intervening \cdf{read-char}
operation.  The result is that one may back up only by one character,
and one may not insert any characters into the input stream that were not already there.

\begin{new}
X3J13 voted in January 1989
\issue{UNREAD-CHAR-AFTER-PEEK-CHAR}
to clarify that one also may not invoke
\cdf{unread-char} after invoking \cdf{peek-char} without an intervening
\cdf{read-char} operation.  This is consistent with the notion that
\cdf{peek-char} behaves much like \cdf{read-char} followed by \cdf{unread-char}.
\end{new}

\beforenoterule
\begin{rationale}
This is not intended to be a general mechanism, but rather
an efficient mechanism for allowing the Lisp reader and other parsers
to perform one-character lookahead in the input stream.
This protocol admits a wide variety of efficient implementations,
such as simply decrementing a buffer pointer.
To have to specify the character in the call to \cdf{unread-char} is
admittedly redundant, since at any given time there is only one character
that may be legally specified.  The redundancy is intentional,
again to give the implementation latitude.
\end{rationale}
\afternoterule

\begin{new}
X3J13 voted in January 1989
\issue{PEEK-CHAR-READ-CHAR-ECHO}
to clarify the interaction of \cdf{unread-char} with echo streams
(as created by \cdf{make-echo-stream}).  When a character is ``unread'' from an echo
stream, no attempt is made to ``unecho'' the character.  However, a character
placed back into an echo stream by \cdf{unread-char} will not be re-echoed
when it is subsequently re-read by \cdf{read-char}.
\end{new}
\end{defun}

\newpage%required

\begin{defun}[Function]
peek-char &optional peek-type input-stream eof-error-p eof-value recursive-p

What \cdf{peek-char} does depends on the \emph{peek-type}, which
defaults to {\false}.  With a \emph{peek-type} of {\false},
\cdf{peek-char} returns the next character to be read from 
\emph{input-stream}, without actually removing it from the input stream.
The next time input is done from \emph{input-stream}, the character will still
be there.  It is as if one had called \cdf{read-char} and then \cdf{unread-char}
in succession.

If \emph{peek-type} is {\true}, then \cdf{peek-char} skips over whitespace
characters (but not comments)
and then performs the peeking operation on the next
character.
This is useful for finding the (possible) beginning
of the next printed representation of a Lisp object.
The last character examined (the one that starts an object)
is not removed from the input stream.

If \emph{peek-type} is a character object, then \cdf{peek-char} skips
over input characters until a character that
is \cdf{char=} to that object is found;
that character is left in the input stream.

\begin{new}
X3J13 voted in January 1989
\issue{PEEK-CHAR-READ-CHAR-ECHO}
to clarify the interaction of \cdf{peek-char} with echo streams
(as created by \cdf{make-echo-stream}).  When a character from an echo
stream is only peeked at, it is not echoed at that time.  The character remains in
the input stream and may be echoed when read by \cdf{read-char} at a later time.
Note, however, that if the \emph{peek-type} is not \cdf{nil}, then characters
skipped over (and therefore consumed) by \cdf{peek-char} are treated as if they had been read
by \cdf{read-char}, and will be echoed if \cdf{read-char} would have echoed them.
\end{new}
\end{defun}

\begin{defun}[Function]
listen &optional input-stream

The predicate \cdf{listen} is true if there is a character
immediately available from \emph{input-stream}, and is false if not.
This is particularly useful when the stream obtains characters
from an interactive device such as a keyboard.  A call to \cdf{read-char}
would simply wait until a character was available, but \cdf{listen} can
sense whether or not input is available and allow the program to
decide whether or not to attempt input.  On a non-interactive stream,
the general rule is that \cdf{listen} is true except when at
end-of-file.
\end{defun}

\begin{defun}[Function]
read-char-no-hang &optional input-stream eof-error-p eof-value recursive-p

This function is exactly like \cdf{read-char}, except
that if it would be necessary to wait in order to get a character (as
from a keyboard), {\false} is immediately returned without waiting.  This
allows one to efficiently check for input availability and get the
input if it is available.
This is different from the \cdf{listen} operation in
two~ways.  First, \cdf{read-char-no-hang} potentially reads a character,
whereas \cdf{listen} never inputs a character.  Second,
\cdf{listen} does not distinguish between end-of-file and no input being
available, whereas \cdf{read-char-no-hang} does make that distinction, returning
\emph{eof-value} at end-of-file (or signaling an error if no \emph{eof-error-p}
is true) but always returning {\false} if no input
is available.
\end{defun}

\begin{defun}[Function]
clear-input &optional input-stream

This clears any buffered input associated with \emph{input-stream}.
It is primarily useful for clearing type-ahead from keyboards when
some kind of asynchronous error has occurred.  If this operation
doesn't make sense for the stream involved, then \cdf{clear-input}
does nothing.  \cdf{clear-input} returns {\false}.
\end{defun}

\begin{defun}[Function]
read-from-string string &optional eof-error-p eof-value &key :start :end :preserve-whitespace

The characters of \emph{string} are given successively to the Lisp reader,
and the Lisp object built by the reader is returned.  Macro characters
and so on will all take effect.

The arguments \cd{:start} and \cd{:end} delimit a substring of \emph{string}
beginning at the character indexed by \cd{:start} and up to but not
including the character indexed by \cd{:end}.  By default \cd{:start} is \cd{0}
(the beginning of the string) and \cd{:end} is \cd{(length \emph{string})}.
This is the same as for other string functions.

The flag \cd{:preserve-whitespace}, if provided and not {\nil}, indicates
that the operation should preserve whitespace as
for \cdf{read-preserving-whitespace}.  It defaults to {\nil}.

As with other reading functions,
the arguments \emph{eof-error-p} and \emph{eof-value} control the action
if the end of the (sub)string is reached
before the operation is completed;
reaching the end of the string is treated as any other end-of-file event.

\cdf{read-from-string} returns two values: the first is the object read,
and the second is the index of the first character in the string not
read.  If the entire string was read, the second result
will be either the length of
the string or one greater than the length of the string.  The parameter
\cd{:preserve-whitespace} may affect this second value.

\begin{lisp}
(read-from-string "(a b c)") \EV\ (a b c) \textrm{and} 7
\end{lisp}
\end{defun}

\begin{defun}[Function]
parse-integer string &key :start :end :radix :junk-allowed

This function examines the substring of \emph{string} delimited by \cd{:start}
and \cd{:end} (which default to the beginning and end of the string).
It skips over whitespace characters and then attempts to
parse an integer.  The \cd{:radix} parameter defaults to \cd{10}
and must be an integer between 2 and 36.

If \cd{:junk-allowed} is not {\false}, then the first value
returned is the value of the number parsed
as an integer or {\false} if no syntactically correct integer
was seen.

If \cd{:junk-allowed} is {\false} (the default), then the entire substring is scanned.
The returned value is the value of the number parsed as an integer.
An error is signaled if the substring does not consist entirely of
the representation of an integer, possibly surrounded on either side by
whitespace characters.

In either case, the second value is the index into the string of the delimiter
that terminated the parse, or it is the index beyond the substring if the
parse terminated at the end of the substring (as will always be the case if
\cd{:junk-allowed} is false).

Note that \cdf{parse-integer} does not recognize the syntactic radix-specifier
prefixes \cd{\#O}, \cd{\#B}, \cd{\#X}, and \cd{\#\emph{n}R}, nor does it recognize
a trailing decimal point.  It permits only an optional sign
(\cdf{+} or \cdf{-}) followed
by a non-empty sequence of digits in the specified radix.
\end{defun}

\begin{defun}[Function]
read-sequence sequence input-stream &key :start :end

This function reads elements from input-stream into sequence. The position of
the first unchanged element of sequence is returned.
\end{defun}

\subsection {Input from Binary Streams}

Common Lisp currently specifies only a very simple facility for binary input:
the reading of a single byte as an integer.

\begin{defun}[Function]
read-byte binary-input-stream &optional eof-error-p eof-value

\cdf{read-byte} reads one byte from the \emph{binary-input-stream} and returns
it in the form of an integer.
\end{defun}

\section {Output Functions}

The output functions are divided into two groups: those that
operate on streams of characters and those that operate on
streams of binary data.  The function \cdf{format} operates
on streams of characters but is described in a section
separate from the other character-output functions
because of its great complexity.

\subsection {Output to Character Streams}

These functions all take an optional argument called \emph{output-stream},
which is where to send the output.  If unsupplied or {\false}, \emph{output-stream}
defaults to the value of the variable
\cdf{*standard-output*}.  If it is {\true}, the value of the variable
\cdf{*terminal-io*} is used.

X3J13 voted in June 1989 \issue{DATA-IO} to add the keyword argument
\cd{:readably} to the function \cdf{write}, and voted in June 1989 \issue{PRETTY-PRINT-INTERFACE}
to add the keyword arguments \cd{:right-margin}, \cd{:miser-width}, \cd{:lines},
and \cd{:pprint-dispatch}.
The revised description
is as follows.

\begin{defun}[Function]
write object &key :stream :escape :radix :base :circle
   :pretty :level :length :case :gensym :array :readably
   :right-margin :miser-width :lines :pprint-dispatch

The printed representation of \emph{object} is written to the output stream
specified by \cd{:stream}, which defaults to the value of \cdf{*standard-output*}.

The other keyword arguments specify values used to control the
generation of the printed representation.  Each defaults to the
value of the corresponding global variable: see \cdf{*print-escape*},
\cd{*print-radix*}, \cd{*print-base*}, \cdf{*print-circle*}, \cdf{*print-pretty*},
\cd{*print-level*}, \cd{*print-length*}, and \cdf{*print-case*}, in addition to
\cd{*print-array*},
\cd{*print-gensym*},
\cd{*print-readably*},
\cd{*print-right-margin*},
\cd{*print-miser-width*},
\cd{*print-lines*},
and \cd{*print-pprint-dispatch*}.
(This is the means by which these variables affect printing operations:
supplying default values for the \cdf{write} function.)
Note that the printing of symbols is also affected by the value
of the variable \cdf{*package*}.
\cdf{write} returns \emph{object}.
\end{defun}

\begin{defun}[Function]
prin1 object &optional output-stream \\
print object &optional output-stream \\
pprint object &optional output-stream \\
princ object &optional output-stream

\cd{prin1} outputs the printed representation of \emph{object} to
\emph{output-stream}.  Escape characters are used as appropriate.
Roughly speaking, the output from \cd{prin1} is suitable for input to
the function \cdf{read}.  \cd{prin1} returns the \emph{object} as its value.
\begin{lisp}
(prin1 \emph{object} \emph{output-stream}) \\
~~~\EQ\ (write \emph{object} :stream \emph{output-stream} :escape t)
\end{lisp}

\cdf{print} is just like \cd{prin1} except that the printed representation
of \emph{object} is preceded by a newline (see \cdf{terpri})
and followed by a space.
\cdf{print} returns \emph{object}.

\cdf{pprint} is just like \cdf{print} except that the trailing
space is omitted and the
\emph{object} is printed with the \cdf{*print-pretty*} flag non-{\nil}
to produce ``pretty'' output.
\cdf{pprint} returns no values (that is, what the expression
\cd{(values)} returns: zero values).

\begin{new}
X3J13 voted in January 1989
\issue{PRETTY-PRINT-INTERFACE}
to adopt a facility for user-controlled pretty printing
(see chapter~\ref{PPRINT}).
\end{new}

\cdf{princ} is just like \cd{prin1} except that the
output has no escape characters.  A symbol is printed as simply the characters
of its print name; a string is printed without surrounding double quotes;
and there may be differences for other data types as well.
The general rule is that output from \cdf{princ} is intended to look
good to people, while output from \cd{prin1} is intended to
be acceptable to the function \cdf{read}.
\begin{newer}
X3J13 voted in June 1987 \issue{PRINC-CHARACTER}
to clarify that \cdf{princ} prints a character in exactly
the same manner as \cdf{write-char}: the character is simply sent to the output \emph{stream}.
This was implied by the specification in section~\ref{PRINTER} in the first edition,
but is worth pointing out explicitly here.
\end{newer}
\cdf{princ} returns the \emph{object} as its value.
\begin{lisp}
(princ \emph{object} \emph{output-stream}) \\
~~~\EQ\ (write \emph{object} :stream \emph{output-stream} :escape {\false})
\end{lisp}
\end{defun}


\begin{defun}[Function]
write-to-string object &key :escape :radix :base :circle :pretty
   :level :length :case :gensym :array :readably
   :right-margin :miser-width :lines :pprint-dispatch \\
prin1-to-string object \\
princ-to-string object

The object is effectively printed as if by \cdf{write},
\cd{prin1}, or \cdf{princ}, respectively,
and the characters that would be output are made into a string,
which is returned.
\end{defun}


\begin{defun}[Function]
write-char character &optional output-stream

\cdf{write-char} outputs the \emph{character} to \emph{output-stream},
and returns \emph{character}.
\end{defun}


\begin{defun}[Function]
write-string string &optional output-stream &key :start :end{\negthinspace\negthinspace} \\
write-line string &optional output-stream &key :start :end

\cdf{write-string} writes the characters of the specified
substring of \emph{string} to
the \emph{output-stream}.  The \cd{:start} and \cd{:end} parameters
delimit a substring of \emph{string} in the usual manner
(see chapter~\ref{KSEQUE}).
\cdf{write-line} does the same thing but then
outputs a newline afterwards.  (See \cdf{read-line}.)
In either case, the \emph{string} is returned (\emph{not} the substring
delimited by \cd{:start} and \cd{:end}).
In some implementations these may be much
more efficient than an explicit loop using \cdf{write-char}.
\end{defun}

\begin{defun}[Function]
write-sequence sequence output-stream &key :start :end


write-sequence writes the elements of the subsequence of \emph{sequence} bounded by \emph{start} and \emph{end} to \emph{output-stream}.
\end{defun}

\begin{defun}[Function]
terpri &optional output-stream \\
fresh-line &optional output-stream

The function \cdf{terpri} outputs a newline to \emph{output-stream}.
It is identical in effect to
\cd{(write-char \#{\Xbackslash}Newline \emph{output-stream})}; however,
\cdf{terpri} always returns {\false}.

\cdf{fresh-line} is similar to \cdf{terpri} but outputs a newline
only if the stream is not already at the start of a line.
(If for some reason this cannot be determined, then a newline
is output anyway.)
This guarantees that the stream will be on a ``fresh line'' while
consuming as little vertical distance as possible.
\cdf{fresh-line} is a predicate that is true if it output a
newline, and otherwise false.
\end{defun}

\begin{defun}[Function]
finish-output &optional output-stream \\
force-output &optional output-stream \\
clear-output &optional output-stream

Some streams may be implemented in an asynchronous or buffered manner.
The function \cdf{finish-output} attempts to ensure that all output
sent to \emph{output-stream} has reached its destination, and only then
returns {\false}.  \cdf{force-output} initiates the emptying of any
internal buffers but returns {\nil} without waiting for completion
or acknowledgment.

The function \cdf{clear-output}, on the other hand, attempts to abort any
outstanding output operation in progress in order
to allow as little output as possible
to continue to the destination.  This is useful, for example, to abort
a lengthy output to the terminal when an asynchronous error occurs.
\cdf{clear-output} returns {\false}.

The precise actions of all three of these operations are
implementation-dependent.
\end{defun}

\begin{defmac}
print-unreadable-object (object stream &key type identity) 
  {declaration}* {form}*

    Function will output a printed representation of \emph{object} on \emph{stream},
    beginning with
    \cd{\#<} and ending with \cdf{>}.  Everything output to the \emph{stream} during
    execution of the body
    forms is enclosed in the angle brackets.  If \emph{type} is true, the body
    output is preceded by a brief description of the object's type and a
    space character.  If \emph{id} is true, the body output is followed by
    a space character and a representation of the object's identity,
    typically a storage address.

    If \cd{*print-readably*} is true, \cdf{print-unreadable-object} signals an error
    of type \cdf{print-not-readable} without printing anything.

    The \emph{object}, \emph{stream}, \emph{type}, and \emph{id} arguments are all evaluated
    normally.  The \emph{type} and \emph{id} default to false.  It is valid to provide
    no body forms.  If \emph{type} and \emph{id} are both true and there are no
    body forms, only one space character separates the printed type and the printed identity.

    The value returned by \cdf{print-unreadable-object} is \cdf{nil}.
\begin{lisp}
(defmethod print-object ((obj airplane) stream) \\*
~~(print-unreadable-object (obj stream :type t :identity t) \\*
~~~~(princ (tail-number obj) stream))) \\
(print my-airplane)~~\textrm{prints} \\*
\#<Airplane NW0773 777500123135>~~~~~;\textrm{In implementation A} \\*
~~~~~~~~~~~~~~~~~~~~~\textrm{or perhaps} \\*
\#<FAA:AIRPLANE NW0773 17>~~~~~~~~~~~;\textrm{In implementation B}
\end{lisp}
The big advantage of
\cdf{print-unreadable-object} is that it allows a user to write \cdf{print-object} methods that
  adhere to implementation-specific style without requiring the user to write
  implementation-dependent code.

The X3J13 vote left it unclear whether \cdf{print-unreadable-object}
permits declarations to appear before the body of the macro call.
I believe that was the intent, and this is reflected in the syntax shown above;
but this is only my interpretation.
\end{defmac}

\subsection {Output to Binary Streams}

Common Lisp currently specifies only a very simple facility for binary output:
the writing of a single byte as an integer.

\begin{defun}[Function]
write-byte integer binary-output-stream

\cdf{write-byte} writes one byte, the value of \emph{integer}.
It is an error if \emph{integer} is not of the type
specified as the \cd{:element-type} argument to \cdf{open} when the stream
was created.
The value \emph{integer} is returned.
\end{defun}


\subsection{Formatted Output to Character Streams}
\label{FORMAT-SECTION}
\indexterm{formatted output}

The function \cdf{format} is very useful for producing
nicely formatted text, producing good-looking messages, and so on.
\cdf{format} can generate a string or output to a stream.

Formatted output is performed not only by the \cdf{format} function
itself but by certain other functions that accept a control string
``the way \cdf{format} does.''  For example, error-signaling functions
such as \cdf{cerror} accept \cdf{format} control strings.

\begin{defun}[Function]
format destination control-string &rest arguments

\cdf{format} is used to produce formatted output.
\cdf{format} outputs the characters of \emph{control-string},
except that a tilde (\cd{{\Xtilde}}) introduces a directive.
The character after
the tilde, possibly preceded by prefix parameters and modifiers, specifies
what kind of formatting is desired.  Most directives use one or more
elements of \emph{arguments} to create their output; the typical directive
puts the next element of \emph{arguments} into the output, formatted in
some special way.  It is an error if no argument remains for a directive
requiring an argument, but it is not an error if one or more arguments
remain unprocessed by a directive.

The output is sent to \emph{destination}.  If \emph{destination} is
{\false}, a string is created that contains the output; this string is
returned as the value of the call to \cdf{format}.

When the first argument
to \cdf{format} is \cdf{nil}, \cdf{format} creates a stream
of type \cdf{string-stream} in much the same manner as \cdf{with-output-to-string}.
(This stream may be visible to the user if, for example, the \cd{{\Xtilde}S}
directive is used to print a \cdf{defstruct} structure that has a user-supplied
print function.)

In all other cases
\cdf{format} returns {\false}, performing output to \emph{destination}
as a side effect.
If \emph{destination} is a stream, the output is sent to it.  If
\emph{destination} is {\true}, the output is sent to the stream
that is the value of the variable \cdf{*standard-output*}.
If \emph{destination} is a string with a fill pointer, then
in effect the output characters are added to the end of the string
(as if by use of \cdf{vector-push-extend}).

The \cdf{format} function includes some extremely complicated and specialized
features.  It is not necessary to understand all or even most of its
features to use \cdf{format} effectively.  The beginner should
skip over anything in the following documentation that is not
immediately useful or clear.  The more sophisticated features
(such as conditionals and iteration) are
there for the convenience of programs with especially complicated formatting
requirements.

A \cdf{format} directive consists of a tilde (\cd{{\Xtilde}}),
optional prefix parameters
separated by commas, optional colon (\cd{:}) and at-sign (\cd{{\Xatsign}}) modifiers,
and a single character indicating what kind of directive this is.
The alphabetic case of the directive character is ignored.
The prefix parameters are generally integers,
notated as optionally signed decimal numbers.

If both colon and at-sign modifiers are present, they may appear
in either order; thus \cd{{\Xtilde}:{\Xatsign}R} and \cd{{\Xtilde}{\Xatsign}:R}
mean the same thing.  However, it is traditional to put the colon first, and all the
examples in this book put colons before at-signs.

Examples of control strings:
\begin{lisp}
"{\Xtilde}S"~~~~~~~~~~~;\textrm{An \cd{{\Xtilde}S} directive with no parameters or modifiers} \\
"{\Xtilde}3,-4:{\Xatsign}s"~~~~~;\textrm{An \cd{{\Xtilde}S} directive with two parameters, 3 and $-4$,} \\
~~~~~~~~~~~~~~~; \textrm{and both the colon and at-sign flags} \\
"{\Xtilde},+4S"~~~~~~~~;\textrm{First prefix parameter is omitted and takes} \\
~~~~~~~~~~~~~~~; \textrm{on its default value; the second parameter is 4}
\end{lisp}
Sometimes a prefix parameter is used to specify a character, for
instance the padding character in a right- or left-justifying operation.
In this case a single quote (\cd{'}) followed by the desired
character may be used as a prefix parameter, to mean the character
object that is the character following the single quote.  For
example, you can use \cd{{\Xtilde}5,'0d}
to print an integer in decimal radix in five columns with leading zeros,
or \cd{{\Xtilde}5,'*d} to get leading asterisks.

In place of a prefix parameter to a directive, you can put the letter
\cdf{V} (or \cdf{v}), which takes an argument from \emph{arguments} for use as a parameter to
the directive.  Normally this should be an integer or character object,
as appropriate.  This feature allows variable-width fields and the like.
If the argument used by a \cdf{V} parameter is {\nil},
the effect is as if the parameter had been omitted.
You may also use the character \cd{\#} in place of a parameter; it
represents the number of arguments remaining to be processed.

It is an error to give a format directive more parameters than
it is described here as accepting.  It is also an error to give
colon or at-sign modifiers to a directive in a combination not
specifically described here as being meaningful.
\end{defun}

\penalty-10000 %manual

\begin{new}
X3J13 voted in January 1989
\issue{FORMAT-PRETTY-PRINT}
to clarify the interaction between \cdf{format}
and the various printer control variables (those named \cd{*print-\emph{xxx}*}).
This is important because many \cdf{format} operations are defined, directly
or indirectly, in terms of \cd{prin1} or \cdf{princ}, which are affected
by the printer control variables.  The general rule is that \cdf{format}
does not bind any of the standard printer control variables except as
specified in the individual descriptions of directives.  An implementation
may not bind any standard printer control variable not specified in the
description of a \cdf{format} directive, nor may an implementation fail
to bind any standard printer control variables that is specified to be bound
by such a description.  (See these
descriptions for specific changes voted by X3J13.)

One consequence of this change is that the user is guaranteed to be able
to use the \cdf{format} \cd{{\Xtilde}A} and \cd{{\Xtilde}S} directives
to do pretty printing, under control of the \cdf{*print-pretty*} variable.
Implementations have differed on this point in their interpretations of
the first edition.  The new \cd{{\Xtilde}W} directive may be more appropriate
than either \cd{{\Xtilde}A} and \cd{{\Xtilde}S} for some purposes,
whether for pretty printing or ordinary printing.
See section~\ref{PPRINT-FORMAT-DIRECTIVES-SECTION} for a discussion of
\cd{{\Xtilde}W} and other new \cdf{format} directives related to pretty printing.
\end{new}


Here are some relatively simple examples to give you the general
flavor of how \cdf{format} is used.
\begin{lisp}
(format {\false} "foo") \EV\ "foo" \\
 \\
(setq x 5) \\
 \\
(format {\false} "The answer is {\Xtilde}D." x) \EV\ "The answer is 5." \\
 \\
(format {\false} "The answer is {\Xtilde}3D." x) \EV\ "The answer is~~~5." \\
 \\
(format {\false} "The answer is {\Xtilde}3,'0D." x) \EV\ "The answer is 005." \\
 \\
(format {\false} "The answer is {\Xtilde}:D." (expt 47 x)) \\
~~~~~~~~~~~~~~~~~~~~~~~~~~~~~~~~\EV\ "The answer is 229,345,007."
\end{lisp}

\begin{lisp}
(setq y "elephant") \\
 \\
(format {\false} "Look at the {\Xtilde}A!" y) \EV\ "Look at the elephant!" \\
 \\
(format {\false} "Type {\Xtilde}:C to {\Xtilde}A." \\
~~~~~~~~(set-char-bit \#{\Xbackslash}D :control t) \\
~~~~~~~~"delete all your files") \\
~~~\EV\ "Type Control-D to delete all your files."
\end{lisp}

\newpage%manual

\begin{lisp}
(setq n 3)
\end{lisp}
\begin{lisp}
(format {\false} "{\Xtilde}D item{\Xtilde}:P found." n) \EV\ "3 items found."
\end{lisp}
\begin{lisp}
(format {\false} "{\Xtilde}R dog{\Xtilde}:{\Xlbracket}s are{\Xtilde}; is{\Xtilde}{\Xrbracket} here." n (= n 1)) \\
~~~~~~\EV\ "three dogs are here."
\end{lisp}
\begin{lisp}
(format {\false} "{\Xtilde}R dog{\Xtilde}:*{\Xtilde}{\Xlbracket}s are{\Xtilde}; is{\Xtilde}:;s are{\Xtilde}{\Xrbracket} here." n) \\
~~~~~~\EV\ "three dogs are here."
\end{lisp}
\begin{lisp}
(format {\false} "Here {\Xtilde}{\Xlbracket}are{\Xtilde};is{\Xtilde}:;are{\Xtilde}{\Xrbracket} {\Xtilde}:*{\Xtilde}R pupp{\Xtilde}:{\Xatsign}P." n) \\
~~~~~~\EV\ "Here are three puppies."
\end{lisp}

In the descriptions of the directives that follow,
the term \emph{arg} in general
refers to the next item of the set of \emph{arguments} to be processed.
The word or phrase at the beginning of each description is a mnemonic
(not necessarily an accurate one) for the directive.
\begin{flushdesc}
\item[\cd{{\Xtilde}A}]
\emph{Ascii}.  An \emph{arg}, any Lisp object, is printed without escape characters
(as by \cdf{princ}).  In particular, if \emph{arg} is a string, its characters
will be output verbatim.
If \emph{arg} is {\nil}, it will
be printed as {\false}; the colon modifier
(\cd{{\Xtilde}:A}) will cause an \emph{arg} of {\nil} to be printed as {\emptylist},
but if \emph{arg} is a composite structure, such as a list or vector,
any contained occurrences of {\nil} will still be printed as {\false}.

\cd{{\Xtilde}\emph{mincol}A} inserts spaces on the right, if necessary, to make the
width at least \emph{mincol} columns.  The \cd{{\Xatsign}} modifier causes the spaces
to be inserted on the left rather than the right.

\cd{{\Xtilde}\emph{mincol},\emph{colinc},\emph{minpad},\emph{padchar}A} is the full form of \cd{{\Xtilde}A},
which allows elaborate control of the padding.
The string is padded on the right (or on the left if the
\cd{{\Xatsign}} modifier is used) with at least \emph{minpad} copies
of \emph{padchar}; padding characters are then inserted \emph{colinc} characters
at a time until the total width is at least \emph{mincol}.
The defaults are \cd{0} for \emph{mincol} and \emph{minpad}, \cd{1} for \emph{colinc},
and the space character for \emph{padchar}.

\cdf{format} binds \cdf{*print-escape*}
to \cdf{nil} during the processing of the \cd{{\Xtilde}A} directive.

\item[\cd{{\Xtilde}S}]
\emph{S-expression}.
This is just like \cd{{\Xtilde}A}, but \emph{arg} is printed \emph{with} escape
characters (as by \cd{prin1} rather than \cdf{princ}).  The output is
therefore suitable for input to \cdf{read}.  \cd{{\Xtilde}S} accepts
all the arguments and modifiers that \cd{{\Xtilde}A} does.

\cdf{format} binds \cdf{*print-escape*}
to \cdf{t} during the processing of the \cd{{\Xtilde}S} directive.

\newpage%required

\item[\cd{{\Xtilde}D}]
\emph{Decimal}.
An \emph{arg}, which should be an integer, is printed in decimal radix.
\cd{{\Xtilde}D} will never put a decimal point after the number.

\cd{{\Xtilde}\emph{mincol}D} uses a column width of \emph{mincol}; spaces are inserted on
the left if the number requires fewer than \emph{mincol} columns for its digits
and sign.  If the number doesn't fit in \emph{mincol} columns, additional columns
are used as needed.

\cd{{\Xtilde}\emph{mincol},\emph{padchar}D} uses \emph{padchar} as the pad character
instead of space.

If \emph{arg} is not an integer, it is printed
in \cd{{\Xtilde}A} format and decimal base.

\cdf{format} binds \cdf{*print-escape*}
to \cdf{nil}, \cd{*print-radix*} to \cdf{nil}, and \cd{*print-base*} to \cd{10}
during processing of \cd{{\Xtilde}D}.

The \cd{{\Xatsign}} modifier causes the number's sign to be printed always; the default
is to print it only if the number is negative.
The \cd{:} modifier causes commas to be printed between groups of three digits;
the third prefix parameter may be used to change the character used as the comma.
Thus the most general form of \cd{{\Xtilde}D} is
\cd{{\Xtilde}\emph{mincol},\emph{padchar},\emph{commachar}D}.

\begin{new}
X3J13 voted in March 1988
\issue{FORMAT-COMMA-INTERVAL}
to add a fourth parameter, the \emph{commainterval}.
This must be an integer; if it is not provided,
it defaults to 3.  This parameter controls the number of digits in each
group separated by the \emph{commachar}.

By extension, each of the \cd{{\Xtilde}B}, \cd{{\Xtilde}O}, and \cd{{\Xtilde}X} directives
accepts a \emph{commainterval} as a fourth parameter,
and the \cd{{\Xtilde}R} directive accepts a \emph{commainterval} as its fifth parameter.
Examples:
\begin{lisp}
(format nil "{\Xtilde},,' ,4B" \#xFACE) \EV\ "1111 1010 1100 1110" \\
(format nil "{\Xtilde},,' ,4B" \#x1CE) \EV\ "1 1100 1110" \\
(format nil "{\Xtilde}19,,' ,4B" \#xFACE) \EV\ "1111 1010 1100 1110" \\
(format nil "{\Xtilde}19,,' ,4B" \#x1CE) \EV\ "0000 0001 1100 1110"
\end{lisp}
This is one of those little improvements that probably don't matter much
but aren't hard to implement either.  It was pretty silly having the number 3 wired
into the definition of comma separation when it is just as easy to make it
a parameter.
\end{new}

\item[\cd{{\Xtilde}B}]
\emph{Binary}.
This is just like \cd{{\Xtilde}D} but prints in binary radix (radix 2)
instead of decimal.  The full form is therefore
\cd{{\Xtilde}\emph{mincol},\emph{padchar},\emph{commachar}B}.

\cdf{format} binds \cdf{*print-escape*}
to \cdf{nil}, \cd{*print-radix*} to \cdf{nil}, and \cd{*print-base*} to \cd{2}
during processing of \cd{{\Xtilde}B}.

\item[\cd{{\Xtilde}O}]
\emph{Octal}.
This is just like \cd{{\Xtilde}D} but prints in octal radix (radix 8)
instead of decimal.  The full form is therefore
\cd{{\Xtilde}\emph{mincol},\emph{padchar},\emph{commachar}O}.

\cdf{format} binds \cdf{*print-escape*}
to \cdf{nil}, \cd{*print-radix*} to \cdf{nil}, and \cd{*print-base*} to \cd{8}
during processing of \cd{{\Xtilde}O}.

\item[\cd{{\Xtilde}X}]
\emph{Hexadecimal}.
This is just like \cd{{\Xtilde}D} but prints in hexadecimal radix
(radix 16) instead of decimal.  The full form is therefore
\cd{{\Xtilde}\emph{mincol},\emph{padchar},\emph{commachar}X}.

\cdf{format} binds \cdf{*print-escape*}
to \cdf{nil}, \cd{*print-radix*} to \cdf{nil}, and \cd{*print-base*} to \cd{16}
during processing of \cd{{\Xtilde}X}.

\item[\cd{{\Xtilde}R}]
\emph{Radix}.
\cd{{\Xtilde}\emph{n}R} prints \emph{arg} in radix \emph{n}.
The modifier flags and any remaining parameters are used as for
the \cd{{\Xtilde}D} directive.
Indeed, \cd{{\Xtilde}D} is the same as \cd{{\Xtilde}10R}.  The full form here is therefore
\cd{{\Xtilde}\emph{radix},\emph{mincol},\emph{padchar},\emph{commachar}R}.

\cdf{format} binds \cdf{*print-escape*}
to \cdf{nil}, \cd{*print-radix*} to \cdf{nil}, and \cd{*print-base*} to the value
of the first parameter
during the processing of the \cd{{\Xtilde}R} directive with a parameter.

If no parameters are given to \cd{{\Xtilde}R}, then an entirely different
interpretation is given.
\begin{new}%CORR
\emph{Notice of correction.}
In the first edition, this sentence referred to ``arguments'' given to \cd{{\Xtilde}R}.
The correct term is ``parameters.''
\end{new}
The argument should be an integer;
suppose it is \cd{4}. Then
\cd{{\Xtilde}R} prints \emph{arg} as a cardinal English number: \cdf{four};
\cd{{\Xtilde}:R} prints \emph{arg} as an ordinal English number: \cdf{fourth};
\cd{{\Xtilde}{\Xatsign}R} prints \emph{arg} as a Roman numeral: \cdf{IV}; and
\cd{{\Xtilde}:{\Xatsign}R} prints \emph{arg} as an old Roman numeral: \cdf{IIII}.

\cdf{format} binds \cd{*print-base*} to \cd{10}
during the processing of the \cd{{\Xtilde}R} directive with no parameter.

\begin{new}
The first edition did not specify how \cd{{\Xtilde}R} and its variants should
handle arguments that are very large or not positive.  Actual practice varies,
and X3J13 has not yet addressed the topic.
Here is a sampling of current practice.

For \cd{{\Xtilde}{\Xatsign}R} and \cd{{\Xtilde}:{\Xatsign}R}, nearly all implementations
produce Roman numerals only for integers in the range 1 to 3999, inclusive.
Some implementations will produce old-style Roman numerals for integers in
the range 1 to 4999, inclusive.  All other integers are printed in decimal
notation, as if \cd{{\Xtilde}D} had been used.

For zero, most implementations print \cdf{zero} for \cd{{\Xtilde}R}
and \cdf{zeroth} for \cd{{\Xtilde}:R}.

For \cd{{\Xtilde}R} with a negative argument, most implementations simply print
the word \cdf{minus} followed by its absolute value as a cardinal in English.

For \cd{{\Xtilde}:R} with a negative argument, some implementations also print
the word \cdf{minus} followed by its absolute value as an ordinal in English;
other implementations print the absolute value followed by the word \cdf{previous}.
Thus the argument \cd{-4} might produce \cd{minus~fourth} or \cd{fourth~previous}.
Each has its charm, but one is not always a suitable substitute for the other;
users should be careful.

There is standard English nomenclature for fairly large integers (up to $10^60$,
at least), based on appending the suffix -illion to Latin names of integers.
Thus we have the names \emph{trillion}, \emph{quadrillion}, \emph{sextillion},
\emph{septillion}, and so on.  For extremely large integers, one may express powers
of ten in English.
One implementation gives
\cd{1606938044258990275541962092341162602522202993782792835301376}
(which is $2^{200}$, the result of \cd{(ash 1 200)})
in this manner:
\begin{flushleft}
\small \cf
one times ten to the sixtieth power six hundred six times ten to the
fifty-seventh power nine hundred thirty-eight septdecillion forty-four
sexdecillion two hundred fifty-eight quindecillion nine hundred ninety
quattuordecillion two hundred seventy-five tredecillion five hundred forty-one
duodecillion nine hundred sixty-two undecillion ninety-two decillion three
hundred forty-one nonillion one hundred sixty-two octillion six hundred two
septillion five hundred twenty-two sextillion two hundred two quintillion nine
hundred ninety-three quadrillion seven hundred eighty-two trillion seven hundred
ninety-two billion eight hundred thirty-five million three hundred one thousand
three hundred seventy-six
\end{flushleft}
Another implementation prints it this way (note the use of \cdf{plus}):
\begin{flushleft}
\small \cf
one times ten to the sixtieth power plus six hundred six times ten to the fifty-seventh power plus
... plus two hundred seventy-five times ten to the
forty-second power plus five hundred forty-one duodecillion nine hundred sixty-two undecillion
...  three hundred seventy-six
\end{flushleft}
(I have elided some of the text here to save space.)

Unfortunately, the meaning of this nomenclature differs between American English (in which {\it
k}-illion means $10^{3(\hbox{\scriptsize\it k}+1)}$, so one trillion is $10^{12}$) and British English (in which {\it
k}-illion means $10^{6\hbox{\scriptsize\it k}}$, so one trillion is $10^{18}$).
To avoid both confusion and prolixity, 
I recommend using decimal notation for all numbers above 999,999,999;
this is similar to the escape hatch used for Roman numerals.
\end{new}

\item[\cd{{\Xtilde}P}]
\emph{Plural}.
If \emph{arg} is not \cdf{eql} to the integer \cd{1}, a lowercase \cdf{s} is
printed; if \emph{arg} is \cdf{eql} to \cd{1}, nothing is printed.  (Notice
that if \emph{arg} is a floating-point \cd{1.0}, the \cdf{s} \emph{is}
printed.)
\cd{{\Xtilde}:P} does the same thing, after doing a \cd{{\Xtilde}:*} to back up one argument;
that is, it prints a lowercase \cdf{s} if the \emph{last} argument was not
\cd{1}.  This is useful after printing a number using \cd{{\Xtilde}D}.
\cd{{\Xtilde}{\Xatsign}P} prints \cdf{y} if the argument is \cd{1}, or \cdf{ies} if it is
not.  \cd{{\Xtilde}:{\Xatsign}P} does the same thing, but backs up first.
\begin{lisp}
(format {\false} "{\Xtilde}D tr{\Xtilde}:{\Xatsign}P/{\Xtilde}D win{\Xtilde}:P" 7 1) \EV\ "7 tries/1 win" \\
(format {\false} "{\Xtilde}D tr{\Xtilde}:{\Xatsign}P/{\Xtilde}D win{\Xtilde}:P" 1 0) \EV\ "1 try/0 wins" \\
(format {\false} "{\Xtilde}D tr{\Xtilde}:{\Xatsign}P/{\Xtilde}D win{\Xtilde}:P" 1 3) \EV\ "1 try/3 wins"
\end{lisp}

\item[\cd{{\Xtilde}C}]
\emph{Character}.  The next \emph{arg} should be a character; it is printed
according to the modifier flags.

\begin{obsolete}
\cd{{\Xtilde}C} prints the character in an implementation-dependent
abbreviated format.  This format should be culturally compatible with the
host environment.
\end{obsolete}

\begin{newer}
X3J13 voted in June 1987 \issue{FORMAT-OP-C} to specify that
\cd{{\Xtilde}C} performs exactly the same action as \cdf{write-char}
if the character to be printed has zero for its bits attributes.
X3J13 voted in March 1989 \issue{CHARACTER-PROPOSAL} to eliminate
the bits and font attributes, replacing them with the notion of
implementation-defined attributes.  The net effect is that characters
whose implementation-defined attributes all have the ``standard''
values should be printed by \cd{{\Xtilde}C} in the same way
that \cdf{write-char} would print them.
\end{newer}

\cd{{\Xtilde}:C} spells out the names of the control bits
and represents non-printing characters
by their names: \cdf{Control-Meta-F}, \cdf{Control-Return}, \cdf{Space}.
This is a ``pretty'' format for printing characters.

\cd{{\Xtilde}:{\Xatsign}C} prints what \cd{{\Xtilde}:C} would, and then
if the character requires unusual shift keys on the keyboard to type it,
this fact is mentioned: \cd{Control-$\partial$ (Top-F)}.  This is the
format for telling the user about a key he or she is expected to type,
in prompts, for instance.  The precise output may depend not only
on the implementation but on the particular I/O devices in use.

\cd{{\Xtilde}{\Xatsign}C} prints the character so that the Lisp reader can read it,
using \cd{\#{\Xbackslash}} syntax.

\begin{new}
X3J13 voted in January 1989
\issue{FORMAT-PRETTY-PRINT}
to specify that \cdf{format} binds \cdf{*print-escape*} to \cdf{t}
during the processing of the \cd{{\Xtilde}{\Xatsign}C} directive.
Other variants of the \cd{{\Xtilde}C} directive do not bind any printer control variables.
\end{new}

\beforenoterule
\begin{rationale}
In some implementations the \cd{{\Xtilde}S} directive would
do what \cd{{\Xtilde}C} does,
but \cd{{\Xtilde}C} is compatible
with Lisp dialects such as MacLisp that do not have a character data type.
\end{rationale}
\afternoterule

\item[\cd{{\Xtilde}F}]
\emph{Fixed-format floating-point}.
The next \emph{arg} is printed as a floating-point
number.

The full form is \cd{{\Xtilde}\emph{w},\emph{d},\emph{k},\emph{overflowchar},\emph{padchar\/}F}.
The parameter \emph{w}
is the width of the field to be printed; \emph{d} is the number
of digits to print after the decimal point; \emph{k} is a scale factor
that defaults to zero.

Exactly \emph{w} characters will
be output.  First, leading copies of the character \emph{padchar}
(which defaults to a space) are printed, if necessary, to pad the
field on the left.
If the \emph{arg} is negative, then a minus sign is printed;
if the \emph{arg} is not negative, then a plus sign is printed
if and only if the \cd{{\Xatsign}} modifier was specified.  Then a sequence
of digits, containing a single embedded decimal point, is printed;
this represents the magnitude of the value of \emph{arg} times $10^{\hbox{\scriptsize\it k}}$,
rounded to \emph{d} fractional digits.
(When rounding up and rounding down would produce printed values
equidistant from the scaled value of \emph{arg}, then the implementation
is free to use either one.  For example, printing the argument
\cd{6.375} using the format \cd{{\Xtilde}4,2F} may correctly produce
either \cd{6.37} or \cd{6.38}.)
Leading zeros are not permitted, except that a single
zero digit is output before the decimal point if the printed value
is less than 1, and this single zero digit is not output
after all if $\emph{w}=\emph{d}+1$.

If it is impossible to print the value in the required format in a field
of width \emph{w}, then one of two actions is taken.  If the
parameter \emph{overflowchar} is specified, then \emph{w} copies of that
parameter are printed instead of the scaled value of \emph{arg}.
If the \emph{overflowchar} parameter is omitted, then the scaled value
is printed using more than \emph{w} characters, as many more as may be
needed.

If the \emph{w} parameter is omitted, then the field is of variable width.
In effect, a value is chosen
for \emph{w} in such a way that no leading pad characters need to be printed
and exactly \emph{d} characters will follow the decimal point.
For example, the directive \cd{{\Xtilde},2F} will print exactly
two digits after the decimal point and as many as necessary before the
decimal point.

If the parameter \emph{d} is omitted, then there is no constraint
on the number of digits to appear after the decimal point.
A value is chosen for \emph{d} in such a way that as many digits
as possible may be printed subject to the width constraint
imposed by the parameter \emph{w} and the constraint that no trailing
zero digits may appear in the fraction, except that if the
fraction to be printed is zero, then a single zero digit should
appear after the decimal point if permitted by the width constraint.

If both \emph{w} and \emph{d} are omitted, then the effect is to print
the value using ordinary free-format output; \cd{prin1} uses this format
for any number whose magnitude is either zero or between
$10^{-3}$ (inclusive) and $10^7$ (exclusive).

If \emph{w} is omitted, then if the magnitude of \emph{arg} is so large (or, if
\emph{d} is also omitted, so small) that more than 100 digits would have to
be printed, then an implementation is free, at its discretion, to print
the number using exponential notation instead, as if by the directive
\cd{{\Xtilde}E} (with all parameters to \cd{{\Xtilde}E} defaulted, not
taking their values from the \cd{{\Xtilde}F} directive).

If \emph{arg} is a rational number, then it is coerced to be a \cdf{single-float}
and then printed.  (Alternatively, an implementation is permitted to
process a rational number by any other method that has essentially the
same behavior but avoids such hazards as loss of precision or overflow
because of the coercion.  However, note that if \emph{w} and \emph{d} are
unspecified and the number has no exact decimal representation,
for example \cd{1/3}, some precision cutoff must be chosen
by the implementation: only a finite number of digits may be printed.)

If \emph{arg} is a complex number or some non-numeric
object, then it is printed using the format directive \cd{{\Xtilde}\emph{w\/}D},
thereby printing it in decimal radix and a minimum field width of \emph{w}.
(If it is desired to print each of the real part and imaginary part
of a complex number using a \cd{{\Xtilde}F} directive, then this must
be done explicitly with two \cd{{\Xtilde}F} directives and code to
extract the two parts of the complex number.)

\begin{new}
X3J13 voted in January 1989
\issue{FORMAT-PRETTY-PRINT}
to specify that \cdf{format} binds \cdf{*print-escape*} to \cdf{nil}
during the processing of the \cd{{\Xtilde}F} directive.
\end{new}

\begin{lisp}
(defun foo (x) \\
~~(format nil "{\Xtilde}6,2F|{\Xtilde}6,2,1,'*F|{\Xtilde}6,2,,'?F|{\Xtilde}6F|{\Xtilde},2F|{\Xtilde}F" \\
~~~~~~~~~~x x x x x x)) \\
(foo 3.14159)~~\EV\ "~~3.14|~31.42|~~3.14|3.1416|3.14|3.14159" \\
(foo -3.14159)~\EV\ "~-3.14|-31.42|~-3.14|-3.142|-3.14|-3.14159" \\
(foo 100.0)~~~~\EV\ "100.00|******|100.00|~100.0|100.00|100.0" \\
(foo 1234.0)~~~\EV\ "1234.00|******|??????|1234.0|1234.00|1234.0" \\
(foo 0.006)~~~~\EV\ "~~0.01|~~0.06|~~0.01|~0.006|0.01|0.006"
\end{lisp}

\item[\cd{{\Xtilde}E}]
\emph{Exponential floating-point}.
The next \emph{arg} is printed in exponential notation.

The full form is \cd{{\Xtilde}\emph{w},\emph{d},\emph{e},\emph{k},\emph{overflowchar},\emph{padchar},\emph{exponentchar}E}.
The parameter \emph{w}
is the width of the field to be printed; \emph{d} is the number
of digits to print after the decimal point; \emph{e} is the number
of digits to use when printing the exponent;
\emph{k} is a scale factor that defaults to 1 (not zero).

Exactly \emph{w} characters will
be output.  First, leading copies of the character \emph{padchar}
(which defaults to a space) are printed, if necessary, to pad the
field on the left.
If the \emph{arg} is negative, then a minus sign is printed;
if the \emph{arg} is not negative, then a plus sign is printed
if and only if the \cd{{\Xatsign}} modifier was specified.  Then a sequence
of digits, containing a single embedded decimal point, is printed.
The form of this sequence of digits depends on the scale factor \emph{k}.
If \emph{k} is zero, then \emph{d} digits are printed after the decimal
point, and a single zero digit appears before the decimal point if
the total field width will permit it.  If \emph{k} is positive,
then it must be strictly less than $\emph{d}+2$;  \emph{k} significant digits
are printed before the decimal point, and $\emph{d}-\emph{k}+1$
digits are printed after the decimal point.  If \emph{k} is negative,
then it must be strictly greater than $-\emph{d}$;
a single zero digit appears before the decimal point if
the total field width will permit it, and after the decimal point
are printed first
$-\emph{k}$ zeros and then $\emph{d}+\emph{k}$ significant digits.
The printed fraction must be properly rounded.
(When rounding up and rounding down would produce printed values
equidistant from the scaled value of \emph{arg}, then the implementation
is free to use either one.  For example, printing
\cd{637.5} using the format \cd{{\Xtilde}8,2E} may correctly produce
either \cd{6.37E+02} or \cd{6.38E+02}.)

Following the digit sequence, the exponent is printed.
First the character parameter \emph{exponentchar} is printed; if this
parameter is omitted, then the exponent marker that
\cd{prin1} would use is printed, as determined from the
type of the floating-point number and the current value of
\cdf{*read-default-float-format*}.
Next, either a plus sign or a minus sign
is printed, followed by \emph{e} digits representing the power of
10 by which the printed fraction must be multiplied
to properly represent the rounded value of \emph{arg}.

If it is impossible to print the value in the required format in a field
of width \emph{w}, possibly because \emph{k} is too large or too small
or because the exponent cannot be printed in \emph{e} character positions,
then one of two actions is taken.  If the
parameter \emph{overflowchar} is specified, then \emph{w} copies of that
parameter are printed instead of the scaled value of \emph{arg}.
If the \emph{overflowchar} parameter is omitted, then the scaled value
is printed using more than \emph{w} characters, as many more as may be
needed; if the problem is that \emph{d} is too small for the specified \emph{k}
or that \emph{e} is too small, then a larger value is used for \emph{d} or \emph{e}
as may be needed.

If the \emph{w} parameter is omitted, then the field is of variable width.
In effect a value is chosen
for \emph{w} in such a way that no leading pad characters need to be printed.

If the parameter \emph{d} is omitted, then there is no constraint
on the number of digits to appear.
A value is chosen for \emph{d} in such a way that as many digits
as possible may be printed subject to the width constraint
imposed by the parameter \emph{w}, the constraint of the scale factor \emph{k},
and the constraint that no trailing
zero digits may appear in the fraction, except that if the
fraction to be printed is zero, then a single zero digit should
appear after the decimal point if the width constraint allows it.

If the parameter \emph{e} is omitted, then the exponent is printed
using the smallest number of digits necessary to represent its value.

If all of \emph{w}, \emph{d}, and \emph{e} are omitted, then the effect is to print
the value using ordinary free-format exponential-notation output;
\cd{prin1} uses this format for any non-zero number whose magnitude
is less than $10^{-3}$ or greater than or equal to $10^7$.

\begin{new}
X3J13 voted in January 1989
\issue{FORMAT-E-EXPONENT-SIGN}
to amend the previous paragraph as follows:

If all of \emph{w}, \emph{d}, and \emph{e} are omitted, then the effect is to print
the value using ordinary free-format exponential-notation output;
\cd{prin1} uses a similar format for any non-zero number whose magnitude
is less than $10^{-3}$ or greater than or equal to $10^7$.
The only difference is that the \cd{{\Xtilde}E} directive always prints
a plus or minus sign before the exponent, while \cd{prin1} omits the plus sign
if the exponent is non-negative.

(The amendment reconciles this paragraph with the specification several
paragraphs above that \cd{{\Xtilde}E} always prints
a plus or minus sign before the exponent.)
\end{new}

If \emph{arg} is a rational number, then it is coerced to be a \cdf{single-float}
and then printed.  (Alternatively, an implementation is permitted to
process a rational number by any other method that has essentially the
same behavior but avoids such hazards as loss of precision or overflow
because of the coercion.  However, note that if \emph{w} and \emph{d} are
unspecified and the number has no exact decimal representation,
for example \cd{1/3}, some precision cutoff must be chosen
by the implementation: only a finite number of digits may be printed.)

If \emph{arg} is a complex number or some non-numeric
object, then it is printed using the format directive \cd{{\Xtilde}\emph{w}D},
thereby printing it in decimal radix and a minimum field width of \emph{w}.
(If it is desired to print each of the real part and imaginary part
of a complex number using a \cd{{\Xtilde}E} directive, then this must
be done explicitly with two \cd{{\Xtilde}E} directives and code to
extract the two parts of the complex number.)

\begin{new}
X3J13 voted in January 1989
\issue{FORMAT-PRETTY-PRINT}
to specify that \cdf{format} binds \cdf{*print-escape*} to \cdf{nil}
during the processing of the \cd{{\Xtilde}E} directive.
\end{new}

\begin{lisp}
(defun foo (x) \\
~~(format {\nil} \\
~~~~~~~~~~"{\Xtilde}9,2,1,,'*E|{\Xtilde}10,3,2,2,'?,,'\$E|{\Xtilde}9,3,2,-2,'\%{\Xatsign}E|{\Xtilde}9,2E" \\
~~~~~~~~~~x x x x)) \\
(foo 3.14159)~~\EV\ "~~3.14E+0|~31.42\$-01|+.003E+03|~~3.14E+0" \\
(foo -3.14159)~\EV\ "~-3.14E+0|-31.42\$-01|-.003E+03|~-3.14E+0" \\
(foo 1100.0) ~~\EV\ "~~1.10E+3|~11.00\$+02|+.001E+06|~~1.10E+3" \\
(foo 1100.0L0)~\EV\ "~~1.10L+3|~11.00\$+02|+.001L+06|~~1.10L+3" \\
(foo 1.1E13)~~~\EV\ "*********|~11.00\$+12|+.001E+16|~1.10E+13" \\
(foo 1.1L120)~~\EV\ "*********|??????????|\%\%\%\%\%\%\%\%\%|1.10L+120" \\
(foo 1.1L1200)~\EV\ "*********|??????????|\%\%\%\%\%\%\%\%\%|1.10L+1200"
\end{lisp}
Here is an example of the effects of varying the scale factor:
\begin{lisp}
(dotimes (k 13) \\
~~(format t "~\%Scale factor ~2D: |~13,6,2,VE|" \\
~~~~~~~~~~(- k 5) 3.14159))\`;\textrm{Prints 13 lines} \\
Scale factor -5: | 0.000003E+06| \\
Scale factor -4: | 0.000031E+05| \\
Scale factor -3: | 0.000314E+04| \\
Scale factor -2: | 0.003142E+03| \\
Scale factor -1: | 0.031416E+02| \\
Scale factor~~0: | 0.314159E+01| \\
Scale factor~~1: | 3.141590E+00| \\
Scale factor~~2: | 31.41590E-01| \\
Scale factor~~3: | 314.1590E-02| \\
Scale factor~~4: | 3141.590E-03| \\
Scale factor~~5: | 31415.90E-04| \\
Scale factor~~6: | 314159.0E-05| \\
Scale factor~~7: | 3141590.E-06|
\end{lisp}

\newpage%required

\item[\cd{{\Xtilde}G}]
\emph{General floating-point}.  The next \emph{arg} is printed as a floating-point
number in either fixed-format or exponential notation as appropriate.

The full form is \cd{{\Xtilde}\emph{w},\emph{d},\emph{e},\emph{k},\emph{overflowchar},\emph{padchar},\emph{exponentchar}G}.
The format in which to print \emph{arg} depends on the magnitude (absolute
value) of the \emph{arg}.  Let \emph{n} be an integer such that
$10^{\hbox{\scriptsize\it n}-1}\leq\hbox\emph{arg}<10^{\hbox{\scriptsize\it n}}$.
(If \emph{arg} is zero, let \emph{n} be 0.)
Let \emph{ee} equal $\emph{e}+2$, or 4 if \emph{e} is omitted.
Let \emph{ww} equal $\emph{w}-\hbox\emph{ee}$,
or {\nil} if \emph{w} is omitted.  If \emph{d} is omitted, first let \emph{q}
be the number of digits needed to print \emph{arg} with no loss
of information and without leading or trailing zeros;
then let \emph{d} equal \cd{(max \emph{q} (min \emph{n} 7))}.
Let \emph{dd} equal $\emph{d}-\emph{n}$.

If $0\leq\hbox\emph{dd}\leq \emph{d}$, then \emph{arg} is printed
as if by the format directives
\begin{lisp}
{\Xtilde}\emph{ww},\emph{dd},,\emph{overflowchar},\emph{padchar}F{\Xtilde}\emph{ee}{\Xatsign}T
\end{lisp}
Note that the scale factor \emph{k} is not passed to the \cd{{\Xtilde}F}
directive.  For all other values of \emph{dd}, \emph{arg} is printed as if
by the format directive
\begin{lisp}
{\Xtilde}\emph{w},\emph{d},\emph{e},\emph{k},\emph{overflowchar},\emph{padchar},\emph{exponentchar}E
\end{lisp}

In either case, an \cd{{\Xatsign}} modifier is specified to the \cd{{\Xtilde}F}
or \cd{{\Xtilde}E} directive if and only if one was specified to the
\cd{{\Xtilde}G} directive.

\cdf{format} binds \cdf{*print-escape*} to \cdf{nil}
during the processing of the \cd{{\Xtilde}G} directive.

Examples:
\begin{lisp}
(defun foo (x) \\
~~(format {\nil} \\
~~~~~~~~~~"{\Xtilde}9,2,1,,'*G|{\Xtilde}9,3,2,3,'?,,'\$G|{\Xtilde}9,3,2,0,'\%G|{\Xtilde}9,2G" \\
~~~~~~~~~~x x x)) \\[10pt]
(foo 0.0314159) \EV\ "~~3.14E-2|314.2\$-04|0.314E-01|~~3.14E-2" \\
(foo 0.314159)~~\EV\ "~~0.31~~~|0.314~~~~|0.314~~~~| 0.31~~~~" \\
(foo 3.14159)~~~\EV\ "~~~3.1~~~|~3.14~~~~|~3.14~~~~|~~3.1~~~~" \\
(foo 31.4159)~~~\EV\ "~~~31.~~~|~31.4~~~~|~31.4~~~~|~~31.~~~~" \\
(foo 314.159)~~~\EV\ "~~3.14E+2|~314.~~~~|~314.~~~~|~~3.14E+2" \\
(foo 3141.59)~~~\EV\ "~~3.14E+3|314.2\$+01|0.314E+04|~~3.14E+3" \\
(foo 3141.59L0)~\EV\ "~~3.14L+3|314.2\$+01|0.314L+04|~~3.14L+3" \\
(foo 3.14E12)~~~\EV\ "*********|314.0\$+10|0.314E+13|~3.14E+12" \\
(foo 3.14L120)~~\EV\ "*********|?????????|\%\%\%\%\%\%\%\%\%|3.14L+120" \\
(foo 3.14L1200)~\EV\ "*********|?????????|\%\%\%\%\%\%\%\%\%|3.14L+1200"
\end{lisp}
\newpage%manual

\item[\cd{{\Xtilde}\$}]
\emph{Dollars floating-point}.  The next \emph{arg} is printed as a floating-point
number in fixed-format notation.  This format is particularly
convenient for printing a value as dollars and cents.

The full form is \cd{{\Xtilde}\emph{d},\emph{n},\emph{w},\emph{padchar}\$}.
The parameter \emph{d} is the number
of digits to print after the decimal point (default value 2);
\emph{n} is the minimum number of digits to print before the decimal
point (default value 1);
\emph{w} is the minimum total width of the field to be printed (default
value 0).

First padding and the sign are output.
If the \emph{arg} is negative, then a minus sign is printed;
if the \emph{arg} is not negative, then a plus sign is printed
if and only if the \cd{{\Xatsign}} modifier was specified.  
If the \cd{:} modifier is used, the sign appears before any padding,
and otherwise after the padding.
If \emph{w} is specified and the number of other characters to be output
is less than \emph{w}, then copies of \emph{padchar} (which defaults
to a space) are output to
make the total field width equal \emph{w}.
Then \emph{n} digits are printed for the integer part of \emph{arg},
with leading zeros if necessary; then a decimal point;
then \emph{d} digits of fraction, properly rounded.

If the magnitude of \emph{arg} is so large that more than \emph{m} digits would
have to be printed, where \emph{m} is the larger of \emph{w} and 100, then an
implementation is free, at its discretion, to print the number using
exponential notation instead, as if by the directive
\cd{{\Xtilde}\emph{w},\emph{q},,,,\emph{padchar}E}, where \emph{w} and \emph{padchar} are
present or omitted according to whether they were present or omitted in
the \cd{{\Xtilde}\$} directive, and where $\emph{q}=\emph{d}+\emph{n}-1$,
where \emph{d} and \emph{n} are the (possibly default) values given to the
\cd{{\Xtilde}\$} directive.

If \emph{arg} is a rational number, then it is coerced to be a \cdf{single-float}
and then printed.  (Alternatively, an implementation is permitted to
process a rational number by any other method that has essentially the
same behavior but avoids such hazards as loss of precision or overflow
because of the coercion.)

If \emph{arg} is a complex number or some non-numeric
object, then it is printed using the format directive \cd{{\Xtilde}\emph{w}D},
thereby printing it in decimal radix and a minimum field width of \emph{w}.
(If it is desired to print each of the real part and imaginary part
of a complex number using a \cd{{\Xtilde}\$} directive, then this must
be done explicitly with two \cd{{\Xtilde}\$} directives and code to
extract the two parts of the complex number.)

\cdf{format} binds \cdf{*print-escape*} to \cdf{nil}
during the processing of the \cd{{\Xtilde}\$} directive.

\item[\cd{{\Xtilde}\%}]
This outputs a \cd{\#{\Xbackslash}Newline} character, thereby terminating the current
output line and beginning a new one
(see \cdf{terpri}).

\cd{{\Xtilde}\emph{n}\%} outputs \emph{n} newlines.

No \emph{arg} is used.  Simply putting a newline in the control string
would work, but \cd{{\Xtilde}\%} is often used because it makes the control string
look nicer in the middle of a Lisp program.

\item[\cd{{\Xtilde}\&}]
Unless it can be determined that the output stream
is already at the beginning of a line,
this outputs a newline (see \cdf{fresh-line}).

\cd{{\Xtilde}\emph{n}\&} calls \cdf{fresh-line}
and then outputs $\emph{n}-1$ newlines.
\cd{{\Xtilde}0\&} does nothing.

\item[\cd{{\Xtilde}|}]
This outputs a page separator character, if possible.
\cd{{\Xtilde}\emph{n}|} does this
\emph{n} times.  \cd{|} is vertical bar, not capital I.

\item[\cd{{\Xtilde}{\Xtilde}}]
\emph{Tilde}.
This outputs a tilde.  \cd{{\Xtilde}\emph{n}{\Xtilde}} outputs \emph{n} tildes.

\item[\cd{{\Xtilde}}$\langle$newline$\rangle$]
Tilde immediately followed by a newline ignores the newline
and any following non-newline whitespace characters.
With a \cd{:}, the newline
is ignored, but any following whitespace is left in place.
With an \cd{{\Xatsign}}, the newline
is left in place, but any following whitespace is ignored.
This directive is typically used when a format control string is too long
to fit nicely into one line of the program:
\begin{lisp}
(defun type-clash-error (fn nargs argnum right-type wrong-type) \\*
~~(format *error-output* \\*
~~~~~~~~~~"{\Xtilde}\&Function {\Xtilde}S requires its {\Xtilde}:{\Xlbracket}{\Xtilde}:R{\Xtilde};{\Xtilde}*{\Xtilde}{\Xrbracket} {\Xtilde} \\*
~~~~~~~~~~~argument to be of type {\Xtilde}S,{\Xtilde}\%but it was called {\Xtilde} \\*
~~~~~~~~~~~with an argument of type {\Xtilde}S.{\Xtilde}\%" \\*
~~~~~~~~~~fn (eql nargs 1) argnum right-type wrong-type)) \\
 \\
(type-clash-error 'aref nil 2 'integer 'vector)  \textrm{prints}: \\*
Function AREF requires its second argument to be of type INTEGER, \\*
but it was called with an argument of type VECTOR. \\
 \\
(type-clash-error 'car 1 1 'list 'short-float)  \textrm{prints}: \\
Function CAR requires its argument to be of type LIST, \\
but it was called with an argument of type SHORT-FLOAT.
\end{lisp}
Note that in this example newlines appear in the output only as specified
by the \cd{{\Xtilde}\&} and \cd{{\Xtilde}\%} directives; the actual newline characters
in the control string are suppressed because each is preceded by a tilde.

\item[\cd{{\Xtilde}T}]
\emph{Tabulate}.
This spaces over to a given column.
\cd{{\Xtilde}\emph{colnum},\emph{colinc}T} will output
sufficient spaces to move the cursor to column \emph{colnum}.  If the cursor
is already at or beyond column \emph{colnum}, it will output spaces to move it to
column \emph{colnum}+\emph{k}*\emph{colinc} for the smallest positive integer
\emph{k} possible, unless \emph{colinc} is zero, in which case no spaces
are output if the cursor is already at or beyond column \emph{colnum}.
\emph{colnum} and \emph{colinc} default to \cd{1}.

Ideally, the current column position is determined by examination of the
destination, whether a stream or string. (Although no user-level
operation for determining the column position of a stream is defined
by Common Lisp, such a facility may exist at the implementation level.)
If for some reason the current absolute column position cannot be determined
by direct inquiry,
\cdf{format} may be able to deduce the current column position by noting
that certain directives (such as \cd{{\Xtilde}\%}, or \cd{{\Xtilde}\&},
or \cd{{\Xtilde}A} with the argument being a string containing a newline) cause
the column position to be reset to zero, and counting the number of characters
emitted since that point.  If that fails, \cdf{format} may attempt a
similar deduction on the riskier assumption that the destination was
at column zero when \cdf{format} was invoked.  If even this heuristic fails
or is implementationally inconvenient, at worst
the \cd{{\Xtilde}T} operation will simply output two spaces.
(All this implies that code that uses \cdf{format} is
more likely to be portable if all format control strings that use 
the \cd{{\Xtilde}T} directive either begin with \cd{{\Xtilde}\%} or \cd{{\Xtilde}\&}
to force a newline
or are designed to be used only when the destination is known from other
considerations to be at column zero.)

\cd{{\Xtilde}{\Xatsign}T} performs \emph{relative} tabulation.
\cd{{\Xtilde}\emph{colrel},\emph{colinc}{\Xatsign}T} outputs \emph{colrel} spaces
and then outputs the smallest non-negative
number of additional spaces necessary to move the cursor
to a column that is a multiple
of \emph{colinc}.  For example, the directive \cd{{\Xtilde}3,8{\Xatsign}T} outputs
three spaces and then moves the cursor to a ``standard multiple-of-eight
tab stop'' if not at one already.
If the current output column cannot be determined, however,
then \emph{colinc} is ignored, and exactly \emph{colrel} spaces are output.

\begin{new}
X3J13 voted in June 1989 \issue{PRETTY-PRINT-INTERFACE} to define \cd{{\Xtilde}:T}
and \cd{{\Xtilde}:{\Xatsign}T} to perform tabulation relative to a point defined
by the pretty printing process (see section~\ref{PPRINT-FORMAT-DIRECTIVES-SECTION}).
\end{new}

\item[\cd{{\Xtilde}*}]
The next \emph{arg} is ignored.  \cd{{\Xtilde}\emph{n}*} ignores the next \emph{n} arguments.

\cd{{\Xtilde}:*} ``ignores backwards''; that is, it backs up in the list of
arguments so that the argument last processed will be processed again.
\cd{{\Xtilde}\emph{n}:*} backs up \emph{n} arguments.

When within a \cd{{\Xtilde}{\Xlbrace}} construct
(see below), the ignoring (in either direction) is relative to the list
of arguments being processed by the iteration.

\cd{{\Xtilde}\emph{n}{\Xatsign}*} is an ``absolute goto'' rather than a ``relative goto'':
it goes to the \emph{n}th \emph{arg}, where 0 means the first one;
\emph{n} defaults to 0, so \cd{{\Xtilde}{\Xatsign}*} goes back to the first \emph{arg}.
Directives after a \cd{{\Xtilde}\emph{n}{\Xatsign}*}
will take arguments in sequence beginning with the one gone to.
When within a \cd{{\Xtilde}{\Xlbrace}} construct, the ``goto''
is relative to the list of arguments being processed by the iteration.

\item[\cd{{\Xtilde}?}]
\emph{Indirection}.
The next \emph{arg} must be a string, and the one after it a list;
both are consumed by the \cd{{\Xtilde}?} directive.
The string is processed as a \cdf{format} control string, with the
elements of the list as the arguments.  Once the recursive processing
of the control string has been finished, then processing of the control
string containing the \cd{{\Xtilde}?} directive is resumed.
Example:
\begin{lisp}
(format nil "{\Xtilde}? {\Xtilde}D" "<{\Xtilde}A {\Xtilde}D>" '("Foo" 5) 7) \EV\ "<Foo 5> 7" \\
(format nil "{\Xtilde}? {\Xtilde}D" "<{\Xtilde}A {\Xtilde}D>" '("Foo" 5 14) 7) \EV\ "<Foo 5> 7"
\end{lisp}
Note that in the second example three arguments are supplied
to the control string \cd{"<{\Xtilde}A {\Xtilde}D>"}, but only two are processed
and the third is therefore ignored.

With the \cd{{\Xatsign}} modifier, only one \emph{arg} is directly consumed.
The \emph{arg} must be a string; it is processed as part of the control
string as if it had appeared in place of the \cd{{\Xtilde}{\Xatsign}?} construct,
and any directives in the recursively processed control string may
consume arguments of the control string containing the \cd{{\Xtilde}{\Xatsign}?}
directive.
Example:
\begin{lisp}
(format nil "{\Xtilde}{\Xatsign}? {\Xtilde}D" "<{\Xtilde}A {\Xtilde}D>" "Foo" 5 7) \EV\ "<Foo 5> 7" \\
(format nil "{\Xtilde}{\Xatsign}? {\Xtilde}D" "<{\Xtilde}A {\Xtilde}D>" "Foo" 5 14 7) \EV\ "<Foo 5> 14"
\end{lisp}

Here is a rather sophisticated example.
The \cdf{format} function itself,
as implemented at one time in Lisp Machine Lisp,
used a routine internal to the \cdf{format} package called \cdf{format-error} to
signal error messages; \cdf{format-error} in turn used \cdf{error}, which used
\cdf{format} recursively.  Now \cdf{format-error} took a string and
arguments, just like \cdf{format}, but also printed the control string
to \cdf{format} (which at this point was available
in the global variable \cd{*ctl-string*}) and a little
arrow showing where in the processing of the control string the error
occurred.  The variable \cd{*ctl-index*} pointed one character after the
place of the error.
\begin{lisp}
(defun format-error (string \&rest args)~~~~~;\textrm{Example} \\
~~(error {\false} "{\Xtilde}?{\Xtilde}\%{\Xtilde}V{\Xatsign}T{\Xarrowdown}{\Xtilde}\%{\Xtilde}3{\Xatsign}T{\Xbackslash}"{\Xtilde}A{\Xbackslash}"{\Xtilde}\%" \\
~~~~~~~~~string args (+ *ctl-index* 3) *ctl-string*))
\end{lisp}
(The character set used in the Lisp Machine Lisp implementation contains a
down-arrow character \cd{\Xarrowdown}, which is not a standard Common Lisp
character.)  This first processed the given string and arguments using
\cd{{\Xtilde}?}, then output a newline, tabbed a variable amount for
printing the down-arrow, and printed the control string between
double quotes (note the use of \cd{{\Xbackslash}"} to include double quotes within
the control string).  The effect was something like this:
\begin{lisp}
(format t "The item is a {\Xtilde}{\Xlbracket}Foo{\Xtilde};Bar{\Xtilde};Loser{\Xtilde}{\Xrbracket}." 'quux) \\
>>ERROR: The argument to the FORMAT "{\Xtilde}{\Xlbracket}" command  \\
~~~~~~~~~must be a number. \\
~~~~~~~~~~~~~~~~~~~{\Xarrowdown} \\
~~~"The item is a {\Xtilde}{\Xlbracket}Foo{\Xtilde};Bar{\Xtilde};Loser{\Xtilde}{\Xrbracket}."
\end{lisp}

\beforenoterule
\begin{implementation}
Implementors may wish to report errors occurring
within \cdf{format} control strings in the manner outlined here.
It looks pretty flashy when done properly.
\end{implementation}
\afternoterule
\end{flushdesc}

\begin{new}
X3J13 voted in June 1989 \issue{PRETTY-PRINT-INTERFACE} to introduce
certain \cdf{format} directives to support the user interface to the pretty
printer described in detail in chapter~\ref{PPRINT}.

\begin{flushdesc}
\item[\cd{{\Xtilde}{\Xunderscore}}]
\emph{Conditional newline.} Without any modifiers, the directive
\cd{{\Xtilde}{\Xunderscore}} is equivalent to
\cd{(pprint-newline :linear)}.  The directive
\cd{{\Xtilde}{\Xatsign}{\Xunderscore}} is
equivalent to \cd{(pprint-newline :miser)}.  The directive
\cd{{\Xtilde}:{\Xunderscore}}
is equivalent to \cd{(pprint-newline :fill)}.  The directive
\cd{{\Xtilde}:{\Xatsign}{\Xunderscore}} is
equivalent to \cd{(pprint-newline :mandatory)}.

\item[\cd{{\Xtilde}W}]
\emph{Write.}  An \emph{arg}, any Lisp object, is printed obeying \emph{every}
printer control variable (as by \cdf{write}).
See section~\ref{PPRINT-FORMAT-DIRECTIVES-SECTION} for details.

\item[\cd{{\Xtilde}I}]
\emph{Indent.} The directive \cd{{\Xtilde}\emph{n}I} is equivalent to
\cd{(pprint-indent~:block~\emph{n})}.  The directive \cd{{\Xtilde}:\emph{n}I} is equivalent to
\cd{(pprint-indent~:current~\emph{n})}.  In both cases, \emph{n} defaults to zero,
if it is omitted.
\end{flushdesc}
\end{new}

The format directives after this point are much more complicated than the
foregoing; they constitute control structures that can perform
case conversion,
conditional selection, iteration, justification, and non-local exits.
Used with restraint, they can perform powerful tasks.  Used with
abandon, they can produce completely unreadable and unmaintainable code.

The case-conversion, conditional, iteration, and justification
constructs can contain other formatting constructs by bracketing them.
These constructs must nest properly with respect to each other.
For example, it is not legitimate to put the start of a case-conversion
construct in each arm of a conditional and the
end of the case-conversion construct outside the conditional:
\begin{lisp}
(format {\false} "{\Xtilde}:{\Xlbracket}abc{\Xtilde}:{\Xatsign}(def{\Xtilde};ghi{\Xtilde}:{\Xatsign}(jkl{\Xtilde}{\Xrbracket}mno{\Xtilde})" x)~~~~~;\textrm{Illegal!}
\end{lisp}
One might expect this to produce either \cd{"abcDEFMNO"} or \cd{"ghiJKLMNO"},
depending on whether \cdf{x} is false or true; but in fact the construction
is illegal because the \cd{{\Xtilde}{\Xlbracket}...{\Xtilde};...{\Xtilde}{\Xrbracket}}
and \cd{{\Xtilde}(...{\Xtilde})} constructs are not properly nested.

The processing indirection caused by the \cd{{\Xtilde}?} directive
is also a kind of nesting for the purposes of this rule of proper nesting.
It is not permitted to
start a bracketing construct within a string processed
under control of a \cd{{\Xtilde}?}
directive and end the construct at some point after the \cd{{\Xtilde}?} construct
in the string containing that construct, or vice versa.
For example, this situation is illegal:
\begin{lisp}
(format {\false} "{\Xtilde}?ghi{\Xtilde})" "abc{\Xtilde}{\Xatsign}(def")~~~~~;\textrm{Illegal!}
\end{lisp}
One might expect it to produce \cd{"abcDEFGHI"}, but in fact the construction
is illegal because the \cd{{\Xtilde}?}
and \cd{{\Xtilde}(...{\Xtilde})} constructs are not properly nested.

\begin{flushdesc}
\item[\cd{{\Xtilde}(\emph{str}{\Xtilde})}]
\emph{Case conversion}.
The contained control string \emph{str} is processed, and what it produces
is subject to case conversion:
\cd{{\Xtilde}(} converts every uppercase character
to the corresponding lowercase character;
\cd{{\Xtilde}:(} capitalizes all words, as if by \cdf{string-capitalize};
\cd{{\Xtilde}{\Xatsign}(} capitalizes just the first word and forces the rest to lowercase;
\cd{{\Xtilde}:{\Xatsign}(} converts every lowercase character
to the corresponding uppercase character.
In this example, \cd{{\Xtilde}{\Xatsign}(} is used to cause the first word
produced by \cd{{\Xtilde}{\Xatsign}R} to be capitalized:
\begin{lisp}
(format nil "{\Xtilde}{\Xatsign}R {\Xtilde}({\Xtilde}{\Xatsign}R{\Xtilde})" 14 14) \EV\ "XIV xiv" \\
(defun f (n) (format nil "{\Xtilde}{\Xatsign}({\Xtilde}R{\Xtilde}) error{\Xtilde}:P detected." n)) \\
(f 0) \EV\ "Zero errors detected." \\
(f 1) \EV\ "One error detected." \\
(f 23) \EV\ "Twenty-three errors detected."
\end{lisp}

\item[\cd{{\Xtilde}{\Xlbracket}\emph{str0}{\Xtilde};\emph{str1}{\Xtilde};\emph{...}{\Xtilde};\emph{strn}{\Xtilde}{\Xrbracket}}]
\emph{Conditional expression}.
This is a set of control strings, called \emph{clauses}, one of which is
chosen and used.  The clauses are separated by \cd{{\Xtilde};} and the construct
is terminated by \cd{{\Xtilde}{\Xrbracket}}.  For example,
\begin{lisp}
"{\Xtilde}{\Xlbracket}Siamese{\Xtilde};Manx{\Xtilde};Persian{\Xtilde}{\Xrbracket} Cat"
\end{lisp}
The \emph{arg}th
clause is selected, where the first clause is number 0.
If a prefix parameter is given (as \cd{{\Xtilde}\emph{n}{\Xlbracket}}),
then the parameter is used instead of an argument.
(This is useful only if the parameter is specified by \cd{\#},
to dispatch on the number of arguments remaining to be processed.)
If \emph{arg} is out of range, then no clause is selected
(and no error is signaled).
After the selected alternative has been processed, the control string
continues after the \cd{{\Xtilde}{\Xrbracket}}.

\cd{{\Xtilde}{\Xlbracket}\emph{str0}{\Xtilde};\emph{str1}{\Xtilde};\emph{...}{\Xtilde};\emph{strn}{\Xtilde}:;\emph{default}{\Xtilde}{\Xrbracket}} has a default case.
If the \emph{last} \cd{{\Xtilde};} used to separate clauses
is \cd{{\Xtilde}:;} instead, then the last clause is an ``else'' clause
that is performed if no other clause is selected.
For example:
\begin{lisp}
"{\Xtilde}{\Xlbracket}Siamese{\Xtilde};Manx{\Xtilde};Persian{\Xtilde}:;Alley{\Xtilde}{\Xrbracket} Cat"
\end{lisp}

\cd{{\Xtilde}:{\Xlbracket}\emph{false}{\Xtilde};\emph{true}{\Xtilde}{\Xrbracket}} selects the \emph{false} control string
if \emph{arg} is {\false}, and selects the \emph{true} control string otherwise.

\cd{{\Xtilde}{\Xatsign}{\Xlbracket}\emph{true}{\Xtilde}{\Xrbracket}} tests the argument.  If it is not {\false},
then the argument is not used up by the \cd{{\Xtilde}{\Xatsign}{\Xlbracket}} command
but remains as the next one to be processed,
and the one clause \emph{true} is processed.
If the \emph{arg} is {\false}, then the argument is used up,
and the clause is not processed.
The clause therefore should normally use exactly one argument,
and may expect it to be non-{\false}.
For example:
\begin{lisp}
(setq *print-level* {\false} *print-length* 5) \\*
(format {\false} "{\Xtilde}{\Xatsign}{\Xlbracket} print level = {\Xtilde}D{\Xtilde}{\Xrbracket}{\Xtilde}{\Xatsign}{\Xlbracket} print length = {\Xtilde}D{\Xtilde}{\Xrbracket}" \\*
~~~~~~~~~~~~*print-level* *print-length*) \\*
~~~\EV\  " print length = 5"
\end{lisp}

The combination of \cd{{\Xtilde}{\Xlbracket}} and \cd{\#} is useful, for
example, for dealing with English conventions for printing lists:
\begin{lisp}
(setq foo "Items:{\Xtilde}\#{\Xlbracket} none{\Xtilde}; {\Xtilde}S{\Xtilde}; {\Xtilde}S and {\Xtilde}S{\Xtilde} \\
~~~~~~~~~~~{\Xtilde}:;{\Xtilde}{\Xatsign}{\Xlbrace}{\Xtilde}\#{\Xlbracket}{\Xtilde}; and{\Xtilde}{\Xrbracket}
{\Xtilde}S{\Xtilde}{\Xcircumflex},{\Xtilde}{\Xrbrace}{\Xtilde}{\Xrbracket}.") \\
(format {\false} foo) \\ ~~~~~~~~\EV\  "Items: none." \\
(format {\false} foo 'foo) \\
~~~~~~~~\EV\  "Items: FOO." \\
(format {\false} foo 'foo 'bar) \\
~~~~~~~~\EV\  "Items: FOO and BAR." \\
(format {\false} foo 'foo 'bar 'baz) \\
~~~~~~~~\EV\  "Items: FOO, BAR, and BAZ." \\
(format {\false} foo 'foo 'bar 'baz 'quux) \\
~~~~~~~~\EV\  "Items: FOO, BAR, BAZ, and QUUX."
\end{lisp}

\item[\cd{{\Xtilde};}]
This separates clauses in \cd{{\Xtilde}{\Xlbracket}} and \cd{{\Xtilde}<}
constructions.  It is an error elsewhere.

\item[\cd{{\Xtilde}{\Xrbracket}}]
This terminates a \cd{{\Xtilde}{\Xlbracket}}.  It is an error elsewhere.

\item[\cd{{\Xtilde}{\Xlbrace}\emph{str}{\Xtilde}{\Xrbrace}}]
\emph{Iteration}.
This is an iteration construct.  The argument should be a list,
which is used as a set of arguments as if for a recursive call to \cdf{format}.
The string \emph{str} is used repeatedly as the control string.
Each iteration can absorb as many elements of the list as it likes
as arguments;
if \emph{str} uses up two arguments by itself, then two elements of the
list will get used up each time around the loop.
If before any iteration step the list is empty, then the iteration is terminated.
Also, if a prefix parameter \emph{n} is given, then there will be at most \emph{n}
repetitions of processing of \emph{str}.  Finally, the
\cd{{\Xtilde}{\Xcircumflex}} directive can be used to terminate the iteration
prematurely.

Here are some simple examples:
\begin{lisp}
(format {\false} \\
~~~~~~~~"The winners are:{\Xtilde}{\Xlbrace} {\Xtilde}S{\Xtilde}{\Xrbrace}." \\
~~~~~~~~'(fred harry jill)) \\
~~~~~\EV\ "The winners are: FRED HARRY JILL." \\
\\
(format {\false} "Pairs:{\Xtilde}{\Xlbrace} <{\Xtilde}S,{\Xtilde}S>{\Xtilde}{\Xrbrace}." '(a 1 b 2 c 3)) \\
~~~~~\EV\ "Pairs: <A,1> <B,2> <C,3>."
\end{lisp}

\cd{{\Xtilde}:{\Xlbrace}\emph{str}{\Xtilde}{\Xrbrace}} is similar, but the argument should be a list of sublists.
At each repetition step, one sublist is used as the set of arguments for
processing \emph{str}; on the next repetition, a new sublist is used, whether
or not all of the last sublist had been processed.  Example:
\begin{lisp}
(format {\false} "Pairs:{\Xtilde}:{\Xlbrace} <{\Xtilde}S,{\Xtilde}S>{\Xtilde}{\Xrbrace}." \\
~~~~~~~~~~~~'((a 1) (b 2) (c 3))) \\
~~~~~\EV\ "Pairs: <A,1> <B,2> <C,3>."
\end{lisp}

\cd{{\Xtilde}{\Xatsign}{\Xlbrace}\emph{str}{\Xtilde}{\Xrbrace}} is similar to \cd{{\Xtilde}{\Xlbrace}\emph{str}{\Xtilde}{\Xrbrace}}, but instead of
using one argument that is a list, all the remaining arguments
are used as the list of arguments for the iteration.
Example:
\begin{lisp}
(format {\false} "Pairs:{\Xtilde}{\Xatsign}{\Xlbrace} <{\Xtilde}S,{\Xtilde}S>{\Xtilde}{\Xrbrace}." \\
~~~~~~~~~~~~'a 1 'b 2 'c 3) \\
~~~~~\EV\ "Pairs: <A,1> <B,2> <C,3>."
\end{lisp}
If the iteration is terminated before all the remaining arguments are
consumed, then any arguments not processed by the iteration remain to be
processed by any directives following the iteration construct.

\cd{{\Xtilde}:{\Xatsign}{\Xlbrace}\emph{str}{\Xtilde}{\Xrbrace}}
combines the features of \cd{{\Xtilde}:{\Xlbrace}\emph{str}{\Xtilde}{\Xrbrace}}
and \cd{{\Xtilde}{\Xatsign}{\Xlbrace}\emph{str}{\Xtilde}{\Xrbrace}}.
All the remaining arguments
are used, and each one must be a list.
On each iteration, the next argument is used as a list of arguments to \emph{str}.
Example:
\begin{lisp}
(format {\false} "Pairs:{\Xtilde}:{\Xatsign}{\Xlbrace} <{\Xtilde}S,{\Xtilde}S>{\Xtilde}{\Xrbrace}." \\
~~~~~~~~~~~~'(a 1) '(b 2) '(c 3)) \\
~~~~~\EV\ "Pairs: <A,1> <B,2> <C,3>."
\end{lisp}

Terminating the repetition construct with \cd{{\Xtilde}:{\Xrbrace}} instead of
\cd{{\Xtilde}{\Xrbrace}} forces \emph{str} to be processed at least once, even if
the initial list of arguments is null (however, it will not override an explicit
prefix parameter of zero).

If \emph{str} is empty, then an argument is used as \emph{str}.  It must be a string
and precede any arguments processed by the iteration.  As an example,
the following are equivalent:
\begin{lisp}
(apply \#'format stream string arguments) \\
(format stream "{\Xtilde}1{\Xlbrace}{\Xtilde}:{\Xrbrace}" string arguments)
\end{lisp}
This will use \cdf{string} as a formatting string.  The \cd{{\Xtilde}1{\Xlbrace}} says it will
be processed at most once, and the \cd{{\Xtilde}:{\Xrbrace}} says it will be processed at least once.
Therefore it is processed exactly once, using \cdf{arguments} as the arguments.
This case may be handled more clearly by the \cd{{\Xtilde}?} directive,
but this general feature of \cd{{\Xtilde}{\Xlbrace}}
is more powerful than \cd{{\Xtilde}?}.

\item[\cd{{\Xtilde}{\Xrbrace}}]
This terminates a \cd{{\Xtilde}{\Xlbrace}}.  It is an error elsewhere.

\item[\cd{{\Xtilde}\emph{mincol},\emph{colinc},\emph{minpad},\emph{padchar}<\emph{str}{\Xtilde}>}]
\emph{Justification}.
This justifies the text produced by processing \emph{str}
within a field at least \emph{mincol} columns wide.  \emph{str}
may be divided up into segments with \cd{{\Xtilde};}, in which case the
spacing is evenly divided between the text segments.

With no modifiers, the leftmost text segment is left-justified in the
field, and the rightmost text segment right-justified;  if there is
only one text element, as a special case, it is right-justified.
The \cd{:} modifier causes
spacing to be introduced before the first text segment;  the \cd{{\Xatsign}}
modifier causes spacing to be added after the last.
The \emph{minpad} parameter (default \cd{0}) is the minimum number of
padding characters to be output between each segment.
The padding character is specified by \emph{padchar},
which defaults to the space character.
If the total width needed to satisfy these constraints is greater
than \emph{mincol}, then the width used is \emph{mincol}+\emph{k}*\emph{colinc}
for the smallest possible non-negative integer value \emph{k};
\emph{colinc} defaults to \cd{1}, and \emph{mincol} defaults to \cd{0}.

\begin{lisp}
(format {\false} "{\Xtilde}10<foo{\Xtilde};bar{\Xtilde}>")~~~~~~~~~~\EV\  "foo~~~~bar" \\*
(format {\false} "{\Xtilde}10:<foo{\Xtilde};bar{\Xtilde}>")~~~~~~~~~\EV\  "~~foo~~bar" \\
(format {\false} "{\Xtilde}10:{\Xatsign}<foo{\Xtilde};bar{\Xtilde}>")~~~~~~~~\EV\  "~~foo~bar~" \\
(format {\false} "{\Xtilde}10<foobar{\Xtilde}>")~~~~~~~~~~~~\EV\  "~~~~foobar" \\
(format {\false} "{\Xtilde}10:<foobar{\Xtilde}>")~~~~~~~~~~~\EV\  "~~~~foobar" \\
(format {\false} "{\Xtilde}10{\Xatsign}<foobar{\Xtilde}>")~~~~~~~~~~~\EV\  "foobar~~~~" \\*
(format {\false} "{\Xtilde}10:{\Xatsign}<foobar{\Xtilde}>")~~~~~~~~~~\EV\  "~~foobar~~"
\end{lisp}

Note that \emph{str} may include \cdf{format} directives.
All the clauses in \emph{str} are processed in order;
it is the resulting pieces of text that are justified.

The \cd{{\Xtilde}{\Xcircumflex}} directive may be used to terminate processing
of the clauses prematurely, in which case only the completely processed clauses
are justified.

If the first clause of a \cd{{\Xtilde}<} is terminated with \cd{{\Xtilde}:;} instead of
\cd{{\Xtilde};}, then it is used in a special way.  All of the clauses are
processed (subject to \cd{{\Xtilde}{\Xcircumflex}}, of course), but the first one is not used
in performing the spacing and padding.  When the padded result has been
determined, then if it will fit on the current line of output, it is
output, and the text for the first clause is discarded.  If, however, the
padded text will not fit on the current line, then the text segment for
the first clause is output before the padded text.  The first clause
ought to contain a newline (such as a \cd{{\Xtilde}\%} directive).  The first
clause is always processed, and so any arguments it refers to will be
used; the decision is whether to use the resulting segment of text, not
whether to process the first clause.  If the \cd{{\Xtilde}:;} has a prefix
parameter \emph{n}, then the padded text must fit on the current line with
\emph{n} character positions to spare to avoid outputting the first clause's
text.  For example, the control string
\begin{lisp}
"{\Xtilde}\%;; {\Xtilde}{\Xlbrace}{\Xtilde}<{\Xtilde}\%;; {\Xtilde}1:; {\Xtilde}S{\Xtilde}>{\Xtilde}{\Xcircumflex},{\Xtilde}{\Xrbrace}.{\Xtilde}\%"
\end{lisp}
can be used to print a list of items separated by commas without
breaking items over line boundaries, beginning each line with
\cd{;; }.  The prefix parameter \cd{1} in \cd{{\Xtilde}1:;} accounts for the width of the
comma that will follow the justified item if it is not the last
element in the list, or the period if it is.  If \cd{{\Xtilde}:;} has a second
prefix parameter, then it is used as the width of the line,
thus overriding the natural line width of the output stream.  To make
the preceding example use a line width of 50, one would write
\begin{lisp}
"{\Xtilde}\%;; {\Xtilde}{\Xlbrace}{\Xtilde}<{\Xtilde}\%;; {\Xtilde}1,50:; {\Xtilde}S{\Xtilde}>{\Xtilde}{\Xcircumflex},{\Xtilde}{\Xrbrace}.{\Xtilde}\%"
\end{lisp}

If the second argument is not specified, then \cdf{format} uses the
line width of the output stream.
If this cannot be determined (for example, when producing a string result),
then \cdf{format} uses \cd{72} as the line length.

\item[\cd{{\Xtilde}>}]
Terminates a \cd{{\Xtilde}<}.  It is an error elsewhere.
\begin{new}
X3J13 voted in June 1989 \issue{PRETTY-PRINT-INTERFACE} to introduce
certain \cdf{format} directives to support the user interface to the pretty
printer.  If \cd{{\Xtilde}:>} is used to terminate a
\cd{{\Xtilde}<...} directive, the directive is equivalent to a call on
\cdf{pprint-logical-block}.
See section~\ref{PPRINT-FORMAT-DIRECTIVES-SECTION} for details.
\end{new}

\item[\cd{{\Xtilde}{\Xcircumflex}}]
\emph{Up and out}.
This is an escape construct.  If there are no more arguments remaining to
be processed, then the immediately enclosing \cd{{\Xtilde}{\Xlbrace}} or \cd{{\Xtilde}<} construct
is terminated.  If there is no such enclosing construct, then the entire
formatting operation is terminated.  In the \cd{{\Xtilde}<} case, the formatting
\emph{is} performed, but no more segments are processed before doing the
justification.  The \cd{{\Xtilde}{\Xcircumflex}} should appear only at the \emph{beginning} of a
\cd{{\Xtilde}<} clause, because it aborts the entire clause it appears in (as well
as all following clauses).
\cd{{\Xtilde}{\Xcircumflex}} may appear anywhere in a \cd{{\Xtilde}{\Xlbrace}}
construct.
\begin{lisp}
(setq donestr "Done.{\Xtilde}{\Xcircumflex}  {\Xtilde}D warning{\Xtilde}:P.{\Xtilde}{\Xcircumflex}  {\Xtilde}D error{\Xtilde}:P.") \\
(format {\false} donestr) \EV\ "Done." \\
(format {\false} donestr 3) \EV\ "Done.  3 warnings." \\
(format {\false} donestr 1 5) \EV\ "Done.  1 warning.  5 errors."
\end{lisp}

If a prefix parameter is given, then termination occurs if the parameter
is zero.  (Hence \cd{{\Xtilde}{\Xcircumflex}} is equivalent to \cd{{\Xtilde}\#{\Xcircumflex}}.)  If two
parameters are given, termination occurs if they are equal.  If three
parameters are given, termination occurs if the first is less than or
equal to the second and the second is less than or equal to the third.
Of course, this is useless if all the prefix parameters are constants; at
least one of them should be a \cd{\#} or a \cdf{V} parameter.

If \cd{{\Xtilde}{\Xcircumflex}} is used within a \cd{{\Xtilde}:{\Xlbrace}} construct, then it merely terminates
the current iteration step (because in the standard case it tests for
remaining arguments of the current step only); the next iteration step
commences immediately.  To terminate the entire iteration process,
use \cd{{\Xtilde}:{\Xcircumflex}}.

\begin{new}
X3J13 voted in March 1988
\issue{FORMAT-COLON-UPARROW-SCOPE}
to clarify the behavior of \cd{{\Xtilde}:{\Xcircumflex}} as follows.
It may be used only if the command it would terminate is \cd{{\Xtilde}:{\Xlbrace}}
or \cd{{\Xtilde}:{\Xatsign}{\Xlbrace}}.  The entire iteration process is terminated
if and only if the sublist that is supplying the arguments for the current iteration step
is the last sublist (in the case of terminating a \cd{{\Xtilde}:{\Xlbrace}} command)
or the last argument to that call to \cdf{format} (in the
case of terminating a \cd{{\Xtilde}:{\Xatsign}{\Xlbrace}} command).
Note furthermore that while \cd{{\Xtilde}{\Xcircumflex}} is equivalent
to \cd{{\Xtilde}\#{\Xcircumflex}} in all circumstances,
\cd{{\Xtilde}:{\Xcircumflex}} is \emph{not} equivalent
to \cd{{\Xtilde}:\#{\Xcircumflex}} because the latter terminates the entire iteration
if and only if no arguments remain for \emph{the current iteration step}
(as opposed to no arguments remaining for the entire iteration process).

Here are some examples of the differences in the
behaviors of \cd{{\Xtilde}{\Xcircumflex}}, \cd{{\Xtilde}:{\Xcircumflex}},
and \cd{{\Xtilde}:\#{\Xcircumflex}}.
\begin{lisp}
(format nil \\*
~~~~~~~~"{\Xtilde}:{\Xlbrace}/{\Xtilde}S{\Xtilde}{\Xcircumflex} ...{\Xtilde}{\Xrbrace}" \\*
~~~~~~~~'((hot dog) (hamburger) (ice cream) (french fries))) \\*
~\EV~"/HOT .../HAMBURGER/ICE .../FRENCH ..."
\end{lisp}
For each sublist, ``\cd{ ...}'' appears after the first word unless there are no
additional words.
\begin{lisp}
(format nil \\*
~~~~~~~~"{\Xtilde}:{\Xlbrace}/{\Xtilde}S{\Xtilde}:{\Xcircumflex} ...{\Xtilde}{\Xrbrace}" \\*
~~~~~~~~'((hot dog) (hamburger) (ice cream) (french fries))) \\*
~\EV~"/HOT .../HAMBURGER .../ICE .../FRENCH"
\end{lisp}
For each sublist, ``\cd{ ...}'' always appears after the first word, unless it is the
last sublist, in which case the entire iteration is terminated.
\begin{lisp}
(format nil \\*
~~~~~~~~"{\Xtilde}:{\Xlbrace}/{\Xtilde}S{\Xtilde}:\#{\Xcircumflex} ...{\Xtilde}{\Xrbrace}" \\*
~~~~~~~~'((hot dog) (hamburger) (ice cream) (french fries))) \\*
~\EV~"/HOT .../HAMBURGER"
\end{lisp}
For each sublist, ``\cd{ ...}'' appears after the first word, but if the sublist
has only one word then the entire iteration is terminated.
\end{new}

If \cd{{\Xtilde}{\Xcircumflex}} appears within a control string being processed
under the control of a \cd{{\Xtilde}?} directive, but not within
any \cd{{\Xtilde}{\Xlbrace}} or \cd{{\Xtilde}<} construct within that string,
then the string being
processed will be terminated, thereby ending processing
of the \cd{{\Xtilde}?} directive.  Processing then
continues within the string
containing the \cd{{\Xtilde}?} directive at the point following that directive.

If \cd{{\Xtilde}{\Xcircumflex}} appears within a \cd{{\Xtilde}{\Xlbracket}} or \cd{{\Xtilde}(} construct,
then all the commands up to the \cd{{\Xtilde}{\Xcircumflex}} are properly selected
or case-converted, the \cd{{\Xtilde}{\Xlbracket}} or \cd{{\Xtilde}(} processing is terminated,
and the outward search continues for a \cd{{\Xtilde}{\Xlbrace}} or \cd{{\Xtilde}<} construct
to be terminated.  For example:
\begin{lisp}
(setq tellstr "{\Xtilde}{\Xatsign}({\Xtilde}{\Xatsign}{\Xlbracket}{\Xtilde}R{\Xtilde}{\Xrbracket}{\Xtilde}{\Xcircumflex} {\Xtilde}A.{\Xtilde})") \\
(format {\false} tellstr 23) \EV\ "Twenty-three." \\
(format {\false} tellstr nil "losers") \EV\ "Losers." \\
(format {\false} tellstr 23 "losers") \EV\ "Twenty-three losers."
\end{lisp}

Here are some examples of the use of \cd{{\Xtilde}{\Xcircumflex}} within a \cd{{\Xtilde}<} construct.
\begin{lisp}
(format {\false} "{\Xtilde}15<{\Xtilde}S{\Xtilde};{\Xtilde}{\Xcircumflex}{\Xtilde}S{\Xtilde};{\Xtilde}{\Xcircumflex}{\Xtilde}S{\Xtilde}>" 'foo) \\
~~~~~~~~\EV\  "~~~~~~~~~~~~FOO" \\
(format {\false} "{\Xtilde}15<{\Xtilde}S{\Xtilde};{\Xtilde}{\Xcircumflex}{\Xtilde}S{\Xtilde};{\Xtilde}{\Xcircumflex}{\Xtilde}S{\Xtilde}>" 'foo 'bar) \\
~~~~~~~~\EV\  "FOO~~~~~~~~~BAR" \\
(format {\false} "{\Xtilde}15<{\Xtilde}S{\Xtilde};{\Xtilde}{\Xcircumflex}{\Xtilde}S{\Xtilde};{\Xtilde}{\Xcircumflex}{\Xtilde}S{\Xtilde}>" 'foo 'bar 'baz) \\
~~~~~~~~\EV\  "FOO~~~BAR~~~BAZ"
\end{lisp}
\end{flushdesc}

\begin{new}
X3J13 voted in June 1989 \issue{PRETTY-PRINT-INTERFACE}
to introduce user-defined directives in the form of the
\cd{{\Xtilde}/.../} directive.
See section~\ref{PPRINT-FORMAT-DIRECTIVES-SECTION} for details.
\end{new}

\begin{new}
The hairiest \cdf{format} control string I have ever seen in shown
in table~\ref{XAPPING-FORMAT-TABLE}.
It started innocently enough as part of the simulator for Connection Machine Lisp
\cite{CONNECTION-MACHINE-LISP,CMLISP-IMPLEMENTATION}; the \emph{xapping} data type,
defined by \cdf{defstruct}, needed a \cd{:print-function} option so that
xappings would print properly.  As this data type became more complicated, step by step,
so did the \cdf{format} control string.
\end{new}
\begin{table}[t]
\caption{Print Function for the Xapping Data Type}
\label{XAPPING-FORMAT-TABLE}
\begingroup
\small
\begin{lisp}
(defun print-xapping (xapping stream depth) \\
~~(declare (ignore depth)) \\
~~(format stream \\
~~~~~~~~~~;; Are you ready for this one? \\
~~~~~~~~~~"{\Xtilde}:[{\Xlbrace}{\Xtilde};[{\Xtilde}]{\Xtilde}:{\Xlbrace}{\Xtilde}S{\Xtilde}:[{\Xarrowright}{\Xtilde}S{\Xtilde};{\Xtilde}*{\Xtilde}]{\Xtilde}:{\Xcircumflex} {\Xtilde}{\Xrbrace}{\Xtilde}:[{\Xtilde}; {\Xtilde}]{\Xtilde} \\
~~~~~~~~~~~{\Xtilde}{\Xlbrace}{\Xtilde}S{\Xarrowright}{\Xtilde}{\Xcircumflex} {\Xtilde}{\Xrbrace}{\Xtilde}:[{\Xtilde}; {\Xtilde}]{\Xtilde}[{\Xtilde}*{\Xtilde};{\Xarrowright}{\Xtilde}S{\Xtilde};{\Xarrowright}{\Xtilde}*{\Xtilde}]{\Xtilde}:[{\Xrbrace}{\Xtilde};]{\Xtilde}]" \\
~~~~~~~~~~;; Is that clear? \\
~~~~~~~~~~(xectorp xapping) \\
~~~~~~~~~~(do ((vp (xectorp xapping)) \\
~~~~~~~~~~~~~~~(sp (finite-part-is-xetp xapping)) \\
~~~~~~~~~~~~~~~(d (xapping-domain xapping) (cdr d)) \\
~~~~~~~~~~~~~~~(r (xapping-range xapping) (cdr r)) \\
~~~~~~~~~~~~~~~(z '() (cons (list (if vp (car r) (car d)) \\
~~~~~~~~~~~~~~~~~~~~~~~~~~~~~~~~~~(or vp sp) \\
~~~~~~~~~~~~~~~~~~~~~~~~~~~~~~~~~~(car r)) \\
~~~~~~~~~~~~~~~~~~~~~~~~~~~~z))) \\
~~~~~~~~~~~~~~((null d) (reverse z))) \\
~~~~~~~~~~(and (xapping-domain xapping) \\
~~~~~~~~~~~~~~~(or (xapping-exceptions xapping) \\
~~~~~~~~~~~~~~~~~~~(xapping-infinite xapping))) \\
~~~~~~~~~~(xapping-exceptions xapping) \\
~~~~~~~~~~(and (xapping-exceptions xapping) \\
~~~~~~~~~~~~~~~(xapping-infinite xapping)) \\
~~~~~~~~~~(ecase (xapping-infinite xapping) \\
~~~~~~~~~~~~((nil) 0) \\
~~~~~~~~~~~~(:constant 1) \\
~~~~~~~~~~~~(:universal 2)) \\
~~~~~~~~~~(xapping-default xapping) \\
~~~~~~~~~~(xectorp xapping)))
\end{lisp}
\endgroup
See section~\ref{READTABLE-SECTION} for the \cdf{defstruct} definition of the \cdf{xapping} data
type, whose accessor functions are used in this code.
\end{table}

\begin{new}
See the description of \cdf{set-macro-character} for a discussion of xappings
and the \cdf{defstruct} definition.  Assume that the predicate \cdf{xectorp}
is true of a xapping if it is a xector, and that the predicate \cdf{finite-part-is-xetp}
is true if every value in the range is the same as its corresponding index.

Here is a blow-by-blow description of the parts of this format string:
\end{new}

\begin{flushleft}
\begin{tabular*}{\textwidth}{@{}l@{\extracolsep{\fill}}p{17pc}@{}}
\cd{{\Xtilde}:[{\Xlbrace}{\Xtilde};[{\Xtilde}]}&
Print ``\cd{[}'' for a xector, and ``\cd{{\Xlbrace}}'' otherwise. \\
\cd{{\Xtilde}:{\Xlbrace}{\Xtilde}S{\Xtilde}:[{\Xarrowright}{\Xtilde}S{\Xtilde};{\Xtilde}*{\Xtilde}]{\Xtilde}:{\Xcircumflex} {\Xtilde}{\Xrbrace}}& 
Given a list of lists, print the pairs.  Each sublist has three elements:
the index (or the value if we're printing a xector); a flag that is true for
either a xector or xet (in which case no arrow is printed);
and the value.  Note the use of \cd{{\Xtilde}:{\Xlbrace}} to iterate, and the use
of \cd{{\Xtilde}:{\Xcircumflex}} to avoid printing a separating space after the
final pair (or at all, if there are no pairs). \\
\cd{{\Xtilde}:[{\Xtilde}; {\Xtilde}]}&
If there were pairs and there are exceptions or an infinite part, print a separating space. \\
\cd{{\Xtilde}$\langle$\textrm{newline}$\rangle$}&
Do nothing.  This merely allows the format control string to be broken across two lines. \\
\cd{{\Xtilde}{\Xlbrace}{\Xtilde}S{\Xarrowright}{\Xtilde}{\Xcircumflex} {\Xtilde}{\Xrbrace}}&
Given a list of exception indices, print them.
Note the use of \cd{{\Xtilde}{\Xlbrace}} to iterate, and the use
of \cd{{\Xtilde}{\Xcircumflex}} to avoid printing a separating space after the
final exception (or at all, if there are no exceptions). \\
\cd{{\Xtilde}:[{\Xtilde}; {\Xtilde}]}&
If there were exceptions and there is an infinite part, print a separating space. \\
\cd{{\Xtilde}[{\Xtilde}*{\Xtilde};{\Xarrowright}{\Xtilde}S{\Xtilde};{\Xarrowright}{\Xtilde}*{\Xtilde}]}&
Use \cd{{\Xtilde}[} to choose one of three cases for printing the infinite part. \\
\cd{{\Xtilde}:[{\Xrbrace}{\Xtilde};]{\Xtilde}]}&
Print ``\cd{]}'' for a xector, and ``\cd{{\Xrbrace}}'' otherwise.
\end{tabular*}
\end{flushleft}

\section{Querying the User}
\indexterm{querying the user}
\indexterm{yes-or-no functions}

The following functions provide a convenient and consistent interface for
asking questions of the user.  Questions are printed and the answers are
read using the stream \cdf{*query-io*}, which normally is synonymous with
\cdf{*terminal-io*} but can be rebound to another stream for special
applications.

\begin{defun}[Function]
y-or-n-p &optional format-string &rest arguments

This predicate is for asking the user a question whose answer is either
``yes'' or ``no.''  It types out a message (if supplied), reads an answer
in some implementation-dependent manner (intended to be short and simple,
like reading a single character such as \cdf{Y} or \cdf{N}), and is
true if the answer was ``yes'' or false if the answer was ``no.''

If the \emph{format-string} argument is supplied and not {\false},
then a \cdf{fresh-line} operation is performed; then
a message is printed as if the \emph{format-string} and \emph{arguments}
were given to \cdf{format}.
Otherwise it is assumed that any message has already been printed
by other means.
If you want a question mark at the end of the message,
you must put it there yourself; \cdf{y-or-n-p} will not add it.
However, the message should not contain an explanatory note such
as \cd{(Y or N)}, because the nature of the interface provided for
\cdf{y-or-n-p} by a given implementation might not involve typing a
character on a keyboard; \cdf{y-or-n-p} will provide such a note
if appropriate.

All input and output are performed using the stream in the global
variable \cdf{*query-io*}.

Here are some examples of the use of \cdf{y-or-n-p}:
\begin{lisp}
(y-or-n-p "Produce listing file?") \\
(y-or-n-p "Cannot connect to network host {\Xtilde}S.  Retry?" host)
\end{lisp}

\cdf{y-or-n-p} should only be used for questions that the user knows
are coming or in situations where the user is known to be waiting for
a response of some kind.
If the user is unlikely to anticipate the question,
or if the consequences of the answer might be grave and irreparable,
then \cdf{y-or-n-p} should not be used because the user might type ahead
and thereby accidentally answer the question.
For such questions as ``Shall I delete all of your files?'' it is better to use
\cdf{yes-or-no-p}.
\end{defun}

\begin{defun}[Function]
yes-or-no-p &optional format-string &rest arguments

This predicate, like \cdf{y-or-n-p}, is for asking the user a question
whose answer is either ``yes'' or ``no.''  It types out a message (if
supplied), attracts the user's attention (for example, by ringing
the terminal's bell), and reads a reply in some
implementation-dependent manner.  It is intended that the reply require
the user to take more action than just a single keystroke, such as typing
the full word \cdf{yes} or \cdf{no} followed by a newline.

If the \emph{format-string} argument is supplied and not {\false},
then a \cdf{fresh-line} operation is performed; then
a message is printed as if the \emph{format-string} and \emph{arguments}
were given to \cdf{format}.
Otherwise it is assumed that any message has already been printed
by other means.
If you want a question mark at the end of the message,
you must put it there yourself; \cdf{yes-or-no-p} will not add it.
However, the message should not contain an explanatory note such
as \cd{(Yes or No)} because the nature of the interface provided for
\cdf{yes-or-no-p} by a given implementation might not involve typing the
reply on a keyboard; \cdf{yes-or-no-p} will provide such a note
if appropriate.

All input and output are performed using the stream in the global
variable \cdf{*query-io*}.

To allow the user to answer a yes-or-no question with a single
character, use \cdf{y-or-n-p}.  \cdf{yes-or-no-p} should be
used for unanticipated or momentous questions;
this is why it attracts attention
and why it requires a multiple-action sequence to answer it.
\end{defun}
          % Input/output functions
%Part{Files, Root = "CLM.MSS"}
%%%Chapter of Common Lisp Manual.  Copyright 1984, 1988, 1989 Guy L. Steele Jr.

\clearpage\def\pagestatus{FINAL PROOF}

\chapter{File System Interface}
\label{FILES}

A frequent use of streams is to communicate with a \emph{file system}
to which groups of data (files) can be written and from which files
can be retrieved.

Common Lisp defines a standard interface for dealing with such a file system.
This interface is designed to be simple and general enough to
accommodate the facilities provided by ``typical'' operating system
environments within which Common Lisp is likely to be implemented.
The goal is to make Common Lisp programs that perform only simple operations
on files reasonably portable.

To this end, Common Lisp assumes that files are named, that given a name one
can construct a stream connected to a file of that name, and that the
names can be fit into a certain canonical, implementation-independent
form called a \emph{pathname}.

Facilities are provided for manipulating pathnames, for creating
streams connected to files, and for manipulating the file system
through pathnames and streams.

\section {File Names}

Common Lisp programs need to use names to designate files.
The main difficulty in dealing with names of files is that different
file systems have different naming formats for files.
For example, here is a table of several file systems (actually,
operating systems that provide file systems) and what equivalent
file names might look like for each one:
\begin{flushleft}
\begin{tabular}{@{}l@{\hskip 2pc}l@{}}
System&File Name \\
\hlinesp
{TOPS-20}&\cd{<LISPIO>FORMAT.FASL.13} \\
{TOPS-10}&\cd{FORMAT.FAS{\Xlbracket}1,4{\Xrbracket}} \\
{ITS}&\cd{LISPIO;FORMAT FASL} \\
{MULTICS}&\cd{>udd>LispIO>format.fasl} \\
{TENEX}&\cd{<LISPIO>FORMAT.FASL;13} \\
{VAX}/{VMS}&\cd{{\Xlbracket}LISPIO{\Xrbracket}FORMAT.FAS;13} \\
{UNIX}&\cd{/usr/lispio/format.fasl} \\
\hline
\end{tabular}
\end{flushleft}
It would be impossible for each program that deals with file names to
know about each different file name format that exists; a new Common Lisp
implementation might use a format different from any of its predecessors.
Therefore, Common Lisp provides \emph{two} ways to represent file names:
\emph{namestrings}, which are strings in the implementation-dependent form
customary for the file system, and \emph{pathnames}, which are special abstract
data objects that represent file names in an implementation-independent
way.  Functions are provided to convert between these two representations,
and all manipulations of files can be expressed in machine-independent
terms by using pathnames.

In order to allow Common Lisp programs to operate in a network environment
that may have more than one kind of file system, the pathname facility
allows a file name to specify which file system is to be used.
In this context, each file system is called a \emph{host}, in keeping
with the usual networking terminology.

\begin{newer}
Different hosts may use different notations for file names.
Common Lisp allows customary notation to be used for each host, but
also supports
a system of logical pathnames that provides a standard framework for naming
files in a portable manner (see section~\ref{LOGICAL-PATHNAMES-SECTION}).
\end{newer}

\subsection{Pathnames}
\label{PATHNAME}
All file systems dealt with by Common Lisp are forced into a common framework,
in which files are named by a Lisp data object of type \cdf{pathname}.

A pathname always has six components, described below.  These components
are the common interface that allows programs to work the same way with
different file systems; the mapping of the pathname components into the
concepts peculiar to each file system is taken care of by the Common Lisp
implementation.


\begin{flushdesc}
\item[\emph{host}]
The name of the file system on which the file resides.

\item[\emph{device}]
Corresponds to the ``device'' or ``file structure'' concept in many
host file systems: the name of a (logical or physical) device containing files.

\item[\emph{directory}]
Corresponds to the ``directory'' concept in many host file systems:
the name of a group of related files
(typically those belonging to a single
user or project).

\item[\emph{name}]
The name of a group of files that can be thought of as
the ``same'' file.

\item[\emph{type}]
Corresponds to the ``filetype'' or ``extension'' concept in many host
file systems; identifies the type of file.  Files with the same names
but different types are usually related in some specific way, for instance,
one being a source file, another the compiled form of that source,
and a third the listing of error messages from the compiler.

\item[\emph{version}]
Corresponds to the ``version number'' concept in many host file systems.
Typically this is a number that is incremented every time the file is modified.
\end{flushdesc}

Note that a pathname is not necessarily the name of a specific file.
Rather, it is a specification (possibly only a partial specification) of
how to access a file.  A pathname need not correspond to any file that
actually exists, and more than one pathname can refer to the same file.
For example, the pathname with a version of ``newest'' may refer to the
same file as a pathname with the same components except a certain number
as the version.  Indeed, a pathname with version ``newest'' may refer to
different files as time passes, because the meaning of such a pathname
depends on the state of the file system.  In file systems with such
facilities as ``links,'' multiple file names, logical devices, and so on,
two pathnames that look quite different may turn out to address the same
file.  To access a file given a pathname, one must do a file system
operation such as
\cdf{open}.

Two important operations involving pathnames are \emph{parsing} and
\emph{merging}.  Parsing is the conversion of a namestring (which might be
something supplied interactively by the user when asked to supply the
name of a file) into a pathname object.  This operation is
implementation-dependent, because the format of namestrings
is implementation-dependent.
Merging takes a pathname with missing components
and supplies values for those components from a source of defaults.

Not all of the components of a pathname need to be specified.  If a
component of a pathname is missing, its value is {\nil}.  Before the file
system interface can do anything interesting with a file, such as opening the
file, all the missing components of a pathname must be filled in
(typically from a set of defaults).  Pathnames with missing components
may be used internally for various purposes;
in particular, parsing a namestring
that does not specify certain components will result in a pathname with
missing components.

\begin{newer}
X3J13 voted in January 1989 \issue{PATHNAME-UNSPECIFIC-COMPONENT}
to permit any component of a pathname to have the value \cd{:unspecific},
meaning that the component simply does not exist,
for file systems in which such a value makes sense.
(For example, a UNIX file system usually does not support version numbers,
so the version component of a pathname for a UNIX host might be
\cd{:unspecific}.  Similarly,
the file type is usually regarded in a UNIX file system as the part
of a name after a period, but some file names contain no periods and therefore have
no file types.)

  When a pathname is converted to a namestring, the values \cdf{nil} and \cd{:unspecific}
  have the same effect: they
  are treated as if the component were empty (that is, they each cause the
  component not to appear in the namestring).
  When merging, however, only a \cdf{nil} value for a component will be
  replaced with the default for that component; the value \cd{:unspecific}
  will be left alone as if the field were filled.

  The results are undefined if \cd{:unspecific} is supplied
  to a file system in a component for which
  \cd{:unspecific} does not make sense for that file system.

  Programming hint:
  portable programs should be prepared to handle the value \cd{:unspecific} in the device,
  directory, type, or version field in some implementations.
  Portable programs should not explicitly place \cd{:unspecific} in any
  field because it might not be permitted in some situations,
  but portable programs may sometimes do so implicitly (by copying
  such a value from another pathname, for example).
\end{newer}

What values are allowed for components of a pathname depends, in general,
on the pathname's host.  However, in order for pathnames to be usable
in a system-independent way, certain global conventions are adhered to.
These conventions are stronger for the type and version than for the
other components, since the type and version are explicitly manipulated by
many programs, while the other components are usually treated as something
supplied by the user that just needs to be remembered and copied
from place to place.

The type is always a string or {\nil} or \cd{:wild}.
It is expected that most
programs that deal with files will supply a default type for each file.

The version is either a positive integer or a special symbol.  The
meanings of {\nil} and \cd{:wild} have been explained
above.  The keyword \cd{:newest} refers to the largest version number
that already exists in the file system when reading a file, or to
a version number
greater than any already existing in the file system
when writing a new file.  Some Common Lisp implementors
may choose to define other special version symbols.
Some semi-standard names, suggested but not required to be supported
by every Common Lisp implementation, are
\cd{:oldest}, to refer to the smallest version number that exists
in the file system;
\cd{:previous}, to refer to the version previous to the newest version;
and \cd{:installed}, to refer to a version that is officially installed
for users (as opposed to a working or development version).
Some Common Lisp implementors may also choose to attach a meaning to
non-positive version numbers (a typical convention is that \cd{0}
is synonymous with \cd{:newest} and \cd{-1} with \cd{:previous}),
but such interpretations are implementation-dependent.

The host may be a string, indicating a file system, or a list
of strings, of which the first names the file system and the rest
may be used for such a purpose as inter-network routing.

\begin{newer}
X3J13 voted in June 1989 \issue{PATHNAME-COMPONENT-VALUE}
to approve the following clarifications and specifications
of precisely what are valid values for the various components
of a pathname.

\makeatletter
\def\@listi{\leftmargin\leftmargini \labelsep\leftmargin
   \parsep 3pt\relax
   \topsep 2pt plus 5pt\relax
   \itemsep\topsep}
\makeatother

  Pathname component value strings never contain the punctuation
  characters that are used to separate fields in a namestring (for example,
  slashes and
  periods as used in UNIX file systems).  Punctuation characters appear only in namestrings.
  Characters used as punctuation can appear in pathname component values
  with a non-punctuation meaning if the file system allows it (for example,
  UNIX file systems allow a file name to begin with a period).

  When examining pathname components, conforming programs must be prepared
  to encounter any of the following siutations:
  \begin{itemize}
    \item Any component can be \cdf{nil}, which means the component has not
    been specified.
  
   \item  Any component can be \cd{:unspecific}, which means the component has
    no meaning in this particular pathname.
  
   \item  The device, directory, name, and type can be strings.
  
   \item  The host can be any object, at the discretion of the implementation.
  
  \item  The directory can be a list of strings and symbols as described in
    section~\ref{STRUCTURED-DIRECTORY-SECTION}.
  
  \item  The version can be any symbol or any integer.  The symbol \cd{:newest}
    refers to the largest version number that already exists in the file
    system when reading, overwriting, appending, superseding, or
    directory-listing an existing file; it refers to the smallest version number
    greater than any existing version number when creating a new file.
    Other symbols and integers have implementation-defined meaning.
    It is suggested, but not required, that implementations use positive
    integers starting at 1 as version numbers, recognize the symbol \cd{:oldest}
    to designate the smallest existing version number, and use keyword
    symbols for other special versions.
  \end{itemize}

  When examining wildcard components of a wildcard pathname, conforming programs
  must be prepared to encounter any of the following additional values
  in any component or any element of a list that is the directory component:
  \begin{itemize}
    \item The symbol \cd{:wild}, which matches anything.

   \item  A string containing implementation-dependent special wildcard
    characters.

    \item  Any object, representing an implementation-dependent wildcard pattern.
  \end{itemize}

  When constructing a pathname from components, conforming programs
  must follow these rules:
  \begin{itemize}
    \item  Any component may be \cdf{nil}.  Specifying \cdf{nil} for the host may,
     in some implementations,
    result in using a default host
    rather than an actual \cdf{nil} value.
    
    \item  The host, device, directory, name, and type may be strings.  There
    are implementation-dependent limits on the number and type of
    characters in these strings.  A plausible assumption is that letters (of a single case)
    and digits are acceptable to most file systems.
  
    \item  The directory may be a list of strings and symbols as described in
    section~\ref{STRUCTURED-DIRECTORY-SECTION}.  There are
    implementation-dependent limits on the length and contents of the list.
  
   \item  The version may be \cd{:newest}.

    \item  Any component may be taken from the corresponding component
    of another pathname.  When the two pathnames are for different
    file systems (in implementations that support multiple file
    systems), an appropriate translation occurs.  If no meaningful
    translation is possible, an error is signaled.  The definitions
    of ``appropriate'' and ``meaningful'' are implementation-dependent.
  
    \item  When constructing a wildcard pathname, the name, type, or version
    may be \cd{:wild}, which matches anything.
  
   \item  An implementation might support other values for some components,
    but a portable program should not use those values.  A conforming program
    can use implementation-dependent values but this can make it
    non-portable; for example, it might work only with UNIX file systems.
   \end{itemize}
\end{newer}

The best way to compare two pathnames for equality is with \cdf{equal},
not \cdf{eql}.
(On pathnames, \cdf{eql} is simply the same as \cdf{eq}.)
Two pathname objects are \cdf{equal} if and only if
all the corresponding components
(host, device, and so on) are equivalent.  (Whether or not
uppercase and lowercase letters are considered equivalent
in strings appearing in components depends on the file
name conventions of the file system.)  Pathnames
that are \cdf{equal} should be functionally equivalent.

\subsection{Case Conventions}
\label{PATHNAME-COMPONENT-CASE-SECTION}

  Issues of alphabetic case in pathnames are a major source of problems.
  In some file systems, the customary case is lowercase, in some uppercase,
  in some mixed.  Some file systems are case-sensitive (that is, they treat
  \cdf{FOO} and \cdf{foo} as different file names) and others are not.

  There are two kinds of pathname case portability problems: moving
  programs from one Common Lisp to another, and moving pathname component
  values from one file system to another.  The solution to the first problem
  is the requirement that all
  Common Lisp implementations that support a particular file system must
  use compatible representations for pathname component values.  The solution to
  the second problem is the use of a common representation for the
  least-common-denominator pathname component values that exist on all
  interesting file systems.

  Requiring a common representation directly conflicts with the
  desire among programmers that use only one file system to work with the
  local conventions and to ignore issues of porting to other file
  systems.  The common representation cannot be the same as local (varying)
  conventions.

X3J13 voted in June 1989 \issue{PATHNAME-COMPONENT-CASE} to
add a keyword argument \cd{:case} to each of the functions
  \cdf{make-pathname}, \cdf{pathname-host},
  \cdf{pathname-device}, \cdf{pathname-directory}, \cdf{pathname-name},
  and \cdf{pathname-type}.
  The possible values for the argument are \cd{:common} and \cd{:local}.
  The default is \cd{:local}.

  The value \cd{:local} means that strings given to \cdf{make-pathname}
  or returned by any of the pathname component accessors
  follow the local file system's conventions for alphabetic case.
  Strings given to \cdf{make-pathname} will be used exactly as written if
  the file system supports both cases.  If the file system
  supports only one case, the strings will be translated to that case.

  The value \cd{:common} means that strings given to \cdf{make-pathname}
  or returned by any of the pathname component accessors
  follow this common convention:
\begin{itemize}
\item All uppercase means that a file system's customary case will be used.
\item All lowercase means that the opposite of the customary case will be used.
\item Mixed case represents itself.
\end{itemize}
  Uppercase is used as the common case for no better reason than
  consistency with Lisp symbols.
  The second and third points allow translation from local representation to
  common and back to be information-preserving.  (Note that translation
  from common to local representation and back may or may not be information-preserving,
  depending on the nature of the local representation.)

  Namestrings always use \cd{:local} file system case conventions.

  Finally, \cdf{merge-pathnames} and \cdf{translate-pathname} map customary case in the
  input pathnames into customary case in the output pathname.

  Examples of possible use of this convention:
\begin{itemize}
  \item TOPS-20 is case-sensitive and prefers uppercase,
  translating lowercase to uppercase unless escaped with \cd{{\Xcircumflex}V};
  for a TOPS-20--based
  file system, a Common Lisp implementation
  should use identical
  representations for common and local.

\item UNIX is case-sensitive and prefers lowercase; for a UNIX-based file system,
  a Common Lisp implementation should translate between
  common and local representations by inverting the case of non-mixed-case strings.

\item VAX/VMS is uppercase-only (that is, the file system translates all file
  name arguments to uppercase); for a VAX/VMS-based file system,
  a Common Lisp implementation should
  translate common representation to local by
  converting to uppercase and should translate local representation
  to common with no change.

\item The Macintosh operating system is case-insensitive and prefers lowercase,
  but remembers the cases of letters actually used to name a file;
  for a Macintosh-based file system, a Common Lisp implementation should translate
  between common and local representations by inverting the case of non-mixed-case strings
  and should ignore case when determining whether two pathnames are \cdf{equal}.
\end{itemize}

\newpage%required

Here are some examples of this behavior.  Assume that the host \cdf{T} runs
TOPS-20, \cdf{U} runs UNIX, \cdf{V} runs VAX/VMS, and \cdf{M} runs the Macintosh
operating system.
\begin{lisp}
;;; Returns two values: the PATHNAME-NAME from a namestring \\*
;;; in :COMMON and :LOCAL representations (in that order). \\*
(defun pathname-example (name) \\*
~~(let ((path (parse-namestring name)))) \\*
~~~~(values (pathname-name path :case :common) \\*
~~~~~~~~~~~~(pathname-name path :case :local)))) \\
\\
~~~~~~~~~~~~~~~~~~~~~~~~~~~~~~~~~~~~~~~~~~~\EV\ "FOO" \textrm{and} \="FOO" \kill
~~~~~~~~~~~~~~~~~~~~~~~~~~~~~~~~~~~~~~~~~~~~~~;\textrm{Common}\>\textrm{Local} \\*
(pathname-example "T:<ME>FOO.LISP")~~~~~~~~\EV\ "FOO" \textrm{and} "FOO" \\*
(pathname-example "T:<ME>foo.LISP")~~~~~~~~\EV\ "FOO" \textrm{and} "FOO" \\*
(pathname-example "T:<ME>{\Xcircumflex}Vf{\Xcircumflex}Vo{\Xcircumflex}Vo.LISP")~~\EV\ "foo" \textrm{and} "foo" \\
(pathname-example "T:<ME>TeX.LISP")~~~~~~~~\EV\ "TEX" \textrm{and} "TEX" \\
(pathname-example "T:<ME>T{\Xcircumflex}VeX.LISP")~~~~~~\EV\ "TeX" \textrm{and} "TeX" \\
(pathname-example "U:/me/FOO.lisp")~~~~~~~~\EV\ "foo" \textrm{and} "FOO" \\
(pathname-example "U:/me/foo.lisp")~~~~~~~~\EV\ "FOO" \textrm{and} "foo" \\
(pathname-example "U:/me/TeX.lisp")~~~~~~~~\EV\ "TeX" \textrm{and} "TeX" \\
(pathname-example "V:[me]FOO.LISP")~~~~~~~~\EV\ "FOO" \textrm{and} "FOO" \\
(pathname-example "V:[me]foo.LISP")~~~~~~~~\EV\ "FOO" \textrm{and} "FOO" \\
(pathname-example "V:[me]TeX.LISP")~~~~~~~~\EV\ "TEX" \textrm{and} "TEX" \\
(pathname-example "M:FOO.LISP")~~~~~~~~~~~~\EV\ "foo" \textrm{and} "FOO" \\*
(pathname-example "M:foo.LISP")~~~~~~~~~~~~\EV\ "FOO" \textrm{and} "foo" \\*
(pathname-example "M:TeX.LISP")~~~~~~~~~~~~\EV\ "TeX" \textrm{and} "TeX"
\end{lisp}
The following example illustrates the creation of new pathnames.
The name is converted from common representation to local because
namestrings always use local conventions.
\begin{lisp}
(defun make-pathname-example (h n) \\*
~~(namestring (make-pathname :host h :name n :case :common)) \\
\\
(make-pathname-example "T" "FOO") \EV\ "T:FOO" \\*
(make-pathname-example "T" "foo") \EV\ "T:{\Xcircumflex}Vf{\Xcircumflex}Vo{\Xcircumflex}Vo" \\
(make-pathname-example "T" "TeX") \EV\ "T:T{\Xcircumflex}VeX" \\
(make-pathname-example "U" "FOO") \EV\ "U:foo" \\
(make-pathname-example "U" "foo") \EV\ "U:FOO" \\
(make-pathname-example "U" "TeX") \EV\ "U:TeX" \\
(make-pathname-example "V" "FOO") \EV\ "V:FOO" \\
(make-pathname-example "V" "foo") \EV\ "V:FOO" \\
(make-pathname-example "V" "TeX") \EV\ "V:TeX" \\
(make-pathname-example "M" "FOO") \EV\ "M:foo" \\
(make-pathname-example "M" "foo") \EV\ "M:FOO" \\*
(make-pathname-example "M" "TeX") \EV\ "M:TeX"
\end{lisp}
A big advantage of this set of conventions is that one can, for example,
call \cdf{make-pathname} with \cd{:type~"LISP"} and \cd{:case~:common},
and the result will appear in a namestring as \cd{.LISP} or \cd{.lisp},
whichever is appropriate.

\subsection{Structured Directories}
\label{STRUCTURED-DIRECTORY-SECTION}

X3J13 voted in June 1989 \issue{PATHNAME-SUBDIRECTORY-LIST}
to define a specific pathname component format for structured directories.

The value of a pathname's directory component may be a list.  The
  \emph{car} of the list should be a keyword, either \cd{:absolute} or \cd{:relative}.
  Each remaining element of the list should be a string or a symbol (see below).
  Each string names a single level of directory structure and should consist
  of only the directory name without any punctuation characters.

  A list whose \emph{car} is the symbol \cd{:absolute} represents a directory path
  starting from the root directory.  For example, the list \cd{(:absolute)} represents
  the root directory itself;  the list \cd{(:absolute "foo" "bar" "baz")} represents
  the directory that in a UNIX file system would be called \cd{/foo/bar/baz}.

  A list whose \emph{car} is the symbol \cd{:relative} represents a directory path
  starting from a default directory.  The list \cd{(:relative)} has the same
  meaning as \cdf{nil} and hence normally is not used.  The list \cd{(:relative "foo" "bar")}
  represents the directory named \cdf{bar} in the directory named \cdf{foo} in the
  default directory.

  In place of a string, at any point in the list, a symbol may occur to
  indicate a special file notation. The following symbols have standard
  meanings.
\begin{indentdesc}{7pc}
\item[\cd{:wild}]            Wildcard match of one level of directory structure
\item[\cd{:wild-inferiors}]  Wildcard match of any number of directory levels
\item[\cd{:up}]              Go upward in directory structure (semantic)
\item[\cd{:back}]            Go upward in directory structure (syntactic)
\end{indentdesc}
  (See section~\ref{WILD-PATHNAME-SECTION} for a discussion of wildcard pathnames.)

  Implementations are permitted to add additional objects of any
  non-string type if necessary to represent features of their file systems
  that cannot be represented with the standard strings and symbols.
  Supplying any non-string, including any of the symbols listed below, to a
  file system for which it does not make sense signals an error of type
  \cdf{file-error}.  For example, most implementations of the UNIX file system
  do not support \cd{:wild-inferiors}.  Any directory list in which
  \cd{:absolute} or \cd{:wild-inferiors} is immediately followed by \cd{:up} or \cd{:back}
  is illegal and when  processed causes an error to be signaled.

  The keyword \cd{:back} has a ``syntactic'' meaning that depends only on the pathname
  and not on the contents of the file system.  The keyword \cd{:up} has a ``semantic''
  meaning that depends on the contents of the file system; to resolve
  a pathname containing \cd{:up} to a pathname whose directory component
  contains only \cd{:absolute} and strings requires a search of the file system.
  Note that use of \cd{:up} instead of \cd{:back} can result in designating a different
  actual directory only in file systems that support multiple
  names for directories, perhaps via symbolic links.  For example,
  suppose that there is a directory link such that
\begin{lisp}
(:absolute "X" "Y")~~\textrm{is linked to}~~(:absolute "A" "B")
\end{lisp}
and there also exist directories
\begin{lisp}
(:absolute "A" "Q")~~\textrm{and}~~(:absolute "X" "Q")
\end{lisp}
Then
\begin{lisp}
(:absolute "X" "Y" :up "Q")~~\textrm{designates}~~(:absolute "A" "Q")
\end{lisp}
but
\begin{lisp}
(:absolute "X" "Y" :back "Q")~~\textrm{designates}~~(:absolute "X" "Q")
\end{lisp}

  If a string is used as the value of the \cd{:directory} argument to
  \cdf{make-pathname}, it should be the name of a top-level directory and should
  not contain any punctuation characters.  Specifying a string \emph{s} is
  equivalent to specifying the list \cd{(:absolute \emph{s\/})}.  Specifying the symbol
  \cd{:wild} is equivalent to specifying the list \cd{(:absolute :wild-inferiors)}
  (or \cd{(:absolute :wild)} in a file system that does not support \cd{:wild-inferiors}).

  The function \cdf{pathname-directory}  always returns \cdf{nil}, \cd{:unspecific}, or a
  list---never a string, never \cd{:wild}.
  If a list is returned, it is not guaranteed to be freshly consed; the
  consequences of modifying this list are undefined.
 
  In non-hierarchical file systems, the only valid list values for the
  directory component of a pathname are \cd{(:absolute \emph{s})} (where \emph{s}
  is a string) and
  \cd{(:absolute :wild)}.  The keywords \cd{:relative},
  \cd{:wild-inferiors}, \cd{:up}, and \cd{:back} are not used in non-hierarchical file
  systems.

  Pathname merging treats a relative directory specially.  Let
  \emph{pathname\/} and \emph{defaults\/} be the first two arguments to
  \cdf{merge-pathnames}.  If \cd{(pathname-directory \emph{pathname\/})} is a list whose
  \emph{car} is \cd{:relative}, and \cd{(pathname-directory \emph{defaults\/})} is a list, then
  the merged directory is the value of
\begin{lisp}
(append (pathname-directory \emph{defaults\/}) \\*
~~~~~~~~(cdr~~~~~;\textrm{Remove \cd{:relative} from the front} \\*
~~~~~~~~~~(pathname-directory \emph{pathname\/})))
\end{lisp}
  except that if the resulting list contains a string or \cd{:wild} immediately
  followed by \cd{:back}, both of them are removed.  This removal of redundant
  occurrences of \cd{:back} is repeated as many times as possible.
  If \cd{(pathname-directory \emph{defaults\/})} is not a list or
  \cd{(pathname-directory \emph{pathname\/})} is not a list whose \emph{car} is \cd{:relative}, the
  merged directory is the value of
\begin{lisp}
(or (pathname-directory \emph{pathname\/}) \\
~~~~(pathname-directory \emph{defaults\/}))
\end{lisp}

  A relative directory in the pathname argument to a function such as
  \cdf{open} is merged with the value of \cd{*default-pathname-defaults*} before the
  file system is accessed.

Here are some examples of the use of structured directories.
Suppose that host \cdf{L} supports a Symbolics Lisp Machine file system,
host \cdf{U} supports a UNIX file system, and
host \cdf{V} supports a VAX/VMS file system.
\begin{lisp}
(pathname-directory (parse-namestring "V:[FOO.BAR]BAZ.LSP")) \\*
~~~\EV\ (:ABSOLUTE "FOO" "BAR")
\end{lisp}
\begin{lisp}
(pathname-directory (parse-namestring "U:/foo/bar/baz.lisp")) \\*
~~~\EV\ (:ABSOLUTE "foo" "bar")
\end{lisp}
\begin{lisp}
(pathname-directory (parse-namestring "U:../baz.lisp")) \\*
~~~\EV\ (:RELATIVE :UP)
\end{lisp}
\begin{lisp}
(pathname-directory (parse-namestring "U:/foo/bar/../mum/baz")) \\*
~~~\EV\ (:ABSOLUTE "foo" "bar" :UP "mum")
\end{lisp}
\begin{lisp}
(pathname-directory (parse-namestring "U:bar/../../ztesch/zip")) \\*
~~~\EV\ (:RELATIVE "bar" :UP :UP "ztesch")
\end{lisp}
\begin{lisp}
(pathname-directory (parse-namestring "L:>foo>**>bar>baz.lisp")) \\*
~~~\EV\ (:ABSOLUTE "FOO" :WILD-INFERIORS "BAR")
\end{lisp}
\begin{lisp}
(pathname-directory (parse-namestring "L:>foo>*>bar>baz.lisp")) \\*
~~~\EV\ (:ABSOLUTE "FOO" :WILD "BAR")
\end{lisp}

\subsection{Extended Wildcards}
\label{WILD-PATHNAME-SECTION}

  Some file systems provide more complex conventions for wildcards than
  simple component-wise wildcards representable by \cd{:wild}.
For example, the namestring \cd{"F*O"} might mean a normal three-character
name; a three-character name with the middle character wild;
a name with at least two characters, beginning with \cdf{F} and ending with \cdf{O};
or perhaps a wild match spanning multiple directories.  Similarly, the
namestring \cd{">foo>**>bar>"} might imply that the middle directory is
named \cd{"**"}; the middle directory is \cd{:wild};
there are zero or more middle directories that are \cd{:wild};
or perhaps that the middle directory name matches any two-letter name.
Some file systems support even more complex wildcards, such as
  regular expressions.

X3J13 voted in June 1989 \issue{PATHNAME-WILD} to provide
some facilities for dealing with more general wildcard pathnames
in a fairly portable manner.

\begin{defun}[Function]
wild-pathname-p pathname &optional field-key
  
    Tests a pathname for the presence of wildcard components.  If the first
    argument is not a pathname, string, or file stream, an error of type
    \cdf{type-error} is signaled.
  
    If no \emph{field-key} is provided, or the \emph{field-key} is \cdf{nil}, the result is
    true if and only if \emph{pathname} has any wildcard components.

    If a non-null \emph{field-key} is provided, it must be one of \cd{:host}, \cd{:device},
    \cd{:directory}, \cd{:name}, \cd{:type}, or \cd{:version}.
    In this case, the result is true if and only
    if the indicated component of \emph{pathname} is a wildcard.

    Note that X3J13 voted in June 1989 
    \issue{PATHNAME-COMPONENT-VALUE}
    to specify that an implementation need not support wildcards in all fields;
    the only requirement is that the name, type, or version may be \cd{:wild}.
    However, portable programs should be prepared to encounter either \cd{:wild}
    or implementation-dependent wildcards in any pathname component.
    The function \cdf{wild-pathname-p} provides a portable way for testing
    the presence of wildcards.
\end{defun}

\begin{defun}[Function]
pathname-match-p pathname wildname

    This predicate is true if and only if the \emph{pathname}
    matches the \emph{wildname}.  The matching rules
    are implementation-defined but should be consistent with the behavior of the
    \cdf{directory} function.  Missing components of \emph{wildname} default to \cd{:wild}.

    If either argument is not a pathname, string, or file stream, an error
    of type \cdf{type-error} is signaled.  It is valid for \emph{pathname} to be a
    wild pathname; a wildcard field in \emph{pathname} will match only a
    wildcard field in \emph{wildname}; that is, \cdf{pathname-match-p} is not commutative.
    It is valid for \emph{wildname} to be a non-wild pathname;
    I believe that in this case \cdf{pathname-match-p} will have the same
    behavior as \cdf{equal}, though the X3J13 specification did not say so.
\end{defun}

\begin{defun}[Function]
translate-pathname source from-wildname to-wildname &key

    Translates the pathname \emph{source}, which must match \emph{from-wildname}, into
    a corresponding pathname (call it \emph{result}), which is constructed
    so as to match \emph{to-wildname}, and returns \emph{result}.

    The pathname \emph{result} is a copy of
    \emph{to-wildname} with each missing or wildcard
    field replaced by a portion of \emph{source}; for this purpose a wildcard field is a
    pathname component with a value of \cd{:wild}, a \cd{:wild} element of a
    list-valued directory component, or an implementation-defined portion
    of a component, such as the \cdf{*} in the complex wildcard string
    \cd{"foo*bar"} that some implementations support.  An implementation that
    adds other wildcard features, such as regular expressions, must define
    how \cdf{translate-pathname} extends to those features.  A missing field is
    a pathname component that is \cdf{nil}.

    The portion of \emph{source} that is copied into \emph{result} is
    implementation-defined.  Typically it is determined by the user interface conventions
    of the file systems involved.  Usually it is the portion of \emph{source}
    that matches a wildcard field of \emph{from-wildname} that is in the same
    position as the missing or wildcard field of \emph{to-wildname}.  If there
    is no wildcard field in \emph{from-wildname} at that position, then usually
    it is the entire corresponding pathname component of \emph{source} or, in
    the case of a list-valued directory component, the entire corresponding
    list element.  For example, if the name components of \emph{source},
    \emph{from-wildname}, and \emph{to-wildname} are \cd{"gazonk"}, \cd{"gaz*"}, and \cd{"h*"}
    respectively, then in most file systems the wildcard fields of the
    name component of \emph{from-wildname} and \emph{to-wildname} are each \cd{"*"}, the
    matching portion of \emph{source} is \cd{"onk"}, and the name component of
    \emph{result} is \cd{"honk"}; however, the exact behavior of \cdf{translate-pathname}
    is not dictated by the Common Lisp language and may
    vary according to the user interface conventions of the file systems
    involved.

    During the copying of a portion of \emph{source} into \emph{result}, additional
    implementation-defined translations of alphabetic case or file naming
    conventions may occur, especially when \emph{from-wildname} and
    \emph{to-wildname} are for different hosts.

    If any of the first three arguments is not a pathname, string, or file
    stream, an error of type \cdf{type-error} is signaled.  It is valid for
    \emph{source} to be a wild pathname; in general this will produce a wild
    \emph{result} pathname.  It is valid for \emph{from-wildname} or \emph{to-wildname} or both
    to be non-wild.  An error is signaled if the \emph{source} pathname does not
    match the \emph{from-wildname}, that is,
    if \cd{(pathname-match-p \emph{source} \emph{from-wildname})} would not be true.
    
    There are no specified keyword arguments for \cdf{translate-pathname}, but
    implementations are permitted to extend it by adding keyword arguments.
    There is one specified return value from \cdf{translate-pathname};
    implementations are permitted to extend it by returning additional
    values.

    Here is an implementation suggestion.   One file system performs this operation by
    examining corresponding pieces of the three pathnames in turn, where a piece is a
    pathname component or a list element of a structured component such as
    a hierarchical directory.  Hierarchical directory elements in
    \emph{from-wildname} and \emph{to-wildname} are matched by whether they are
    wildcards, not by depth in the directory hierarchy.  If the piece in
    \emph{to-wildname} is present and not wild, it is copied into the result.
    If the piece in \emph{to-wildname} is \cd{:wild} or \cdf{nil}, the corresponding
    piece in \emph{source} is
    copied into the result.  Otherwise, the piece in \emph{to-wildname} might be
    a complex wildcard such as \cd{"foo*bar"}; the portion of the piece in \emph{source}
    that matches the
    wildcard portion of the corresponding piece in \emph{from-wildname} (or the entire
    \emph{source} piece, if the \emph{from-wildname} piece is not wild and therefore
    equals the \emph{source} piece) replaces the wildcard
    portion of the piece in \emph{to-wildname} and the value produced is used in
    the result.

X3J13 voted in June 1989 \issue{PATHNAME-COMPONENT-CASE} to
require \cdf{translate-pathname} to map customary case in argument
pathnames to the customary case in returned pathnames
(see section~\ref{PATHNAME-COMPONENT-CASE-SECTION}).

Here are some examples of the use of the new wildcard pathname facilities.
These examples are not portable.  They are written to run
with particular file systems and particular wildcard conventions and are
intended to be illustrative, not prescriptive.
Other implementations may behave differently.
\begin{lisp}
(wild-pathname-p (make-pathname :name :wild)) \EV\ t \\*
(wild-pathname-p (make-pathname :name :wild) :name) \EV\ t \\
(wild-pathname-p (make-pathname :name :wild) :type) \EV\ nil \\
(wild-pathname-p (pathname "S:>foo>**>")) \EV\ t~~~~~~~~~;\textrm{Maybe} \\*
(wild-pathname-p (make-pathname :name "F*O")) \EV\ t~~~~~;\textrm{Probably}
\end{lisp}
One cannot rely on \cdf{rename-file} to handle wild pathnames in a predictable
manner.  However, one can use \cdf{translate-pathname} explicitly to control
the process.
\begin{lisp}
(defun rename-files (from to) \\*
~~"Rename all files that match the first argument by \\*
~~~translating their names to the form of the second \\*
~~~argument.  Both arguments may be wild pathnames." \\
~~(dolist (file (directory from)) \\*
~~~~;; DIRECTORY produces only pathnames that match from-wildname. \\*
~~~~(rename-file file (translate-pathname file from to))))
\end{lisp}
\end{defun}

Assuming one particular set of popular wildcard conventions,
this function might exhibit the following behavior.
Not all file systems will run this example exactly as written.
\begin{lisp}
(rename-files "/usr/me/*.lisp" "/dev/her/*.l") \\*
~~~\textrm{renames}\=~~/usr/me/init.lisp \\
\textrm{to}~~/dev/her/init.l \\
\\
(rename-files "/usr/me/pcl*/*" "/sys/pcl/*/") \\*
~~~\textrm{renames}~~/usr/me/pcl-5-may/low.lisp \\*
\textrm{to}~~/sys/pcl/pcl-5-may/low.lisp \\*
~~~\textrm{(in some file systems the result might be}~/sys/pcl/5-may/low.lisp\textrm{)} \\
\\
(rename-files "/usr/me/pcl*/*" "/sys/library/*/") \\*
~~~\textrm{renames}~~/usr/me/pcl-5-may/low.lisp \\*
\textrm{to}~~/sys/library/pcl-5-may/low.lisp \\*
~~~\textrm{(in some file systems the result might be}~/sys/library/5-may/low.lisp\textrm{)} \\
\\
(rename-files "/usr/me/foo.bar" "/usr/me2/") \\*
~~~\textrm{renames}~~/usr/me/foo.bar \\
\textrm{to}~~/usr/me2/foo.bar \\
\\
(rename-files "/usr/joe/*-recipes.text" \\*
~~~~~~~~~~~~~~"/usr/jim/personal/cookbook/joe's-*-rec.text") \\*
~~~\textrm{renames}~~/usr/joe/lamb-recipes.text \\*
\textrm{to}~~/usr/jim/personal/cookbook/joe's-lamb-rec.text~~~~ \\
~~~\textrm{renames}~~/usr/joe/veg-recipes.text \\*
\textrm{to}~~/usr/jim/personal/cookbook/joe's-veg-rec.text~~~~~ \\
~~~\textrm{renames}~~/usr/joe/cajun-recipes.text \\*
\textrm{to}~~/usr/jim/personal/cookbook/joe's-cajun-rec.text~~~ \\
~~~\textrm{renames}~~/usr/joe/szechuan-recipes.text \\*
\textrm{to}~~/usr/jim/personal/cookbook/joe's-szechuan-rec.text
\end{lisp}

The following examples use UNIX syntax and the wildcard conventions of one
particular version of UNIX.
\begin{lisp}
(namestring \\*
~~(translate-pathname "/usr/dmr/hacks/frob.l" \\*
~~~~~~~~~~~~~~~~~~~~~~"/usr/d*/hacks/*.l" \\*
~~~~~~~~~~~~~~~~~~~~~~"/usr/d*/backup/hacks/backup-*.*")) \\*
~~~\EV\ "/usr/dmr/backup/hacks/backup-frob.l"
\end{lisp}
\goodbreak
\begin{lisp}
(namestring \\*
~~(translate-pathname "/usr/dmr/hacks/frob.l" \\*
~~~~~~~~~~~~~~~~~~~~~~"/usr/d*/hacks/fr*.l" \\*
~~~~~~~~~~~~~~~~~~~~~~"/usr/d*/backup/hacks/backup-*.*")) \\*
~~~\EV\ "/usr/dmr/backup/hacks/backup-ob.l"
\end{lisp}
The following examples are similar to the preceding examples
but use two different hosts; host \cdf{U} supports a UNIX file system
and host \cdf{V} supports a VAX/VMS file system.  Note the translation
of file type (from \cdf{l} to \cdf{LSP}) and the change of alphabetic case conventions.
\begin{lisp}
(namestring \\*
~~(translate-pathname "U:/usr/dmr/hacks/frob.l" \\*
~~~~~~~~~~~~~~~~~~~~~~"U:/usr/d*/hacks/*.l" \\*
~~~~~~~~~~~~~~~~~~~~~~"V:SYS\$DISK:[D*.BACKUP.HACKS]BACKUP-*.*")) \\*
~~~\EV\ "V:SYS\$DISK:[DMR.BACKUP.HACKS]BACKUP-FROB.LSP"
\end{lisp}
\begin{lisp}
(namestring \\*
~~(translate-pathname "U:/usr/dmr/hacks/frob.l" \\*
~~~~~~~~~~~~~~~~~~~~~~"U:/usr/d*/hacks/fr*.l" \\*
~~~~~~~~~~~~~~~~~~~~~~"V:SYS\$DISK:[D*.BACKUP.HACKS]BACKUP-*.*")) \\*
~~~\EV\ "V:SYS\$DISK:[DMR.BACKUP.HACKS]BACKUP-OB.LSP"
\end{lisp}
The next example is a version of the
function \cdf{translate-logical-pathname} (simplified a bit) for a logical host named \cdf{FOO}.
The points of interest are the use of \cdf{pathname-match-p} as
a \cd{:test} argument for \cdf{assoc} and the use of \cdf{translate-pathname}
as a substrate for \cdf{translate-logical-pathname}.
\begin{lisp}
(define-condition logical-translation-error (file-error)) \\
\\
(defun my-translate-logical-pathname (pathname \&key rules) \\*
~~(let ((rule (assoc pathname rules :test \#'pathname-match-p))) \\
~~~~(unless rule \\*
~~~~~~(error 'logical-translation-error :pathname pathname)) \\*
~~~~(translate-pathname pathname (first rule) (second rule)))) \\
\\
(my-translate-logical-pathname \\*
~~"FOO:CODE;BASIC.LISP" \\*
~~:rules '(("FOO:DOCUMENTATION;"~"U:/doc/foo/") \\*
~~~~~~~~~~~("FOO:CODE;"~~~~~~~~~~"U:/lib/foo/") \\*
~~~~~~~~~~~("FOO:PATCHES;*;"~~~~~"U:/lib/foo/patch/*/"))) \\*
~~~\EV\ \#P"U:/lib/foo/basic.l"
\end{lisp}

\newpage%required

\subsection{Logical Pathnames}
\label{LOGICAL-PATHNAMES-SECTION}

  Pathname values are not portable, but sometimes they must be mentioned in a
  program (for example, the names of files containing the program and the
  data used by the program).

X3J13 voted in June 1989 \issue{PATHNAME-LOGICAL} to provide
some facilities for portable pathname values.  The idea is to provide
a portable framework for pathname values; these logical pathnames
are then mapped to physical (that is, actual) pathnames by a set of implementation-dependent
or site-dependent rules.  The logical pathname facility therefore
separates the concerns of program writing and user software architecture
from the details of how a software system is embedded in a particular
file system or operating environment.

  Pathname values are not portable because not all Common Lisp
  implementations use the same operating system and file name syntax varies
  widely among operating systems.  In addition, corresponding files at two
  different sites may have different names even when the operating system
  is the same; for example, they may be on different directories or
  different devices.  The Common Lisp logical pathname system defines
  a particular pathname structure and namestring syntax that must be supported
  by all implementations.

\begin{defun}[Class]
logical-pathname

This is a subclass of \cdf{pathname}.
\end{defun}

\subsubsection{Syntax of Logical Pathname Namestrings}

The syntax of a logical pathname namestring is as follows:
\begin{tabbing}
\emph{logical-namestring\/} ::= \Mopt{host\/ \cd{:}} \Mopt{\cd{;}} \Mstar{directory\/ \cd{;}}
      \Mopt{name} \Mopt{\cd{.} type\/ \Mopt{\cd{.} version}}
\end{tabbing}
Note that a logical namestring has no \emph{device} portion.

\begin{tabbing}
\emph{host\/} ::= \emph{word\/} \\*
\emph{directory\/} ::= \emph{word\/} {\Mor} \emph{wildcard-word\/} {\Mor} \emph{wildcard-inferiors\/} \\
\emph{name\/} ::= \emph{word\/} {\Mor} \emph{wildcard-word\/} \\
\emph{type\/} ::= \emph{word\/} {\Mor} \emph{wildcard-word\/} \\
\emph{version\/} ::= \emph{word\/} {\Mor} \emph{wildcard-word\/} \\
\emph{word\/} ::= \Mplus{letter\/ {\Mor} digit\/ {\Mor} \cdf{-}} \\
\emph{wildcard-word\/} ::= \Mopt{word} \cdf{*} \Mstar{word\/ \cdf{*}} \Mopt{word} \\*
\emph{wildcard-inferiors\/} ::= \cdf{**}
\end{tabbing}

  A \emph{word} consists of one or more uppercase letters, digits, and hyphens.
  
  A \emph{wildcard word} consists of one or more asterisks, uppercase letters,
  digits, and hyphens, including at least one asterisk, with no two
  asterisks adjacent.
  Each asterisk matches a sequence of zero or more
  characters.  The wildcard word \cdf{*} parses as \cd{:wild}; all others parse
  as strings.

  Lowercase letters may also appear in a word or wildcard word
  occurring in a namestring.  Such letters are converted to uppercase
  when the namestring is converted to a pathname.
  The consequences of using other characters are unspecified.

  The \emph{host} is a word that has been defined as a logical pathname host by
  using \cdf{setf} with the function \cdf{logical-pathname-translations}.

  There is no device, so the device component of a logical pathname is
  always \cd{:unspecific}.  No other component of a logical pathname can be \cd{:unspecific}.

  Each \emph{directory} is a word, a wildcard word, or \cdf{**} (which is parsed as \cd{:wild-inferiors}).
  If a semicolon precedes the directories, the directory component is
  relative; otherwise it is absolute.

  The \emph{name} is a word or a wildcard word.

  The \emph{type} is a word or a wildcard word.

  The \emph{version} is a positive decimal integer or the word \cdf{NEWEST} (which is parsed
  as \cd{:newest}) or \cdf{*} (which is parsed as \cd{:wild}).
  The letters in \cdf{NEWEST} can be in either alphabetic case.

  The consequences of using any value not specified here as a logical
  pathname component are unspecified.
  The null string \cd{""} is not a valid value for any component of a logical pathname,
  since the null string is not a word or a wildcard word.

\subsubsection{Parsing of Logical Pathname Namestrings}

  Logical pathname namestrings are recognized by the functions \cdf{logical-pathname}
  and \cdf{translate-logical-pathname}.  The host portion
  of the logical pathname namestring and its following colon must appear in
  the namestring arguments to these functions.

  The function \cdf{parse-namestring} recognizes a logical pathname
  namestring when the \emph{host} argument is logical or the \emph{defaults} argument is
  a logical pathname.  In this case the host portion of the logical
  pathname namestring and its following colon are optional.  If the host
  portion of the namestring and the \emph{host} argument are both present and do
  not match, an error is signaled.
  The host argument is logical if it is supplied and came from
  \cdf{pathname-host} of a logical pathname.  Whether a host argument is logical
  if it is a string \cdf{equal} to a logical pathname host name is
  implementation-defined.

  The function \cdf{merge-pathnames} recognizes a logical pathname namestring
  when the \emph{defaults} argument is a logical pathname.  In this case the host
  portion of the logical pathname namestring and its following colon are
  optional.

  Whether the other functions that coerce strings to pathnames
  recognize logical pathname namestrings is implementation-defined.
  These functions include \cdf{parse-namestring} in circumstances other than those described above,
  \cd{merge-\discretionary{}{}{}pathnames} in circumstances other than those described above,
  the \cd{:defaults} argument to \cdf{make-pathname}, and the following functions:
  \begin{flushleft}
  \begin{tabular*}{\linewidth}{@{\extracolsep{\fill}}lll@{}}
  \cdf{compile-file} & \cdf{file-write-date} & \cdf{pathname-name} \\
  \cdf{compile-file-pathname} & \cdf{host-namestring} & \cdf{pathname-type} \\
  \cdf{delete-file} & \cdf{load} & \cdf{pathname-version} \\
  \cdf{directory} & \cdf{namestring} & \cdf{probe-file} \\
  \cdf{directory-namestring} & \cdf{open} & \cdf{rename-file} \\
  \cdf{dribble} & \cdf{pathname} & \cdf{translate-pathname} \\
  \cdf{ed} & \cdf{pathname-device} & \cdf{truename} \\
  \cdf{enough-namestring} & \cdf{pathname-directory} & \cdf{wild-pathname-p} \\
  \cdf{file-author} & \cdf{pathname-host} & \cdf{with-open-file} \\
  \cdf{file-namestring} & \cdf{pathname-match-p} &
  \end{tabular*}
  \end{flushleft}
  Note that many of these functions must accept logical pathnames even though
  they do not accept logical pathname namestrings.
  

\subsubsection{Using Logical Pathnames}

  Some real file systems do not have versions.  Logical pathname
  translation to such a file system ignores the version.  This implies that
  a portable program cannot rely on being able to store in a file system
  more than one version of a
  file named by a logical pathname.

  The type of a logical pathname for a Common Lisp source file is \cdf{LISP}.
  This should be translated into whatever implementation-defined
  type is appropriate in a physical pathname.

  The logical pathname host name \cdf{SYS} is reserved for the implementation.
  The existence and meaning of logical pathnames for logical host \cdf{SYS} is
  implementation-defined.

  File manipulation functions must operate with logical pathnames
according to the following requirements:
\begin{itemize}
\item  The following accept logical pathnames
  and translate them into physical pathnames as if by calling the
  function \cdf{translate-logical-pathname}:
  \begin{flushleft}
  \begin{tabular*}{\linewidth}{@{}l@{\extracolsep{\fill}}ll@{}}
  \cdf{compile-file} & \cdf{ed} & \cdf{probe-file} \\
  \cdf{compile-file-pathname} & \cdf{file-author} & \cdf{rename-file} \\
  \cdf{delete-file} & \cdf{file-write-date} & \cdf{truename} \\
  \cdf{directory} & \cdf{load} & \cdf{with-open-file} \\
  \cdf{dribble} & \cdf{open} & 
  \end{tabular*}
  \end{flushleft}

\item Applying the function \cdf{pathname} to a stream created by the function \cdf{open}
  or the macro \cdf{with-open-file} using a logical pathname produces a logical pathname.

\item The functions \cdf{truename}, \cdf{probe-file}, and \cdf{directory}
      never return logical pathnames.

\item Calling \cdf{rename-file} with a logical pathname as the second argument returns a
  logical pathname as the first value.

\item \cdf{make-pathname} returns a logical pathname if and only if the host is
  logical.  If the \cd{:host} argument to \cdf{make-pathname} is supplied, the host is
  logical if it came from the \cdf{pathname-host} of a logical pathname.  Whether a
  \cd{:host} argument is logical if it is a string equal to a logical pathname
  host name is implementation-defined.
\end{itemize}

\begin{defun}[Function]
logical-pathname pathname

    Converts the argument to a logical pathname and returns it.  The
    argument can be a logical pathname, a logical pathname namestring
    containing a host component, or a stream for which the \cdf{pathname}
    function returns a logical pathname.  For any other argument,
    \cdf{logical-pathname} signals an error of type \cdf{type-error}.
\end{defun}

\begin{defun}[Function]
translate-logical-pathname pathname &key

    Translates a logical pathname to the corresponding physical pathname.
    The \emph{pathname} argument is first coerced to a pathname.  If it is not a
    pathname, string, or file stream, an error of type \cdf{type-error} is
    signaled.

    If the coerced argument is a physical pathname, it is returned.

    If the coerced argument is a logical pathname, the first matching
    translation (according to \cdf{pathname-match-p}) of the logical pathname
    host is applied, as if by calling \cdf{translate-pathname}.  If the result is
    a logical pathname, this process is repeated.  When the result is
    finally a physical pathname, it is returned.

    If no translation matches a logical pathname,
    an error of type \cdf{file-error} is signaled.

    \cdf{translate-logical-pathname} may perform additional translations,
    typically to provide translation of file types to local naming
    conventions, to accommodate physical file systems with names of limited length,
    or to deal with special character requirements such as
    translating hyphens to underscores or uppercase letters to lowercase.
    Any such additional translations are implementation-defined.  Some
    implementations do no additional translations.

    There are no specified keyword arguments for
    \cdf{translate-logical-pathname} but implementations are permitted to extend
    it by adding keyword arguments.  There is one specified return value
    from \cdf{translate-logical-pathname}; implementations are permitted to
    extend it by returning additional values.
\end{defun}

\begin{defun}[Function]
logical-pathname-translations host

    If the specified \emph{host} is not the host component of a logical pathname and is not a
    string that has been defined as a logical pathname host name by \cdf{setf} of
    \cdf{logical-pathname-translations}, this function signals an error of type \cdf{type-error};
    otherwise, it returns the list of translations for the specified \emph{host}.  Each translation is
    a list of at least two elements, from-wildname and to-wildname.  Any
    additional elements are implementation-defined.  A from-wildname is a
    logical pathname whose host is the specified \emph{host}.  A to-wildname is any pathname.
    Translations are searched in the order listed, so more specific
    from-wildnames must precede more general ones.

    \cd{(setf (logical-pathname-translations \emph{host\/}) \emph{translations\/})}
    sets the list of translations for the logical
    pathname \emph{host} to \emph{translations}.  If \emph{host} is a string that has
    not previously been used as logical pathname host, a new logical
    pathname host is defined; otherwise an existing host's translations are
    replaced.  Logical pathname host names are compared with \cdf{string-equal}.

    When setting the translations list, each from-wildname can be a logical
    pathname whose host is \emph{host} or a logical pathname namestring \emph{s}
    parseable by \cd{(parse-namestring \emph{s} \emph{host-object})}, where \emph{host-object}
    is an appropriate object for representing the specified \emph{host} to
    \cdf{parse-namestring}.  (This circuitous specification dodges the fact
    that \cdf{parse-namestring} does not necessarily accept as its second argument
    any old string that names a logical host.)
    Each to-wildname can be anything coercible to a pathname by application of
    the function \cdf{pathname}.
    If to-wildname coerces to a logical pathname,
    \cdf{translate-logical-pathname} will retranslate the result, repeatedly if
    necessary.

    Implementations may define additional functions that operate on
    logical pathname hosts (for example, to specify additional translation
    rules or options).
\end{defun}

\begin{defun}[Function]
load-logical-pathname-translations host

    If a logical pathname host named \emph{host} (a string) is already defined,
    this function returns \cdf{nil}.  Otherwise, it searches for a logical pathname host definition
    in an implementation-defined manner.  If none is found, it signals an
    error.  If a definition is found, it installs the definition and returns \cdf{t}.

    The search used by \cdf{load-logical-pathname-translations} should be
    documented, as logical pathname definitions will be created by users as well as
    by Lisp implementors.  A typical search technique is to
    look in an implementation-defined directory for a file whose name is derived from
    the host name in an implementation-defined fashion.
\end{defun}

\begin{defun}[Function]
compile-file-pathname pathname &key :output-file           

    Returns the pathname that \cdf{compile-file} would write into, if given the
    same arguments.  If the pathname argument is a logical pathname and the
    \cd{:output-file} argument is unspecified, the result is a logical pathname.
    If an implementation supports additional keyword arguments to
    \cdf{compile-file}, \cdf{compile-file-pathname} must accept the same arguments.
\end{defun}

\subsubsection{Examples of the Use of Logical Pathnames}

  Here is a very simple example of setting up a logical pathname host named \cdf{FOO}.
  Suppose that no
  translations are necessary to get around file system restrictions, so
  all that is necessary is to specify the root of the physical directory
  tree that contains the logical file system.
 The namestring syntax in the to-wildname is implementation-specific.
\begin{lisp}
(setf (logical-pathname-translations "foo") \\*
~~~~~~'(("**;*.*.*"~~~~~~~~~~"MY-LISPM:>library>foo>**>")))
\end{lisp}
The following is a sample use of that logical pathname.  All return values
are of course implementation-specific; all of the examples in this section
are of course meant to be illustrative and not prescriptive.
\begin{lisp}
(translate-logical-pathname "foo:bar;baz;mum.quux.3") \\*
~~~\EV\ \#P"MY-LISPM:>library>foo>bar>baz>mum.quux.3"
\end{lisp}

  Next we have a more complex example, dividing the files among two file servers
  (\cdf{U}, supporting a UNIX file system, and \cdf{V}, supporting a VAX/VMS file system)
  and several different directories.  This UNIX file system doesn't support
  \cd{:wild-inferiors} in the directory, so each directory level must
  be translated individually.  No file name or type translations
  are required except for \cd{.MAIL} to \cd{.MBX}.
  The namestring syntax used for the to-wildnames is implementation-specific.
\begin{lisp}
(setf (logical-pathname-translations "prog") \\*
~~~~~~'(("RELEASED;*.*.*"~~~~"U:/sys/bin/my-prog/") \\*
~~~~~~~~("RELEASED;*;*.*.*"~~"U:/sys/bin/my-prog/*/") \\
~~~~~~~~("EXPERIMENTAL;*.*.*" \\*
~~~~~~~~~~~~~~~~~~~~~~~~~~~~~"U:/usr/Joe/development/prog/") \\
~~~~~~~~("EXPERIMENTAL;DOCUMENTATION;*.*.*" \\*
~~~~~~~~~~~~~~~~~~~~~~~~~~~~~"V:SYS\$DISK:[JOE.DOC]") \\
~~~~~~~~("EXPERIMENTAL;*;*.*.*" \\*
~~~~~~~~~~~~~~~~~~~~~~~~~~~~~"U:/usr/Joe/development/prog/*/") \\*
~~~~~~~~("MAIL;**;*.MAIL"~~~~"V:SYS\$DISK:[JOE.MAIL.PROG...]*.MBX") \\*
~~~~~~~~))
\end{lisp}
  Here are sample uses of logical host \cdf{PROG}.  All return values
  are of course implementation-specific.
\begin{lisp}
(translate-logical-pathname "prog:mail;save;ideas.mail.3") \\*
~~~\EV\ \#P"V:SYS\$DISK:[JOE.MAIL.PROG.SAVE]IDEAS.MBX.3" \\
\\
(translate-logical-pathname "prog:experimental;spreadsheet.c") \\*
~~~\EV\ \#P"U:/usr/Joe/development/prog/spreadsheet.c"
\end{lisp}

  Suppose now that we have a program that uses three files logically named \cd{MAIN.LISP},
  \cd{AUXILIARY.LISP}, and \cd{DOCUMENTATION.LISP}.  The following translations might be
  provided by a software supplier as examples.

For a UNIX file system with long file names:
\begin{lisp}
(setf (logical-pathname-translations "prog") \\*
~~~~~~'(("CODE;*.*.*"~~~~~~~~"/lib/prog/"))) \\
\\
(translate-logical-pathname "prog:code;documentation.lisp") \\*
~~~\EV\ \#P"/lib/prog/documentation.lisp"
\end{lisp}
For a UNIX file system with 14-character file names, using \cd{.lisp} as the type:
\begin{lisp}
(setf (logical-pathname-translations "prog") \\*
~~~~~~'(("CODE;DOCUMENTATION.*.*" "/lib/prog/docum.*") \\*
~~~~~~~~("CODE;*.*.*"~~~~~~~~~~~~~"/lib/prog/"))) \\
\\
(translate-logical-pathname "prog:code;documentation.lisp") \\*
~~~\EV\ \#P"/lib/prog/docum.lisp"
\end{lisp}
For a UNIX file system with 14-character file names, using \cd{.l} as the type
(the second translation shortens the compiled file type to \cd{.b}):
\begin{lisp}
(setf (logical-pathname-translations "prog") \\*
~~~~~~{\Xbq}(("**;*.LISP.*"~~~~~~,(logical-pathname "PROG:**;*.L.*")) \\*
~~~~~~~~(,(compile-file-pathname \\*
~~~~~~~~~~~~(logical-pathname "PROG:**;*.LISP.*")) \\*
~~~~~~~~~~~~~~~~~~~~~~~~~~~~,(logical-pathname "PROG:**;*.B.*")) \\
~~~~~~~~("CODE;DOCUMENTATION.*.*" "/lib/prog/documentatio.*") \\*
~~~~~~~~("CODE;*.*.*"~~~~~~~~~~~~~"/lib/prog/"))) \\
\\
(translate-logical-pathname "prog:code;documentation.lisp") \\*
~~~\EV\ \#P"/lib/prog/documentatio.l"
\end{lisp}

\subsubsection{Discussion of Logical Pathnames}

  Large programs can be moved between sites without changing any
  pathnames, provided all pathnames used are logical.  A portable system
  construction tool can be created that operates on programs defined as
  sets of files named by logical pathnames.

   Logical pathname syntax was chosen to be easily translated into the formats of most
  popular file systems, while still being powerful enough for storing large
  programs.  Although they have hierarchical directories, extended wildcard
  matching, versions, and no limit on the length of names, logical pathnames can be
  mapped onto a less capable real file system by translating each
  directory that is used into a flat directory name, processing wildcards in
  the Lisp implementation rather than in the file system, treating all versions as \cd{:newest},
  and using translations to shorten long names.

  Logical pathname words are restricted to non-case-sensitive letters,
  digits, and hyphens to avoid creating problems with real file systems
  that support limited character sets for file naming.
  (If logical pathnames were
  case-sensitive, it would be very difficult to map them into a
  file system that is not sensitive to case in its file names.)

  It is not a goal of logical pathnames to be able to represent all
  possible file names.  Their goal is rather to represent just enough file
  names to be useful for storing software.  Real pathnames, in contrast,
  need to provide a uniform interface to all possible file names, including
  names and naming conventions that are not under the control of Common
  Lisp.

  The choice of logical pathname syntax, using colon, semicolon, and
  period, was guided by the goals of being visually distinct from real file
  systems and minimizing the use of special characters.

  The \cdf{logical-pathname} function is separate from the \cdf{pathname} function
  so that the syntax of logical pathname namestrings does not constrain the
  syntax of physical pathname namestrings in any way.  Logical pathname
  syntax must be defined by Common Lisp so that logical pathnames can be
  conveniently exchanged between implementations, but physical pathname
  syntax is dictated by the operating environments.

\vskip 0pt plus 0.3pt

  The \cdf{compile-file-pathname} function and the specification of \cdf{LISP}
  as the type of a logical pathname for a Common Lisp source file together
  provide enough information about compilation to make possible a portable system
  construction tool.  Suppose that it is desirable
  to call \cdf{compile-file} only if the source file is newer than the compiled
  file.  For this to succeed, it must be possible to know the name of the
  compiled file without actually calling \cdf{compile-file}.
  In some implementations the compiler produces one of several file types,
  depending on a variety of implementation-dependent circumstances,
  so it is not sufficient simply to prescribe a standard logical file type
  for compiled files;
  \cdf{compile-file-pathname} provides access to the defaulting that is performed
  by \cdf{compile-file} ``in a manner
  appropriate to the implementation's file system conventions.''

\vskip 0pt plus 0.3pt

  The use of the logical pathname host name \cdf{SYS} for the implementation
  is current practice.  Standardizing on this name helps users choose
  logical pathname host names that avoid conflicting with
  implementation-defined names.

\vskip 0pt plus 0.3pt

  Loading of logical pathname translations from a site-dependent file
  allows software to be distributed using logical pathnames.  The assumed
  model of software distribution is a division of labor between the
  supplier of the software and the user installing it.  The supplier
  chooses logical pathnames to name all the files used or created by the
  software, and supplies examples of logical pathname translations for a
  few popular file systems.  Each example uses an assumed directory and/or
  device name, assumes local file naming conventions, and provides
  translations that will translate all the logical pathnames used or
  generated by the particular software into valid physical pathnames.
  For a powerful file system these translations can be quite simple.  For
  a more restricted file system, it may be necessary to list an explicit
  translation for every logical pathname used (for example, when dealing
  with restrictions on the maximum length of a file name).

\vskip 0pt plus 0.3pt

  The user installing the software decides on which device and directory
  to store the files and edits the example logical pathname translations
  accordingly.  If necessary, the user also adjusts the translations for
  local file naming conventions and any other special aspects of the user's
  local file system policy and local Common Lisp implementation.  For
  example, the files might be divided among several file server hosts to
  share the load.  The process of defining site-customized logical pathname
  translations is quite easy for a user of a popular file system for which
  the software supplier has provided an example.  A user of a more unusual
  file system might have to take more time; the supplier can help by
  providing a list of all the logical pathnames used or generated by the
  software.\strut

  Once the user has created and executed
  a suitable \cdf{setf} form for setting the \cdf{logical-pathname-translations}
  of the relevant logical host, the software can be loaded and run.  It
  may be necessary to use the translations again, or on another workstation
  at the same site, so it is best to save the \cdf{setf} form in the standard
  place where it can be found later by \cdf{load-logical-pathname-translations}.
  Often a software supplier will include a program for restoring software
  from the distribution medium to the file system and a program for loading
  the software from the file system into a Common Lisp; these programs
  will start by calling \cdf{load-logical-pathname-translations} to make sure that
  the logical pathname host is defined.

  Note that the \cdf{setf} of \cdf{logical-pathname-translations} form isn't part of
  the program; it is separate and is written by the user, not by the
  software supplier.  That separation and a uniform convention for
  doing the separation are the key aspects of logical pathnames.  For small
  programs involving only a handful of files, it doesn't matter much.  The
  real benefits come with large programs with hundreds or thousands of
  files and more complicated situations such as program-generated file
  names or porting a program developed on a system with long file names
  onto a system with a very restrictive limit on the length of file names.

\subsection{Pathname Functions}
\label{PATHNAME-FUNCTIONS}

These functions are what programs use to parse and default file names
that have been typed in or otherwise supplied by the user.

\begin{obsolete}
Any argument called \emph{pathname} in this book may actually be a pathname,
a string or symbol, or a stream.  Any argument called \emph{defaults} may
likewise be a pathname, a string or symbol, or a stream.
\end{obsolete}

\begin{new}
X3J13 voted in March 1988
\issue{PATHNAME-SYMBOL}
to change the language so that a symbol is
\emph{never} allowed as a pathname argument.  More specifically, the
following functions are changed to disallow a symbol as a \emph{pathname}
argument:
\begin{flushleft}
\begin{tabular*}{\textwidth}{@{\extracolsep{\fill}}lll@{}}
\cdf{pathname} & \cdf{pathname-device} & \cdf{namestring} \\
\cdf{truename} & \cdf{pathname-directory} & \cdf{file-namestring} \\
\cdf{parse-namestring} & \cdf{pathname-name} & \cdf{directory-namestring} \\
\cdf{merge-pathnames} & \cdf{pathname-type} & \cdf{host-namestring} \\
\cdf{pathname-host} & \cdf{pathname-version} & \cdf{enough-namestring}
\end{tabular*}
\end{flushleft}
(The function \cdf{require} was
also changed by this vote but was
deleted from the language by a vote in January 1989
\issue{REQUIRE-PATHNAME-DEFAULTS}.)
Furthermore, the vote reaffirmed that the following functions
do not accept symbols as \emph{file}, \emph{filename}, or \emph{pathname} arguments:
\begin{flushleft}
\begin{tabular*}{\textwidth}{@{\extracolsep{\fill}}lll@{}}
\cdf{open} & \cd{rename-file~~~~~~~} & \cd{file-write-date~~~~~} \\
\cd{with-open-file~} & \cdf{delete-file} & \cdf{file-author} \\
\cdf{load} & \cdf{probe-file} & \cdf{directory} \\
\cdf{compile-file}
\end{tabular*}
\end{flushleft}
In older implementations of Lisp that did not have strings, for example
MacLisp, symbols were the only means for specifying pathnames.
This was convenient only because the file systems of the time allowed
only uppercase letters in file names.  Typing \cd{(load 'foo)} caused
the function \cdf{load} to receive the symbol \cdf{FOO} (with uppercase
letters because of the way symbols are parsed) and therefore to
load the file named \cdf{FOO}.
Now that many file systems, most notably {UNIX}, support
case-sensitive file names, the use of symbols is less convenient
and more error-prone.
\end{new}

\begin{new}
X3J13 voted in March 1988
\issue{PATHNAME-STREAM}
to specify that a stream may be used
as a \cdf{pathname}, \cdf{file}, or \cdf{filename} argument
only if it was created by use of \cdf{open} or \cdf{with-open-file},
or if it is a synonym stream whose symbol is bound to a stream that
may be used as a pathname.

If such a stream is used as a pathname, it is as if the \cdf{pathname} function
were applied to the stream and the resulting pathname used in place of the
stream.  This represents the name used to open the file.
This may be, but is not required to be, the actual name of the file.

It is an error to attempt to obtain a pathname
from a stream created by any of the following:
\begin{flushleft}
\begin{tabular*}{\textwidth}{@{\extracolsep{\fill}}ll@{}}
\cdf{make-two-way-stream} & \cdf{make-string-input-stream} \\
\cdf{make-echo-stream} & \cdf{make-string-output-stream} \\
\cdf{make-broadcast-stream} & \cdf{with-input-from-string} \\
\cdf{make-concatenated-stream} & \cdf{with-output-to-string}
\end{tabular*}
\end{flushleft}
\end{new}

In the examples, it is assumed that the host named \cdf{CMUC} runs
the {TOPS-20} operating system, and therefore uses {TOPS-20}
file system syntax; furthermore, an explicit host name is
indicated by following the host name with a double colon.
Remember, however, that namestring syntax is implementation-dependent,
and this syntax is used here purely for the sake of examples.

\begin{defun}[Function]
pathname pathname

The \cdf{pathname} function converts its argument to be a pathname.
The argument may be a pathname, a string or symbol, or a stream;
the result is always a pathname.

\begin{new}
X3J13 voted in March 1988
not to permit symbols as pathnames
\issue{PATHNAME-SYMBOL} and
to specify exactly which streams may be used as pathnames
\issue{PATHNAME-STREAM}.
\end{new}

\begin{new}
X3J13 voted in January 1989
\issue{CLOSED-STREAM-OPERATIONS}
to specify that \cdf{pathname} is unaffected
by whether its argument, if a stream, is open or closed.
X3J13 further commented that because some implementations cannot
provide the ``true name'' of a file until the file is closed,
in such an implementation \cdf{pathname} might, in principle,
return a different (perhaps more specific) file name after the stream is closed.
However, such behavior is prohibited; \cdf{pathname} must return the
same pathname after a stream is closed as it would have while the stream
was open.  See \cdf{truename}.
\end{new}
\end{defun}

\begin{defun}[Function]
truename pathname

The \cdf{truename} function
endeavors to discover the ``true name'' of the file
associated with the \emph{pathname} within the file system.
If the \emph{pathname} is an open stream already associated with a file
in the file system, that file is used.
The ``true name'' is returned as a pathname.
An error is signaled if an appropriate file cannot be located
within the file system for the given \emph{pathname}.

The \cdf{truename} function may be used to
account for any file name translations performed by the file system,
for example.

For example, suppose that \cd{DOC:} is a {TOPS-20} logical
device name that is translated by the {TOPS-20} file system
to be \cd{PS:<DOCUMENTATION>}.
\begin{lisp}
(setq file (open "CMUC::DOC:DUMPER.HLP")) \\
(namestring (pathname file)) \EV\ "CMUC::DOC:DUMPER.HLP" \\
(namestring (truename file)) \\
~~~\EV\ "CMUC::PS:<DOCUMENTATION>DUMPER.HLP.13"
\end{lisp}

\begin{new}
X3J13 voted in March 1988
not to permit symbols as pathnames
\issue{PATHNAME-SYMBOL} and
to specify exactly which streams may be used as pathnames
\issue{PATHNAME-STREAM}.
\end{new}

\begin{new}
X3J13 voted in January 1989
\issue{CLOSED-STREAM-OPERATIONS}
to specify that \cdf{truename} may be
applied to a stream whether the stream is open or closed.
X3J13 further commented that because some implementations cannot
provide the ``true name'' of a file until the file is closed, in principle
it would be possible in such an implementation for \cdf{truename} to
return a different file name after the stream is closed.
Such behavior is permitted; in this respect \cdf{truename}
differs from \cdf{pathname}.
\end{new}

\begin{newer}
X3J13 voted in June 1989 \issue{PATHNAME-WILD}
to clarify that \cdf{truename} accepts only non-wild pathnames;
an error is signaled if \cdf{wild-pathname-p} would be true of
the \emph{pathname} argument.
\end{newer}

\begin{newer}
X3J13 voted in June 1989 \issue{PATHNAME-LOGICAL} to require \cdf{truename}
to accept logical pathnames (see section~\ref{LOGICAL-PATHNAMES-SECTION}).
However, \cdf{truename} never returns a logical pathname.
\end{newer}
\end{defun}

\begin{defun}[Function]
parse-namestring thing &optional host defaults &key :start :end :junk-allowed

\begin{obsolete}\noindent
This turns \emph{thing} into a pathname.  The \emph{thing} is usually a string
(that is, a namestring), but it may be a symbol (in which case the print
name is used) or a pathname or stream
(in which case no parsing is needed, but
an error check may be made for matching hosts).
\end{obsolete}

\begin{new}
X3J13 voted in March 1988
not to permit symbols as pathnames
\issue{PATHNAME-SYMBOL} and
to specify exactly which streams may be used as pathnames
\issue{PATHNAME-STREAM}.  The \emph{thing} argument may not be a symbol.
\end{new}

\begin{newer}
X3J13 voted in June 1989 \issue{PATHNAME-LOGICAL} to require \cdf{parse-namestring}
to accept logical pathname namestrings (see section~\ref{LOGICAL-PATHNAMES-SECTION}).
\end{newer}

This function does \emph{not}, in general, do defaulting of pathname components,
even though it has an argument named \emph{defaults};
it only does parsing.  The
\emph{host} and \emph{defaults} arguments are present because in some implementations
it may be that a namestring can only be parsed with reference to a
particular file name syntax of several available in the implementation.
If \emph{host} is
non-{\nil}, it must be a host name that could appear in the
host component of a pathname, or {\nil};
if \emph{host} is {\nil} then the host 
name is extracted from the default pathname in \emph{defaults}
and used to determine the syntax convention.  The \emph{defaults} argument
defaults to the value of \cd{*default-pathname-defaults*}.

For a string (or symbol) argument, \cdf{parse-namestring}
parses a file name within it in the
range delimited by the \cd{:start} and \cd{:end} arguments
(which are integer indices
into \emph{string}, defaulting to the beginning and end of the string).

\begin{newer}
See chapter~\ref{KSEQUE} for a discussion of \cd{:start} and \cd{:end} arguments.
\end{newer}

If \cd{:junk-allowed} is not {\false}, then the first value
returned is the pathname parsed, or {\false} if no syntactically correct
pathname was seen.

If \cd{:junk-allowed} is {\false} (the default),
then the entire substring is scanned.
The returned value is the pathname parsed.
An error is signaled if the substring does not consist entirely of
the representation of a pathname, possibly surrounded on either side by
whitespace characters if that is appropriate to the cultural conventions
of the implementation.

In either case, the second value is the index into the string of the delimiter
that terminated the parse, or the index beyond the substring if the
parse terminated at the end of the substring (as will always be the case if
\cd{:junk-allowed} is false).

If \emph{thing} is not a string or symbol, then \emph{start} (which defaults
to zero in any case) is always returned as the second value.

Parsing an empty string always succeeds, producing a pathname with
all components (except the host) equal to {\nil}.

Note that if \emph{host} is specified and not {\nil},
and \emph{thing} contains a manifest host name, an
error is signaled if the hosts do not match.

If \emph{thing} contains an explicit host name and no explicit device name,
then it might be appropriate, depending on the
implementation environment, for \cdf{parse-namestring} to supply the
standard default device for that host as the device component
of the resulting pathname.
\end{defun}


\begin{defun}[Function]
merge-pathnames pathname &optional defaults default-version

\begin{new}
X3J13 voted in March 1988
not to permit symbols as pathnames
\issue{PATHNAME-SYMBOL} and
to specify exactly which streams may be used as pathnames
\issue{PATHNAME-STREAM}.
\end{new}

\begin{newer}
X3J13 voted in June 1989 \issue{PATHNAME-LOGICAL} to require \cdf{merge-namestrings}
to recognize a logical pathname namestring as its first argument
if its second argument is a logical pathname (see section~\ref{LOGICAL-PATHNAMES-SECTION}).
\end{newer}

\begin{new}
X3J13 voted in January 1989
\issue{CLOSED-STREAM-OPERATIONS}
to specify that \cdf{merge-pathname} is unaffected by
whether the first argument, if a stream, is open or closed. If the first
argument is a stream, \cdf{merge-pathname} behaves as if the function
\cdf{pathname} were applied to the stream and the resulting pathname used instead.
\end{new}

\begin{newer}
X3J13 voted in June 1989 \issue{PATHNAME-COMPONENT-CASE} to
require \cdf{merge-pathnames} to map customary case in argument
pathnames to the customary case in returned pathnames
(see section~\ref{PATHNAME-COMPONENT-CASE-SECTION}).
\end{newer}

\emph{defaults} defaults to the value of \cd{*default-pathname-defaults*}.

\emph{default-version} defaults to \cd{:newest}.

Here is an example of the use of \cdf{merge-pathnames}:
\begin{lisp}
(merge-pathnames "CMUC::FORMAT" \\
~~~~~~~~~~~~~~~~~"CMUC::PS:<LISPIO>.FASL") \\
~~~\EV\ \textrm{a pathname object that re-expressed as a namestring would be} \\
~~~~~~"CMUC::PS:<LISPIO>FORMAT.FASL.0"
\end{lisp}

Defaulting of pathname components is done by filling in components taken
from another pathname.
This is especially useful for
cases such as a program that has an input file and an output file, and
asks the user for the name of both, letting the unsupplied components of
one name default from the other.  Unspecified components of the output
pathname will come from the input pathname, except that the type should
default not to the type of the input but to the appropriate default type
for output from this program.

The pathname merging operation takes as input a given pathname, a
defaults pathname, and a default version, and returns a
new pathname.  Basically, the missing components in the given pathname
are filled in from the defaults pathname, except that
if no version is specified the
default version is used.
The default version is usually \cd{:newest}; if no version is specified
the newest version in existence should be used.  The default
version can be {\nil}, to preserve the information that it was missing
in the input pathname.

If the
given pathname explicitly specifies a host and does not supply a device, then
if the host component of the defaults matches the host component
of the given pathname, then the device is taken from the defaults;
otherwise
the device will be the default file device for that host.  Next, if
the
given pathname does not specify a host, device, directory, name,
or type, each such
component is copied from the defaults.
The merging rules for the version are more complicated and
depend on whether the pathname specifies a name.  If the pathname
doesn't specify a name, then the version, if not provided, will
come from the defaults, just like the other components.  However, if the
pathname does specify a name, then the version is not affected
by the defaults.  The reason is that the version
``belongs to'' some other file name and is unlikely to have anything to do
with the new one.  Finally, if this process leaves the
version missing, the default version is used.

The net effect is that if the user supplies just a name, then the
host, device, directory, and type will come from the defaults, but the
version will come from the default version
argument to the merging operation.  If the user supplies nothing, or
just a directory, the name, type, and version will come over from
the defaults together.  If the host's file name syntax provides a way
to input a version without a name or type, the user can let the name
and type
default but supply a version different from the one in the defaults.

\begin{newer}
X3J13 voted in June 1989 \issue{PATHNAME-SYNTAX-ERROR-TIME} to agree to disagree:
\cdf{merge-pathname} might or might not perform plausibility checking
on its arguments to ensure that the resulting pathname can be converted
a valid namestring.  User beware: this could cause portability problems.

For example, suppose that host \cdf{LOSER} constrains file types to be three characters
or fewer but host \cdf{CMUC} does not.  Then \cd{"LOSER::FORMAT"} is a valid
namestring and \cd{"CMUC::PS:<LISPIO>.FASL"} is a valid namestring, but
\begin{lisp}
(merge-pathnames "LOSER::FORMAT" "CMUC::PS:<LISPIO>.FASL")
\end{lisp}
might signal an error in some implementations because the hypothetical result would be a pathname
equivalent to the namestring \cd{"LOSER::FORMAT.FASL"} which is illegal
because the file type \cdf{FASL} has more than three characters.
In other implementations \cdf{merge-pathname} might return a pathname but that pathname might
cause \cdf{namestring} to signal an error.
\end{newer}
\end{defun}

\begin{defun}[Variable]
*default-pathname-defaults*

This is the default pathname-defaults pathname; if any pathname primitive
that needs a set of defaults is not given one, it uses this one.
As a general rule, however, each program
should have its own pathname defaults rather than using this one.
\end{defun}

The following example assumes the use of UNIX syntax and conventions.
\begin{lisp}
(make-pathname :host "technodrome" \\
~~~~~~~~~~~~~~~:directory '(:absolute "usr" "krang") \\
~~~~~~~~~~~~~~~:name "shredder") \\
~~\EV\ \#P"technodrome:/usr/krang/shredder"
\end{lisp}
X3J13 voted in June 1989 \issue{PATHNAME-COMPONENT-CASE} to add a new keyword
argument \cd{:case} to \cdf{make-pathname}.  The new argument description
is therefore as follows:

\begin{defun}[Function]
make-pathname &key :host :device :directory :name :type :version :defaults :case

See section~\ref{PATHNAME-COMPONENT-CASE-SECTION} for a description
of the \cd{:case} argument.

X3J13 voted in June 1989 \issue{PATHNAME-SYNTAX-ERROR-TIME} to agree to disagree:
\cdf{make-pathname} might or might not check
on its arguments to ensure that the resulting pathname can be converted to
a valid namestring.  If \cdf{make-pathname} does not check its arguments
and signal an error in problematical cases,
\cdf{namestring} yet might or might not signal an error when given the resulting
pathname.  User beware: this could cause portability problems.
\end{defun}

\begin{defun}[Function]
pathnamep object

This predicate is true if \emph{object} is a pathname, and otherwise is false.
\begin{lisp}
(pathnamep x) \EQ\ (typep x 'pathname)
\end{lisp}
\end{defun}

\begin{new}
X3J13 voted in March 1988
not to permit symbols as pathnames
\issue{PATHNAME-SYMBOL} and
to specify exactly which streams may be used as pathnames
\issue{PATHNAME-STREAM}.
\end{new}

\begin{new}
X3J13 voted in January 1989
\issue{CLOSED-STREAM-OPERATIONS}
to specify that these operations are unaffected by
whether the first argument, if a stream, is open or closed. If the first
argument is a stream, each operation behaves as if the function \cdf{pathname}
were applied to the stream and the resulting pathname used instead.
\end{new}

X3J13 voted in June 1989 \issue{PATHNAME-COMPONENT-CASE} to add a keyword
argument \cd{:case} to all of the pathname accessor functions except
\cdf{pathname-version}.  The new argument descriptions
are therefore as follows:

\begin{defun}[Function]
pathname-host pathname &key :case \\
pathname-device pathname &key :case \\
pathname-directory pathname &key :case \\
pathname-name pathname &key :case \\
pathname-type pathname &key :case \\
pathname-version pathname

See section~\ref{PATHNAME-COMPONENT-CASE-SECTION} for a description
of the \cd{:case} argument.

X3J13 voted in June 1989 \issue{PATHNAME-SUBDIRECTORY-LIST}
to specify that
  \cdf{pathname-directory}  always returns \cdf{nil}, \cd{:unspecific}, or a
  list---never a string, never \cd{:wild} (see section~\ref{STRUCTURED-DIRECTORY-SECTION}).
  If a list is returned, it is not guaranteed to be freshly consed; the
  consequences of modifying this list are undefined.
\end{defun}

\begin{defun}[Function]
namestring pathname \\
file-namestring pathname \\
directory-namestring pathname \\
host-namestring pathname \\
enough-namestring pathname &optional defaults

The \emph{pathname} argument may be a pathname, a string or symbol,
or a stream that is or was open to a file.
The name represented by \emph{pathname} is returned as a namelist
in canonical form.

If \emph{pathname} is a stream, the name returned represents the
name used to \emph{open} the file, which may not be the \emph{actual}
name of the file (see \cdf{truename}).

\begin{new}
X3J13 voted in March 1988
not to permit symbols as pathnames
\issue{PATHNAME-SYMBOL} and
to specify exactly which streams may be used as pathnames
\issue{PATHNAME-STREAM}.
\end{new}

\begin{new}
X3J13 voted in January 1989
\issue{CLOSED-STREAM-OPERATIONS}
to specify that these operations are unaffected by
whether the first argument, if a stream, is open or closed. If the first
argument is a stream, each operation behaves as if the function \cdf{pathname}
were applied to the stream and the resulting pathname used instead.
\end{new}

\cdf{namestring} returns the full form of the \emph{pathname} as a string.
\cdf{file-namestring} returns a string representing just the \emph{name},
\emph{type}, and \emph{version} components of the \emph{pathname};
the result of \cdf{directory-namestring}
represents just the \emph{directory-name} portion; and \cdf{host-namestring}
returns a string for just the \emph{host-name} portion.
Note that a valid namestring cannot necessarily be constructed
simply by concatenating some of the three shorter strings in some order.

\cdf{enough-namestring} takes another argument, \emph{defaults}.
It returns an abbreviated namestring that is just sufficient to
identify the file named by \emph{pathname} when considered relative
to the \emph{defaults} (which defaults to the value of
\cd{*default-pathname-defaults*}).  That is, it is required
that
\begin{lisp}
(merge-pathnames (enough-namestring \emph{pathname} \emph{defaults}) \emph{defaults}) {\EQ} \\
~(merge-pathnames (parse-namestring \emph{pathname} nil \emph{defaults}) \emph{defaults})
\end{lisp}
in all cases; and the result of \cdf{enough-namestring} is, roughly speaking,
the shortest reasonable string that will still satisfy this criterion.
\begin{newer}
X3J13 voted in June 1989 \issue{PATHNAME-SYNTAX-ERROR-TIME} to agree to disagree:
\cdf{make-pathname} and \cdf{merge-pathnames} might or might not be able to produce pathnames
that cannot be converted to valid namestrings.
User beware: this could cause portability problems.
\end{newer}
\end{defun}

\begin{defun}[Function]
user-homedir-pathname &optional host

Returns a pathname for the user's ``home directory'' on \emph{host}.
The \emph{host} argument
defaults in some appropriate implementation-dependent manner.  The
concept of ``home directory'' is itself somewhat
implementation-dependent, but from the point of view of Common Lisp it is the
directory where the user keeps personal files such as initialization
files and mail.  If it is impossible to determine this information,
then {\nil} is returned instead of a pathname; however,
\cdf{user-homedir-pathname} never returns {\nil} if the \emph{host} argument
is not specified.
This function returns a pathname without any name, type,
or version component (those components are all {\nil}).
\end{defun}


\section{Opening and Closing Files}

When a file is \emph{opened}, a stream object is constructed to serve
as the file system's ambassador to the Lisp environment;
operations on the stream are reflected by operations on the file
in the file system.  The act of \emph{closing} the file (actually,
the stream) ends the association; the transaction with the file
system is terminated, and input/output may no longer be performed
on the stream.  The stream function \cdf{close} may be used
to close a file; the functions described below may be used to open them.
The basic operation is \cdf{open}, but \cdf{with-open-file} is usually
more convenient for most applications.

\begin{defun}[Function]
open filename &key :direction :element-type :if-exists :if-does-not-exist :external-format

\begin{newer}
X3J13 voted in June 1989 \issue{MORE-CHARACTER-PROPOSAL}
to add to the function \cdf{open} a new keyword argument \cd{:external-format}.
This argument did not appear in the preceding argument description in the
first edition.
\end{newer}

This returns a stream that is connected to the file specified by \emph{filename}.
The \emph{filename} is the name of the file to be opened; it may be a string,
a pathname, or a stream.  (If the \emph{filename} is a stream, then it is not
closed first or otherwise affected; it is used merely to provide a file name
for the opening of a new stream.)

\begin{new}
X3J13 voted in January 1989
\issue{STREAM-ACCESS}
to specify that the result of
\cdf{open}, if it is a stream, is always a stream of type \cdf{file-stream}.
\end{new}

\begin{new}
X3J13 voted in March 1988
\issue{PATHNAME-STREAM}
to specify exactly which streams may be used as pathnames.
See section \ref{PATHNAME-FUNCTIONS}.
\end{new}

\begin{new}
X3J13 voted in January 1989
\issue{CLOSED-STREAM-OPERATIONS}
to specify that \cdf{open} is unaffected by
whether the first argument, if a stream, is open or closed. If the first
argument is a stream, \cdf{open} behaves as if the function \cdf{pathname}
were applied to the stream and the resulting pathname used instead.
\end{new}

\begin{newer}
X3J13 voted in June 1989 \issue{PATHNAME-WILD}
to clarify that \cdf{open} accepts only non-wild pathnames;
an error is signaled if \cdf{wild-pathname-p} would be true of \emph{filename}.
\end{newer}

\begin{newer}
X3J13 voted in June 1989 \issue{PATHNAME-LOGICAL} to require \cdf{open}
to accept logical pathnames (see section~\ref{LOGICAL-PATHNAMES-SECTION}).
\end{newer}

The keyword arguments specify what kind of stream to produce and how
to handle errors:
\begin{flushdesc}
\item[\cd{:direction}]
This argument specifies whether the stream should handle input, output,
or both.
\begin{quotation}
\begin{flushdesc}
\item[\cd{:input}]
The result will be an input stream.  This is the default.

\item[\cd{:output}]
The result will be an output stream.

\item[\cd{:io}]
The result will be a bidirectional stream.

\item[\cd{:probe}]
The result will be a no-directional stream (in effect, the stream
is created and then closed).  This is useful for determining whether
a file exists without actually setting up a complete stream.
\end{flushdesc}
\end{quotation}

\item[\cd{:element-type}]
This argument specifies the type of the unit of transaction for the stream.
Anything that can be recognized as being a finite subtype of
\cdf{character} or \cdf{integer} is acceptable.  In particular,
the following types are recognized:

\begin{quotation}
\begin{flushdesc}
\item[\cdf{character}]
The unit of transaction is any character, not just a string-character.
The functions \cdf{read-char} and \cdf{write-char} (depending on the value of the
\cd{:direction} argument) may be used on the stream.  This is the default.

\item[\cdf{base-char}]
The unit of transaction is a base character.
The functions \cdf{read-char} and \cdf{write-char} (depending on the value of the
\cd{:direction} argument) may be used on the stream.
\end{flushdesc}
\end{quotation}

\begin{quotation}
\begin{flushdesc}
\item[\cd{(unsigned-byte \emph{n})}]
The unit of transaction is an unsigned byte (a non-negative integer) of size \emph{n}.
The functions \cdf{read-byte} and/or \cdf{write-byte} may be
used on the stream.

\item[\cdf{unsigned-byte}]
The unit of transaction is an unsigned byte (a non-negative integer);
the size of the byte is determined by the file system.
The functions \cdf{read-byte} and/or \cdf{write-byte} may be
used on the stream.

\item[\cd{(signed-byte \emph{n})}]
The unit of transaction is a signed byte of size \emph{n}.
The functions \cdf{read-byte} and/or \cdf{write-byte} may be
used on the stream.

\item[\cdf{signed-byte}]
The unit of transaction is a signed byte;
the size of the byte is determined by the file system.
The functions \cdf{read-byte} and/or \cdf{write-byte} may be
used on the stream.

\item[\cdf{bit}]
The unit of transaction is a bit (values \cd{0} and \cd{1}).
The functions \cdf{read-byte} and/or \cdf{write-byte} may be
used on the stream.

\item[\cd{(mod \emph{n})}]
The unit of transaction is a non-negative integer less than \emph{n}.
The functions \cdf{read-byte} and/or \cdf{write-byte} may be
used on the stream.

\item[\cd{:default}]
The unit of transaction is to be determined by the file system, based
on the file it finds.
The type can be determined by using the function \cdf{stream-element-type}.
\end{flushdesc}
\end{quotation}

\item[\cd{:if-exists}]
This argument specifies the action to be taken if the \cd{:direction} is
\cd{:output} or \cd{:io} and a file of the specified name already exists.
If the direction is \cd{:input} or \cd{:probe}, this argument is ignored.
\begin{quotation}
\begin{flushdesc}
\item[\cd{:error}]
Signals an error.  This is the default when the version component of
the \emph{filename} is not \cd{:newest}.

\item[\cd{:new-version}]
Creates a new file with the same file name but with a larger version number.
This is the default when the version component of the \emph{filename} is \cd{:newest}.

\item[\cd{:rename}]
Renames the existing file to some other name and then creates a new file
with the specified name.

\item[\cd{:rename-and-delete}]
Renames the existing file to some other name and then deletes it (but
does not expunge it, on those systems that distinguish deletion from
expunging).  Then create a new file with the specified name.

\item[\cd{:overwrite}]
Uses the existing file.  Output operations on the stream
will destructively modify the file.
If the \cd{:direction} is \cd{:io},
the file is opened in a bidirectional mode that allows both
reading and writing.  The file pointer is initially positioned
at the beginning of the file; however, the file is not truncated
back to length zero when it is opened.
This mode is most useful when
the \cdf{file-position} function can be used on the stream.

\item[\cd{:append}]
Uses the existing file.  Output operations on the stream
will destructively modify the file.  The file pointer is
initially positioned at the end of the file.
If the \cd{:direction} is \cd{:io},
the file is opened in a bidirectional mode that allows both
reading and writing.

\item[\cd{:supersede}]
Supersedes the existing file.  If possible, the implementation should
arrange not to destroy the old file until the new stream is closed,
against the possibility that the stream will be closed in ``abort'' mode
(see \cdf{close}).
This differs from \cd{:new-version} in that \cd{:supersede} creates
a new file with the same name as the old one, rather than a file
name with a higher version number.

\item[\cd{\false}]
Does not create a file or even a stream, but instead
simply returns {\false} to indicate failure.
\end{flushdesc}
\end{quotation}

If the \cd{:direction} is \cd{:output} or \cd{:io}
and the value of \cd{:if-exists} is \cd{:new-version},
then the version of the (newly created) file that is opened will
be a version greater than that of any other file in the file system
whose other pathname components are the same as those of \emph{filename}.

If the \cd{:direction} is \cd{:input} or \cd{:probe}
or the value of \cd{:if-exists} is not \cd{:new-version},
\emph{and} the version component of the \emph{filename} is \cd{:newest},
then the file opened is that file already existing in the file system
that has a version greater than that of any other file in the file system
whose other pathname components are the same as those of \emph{filename}.

\begin{new}
Some file systems permit yet other actions to be taken when a file
already exists; therefore,
some implementations provide implementation-specific \cd{:if-exist} options.
\end{new}
\end{flushdesc}

\beforenoterule
\begin{implementation}
The various file systems in existence today
have widely differing capabilities.  A given implementation may not
be able to support all of these options in exactly the manner stated.
An implementation is required to recognize all of these option keywords
and to try to do something ``reasonable'' in the context of the host operating
system.  Implementors are encouraged to approximate the semantics specified
here as closely as possible.

As an example, suppose that a file system does not support distinct file
versions and does not distinguish the notions of deletion and expunging
(in some file systems file deletion is reversible until an expunge operation
is performed).  Then \cd{:new-version} might be treated the same as
\cd{:rename} or \cd{:supersede}, and \cd{:rename-and-delete} might
be treated the same as \cd{:supersede}.

If it is utterly impossible for an implementation to handle some option
in a manner close to what is specified here, it may simply signal an error.
The opening of files is an area where complete portability is too much to
hope for; the intent here is simply to make things as portable as possible
by providing specific names for a range of commonly supportable options.
\end{implementation}
\afternoterule

\begin{flushdesc}
\item[\cd{:if-does-not-exist}]
This argument specifies the action to be taken if
a file of the specified name does not already exist.\vadjust{\vskip2pt}
\begin{quotation}
\begin{flushdesc}
\item[\cd{:error}]
Signals an error.  This is the default if the \cd{:direction} is \cd{:input},
or if the \cd{:if-exists} argument is \cd{:overwrite} or \cd{:append}.

\item[\cd{:create}]
Creates an empty file with the specified name and then proceeds as if it
had already existed (but do not perform any processing directed by the
\cd{:if-exists} argument).
This is the default if the \cd{:direction} is \cd{:output}
or \cd{:io}, and the \cd{:if-exists} argument is anything but \cd{:overwrite}
or \cd{:append}.

\item[\cd{\false}]
Does not create a file or even a stream, but
instead simply returns {\false} to indicate failure.
This is the default if the \cd{:direction} is \cd{:probe}.
\end{flushdesc}
\end{quotation}
\end{flushdesc}

\begin{newer}
X3J13 voted in June 1989 \issue{MORE-CHARACTER-PROPOSAL}
to add to the function \cdf{open} a new keyword argument \cd{:external-format}.
\begin{flushdesc}
\item[\cd{:external-format}]
This argument specifies an implementation-recognized scheme for
representing characters in files.  The default value is \cd{:default}
and is implementation-defined but must support the base characters.
An error is signaled if the implementation does recognize the specified format.

This argument may be specified if the \cd{:direction} argument is
\cd{:input}, \cd{:output}, or \cd{:io}.  It is an error to write a character
to the resulting stream that cannot be represented by the specified file format.
(However, the \cd{\#{\Xbackslash}Newline} character cannot produce such an error;
implementations must provide appropriate line division behavior for all character
streams.)

See \cdf{stream-external-format}.
\end{flushdesc}
\end{newer}

When the caller is finished with the stream, it should close the file by
using the \cdf{close} function.  The \cdf{with-open-file}
form does this automatically, and so is preferred for most purposes.
\cdf{open} should be used only when the control structure of the program
necessitates opening and closing of a file in some way more complex than
provided by \cdf{with-open-file}.  It is suggested that any program that uses
\cdf{open} directly should use the special operator \cdf{unwind-protect} to
close the file if an abnormal exit occurs.
\end{defun}

\begin{defmac}
with-open-file (stream filename {options}*)
               {declaration}* {form}*

\cdf{with-open-file}
evaluates the \emph{forms} of the body (an implicit \cdf{progn}) with the variable
\emph{stream} bound to a stream that reads or writes the file named by the
value of \emph{filename}.
The \emph{options} are evaluated and are used as keyword arguments to
the function \cdf{open}.

When control leaves the body, either normally or abnormally (such as by
use of \cdf{throw}), the file is automatically closed.  If a new
output file is being written, and control leaves abnormally, the file is
aborted and the file system is left, so far as possible, as if the file
had never been opened.  Because \cdf{with-open-file} always closes the
file, even when an error exit is taken, it is preferred over \cdf{open} for
most applications.

\emph{filename} is the name of the file to be opened; it may be a string,
a pathname, or a stream.

\begin{new}
X3J13 voted in March 1988
\issue{PATHNAME-STREAM}
to specify exactly which streams may be used as pathnames.
See section \ref{PATHNAME-FUNCTIONS}.
\end{new}

\begin{newer}
X3J13 voted in June 1989 \issue{PATHNAME-WILD}
to clarify that \cdf{with-open-file} accepts only non-wild pathnames;
an error is signaled if \cdf{wild-pathname-p} would be true of
the \emph{filename} argument.
\end{newer}

\begin{newer}
X3J13 voted in June 1989 \issue{PATHNAME-LOGICAL} to require \cdf{with-open-file}
to accept logical pathnames (see section~\ref{LOGICAL-PATHNAMES-SECTION}).
\end{newer}


For example:
\begin{lisp}
(with-open-file (ifile name \\
~~~~~~~~~~~~~~~~~:direction :input) \\
~~(with-open-file (ofile (merge-pathname-defaults ifile \\
~~~~~~~~~~~~~~~~~~~~~~~~~~~~~~~~~~~~~~~~~~~~~~~~~~nil \\
~~~~~~~~~~~~~~~~~~~~~~~~~~~~~~~~~~~~~~~~~~~~~~~~~~"out") \\
~~~~~~~~~~~~~~~~~~~~~~~~~:direction :output \\
~~~~~~~~~~~~~~~~~~~~~~~~~:if-exists :supersede) \\
~~~~(transduce-file ifile ofile)))
\end{lisp}

\begin{newer}
X3J13 voted in June 1989 \issue{WITH-OPEN-FILE-DOES-NOT-EXIST}
to specify that the variable \emph{stream} is not always bound to
a stream; rather it is bound to whatever would be returned by
a call to \cdf{open}.  For example, if the options include
\cd{:if-does-not-exist~nil}, \emph{stream} will be bound to \cdf{nil}
if the file does not exist.  In this case the value of \emph{stream}
should be tested within the body of the \cdf{with-open-file} form
before it is used as a stream.  For example:
\begin{lisp}
(with-open-file (ifile name \\*
~~~~~~~~~~~~~~~~~:direction :input \\*
~~~~~~~~~~~~~~~~~:if-does-not-exist nil) \\*
~~;; Process the file only if it actually exists. \\*
~~(when (streamp name)\\*
~~~~(compile-cobol-program ifile)))
\end{lisp}
\end{newer}
\end{defmac}

\beforenoterule
\begin{implementation}
While \cdf{with-open-file} tries to automatically close
the stream on exit from the construct, for robustness it is helpful
if the garbage collector can detect discarded streams and automatically
close them.
\end{implementation}
\afternoterule

\section{Renaming, Deleting, and Other File Operations}

These functions provide a standard interface to operations provided
in some form by most file systems.  It may be that some implementations
of Common Lisp cannot support them all completely.

\begin{defun}[Function]
rename-file file new-name

The specified \emph{file} is renamed to \emph{new-name} (which must be a file name).
The \emph{file} may be a string, a pathname, or a stream.  If it is an open stream
associated with a file, then the stream itself and the file associated
with it are affected (if the file system permits).

\begin{new}
X3J13 voted in March 1988
\issue{PATHNAME-STREAM}
to specify exactly which streams may be used as pathnames.
See section \ref{PATHNAME-FUNCTIONS}.
\end{new}

\cdf{rename-file} returns three values if successful.  The first value
is the \emph{new-name} with any missing components filled in by performing
a \cdf{merge-pathnames} operation using \emph{file} as the defaults.
The second value is the \cdf{truename} of the file before it was renamed.
The third value is the \cdf{truename} of the file after it was renamed.

If the renaming operation is not successful, an error is signaled.

\begin{newer}
X3J13 voted in June 1989 \issue{PATHNAME-LOGICAL} to require \cdf{rename-file}
to accept logical pathnames (see section~\ref{LOGICAL-PATHNAMES-SECTION}).
\end{newer}
\end{defun}

\begin{defun}[Function]
delete-file file

The specified \emph{file} is deleted.  The \emph{file} may be a string, a
pathname, or a stream.  If it is an open stream associated with a file,
then the stream itself and the file associated with it are affected (if
the file system permits), in which case the stream may or may not be
closed immediately, and the deletion may be immediate or delayed until
the stream is explicitly closed, depending on the requirements of the
file system.

\begin{new}
X3J13 voted in March 1988
\issue{PATHNAME-STREAM}
to specify exactly which streams may be used as pathnames.
See section \ref{PATHNAME-FUNCTIONS}.
\end{new}

\cdf{delete-file} returns a non-{\nil} value if successful.
It is left to the discretion of the implementation whether an attempt
to delete a non-existent file is considered to be successful.
If the deleting operation is not successful, an error is signaled.

\begin{newer}
X3J13 voted in June 1989 \issue{PATHNAME-LOGICAL} to require \cdf{delete-file}
to accept logical pathnames (see section~\ref{LOGICAL-PATHNAMES-SECTION}).
\end{newer}
\end{defun}

\begin{defun}[Function]
probe-file file

This predicate is false if there is no file named \emph{file},
and otherwise returns a pathname that is the true name of the file
(which may be different from \emph{file} because of file links, version
numbers, or other artifacts of the file system).
Note that if the \emph{file} is an open stream associated with a file,
then \cdf{probe-file} cannot return {\nil} but will produce the
true name of the associated file.
See \cdf{truename} and the \cd{:probe} value for the
\cd{:direction} argument to \cdf{open}.

\begin{new}
X3J13 voted in March 1988
\issue{PATHNAME-STREAM}
to specify exactly which streams may be used as pathnames.
See section \ref{PATHNAME-FUNCTIONS}.
\end{new}

\begin{newer}
X3J13 voted in June 1989 \issue{PATHNAME-WILD}
to clarify that \cdf{probe-file} accepts only non-wild pathnames;
an error is signaled if \cdf{wild-pathname-p} would be true of
the \emph{file} argument.
\end{newer}

\begin{newer}
X3J13 voted in June 1989 \issue{PATHNAME-LOGICAL} to require \cdf{probe-file}
to accept logical pathnames (see section~\ref{LOGICAL-PATHNAMES-SECTION}).
However, \cdf{probe-file} never returns a logical pathname.
\end{newer}

\begin{new}
X3J13 voted in January 1989
\issue{CLOSED-STREAM-OPERATIONS}
to specify that \cdf{probe-file} is unaffected by
whether the first argument, if a stream, is open or closed. If the first
argument is a stream, \cdf{probe-file} behaves as if the function \cdf{pathname}
were applied to the stream and the resulting pathname used instead.
However, X3J13 further commented that the treatment of open streams
may differ considerably from one implementation to another; for example,
in some operating systems open files are written under a temporary or
invisible name and later renamed when closed.  In general, programmers writing
code intended to be portable should be very careful when using \cdf{probe-file}.
\end{new}
\end{defun}

\begin{defun}[Function]
file-write-date file

\emph{file} can be a file name or a stream that is open to a file.
This returns the time at which the file was created
or last written as an integer in
universal time format (see section \ref{TIME-SECTION}),
or {\false} if this cannot be determined.

\begin{new}
X3J13 voted in March 1988
\issue{PATHNAME-STREAM}
to specify exactly which streams may be used as pathnames.
See section \ref{PATHNAME-FUNCTIONS}.
\end{new}

\begin{newer}
X3J13 voted in June 1989 \issue{PATHNAME-WILD}
to clarify that \cdf{file-write-date} accepts only non-wild pathnames;
an error is signaled if \cdf{wild-pathname-p} would be true of
the \emph{file} argument.
\end{newer}

\begin{newer}
X3J13 voted in June 1989 \issue{PATHNAME-LOGICAL} to require \cdf{file-write-date}
to accept logical pathnames (see section~\ref{LOGICAL-PATHNAMES-SECTION}).
\end{newer}
\end{defun}

\begin{defun}[Function]
file-author file

\emph{file} can be a file name or a stream that is open to a file.
This returns the name of the author of the file as a string,
or {\false} if this cannot be determined.

\begin{new}
X3J13 voted in March 1988
\issue{PATHNAME-STREAM}
to specify exactly which streams may be used as pathnames.
See section \ref{PATHNAME-FUNCTIONS}.
\end{new}

\begin{newer}
X3J13 voted in June 1989 \issue{PATHNAME-WILD}
to clarify that \cdf{file-author} accepts only non-wild pathnames;
an error is signaled if \cdf{wild-pathname-p} would be true of
the \emph{file} argument.
\end{newer}

\begin{newer}
X3J13 voted in June 1989 \issue{PATHNAME-LOGICAL} to require \cdf{file-author}
to accept logical pathnames (see section~\ref{LOGICAL-PATHNAMES-SECTION}).
\end{newer}
\end{defun}

\begin{defun}[Function]
file-position file-stream &optional position

\cdf{file-position} returns or sets the current position within
a random-access file.

\cd{(file-position \emph{file-stream})} returns a non-negative integer
indicating the current position within the \emph{file-stream}, or {\false} if
this cannot be determined.  The file position at the start of a file will
be zero.  The value returned by \cdf{file-position} increases monotonically
as input or output operations are performed.  For a character file,
performing a single \cdf{read-char} or \cdf{write-char} operation
may cause the file position to be increased by more than 1 because of
character-set translations (such as translating between the Common Lisp
\cd{\#{\Xbackslash}Newline} character and an external {ASCII}
carriage-return/line-feed sequence) and other aspects of the
implementation.  For a binary file, every \cdf{read-byte} or \cdf{write-byte}
operation increases the file position by 1.

\cd{(file-position \emph{file-stream} \emph{position})} sets the position within
\emph{file-stream} to be \emph{position}.  The \emph{position} may be an integer,
or \cd{:start} for the beginning of the stream, or \cd{:end} for the end of the
stream.
If the integer is too large or otherwise inappropriate, an error
is signaled (the \cdf{file-length} function returns the length beyond
which \cdf{file-position} may not access).  An integer returned by
\cdf{file-position} of one argument should, in general, be acceptable
as a second argument for use with the same file.
With two arguments,
\cdf{file-position} returns {\true} if the repositioning was performed
successfully, or {\false} if it was not (for example,
because the file was not random-access).

\begin{implementation}
Implementations that have character files represented
as a sequence of records of bounded size might choose to encode the
file position as, for example,
\emph{record-number}*256+\emph{character-within-record}.
This is a valid encoding because it increases monotonically as
each character is read or written, though not necessarily by 1 at
each step.  An integer might then be considered ``inappropriate''
as a second argument to \cdf{file-position} if, when decoded into
record number and character number, it turned out that the
specified record was too short for the specified character number.
\end{implementation}
\end{defun}

\begin{defun}[Function]
file-length file-stream

\emph{file-stream} must be a stream that is open to a file.
The length of the file is returned as a non-negative integer,
or {\false} if the length cannot be determined.
For a binary file,
the length is specifically
measured in units of the \cd{:element-type} specified
when the file was opened (see \cdf{open}).
\end{defun}


\begin{newer}
\begin{defun}[Function]
file-string-length file-stream object

X3J13 voted in June 1989 \issue{MORE-CHARACTER-PROPOSAL} to add
the function \cdf{file-string-length}.
The \emph{object} must be a string or a character.  The function
\cdf{file-string-length} returns a non-negative integer
that is the difference between what the \cdf{file-position} of the
\emph{file-stream} would be after and before writing the \emph{object}
to the \emph{file-stream}, or \cdf{nil} if this
difference cannot be determined.  The value returned may
depend on the current state of the \emph{file-stream}; that is, calling
\cdf{file-string-length} on the same arguments twice may in certain circumstances
produce two different integers.
\end{defun}
\end{newer}

\section{Loading Files}

To \emph{load} a file is to read through the file, evaluating each form in
it.  Programs are typically stored in files containing calls to
constructs such as \cdf{defun}, \cdf{defmacro},
and \cdf{defvar}, which define
the functions and variables of the program.

Loading a compiled (``fasload'') file is similar, except that the file does not
contain text but rather pre-digested expressions created by the
compiler that can be loaded more quickly.

\begin{defun}[Function]
load filename &key :verbose :print :if-does-not-exist

This function loads the file named by \emph{filename} into the Lisp
environment.  It is assumed that a text (character file) can be
automatically distinguished from an object (binary) file by some appropriate
implementation-dependent means, possibly by the file type.
The defaults for \emph{filename} are taken from the variable
\cd{*default-pathname-defaults*}.
If the \emph{filename} (after the merging in of the defaults)
does not explicitly specify a type,
and both text and object types of the file are available in the file system,
\cdf{load} should
try to select the more appropriate file by some implementation-dependent means.

If the first argument is a stream rather than a pathname,
then \cdf{load} determines what kind of stream it is and loads
directly from the stream.

The \cd{:verbose} argument (which defaults to the value of
\cd{*load-verbose*}), if true, permits \cdf{load} to print a message
in the form of a comment (that is, with a leading
semicolon) to \cdf{*standard-output*} indicating what
file is being loaded and other useful information.

\begin{obsolete}
The \cd{:print} argument (default {\nil}),
if true, causes the value of each expression
loaded to be printed to \cdf{*standard-output*}.  If a binary file is
being loaded, then what is printed may not reflect precisely the contents
of the source file, but nevertheless some information will be printed.
\end{obsolete}
\begin{newer}
X3J13 voted in March 1989 \issue{COMPILER-VERBOSITY}
to add the variable \cd{*load-print*}; its value is used as the default
for the \cd{:print} argument to \cdf{load}.
\end{newer}

\begin{newer}
The function \cdf{load} rebinds \cdf{*package*} to its current value.  If
some form in the file changes the value of \cdf{*package*} during loading,
the old value will be restored when the loading is completed.
(This was specified in the first edition under the description of \cdf{*package*};
for convenience I now mention it here as well.)
\end{newer}

\begin{new}
X3J13 voted in March 1988
\issue{PATHNAME-STREAM}
to specify exactly which streams may be used as pathnames.
See section \ref{PATHNAME-FUNCTIONS}.
\end{new}

\begin{newer}
X3J13 voted in June 1989 \issue{PATHNAME-WILD}
to clarify that supplying a wild pathname
as the \emph{filename} argument to \cdf{load} has implementation-dependent consequences;
\cdf{load} might signal an error, for example,
or might load all files that match the pathname.
\end{newer}

\begin{newer}
X3J13 voted in June 1989 \issue{PATHNAME-LOGICAL} to require \cdf{load}
to accept logical pathnames (see section~\ref{LOGICAL-PATHNAMES-SECTION}).
\end{newer}

If a file is successfully loaded, \cdf{load} always returns a non-{\false}
value.  If \cd{:if-does-not-exist} is specified and is {\false},
\cdf{load} just returns {\false} rather than signaling an error if the file
does not exist.

\begin{newer}
X3J13 voted in March 1989 \issue{IN-SYNTAX}
to require that \cdf{load} bind \cd{*readtable*} to its current value
at the time \cdf{load} is called; the dynamic extent of the binding
should encompass all of the file-loading activity.
This allows a portable program to include forms such as
\begin{lisp}
(in-package "FOO") \\*
\\*
(eval-when (:execute :load-toplevel :compile-toplevel) \\*
~~(setq *readtable* foo:my-readtable))
\end{lisp}
without performing a net global side effect on the loading environment.
Such statements allow the remainder of such a file to be read either as
interpreted code or by \cdf{compile-file} in a syntax determined by
an alternative readtable.
\end{newer}

\begin{newer}
X3J13 voted in June 1989 \issue{LOAD-TRUENAME}
to require that \cdf{load} bind two new variables
\cd{*load-pathname*} and \cd{*load-truename*}; the dynamic extent of the bindings
should encompass all of the file-loading activity.
\end{newer}
\end{defun}

\begin{defun}[Variable]
*load-verbose*

This variable provides the default for the \cd{:verbose} argument
to \cdf{load}.  Its initial value is implementation-dependent.
\end{defun}


\begin{newer}
\begin{defun}[Variable]
*load-print*

X3J13 voted in March 1989 \issue{COMPILER-VERBOSITY}
to add \cd{*load-print*}.
This variable provides the default for the \cd{:print} argument
to \cdf{load}.  Its initial value is \cdf{nil}.
\end{defun}
\end{newer}

\begin{newer}
\begin{defun}[Variable]
*load-pathname*

X3J13 voted in June 1989 \issue{LOAD-TRUENAME} to introduce \cd{*load-pathname*};
it is initially \cdf{nil} but \cdf{load} binds it to a pathname that
represents the file name given as the first argument to \cdf{load} merged
with the defaults (see \cdf{merge-pathname}).
\end{defun}

\begin{defun}[Variable]
*load-truename*

X3J13 voted in June 1989 \issue{LOAD-TRUENAME} to introduce \cd{*load-truename*};
it is initially \cdf{nil} but \cdf{load} binds it to the ``true name'' of
the file being loaded.  See \cdf{truename}.
\end{defun}
\end{newer}

\begin{newer}
X3J13 voted in March 1989 \issue{LOAD-OBJECTS} to introduce a facility
based on the Object System
whereby a user can specify how \cdf{compile-file} and \cdf{load}
must cooperate to reconstruct compile-time constant objects at load time.
The protocol is simply this:
  \cdf{compile-file} calls the generic
  function \cdf{make-load-form} on any object that is referenced as
  a constant or as a self-evaluating form, if the object's metaclass is
  \cdf{standard-class}, \cdf{structure-class}, any user-defined metaclass (not a
  subclass of \cdf{built-in-class}), or any of a possibly empty
  implementation-defined list of other metaclasses; \cdf{compile-file} will
  call \cdf{make-load-form} only once for any given object (as determined by \cdf{eq})
  within a single file.  The user-programmability stems from the possibility
  of user-defined methods for \cdf{make-load-form}.  The helper function
  \cdf{make-load-form-saving-slots} makes it easy to write commonly used
  versions of such methods.

\begin{defun}[Generic function]
make-load-form object

The argument is an object that is
  referenced as a constant or as a self-evaluating form in a file being
  compiled by \cdf{compile-file}.  The objective is to enable \cdf{load} to
  construct an equivalent object.

  The first value, called the \emph{creation form}, is a form that, when
  evaluated at load time, should return an object that is equivalent to
  the argument.  The exact meaning of ``equivalent'' depends on the type
  of object and is up to the programmer who defines a method for
  \cdf{make-load-form}.  This allows the user to program the notion
  of ``similar as a constant'' (see section~\ref{COMPILER-SECTION}).

  The second value, called the \emph{initialization form}, is a form that,
  when evaluated at load time, should perform further initialization of
  the object.  The value returned by the initialization form is ignored.
  If the \cdf{make-load-form} method returns only one value, the
  initialization form is \cdf{nil}, which has no effect.  If the object used
  as the argument to \cdf{make-load-form} appears as a constant in the
  initialization form, at load time it will be replaced by the
  equivalent object constructed by the creation form; this is how the
  further initialization gains access to the object.

  Two values are returned so that circular structures may be handled.
  The order of evaluation rules discussed below
  for creation and initialization forms
  eliminates the possibility of partially initialized objects in the
  absence of circular structures and reduces the possibility to a minimum
  in the presence of circular structures.  This allows nodes in
  non-circular structures to be built out of fully initialized subparts.

  Both the creation form and the initialization form can contain
  references to objects of user-defined types (defined precisely below).
  However, there must not be any circular dependencies in creation forms.
  An example of a circular dependency: the creation form for the
  object \emph{X\/} contains a reference to the object \emph{Y\/}, and the creation form
  for the object \emph{Y\/} contains a reference to the object \emph{X\/}.  A simpler
  example: the creation form for the object \emph{X\/} contains
  a reference to \emph{X\/} itself.  Initialization forms are not subject to
  any restriction against circular dependencies, which is the entire
  reason for having initialization forms.  See the example of circular
  data structures below.

  The creation form for an object is always evaluated before the
  initialization form for that object.  When either the creation form or
  the initialization form refers to other objects of user-defined types
  that have not been referenced earlier in the \cdf{compile-file}, the
  compiler collects all of the creation and initialization forms.  Each
  initialization form is evaluated as soon as possible after its
  creation form, as determined by data flow.  If the initialization form
  for an object does not refer to any other objects of user-defined
  types that have not been referenced earlier in the \cdf{compile-file}, the
  initialization form is evaluated immediately after the creation form.
  If a creation or initialization form \emph{F\/} references other objects of
  user-defined types that have not been referenced earlier in the
  \cdf{compile-file}, the creation forms for those other objects are evaluated
  before \emph{F\/} and the initialization forms for those other objects are
  also evaluated before \emph{F\/} whenever they do not depend on the object
  created or initialized by \emph{F}.  Where the above rules do not uniquely
  determine an order of evaluation, it is unspecified
  which of the possible orders of evaluation is chosen.

  While these creation and initialization forms are being evaluated, the
  objects are possibly in an uninitialized state, analogous to the state
  of an object between the time it has been created by \cdf{allocate-instance}
  and it has been processed fully by \cdf{initialize-instance}.  Programmers
  writing methods for \cdf{make-load-form} must take care in manipulating
  objects not to depend on slots that have not yet been initialized.

  It is unspecified whether \cdf{load} calls \cdf{eval} on the forms or does some
  other operation that has an equivalent effect.  For example, the
  forms might be translated into different but equivalent forms and
  then evaluated; they might be compiled and the resulting functions
  called by \cdf{load} (after they themselves have been loaded);
  or they might be interpreted by a special-purpose
  interpreter different from \cdf{eval}.  All that is required is that the
  effect be equivalent to evaluating the forms.

  It is valid for user programs to call \cdf{make-load-form} in
  circumstances other than compilation, providing the argument's
  metaclass is not \cdf{built-in-class} or a subclass of \cdf{built-in-class}.

  Applying \cdf{make-load-form} to an object whose metaclass is \cdf{standard-class} or
  \cdf{structure-class} for which no user-defined method is applicable signals
  an error.  It is valid to implement this either by defining default
  methods for the classes \cdf{standard-object} and \cdf{structure-object} that signal an error
  or by having no applicable method for those classes.

See \cdf{load-time-eval}.

In the following example, an equivalent instance of \cdf{my-class} is reconstructed
  by using the values of two of its slots.  The value of the third slot
  is derived from those two values.
\begin{lisp}
(defclass my-class ()
~~((a :initarg :a :reader my-a) \\*
~~~(b :initarg :b :reader my-b) \\*
~~~(c :accessor my-c))) \\
\\
(defmethod shared-initialize ((self my-class) slots \&rest inits) \\*
~~(declare (ignore slots inits)) \\*
~~(unless (slot-boundp self 'c) \\*
~~~~(setf (my-c self) \\*
~~~~~~~~~~(some-computation (my-a self) (my-b self))))) \\
\\
(defmethod make-load-form ((self my-class)) \\*
~~{\Xbq}(make-instance ',(class-name (class-of self)) \\*
~~~~~~~~~~~~~~~~~~:a ',(my-a self) :b ',(my-b self)))
\end{lisp}
This code will fail if either of the first two slots of some instance
of \cdf{my-class} contains the instance itself.
Another way to write the last form in the preceding example is
\begin{lisp}
(defmethod make-load-form ((self my-class)) \\*
~~(make-load-form-saving-slots self '(a b)))
\end{lisp}
This has the advantages of conciseness and handling circularities correctly.

In the next example, instances of class \cdf{my-frob} are ``interned'' in some way.
  An equivalent instance is reconstructed by using the value of the
  \cdf{name} slot as a key for searching for existing objects.  In this case
  the programmer has chosen to create a new object if no existing
  object is found; an alternative possibility would be to signal an
  error in that case.
\begin{lisp}
(defclass my-frob () \\*
~~((name :initarg :name :reader my-name))) \\
\\
(defmethod make-load-form ((self my-frob)) \\*
~~{\Xbq}(find-my-frob ',(my-name self) :if-does-not-exist :create))
\end{lisp}

  In the following example, the data structure to be dumped is circular, because
  each node of a tree has a list of its children and each child has a reference
  back to its parent.  
\begin{lisp}
(defclass tree-with-parent () ((parent :accessor tree-parent) \\*
~~~~~~~~~~~~~~~~~~~~~~~~~~~~~~~(children :initarg :children)))
\end{lisp}

\begin{lisp}
(defmethod make-load-form ((x tree-with-parent)) \\*
~~(values \\*
~~~~{\Xbq}(make-instance ',(class-of x) \\*
~~~~~~~~~~~~~~~~~~~~:children ',(slot-value x 'children)) \\*
~~~~{\Xbq}(setf (tree-parent ',x) ',(slot-value x 'parent))))
\end{lisp}
Suppose \cdf{make-load-form} is called on one object in
  such a structure.  The creation form creates an equivalent object and
  fills in the \cdf{children} slot, which forces creation of equivalent
  objects for all of its children, grandchildren, etc.  At this point
  none of the parent slots have been filled in.  The initialization form
  fills in the \cdf{parent} slot, which forces creation of an equivalent
  object for the parent if it was not already created.  Thus the entire
  tree is recreated at load time.  At compile time, \cdf{make-load-form} is
  called once for each object in the tree.  All the creation forms
  are evaluated, in unspecified order, and then all  the
  initialization forms are evaluated, also in unspecified order.

  In this final example, the data structure to be dumped has no special
  properties and an equivalent structure can be reconstructed
  simply by reconstructing the slots' contents.
\begin{lisp}
(defstruct my-struct a b c) \\[2pt]
(defmethod make-load-form ((s my-struct)) \\*
~~(make-load-form-saving-slots s))
\end{lisp}
This is easy to code using \cdf{make-load-form-saving-slots}.
\end{defun}

\begin{defun}[Function]
make-load-form-saving-slots object &optional slots

This returns two values suitable for return from a \cdf{make-load-form} method.
  The first argument is the object.  The optional second argument is a
  list of the names of slots to preserve; it defaults to all of the
  local slots.

\cdf{make-load-form-saving-slots} returns forms that construct
  an equivalent object using \cdf{make-instance}
  and \cdf{setf} of \cdf{slot-value} for
  slots with values, or \cdf{slot-makunbound} for slots without values, or
  other functions of equivalent effect.

  Because \cdf{make-load-form-saving-slots} returns two values, it can deal with
  circular structures; it works for any object
  of metaclass \cdf{standard-class} or \cdf{structure-class}.
  Whether the result is
  useful depends on whether the object's type and slot
  contents fully capture an application's idea of the object's state.
\end{defun}
\end{newer}


\section {Accessing Directories}

The following function is a very simple portable primitive for examining
a directory.  Most file systems can support much more powerful
directory-searching primitives, but no two are alike.
It is expected that most implementations of Common Lisp will extend the
\cdf{directory} function or provide more powerful
primitives.

\begin{defun}[Function]
directory pathname &key

A list of pathnames is returned, one for each
file in the file system that matches the given \emph{pathname}.
(The \emph{pathname} argument may be a pathname, a string,
or a stream associated with a file.)
For a file that matches, the \cdf{truename} appears
in the result list.
If no file matches the \emph{pathname}, it is not an error;
\cdf{directory} simply returns {\nil}, the list of no results.
Keywords such as \cd{:wild} and \cd{:newest} may
be used in \emph{pathname} to indicate the search space.

\begin{new}
X3J13 voted in March 1988
\issue{PATHNAME-STREAM}
to specify exactly which streams may be used as pathnames.
See section \ref{PATHNAME-FUNCTIONS}.
\end{new}

\begin{new}
X3J13 voted in January 1989
\issue{CLOSED-STREAM-OPERATIONS}
to specify that \cdf{directory} is unaffected by
whether the first argument, if a stream, is open or closed. If the first
argument is a stream, \cdf{directory} behaves as if the function \cdf{pathname}
were applied to the stream and the resulting pathname used instead.
However, X3J13 commented that the treatment of open streams
may differ considerably from one implementation to another; for example,
in some operating systems open files are written under a temporary or
invisible name and later renamed when closed.  In general, programmers writing
code intended to be portable should be careful when using \cdf{directory}.
\end{new}

\begin{newer}
X3J13 voted in June 1989 \issue{PATHNAME-LOGICAL} to require \cdf{directory}
to accept logical pathnames (see section~\ref{LOGICAL-PATHNAMES-SECTION}).
However, the result returned by \cdf{directory} never contains a logical pathname.
\end{newer}

\beforenoterule
\begin{implementation}
It is anticipated that
an implementation may need to provide additional
parameters to control the directory search.  Therefore \cdf{directory}
is specified to take additional keyword arguments so that implementations
may experiment with extensions,
even though no particular keywords are specified here.

As a simple example of such an extension, for a file system that
supports the notion of cross-directory file links,
a keyword argument \cd{:links} might, if non-{\nil},
specify that such links be included in the result list.
\end{implementation}
\afternoterule
\end{defun}

\begin{defun}[Function]
ensure-directories-exist file &key :verbose

If the directories of file do not exist then this function creates them
returning two values, file and a second value true if the directories were
created or nil if not.
\end{defun}

       % File system interface
%%%Part{XERROR, Root = "CLM.MSS"}
%%%Chapter of Common Lisp Manual.  Copyright 1984, 1988, 1989 Guy L. Steele Jr.

\clearpage\def\pagestatus{FINAL PROOF}

\chapter{Errors}
\label{XERROR}

Errors may be signaled for a variety of reasons.
Many built-in Common Lisp functions may signal an error when given incorrect
arguments.  Other functions, described in this chapter,
may be called by user programs for the purpose of signaling
an error.

When an error is signaled, it is
handled in an implementation-dependent way.  It is expected
that each implementation of Common Lisp will provide an interactive debugger that
prints the error message along with suitable contextual information
such as which function detected the error.  The user may interact with
the debugger to examine or modify the state of the program in various
ways, including abandoning the current computation (``aborting to top
level'') and continuing from the error.  What ``continuing'' means
depends on how the error is signaled; the details of this are specified below
for each error-signaling function.

\begin{obsolete}
An implementation may also choose to provide means (such as the
\cd{errset} special form in MacLisp) for a program to trap
all errors and prevent the debugger from stepping in for
certain errors.

\beforenoterule
\begin{rationale}
Error handling of adequate
flexibility and power for all systems written in Common Lisp appears to
require a complex error classification system.
Experience with several error-handling systems
in such dialects as MacLisp and Lisp Machine Lisp indicates that
further experimentation is needed in this area;
it is too early to define a standard error-handling mechanism.
Therefore Common Lisp provides standard ways to {\it signal} errors,
but no standard ways to {\it handle} errors.
Of course a
complete Lisp system requires error-handling mechanisms, but many useful
portable programs do not require them.  It is expected that a future
revision of Common Lisp will address the problem of portable error-handling
mechanisms.
\end{rationale}
\afternoterule
\end{obsolete}

\begin{newer}
X3J13 voted in June 1988
\issue{CONDITION-SYSTEM}
to adopt a proposal for a Common Lisp Condition System.
This was the result of the research and experimentation
alluded to in the preceding paragraph.
Conditions subsume and generalize the notion of errors.
The condition system also provides means for handling
conditions (of which errors are a special case) and
for restarting a computation after a condition has been signaled.
See chapter~\ref{CONDITION}.
\end{newer}

\beforenoterule
\begin{incompatibility}
What is here called ``continuing,''
Lisp Machine Lisp calls ``proceeding'' from an error.
\begin{new}
In the new terminology introduced in chapter~\ref{CONDITION},
what Lisp Machine Lisp called ``proceeding'' would be called
``restarting,'' and ``continuing'' refers to the particular
restart named \cd{continue}.
\end{new}
\end{incompatibility}
\afternoterule

\section{General Error-Signaling Functions}
\label{ERROR-SIGNALLING-FUNCTIONS}

The functions in this section provide various mechanisms
for signaling warnings, breaks, continuable errors, and fatal errors.

In each case, the caller specifies an error message (a string) that may be
processed (and perhaps displayed to the user) by the error-handling
mechanism.  All messages are
constructed by applying the function
\cd{format} to the quantities {\nil}, {\it format-string},
and all the {\it args} to produce a string.

An error message string should not contain a newline character
at either the beginning or end, and should not contain any sort of
herald indicating that it is an error.  The system will take care of
these according to whatever its preferred style may be.  

Conventionally,
error messages are complete English sentences ending with a period.
Newlines in the middle of long messages are acceptable.  There
should be no indentation after a newline in the middle of an
error message.  The error message need not mention the name of the function
that signals the error; it is assumed that the debugger will make this
information available.

\beforenoterule
\begin{implementation}
If the debugger in a particular implementation
displays error messages indented from the prevailing left margin
(for example, indented by seven spaces because
they are prefixed by the seven-character herald ``\cd{Error: }''),
then the debugger should take care of inserting
the appropriate indentation into a multi-line error message.
Similarly, a debugger that prefixes error messages with semicolons
so that they appear to be comments
should take care of inserting a semicolon at the beginning of each
line in a multi-line error message.  These rules are suggested
because, even within a single
implementation, there may be more than one program that presents error
messages to the user, and they may use different styles of
presentation.  The caller
of \cd{error} cannot anticipate all such possible styles,
and so it is incumbent upon the presenter of the message
to make any necessary adjustments.
\end{implementation}
\afternoterule

Common Lisp does not specify the manner in which error messages and
other messages are displayed.  For the purposes of exposition,
a fairly simple style of textual presentation will be used in the
examples in this chapter.  The character \cd{>} is used
to represent the command prompt symbol for a debugger.

\begin{defun}[Function]
error format-string &rest args

\begin{obsolete}\noindent
This function signals a fatal error.  It is impossible to continue
from this kind of 
error; thus \cd{error} will never return to its caller.

The debugger printout in the following example is typical of what
an implementation might print when \cd{error} is called.
Suppose that the (misspelled) symbol \cd{emergnecy-shutdown} has no property
named \cd{command} (all too likely, as it is probably a typographical
error for \cd{emergency-shutdown}).
\begin{lisp}
(defun command-dispatch (cmd) \\
~~(let ((fn (get cmd 'command))) \\
~~~~(if (not (null fn)) \\
~~~~~~~~(funcall fn)) \\
~~~~~~~~(error "The command {\Xtilde}S is unrecognized." cmd)))) \\
 \\
(command-dispatch 'emergnecy-shutdown) \\
Error: The command EMERGNECY-SHUTDOWN is unrecognized. \\
Error signaled by function COMMAND-DISPATCH. \\
> 
\end{lisp}
\end{obsolete}

\begin{new}
X3J13 voted in June 1988
\issue{CONDITION-SYSTEM}
to adopt a proposal for a Common Lisp Condition System. 
This proposal modifies the definition of \cd{error} to specify its interaction
with the condition system.  See section~\ref{SIGNALLING-CONDITIONS}.
\end{new}

\beforenoterule
\begin{incompatibility}
Lisp Machine Lisp calls this function \cd{ferror}.
MacLisp has a function named \cd{error} that takes
different arguments and can signal either a fatal or a continuable error.
\end{incompatibility}
\afternoterule
\end{defun}

\begin{defun}[Function]
cerror continue-format-string error-format-string &rest args

\begin{obsolete}\noindent
\cd{cerror} is used to signal continuable errors.  Like \cd{error}, it
signals an error and enters the debugger.  However, \cd{cerror} allows
the program to be continued from the debugger after resolving the
error.  

If the program is continued after encountering the error, \cd{cerror}
returns {\false}.  The code that follows the call to \cd{cerror} will
then be executed.   This code should correct the problem, perhaps by
accepting a new value from the user if a variable was invalid.

If the code that corrects the problem interacts with the program's 
use and might possibly be misled,
it should make sure the error has really been corrected before
continuing.  One way to do this is to put the call to \cd{cerror} and
the correction code in a loop, checking each time to see if the error
has been corrected before terminating the loop.

The {\it continue-format-string} argument, like the {\it error-format-string}
argument, is given as a control string to \cd{format} along with
the {\it args} to construct a message string.
The error message string is used in the same way that \cd{error} uses it.
The continue message string should
describe the effect of continuing.  The intent is that this
message can be displayed as an aid to the user in deciding whether and
how to continue.  For example, it might
be used by an interactive debugger as part of the documentation of its
``continue'' command.  

The content of the continue message should adhere to the rules
of style for error messages.  It should not include
any statement of how the ``continue'' command is given, since this may be
different for each debugger.  (It is up to the debugger to supply this
information according to its own particular style of presentation and user
interaction.)
\end{obsolete}

\begin{new}
X3J13 voted in June 1988
\issue{CONDITION-SYSTEM}
to adopt a proposal for a Common Lisp Condition System. 
This proposal modifies the definition of \cd{cerror} to specify its interaction
with the condition system.  See section~\ref{SIGNALLING-CONDITIONS}.
\end{new}

Here is an example where the caller of \cd{cerror}, if continued,
fixes the problem without any further user interaction:
\begin{lisp}
(let ((nvals (list-length vals))) \\
~~(unless (= nvals 3) \\
~~~~(cond ((< nvals 3) \\
~~~~~~~~~~~(cerror "Assume missing values are zero." \\
~~~~~~~~~~~~~~~~~~~"Too few values in {\Xtilde}S;{\Xtilde}\%{\Xtilde} \\
~~~~~~~~~~~~~~~~~~~~three are required, {\Xtilde} \\
~~~~~~~~~~~~~~~~~~~~but {\Xtilde}R {\Xtilde}:{\Xlbracket}were{\Xtilde};was{\Xtilde}{\Xrbracket} supplied." \\
~~~~~~~~~~~~~~~~~~~nvals (= nvals 1)) \\
~~~~~~~~~~~(setq vals (append vals (subseq '(0 0 0) nvals)))) \\
~~~~~~~~~~(t (cerror "Ignore all values after the first three." \\
~~~~~~~~~~~~~~~~~~~~~"Too many values in {\Xtilde}S;{\Xtilde}\%{\Xtilde} \\
~~~~~~~~~~~~~~~~~~~~~~three are required, {\Xtilde} \\
~~~~~~~~~~~~~~~~~~~~~~but {\Xtilde}R were supplied." \\
~~~~~~~~~~~~~~~~~~~~~~nvals) \\
~~~~~~~~~~~~~(setq vals (subseq vals 0 3))))))
\end{lisp}
If \cd{vals} were the list \cd{(-47)}, the interaction might look
like this:
\begin{lisp}
Error: Too few values in (-47); \\
~~~~~~~three are required, but one was supplied. \\
Error signaled by function EXAMPLE. \\
If continued: Assume missing values are zero. \\
>
\end{lisp}
In this example, a loop is used to ensure that a test is satisfied.
(This example could be written more succinctly using \cd{assert}
or \cd{check-type}, which indeed supply such loops.)
\begin{lisp}
(do () \\
~~~~((known-wordp word) word) \\
~~(cerror "You will be prompted for a replacement word." \\
~~~~~~~~~~"{\Xtilde}S is an unknown word (possibly misspelled)." \\
~~~~~~~~~~word) \\
~~(format *query-io* "{\Xtilde}\&New word: ") \\
~~(setq word (read *query-io*)))
\end{lisp}

In complex cases where the {\it error-format-string}
uses some of the {\it args} and the
{\it continue-format-string} uses others, it may be necessary to use the
\cd{format} directives \cd{{\Xtilde}*} and \cd{{\Xtilde}{\Xatsign}*}
to skip over unwanted arguments in one or both of the
format control strings.

\beforenoterule
\begin{incompatibility}
The Lisp Machine Lisp function \cd{fsignal} is similar to this, but
returns \cd{:no-action}
rather than {\false}, and fails to distinguish between the error message
and the continue message.
\end{incompatibility}
\afternoterule
\end{defun}

\begin{defun}[Function]
warn format-string &rest args

\begin{obsolete}
\noindent
\cd{warn} prints an error message but normally
doesn't go into the debugger.  (However, this may be controlled
by the variable \cd{*break-on-warnings*}.)
\end{obsolete}
\begin{newer}
X3J13 voted in March 1989
\issue{BREAK-ON-WARNINGS-OBSOLETE}
to remove \cd{*break-on-warnings*} from the language.
See \cd{*break-on-signals*}.
\end{newer}
\begin{obsolete}
\cd{warn} returns {\false}.

This function would be just the same as
\cd{format} with the output directed 
to the stream in \cd{error-output}, except that \cd{warn}
may perform various implementation-dependent formatting and
other actions.  For example, an implementation of \cd{warn} should take
care of advancing to a fresh line before and after the error message
and perhaps supplying the name of the function that called \cd{warn}.
\end{obsolete}

\beforenoterule
\begin{incompatibility}
The Lisp Machine Lisp function \cd{compiler:warn} is an
approximate equivalent to this.
\end{incompatibility}
\afternoterule

\begin{new}
X3J13 voted in June 1988
\issue{CONDITION-SYSTEM}
to adopt a proposal for a Common Lisp Condition System. 
This proposal modifies the definition of \cd{warn} to specify its interaction
with the condition system.  See section~\ref{WARNING-CONDITIONS}.
\end{new}
\end{defun}

\begin{obsolete}
\begin{defun}[Variable]
*break-on-warnings*

If \cd{*break-on-warnings*} is not {\false}, then the function
\cd{warn} behaves like
\cd{break}.  It prints its message and then goes to the debugger or break
loop.  Continuing causes \cd{warn} to return {\false}.  This flag is intended
primarily for use when the user is debugging programs that issue warnings;
in ``production'' use, the value of \cd{*break-on-warnings*} should be {\false}.
\end{defun}
\end{obsolete}

\begin{newer}
X3J13 voted in March 1989
\issue{BREAK-ON-WARNINGS-OBSOLETE}
to remove \cd{*break-on-warnings*} from the language.
See \cd{*break-on-signals*}.
\end{newer}

\begin{defun}[Function]
break &optional format-string &rest args

\begin{obsolete}\noindent
\cd{break} prints the message and goes directly into the debugger,
without allowing 
any possibility of interception by programmed error-handling facilities.
(Right now, there aren't any error-handling facilities defined in Common Lisp,
but there might be in particular implementations, and there will be some
defined by Common Lisp in the future.)
When continued, \cd{break} returns {\false}.  It is permissible to
call \cd{break} with no arguments; a suitable default message will be provided.

\cd{break} is presumed to be used as a way of inserting temporary debugging
``breakpoints'' in a program, not as a way of signaling errors;
it is expected that
continuing from a \cd{break} will not trigger any unusual recovery action.
For this reason, \cd{break} does not
take the additional \cd{format} control string argument that \cd{cerror}
takes.  This and the lack of any possibility of interception by programmed
error handling are the only program-visible differences between \cd{break}
and \cd{cerror}.
The interactive debugger may choose to display them
differently; for instance, a \cd{cerror} message might be prefixed with
the herald
``\cd{Error:~}'' and a \cd{break} message with
``\cd{Break:~}''.  This depends on
the user-interface style of the particular implementation.  A particular
implementation may choose, according to its own style and needs,
when \cd{break} is called to go
into a debugger different from the one used for handling errors.
For example, it might go into an ordinary read-eval-print loop identical to
the top-level one except for the provision of a ``continue'' command that
causes \cd{break} to return {\false}.
\end{obsolete}

\beforenoterule
\begin{incompatibility}
In MacLisp, \cd{break} is a special form (FEXPR)
that takes two optional arguments.  The first is a symbol (it would be a
string if MacLisp had strings), which is not evaluated.  The second is
evaluated to produce a truth value specifying whether \cd{break} should
break (true) or return immediately (false).  In Common Lisp one makes a call
to \cd{break} conditional by putting it inside a conditional form such as
\cd{when} or \cd{unless}.
\end{incompatibility}
\afternoterule

\begin{new}
X3J13 voted in June 1988
\issue{CONDITION-SYSTEM}
to adopt a proposal for a Common Lisp Condition System. 
This proposal modifies the definition of \cd{break} to specify its interaction
with the condition system.  See section~\ref{DEBUGGING-UTILITIES}.
\end{new}
\end{defun}

\section{Specialized Error-Signaling Forms and Macros}
\label{SPECIALIZED-ERROR-SIGNALLING}

These facilities are designed to make it convenient for the user
to insert error checks into code.

\begin{defmac}
check-type place typespec [string]

\begin{obsolete}\noindent
\cd{check-type} signals an error if the contents of {\it place} are not
of the desired type.
Upon continuing from this error, the user will be asked for a new value;
\cd{check-type} will store the new value in {\it place} and start over, 
checking the type of the new value and signaling
another error if it is still not of the desired type.  Subforms of
{\it place} may be evaluated multiple times because of the implicit
loop generated.  \cd{check-type} returns {\false}.

The {\it place} must be a generalized variable reference acceptable to
\cd{setf}.
The {\it typespec} must be a type specifier; it is not evaluated.
The {\it string} should be an English description of the type, starting with
an indefinite article (``a'' or ``an''); it is evaluated.
If {\it string} is
not supplied, it is computed automatically from {\it typespec}.
(The optional {\it string} argument is allowed because some applications
of \cd{check-type} may require a more specific description of what is
wanted than can be generated automatically from the type specifier.)

The error message will mention \cd{place}, its contents, and the desired type.
\end{obsolete}

\begin{newer}
The precise format and content of the error message
is implementation-dependent.  The example shown below
is representative of current practice.
\end{newer}

\beforenoterule
\begin{implementation}
An implementation may choose to
generate a somewhat differently worded
error message if it recognizes that {\it place} is of a particular
form, such as one of the arguments to
the function that called \cd{check-type}.
\end{implementation}
\afternoterule

\begin{new}
X3J13 voted in June 1988
\issue{CONDITION-SYSTEM}
to adopt a proposal for a Common Lisp Condition System. 
This proposal modifies the definition of \cd{check-type} to specify its
interaction with the condition system.  See section~\ref{CONDITION-ASSERTIONS}.
\end{new}

\begin{newer}
X3J13 voted in March 1988 \issue{PUSH-EVALUATION-ORDER}
to clarify order of evaluation (see section~\ref{SETF-SECTION}).
\end{newer}

Examples:
\begin{lisp}
(setq aardvarks '(sam harry fred)) \\
(check-type aardvarks (vector integer)) \\
Error: The value of AARDVARKS, (SAM HARRY FRED), \\
~~~~~~~is not a vector of integers. \\
 \\
(setq naards 'foo) \\
(check-type naards (integer 0 *) "a positive integer") \\
Error: The value of NAARDS, FOO, is not a positive integer.
\end{lisp}

\beforenoterule
\begin{incompatibility}
In Lisp Machine Lisp the equivalent facility
is called \cd{check-arg-type}.
\end{incompatibility}
\afternoterule
\end{defmac}


\begin{defmac}
assert test-form [({place}*) [string {arg}*]]

\begin{obsolete}\noindent
\cd{assert} signals an error if the value of {\it test-form} is {\false}.
Continuing 
from this error will allow the user to alter the values of some
variables, and \cd{assert} will then start over, evaluating 
{\it test-form} again.  \cd{assert} returns {\false}.

{\it test-form} is any form.  Each {\it place} (there may be any number of
them, or none) must be a generalized-variable reference acceptable to
\cd{setf}.  These should be variables on which {\it test-form} depends,
whose values may sensibly be changed by the user in attempting to correct
the error.  Subforms of each {\it place} are only evaluated if an error is
signaled, and may be re-evaluated if the error is re-signaled (after
continuing without actually fixing the problem).

The {\it string} is an
error message string, and the {\it args} are additional arguments; they are
evaluated only if an error is signaled, and re-evaluated if the error is
signaled again.
The function \cd{format} is applied in the usual way to
{\it string} and {\it args} to produce
the actual error message.  If {\it string} is omitted (and therefore also
the {\it args}), a default error message is used.
\end{obsolete}

\beforenoterule
\begin{implementation}
The debugger need not include
the {\it test-form} in the error message,
and the {\it places} should not be included in the message, but they
should be made available for the user's perusal.
If the user gives the ``continue'' command, he should be
presented with the opportunity to alter the values of any or all of the
references.  The details of this depend on the
implementation's style of user interface, of course.
\end{implementation}
\afternoterule

\begin{new}
X3J13 voted in June 1988
\issue{CONDITION-SYSTEM}
to adopt a proposal for a Common Lisp Condition System. 
This proposal modifies the definition of \cd{assert} to specify its
interaction with the condition system.  See section~\ref{CONDITION-ASSERTIONS}.
\end{new}

\begin{newer}
X3J13 voted in March 1988 \issue{PUSH-EVALUATION-ORDER}
to clarify order of evaluation (see section~\ref{SETF-SECTION}).
\end{newer}

\begin{newer}
X3J13 voted in June 1989 \issue{SETF-MULTIPLE-STORE-VARIABLES}
to extend the specification of \cd{assert} to allow a {\it place}
whose \cd{setf} method has more than one store variable (see \cd{define-setf-method}).
\end{newer}

Examples:
\begin{lisp}
(assert (valve-closed-p v1)) \\
 \\
(assert (valve-closed-p v1) () \\
~~~~~~~~"Live steam is escaping!") \\
 \\
(assert (valve-closed-p v1) \\
~~~~~~~~((valve-manual-control v1)) \\
~~~~~~~~"Live steam is escaping!") \\
 \\
;; Note here that the user is invited to change BASE,  \\
;; but not the bounds MINBASE and MAXBASE. \\[3pt]
\\
(assert (<= minbase base maxbase) \\
~~~~~~~~(base) \\
~~~~~~~~"Base {\Xtilde}D is not in the range {{\Xtilde}D, {\Xtilde}D}" \\
~~~~~~~~base minbase maxbase) \\
 \\
;; Note here that it is probably not desirable to include the \\
;; entire contents of the two matrices in the error message. \\
;; It is reasonable to assume that the debugger will give \\
;; the user access to the values of the places A and B. \\
\\
(assert (= (array-dimension a 1)  \\
~~~~~~~~~~~(array-dimension b 0)) \\
~~~~~~~~(a b) \\
~~~~~~~~"Cannot multiply a {\Xtilde}D-by-{\Xtilde}D matrix {\Xtilde} \\
~~~~~~~~~and a {\Xtilde}D-by-{\Xtilde}D matrix." \\
~~~~~~~~(array-dimension a 0) \\
~~~~~~~~(array-dimension a 1) \\
~~~~~~~~(array-dimension b 0) \\
~~~~~~~~(array-dimension b 1))
\end{lisp}
\end{defmac}

\section{Special Forms for Exhaustive Case Analysis}
\label{EXHAUSTIVE-CASE-ANALYSIS}

The syntax for \cd{etypecase} and \cd{ctypecase} is the same as for
\cd{typecase}, except that no \cd{otherwise} clause is permitted.
Similarly, the syntax for \cd{ecase} and \cd{ccase} is the same as for
\cd{case} except for the \cd{otherwise} clause.

\cd{etypecase} and \cd{ecase} are similar to \cd{typecase} and \cd{case},
respectively, but signal a non-continuable error rather than returning
{\false} if no clause is selected.

\cd{ctypecase} and \cd{ccase} are also similar to \cd{typecase} and \cd{case},
but signal a continuable error if no clause is selected.

\begin{defmac}
etypecase keyform {(type {\,form}*)}*

\begin{obsolete}\noindent
This control construct is similar to \cd{typecase},
but no explicit \cd{otherwise} or \cd{t} clause is permitted.
If no clause is satisfied, \cd{etypecase} signals an error with
a message constructed from the clauses.  It is not permissible to
continue from this error.  To supply an application-specific error message, the
user should use \cd{typecase} with an \cd{otherwise} clause containing a call
to \cd{error}.  The name of this function stands for ``exhaustive
type case'' or ``error-checking type case.''
For example:
\begin{lisp}
(setq x 1/3) \\
(etypecase x \\
~~(integer x) \\
~~(symbol (symbol-value x))) \\
Error: The value of X, 1/3, is neither \\
~~~~~~~an integer nor a symbol. \\
>
\end{lisp}
\end{obsolete}

\begin{new}
X3J13 voted in June 1988
\issue{CONDITION-SYSTEM}
to adopt a proposal for a Common Lisp Condition System. 
This proposal modifies the definition of \cd{etypecase} to specify its
interaction with the condition system.
See section~\ref{EXHAUSTIVE-CASE-ANALYSIS-CONDITIONS}.
\end{new}
\end{defmac}

\begin{defmac}
ctypecase keyplace {(type {\,form}*)}*

\begin{obsolete}\noindent
This control construct is similar to \cd{typecase},
but no explicit \cd{otherwise} or \cd{t} clause is permitted.
The {\it keyplace} must be a generalized variable reference
acceptable to \cd{setf}.  If no clause is satisfied, \cd{ctypecase} signals an
error with a message constructed from the clauses.  Continuing from this
error causes \cd{ctypecase} to accept a new value from the user, store
it into {\it keyplace}, and start over, making the type tests again.
Subforms of {\it keyplace} may be evaluated multiple times.  The name
of this function stands for ``continuable exhaustive type case.''
\end{obsolete}

\begin{new}
X3J13 voted in June 1988
\issue{CONDITION-SYSTEM}
to adopt a proposal for a Common Lisp Condition System. 
This proposal modifies the definition of \cd{ctypecase} to specify its
interaction with the condition system.
See section~\ref{EXHAUSTIVE-CASE-ANALYSIS-CONDITIONS}.
\end{new}

\begin{newer}
X3J13 voted in March 1988 \issue{PUSH-EVALUATION-ORDER}
to clarify order of evaluation (see section~\ref{SETF-SECTION}).
\end{newer}
\end{defmac}

\begin{defmac}
ecase keyform {({({key}*) | key} {\,form}*)}*

\begin{obsolete}\noindent
This control construct is similar to \cd{case},
but no explicit \cd{otherwise} or \cd{t} clause is permitted.
If no clause is satisfied, \cd{ecase} signals an error with a
message constructed from the clauses.  It is not permissible to continue
from this error.  To supply an error message, the user should use
\cd{case} with an \cd{otherwise} clause containing a call to \cd{error}.
The name of this function stands for ``exhaustive case'' or
``error-checking case.'' 
For example:
\begin{lisp}
(setq x 1/3)	 \\
(ecase x \\
~~(alpha (foo)) \\
~~(omega (bar)) \\
~~((zeta phi) (baz))) \\
Error: The value of X, 1/3, is not \\
~~~~~~~ALPHA, OMEGA, ZETA, or PHI.
\end{lisp}
\end{obsolete}

\begin{new}
X3J13 voted in June 1988
\issue{CONDITION-SYSTEM}
to adopt a proposal for a Common Lisp Condition System. 
This proposal modifies the definition of \cd{ecase} to specify its
interaction with the condition system.
See section~\ref{EXHAUSTIVE-CASE-ANALYSIS-CONDITIONS}.
\end{new}
\end{defmac}

\begin{defmac}
ccase keyplace {({({key}*) | key} {\,form}*)}*

\begin{obsolete}\noindent
This control construct is similar to \cd{case},
but no explicit \cd{otherwise} or \cd{t} clause is permitted.
The {\it keyplace} must be a generalized variable reference
acceptable to \cd{setf}.  If no clause is satisfied, \cd{ccase} signals an error
with a message constructed from the clauses.  Continuing from this error
causes \cd{ccase} to accept a new value from the user, store it into
{\it keyplace}, and start over, making the clause tests again.  Subforms of
{\it keyplace} may be evaluated multiple times.  The name of this function
stands for ``continuable exhaustive case.''
\end{obsolete}

\begin{new}
X3J13 voted in June 1988
\issue{CONDITION-SYSTEM}
to adopt a proposal for a Common Lisp Condition System. 
This proposal modifies the definition of \cd{ccase} to specify its
interaction with the condition system.
See section~\ref{EXHAUSTIVE-CASE-ANALYSIS-CONDITIONS}.
\end{new}

\begin{newer}
X3J13 voted in March 1988 \issue{PUSH-EVALUATION-ORDER}
to clarify order of evaluation (see section~\ref{SETF-SECTION}).
\end{newer}
\end{defmac}

\beforenoterule
\begin{rationale}
The special forms
\cd{etypecase}, \cd{ctypecase}, \cd{ecase}, and \cd{ccase}
are included in Common Lisp, even though a user
could write them himself using the other standard facilities provided,
because it is likely that many users will want these.
Common Lisp therefore provides
a standard consistent set rather than allowing
a variety of incompatible dialects to develop.

In addition, experience has shown that
some Lisp programmers are too lazy to put an appropriate
\cd{otherwise} clause into every \cd{case} statement to
check for cases they
didn't anticipate, even if they would agree that it will probably 
hurt them later.  If an \cd{otherwise} clause can be included
very easily by adding one character to the name of the construct,
it is perhaps more likely that programmers will take the trouble to do it. 

The \cd{e} versions do nothing more than supply
automatically generated \cd{otherwise} clauses, but correct
implementation of the \cd{c} versions
requires some care.  It is therefore especially
important that the \cd{c} versions be provided
by the system so users don't have to puzzle them out on
their own.  Individual implementations may be able to do a better job
of supporting these special forms,
using their own idiosyncratic facilities, than can be done
using the error-signaling facilities defined by Common Lisp.
\end{rationale}
\afternoterule
    % Errors and the debugger
%Part{MISC, Root = "CLM.MSS"}
%%%Chapter of Common Lisp Manual.  Copyright 1984, 1988, 1989 Guy L. Steele Jr.

\clearpage\def\pagestatus{FINAL PROOF}

\chapter{Miscellaneous Features}

In this chapter are described various things that don't
seem to fit neatly anywhere else in this book:
the compiler, the \cdf{documentation}
function, debugging aids, environment inquiries (including facilities
for calculating and measuring time), and the \cdf{identity} function.



\section{The Compiler}
\label{COMPILER-SECTION}

\begin{obsolete}\noindent
The compiler is a program that may make code run faster by translating
programs into an implementation-dependent form that can
be executed more efficiently by the computer.  Most of the time
you can write programs without worrying about the compiler;
compiling a file of code should produce an equivalent but more
efficient program.  When doing more esoteric things, you may need to
think carefully about what happens at ``compile time'' and what happens
at ``load time.''  Then the difference between the syntaxes \cd{\#.}
and \cd{\#,} becomes important, and the \cdf{eval-when} construct
becomes particularly useful.
\end{obsolete}

\begin{newer}
X3J13 voted in January 1989
\issue{SHARP-COMMA-CONFUSION} to remove \cd{\#,} from the language.
\end{newer}

\everypar{}
Most declarations are not used by the Common Lisp interpreter;
they may be used to give advice to the compiler.  The compiler may attempt
to check your advice and warn you if it is inconsistent.

Unlike most other Lisp dialects, Common Lisp recognizes \cdf{special}
declarations in interpreted code as well as compiled code.
This potential source of incompatibility between interpreted and compiled
code is thereby \emph{eliminated} in Common Lisp.

The internal workings of a compiler will of course be highly
implementation-dependent.  The following functions provide a standard
interface to the compiler, however.


\begin{defun}[Function]
compile name &optional definition

\begin{obsolete}\noindent
If \emph{definition} is supplied, it should be a lambda-expression,
the interpreted function to be compiled.  If it is not supplied,
then \emph{name} should be a symbol with a definition that is a
lambda-expression; that definition is compiled
and the resulting compiled code is put back into the symbol
as its function definition.
\end{obsolete}

\emph{name} may be any function-name (a symbol or a list
whose car is \cdf{setf}---see section~\ref{FUNCTION-NAME-SECTION}).
One may write \cd{(compile '(setf cadr))} to compile the \cdf{setf}
expansion function for \cdf{cadr}.

\begin{newer}
X3J13 voted in October 1988 \issue{COMPILE-ARGUMENT-PROBLEMS}
to restate the preceding paragraph more precisely and to extend the
capabilities of \cdf{compile}.
If the optional \emph{definition} argument is supplied,
it may be either a lambda-expression (which is coerced to a function)
or a function to be compiled; if no \emph{definition} is supplied,
the \cdf{symbol-function} of the symbol is extracted and compiled.
It is permissible for the symbol to have a macro definition rather than
a function definition; both macros and functions may be compiled.

It is an error if the function to be compiled was defined interpretively
in a non-null lexical environment.  (An implementation is free to extend
the behavior of \cdf{compile} to compile such functions properly, but
portable programs may not depend on this capability.)  The consequences
of calling \cdf{compile} on a function that is already compiled
are unspecified.
\end{newer}

\begin{obsolete}
The definition is compiled and a compiled-function object produced.
If \emph{name} is a non-{\nil}
symbol, then the compiled-function object is installed as the
global function definition of the symbol and the symbol is returned.
If \emph{name} is {\false}, then the compiled-function object itself is returned.
For example:
\begin{lisp}
\\
(defun foo ...) \EV\ foo~~~~~~~~\=;\textrm{A function definition} \\
(compile 'foo) \EV\ foo\>;\textrm{Compile it} \\
\>;\textrm{Now \cdf{foo} runs faster (maybe)} \\[4pt]
(compile {\false} \\
~~~~~~~~~'(lambda (a b c) (- (* b b) (* 4 a c)))) \\
~~~\EV\ \textrm{a compiled function of three arguments that computes $b^2-4ac$}
\end{lisp}
\end{obsolete}

\begin{newer}
X3J13 voted in June 1989 \issue{COMPILER-DIAGNOSTICS} to specify that
\cdf{compile} returns two additional values
indicating whether the compiler issued any diagnostics
(see section~\ref{COMPILER-DIAGNOSTICS-SECTION}).
\end{newer}
\end{defun}

X3J13 voted in March 1989 \issue{COMPILER-VERBOSITY} to add two new
keyword arguments \cd{:verbose} and \cd{:print}
to \cdf{compile-file} by analogy with \cdf{load}.
The new function definition is as follows.

\begin{defun}[Function]
compile-file input-pathname &key :output-file :verbose :print

The \emph{input-pathname} must be a valid file specifier, such as a pathname.
The defaults for \emph{input-filename} are taken from the variable
\cd{*default-pathname-defaults*}.
The file should be a Lisp source file;
its contents are compiled and written as a binary object file.

The \cd{:verbose} argument (which defaults to the value of
\cd{*compile-verbose*}), if true, permits \cdf{compile-file} to print a message
in the form of a comment to \cdf{*standard-output*} indicating what file is
being compiled and other useful information.

The \cd{:print} argument (which defaults to the value of \cd{*compile-print*}),
if true, causes information about top-level forms in the file being
compiled to be printed to \cdf{*standard-output*}.  Exactly what is printed
is implementation-dependent; nevertheless something will be printed.
\end{defun}

\begin{new}
X3J13 voted in March 1988
\issue{PATHNAME-STREAM}
to specify exactly which streams may be used as pathnames
(see section~\ref{PATHNAME-FUNCTIONS}).
\end{new}
\begin{newer}
X3J13 voted in June 1989 \issue{PATHNAME-WILD}
to clarify that supplying a wild pathname
as the \emph{input-pathname} argument to \cdf{compile-file} has implementation-dependent consequences;
\cdf{compile-file} might signal an error, for example,
or might compile all files that match the wild pathname.
\end{newer}

\begin{newer}
X3J13 voted in June 1989 \issue{PATHNAME-LOGICAL} to require \cdf{compile-file}
to accept logical pathnames (see section~\ref{LOGICAL-PATHNAMES-SECTION}).
\end{newer}

The \cd{:output-file} argument may be used to specify an output pathname;
it defaults in a manner
appropriate to the implementation's file system conventions.


\begin{newer}
X3J13 voted in June 1989 \issue{COMPILER-DIAGNOSTICS} to specify that
\cdf{compile-file} returns three values: the \cdf{truename} of the output
file (or \cdf{nil} if the file could not be created) and two values
indicating whether the compiler issued any diagnostics
(see section~\ref{COMPILER-DIAGNOSTICS-SECTION}).
\end{newer}

\begin{newer}
X3J13 voted in October 1988 \issue{COMPILE-FILE-PACKAGE} to specify that
\cdf{compile-file}, like \cdf{load}, rebinds \cdf{*package*} to its current value.  If
some form in the file changes the value of \cdf{*package*},
the old value will be restored when compilation is completed.
\end{newer}


\begin{newer}
X3J13 voted in June 1989 \issue{COMPILE-FILE-SYMBOL-HANDLING} to specify
restrictions on conforming programs to ensure consistent handling of symbols
and packages.

  In order to guarantee that compiled files can be loaded correctly,
  the user must ensure that the packages referenced in the file are defined
  consistently at compile and load time.  Conforming Common Lisp programs
  must satisfy the following requirements.
\begin{itemize}
\item The value of \cdf{*package*} when a top-level form in the file is processed
      by \cdf{compile-file} must be the same as the value of \cdf{*package*} when the
      code corresponding to that top-level form in the compiled file is
      executed by the loader.  In particular,
      any top-level form in a file that alters the value of \cdf{*package*}
          must change it to a package of the same name at both compile and
          load time; moreover, if the first non-atomic top-level form
          in the file is not a call to
          \cdf{in-package}, then the value of \cdf{*package*} at the time \cdf{load} is
          called must be a package with the same name as the package that
          was the value of \cdf{*package*} at the time \cdf{compile-file} was called.

\item For every symbol appearing lexically within a top-level form that
      was accessible in the package that was the value of \cdf{*package*}
      during processing of that top-level form at compile time, but
      whose home package was another package, at load time there must
      be a symbol with the same name that is accessible in both the
      load-time \cdf{*package*} and in the package with the same name as the
      compile-time home package. 
  
\item For every symbol in the compiled file that was an external symbol in
      its home package at compile time, there must be a symbol with the
      same name that is an external symbol in the package with the same name
      at load time.
\end{itemize}
  If any of these conditions do not hold, the package in which \cdf{load} looks
  for the affected symbols is unspecified.  Implementations are permitted 
  to signal an error or otherwise define this behavior.

    These requirements are merely an explicit statement of the status quo,
    namely that users cannot depend on any particular behavior if the
    package environment at load time is inconsistent with what existed
    at compile time. 
\end{newer}


\begin{newer}
X3J13 voted in March 1989 \issue{IN-SYNTAX}
to specify that \cdf{compile-file} must bind \cd{*readtable*} to its current value
at the time \cdf{compile-file} is called; the dynamic extent of the binding
should encompass all of the file-loading activity.
This allows a portable program to include forms such as
\begin{lisp}
(in-package "FOO") \\*
\\*
(eval-when (:execute :load-toplevel :compile-toplevel) \\*
~~(setq *readtable* foo:my-readtable))
\end{lisp}
without performing a net global side effect on the loading environment.
Such statements allow the remainder of such a file to be read either as
interpreted code or by \cdf{compile-file} in a syntax determined by
an alternative readtable.
\end{newer}

\begin{newer}
X3J13 voted in June 1989 \issue{LOAD-TRUENAME}
to require that \cdf{compile-file} bind two new variables
\cd{*compile-file-pathname*} and \cd{*compile-file-truename*}; the dynamic extent of the bindings
should encompass all of the file-compiling activity.
\end{newer}

\begin{newer}
\begin{defun}[Variable]
*compile-verbose*

X3J13 voted in March 1989 \issue{COMPILER-VERBOSITY}
to add \cd{*compile-verbose*}.
This variable provides the default for the \cd{:verbose} argument
to \cdf{compile-file}.  Its initial value is implementation-dependent.

A proposal was submitted to X3J13 in October 1989
to rename this \cd{*compile-\discretionary{}{}{}file-\discretionary{}{}{}verbose*} for consistency.
\end{defun}
\end{newer}

\begin{newer}
\begin{defun}[Variable]
*compile-print*

X3J13 voted in March 1989 \issue{COMPILER-VERBOSITY}
to add \cd{*compile-print*}.
This variable provides the default for the \cd{:print} argument
to \cdf{compile-file}.  Its initial value is implementation-dependent.

A proposal was submitted to X3J13 in October 1989
to rename this  \cd{*compile-\discretionary{}{}{}file-\discretionary{}{}{}print*} for consistency.
\end{defun}
\end{newer}


\begin{defun}[Variable]
*compile-file-pathname*

X3J13 voted in June 1989 \issue{LOAD-TRUENAME} to introduce \cd{*compile-file-pathname*};
it is initially \cdf{nil} but \cdf{compile-file} binds it to a pathname that
represents the file name given as the first argument to \cdf{compile-file} merged
with the defaults (see \cdf{merge-pathname}).
\end{defun}

\begin{defun}[Variable]
*compile-file-truename*

X3J13 voted in June 1989 \issue{LOAD-TRUENAME} to introduce \cd{*compile-file-truename*};
it is initially \cdf{nil} but \cdf{compile-file} binds it to the ``true name'' of
the pathname of the file being compiled.  See \cdf{truename}.
\end{defun}

\begin{newer}
\begin{defspec}
load-time-value form [read-only-p]

X3J13 voted in March 1989 \issue{LOAD-TIME-EVAL} to add
   a mechanism for delaying evaluation of a \emph{form}
   until it can be done in the run-time environment.  

   If a \cdf{load-time-value} expression is seen by \cdf{compile-file}, the compiler
   performs its normal semantic processing (such as macro expansion and
   translation into machine code) on the form, but arranges for the
   execution of the \emph{form} to occur at load time in a null
   lexical environment, with the result of this evaluation then being
   treated as an immediate quantity (that is, as if originally quoted)
   at run time.  It is guaranteed that 
   the evaluation of the \emph{form} will take place only once when the file is 
   loaded, but the order of evaluation with respect to the execution
   of top-level forms in the file is unspecified.

   If a \cdf{load-time-value} expression appears within a function compiled
   with \cdf{compile}, the \emph{form} is evaluated at compile time in a null lexical
   environment.  The result of this compile-time evaluation is treated as 
   an immediate quantity in the compiled code.  

   In interpreted code, \emph{form} is evaluated (by \cdf{eval}) in a null
   lexical environment and one value is returned.  Implementations that
   implicitly compile (or partially compile) expressions passed to
   \cdf{eval} may evaluate the \emph{form} only once, at the time this
   compilation is performed.  This is intentionally similar to the
   freedom that implementations are given for the time of expanding
   macros in interpreted code.

  If the same (as determined by \cdf{eq}) list \cd{(load-time-value \emph{form})} is
  evaluated or compiled more than once, it is unspecified whether the \emph{form}
  is evaluated only once or is evaluated more than once.  This can
  happen both when an expression being evaluated or compiled shares
  substructure and when the same expression is passed to \cdf{eval} or to
  \cdf{compile} multiple times.  Since a \cdf{load-time-value} expression may be
  referenced in more than one place and may be evaluated multiple times
  by the interpreter, it is unspecified whether each execution returns
  a ``fresh'' object or returns the same object as some other execution.
  Users must use caution when destructively modifying the resulting
  object.

  If two lists \cd{(load-time-value \emph{form})} are \cdf{equal} but not \cdf{eq}, their
  values always come from distinct evaluations of \emph{form}.  Coalescing
  of these forms is not permitted.

   The optional \emph{read-only-p} argument designates whether the result
   may be considered a
   read-only constant. If \cdf{nil} (the default), the result must be considered
   ordinary, modifiable data. If \cdf{t}, the result is a read-only quantity
   that may, as appropriate, be copied into read-only space and may,
   as appropriate, be shared
   with other programs.  The \emph{read-only-p} argument is
   not evaluated and only the literal symbols \cdf{t} and \cdf{nil} are permitted.

   This new feature addresses the same set of needs as the now-defunct
   \cd{\#,} reader syntax but in a cleaner and more general manner.
   Note that \cd{\#,} syntax was reliably useful only inside quoted structure
   (though this was not explicitly mentioned in the first edition),
   whereas a \cdf{load-time-value} form must appear outside quoted structure in a
   for-evaluation position.

   See \cdf{make-load-form}.
\end{defspec}
\end{newer}

\begin{defun}[Function]
disassemble name-or-compiled-function

The argument should be a function object, a lambda-expression, or
a symbol with a function definition.  If the relevant function is not a
compiled function, it is first compiled.  In any case, the compiled code
is then ``reverse-assembled'' and printed out in a symbolic format.  This
is primarily useful for debugging the compiler, but also often of use to
the novice who wishes to understand the workings of compiled code.

\beforenoterule
\begin{implementation}
Implementors are encouraged to make the output
readable, preferably with helpful comments.
\end{implementation}
\afternoterule

\begin{newer}
X3J13 voted in March 1988 \issue{DISASSEMBLE-SIDE-EFFECT}
to clarify that when \cdf{disassemble} compiles a function, it never
installs the resulting compiled-function object in the
\cdf{symbol-function} of a symbol.
\end{newer}

\begin{newer}
X3J13 voted in March 1989 \issue{FUNCTION-NAME} to extend \cdf{disassemble}
to accept as a \emph{name} any function-name (a symbol or a list
whose car is \cdf{setf}---see section~\ref{FUNCTION-NAME-SECTION}).
Thus one may write \cd{(disassemble '(setf cadr))} to disassemble the \cdf{setf}
expansion function for \cdf{cadr}.
\end{newer}
\end{defun}

\begin{new}
\begin{defun}[Function]
function-lambda-expression fn

X3J13 voted in January 1989
\issue{FUNCTION-DEFINITION}
to add a new function to allow the
source code for a defined function to be recovered.
(The committee noted that the first edition provided no
portable way to recover a lambda-expression once it had
been compiled or evaluated to produce a function.)

This function takes one argument, which must be a function, and returns
three values.

The first value is the defining lambda-expression for the
function, or {\false} if that information is not available.
The lambda-expression may have been preprocessed in some ways
but should nevertheless be of a form suitable as an argument
to the function \cdf{compile} or for use in the \cdf{function} special form.

The second value is {\false} if the function was definitely
produced by closing
a lambda-expression in the null lexical environment; it is some
non-{\false} value if the function might have been closed in some
non-null lexical environment.

The third value is the ``name'' of the function; this is {\false} if the
name is not available or if the function had no name.
The name is intended for debugging purposes only and may be
any Lisp object (not necessarily one that would be valid for use as a name
in a \cdf{defun} or \cdf{function} special form, for example).

\beforenoterule
\begin{implementation}
An implementation is always free to return the values
{\false}, \cdf{t}, {\false} from this function but is encouraged to
make more useful information available as appropriate.
For example, it may not be desirable for files of compiled code
to retain the source lambda-expressions for use after the file is loaded,
but it is probably desirable for
functions produced by ``in-core'' calls to \cdf{eval},
\cdf{compile}, or \cdf{defun} to retain the defining lambda-expression
for debugging purposes.  The function \cdf{function-lambda-expression}
makes this information, if retained, accessible in a standard and portable
manner.
\end{implementation}
\afternoterule
\end{defun}
\end{new}


\begin{newer}
\begin{defmac}
with-compilation-unit ({option-name option-value}*) {\,form}*

X3J13 voted in March 1989 \issue{WITH-COMPILATION-UNIT}
to add \cdf{with-compilation-unit}, which
   executes the body forms as an implicit \cdf{progn}. Within the dynamic context
   of this form, warnings deferred by the compiler until ``the end of
   compilation'' will be deferred until the end of the outermost call
   to \cdf{with-compilation-unit}. The results are the same as those of
   the last of the forms (or \cdf{nil} if there is no \emph{form}).

   Each \emph{option-name} is an unevaluated keyword; each \emph{option-value}
   is evaluated. The set of keywords permitted may be extended by the
   implementation, but the only standard option keyword is \cd{:override};
   the default value for this option is \cdf{nil}.
   If \cdf{with-compilation-unit} forms are nested dynamically, only the outermost
   such call has any effect unless the \cd{:override} value of an
   inner call is true.

  The function \cdf{compile-file} should
  provide the effect of
  \begin{lisp}
  (with-compilation-unit (:override nil) ...)
  \end{lisp}
  around its code.

  Any implementation-dependent extensions to this behavior may be provided only
  as the result of an explicit programmer request by use of 
  an implementation-dependent keyword.  It is forbidden for an implementation
  to attach additional meaning to a conforming use of this
  macro.

  Note that not all compiler warnings are deferred. In some implementations,
  it may be that none are deferred. This macro only creates an
  interface to the capability where it exists, it does not require the
  creation of the capability. An implementation that does not 
  defer any compiler warnings may correctly implement this macro
  as an expansion into a simple \cdf{progn}.
\end{defmac}
\end{newer}

\subsection{Compiler Diagnostics}
\label{COMPILER-DIAGNOSTICS-SECTION}

X3J13 voted in June 1987 \issue{COMPILER-WARNING-STREAM} to specify
that \cdf{compile} and \cdf{compile-file}
may output warning messages; any such messages should
go to the stream that is the value of \cdf{*error-output*}.

X3J13 voted in June 1989 \issue{COMPILER-DIAGNOSTICS}
to specify the use of conditions to signal various erroneous situations
during compilation.
First, note that
\cdf{error} and \cdf{warning} conditions may be signaled either by the compiler itself
or by code being processed by the compiler (for example, arbitrary errors may 
    occur during compile-time macro expansion or processing of \cdf{eval-when}
    forms).
Considering only those conditions signaled \emph{by the compiler} (as
    opposed to \emph{during compilation}):
\begin{itemize}

\item   Conditions of type \cdf{error} may be signaled by the compiler in
        situations where the compilation cannot proceed without
        intervention.  Examples of such situations may include errors when opening
        a file or syntax errors.

\item  Conditions of type \cdf{warning} may be signaled by the compiler in 
        situations where the standard explicitly states that a warning must,
        should, or may be signaled.  They may also be signaled
        when the compiler can determine 
        that a situation would result at runtime that would have
        undefined consequences or would cause
        an error to be signaled.
        Examples of such situations may include
            violations of type declarations,
            altering or rebinding a constant defined with \cdf{defconstant},
            calls to built-in Lisp functions with too few or too many arguments
                or with malformed keyword argument lists,
            referring to a variable declared \cdf{ignore}, or
            unrecognized declaration specifiers.

\item  The compiler is permitted to signal diagnostics about matters of
        programming style as conditions of type \cdf{style-warning}, a subtype
    of \cdf{warning}.  Although 
        a \cdf{style-warning} condition \emph{may} be signaled in these situations, no 
        implementation is \emph{required} to do so.  However, if an 
        implementation does choose to signal a condition, that condition 
        will be of type \cdf{style-warning} and will be signaled by a call to 
        the function \cdf{warn}.
        Examples of such situations may include
            redefinition of a function with an incompatible argument list,
            calls to functions (other than built-in functions)
                with too few or too many arguments
                or with malformed keyword argument lists,
            unreferenced local variables not declared \cdf{ignore}, or
            standard declaration specifiers that are ignored by 
                the particular compiler in question.
\end{itemize}

Both \cdf{compile} and \cdf{compile-file} are permitted (but not
    required) to establish a handler for conditions of type \cdf{error}.
    Such a handler
    might, for example, issue a warning and restart compilation from some
    implementation-dependent point in order to let the compilation
    proceed without manual intervention.

The functions \cdf{compile} and \cdf{compile-file} each return three values.
See the definitions of these functions for descriptions of the first value.
    The second value is \cdf{nil} if no compiler diagnostics were issued, and
    true otherwise.
    The third value is \cdf{nil} if no compiler diagnostics other than style
    warnings were issued; a non-\cdf{nil} value indicates that there were 
    ``serious'' compiler diagnostics issued or that other conditions of
    type \cdf{error} or \cdf{warning} (but not \cdf{style-warning}) were signaled during
    compilation.


\subsection{Compiled Functions}

X3J13 voted in June 1989 \issue{COMPILED-FUNCTION-REQUIREMENTS}
to impose certain requirements on the functions produced by the compilation
process.


If a function is of type \cdf{compiled-function}, then
all macro calls appearing lexically within the function have 
        already been expanded and will not be expanded again when the
        function is called.  The process of
        compilation effectively turns every \cdf{macrolet} or \cdf{symbol-macrolet}
        construct into a \cdf{progn} (or a \cdf{locally}) with all
        instances of the local macros in the body fully expanded.

If a function is of type \cdf{compiled-function}, then
all \cdf{load-time-value} forms appearing lexically within the function have
        already been pre-evaluated and will not be evaluated
        again when the function is called.
  
Implementations are free to classify every function as 
   a \cdf{compiled-function} provided that all functions
satisfy the preceding requirements.
Conversely, it is permissible for a function that is
      not a \cdf{compiled-function} to satisfy the preceding requirements.
  
If one or more functions are defined in a file that is compiled
      with \cdf{compile-file} and the compiled file is subsequently loaded
by the function \cdf{load},
the resulting loaded function definitions must be of
    type \cdf{compiled-function}.
  
The function \cdf{compile} must produce an object of type
      \cdf{compiled-function} as the value that is either returned
or stored into the \cdf{symbol-function} of a symbol argument.

Note that none of these restrictions addresses questions of the compilation technology or
target instruction set.  For example, a compiled function does not necessarily consist of
native machine instructions.  These requirements merely specify the behavior of the type
system with respect to certain actions taken by \cdf{compile}, \cdf{compile-file}, and
\cdf{load}.

\subsection{Compilation Environment}

X3J13 voted in June 1989 \issue{COMPILE-ENVIRONMENT-CONSISTENCY}
to specify what information must be available at compile time
for correct compilation
and what need not be available until run time.

The following information must be present in the compile-time
environment for a program to be compiled correctly.  This
information need not also be present in the run-time environment.
\begin{itemize}
\item In conforming code, macros referenced in the code being compiled
        must have been previously defined in the compile-time environment.
	The compiler must treat as a function call any form that is a list whose \emph{car} is
	a symbol that does not name a macro or special form.
  (This implies that \cdf{setf} methods must also be available at
	compile time.)

\item In conforming code, proclamations for \cdf{special} variables must
        be made in the compile-time environment before any bindings of
        those variables are processed by the compiler.  The compiler
        must treat any binding of an undeclared variable as a lexical
        binding.
\end{itemize}


The compiler may incorporate the following kinds of information
into the code it produces, if the information is present in the
compile-time environment and is referenced within the code being
compiled; however, the compiler is not required to do so.
When compile-time and run-time definitions differ, it is
unspecified which will prevail within the compiled code
(unless some other behavior is explicitly specified below).  It is also
permissible for an implementation to signal an error at run time on
detecting such a discrepancy.  In all cases, the absence of the
information at compile time is not an error, but its presence may
enable the compiler to generate more efficient code.
\begin{itemize}
\item The compiler may assume that functions that are defined and
	declared \cdf{inline} in the compile-time environment will retain the
        same definitions at run time.

\item The compiler may assume that, within a named function, a
	recursive call to a function of the same name refers to the
	same function, unless that function has been declared \cdf{notinline}.
(This permits tail-recursive calls of a function to itself
to be compiled as jumps, for example, thereby turning certain recursive
schemas into efficient loops.)

\item In the absence of \cdf{notinline}
	declarations to the contrary,
 \cdf{compile-file} may assume that a call within the file being compiled to a named
	function that is defined in that file refers to that function.
	(This rule permits \emph{block compilation} of files.)  The behavior of
	the program is unspecified if functions are redefined individually 
	at run time.

\item The compiler may assume that the signature (or ``interface contract'') of
	all built-in Common Lisp functions will not change.  In addition,
	the compiler may treat all built-in Common Lisp functions as if
	they had been proclaimed \cdf{inline}.

\item The compiler may assume that the signature (or ``interface contract'') of
	functions with \cdf{ftype} information available will not change.

\item The compiler may ``wire in'' (that is, open-code or inline)
the values of symbolic constants
	that have been defined with \cdf{defconstant} in the compile-time
	environment.

\item The compiler may assume that any type definition made with \cdf{defstruct} 
        or \cdf{deftype} in the compile-time environment will retain the same 
        definition in the run-time environment.  It may also assume that
        a class defined by \cdf{defclass} in the compile-time environment will
        be defined in the run-time environment in such a way as to have
        the same superclasses and metaclass.  This implies that
        subtype/supertype relationships of type specifiers will not 
        change between compile time and run time.  (Note that it is not 
        an error for an	unknown type to appear in a declaration at
        compile time, although it is reasonable for the compiler to 
        emit a warning in such a case.)

\item The compiler may assume that if type declarations are present
	in the compile-time environment, the corresponding variables and 
	functions present in the run-time environment will actually be of
	those types.  If this assumption is violated, the run-time behavior of the program is 
	undefined.
\end{itemize}

The compiler must not make any additional assumptions about
consistency between the compile-time and run-time environments.  In 
particular, the compiler may not assume that functions that are defined
	in the compile-time environment will retain either the
	same definition or the same signature at run time, except
as described above.
Similarly,
the compiler may not signal an error if it sees a call to a
	function that is not defined at compile time, since that function
	may be provided at run time.

X3J13 voted in January 1989 \issue{COMPILE-FILE-HANDLING-OF-TOP-LEVEL-FORMS}
to specify the compile-time side effects of processing various macro forms.

Calls to defining macros such as \cdf{defmacro} or \cdf{defvar} appearing
    within a file being processed by \cdf{compile-file} normally have
    compile-time side effects that affect how subsequent forms in the
    same file are compiled.  A convenient model for explaining how these
    side effects happen is that each defining macro expands into one or
    more \cdf{eval-when} forms and that compile-time
    side effects are caused by calls occurring in the body of an
    \cd{(eval-when (:compile-toplevel) ...)} form.

The affected defining macros and their specific side effects are
    as follows.  In each case, it is identified what a user must do to
    ensure that a program is conforming, and what a compiler must do
    in order to correctly process a conforming program.

\begin{flushdesc}
\item[\cdf{deftype}]
The user must ensure that the body of a \cdf{deftype} form is
    evaluable at compile time if the type is referenced in subsequent type
    declarations.  The compiler must ensure that a type
    specifier defined by \cdf{deftype}
    is recognized in subsequent type declarations.  If the
    expansion of a type specifier is not defined fully at compile time
    (perhaps because it expands into an unknown type specifier or a
    \cdf{satisfies} of a named function that isn't defined in the compile-time
    environment), an implementation may ignore any references to this type
    in declarations and may signal a warning.

\item[\cdf{defmacro} and \cdf{define-modify-macro}]   
The compiler must store macro
    definitions at compile time, so that occurrences of the macro later on
    in the file can be expanded correctly.  The user must ensure that the
    body of the macro is evaluable at compile time if it is referenced
    within the file being compiled.

\item[\cdf{defun}]
No required compile-time side effects are associated with \cdf{defun} forms.
    In particular, \cdf{defun} does not make the function definition available
    at compile time.  An implementation may choose to store information
    about the function for the purposes of compile-time error checking
    (such as checking the number of arguments on calls) or to permit later
    \cdf{inline} expansion of the function.

\item[\cdf{defvar} and \cdf{defparameter}]
The compiler must recognize that the variables
    named by these forms have been proclaimed \cdf{special}.  However, it must
    not evaluate the \emph{initial-value} form or \cdf{set} the variable at compile
    time.

\item[\cdf{defconstant}]
The compiler must recognize that the symbol names a
    constant.  An implementation may choose to evaluate the \emph{value-form} at
    compile time, load time, or both.  Therefore the user must ensure that
    the \emph{value-form} is evaluable at compile time (regardless of whether or
    not references to the constant appear in the file) and that it always
    evaluates to the same value.  
    (There has been considerable
variance among implementations on this point.  The effect of this specification is
to legitimize all of the implementation variants by requiring care of the user.)

\item[\cdf{defsetf} and \cdf{define-setf-method}]
The compiler must make \cdf{setf} methods
    available so that they may be used to expand calls to \cdf{setf} later on in
    the file.  Users must ensure that the body of a call
    to \cdf{define-setf-method} or
    the complex form of \cdf{defsetf} is evaluable at compile time if the
    corresponding place is referred to in a subsequent \cdf{setf} in the same
    file.  The compiler must make these \cdf{setf} methods available to 
    compile-time calls to \cdf{get-setf-method} when its environment argument is
    a value received as the \cd{\&environment} parameter of a macro.
     
\item[\cdf{defstruct}]
The compiler must make the structure type name recognized
    as a valid type name in subsequent declarations (as described above
    for \cdf{deftype}) and
    make the structure slot accessors known to \cdf{setf}.
    In addition, the
    compiler must save enough information so that
    further \cdf{defstruct} definitions can include (with the \cd{:include}
    option) a structure type defined
    earlier in the file being compiled.  The functions that \cdf{defstruct}
    generates are not defined in the compile-time environment, although
    the compiler may save enough information about the functions to allow
    \cdf{inline} expansion of
    subsequent calls to these functions.  The \cd{\#S} reader syntax may or may not be 
    available for that structure type at compile time.

\item[\cdf{define-condition}]
The rules are essentially the same as those for
    \cdf{defstruct}. The compiler must make the condition type recognizable as a
    valid type name, and it must be possible to reference the condition
    type as the \emph{parent-type} of another condition type in a subsequent
    \cdf{define-condition} form in the file being compiled.

\item[\cdf{defpackage}]
 All of the actions normally performed by the \cdf{defpackage} macro at load
    time must also be performed at compile time.
\end{flushdesc}

Compile-time side effects may cause information about a
    definition to be stored in a different manner from
information about definitions
    processed either interpretively or by loading
    a compiled file.
    In particular, the information stored by a defining macro at
    compile time may or may not be available to the interpreter (either
    during or after compilation) or during subsequent calls to \cdf{compile} or
    \cdf{compile-file}.  For example, the following code is not portable because
    it assumes that the compiler stores the macro definition of \cdf{foo} where
    it is available to the interpreter.
\begin{lisp}
(defmacro foo (x) {\Xbq}(car ,x)) \\
\\
(eval-when (:execute :compile-toplevel :load-toplevel) \\*
~~(print (foo '(a b c))))~~~~~;\textrm{Wrong}
\end{lisp}
    The goal may be accomplished portably by including the macro
    definition within the \cdf{eval-when} form:
\begin{lisp}  
(eval-when (eval compile load) \\*
~~(defmacro foo (x) {\Xbq}(car ,x)) \\*
~~(print (foo '(a b c))))~~~~~;\textrm{Right}
\end{lisp}

\begin{flushdesc}
\item[\cdf{declaim}]

X3J13 voted in June 1989 \issue{PROCLAIM-ETC-IN-COMPILE-FILE}
to add a new macro \cdf{declaim} for making proclamations recognizable
at compile time.  The declaration specifiers in the \cdf{declaim} form
are effectively proclaimed at compile time so as to affect
compilation of subsequent forms.  (Note that compiler processing
of a call to \cdf{proclaim}
does not have any compile-time side effects, for \cdf{proclaim}
is a function.)
\end{flushdesc}

\begin{flushdesc}
\item[\cdf{in-package}]

X3J13 voted in March 1989 \issue{IN-PACKAGE-FUNCTIONALITY} to specify that
all of the actions normally performed by the \cdf{in-package} macro at load
time must also be performed at compile time.
\end{flushdesc}

X3J13 voted in June 1989 \issue{CLOS-MACRO-COMPILATION}
to specify the compile-time side effects of processing various CLOS-related
macro forms.  Top-level calls to the CLOS defining macros have the
 following compile-time side effects; any other compile-time behavior
 is explicitly left unspecified.

\begin{flushdesc}
\item[\cdf{defclass}]
The class name may appear in subsequent type declarations and
can be used as a specializer in subsequent \cdf{defmethod} forms.
Thus the compile-time behavior of \cdf{defclass} is similar to that of
\cdf{deftype} or \cdf{defstruct}.

\item[\cdf{defgeneric}]
The generic function can be referenced in subsequent \cdf{defmethod} forms,
but the compiler does not arrange for the generic function to be callable
    at compile time.

\item[\cdf{defmethod}]  
The compiler does not arrange for the method to be callable at compile
    time.  If there is a generic function with the same name defined at
    compile time, compiling a \cdf{defmethod} form does not add the method to that 
    generic function; the method is added to the generic
    function only when the \cdf{defmethod} form is actually executed.

    The error-signaling behavior described in the specification of
    \cdf{defmethod} in chapter~\ref{CLOS} (if the function isn't a generic function
    or if the lambda-list is not congruent) occurs only when the defining
    form is executed, not at compile time.

    The forms in \cdf{eql} parameter specializers are evaluated when the \cdf{defmethod}
    form is executed.  The compiler is permitted to build in knowledge
    about what the form in an \cdf{eql} specializer will evaluate to in cases
    where the ultimate result can be syntactically inferred without
    actually evaluating it.

\item[\cdf{define-method-combination}]
The method combination can be used in subsequent \cdf{defgeneric} forms.  

    The body of a \cdf{define-method-combination} form is evaluated no earlier 
    than when the defining macro is executed and possibly as late as 
    generic function invocation time.  The compiler may attempt to
    evaluate these forms at compile time but must not depend on being able
    to do so.
\end{flushdesc}

\subsection{Similarity of Constants}
\label{SIMILAR-AS-A-CONSTANT-SECTION}

X3J13 voted in March 1989 \issue{CONSTANT-COMPILABLE-TYPES}
to specify what objects can be in compiled constants and
what relationship there must be between a constant
passed to the compiler and the one that is established by compiling it
and then loading its file.

The key is a definition of an equivalence relationship called
``similarity as constants''
between Lisp
objects.  Code passed through the file
compiler and then loaded must behave as though quoted constants in it
are similar in this sense to quoted constants in the corresponding source code.
An object may be used as a quoted constant processed by \cdf{compile-file}
if and only if the compiler can guarantee that the resulting constant established
by loading the compiled file is ``similar as a constant'' to the
original.  Specific requirements are spelled out below.

Some types of objects, such as streams, are not supported in constants
processed by the file compiler.  Such objects may not portably appear
as constants in code processed with \cdf{compile-file}.  Conforming
implementations are required to handle such objects either by having
the compiler or loader reconstruct an equivalent copy of the
object in some implementation-specific manner or by having the
compiler signal an error.

Of the types supported in constants, some are treated as aggregate
objects.  For these types, being similar as constants is defined
recursively.  We say that an object of such a type has certain ``basic
attributes''; to be similar as a constant to another object, the
values of the corresponding attributes of the two objects must also be
similar as constants.

A definition of this recursive form has problems with any circular or infinitely
recursive object such as a list that is an element of itself.  We use
the idea of depth-limited comparison and say that two objects are
similar as constants if they are similar at all finite levels.  This
idea is implicit in the definitions below, and it applies in all the
places where attributes of two objects are required to be similar as
constants.  The question of handling circular constants is the subject
of a separate vote by X3J13 (see below).

The following terms are used throughout this section.
  The term \emph{constant} refers to a quoted or self-evaluating constant,
  not a named constant defined by \cdf{defconstant}.
  The term \emph{source code} is used to refer to the objects constructed
  when \cdf{compile-file} calls \cdf{read} (or the equivalent) and to
  additional objects constructed by
  macro expansion during file compilation.
  The term \emph{compiled code} is used to refer to objects constructed by 
  \cdf{load}.

Two objects are \emph{similar as a constant} if and only if
they are both of one of the types listed below and satisfy the
additional requirements listed for that type.

\begin{flushdesc}
\item[\cdf{number}]

  Two numbers are similar as constants if they are of the same type
  and represent the same mathematical value.
  
\item[\cdf{character}]

  Two characters are similar as constants if they both represent
  the same character.  (The intent is that this be compatible with
  how \cdf{eql} is defined on characters.)

\item[\cdf{symbol}]
  X3J13 voted in June 1989 \issue{COMPILE-FILE-SYMBOL-HANDLING}
  to define similarity as a constant for interned symbols.
  A symbol $\emph{S}$ appearing in the source code is similar as a constant to 
  a symbol $\emph{S}'$ in the compiled code if their print names are similar as constants
   and either of the following conditions holds:
\begin{itemize}
\item  $\emph{S}$ is accessible in \cdf{*package*} at compile time and $\emph{S}'$ is accessible in
       \cdf{*package*} at load time.
\item  $\emph{S}'$ is accessible in the package that is similar as a constant to the
       home package of symbol \emph{S}.
\end{itemize}
  The ``similar as constants'' relationship for interned symbols has nothing
  to do with \cd{*readtable*} or how the function \cdf{read} would parse the 
  characters in the print name of the symbol.

  An uninterned symbol in the source code is similar as a constant
  to an uninterned symbol in the compiled code if their print names
  are similar as constants.

\item[\cdf{package}]

  A package in the source code is similar as a constant to a package in
  the compiled code if their names are similar as constants.  Note that
  the loader finds the corresponding package object as if by calling
  \cdf{find-package} with the package name as an argument.  An error is
  signaled if no package of that name exists at load time.

\item[\cdf{random-state}]

 We say that two \cdf{random-state} objects are \emph{functionally equivalent} if 
  applying \cdf{random} to them repeatedly always produces the same 
  pseudo-random numbers in the same order.  
  
  Two random-states are similar as constants if and only if copies of
  them made via \cdf{make-random-state} are functionally equivalent.
  (Note that a constant \cdf{random-state} object cannot be used as the \emph{state}
  argument to the function \cdf{random} because \cdf{random} performs
  a side effect on that argument.)

\item[\cdf{cons}]

  Two conses are similar as constants if the values of their respective
  \emph{car} and \emph{cdr} attributes are similar as constants.

\item[\cdf{array}]

  Two arrays are similar as constants if the corresponding values of each
  of the following attributes are similar as constants:
  for vectors (one-dimensional arrays), the \cdf{length} and \cdf{element-type}
  and the result of \cdf{elt} for all valid indices;
  for all other arrays, the \cdf{array-rank}, the result of \cdf{array-dimension}
  for all valid axis numbers, the \cdf{array-element-type},
  and the result of \cdf{aref} for all valid indices.  (The point of
distinguishing vectors is to take any fill pointers into account.)

  If the array in the source code is a \cdf{simple-array}, then
  the corresponding array in the compiled code must also be a
  \cdf{simple-array}, but if the array in the source code is displaced, has a
  fill pointer, or is adjustable, the corresponding array in the
  compiled code is permitted to lack any or all of these qualities.

\item[\cdf{hash-table}]

  Two hash tables are similar as constants if they meet
  three requirements.
  First, they must have the same test (for example, both are \cdf{eql} hash tables
  or both are \cdf{equal} hash tables).
  Second, there must be a unique bijective correspondence between the keys of
      the two tables, such that the corresponding keys are similar as
      constants.
  Third, for all keys, the values associated with two corresponding keys
      must be similar as constants.

  If there is more than one possible one-to-one correspondence between
  the keys of the two tables, it is unspecified whether the two
  tables are similar as constants.  A conforming
  program cannot use such a table as a constant.

\item[\cdf{pathname}]

  Two pathnames are similar as constants if all corresponding pathname
  components are similar as constants.

\item[\cdf{stream}, \cdf{readtable}, and \cdf{method}]

  Objects of these types are not supported in compiled constants.

\item[\cdf{function}]


   X3J13 voted in June 1989 \issue{CONSTANT-FUNCTION-COMPILATION}
   to specify that objects of type \cdf{function}
   are not supported in compiled constants.

\item[\cdf{structure} and \cdf{standard-object}]

   X3J13 voted in March 1989 \issue{LOAD-OBJECTS} to introduce a facility
based on the Common Lisp Object System
whereby a user can specify how \cdf{compile-file} and \cdf{load}
must cooperate to reconstruct compile-time constant objects at load time
(see \cdf{make-load-form}).
\end{flushdesc}

   X3J13 voted in March 1989 \issue{CONSTANT-COLLAPSING} to specify
the circumstances under which constants may be coalesced in compiled code.

Suppose $\emph{A}$ and $\emph{B}$ are two
objects used as quoted constants in the source code, and that $\emph{A}'$ and
$\emph{B}'$ are the corresponding objects in the compiled code.  If $\emph{A}'$ and $\emph{B}'$
are \cdf{eql} but $\emph{A}$ and $\emph{B}$ were not \cdf{eql}, then we say that $\emph{A}$ and $\emph{B}$ have been
\emph{coalesced} by the compiler.

An implementation is permitted to coalesce constants
appearing in code to be compiled if and only if they are similar as
constants, except that objects of type \cdf{symbol}, \cdf{package},
\cdf{structure}, or \cdf{standard-object} obey their own rules
and may not be coalesced by a separate mechanism.

\beforenoterule
\begin{rationale}
Objects of type \cdf{symbol} and \cdf{package} cannot be coalesced because the fact
that they are named, interned objects means they are already as
coalesced as it is useful for them to be.  Uninterned symbols could
perhaps be coalesced, but that was thought to be more dangerous than
useful.  Structures and objects could be
coalesced if a ``similar as a constant'' predicate were defined for them;
it would be a generic function.  However, at present there is no such
predicate.  Currently \cdf{make-load-form} provides a protocol by which
\cdf{compile-file} and \cdf{load} work together to construct an object in the
compiled code that is equivalent to the object in the source code;
a different mechanism would have to be added to permit coalescing.
\end{rationale}
\afternoterule

Note that coalescing is possible only because it is forbidden to
destructively modify constants \issue{CONSTANT-MODIFICATION} (see \cdf{quote}).

   X3J13 voted in March 1989 \issue{CONSTANT-CIRCULAR-COMPILATION} to specify
that objects containing circular or infinitely recursive references may legitimately
appear as constants to be compiled.  The compiler is
required to preserve \cdf{eql}-ness of substructures within a file compiled
by \cdf{compile-file}.

\section{Debugging Tools Отладочные средства}

The utilities described in this section are sufficiently complex
and sufficiently dependent on the host environment that their
complete definition is beyond the scope of this book.
However, they are also sufficiently
useful to warrant mention here.  It is expected that
every implementation will
provide some version of these utilities, however clever or however simple.

Коммунальные услуги, описанные в этом разделе достаточно сложны
и достаточно зависит от внешней среды, что их
полное описание выходит за рамки этой книги.
Тем не менее, они также достаточно
полезно, чтобы оправдать упомянуть здесь. Ожидается, что
каждая реализация
предоставить некоторые версии этих программ, однако умный или же просто.

Описанные в этом разделе утилиты достаточно сложны и зависят от внешней среды
ОС, что их полное описание выходит за рамки книги. Тем не менее их описание
будет полезным. Предполагается, что каждая реализация будет представлять
некоторую версию этих утилит.

\begin{defmac}
trace {\,function-name}* \\
untrace {\,function-name}*

Invoking \cdf{trace} with one or more function-names (symbols or lists, whose
\emph{car} is \cdf{setf}---see section~\ref{FUNCTION-NAME-SECTION}),
 causes
the functions named to be traced.  Henceforth, whenever such
a function is invoked, information about the call, the arguments
passed, and the eventually returned values, if any, will be printed
to the stream that is the value of \cdf{*trace-output*}.
For example:
\begin{lisp}
(trace fft gcd string-upcase)
\end{lisp}
If a function call is open-coded (possibly as a result of an \cdf{inline}
declaration), then such a call may not produce trace output.

Invoking \cdf{untrace} with one or more function names will cause those
functions not to be traced any more.

Tracing an already traced function, or untracing a function not
currently being traced, should produce no harmful effects but may
produce a warning message.

Calling \cdf{trace} with no argument forms will return a list of functions
currently being traced.

Calling \cdf{untrace} with no argument forms will cause all currently
traced functions to be no longer traced.

The values returned by \cdf{trace} and \cdf{untrace} when
given argument forms are implementation-dependent.

\cdf{trace} and \cdf{untrace} may also accept additional
implementation-dependent argument formats.  The format of the trace
output is implementation-dependent.
\end{defmac}

\begin{defmac}
step form

This evaluates \emph{form} and returns what \emph{form} returns.
However, the user is allowed to interactively
``single-step'' through the evaluation of \emph{form}, at least
through those evaluation steps that are performed interpretively.
The nature of the interaction is implementation-dependent.
However, implementations are encouraged to respond to the typing
of the character \cd{?} by providing help, including a list
of commands.

\cdf{step} evaluates its argument \emph{form}
in the current lexical environment (not simply a null environment),
and that calls to \cdf{step} may be compiled, in which case
an implementation may step through only those parts of the
evaluation that are interpreted.  (In other words, the \emph{form}
itself is unlikely to be stepped, but if executing it happens to
invoke interpreted code, then that code may be stepped.)
\end{defmac}

\begin{defmac}
time form

This evaluates \emph{form} and returns what \emph{form} returns.  However, as
a side effect, various timing data and other information are printed to
the stream that is the value of \cdf{*trace-output*}.  The nature and
format of the printed information is implementation-dependent.  However,
implementations are encouraged to provide such information as elapsed
real time, machine run time, storage management statistics, and so on.

\cdf{time} evaluates its argument \emph{form}
in the current lexical environment (not simply a null environment),
and that calls to \cdf{time} may be compiled.
\end{defmac}

\begin{defun}[Function]
describe object &optional stream

\cdf{describe} prints, to the stream information about the \emph{object}.
Sometimes it will describe something that it finds inside something else;
such recursive descriptions are indented appropriately.  For instance,
\cdf{describe} of a symbol will exhibit the symbol's value,
its definition, and each of its properties.  \cdf{describe} of a
floating-point number will exhibit its internal representation in a way
that is useful for tracking down round-off errors and the like.
The nature and format of the output is implementation-dependent.

\cdf{describe} returns no values (that is, it returns what the expression
\cd{(values)} returns: zero values).

The output is sent to the specified \emph{stream}, which
 defaults to the value of \cdf{*standard-output*};
 the \emph{stream} may also be \cdf{nil} (meaning \cdf{*standard-output*})
 or \cdf{t} (meaning \cdf{*terminal-io*}).

The behavior of \cdf{describe} depends on the generic function
\cdf{describe-object} (see below).
\end{defun}

That \cdf{describe} is forbidden
to prompt for or require user input when given exactly one argument;
It is permitted implementations
to extend \cdf{describe} to accept keyword arguments that may cause
it to prompt for or to require user input.

\begin{defun}[Generic function][Primary method]
describe-object object stream \\
describe-object (object standard-object) stream

X3J13 voted in March 1989 \issue{DESCRIBE-UNDERSPECIFIED} to add
  the generic function \cdf{describe-object}, which writes a description of an object to a
  stream.  The function \cdf{describe-object} is called by the \cdf{describe} function; it
  should not be called by the user.

  Each implementation must provide a method on the class
  \cdf{standard-object} and methods on enough other classes to ensure that
  there is always an applicable method.  Implementations are free to add
  methods for other classes.  Users can write methods for \cdf{describe-object} for
  their own classes if they do not wish to inherit an implementation-supplied
  method.

   The first argument may be any Lisp object.  The second argument is a stream; it
   cannot be \cdf{t} or \cdf{nil}.
   The values returned by \cdf{describe-object} are unspecified.

   Methods on \cdf{describe-object} may recursively call \cdf{describe}.  Indentation,
   depth limits, and circularity detection are all taken care of automatically,
   provided that each method handles exactly one level of structure and calls
   \cdf{describe} recursively if there are more structural levels.
   If this rule is not obeyed, the results are undefined.

   In some implementations the \emph{stream} argument passed to a \cdf{describe-object}
   method is not the original stream but is an intermediate stream that
   implements parts of \cdf{describe}.  Methods should therefore not depend on the
   identity of this stream.

\beforenoterule
\begin{rationale}
 This proposal was closely modeled on the CLOS description of \cdf{print-object},
 which was well thought out and provides a great deal of functionality and
 implementation freedom.  Implementation techniques for
 \cdf{print-object} are applicable to \cdf{describe-object}.

 The reason for making the return values for \cdf{describe-object} unspecified is to
 avoid forcing users to write \cd{(values)} explicitly in all their methods;
 \cdf{describe} should take care of that.
\end{rationale}
\afternoterule
\end{defun}

\begin{defun}[Function]
inspect object

\cdf{inspect} is an interactive version of \cdf{describe}.
The nature of the interaction is implementation-dependent,
but the purpose of \cdf{inspect} is to make it easy to wander
through a data structure, examining and modifying parts of it.
Implementations are encouraged to respond to the typing
of the character \cd{?} by providing help, including a list
of commands.

The values returned by \cdf{inspect}
are implementation-dependent.
\end{defun}

\begin{defun}[Function]
room &optional x

\cdf{room} prints, to the stream in the variable \cdf{*standard-output*},
information about the state of internal storage and its management.  This
might include descriptions of the amount of memory in use and the degree
of memory compaction, possibly broken down by internal data type if that
is appropriate.  The nature and format of the printed information is
implementation-dependent.  The intent is to provide information that may
help a user to tune a program to a particular implementation.

\cd{(room nil)} prints out a minimal amount of information.
\cd{(room t)} prints out a maximal amount of information.
Simply \cd{(room)} prints out an intermediate amount
of information that is likely to be useful.

The argument \emph{x} may also be the keyword \cd{:default},
which has the same effect as passing no argument at all.
\end{defun}

\begin{defun}[Function]
ed &optional x

If the implementation provides a resident editor, this function
should invoke it.

\cd{(ed)} or \cd{(ed nil)} simply enters the editor, leaving you in the same
state as the last time you were in the editor.

\cd{(ed \emph{pathname})} edits the contents of the file specified
by \emph{pathname}.  The \emph{pathname} may be an actual pathname
or a string.

\cdf{ed} accepts logical pathnames (see
section~\ref{LOGICAL-PATHNAMES-SECTION}).

\cd{(ed \emph{symbol})} tries to let you edit the text for the function
named \emph{symbol}.  The means by which the function text is obtained
is implementation-dependent; it might involve searching the file system,
or pretty printing resident interpreted code, for example.

Function name may be any function-name (a symbol or a list
whose \emph{car} is \cdf{setf}---see section~\ref{FUNCTION-NAME-SECTION}).
Thus one may write \cd{(ed '(setf cadr))} to edit the \cdf{setf}
expansion function for \cdf{cadr}.
\end{defun}


\begin{defun}[Function]
dribble &optional pathname

\cd{(dribble \emph{pathname})} may rebind \cdf{*standard-input*}
and \cdf{*standard-output*}, and may take other appropriate
action, so as to send a record of the
input/output interaction to a file named by \emph{pathname}.
The primary purpose of this is to create a readable record of an interactive
session.

\cd{(dribble)} terminates the recording of input and output and
closes the dribble file.

\cdf{dribble} also accepts logical pathnames (see
section~\ref{LOGICAL-PATHNAMES-SECTION}).

\begin{new}
X3J13 voted in March 1988
\issue{DRIBBLE-TECHNIQUE}
to clarify that \cdf{dribble} is intended primarily
for interactive debugging and that its effect cannot be
relied upon for use in portable
programs.

Different implementations of Common Lisp have used radically different
techniques for implementing \cdf{dribble}.  All are reasonable interpretations
of the original specification, and all behave in approximately the same
way if \cdf{dribble} is called only from the interactive top level.
However, they may have quite different behaviors if \cdf{dribble} is
called from within compound forms.

Consider two models of the operation of \cdf{dribble}.  In the ``redirecting''
model, a call to \cdf{dribble} with a pathname argument
alters certain global variables such as \cdf{*standard-output*},
perhaps by constructing a broadcast stream directed to both the original
value of \cdf{*standard-output*} and to the dribble file; other streams
may be affected as well.  A call to \cdf{dribble} with no arguments
undoes these side effects.

In the ``recursive'' model, by contrast, a call to \cdf{dribble} with a
pathname argument creates a new interactive command loop and calls it
recursively.  This new command loop is just like an ordinary
read-eval-print loop except that it also echoes the interaction to
the dribble file.  A call to \cdf{dribble} with no arguments
does a \cdf{throw} that exits the recursive command loop and returns
to the original caller of \cdf{dribble} with an argument.

The two models may be distinguished by this test case:
\begin{lisp}
(progn (dribble "basketball") \\
~~~~~~~(print "Larry") \\
~~~~~~~(dribble) \\
~~~~~~~(princ "Bird"))
\end{lisp}
If this form is input to the Lisp top level, in either model
a newline (provided by the function \cdf{print}) and the words
\cd{Larry Bird} will be printed to the standard output.
The redirecting dribble model will additionally print all but the word
\cdf{Bird} to a file named \cdf{basketball}.

By contrast, the recursive dribble model will enter a recursive command
loop and not print anything until \cd{(dribble)} is executed from within
the new interactive command loop.  At that time the file named
\cdf{basketball} will be closed, and then execution of the
\cdf{progn} form will be resumed.  A newline and ``\cd{Larry~}'' (note the trailing space)
will be printed to the standard output, and then the call
\cd{(dribble)} may complain that there is no active dribble file.
Once this error is resolved, the word \cdf{Bird} may be printed
to the standard output.

Here is a slightly different test case:
\begin{lisp}
(dribble "baby-food")
\end{lisp}
\begin{lisp}
(progn (print "Mashed banana") \\*
~~~~~~~(dribble) \\*
~~~~~~~(princ "and cream of rice"))
\end{lisp}
If this form is input to the Lisp top level, in the redirecting model
a newline and the words
\cd{Mashed banana and cream of rice} will be printed to the standard output
and all but the words
\cd{and cream of rice} will be sent to a file named \cdf{baby-food}.

The recursive model will direct exactly the same output to the file
named \cdf{baby-food} but will never print the words
\cd{and cream of rice} to the standard output because the call
\cd{(dribble)} does not return normally; it throws.

The redirecting model may be intuitively more appealing to some.
The recursive model, however, may be more robust; it carefully limits
the extent of the dribble operation and disables dribbling if a
throw of any kind occurs.  The vote by X3J13 was an explicit decision
not to decide which model to use.  Users are advised to call \cdf{dribble}
only interactively, at top level.
\end{new}
\end{defun}


\begin{defun}[Function]
apropos string &optional package \\
apropos-list string &optional package

\cd{(apropos \emph{string})} tries to find all available symbols whose print names
contain \emph{string} as a substring.  (A symbol may be supplied for
the \emph{string}, in which case the print name of the symbol is used.)
Whenever \cdf{apropos} finds a symbol, it prints
out the symbol's name; in addition,
information about the function definition and dynamic value of the symbol,
if any, is printed.
If \emph{package} is specified and not {\nil}, then only symbols
available in that package are examined;
otherwise ``all'' packages are searched, as if by \cdf{do-all-symbols}.
Because a symbol may be available by way of more than one inheritance
path, \cdf{apropos} may print information about the same symbol more than once.
The information is printed to the stream that is the value
of \cdf{*standard-output*}.
\cdf{apropos} returns no values (that is, it returns what the expression
\cd{(values)} returns: zero values).

\cdf{apropos-list} performs the same search that \cdf{apropos} does but
prints nothing.  It returns a list of the symbols whose print names
contain \emph{string} as a substring.
\end{defun}

\section{Environment Inquiries Справка о среде}

Environment inquiry functions provide information about the
environment in which a Common Lisp program is being executed.
They are described here in two categories: first, those dealing with
determination and measurement of time, and second, all the others,
most of which deal with identification of the computer hardware
and software.

Справочные функции представляют информацию о среде, в которой исполняется Common
Lisp'овая программа.
Функции разделены на две категории: первые для работы со временем, и остальные
для получения имен, версий, типов программ и оборудования.

\subsection{Time Functions}
\label{TIME-SECTION}

Time is represented in three different ways in Common Lisp:
Decoded Time, Universal Time, and Internal Time.
The first two representations
are used primarily to represent calendar time and are
precise only to one second.
Internal Time is used primarily to represent measurements of computer
time (such as run time) and is precise to some implementation-dependent
fraction of a second, as specified by \cdf{internal-time-units-per-second}.
Decoded Time format is used only for absolute time indications.
Universal Time and Internal Time formats are used for both absolute
and relative times.

Decoded Time format represents calendar time as a number of components:
\begin{itemize}
\item
\emph{Second}: an integer between 0 and 59, inclusive.

\item
\emph{Minute}: an integer between 0 and 59, inclusive.

\item
\emph{Hour}: an integer between 0 and 23, inclusive.

\item
\emph{Date}: an integer between 1 and 31, inclusive (the upper limit actually
depends on the month and year, of course).

\item
\emph{Month}: an integer between 1 and 12, inclusive; 1 means January,
12 means December.

\item
\emph{Year}: an integer indicating the year A.D.  However, if this integer
is between 0 and 99, the ``obvious'' year is used; more precisely,
that year is assumed that is equal to the integer modulo 100 and
within fifty years of the current year (inclusive backwards
and exclusive forwards).  Thus, in the year 1978, year 28 is 1928
but year 27 is 2027.  (Functions that return time in this format always return
a full year number.)
\end{itemize}

\begin{itemize}
\item
\emph{Day-of-week}: an integer between 0 and 6, inclusive;
0 means Monday, 1 means Tuesday, and so on; 6 means Sunday.

\item
\emph{Daylight-saving-time-p}: a flag that, if not {\false}, indicates that
daylight saving time is in effect.

\item
\emph{Time-zone}: an integer specified as the number of hours west of GMT
(Greenwich Mean Time).  For example, in Massachusetts the time zone is 5,
and in California it is 8.  Any adjustment for daylight saving time is
separate from this.
\end{itemize}

\begin{newer}
X3J13 voted in March 1989 \issue{TIME-ZONE-NON-INTEGER}
to specify that the time zone part of Decoded Time need not be an integer,
but may be any rational number (either an integer or a ratio)
in the range -24 to 24 (inclusive on both ends)
that is an integral multiple of \cd{1/3600}.

\beforenoterule
\begin{rationale}
For all possible time designations to be accommodated, it is
    necessary to allow the time zone to be non-integral, for some places
    in the world have time standards offset from Greenwich Mean Time
    by a non-integral number of hours.

    There appears to be no user demand for floating-point time zones.  Since such
    zones would introduce inexact arithmetic, X3J13 did not consider
    adding them at this time.

This specification does require time zones to be represented as integral multiples
    of 1 second (rather than 1 hour).  This prevents problems that could otherwise
occur in converting Decoded Time to Universal Time.
\end{rationale}
\afternoterule
\end{newer}

Universal Time represents time as a single non-negative integer.
For relative time
purposes, this is a number of seconds.  For absolute time, this is the
number of seconds since midnight, January 1, 1900 {GMT}.  Thus the time 1
is 00:00:01 (that is, 12:00:01 A.M.) on January 1, 1900 {GMT}.
Similarly, the time 2398291201 corresponds to time 00:00:01 on January 1,
1976 {GMT}.
Recall that the year 1900 was \emph{not} a leap year; for the purposes of
Common Lisp, a year is a leap year if and only if its number is divisible by 4, except
that years divisible by 100 are \emph{not} leap years, except that years
divisible by 400 \emph{are} leap years.  Therefore the year 2000 will
be a leap year.
(Note that the ``leap seconds'' that
are sporadically inserted by the world's official timekeepers as an additional
correction are ignored; Common Lisp assumes that every day is exactly 86400
seconds long.)
Universal Time format is used as a standard time
representation within the {ARPANET}; see reference \cite{KLH-TIME-SERVER}.
Because the Common Lisp Universal Time representation uses only
non-negative integers, times before the base time of midnight,
January 1, 1900 {GMT} cannot be processed by Common Lisp.

Internal Time also represents time as a single integer, but
in terms of an implementation-dependent unit.
Relative time is measured as a number of these units.
Absolute time is relative to an arbitrary time base, typically
the time at which the system began running.

\begin{defun}[Function]
get-decoded-time 

The current time is returned in Decoded Time format.  Nine values
are returned: \emph{second}, \emph{minute}, \emph{hour}, \emph{date}, \emph{month},
\emph{year}, \emph{day-of-week}, \emph{daylight-saving-time-p}, and \emph{time-zone}.
\end{defun}

\begin{defun}[Function]
get-universal-time 

The current time of day is returned as a single integer
in Universal Time format.

Функция возвращает текущее время всемирное время в 
\end{defun}

\begin{defun}[Function]
decode-universal-time universal-time &optional time-zone

The time specified by \emph{universal-time} in Universal Time format
is converted to Decoded Time format.  Nine values
are returned: \emph{second}, \emph{minute}, \emph{hour}, \emph{date}, \emph{month},
\emph{year}, \emph{day-of-week}, \emph{daylight-saving-time-p}, and \emph{time-zone}.

The \emph{time-zone} argument defaults to the current time zone.

\cdf{decode-universal-time},
like \cdf{encode-universal-time}, ignores daylight saving time information
if a \emph{time-zone} is explicitly specified; in this case
the returned \emph{daylight-saving-time-p} value will necessarily be
\cdf{nil} even if daylight saving time happens to be in effect in that
time zone at the specified time.
\end{defun}

\begin{defun}[Function]
encode-universal-time second minute hour date month year &optional time-zone

The time specified by the given components of Decoded Time format is
encoded into Universal Time format and returned.  If you do not specify
\emph{time-zone}, it defaults to the current time zone adjusted for daylight
saving time.  If you provide \emph{time-zone} explicitly, no adjustment for
daylight saving time is performed.

Функция преобразует время, заданное компонентами формата декодированного
времени, в формат всемирного времени. Если вы не укажете часовой пояс
\emph{time-zone}, то он будет по-умолчанию равен текущему часовому поясу с
учетом перехода на летнее время. Если вы укажете явно часовой пояс
\emph{time-zon}, учет летнего времени производиться не будет.
\end{defun}

\begin{defun}[Constant]
internal-time-units-per-second

This value is an integer, the implementation-dependent
number of internal time units in a second.  (The internal time unit must
be chosen so that one second is an integral multiple of it.)

Значение константы является целым числом и зависит от того, сколько единиц
внутреннего времени в секунде для данной реализации Common Lisp'а.
(Единица внутреннего времени должна быть выбрана так, чтобы в секунде помещалось
целое число таких единиц.)

\beforenoterule
\begin{rationale}
The reason for allowing the internal time
units to be implementation-dependent is so that
\cdf{get-internal-run-time} and \cdf{get-internal-real-time}
can execute with minimum overhead.
The idea is that it should be very likely that a fixnum will suffice as the
returned value from these functions.  This probability can be
tuned to the implementation by trading off the speed of the machine
against the word size.  Any particular unit will
be inappropriate for some implementations: a microsecond is too long
for a very fast machine, while a much smaller unit would
force many implementations to return bignums for most calls
to \cdf{get-internal-time}, rendering that function less useful for accurate
timing measurements.
\end{rationale}
\afternoterule
\end{defun}

\begin{defun}[Function]
get-internal-run-time 

The current run time is returned as a single integer in Internal Time
format.
The precise meaning of this quantity is implementation-dependent;
it may measure real time, run time, CPU cycles, or some other quantity.
The intent is that the difference between the values of two calls
to this function be the amount of time between the two calls
during which computational effort was expended on behalf of the
executing program.

Функция возвращает целое число в формате Внутреннего Времени. 
Точное значение этой величины зависит от реализации и может измерятся в реальном
времени, времени выполнения, циклов центрального процессора, или в чем-либо
другом.
Суть в том, что разница между значениями двух вызовов будет количеством времени,
потраченным на исполнение программы между этими вызовами.
\end{defun}

\begin{defun}[Function]
get-internal-real-time 

The current time is returned as a single integer in Internal Time
format.  This time is relative to an arbitrary time base,
but the difference between the values of two calls
to this function will be the amount of elapsed real time between the two calls,
measured in the units defined by \cdf{internal-time-units-per-second}.

Функция возвращает целое число в формате Внутреннего Времени. Данное время
относительно некоторого базового значения, но разница между значениями двух
вызовов этой функции будет количеством прошедшего (между этими вызовами)
реального времени, выраженным в \cdf{internal-time-units-per-second}.
\end{defun}

\begin{defun}[Function]
sleep seconds

\cd{(sleep \emph{n})} causes execution to cease and become dormant for
approximately \emph{n} seconds of real time, whereupon execution is resumed.
The argument may be any non-negative non-complex number.
\cdf{sleep} returns {\nil}.

\cd{(sleep \emph{n})} приостанавливает исполнение примерно на \emph{n} секунд в
реальном времени.
Аргумент может быть любым неотрицательным некомплексным числом.
\cdf{sleep} возвращает {\nil}.
\end{defun}

\subsection{Other Environment Inquiries Справочные функции о среде}

For any of the following functions, if no appropriate
and relevant result can be produced, {\nil} is returned instead
of a string.

Любая из этих функций вместо строки может возвращать результат {\nil}, если
подходящей информации нет.

\beforenoterule
\begin{rationale}
These inquiry facilities are functions rather than variables
against the possibility that a Common Lisp process might migrate from
machine to machine.  This need not happen in a distributed
environment; consider, for example, dumping a core image file
containing a compiler and then shipping it to another site.
\end{rationale}
\afternoterule

\beforenoterule
\begin{rationale}
Эти справочные данные возвращаются функциями, а не хранятся переменных, так как
процесс Common Lisp'а может мигрировать между компьютерами (машинами).
Это необязтально случается в распределенной среде.
Например может служить сохранение образа содержащего компилятор и затем
восстановление данного образа на другой машине. 
\end{rationale}
\afternoterule

\begin{defun}[Function]
lisp-implementation-type 

A string is returned that identifies the generic name of
the particular Common Lisp implementation.
Examples: \cd{"Spice LISP"}, \cd{"Zetalisp"}, \cd{"SBCL"}.

Функция возвращают имя текущей реализации Common Lisp'а.
Примеры: \cd{"Spice LISP"}, \cd{"Zetalisp"}, \cd{"SBCL"}.
\end{defun}

\begin{defun}[Function]
lisp-implementation-version 

A string is returned that identifies the version of
the particular Common Lisp implementation; this information
should be of use to maintainers of the implementation.
Examples: \cd{"1192"}, \cd{"53.7 with complex numbers"},
\cd{"1746.9A, NEWIO 53, ETHER 5.3"}.

Функция возвращает версию текущей реализации Common Lisp'а.
Эта информация должно быть использована сопровождающими реализацию.
Примеры: \cd{"1192"}, \cd{"53.7 with complex numbers"},
\cd{"1746.9A, NEWIO 53, ETHER 5.3"}.
\end{defun}

\begin{defun}[Function]
machine-type 

A string is returned that identifies the generic name of
the computer hardware on which Common Lisp is running.
Examples: \cd{"IMLAC"}, \cd{"DEC PDP-10"}, \cd{"DEC VAX-11/780"}, \cd{"X86-64"}.

Функция возвращает имя типа аппаратного обеспечения, на котором запущен Common
Lisp.
Примеры: \cd{"IMLAC"}, \cd{"DEC PDP-10"}, \cd{"DEC VAX-11/780"}, \cd{"X86-64"}.
\end{defun}

\begin{defun}[Function]
machine-version 

A string is returned that identifies the version of
the computer hardware on which Common Lisp is running.
Example: \cd{"KL10, microcode 9"}, \cd{"AMD Athlon(tm) 64 X2 Dual Core Processor 3600+"}.

Функция возвращает версию аппаратного обеспечения, на котором запущен Common
Lisp.
Примеры: \cd{"KL10, microcode 9"}, \cd{"AMD Athlon(tm) 64 X2 Dual Core Processor 3600+"}.
\end{defun}

\begin{defun}[Function]
machine-instance 

A string is returned that identifies the particular
instance of the computer hardware on which Common Lisp is running;
this might be a local nickname, for example, or a serial number.
Examples: \cd{"MIT-MC"}, \cd{"CMU GP-VAX"}.

Функция возвращает имя компьютера, на котором запущена реализация Common Lisp'а.
Примеры: \cd{"MIT-MC"}, \cd{"CMU GP-VAX"}.
\end{defun}

\begin{defun}[Function]
software-type 

A string is returned that identifies the generic name of
any relevant supporting software.
Examples: \cd{"Spice"}, \cd{"TOPS-20"}, \cd{"ITS"}, \cd{Linux}.

Функция возвращает имя типа текущей операционной системы.
Примеры: \cd{"Spice"}, \cd{"TOPS-20"}, \cd{"ITS"}, \cd{Linux}.
\end{defun}

\begin{defun}[Function]
software-version 

A string is returned that identifies the version of
any relevant supporting software; this information
should be of use to maintainer of the implementation.

Функция возвращает версию текущей операционной системы. Данная информация должно
использоваться сопровождающими ОС.
Пример для ArchLinux'а: \cd{"3.2.13-1-ARCH"}.
\end{defun}

\begin{defun}[Function]
short-site-name  \\
long-site-name 

A string is returned that identifies the physical location
of the computer hardware.
Examples of short names: \cd{"MIT AI Lab"}, \cd{"CMU-CSD"}.
Examples of long names:
\begin{lisp}
"MIT Artificial Intelligence Laboratory" \\
"Massachusetts Institute of Technology \\
Artificial Intelligence Laboratory" \\
"Carnegie-Mellon University Computer Science Department"
\end{lisp}

Функции возвращают строки, обозначающие физическое расположение аппаратной части
компьютера.
Примеры для кратких имен: \cd{"MIT AI Lab"}, \cd{"CMU-CSD"}.
Примеры для длинных имен:
\begin{lisp}
"MIT Artificial Intelligence Laboratory" \\
"Massachusetts Institute of Technology \\
Artificial Intelligence Laboratory" \\
"Carnegie-Mellon University Computer Science Department"
\end{lisp}
\end{defun}

\noindent
See also \cdf{user-homedir-pathname}.

\noindent
Смотрите также \cd{user-homedir-pathname}.


\begin{defun}[Variable]
*features*

The value of the variable \cdf{*features*} should be a list of symbols
that name ``features'' provided by the implementation.
Most such names will be implementation-specific; typically
a name for the implementation will be included.

Значение переменной \cdf{*features} должно быть списком символов, которые
указывают на имена <<возможностей>> данной реализации.
Большинство этих имен специфичны. 

The value of this variable is used by the \cd{\#+} and \cd{\#-}
reader syntax.

Значение этой переменной используется с помощью синтаксических конструкций считывателя: \cd{\#+} и \cd{\#-}.

Feature names used with \cd{\#+} and \cd{\#-}
are read in the \cdf{keyword} package unless an explicit prefix
designating some other package appears.  The standard
feature name \cdf{ieee-floating-point} is therefore actually the
keyword \cd{:ieee-floating-point}, though one need not write the colon
when using it with \cd{\#+} or \cd{\#-}; thus \cd{\#+ieee-floating-point}
and \cd{\#+:ieee-floating-point} mean the same thing.

По-умолчанию символы для имен <<возможностей>>, использованные в конструкциях
\cd{\#+} и \cd{\#-}, ищутся в пакете
\cdf{keyword}. Таким образом \cd{\#+ieee-floating-point}
и \cd{\#+:ieee-floating-point} означают одно и то же.
\end{defun}


\section{Identity Function Функция идентичность (identity)}

This function is occasionally useful as an argument to
other functions that require functions as arguments.  (Got that?)

Эта функция иногда бывает полезна для использования в качестве аргумента других
функций, которые требуют функции в качестве аргументов. (Понял?)

\begin{defun}[Function]
identity object

The \emph{object} is returned as the value of \cdf{identity}.

The \cdf{identity} function is the default value for the \cd{:key}
argument to many sequence functions (see chapter~\ref{KSEQUE}).

Table~\ref{IDENTITY-PLOT} illustrates the behavior in the complex plane
of the
\cdf{identity} function regarded as a function of a complex numerical argument.

Many other constructs in Common Lisp have the behavior of \cdf{identity}
when given a single argument.  For example, one might well use \cdf{values}
in place of \cdf{identity}.  However, writing \cdf{values} of a single
argument conventionally indicates that the argument form might deliver
multiple values and that the intent is to pass on only the first of
those values.

Результатом функции является переданный объект \emph{object}.

Функция \cdf{identity} используется по-умолчанию для аргумента \cd{:key} для
большинства функций для последовательностей (смотрите главу~\ref{KSEQUE}).

Поведение функции \cdf{identity} для комплексного числа проиллюстрировано в
таблице~\ref{IDENTITY-PLOT}.

Множество других Common Lisp'овых конструкций с одним аргументом ведут себя так
же как и \emph{identity}. Например, можно использовать \cdf{values} вместо
\cdf{identity}. Однако, запись \cdf{values} с одним аргументом означает, что
форма аргумента возвращает несколько значений, но необходимо вернуть только одно
из них.
\end{defun}
        % Miscellaneous stuff
%%%Chapter of Common Lisp Manual.  Copyright 1989 Guy L. Steele Jr.
%%% Based on ANSI X3J13 document X3J13/89-004 which in turn is
%%% based on documentation by Lucid, Inc.

\clearpage\def\pagestatus{FINAL PROOF}

\chapterauthor{Jon L White}

\chapter{Loop}
\label{LOOP}

\begin{new}
\prefaceword
X3J13 voted in January 1989
\issue{LOOP-FACILITY}
to adopt an extended definition of the \cd{loop} macro
as a part of the forthcoming draft Common Lisp standard.
\end{new}
This chapter presents the bulk of the Common Lisp
Loop Facility proposal, written by Jon L White.  I have
edited it only very lightly
to conform to the overall style of this book and have inserted a small
number of bracketed remarks, identified by the initials GLS.
(See the Acknowledgments to this second edition for
acknowledgments to others who contributed to the Loop Facility proposal.)

\noindent\hbox to \textwidth{\hss---Guy L. Steele Jr.}
\vskip 8pt plus 3pt minus 2pt

\section{Introduction}

A {\it loop\/} is a series of expressions that are executed one or more times,
a process known as {\it iteration}.
The {\it Loop Facility\/} defines a
variety of useful methods, indicated by
{\it loop keywords}, to iterate and to
accumulate values in a loop.


Loop keywords are not true Common Lisp keywords; they are symbols that
are recognized by the Loop Facility and that provide such capabilities
as controlling the direction of iteration, accumulating values inside
the body of a loop, and evaluating expressions that precede or follow
the loop body.  If you do not use any loop keywords, the Loop Facility
simply executes the loop body repeatedly.


\section{How the Loop Facility Works}

The driving element of the Loop Facility is the \cd{loop} macro.
When Lisp encounters a \cd{loop} macro call
form, it invokes the Loop Facility and passes to it the loop clauses
as a list of unevaluated forms, as with any macro.
The loop clauses contain Common Lisp forms and loop keywords.  The
loop keywords are recognized by their symbol name, regardless of the
packages that contain them.  The \cd{loop} macro translates the
given form into Common Lisp code and returns the expanded form.

The expanded loop form is
one or more lambda-expressions for the local binding of loop variables
and a block and a tagbody that express a looping control structure.   
  The variables established in the loop construct are bound as
  if by using \cd{let} or \cd{lambda}.  Implementations can interleave the
  setting of initial values with the bindings.  However, the assignment
  of the initial values is always calculated in the order specified by
  the user.  A variable is thus sometimes bound to a harmless value of the
  correct data type, and then later in the prologue it is set to the true
  initial value by using \cd{setq}.

The expanded form consists of three basic parts in the tagbody:

\begin{itemize}
\item
The {\it loop prologue\/} contains forms that are executed before iteration begins, 
such as initial settings of loop variables and possibly an initial
termination test.

\item
The {\it loop body\/}  contains those forms that are executed during iteration, 
including application-specific calculations, termination tests,
and variable stepping.  {\it Stepping\/} is the process of assigning a
variable the next item in a series of items.

\item
The {\it loop epilogue} contains forms that are executed after iteration 
terminates,
such as code to return values from the loop.
\end{itemize}


Expansion of the \cd{loop} macro produces an implicit block 
(named \cd{nil}).
Thus, the Common Lisp macro \cd{return} and the special form 
\cd{return-from} can be 
used to return values from a loop or to exit a loop.

  Within the executable parts of loop clauses and around the entire
  loop form, you can still bind variables by using the Common Lisp
  special form \cd{let}.



  \section{Parsing Loop Clauses}

  The syntactic parts of a loop construct are called {\it clauses}; the scope
  of each clause is determined by the top-level parsing of that clause's
  keyword.  The following example shows a loop construct with six
  clauses:

\begin{lisp}
(loop for i from 1 to (compute-top-value)~~~~~~~~~;{\rm First clause} \\*
~~~~~~while (not (unacceptable i))~~~~~~~~~~~~~~~~;{\rm Second clause} \\*
~~~~~~collect (square i)~~~~~~~~~~~~~~~~~~~~~~~~~~;{\rm Third clause} \\*
~~~~~~do (format t "Working on {\Xtilde}D now" i)~~~~~~~~~;{\rm Fourth clause} \\*
~~~~~~when (evenp i)~~~~~~~~~~~~~~~~~~~~~~~~~~~~~~;{\rm Fifth clause} \\*
~~~~~~~~do (format t "{\Xtilde}D is a non-odd number" i) \\*
~~~~~~finally (format t "About to exit!"))~~~~~~~~;{\rm Sixth clause}
\end{lisp}


  Each loop keyword introduces either a compound loop clause or a simple
  loop clause that can consist of a loop keyword followed by a 
  single Lisp form.  The number of
  forms in a clause is determined by the loop keyword that begins the
  clause and by the auxiliary keywords in the clause.  The keywords
  \cd{do}, \cd{initially}, and \cd{finally} are the only loop
  keywords that can take any number of Lisp forms and group them as if
  in a single \cd{progn} form.

  Loop clauses can contain auxiliary keywords, which are sometimes
  called {\it prepositions}.  For example, the first clause in the preceding code
  includes the prepositions \cd{from} and \cd{to}, which mark
  the value from which stepping begins and the value at which stepping
  ends.

  \subsection{Order of Execution}

  With the exceptions listed below, clauses are executed in the loop body
  in the order in which they appear in the source.  Execution is repeated 
  until a clause
  terminates the loop or until a Common Lisp \cd{return},
  \cd{go}, or \cd{throw} form is encountered.  The following actions are
  exceptions to the linear order of execution:

  \begin{itemize}

  \item
  All variables are initialized first, regardless of where the establishing
  clauses appear in the source.  The order of initialization follows the
  order of these clauses.

  \item
  The code for any \cd{initially} clauses is collected
  into one \cd{progn} in the order in which the clauses appear in
  the source.  The collected code is executed once in the loop prologue
  after any implicit variable initializations.

  \item
  The code for any \cd{finally} clauses is collected 
  into one \cd{progn} in the order in which the clauses appear in
  the source.  The collected code is executed once in the loop epilogue
  before any implicit values from the accumulation clauses are returned.
  Explicit returns anywhere in the source, however, will exit the loop
  without executing the epilogue code.

  \item 
  A \cd{with} clause introduces a variable binding and an optional
  initial value.  The initial values are calculated in the order in
  which the \cd{with} clauses occur.

  \item 
  Iteration control clauses implicitly perform the following actions:
  \begin{itemize}
  \item
  initializing variables

  \item
  stepping variables, generally between each execution of the loop body

  \item
  performing termination tests, generally just before the execution of the
  loop body
  \end{itemize}
  \end{itemize}

  \subsection{Kinds of Loop Clauses}
\label{LOOP-KINDS-SECTION}

  Loop clauses fall into one of the following categories:

  \begin{itemize}

  \item 
  variable initialization and stepping

  \begin{itemize}
  \item
  The \cd{for} and \cd{as} constructs provide iteration control clauses
  that establish a variable to be initialized.
  You can combine \cd{for} and \cd{as} clauses with the loop
  keyword \cd{and} to get parallel initialization and stepping.

  \item
  The \cd{with} construct is similar to a single \cd{let} clause.
  You can combine \cd{with} clauses using
  \cd{and} to get parallel initialization.

  \item
  The \cd{repeat} construct causes iteration to terminate after a specified
   number of times.  It uses an internal variable to keep track of the
   number of iterations.
  \end{itemize}

  You can specify data types for loop variables (see
  section~\ref{LOOP-TYPES-SECTION}).
  It is an error to bind the same variable twice in any variable-binding
  clause of a single loop expression.  Such variables include
  local variables, iteration control variables, and variables found by
  destructuring.


  \item value accumulation

  \begin{itemize}
  \item The \cd{collect} construct takes one form in its clause
  and adds the value of that form to the end of a list of values.  By
  default, the list of values is returned when the loop finishes.

  \item
  The \cd{append} construct  takes one form in its clause
  and appends the value of that form to the end of a list of values.  By
  default, the list of values is returned when the loop finishes.

  \item The \cd{nconc} construct is similar to \cd{append}, but
  its list values are concatenated as if by the Common Lisp function
  \cd{nconc}.  By
  default, the list of values is returned when the loop finishes.

  \item The \cd{sum} construct takes one form in its clause that
  must evaluate to a number and adds that number into a running total.
  By default, the cumulative sum is returned when the loop finishes.

  \item 
  The \cd{count} construct takes one form in its clause and counts the
  number of times that the form evaluates to a non-\cd{nil} value.  By
  default, the count is returned when the loop finishes.

  \item
  The \cd{minimize} construct takes one form in its clause and determines
  the minimum value obtained by evaluating that form.  By default, the
  minimum value is returned when the loop finishes.

  \item
  The \cd{maximize} construct takes one form in its clause and 
  determines the maximum value obtained by evaluating that form.  By
  default, the maximum value is returned when the loop finishes.
  \end{itemize}

  \item
  termination conditions

  \begin{itemize}
  \item
  The \cd{loop-finish} Lisp macro terminates iteration and returns any
  accumulated result.  If specified, any \cd{finally} clauses are evaluated.

  \item
  The \cd{for} and \cd{as} constructs provide a termination test
  that is determined by the iteration control clause.

  \item
   The \cd{repeat} construct causes termination after a specified
  number of iterations.

  \item 
  The \cd{while} construct takes one form, a condition, and terminates
  the iteration if
  the condition evaluates to \cd{nil}.  A \cd{while} clause is
  equivalent to the expression \cd{(if~(not~{\it condition}) (loop-finish))}.  

  \item 
   The \cd{until} construct is the inverse of \cd{while};
   it terminates the iteration if the condition evaluates to any non-\cd{nil}
   value.  An \cd{until} clause is equivalent to the expression
  \cd{(if~{\it condition} (loop-finish))}.

  \item
  The \cd{always} construct takes one form and terminates the loop 
  if the form ever evaluates to \cd{nil}; in this case, it returns
  \cd{nil}.  Otherwise, it provides a default return value of \cd{t}.

  \item
  The \cd{never} construct takes one form and terminates the loop
  if the form ever evaluates to non-\cd{nil}; in this case, it returns
  \cd{nil}.  Otherwise, it provides a default return value of \cd{t}.

  \item
  The \cd{thereis} construct takes one form and terminates the loop
  if the form ever evaluates to non-\cd{nil}; in this case, it returns
  that value.
  \end{itemize}

  \item unconditional execution

  \begin{itemize}
  \item
  The \cd{do} construct simply evaluates all forms in its clause.

  \item
  The \cd{return} construct takes one form and returns its value.  It is 
  equivalent to the clause \cd{do (return {\it value})}.
  \end{itemize}

  \item conditional execution

  \begin{itemize}
  \item
  The \cd{if} construct takes one form as a predicate and a clause that 
  is executed when the predicate is true. The clause can be a value 
  accumulation, unconditional, or another conditional clause; it can also
  be any combination of such clauses connected by the loop keyword \cd{and}.

  \item
  The \cd{when} construct is a synonym for \cd{if}.

  \item
  The \cd{unless} construct is similar to \cd{when} except that it complements
  the predicate; it executes the following clause if the predicate is false.

  \item
  The \cd{else} construct provides an optional component of \cd{if},
  \cd{when}, and \cd{unless} clauses that is executed when the
  predicate is false.  The component is one of the clauses described under
  \cd{if}.

  \item
  The \cd{end} construct provides an optional component to mark the
  end of a conditional clause.
  \end{itemize}

  \item miscellaneous operations

  \begin{itemize}
  \item The \cd{named} construct assigns a name to a loop construct.

  \item The \cd{initially} construct causes its forms to be evaluated
   in the loop prologue, which precedes all loop code except for initial 
   settings specified by the constructs \cd{with}, \cd{for}, or \cd{as}.

  \item  The \cd{finally} construct causes its forms to be evaluated
   in the loop epilogue after normal iteration terminates.  An unconditional
   clause can also follow the loop keyword \cd{finally}.
  \end{itemize}
  \end{itemize}



  \subsection{Loop Syntax}

  The following syntax description provides an overview of the syntax
  for loop clauses.  Detailed syntax descriptions of individual clauses
  appear in sections~\ref{LOOP-ITERATION-SECTION} through~\ref{LOOP-MISC-SECTION}.
  A loop consists of the
  following types of clauses:

\begin{tabbing}
{\it initial-final\/} ::= {\it initially\/} {\Mor} {\it finally\/} \\*
{\it variables\/} ::= {\it with\/} {\Mor} {\it initial-final\/} {\Mor} {\it for-as\/} {\Mor} {\it repeat} \\*
{\it main\/} ::= {\it unconditional\/} {\Mor} {\it accumulation\/} {\Mor} {\it conditional\/} 
      {\Mor} {\it termination\/} {\Mor} {\it initial-final\/} \\*
{\it loop\/} ::= \cd{(loop \Mopt{\cd{named {\it name\/}}} \Mstar{\it variables\/} \Mstar{\it main\/})}
\end{tabbing}

Note that a loop must have at least one clause; however, for
backward compatibility, the following format is also supported:
\begin{lisp}
(loop \Mstar{{\it tag} {\Mor} {\it expr}})
\end{lisp}
where {\it expr} is any Common Lisp expression that can be evaluated, and 
{\it tag} is any symbol not identifiable as a loop keyword.  Such a format
is roughly equivalent to the following one:

\begin{lisp}
(loop do \Mstar{{\it tag} {\Mor} {\it expr}})
\end{lisp}

  A loop prologue consists of any automatic variable initializations prescribed 
  by the {\it variable\/} clauses, along with any {\it initially\/} clauses
  in the order they appear in the source.

  A loop epilogue consists of {\it finally\/} clauses, if any, along
  with any implicit return value from an {\it accumulation\/} clause or
  an {\it end-test\/} clause.


  \section{User Extensibility}

  There is currently no specified portable method for users to add
  extensions to the Loop Facility.  The names \cd{defloop} and
  \cd{define-loop-method} have been suggested as candidates for such a method.



\section{Loop Constructs}

The remaining sections of this chapter describe the constructs that the Loop Facility
provides.  The descriptions are organized according to the functionality
of the constructs.  Each section begins with a general discussion of
a particular operation; it then presents the constructs that perform the 
operation.

\begin{itemize}

\item Section~\ref{LOOP-ITERATION-SECTION},
``Iteration Control,'' describes iteration
control clauses that allow directed loop iteration.  

\item Section~\ref{LOOP-TEST-SECTION}, ``End-Test Control,'' 
describes clauses that stop iteration by providing a conditional expression
that can be tested after each execution of the loop body.  

\item Section~\ref{LOOP-ACCUM-SECTION},
``Value Accumulation,'' describes constructs
that accumulate values during iteration and return them from a loop.  This section also
discusses ways in which accumulation clauses can be combined within the
Loop Facility.  

\item Section~\ref{LOOP-VAR-SECTION},
``Variable Initializations,'' describes the \cd{with} 
construct, which provides local variables for use within the loop
body, and other constructs that provide local variables.

\item Section~\ref{LOOP-COND-SECTION},
``Conditional Execution,'' describes how to execute loop
clauses conditionally.

\item Section~\ref{LOOP-UNCOND-SECTION},
``Unconditional Execution,'' describes the \cd{do}
and \cd{return} constructs.  It also describes constructs that are
used in the loop prologue and loop epilogue.

\item Section~\ref{LOOP-MISC-SECTION},
``Miscellaneous Features,'' discusses loop data types
and destructuring.  It also presents constructs for naming a loop and
for specifying initial and final actions.
\end{itemize}


\section{Iteration Control}
\label{LOOP-ITERATION-SECTION}

Iteration control clauses allow you to direct loop iteration.  The
loop keywords \cd{as}, \cd{for}, and \cd{repeat} designate iteration control clauses.

Iteration control clauses differ with respect to the specification of
termination conditions and the initialization and stepping
of loop variables.  Iteration clauses by themselves
do not cause the Loop Facility to return values, but they
can be used in conjunction with value-accumulation clauses to
return values (see section~\ref{LOOP-ACCUM-SECTION}).

All variables are initialized in the loop prologue.  The scope of
the variable binding is {\it lexical} unless it is proclaimed
special; thus, the variable can be
accessed only by expressions that lie textually within the loop.
Stepping assignments are made in the loop body before any other expressions
are evaluated in the body.

The variable argument in iteration control clauses can be a 
{\it destructuring list}.  A destructuring list
is a tree whose non-null atoms are symbols that
can be assigned a value (see section~\ref{LOOP-DESTRUCTURING-SECTION}).

The iteration control clauses \cd{for}, \cd{as}, and \cd{repeat} 
must precede any other loop clauses except
\cd{initially}, \cd{with}, and \cd{named},
since they establish variable bindings.  When iteration control clauses are
used in a loop, termination tests in the loop body are evaluated
before any other loop body code is executed.

If you use multiple iteration clauses to control iteration, variable
initialization and stepping occur sequentially by default.  
You can use the \cd{and} construct to connect two or more
iteration clauses when sequential binding and stepping are not necessary.
The iteration behavior of clauses joined by \cd{and}
is analogous to the behavior of the Common Lisp macro \cd{do}
relative to \cd{do*}.

[X3J13 voted in March 1989 \issue{LOOP-AND-DISCREPANCY} to correct a minor
inconsistency in the original syntactic specification for \cd{loop}.  Only \cd{for}
and \cd{as} clauses (not \cd{repeat} clauses) may be joined by the \cd{and} construct. The
precise syntax is as follows.
\begin{tabbing}
{\it for-as\/} ::= \Mgroup{\cd{for} {\Mor} \cd{as}} {\it for-as-subclause\/} \Mstar{\cd{and} {\it for-as-subclause\/}} \\*
\pushtabs{\it for-as-subclause\/} ::= \={\it for-as-arithmetic\/} {\Mor} {\it for-as-in-list\/} \\*
\>\hbox to 0pt{\hss\Mor~}{\it for-as-on-list\/} {\Mor} {\it for-as-equals-then\/}  \\*
\>\hbox to 0pt{\hss\Mor~}{\it for-as-across\/} {\Mor} {\it for-as-hash\/} {\Mor} {\it for-as-package\/} \poptabs \\
\pushtabs{\it for-as-arithmetic\/} ::= \={\it var\/} \Mopt{type-spec} \Mopt{\Mgroup{\cd{from} {\Mor} \cd{downfrom} {\Mor} \cd{upfrom}} expr1} \\*
\>\Mopt{\Mgroup{\cd{to} {\Mor} \cd{downto} {\Mor} \cd{upto} {\Mor} \cd{below} {\Mor} \cd{above}} expr2} \\*
\>\Mopt{\cd{by} expr3} \poptabs \\
{\it for-as-in-list\/} ::= {\it var\/} \Mopt{type-spec} \cd{in} {\it expr1\/} \Mopt{\cd{by} step-fun} \\
{\it for-as-on-list\/} ::= {\it var\/} \Mopt{type-spec} \cd{on} {\it expr1\/} \Mopt{\cd{by} step-fun} \\
{\it for-as-equals-then\/} ::= {\it var\/} \Mopt{type-spec} \cd{=} {\it expr1\/} \Mopt{\cd{then} step-fun} \\
{\it for-as-across\/} ::= {\it var\/} \Mopt{type-spec} \cd{across} {\it vector\/} \\
\pushtabs{\it for-as-hash\/} ::= \={\it var\/} \Mopt{type-spec} \cd{being} \Mgroup{\cd{each} {\Mor} \cd{the}} \\*
\>\Mgroup{\cd{hash-key} {\Mor} \cd{hash-keys} {\Mor} \cd{hash-value} {\Mor} \cd{hash-values}} \\*
\>\Mgroup{\cd{in} {\Mor} \cd{of}} {\it hash-table\/} \\*
\>\Mopt{\cd{using} \cd{(}\Mgroup{\cd{hash-value} {\Mor} \cd{hash-key}} other-var\/\cd{)}} \poptabs \\
\pushtabs{\it for-as-package\/} ::= \={\it var\/} \Mopt{type-spec} \cd{being} \Mgroup{\cd{each} {\Mor} \cd{the}} \\*
\>{\it for-as-package-keyword\/} \\*
\>\Mgroup{\cd{in} {\Mor} \cd{of}} {\it package\/} \poptabs \\
\pushtabs{\it for-as-package-keyword\/} ::= \=\cd{symbol} {\Mor} \cd{present-symbol} {\Mor} \cd{external-symbol} \\*
\>\hbox to 0pt{\hss\Mor~}\cd{symbols} {\Mor} \cd{present-symbols} {\Mor} \cd{external-symbols} \poptabs
\end{tabbing}
This correction made \cd{for} and \cd{as} clauses syntactically
similar to \cd{with} clauses.  I have changed all examples in this
chapter to reflect the corrected syntax.---GLS]

In the following example, the variable \cd{x} is stepped
before \cd{y} is stepped; thus, the value of \cd{y}
reflects the updated value of \cd{x}:
\begin{lisp}
(loop for x from 1 to 9 \\*
~~~~~~for y = nil then x  \\*
~~~~~~collect (list x y)) \\*
~~~\EV~((1 NIL) (2 2) (3 3) (4 4) (5 5) (6 6) (7 7) (8 8) (9 9))
\end{lisp}

In the following example, \cd{x} and \cd{y} are stepped in parallel:
\begin{lisp}
(loop for x from 1 to 9 \\*
~~~~~~and y = nil then x \\*
~~~~~~collect (list x y)) \\*
~~~\EV~((1 NIL) (2 1) (3 2) (4 3) (5 4) (6 5) (7 6) (8 7) (9 8))
\end{lisp}

The \cd{for} and \cd{as} clauses iterate by using one or more local 
loop  variables that are initialized to some value and that 
can be modified or stepped after each iteration.  
For these clauses, iteration terminates when a local
variable reaches some specified value or when some other loop clause
terminates iteration.  At each iteration, variables can be stepped by an
increment or a decrement or can be assigned a new value by 
the evaluation of 
an expression.  Destructuring can be used to assign initial values to 
variables during iteration.

The \cd{for} and \cd{as} keywords are synonyms and may be used
interchangeably.  There are
seven syntactic representations for these constructs.
In each syntactic description, the data type of
{\it var\/} can be specified by the optional {\it type-spec\/}
argument.  If {\it var\/} is a destructuring list, the data type
specified by the {\it type-spec\/} argument must appropriately match
the elements of the list (see sections~\ref{LOOP-TYPES-SECTION}
and~\ref{LOOP-DESTRUCTURING-SECTION}).

\begin{defloop}
for var [type-spec] [{\!from! | \!downfrom! | \!upfrom!} expr1]
                    [{\!to! | \!downto! | \!upto! | \!below! | \!above!} expr2]
                    [\!by! expr3] \\
as var [type-spec] [{\!from! | \!downfrom! | \!upfrom!} expr1]
                   [{\!to! | \!downto! | \!upto! | \!below! | \!above!} expr2]
                   [\!by! expr3]

[This is the first of seven \cd{for}/\cd{as} syntaxes.---GLS]

The \cd{for} or \cd{as} construct iterates from the value specified by
{\it expr1\/} to the value specified by {\it expr2\/} in increments or
decrements denoted by {\it expr3}. Each
expression is evaluated only once and must evaluate to a number.  

The variable {\it var\/} is bound to the value of 
{\it expr1\/} in the first iteration and is stepped
by the value of {\it expr3\/} in each succeeding iteration,
or by 1 if {\it expr3\/} is not provided.  

The following loop keywords serve as valid prepositions within this 
syntax.

\begin{flushdesc}
\item[\cd{from}]
The loop keyword \cd{from} marks the value from which
stepping begins, as specified by {\it expr1}.  
Stepping is incremental by default.  For
decremental stepping, use \cd{above}
or \cd{downto} with {\it expr2}.  For incremental
stepping, the default \cd{from} value is \cd{0}.

\item[\cd{downfrom}, \cd{upfrom}]
The loop keyword \cd{downfrom} 
indicates that the variable {\it var\/} is decreased in decrements
specified by {\it expr3}; the loop keyword \cd{upfrom} indicates that 
{\it var\/} is increased in increments specified by {\it expr3}.

\item[\cd{to}]
The loop keyword \cd{to} marks the end value for stepping specified in 
{\it expr2}. Stepping is incremental by default.  For
decremental stepping, use \cd{downto},
\cd{downfrom}, or \cd{above} with {\it expr2}.

\item[\cd{downto}, \cd{upto}]
The loop keyword \cd{downto} allows iteration to proceed
from a larger number to a smaller number by the decrement 
{\it expr3}.  The loop keyword \cd{upto} allows iteration to proceed
from a smaller number to a larger number by the increment {\it expr3}.
Since there is no default for {\it expr1\/} in decremental stepping,
you must supply a value with \cd{downto}.

\item[\cd{below}, \cd{above}]
The loop keywords \cd{below} and \cd{above} are analogous to
\cd{upto} and \cd{downto}, respectively.  These keywords stop
iteration just before the value of the variable {\it var} reaches the value 
specified by {\it expr2\/}; the end value of {\it expr2\/} is not included.
Since there is no default for {\it expr1\/} in decremental stepping,
you must supply a value with \cd{above}.

\item[\cd{by}]
The loop keyword \cd{by} marks the increment or decrement specified by
{\it expr3}.  The value of {\it expr3\/} can be any positive number.
The default value is \cd{1}.
\end{flushdesc}

At least one of these prepositions must be used with this syntax.

In an iteration control clause, the \cd{for} or \cd{as} construct
causes termination when the specified limit is reached.  That is,
iteration continues until the value {\it var\/} is stepped to the
exclusive or inclusive limit specified by {\it expr2\/}.  The range is
{\it exclusive\/} if {\it expr3\/} increases or decreases {\it var\/}
to the value of {\it expr2\/} without reaching that value; the loop
keywords \cd{below} and \cd{above} provide exclusive limits.  An
{\it inclusive\/} limit allows {\it var\/} to attain the value of
{\it expr2}; \cd{to}, \cd{downto}, and \cd{upto} provide inclusive
limits.

A common convention is to use \cd{for} to introduce new iterations and \cd{as}
to introduce iterations that depend on a previous iteration specification.
[However, \cd{loop} does not enforce this convention, and some of the examples
below violate it.  {\it De gustibus non disputandum est.}---GLS]

Examples:
\begin{lisp}
;;; Print some numbers. \\[3pt]
(loop as i from 1 to 5 \\*
~~~~~~do (print i)) \`;{\rm Prints 5 lines} \\
1 \\*
2 \\*
3 \\*
4 \\*
5 \\*
~~~\EV~NIL \\
 \\
;;; Print every third number. \\[3pt]
(loop for i from 10 downto 1 by 3 \\*
~~~~~~do (print i)) \`;{\rm Prints 4 lines}\\
10  \\*
7  \\*
4  \\*
1  \\*
~~~\EV~NIL
\end{lisp}

\begin{lisp}
;;; Step incrementally from the default starting value. \\[3pt]
(loop as i below 5 \\*
~~~~~~do (print i)) \`;{\rm Prints 5 lines} \\
0 \\*
1 \\*
2 \\*
3 \\*
4 \\*
~~~\EV~NIL
\end{lisp}
\end{defloop}

\begin{defloop}
for var [type-spec] \!in! expr1 [\!by! step-fun] \\
as var [type-spec] \!in! expr1 [\!by! step-fun]

[This is the second of seven \cd{for}/\cd{as} syntaxes.---GLS]

This construct iterates over the contents of a list.  It checks for 
the end of the list as if using the Common Lisp function \cd{endp}.  
The variable {\it var\/} is bound to the successive elements  of 
the list {\it expr1\/} before each
iteration.  At the end of each iteration, the function {\it step-fun\/}
is called on the list and is expected to produce a successor list;
the default value for {\it step-fun\/} is the \cd{cdr} function.

The \cd{for} or \cd{as} construct causes termination when the
end of the list is reached.
The loop keywords \cd{in} and \cd{by} serve as valid prepositions in
this syntax.

Examples:
\begin{lisp}
;;; Print every item in a list. \\[3pt]
(loop for item in '(1 2 3 4 5) do (print item)) \`;{\rm Prints 5 lines} \\
1 \\*
2 \\*
3 \\*
4 \\*
5 \\*
~~~\EV~NIL \\
 \\
;;; Print every other item in a list. \\[3pt]
(loop for item in '(1 2 3 4 5) by \#'cddr \\*
~~~~~~do (print item))  \`;{\rm Prints 3 lines} \\
1 \\*
3 \\*
5 \\*
~~~\EV~NIL
\end{lisp}
\begin{lisp}
;;; Destructure items of a list, and sum the x values \\*
;;; using fixnum arithmetic. \\*
(loop for (item . x) (t . fixnum) \\*
~~~~~~~~~~in '((A . 1) (B . 2) (C . 3)) \\*
~~~~~~unless (eq item 'B) sum x) \\*
~~~\EV~4
\end{lisp}
\end{defloop}

\begin{defloop}
for var [type-spec] \!on! expr1 [\!by! step-fun] \\
as var [type-spec] \!on! expr1 [\!by! step-fun]

[This is the third of seven \cd{for}/\cd{as} syntaxes.---GLS]

This construct iterates over the contents of a list. It checks for the
end of the list as if using the Common Lisp function 
\cd{endp}.  
The variable {\it var\/} is bound to the successive tails of the list
{\it expr1}.  At the end of each iteration, the function {\it step-fun\/}
is called on the list and is expected to produce a successor list;
the default value for {\it step-fun\/} is the \cd{cdr} function.

The loop keywords \cd{on} and \cd{by} serve as valid
prepositions in this syntax.
The \cd{for} or \cd{as} construct causes termination when the
end of the list is reached.

Examples:
\begin{lisp}
;;; Collect successive tails of a list. \\*
(loop for sublist on '(a b c d) \\*
~~~~~~collect sublist) \\*
~~~\EV~((A B C D) (B C D) (C D) (D)) \\
 \\
;;; Print a list by using destructuring with the loop keyword ON. \\*
(loop for (item) on '(1 2 3) \\*
~~~~~~do (print item))  \`;{\rm Prints 3 lines}\\
1  \\*
2  \\*
3  \\*
~~~\EV~NIL \\
 \\
;;; Print items in a list without using destructuring. \\*
(loop for item in '(1 2 3) \\*
~~~~~~do (print item))  \`;{\rm Prints 3 lines}\\
1  \\*
2  \\*
3  \\*
~~~\EV~NIL
\end{lisp}
\end{defloop}

\begin{defloop}
for var [type-spec] \!\Xequal! expr1 [\!then! expr2] \\
as var [type-spec] \!\Xequal! expr1 [\!then! expr2]

[This is the fourth of seven \cd{for}/\cd{as} syntaxes.---GLS]


  This construct initializes the variable {\it var\/} by setting it to the
  result of evaluating {\it expr1\/} on the first iteration, then setting
  it to the result of evaluating {\it expr2\/} on the second and
  subsequent iterations.  If {\it expr2\/} is omitted, the construct
  uses {\it expr1\/} on the second and
  subsequent iterations.  When {\it expr2\/} is omitted, the expanded
  code shows the following optimization:

\begin{lisp}
;;; Sample original code: \\*
(loop for x = {\it expr1\/} then {\it expr2\/} do (print x))
\end{lisp}
\begin{lisp}
;;; The usual expansion: \\*
(tagbody \\*
~~~~~~(setq x {\it expr1\/}) \\*
~~tag (print x) \\*
~~~~~~(setq x {\it expr2\/}) \\*
~~~~~~(go tag))
\end{lisp}
\begin{lisp}
;;; The optimized expansion: \\*
(tagbody \\*
~~tag (setq x {\it expr1\/}) \\*
~~~~~~(print x) \\*
~~~~~~(go tag))
\end{lisp}

The loop keywords \cd{=} and  \cd{then} serve as valid prepositions
in this syntax.
This construct does not provide any termination conditions.

Example:
\begin{lisp}
;;; Collect some numbers. \\*
(loop for item = 1 then (+ item 10) \\*
~~~~~~repeat 5 \\*
~~~~~~collect item) \\*
~~~\EV~(1 11 21 31 41)
\end{lisp}
\end{defloop}


\begin{defloop}
for var [type-spec] \!across! vector \\
as var [type-spec] \!across! vector

[This is the fifth of seven \cd{for}/\cd{as} syntaxes.---GLS]

    This construct binds the variable {\it var\/} to
    the value of each element in the array {\it vector}.

  The loop keyword \cd{across} marks the array {\it vector}; \cd{across}
  is used as a preposition in this syntax.
  Iteration stops when there are no more elements in the specified
  array that can be referenced.

  Some implementations might use a [user-supplied---GLS] \cd{the} special form
  in the {\it vector} form to produce more efficient code.

  Example:
\begin{lisp}
(loop for char across (the simple-string (find-message port)) \\*
~~~~~~do (write-char char stream))
\end{lisp}
\end{defloop}

\begin{defloop}
for var [type-spec] \!being! {\!each! | \!the!}
                    {\!hash-key! | \!hash-keys! | \!hash-value! | \!hash-values!}
                    {\!in! | \!of!} hash-table [\!using! ({\!hash-value! | \!hash-key!} other-var)] \\
as var [type-spec] \!being! {\!each! | \!the!}
                    {\!hash-key! | \!hash-keys! | \!hash-value! | \!hash-values!}
                    {\!in! | \!of!} hash-table [\!using! ({\!hash-value! | \!hash-key!} other-var)]

[This is the sixth of seven \cd{for}/\cd{as} syntaxes.---GLS]

This construct iterates over the elements, keys, and values of a hash
table.  The variable {\it var\/} takes on the value of each hash key
or hash value in the specified hash table. 

The following loop keywords serve as valid prepositions within this syntax.

\begin{flushdesc}
\item[\cd{being}]
The keyword \cd{being} marks the loop method to be used, either 
\cd{hash-key} or \cd{hash-value}.

\item[\cd{each}, \cd{the}]
For purposes of readability, the loop keyword \cd{each}
should follow the loop keyword \cd{being} when \cd{hash-key} or
\cd{hash-value} is used.  The loop keyword \cd{the} is used with
\cd{hash-keys} and \cd{hash-values}.

\item[\cd{hash-key}, \cd{hash-keys}]
These loop keywords access each key entry of the hash table.  If
the name \cd{hash-value} is specified in a \cd{using} construct with one
of these loop methods, the iteration can optionally access the keyed
value. The order in which the keys are accessed is undefined; empty
slots in the hash table are ignored.

\item[\cd{hash-value}, \cd{hash-values}]
These loop keywords access each value entry of a hash table.  If
the name \cd{hash-key} is specified in a \cd{using} construct with one of
these loop methods, the iteration can optionally access the key that
corresponds to the value.  The order in which the keys are accessed is
undefined; empty slots in the hash table are ignored.

\item[\cd{using}]
The loop keyword \cd{using} marks the optional key or the keyed value to
be accessed.  It allows you to access the hash key if
iterating over the hash values, and the hash value if
iterating over the hash keys.

\item[\cd{in}, \cd{of}]
These loop prepositions mark the hash table {\it hash-table}.
\end{flushdesc}

Iteration stops when there are no more hash keys or hash values to be
referenced in the specified hash table.
\end{defloop}


\begin{defloop}
for var [type-spec] \!being! {\!each! | \!the!}
                    {\!symbol! | \!present-symbol! | \!external-symbol! |
                     \!symbols! | \!present-symbols! | \!external-symbols!}
                    {\!in! | \!of!} package \\
as var [type-spec] \!being! {\!each! | \!the!}
                    {\!symbol! | \!present-symbol! | \!external-symbol! |
                     \!symbols! | \!present-symbols! | \!external-symbols!}
                    {\!in! | \!of!} package

[This is the last of seven \cd{for}/\cd{as} syntaxes.---GLS]


This construct iterates over the symbols in a package.
The variable {\it var\/} takes on the value of each symbol
in the specified package.  

The following loop keywords serve as valid prepositions within this syntax.

\begin{flushdesc}

\item[\cd{being}]
The keyword \cd{being} marks the loop method to be used:
\cd{symbol}, \cd{present-\discretionary{}{}{}symbol},  or \cd{external-symbol}.

\item[\cd{each}, \cd{the}]
For purposes of readability, the loop keyword \cd{each}
should follow the loop keyword \cd{being} when \cd{symbol}, 
\cd{present-symbol}, or \cd{external-symbol} is used.  The loop keyword
\cd{the} is used with \cd{symbols}, \cd{present-symbols}, and 
\cd{external-symbols}.

\item[\cd{present-symbol}, \cd{present-symbols}]
These loop methods iterate over the symbols that are present but not
external in a package.
The package to be iterated over is
specified in the same way that package arguments to the Common Lisp function
\cd{find-package} are specified.  If you do not specify the package 
for the iteration, the current package is used.  If you specify a 
package that does not exist, an error is signaled.

\item[\cd{symbol}, \cd{symbols}]
These loop methods iterate over symbols that are
accessible from a given package.  The package to be iterated over is specified
in the same way that package arguments to the Common Lisp function
\cd{find-package} are specified.  If you do not specify the package 
for the iteration, the current package is used.  If you specify a 
package that does not exist, an error is signaled.

\item[\cd{external-symbol}, \cd{external-symbols}]
These loop methods iterate over the external symbols of a package.
The package to be iterated over is specified in
the same way that package arguments to the Common Lisp function
\cd{find-package} are specified.  If you do not specify the package 
for the iteration, the current package is used.  If you specify a 
package that does not exist, an error is signaled.

\item[\cd{in}, \cd{of}]
These loop prepositions mark the package {\it package}.
\end{flushdesc}

Iteration stops when there are no more symbols to be referenced in the
specified package.

Example:
\begin{lisp}
(loop for x being each present-symbol of "COMMON-LISP-USER"  \\*
~~~~~~do (print x)) \`;{\rm Prints 7 lines in this example}\\*
COMMON-LISP-USER::IN  \\*
COMMON-LISP-USER::X  \\
COMMON-LISP-USER::ALWAYS  \\
COMMON-LISP-USER::FOO  \\
COMMON-LISP-USER::Y  \\
COMMON-LISP-USER::FOR  \\*
COMMON-LISP-USER::LUCID  \\*
~~~\EV~NIL
\end{lisp}
\end{defloop}



\begin{defloop}
repeat expr

The \cd{repeat} construct causes iteration to terminate after a
specified number of times.
The loop body is executed {\it n} times, where {\it n} is the value 
of the expression {\it expr}.  The {\it expr} argument is evaluated one time
in the loop prologue.  If the expression evaluates to zero or 
to a negative number, the loop body is not evaluated.


  The clause \cd{repeat} {\it n} is roughly equivalent to a clause
  such as 
\begin{lisp}
for {\it internal-variable} downfrom (- {\it n} 1) to 0
\end{lisp}
but, in some implementations, the \cd{repeat} construct might 
   be more efficient.


Examples:
\begin{lisp}
(loop repeat 3 \`;{\rm Prints 3 lines}\\*
~~~~~~do (format t "What I say three times is true{\Xtilde}\%")) \\*
What I say three times is true \\*
What I say three times is true \\*
What I say three times is true \\*
~~~\EV~NIL \\
 \\
(loop repeat -15 \`;{\rm Prints nothing}\\*
~~~~~~do (format t "What you see is what you expect{\Xtilde}\%")) \\*
~~~\EV~NIL
\end{lisp}
\end{defloop}



\section{End-Test Control}
\label{LOOP-TEST-SECTION}

The loop keywords \cd{always}, \cd{never}, \cd{thereis},
\cd{until}, and \cd{while} designate constructs that use a single test 
condition to determine when loop iteration should terminate.

The constructs \cd{always}, \cd{never}, and \cd{thereis} provide
specific values to be returned when a loop terminates.  
Using \cd{always}, \cd{never}, or \cd{thereis} with 
value-returning accumulation clauses can produce unpredictable results.
In all other respects these
constructs behave like the \cd{while} and \cd{until} constructs.

The macro \cd{loop-finish} can be used at any time to cause normal
termination.  In normal termination, \cd{finally} clauses are 
executed and default return values are returned.

End-test control constructs can be used anywhere within the loop
body.  The termination conditions are tested in the order in which
they appear.

\begin{defloop}
while expr \\
until expr

The \cd{while} construct allows iteration to continue until the specified
expression {\it expr} evaluates to \cd{nil}.  The expression
is re-evaluated at the location of the \cd{while} clause.

The \cd{until} construct is equivalent to 
{\cd{while} (\cd{not} {\it expr})}.  If the value of the
specified expression is non-\cd{nil}, iteration terminates.

You can use \cd{while} and \cd{until} 
at any point in a loop.  If a \cd{while} or \cd{until} clause causes
termination, any clauses that precede it in the source
are still evaluated.  

Examples:
\begin{lisp}
;;; A classic "while-loop". \\
(loop while (hungry-p) do (eat)) \\
 \\
;;; UNTIL NOT is equivalent to WHILE. \\*
(loop until (not (hungry-p)) do (eat)) \\
 \\
;;; Collect the length and the items of STACK. \\*
(let ((stack '(a b c d e f))) \\*
~~(loop while stack \\*
~~~~~~~~for item = (length stack) then (pop stack) \\*
~~~~~~~~collect item)) \\*
~~~\EV~(6 A B C D E F) \\
 \\
;;; Use WHILE to terminate a loop that otherwise wouldn't \\*
;;; terminate.  Note that WHILE occurs after the WHEN. \\*
(loop for i fixnum from 3 \\*
~~~~~~when (oddp i) collect i \\*
~~~~~~while (< i 5)) \\*
~~~\EV~(3 5)
\end{lisp}
\end{defloop}


\begin{defloop}
always expr \\
never expr \\
thereis expr

The \cd{always} construct takes one form and terminates the loop 
  if the form ever evaluates to \cd{nil}; in this case, it returns
  \cd{nil}.  Otherwise, it provides a default return value of \cd{t}.

  The \cd{never} construct takes one form and terminates the loop
  if the form ever evaluates to non-\cd{nil}; in this case, it returns
  \cd{nil}.  Otherwise, it provides a default return value of \cd{t}.

  The \cd{thereis} construct takes one form and terminates the loop
  if the form ever evaluates to non-\cd{nil}; in this case, it returns
  that value.

If the \cd{while} or \cd{until} construct causes termination,
control is passed to the loop epilogue, where any \cd{finally}
clauses will be executed.  Since \cd{always}, \cd{never}, and 
\cd{thereis} use the Common Lisp macro \cd{return} to terminate
iteration, any \cd{finally} clause that is specified is not
evaluated.

Examples:
\begin{lisp}
;;; Make sure I is always less than 11 (two ways). \\*
;;; The FOR construct terminates these loops.
\end{lisp}
\begin{lisp}
(loop for i from 0 to 10 \\*
~~~~~~always (< i 11)) \\*
~~~\EV~T
\end{lisp}
\begin{lisp}
(loop for i from 0 to 10 \\*
~~~~~~never (> i 11)) \\*
~~~\EV~T
\end{lisp}
\begin{lisp}
;;; If I exceeds 10, return I; otherwise, return NIL. \\*
;;; The THEREIS construct terminates this loop.
\end{lisp}
\begin{lisp}
(loop for i from 0 \\*
~~~~~~thereis (when (> i 10) i) ) \\*
~~~\EV~11
\end{lisp}
\begin{lisp}
;;; The FINALLY clause is not evaluated in these examples.
\end{lisp}
\begin{lisp}
(loop for i from 0 to 10 \\*
~~~~~~always (< i 9) \\*
~~~~~~finally (print "you won't see this")) \\*
~~~\EV~NIL
\end{lisp}
\begin{lisp}
(loop never t \\*
~~~~~~finally (print "you won't see this")) \\*
~~~\EV~NIL
\end{lisp}
\begin{lisp}
(loop thereis "Here is my value" \\*
~~~~~~finally (print "you won't see this")) \\*
~~~\EV~"Here is my value"
\end{lisp}
\begin{lisp}
;;; The FOR construct terminates this loop, \\*
;;; so the FINALLY clause is evaluated.
\end{lisp}
\begin{lisp}
(loop for i from 1 to 10 \\*
~~~~~~thereis (> i 11) \\*
~~~~~~finally (print i)) \`;{\rm Prints 1 line}\\*
11 \\*
~~~\EV~NIL
\end{lisp}
\goodbreak
\begin{lisp}
(defstruct mountain height difficulty (why "because it is there")) \\*
(setq everest (make-mountain :height '(2.86e-13 parsecs))) \\*
(setq chocorua (make-mountain :height '(1059180001 microns))) \\*
(defstruct desert area (humidity 0)) \\*
(setq sahara (make-desert :area '(212480000 square furlongs))) \\
\`;{\rm First there is a mountain, then there is no mountain, then there is $\ldots$} \\
(loop for x in (list everest sahara chocorua) \`;~{\rm ---GLS} \\*
~~~~~~thereis (and (mountain-p x) (mountain-height x))) \\*
~~~\EV~(2.86E-13 PARSECS) \\
 \\
;;; If you could use this code to find a counterexample to \\*
;;; Fermat's last theorem, it would still not return the value \\*
;;; of the counterexample because all of the THEREIS clauses \\*
;;; in this example return only T.~~ Of course, this code has \\*
;;; never been observed to terminate. \\*
 \\*
(loop for z upfrom 2 \\*
~~~~~~thereis \\*
~~~~~~~~(loop for n upfrom 3 below (log z 2) \\*
~~~~~~~~~~~~~~thereis \\*
~~~~~~~~~~~~~~~~(loop for x below z \\*
~~~~~~~~~~~~~~~~~~~~~~thereis \\*
~~~~~~~~~~~~~~~~~~~~~~~~(loop for y below z \\*
~~~~~~~~~~~~~~~~~~~~~~~~~~~~~~thereis (= (+ (expt x n) \\*
~~~~~~~~~~~~~~~~~~~~~~~~~~~~~~~~~~~~~~~~~~~~(expt y n)) \\*
~~~~~~~~~~~~~~~~~~~~~~~~~~~~~~~~~~~~~~~~~(expt z n))))))
\end{lisp}
\end{defloop}

\begin{defmac}
loop-finish \!!

The macro \cd{loop-finish} terminates iteration normally
and returns any accumulated result.  If specified, a \cd{finally} clause
is evaluated.

In most cases it is not necessary to use \cd{loop-finish}
because other loop control clauses terminate the loop.  
Use \cd{loop-finish} to provide a normal exit
from a nested condition inside a loop.

You can use \cd{loop-finish}
inside nested Lisp code to provide a normal exit from a loop.
Since \cd{loop-finish} transfers control to the loop epilogue,
using \cd{loop-finish} within a \cd{finally} expression can cause
infinite looping.


  Implementations are allowed to provide this construct as a local macro
  by using \cd{macrolet}.

Examples:
\begin{lisp}
;;; Print a date in February, but exclude leap day. \\*
;;; LOOP-FINISH exits from the nested condition. \\*
(loop for date in date-list \\*
~~~~~~do (case date \\*
~~~~~~~~~~~(29 (when (eq month 'february) \\*
~~~~~~~~~~~~~~~~~~~~~(loop-finish)) \\*
~~~~~~~~~~~~~(format t "{\Xtilde}:@({\Xtilde}A{\Xtilde}) {\Xtilde}A" month date)))) \\
 \\
;;; Terminate the loop, but return the accumulated count. \\*
(loop for i in '(1 2 3 stop-here 4 5 6) \\*
~~~~~~when (symbolp i) do (loop-finish) \\*
~~~~~~count i) \\*
~~~\EV~3 \\
 \\
;;; This loop works just as well as the previous example. \\*
(loop for i in '(1 2 3 stop-here 4 5 6) \\*
~~~~~~until (symbolp i) \\*
~~~~~~count i) \\*
~~~\EV~3
\end{lisp}
\end{defmac}

\section{Value Accumulation}
\label{LOOP-ACCUM-SECTION}

Accumulating values during iteration and returning them from a loop
is often useful.  Some of these accumulations occur so
frequently that special loop clauses have been developed to handle them.

The loop keywords \cd{append}, \cd{appending},
\cd{collect}, \cd{collecting},
\cd{nconc}, and \cd{nconcing}
designate clauses that
accumulate values in lists and return them.

The loop keywords \cd{count}, \cd{counting},
\cd{maximize}, \cd{maximizing},
\cd{minimize}, \cd{minimizing},
\cd{sum}, and \cd{summing}
designate clauses that accumulate and
return numerical values.
[There is no semantic difference between the ``ing'' keywords and their non-``ing''
counterparts.  They are provided purely for the sake of stylistic diversity among users.
I happen to prefer the non-``ing'' forms---when I use \cd{loop} at all.---GLS]

The loop preposition \cd{into} can be used to name the 
variable used to hold partial accumulations.
The variable is bound as if by the loop
construct \cd{with} (see section~\ref{LOOP-VAR-SECTION}).  If 
\cd{into} is used, the construct does not provide a default return value;
however, the variable is available
for use in any \cd{finally} clause.

You can combine value-returning accumulation clauses in a loop if
all the clauses accumulate the same type of data object.  
By default, the Loop Facility returns only one value;
thus, the data objects collected by multiple accumulation clauses 
as return values must have compatible types. For example, since both
the \cd{collect} and \cd{append} constructs accumulate objects into a
list that is returned from a loop, you can combine them safely.


\begin{lisp}
;;; Collect every name and the kids in one list by using \\*
;;; COLLECT and APPEND. \\*
(loop for name in '(fred sue alice joe june) \\*
~~~~~~for kids in '((bob ken) () () (kris sunshine) ()) \\*
~~~~~~collect name \\*
~~~~~~append kids) \\*
~~~\EV~(FRED BOB KEN SUE ALICE JOE KRIS SUNSHINE JUNE)
\end{lisp}
[In the preceding example, note that the items accumulated by the
\cd{collect} and \cd{append} clauses are interleaved in the result list, according to
the order in which the clauses were executed.---GLS]

Multiple clauses that do not accumulate the same type of data object 
can coexist in a loop only if each clause accumulates its values into 
a different user-specified variable.  Any number of values can
be returned from a loop if you use the Common Lisp function \cd{values},
as the next example shows:
\begin{lisp}
;;; Count and collect names and ages. \\*
(loop for name in '(fred sue alice joe june) \\*
~~~~~~as age in '(22 26 19 20 10) \\*
~~~~~~append (list name age) into name-and-age-list \\*
~~~~~~count name into name-count \\*
~~~~~~sum age into total-age \\*
~~~~~~finally \\*
~~~~~~~~(return (values (round total-age name-count) \\*
~~~~~~~~~~~~~~~~~~~~~~~~name-and-age-list))) \\*
~~~\EV~19 {\rm and} (FRED 22 SUE 26 ALICE 19 JOE 20 JUNE 10)
\end{lisp}

\begin{defloop}
collect expr [\!into! var] \\
collecting expr [\!into! var]

During each iteration, these constructs collect the value of the specified 
expression into a list. When iteration terminates, the list is returned.

The argument {\it var\/} is 
set to the list of collected values; if {\it var} is specified, the loop
does not return the final list automatically.  If {\it var} is not
specified, it is equivalent to specifying an internal name for
{\it var} and returning its value in a \cd{finally} clause.
The {\it var\/} argument
is bound as if by the construct \cd{with}.
You cannot specify a data type for {\it var\/}; it must be of type \cd{list}.


Examples:
\begin{lisp}
;;; Collect all the symbols in a list. \\*
(loop for i in '(bird 3 4 turtle (1 . 4) horse cat) \\*
~~~~~~when (symbolp i) collect i) \\*
~~~\EV~(BIRD TURTLE HORSE CAT) \\
 \\
;;; Collect and return odd numbers. \\*
(loop for i from 1 to 10 \\*
~~~~~~if (oddp i) collect i) \\*
~~~\EV~(1 3 5 7 9) \\
 \\
;;; Collect items into local variable, but don't return them. \\*
(loop for i in '(a b c d) by \#'cddr \\*
~~~~~~collect i into my-list \\*
~~~~~~finally (print my-list)) \`;{\rm Prints 1 line}\\*
(A C)  \\*
~~~\EV~NIL
\end{lisp}
\end{defloop}

\begin{defloop}
append expr [\!into! var] \\
appending expr [\!into! var] \\
nconc expr [\!into! var] \\
nconcing expr [\!into! var]

These constructs are similar to \cd{collect} except that the
values of the specified expression must be lists.  

The \cd{append} keyword causes its list values to be concatenated 
into a single list, as if 
they were arguments to the Common Lisp function \cd{append}.

The \cd{nconc} keyword causes its list values to be concatenated
into a single list,
as if they were arguments to the Common Lisp function \cd{nconc}.  
Note that the \cd{nconc} keyword destructively modifies its argument lists.

The argument {\it var\/} is 
set to the list of concatenated values; if you specify {\it var}, the loop
does not return the final list automatically.  The {\it var\/} argument
is bound as if by the construct \cd{with}.
You cannot specify a data type for {\it var\/}; it must be of type \cd{list}.


Examples:
\begin{lisp}
;;; Use APPEND to concatenate some sublists. \\*
(loop for x in '((a) (b) ((c))) \\*
~~~~~~append x) \\*
~~~\EV~(A B (C))
\end{lisp}
\begin{lisp}
;;; NCONC some sublists together.  Note that only lists \\*
;;; made by the call to LIST are modified. \\*
(loop for i upfrom 0  \\*
~~~~~~as x in '(a b (c)) \\*
~~~~~~nconc (if (evenp i) (list x) '())) \\*
~~~\EV~(A (C))
\end{lisp}
\end{defloop}


\begin{defloop}
count expr [\!into! var] [type-spec] \\
counting expr [\!into! var] [type-spec]

The \cd{count} construct counts the number of times that the specified 
expression has a non-\cd{nil} value.

The argument {\it var\/} accumulates the number of occurrences; if 
{\it var} is specified, the loop
does not return the final count automatically.  The {\it var\/} argument
is bound as if by the construct \cd{with}.

If \cd{into} {\it var\/} is used, the optional
{\it type-spec\/} argument specifies a data type for {\it var\/}.
If there is no \cd{into} variable, the optional {\it type-spec\/}
argument applies to the internal variable that is keeping the count.
In either case it is an error to specify a non-numeric 
data type.
The default type is implementation-dependent, but it must be a subtype
of \cd{(or~integer float)}.

Example:
\begin{lisp}
(loop for i in '(a b nil c nil d e) \\*
~~~~~~count i) \\*
~~~\EV~5
\end{lisp}
\end{defloop}


\begin{defloop}
sum expr [\!into! var] [type-spec] \\
summing expr [\!into! var] [type-spec]

The \cd{sum} construct forms a cumulative sum of the values of the
specified expression at each iteration.

The argument {\it var\/} is used to accumulate
the sum; if {\it var} is specified, the loop
does not return the final sum automatically.  The {\it var\/} argument
is bound as if by the construct \cd{with}.

If \cd{into} {\it var\/} is used, the optional
{\it type-spec\/} argument specifies a data type for {\it var\/}.
If there is no \cd{into} variable, the optional {\it type-spec\/}
argument applies to the internal variable that is keeping the sum.
In either case it is an error to specify a non-numeric 
data type.
The default type is implementation-dependent, but it must be a subtype
of \cd{number}.

Examples:
\begin{lisp}
;;; Sum the elements of a list. \\*
\\*
(loop for i fixnum in '(1 2 3 4 5) \\*
~~~~~~sum i) \\*
~~~\EV~15 \\
\\
;;; Sum a function of elements of a list. \\*
\\*
(setq series \\
~~~~~~'(1.2 4.3 5.7)) \\*
~~~\EV~(1.2 4.3 5.7) \\
\\
(loop for v in series  \\*
~~~~~~sum (* 2.0 v)) \\*
~~~\EV~22.4
\end{lisp}
\end{defloop}

\begin{defloop}
maximize expr [\!into! var] [type-spec] \\
maximizing expr [\!into! var] [type-spec] \\
minimize expr [\!into! var] [type-spec] \\
minimizing expr [\!into! var] [type-spec]

The \cd{maximize} construct compares the value of the specified expression
obtained during the first iteration with values obtained in successive
iterations. The maximum value encountered is determined and returned.  If the
loop never executes the body, the returned value is not meaningful.

The \cd{minimize} construct is similar to \cd{maximize}; it
determines and returns the minimum value.

The argument {\it var\/} accumulates the maximum or
minimum value; if {\it var} is specified, the loop
does not return the maximum or minimum automatically.  The {\it var\/} argument
is bound as if by the construct \cd{with}.

If \cd{into} {\it var\/} is used, the optional
{\it type-spec\/} argument specifies a data type for {\it var\/}.
If there is no \cd{into} variable, the optional {\it type-spec\/}
argument applies to the internal variable that is keeping the intermediate result.
In either case it is an error to specify a non-numeric 
data type.
The default type is implementation-dependent, but it must be a subtype
of \cd{(or integer float)}.

Examples:
\begin{lisp}
(loop for i in '(2 1 5 3 4) \\*
~~~~~~maximize i) \\*
~~~\EV~5
\end{lisp}
\penalty-10000 % required
\begin{lisp}
(loop for i in '(2 1 5 3 4) \\*
~~~~~~minimize i) \\*
~~~\EV~1 \\
 \\
;;; In this example, FIXNUM applies to the internal \\*
;;; variable that holds the maximum value. \\*
\\
(setq series '(1.2 4.3 5.7)) \\*
~~~\EV~(1.2 4.3 5.7) \\
\\
(loop for v in series  \\*
~~~~~~maximize (round v) fixnum) \\*
~~~\EV~6 \\
 \\
;;; In this example, FIXNUM applies to the variable RESULT. \\
\\
(loop for v float in series \\*
~~~~~~minimize (round v) into result fixnum \\*
~~~~~~finally (return result)) \\*
~~~\EV~1
\end{lisp}
\end{defloop}



\section{Variable Initializations}
\label{LOOP-VAR-SECTION}

A local loop variable is one that exists only when the Loop Facility
is invoked.  At that time, the variables are declared and are
initialized to some value.  These local variables exist until loop
iteration terminates, at which point they cease to exist.  Implicitly
variables are also established by iteration control clauses and the
\cd{into} preposition of accumulation clauses.


The loop keyword \cd{with} designates a loop clause that allows you to 
declare and initialize variables
that are local to a loop.  The variables are initialized one time
only; they can be initialized sequentially or in parallel.

By default, the \cd{with} construct initializes variables
sequentially; that is, one variable is assigned a value before the
next expression is evaluated.  However, by using the loop keyword 
\cd{and} to join several \cd{with} clauses, you can force
initializations to occur in parallel; that is, all of the specified
expressions are evaluated, and the results are bound to the respective
variables simultaneously.

Use sequential binding for making the initialization of
some variables depend on the values of previously bound variables.
For example, suppose you want to bind the variables \cd{a}, \cd{b},
and \cd{c} in sequence:
\begin{lisp}
(loop with a = 1  \\*
~~~~~~with b = (+ a 2)  \\*
~~~~~~with c = (+ b 3) \\*
~~~~~~with d = (+ c 4) \\*
~~~~~~return (list a b c d)) \\*
~~~\EV~(1 3 6 10)
\end{lisp}

The execution of the preceding loop is equivalent to the execution of
the following code:
\begin{lisp}
(let* ((a 1) \\*
~~~~~~~(b (+ a 2)) \\*
~~~~~~~(c (+ b 3)) \\*
~~~~~~~(d (+ c 4))) \\
~~(block nil \\*
~~~~(tagbody \\*
~~~~~~next-loop (return (list a b c d)) \\*
~~~~~~~~~~~~~~~~(go next-loop) \\*
~~~~~~end-loop)))
\end{lisp}


If you are not depending on the value of previously bound variables
for the initialization of other local variables, you can use
parallel bindings as follows:
\begin{lisp}
(loop with a = 1  \\*
~~~~~~~and b = 2  \\*
~~~~~~~and c = 3 \\*
~~~~~~~and d = 4 \\*
~~~~~~return (list a b c d)) \\*
~~~\EV~(1 2 3 4)
\end{lisp}

The execution of the preceding loop is equivalent to the execution of
the following code:

\begin{lisp}
(let ((a 1) \\*
~~~~~~(b 2) \\*
~~~~~~(c 3) \\*
~~~~~~(d 4)) \\
~~(block nil \\*
~~~~(tagbody \\*
~~~~~~next-loop (return (list a b c)) \\*
~~~~~~~~~~~~~~~~(go next-loop) \\*
~~~~~~end-loop)))
\end{lisp}

\begin{defloop}
with var [type-spec] [\!\Xequal! expr] {\!and! var [type-spec] [\!\Xequal! expr]}*

The \cd{with} construct initializes variables that are local to 
a loop.  The variables are initialized one time only.

If the optional {\it type-spec\/} argument is specified for any variable 
{\it var\/}, but there is no related expression {\it expr} to be evaluated, {\it var\/}
is initialized to an appropriate default value for its data type.
For example, for the data types \cd{t}, \cd{number}, and \cd{float},
the default values are \cd{nil}, \cd{0}, and \cd{0.0}, respectively.
It is an error to specify a {\it type-spec\/} argument for {\it var\/} if
the related expression returns a value that is not of the specified type.
The optional \cd{and} clause forces parallel rather than sequential 
initializations.


Examples:
\begin{lisp}
;;; These bindings occur in sequence. \\*
(loop with a = 1  \\*
~~~~~~with b = (+ a 2)  \\*
~~~~~~with c = (+ b 3) \\*
~~~~~~with d = (+ c 4) \\*
~~~~~~return (list a b c d)) \\*
~~~\EV~(1 3 6 10) \\
 \\
;;; These bindings occur in parallel. \\*
(setq a 5 b 10 c 1729) \\*
(loop with a = 1 \\*
~~~~~~~and b = (+ a 2) \\*
~~~~~~~and c = (+ b 3) \\*
~~~~~~~and d = (+ c 4) \\*
~~~~~~return (list a b c d)) \\*
~~~\EV~(1 7 13 1733) \\
 \\
;;; This example shows a shorthand way to declare \\*
;;; local variables that are of different types. \\*
(loop with (a b c) (float integer float) \\*
~~~~~~return (format nil "{\Xtilde}A {\Xtilde}A {\Xtilde}A" a b c)) \\*
~~~\EV~"0.0 0 0.0" \\
 \\
;;; This example shows a shorthand way to declare \\*
;;; local variables that are of the same type. \\*
(loop with (a b c) float  \\*
~~~~~~return (format nil "{\Xtilde}A {\Xtilde}A {\Xtilde}A" a b c)) \\*
~~~\EV~"0.0 0.0 0.0"
\end{lisp}
\end{defloop}


\section{Conditional Execution}
\label{LOOP-COND-SECTION}

The loop keywords \cd{if}, \cd{when}, and \cd{unless} designate constructs that 
are useful when you want some loop clauses to operate under a specified
condition.

If the specified condition is true, the succeeding loop clause
is executed.  If the specified condition is not true, the succeeding clause is
skipped, and program control moves to the clause that follows the loop
keyword \cd{else}.  If the specified condition is not true and no
\cd{else} clause is specified, the entire conditional construct
is skipped.  Several clauses can be connected into
one compound clause with the loop keyword \cd{and}.
The end of the conditional clause can be marked with the keyword \cd{end}.

\begin{defloop}
if expr clause {\!and! clause}*
   [\!else! clause {\!and! clause}*] [\!end!] \\
when expr clause {\!and! clause}*
     [\!else! clause {\!and! clause}*] [\!end!] \\
unless expr clause {\!and! clause}*
       [\!else! clause {\!and! clause}*] [\!end!]

The constructs \cd{when} and \cd{if} allow conditional execution of
loop clauses.  These constructs are synonyms and
can be used interchangeably.  [Compare this to the {\it macro} \cd{when},
which does not allow an ``else'' part.---GLS]

If the value of the test expression {\it expr\/} is non-\cd{nil}, the expression
{\it clause1\/} is evaluated. If the test expression evaluates to \cd{nil}
and an \cd{else} construct is specified, the statements that follow the
\cd{else} are evaluated; otherwise, control passes to the next clause.

The \cd{unless} construct is equivalent to \cd{when} (\cd{not} 
{\it expr\/}) and \cd{if} (\cd{not} {\it expr\/}).
If the value of the test expression {\it expr\/} is \cd{nil}, the expression
{\it clause1\/} is evaluated. If the test expression evaluates to 
non-\cd{nil}
and an \cd{else} construct is specified, the statements that follow the
\cd{else} are evaluated; otherwise, control passes to the next clause.
[Compare this to the {\it macro} \cd{unless},
which does not allow an ``else'' part---or do I mean a ``then'' part?!  Ugh.
To prevent confusion, I strongly recommend as a matter of style
that \cd{else} not be used with \cd{unless} loop clauses.---GLS]

The {\it clause\/} arguments must be either accumulation, unconditional,
or conditional clauses (see section~\ref{LOOP-KINDS-SECTION}).
Clauses that follow the test expression can be grouped by using the 
loop keyword \cd{and} to produce a compound 
clause.

The loop keyword \cd{it} can be used to refer to the result of
the test expression in a clause.  If multiple clauses are connected with \cd{and},
the \cd{it} construct must be used in the first
clause in the block.  Since \cd{it} is a loop keyword, \cd{it} may not be used
as a local variable within a loop.

If \cd{when} or \cd{if} clauses are nested, each \cd{else} is
paired with the closest preceding \cd{when} or \cd{if} construct that has
no associated \cd{else}.

The optional loop keyword \cd{end} marks the end of the clause.  If this
keyword is not specified, the next loop keyword marks the end.  You can use
\cd{end} to distinguish the scoping of compound clauses.
\begin{lisp}
;;; Group conditional clauses into a block. \\*
(loop for i in numbers-list \\*
~~~~~~when (oddp i) \\*
~~~~~~~~do (print i) \\*
~~~~~~~~and collect i into odd-numbers \\*
~~~~~~~~and do (terpri) \\
~~~~~~else~~~~~;{\rm \cd{I} is even} \\*
~~~~~~~~collect i into even-numbers \\*
~~~~~~finally \\*
~~~~~~~~(return (values odd-numbers even-numbers)))
\end{lisp}
\begin{lisp}
;;; Collect numbers larger than 3. \\*
(loop for i in '(1 2 3 4 5 6) \\*
~~~~~~when (and (> i 3) i) \\*
~~~~~~collect it)~~~~~;{\rm \cd{it} refers to \cd{(and (> i 3) i)}} \\*
~~~\EV~(4 5 6)
\end{lisp}
\begin{lisp}
;;; Find a number in a list. \\*
(loop for i in '(1 2 3 4 5 6) \\*
~~~~~~when (and (> i 3) i) \\*
~~~~~~return it) \\*
~~~\EV~4
\end{lisp}
\begin{lisp}
;;; The preceding example is similar to the following one. \\*
(loop for i in '(1 2 3 4 5 6) \\*
~~~~~~thereis (and (> i 3) i)) \\*
~~~\EV~4
\end{lisp}
\begin{lisp}
;;; An example of using UNLESS with ELSE (yuk).\`{\rm ---GLS} \\*
(loop for turtle in teenage-mutant-ninja-turtles do\\*
~~(loop for x in '(joker brainiac shredder krazy-kat) \\*
~~~~~~~~unless (evil x) \\*
~~~~~~~~~~do (eat (make-pizza :anchovies t)) \\*
~~~~~~~~else unless (and (eq x 'shredder) (attacking-p x))\\*
~~~~~~~~~~~~~~~do (cut turtle slack)\,;{\rm When the evil Shredder attacks,} \\*
~~~~~~~~~~~~~else (fight turtle x)))\,;\,{\rm those turtle boys don't cut no slack}
\end{lisp}
\goodbreak
\begin{lisp}
;;; Nest conditional clauses. \\*
(loop for i in list \\*
~~~~~~when (numberp i) \\*
~~~~~~~~when (bignump i) \\*
~~~~~~~~~~collect i into big-numbers \\*
~~~~~~~~else~~~~~;{\rm Not \cd{(bignump i)}} \\*
~~~~~~~~~~collect i into other-numbers \\
~~~~~~else~~~~~;{\rm Not \cd{(numberp i)}} \\*
~~~~~~~~when (symbolp i)  \\*
~~~~~~~~~~collect i into symbol-list \\*
~~~~~~~~else~~~~~;{\rm Not \cd{(symbolp i)}} \\*
~~~~~~~~~~(error "found a funny value in list {\Xtilde}S, value {\Xtilde}S{\Xtilde}\%" \\*
~~~~~~~~~~~~~~~~"list i)) \\
 \\
;;; Without the END marker, the last AND would apply to the \\*
;;; inner IF rather than the outer one. \\*
(loop for x from 0 to 3  \\*
~~~~~~do (print x) \\*
~~~~~~if (zerop (mod x 2)) \\*
~~~~~~~~do (princ " a") \\*
~~~~~~~~~~and if (zerop (floor x 2)) \\*
~~~~~~~~~~~~~~~~do (princ " b") \\*
~~~~~~~~~~~~~~end \\*
~~~~~~~~~~and do (princ " c"))
\end{lisp}
\end{defloop}


\section{Unconditional Execution}
\label{LOOP-UNCOND-SECTION}


The loop construct \cd{do} (or \cd{doing}) takes one or more expressions
and simply evaluates them in order.

The loop construct \cd{return} takes one expression and returns its value.  It 
is equivalent to the clause \cd{do~(return {\it value\/})}.

\begin{defloop}
do {expr}* \\
doing {expr}*

The \cd{do} construct simply evaluates the specified expressions
wherever they occur in the expanded form of \cd{loop}.

The {\it expr\/} argument can be any non-atomic Common Lisp form.
Each {\it expr\/} is evaluated in every iteration.

The constructs \cd{do}, \cd{initially}, and \cd{finally} are the
only loop keywords that take an arbitrary number of forms and group
them as if using an implicit \cd{progn}.  
Because every loop clause must begin with a loop keyword, you would use
the keyword \cd{do} when no control action other than execution is 
required.

Examples:
\begin{lisp}
;;; Print some numbers. \\*
(loop for i from 1 to 5 \\*
~~~~~~do (print i)) \`;{\rm Prints 5 lines} \\*
1 \\*
2 \\*
3 \\*
4 \\*
5 \\*
~~~\EV~NIL \\
 \\
;;; Print numbers and their squares. \\*
;;; The DO construct applies to multiple forms. \\*
(loop for i from 1 to 4 \\*
~~~~~~do (print i) \\*
~~~~~~~~~(print (* i i))) \`;{\rm Prints 8 lines} \\
1  \\*
1  \\*
2  \\*
4  \\*
3  \\*
9  \\*
4  \\*
16  \\*
~~~\EV~NIL
\end{lisp}
\end{defloop}


\begin{defloop}
return expr

The \cd{return} construct terminates a 
loop and returns the value of 
the specified expression as the value of the loop.   This construct
is similar to the Common Lisp special form \cd{return-from} and the
Common Lisp macro \cd{return}.

The Loop Facility supports the \cd{return} construct for backward
compatibility with older \cd{loop} implementations.  
The \cd{return} construct returns immediately and does not execute
any \cd{finally} clause that is given.

Examples:
\begin{lisp}
;;; Signal an exceptional condition. \\*
(loop for item in '(1 2 3 a 4 5) \\*
~~~~~~when (not (numberp item)) \\*
~~~~~~return (cerror "enter new value" \\*
~~~~~~~~~~~~~~~~~~~~~"non-numeric value: {\Xtilde}s" \\*
~~~~~~~~~~~~~~~~~~~~~item)) \`;{\rm Signals an error} \\*
>>Error: non-numeric value: A \\
 \\
;;; The previous example is equivalent to the following one. \\*
(loop for item in '(1 2 3 a 4 5) \\*
~~~~~when (not (numberp item)) \\*
~~~~~~do (return  \\*
~~~~~~~~~~~(cerror "enter new value" \\*
~~~~~~~~~~~~~~~~~~~"non-numeric value: {\Xtilde}s" \\*
~~~~~~~~~~~~~~~~~~~item)))  \`;{\rm Signals an error} \\*
>>Error: non-numeric value: A
\end{lisp}
\end{defloop}


\section{Miscellaneous Features}
\label{LOOP-MISC-SECTION}

The Loop Facility provides the \cd{named} construct to name a loop so that
the Common Lisp special form \cd{return-from} can be used.

The loop keywords \cd{initially} and \cd{finally} designate loop constructs that cause
expressions to be evaluated before and after the loop body, respectively.


  The code for any \cd{initially} clauses is collected
  into one \cd{progn} in the order in which the clauses appeared in
  the loop.  The collected code is executed once in the loop prologue
  after any implicit variable initializations.

  The code for any \cd{finally} clauses is collected 
  into one \cd{progn} in the order in which the clauses appeared in
  the loop.  The collected code is executed once in the loop epilogue
  before any implicit values are returned from the accumulation clauses.
  Explicit returns in the loop body, however, will exit the loop
  without executing the epilogue code.

\subsection{Data Types}
\label{LOOP-TYPES-SECTION}

Many loop constructs take a {\it type-spec\/} argument that
allows you to specify  certain data types for loop variables.
While it is not necessary to specify a data type for any variable,
by doing so you ensure that the variable has a correctly typed initial
value.  The type declaration is made available to the compiler for
more efficient \cd{loop} 
expansion. 
In some implementations,
fixnum and float declarations are especially
useful; the compiler notices them and emits more efficient code.  



The {\it type-spec\/} argument has the following syntax:
\begin{tabbing}
{\it type-spec\/} ::= \cd{of-type} {\it d-type-spec} \\
{\it d-type-spec\/} ::= {\it type-specifier\/} {\Mor} \cd{({\it d-type-spec} . {\it d-type-spec})}
\end{tabbing}
A {\it type-specifier} in this syntax can be any Common Lisp type
specifier.  The {\it d-type-spec} argument is used for destructuring,
as described in section~\ref{LOOP-DESTRUCTURING-SECTION}.  If the
{\it d-type-spec} argument consists solely of the types \cd{fixnum},
\cd{float}, \cd{t}, or \nil, the \cd{of-type} keyword is optional.  The
\cd{of-type} construct is optional in these cases to provide backward
compatibility; thus the following two expressions are the same:

\begin{lisp}
;;; This expression uses the old syntax for type specifiers. \\*
(loop for i fixnum upfrom 3 ...) \\
 \\
;;; This expression uses the new syntax for type specifiers. \\*
(loop for i of-type fixnum upfrom 3 ...)
\end{lisp}

\subsection{Destructuring}
\label{LOOP-DESTRUCTURING-SECTION}

Destructuring allows you to bind a set of variables to a corresponding
set of values anywhere that you can normally bind a value to a single
variable.  During \cd{loop} expansion, each variable in the variable list
is matched with the values in the values list.  If there are more variables
in the variable list than there are values in the values list, the 
remaining variables are given a value of \cd{nil}.  If there are more
values than variables listed, the extra values are discarded.


Suppose you want to assign values from a list to the variables \cd{a},
\cd{b}, and \cd{c}.  You could use one \cd{for} clause to
bind the variable \cd{numlist} to the {\it car} of the specified expression,
and then you could use another \cd{for} clause to bind the variables
\cd{a}, \cd{b}, and \cd{c} sequentially.  
\begin{lisp}
;;; Collect values by using FOR constructs. \\*
(loop for numlist in '((1 2 4.0) (5 6 8.3) (8 9 10.4)) \\*
~~~~~~for a integer = (first numlist) \\*
~~~~~~and for b integer = (second numlist) \\*
~~~~~~and for c float = (third numlist) \\*
~~~~~~collect (list c b a)) \\*
~~~\EV~((4.0 2 1) (8.3 6 5) (10.4 9 8))
\end{lisp}
Destructuring makes this process easier by allowing the variables to
be bound in parallel in each loop iteration.  You can declare data
types by using a list of {\it type-spec\/} arguments.  If all the types
are the same, you can use a shorthand destructuring syntax, as the second
example following illustrates.
\begin{lisp}
;;; Destructuring simplifies the process. \\*
(loop for (a b c) (integer integer float) in \\*
~~~~~~'((1 2 4.0) (5 6 8.3) (8 9 10.4)) \\*
~~~~~~collect (list c b a))) \\*
~~~\EV~((4.0 2 1) (8.3 6 5) (10.4 9 8)) \\
 \\
;;; If all the types are the same, this way is even simpler. \\*
(loop for (a b c) float in \\*
~~~~~~'((1.0 2.0 4.0) (5.0 6.0 8.3) (8.0 9.0 10.4)) \\*
~~~~~~collect (list c b a)) \\*
~~~\EV~((4.0 2.0 1.0) (8.3 6.0 5.0) (10.4 9.0 8.0))
\end{lisp}


If you use destructuring to declare or initialize a number of groups
of variables into types, you can use the loop keyword \cd{and}
to simplify the process further.
\begin{lisp}
;;; Initialize and declare variables in parallel \\*
;;; by using the AND construct. \\*
(loop with (a b) float = '(1.0 2.0) \\*
~~~~~~and (c d) integer = '(3 4) \\*
~~~~~~and (e f) \\*
~~~~~~return (list a b c d e f)) \\*
~~~\EV~(1.0 2.0 3 4 NIL NIL)
\end{lisp}

  A data type specifier for a destructuring pattern is a tree of type
  specifiers with the same shape as the tree of variables, with the
  following exceptions:

  \begin{itemize}

  \item  When aligning the trees, an atom in the type specifier tree that
  matches a cons in the variable tree declares the same type for each
  variable.

  \item  A cons in the type specifier tree that matches an atom in the
  variable tree is a non-atomic type specifer.

  \end{itemize}
   
\begin{lisp}
;;; Declare X and Y to be of type VECTOR and FIXNUM, respectively. \\*
(loop for (x y) of-type (vector fixnum) in my-list do ...)
\end{lisp}



If \cd{nil} is used in a destructuring list, no variable is provided for
its place.
\begin{lisp}
(loop for (a nil b) = '(1 2 3) \\*
~~~~~~do (return (list a b))) \\*
~~~\EV~(1 3)
\end{lisp}

Note that nonstandard lists can specify destructuring.
\begin{lisp}
(loop for (x . y) = '(1 . 2) \\*
~~~~~~do (return y)) \\*
~~~\EV~2 \\
 \\
(loop for ((a . b) (c . d)) \\*
~~~~~~~~~~of-type ((float . float) (integer . integer)) \\*
~~~~~~~~~~in '(((1.2 . 2.4) (3 . 4)) ((3.4 . 4.6) (5 . 6))) \\*
~~~~~~collect (list a b c d)) \\*
~~~\EV~((1.2 2.4 3 4) (3.4 4.6 5 6))
\end{lisp}

[It is worth noting that the destructuring facility of \cd{loop} predates,
and differs in some details from, that
of \cd{destructuring-bind}, an extension that has been provided by many implementors
of Common Lisp.---GLS]

\begin{defloop}
initially {expr}* \\
finally [\!do! | \!doing!] {expr}* \\
finally \!return! expr

The \cd{initially} construct causes the specified expression to be evaluated
in the loop prologue, which precedes all loop code except for 
initial settings specified by constructs \cd{with}, \cd{for}, or
\cd{as}.
The \cd{finally} construct causes the specified expression to be evaluated
in the loop epilogue after normal iteration terminates.

The {\it expr\/} argument can be any non-atomic Common Lisp form.

Clauses such as \cd{return}, \cd{always}, \cd{never}, and \cd{thereis}
can bypass the \cd{finally} clause.

The Common Lisp macro \cd{return} (or the \cd{return} loop construct) can be used
after \cd{finally} to return
values from a loop.  The evaluation of the \cd{return} form inside the
\cd{finally} clause takes precedence over returning the accumulation
from clauses specified by such keywords as \cd{collect}, \cd{nconc}, 
\cd{append}, \cd{sum}, \cd{count}, \cd{maximize}, and \cd{minimize}; 
the accumulation values for these pre-empted clauses are not returned by 
the loop if \cd{return} is used.

The constructs \cd{do}, \cd{initially}, and \cd{finally} are the
only loop keywords that take an arbitrary number of (non-atomic) forms and group
them as if by using an implicit \cd{progn}.  

\goodbreak

Examples:
\begin{lisp}
;;; This example parses a simple printed string representation  \\*
;;; from BUFFER (which is itself a string) and returns the \\*
;;; index of the closing double-quote character. \\*
\\*
(loop initially (unless (char= (char buffer 0) \#{\Xbackslash}") \\*
~~~~~~~~~~~~~~~~~~(loop-finish)) \\*
~~~~~~for i fixnum from 1 below (string-length buffer) \\*
~~~~~~when (char= (char buffer i) \#{\Xbackslash}") \\*
~~~~~~~~return i) \\
 \\
;;; The FINALLY clause prints the last value of I. \\*
;;; The collected value is returned. \\*
\\*
(loop for i from 1 to 10 \\*
~~~~~~when (> i 5) \\*
~~~~~~~~collect i \\*
~~~~~~finally (print i)) \`;{\rm Prints 1 line}\\*
11 \\*
~~~\EV~(6 7 8 9 10) \\
 \\
;;; Return both the count of collected numbers \\*
;;; as well as the numbers themselves. \\*
\\*
(loop for i from 1 to 10 \\*
~~~~~~when (> i 5) \\*
~~~~~~~~collect i into number-list \\*
~~~~~~~~and count i into number-count \\*
~~~~~~finally (return (values number-count number-list))) \\*
~~~\EV~5 {\rm and} (6 7 8 9 10)
\end{lisp}
\end{defloop}

\begin{defloop}
named name

The \cd{named} construct allows you to assign a name to a \cd{loop}
construct so that you can use the Common Lisp special form 
\cd{return-from} to exit the named loop.

Only one name may be assigned per loop; the specified name becomes the
name of the implicit block for the loop.

If used, the \cd{named}
construct must be the first clause in the loop expression, coming right after the
word \cd{loop}.

Example:
\begin{lisp}
;;; Just name and return. \\*
(loop named max \\*
~~~~~~for i from 1 to 10 \\*
~~~~~~do (print i) \\*
~~~~~~do (return-from max 'done)) \`;{\rm Prints 1 line}\\*
1  \\*
~~~\EV~DONE
\end{lisp}
\end{defloop}

        % LOOP
%%%Chapter of Common Lisp Manual.  Copyright 1989 Guy L. Steele Jr.

%  +++  Final version of chapter  +++

\clearpage\def\pagestatus{FINAL PROOF}

\chapterauthor{Richard C. Waters}
\chapter{Pretty Printing}
\label{PPRINT}

\prefaceword
\begin{new}
X3J13 voted in January 1989
\issue{PRETTY-PRINT-INTERFACE}
to adopt a facility for user-controlled pretty printing
as a part of the forthcoming draft Common Lisp standard.
This facility is the culmination of thirteen
years of design, testing, revision, and use of this approach.
\end{new}
This chapter presents the bulk of the Common Lisp
pretty printing specification, written by Richard C.~Waters.  I have
edited it only very lightly
to conform to the overall style of this book.

\noindent\hbox to \textwidth{\hss---Guy L. Steele Jr.}
\vskip 8pt plus 3pt minus 2pt

\section{Introduction}

Pretty printing has traditionally been a black box process, displaying
program code using a set of fixed layout rules.  Its utility can be greatly
enhanced by opening it up to user control.  The facilities described
in this chapter provide general and powerful means for specifying pretty-printing
behavior.

By providing direct access to the mechanisms within the pretty printer that
make dynamic decisions about layout, the macros and functions
\cd{pprint-\discretionary{}{}{}logical-\discretionary{}{}{}block}, \cd{pprint-newline}, and \cd{pprint-indent} make
it possible to specify pretty printing layout rules as a part of any
function that produces output.  They also make it very easy for the
function to support
detection of circularity and sharing and abbreviation based on length and
nesting depth.  Using the function
\cd{set-\discretionary{}{}{}pprint-\discretionary{}{}{}dispatch}, one can associate a user-defined pretty
printing function with any type of object.  A small set of new \cd{format}
directives allows concise implementation of user-defined pretty-printing
functions.
Together, these facilities
enable users to redefine the way code is displayed and allow the full power
of pretty printing to be applied to complex combinations of data
structures.

\penalty-10000 % required

\beforenoterule
\begin{implementation}
This chapter describes the interface of
the XP pretty printer.  XP is described fully
in~\cite{XP-PRETTY-PRINTER},
which also explains how to obtain a portable implementation.  XP uses
a highly efficient linear-time algorithm.  When properly integrated into a
Common Lisp, this algorithm supports pretty printing that is only
fractionally slower than ordinary printing.
\end{implementation}
\afternoterule

\section{Pretty Printing Control Variables}
\label{PPRINT-VARIABLES-SECTION}

The function \cd{write} accepts keyword arguments named
\cd{:pprint-dispatch}, \cd{:miser-width}, \cd{:right-margin}, and \cd{:lines},
corresponding to these variables.

\begin{defun}[Variable]
*print-pprint-dispatch*

When \cd{*print-pretty*} is not \cd{nil}, printing is controlled by the `pprint
dispatch table' stored in the variable \cd{*print-pprint-dispatch*}.  The
initial value of \cd{*print-pprint-dispatch*} is implementation-dependent and
causes traditional pretty printing of Lisp code.  The last section of this
chapter explains how the contents of this table can be changed.
\end{defun}

\begin{defun}[Variable]
*print-right-margin*

A primary goal of pretty printing is to keep the output between a pair of
margins.  The left margin is set at the column where the output begins.  If
this cannot be determined, the left margin is set to zero.

When \cd{*print-right-margin*} is not \cd{nil}, it specifies the right
margin to use when making layout decisions.  When \cd{*print-right-margin*}
is \cd{nil} (the initial value), the right margin is set at the maximum
line length that can be displayed by the output stream without wraparound
or truncation.  If this cannot be determined, the right margin is set to an
implementation-dependent value.

To allow for the possibility of variable-width fonts,
\cd{*print-right-margin*} is in units of ems---the width of an
``m'' in the font being used to display characters on the relevant output
stream at the moment when the variables are consulted.
\end{defun}

\begin{defun}[Variable]
*print-miser-width*

If \cd{*print-miser-width*} is not \cd{nil}, the pretty printer switches to a compact
style of output (called miser style) whenever the width available for
printing a substructure is less than or equal to \cd{*print-miser-width*} ems.
The initial value of \cd{*print-miser-width*} is implementation-dependent.
\end{defun}

\begin{defun}[Variable]
*print-lines*

When given a value other than its initial value of \cd{nil},
\cd{*print-lines*} limits the number of output lines produced when
something is pretty printed.  If an attempt is made to go beyond
\cd{*print-lines*} lines, ``\cd{~..}'' (a space and two periods)
is printed at the end of the last
line followed by all of the suffixes (closing delimiters) that are pending
to be printed.
\begin{lisp}
(let ((*print-right-margin* 25) (*print-lines* 3)) \\*
~~(pprint '(progn (setq a 1 b 2 c 3 d 4)))) \\*
\\
(PROGN (SETQ A 1 \\*
~~~~~~~~~~~~~B 2 \\*
~~~~~~~~~~~~~C 3 ..))
\end{lisp}

(The symbol ``\cd{..}'' is printed out to ensure that a reader error will
occur if the output is later read.  A symbol different from ``\cd{...}'' is
used to indicate that a different kind of abbreviation has occurred.)
\end{defun}



\section{Dynamic Control of the Arrangement of Output}

The following functions and macros support precise control of what should
be done when a piece of output is too large to fit in the space available.
Three concepts underlie the way these operations work: {\it logical blocks},
{\it conditional newlines}, and {\it sections}.  Before proceeding further, it is
important to define these terms.

The first line of figure~\ref{PRETTY-PRINT-SECTIONS-FIGURE} shows a
schematic piece of output.  The characters in the output are represented by
hyphens.  The positions of conditional newlines are indicated by
digits.  The beginnings and ends of logical blocks are indicated in the
figure by ``\cd{<}'' and ``\cd{>}'' respectively.

The output as a whole is a logical block and the outermost section.  This
section is indicated by the \cd{0}'s on the second line of
figure~\ref{PRETTY-PRINT-SECTIONS-FIGURE}.  Logical blocks nested within
the output are specified by the macro
\cd{pprint-logical-block}.  Conditional newline positions are specified by calls
on \cd{pprint-newline}.  Each conditional newline defines two sections (one
before it and one after it) and is associated with a third (the section
immediately containing it).

The section after a conditional newline consists of all the output up to,
but not including, (a) the next conditional newline immediately contained
in the same logical block; or if (a) is not applicable, (b) the next
newline that is at a lesser level of nesting in logical blocks; or if (b)
is not applicable, (c) the end of the output.

The section before a conditional newline consists of all the output back
to, but not including, (a) the previous conditional newline that is
immediately contained in the same logical block; or if (a) is not
applicable, (b) the beginning of the immediately containing logical block.
The last four lines in figure~\ref{PRETTY-PRINT-SECTIONS-FIGURE} indicate
the sections before and after the four conditional newlines.

The section immediately containing a conditional newline is the shortest
section that contains the conditional newline in question.  In
figure~\ref{PRETTY-PRINT-SECTIONS-FIGURE}, the first conditional newline is
immediately contained in the section marked with \cd{0}'s, the second and third
conditional newlines are immediately contained in the section before the
fourth conditional newline, and the fourth conditional newline is
immediately contained in the section after the first conditional newline.

\begin{figure}[t]
\caption{Example of Logical Blocks, Conditional Newlines, and Sections}
\label{PRETTY-PRINT-SECTIONS-FIGURE}
\begin{lisp}
~~~~~~~~~~~~~~~~~<-1---<--<--2---3->--4-->-> \\[4pt]
~~~~~~~~~~~~~~~~~000000000000000000000000000 \\
~~~~~~~~~~~~~~~~~11~111111111111111111111111 \\
~~~~~~~~~~~~~~~~~~~~~~~~~~~22~222            \\
~~~~~~~~~~~~~~~~~~~~~~~~~~~~~~333~3333       \\
~~~~~~~~~~~~~~~~~~~~~~~~44444444444444~44444
\end{lisp}
\end{figure}

Whenever possible, the pretty printer displays the entire contents of a
section on a single line.  However, if the section is too long to fit in
the space available, line breaks are inserted at conditional newline
positions within the section.

\begin{defun}[Function]
pprint-newline kind &optional stream 

The {\it stream} (which defaults to \cd{*standard-output*}) follows the
standard conventions for stream arguments to printing functions (that is,
\cd{nil} stands for \cd{*standard-output*} and \cd{t} stands for
\cd{*terminal-io*}).  The {\it kind} argument specifies the style of
conditional newline.  It must be one of \cd{:linear}, \cd{:fill},
\cd{:miser}, or \cd{:mandatory}.  An error is signaled if any other value is
supplied.  If {\it stream} is a pretty printing stream created by
\cd{pprint-logical-block}, a line break is inserted in the output when the
appropriate condition below is satisfied.  Otherwise, \cd{pprint-newline}
has no effect.  The value \cd{nil} is always returned.

If {\it kind} is \cd{:linear}, it specifies a `linear-style' conditional newline.
 A line break is inserted if and only if the immediately containing section
cannot be printed on one line.  The effect of this is that line breaks are
either inserted at every linear-style conditional newline in a logical
block or at none of them.

If {\it kind} is \cd{:miser}, it specifies a `miser-style' conditional newline. 
A line break is inserted if and only if the immediately containing section
cannot be printed on one line and miser style is in effect in the
immediately containing logical block.  The effect of this is that
miser-style conditional newlines act like linear-style conditional
newlines, but only when miser style is in effect.  Miser style is in effect
for a logical block if and only if the starting position of the logical
block is less than or equal to \cd{*print-miser-width*} from the right margin.

If {\it kind} is \cd{:fill}, it specifies a `fill-style' conditional
newline.  A line break is inserted if and only if either (a) the following
section cannot be printed on the end of the current line, (b) the preceding
section was not printed on a single line, or (c) the immediately containing
section cannot be printed on one line and miser style is in effect in the
immediately containing logical block.  If a logical block is broken up into
a number of subsections by fill-style conditional newlines, the basic
effect is that the logical block is printed with as many subsections as
possible on each line.  However, if miser style is in effect, fill-style
conditional newlines act like linear-style conditional newlines.

If {\it kind} is \cd{:mandatory}, it specifies a `mandatory-style' conditional
newline.  A line break is always inserted.  This implies that none of the
containing sections can be printed on a single line and will therefore
trigger the insertion of line breaks at linear-style conditional newlines
in these sections.

When a line break is inserted by any type of conditional newline, any
blanks that immediately precede the conditional newline are omitted from
the output and indentation is introduced at the beginning of the next line.
By default, the indentation causes the following line to begin in the same
horizontal position as the first character in the immediately containing
logical block.  (The indentation can be changed via \cd{pprint-indent}.)

There are a variety of ways {\it un\/}conditional newlines can be introduced into
the output (for example, via \cd{terpri} or by printing a string containing a newline
character).  As with mandatory conditional newlines, this prevents any of
the containing sections from being printed on one line.  In general, when
an unconditional newline is encountered, it is printed out without
suppression of the preceding blanks and without any indentation following
it.  However, if a per-line prefix has been specified (see
\cd{pprint-logical-block}), that prefix will always be printed no matter how a
newline originates.
\end{defun}

\begin{defmac}
pprint-logical-block (stream-symbol list
                      <{\!:prefix! | \!:per-line-prefix!} p | \!:suffix! s>)  
                     {\,form}*

This macro causes printing to be grouped into a logical block.  It returns
\cd{nil}.

The {\it stream-symbol} must be a symbol.  If it is \cd{nil}, it is treated the same
as if it were \cd{*standard-output*}.  If it is \cd{t}, it is treated the same as if
it were \cd{*terminal-io*}.  The run-time value of {\it stream-symbol} must
be a stream (or \cd{nil} standing for \cd{*standard-output*}
or \cd{t} standing for \cd{*terminal-io*}).
The logical block is printed into this destination stream.

The body (which consists of the {\it form\/}s)
can contain any arbitrary Lisp forms.  Within the body,
{\it stream-symbol} (or \cd{*standard-output*} if {\it stream-symbol} is
\cd{nil}, or \cd{*terminal-io*} if {\it stream-symbol} is \cd{t}) is bound
to a ``pretty printing'' stream that supports decisions about the arrangement
of output and then forwards the output to the destination stream.  All the
standard printing functions (for example, \cd{write}, \cd{princ}, \cd{terpri}) can
be used to send output to the pretty printing stream created by
\cd{pprint-logical-block}.  All and only the output sent to this pretty
printing stream is treated as being in the logical block.

\cd{pprint-logical-block} and the pretty printing stream it creates have dynamic
extent.  It is undefined what happens if output is attempted outside of
this extent to the pretty printing stream created.  It is unspecified what
happens if, within this extent, any output is sent directly to the
underlying destination stream (by calling \cd{write-char}, for example).

The \cd{:suffix}, \cd{:prefix}, and \cd{:per-line-prefix} arguments must all
be expressions that (at run time) evaluate to strings.  The \cd{:suffix} argument {\it s}
(which defaults to the null string) specifies a suffix that is printed just
after the logical block.  The \cd{:prefix} and \cd{:per-line-prefix} arguments
are mutually exclusive.  If neither \cd{:prefix} nor \cd{:per-line-prefix} is 
specified, a \cd{:prefix} of the null string is assumed.
The \cd{:prefix} argument
specifies a prefix {\it p} that is printed before the beginning of the logical block.
The \cd{:per-line-prefix} specifies a prefix {\it p} that is printed before the block 
and at the beginning of each subsequent line in the block.
An error is signaled if \cd{:prefix} and \cd{:per-line-prefix} are both used
or if a \cd{:suffix}, \cd{:prefix}, or \cd{:pre-line-prefix} argument does not
evaluate to a string.

The {\it list} is interpreted as being a list that the body is responsible
for printing.  (See \cd{pprint-exit-if-list-exhausted} and
\cd{pprint-pop}.)  If {\it list} does not (at run time) evaluate to a list,
it is printed using \cd{write}.  (This makes it easier to write printing
functions that are robust in the face of malformed arguments.)  If
\cd{*print-circle*} (and possibly also \cd{*print-shared*})
is not \cd{nil} and {\it list} is a circular (or shared) reference
to a cons, then an appropriate ``\cd{\#{\it n}\#}'' marker is printed.
(This makes it easy to write printing functions that provide full support
for circularity and sharing abbreviation.)  If \cd{*print-level*} is not
\cd{nil} and the logical block is at a dynamic nesting depth of greater
than \cd{*print-level*} in logical blocks, ``\cd{\#}'' is printed.  (This
makes it easy to write printing functions that provide full support for depth
abbreviation.)

If any of the three preceding conditions occurs, the indicated output is
printed on {\it stream-symbol} and the {\it body} is skipped along with the
printing of the prefix and suffix.  (If the
body is not responsible for printing a list, then the first two tests
above can be turned off by supplying \cd{nil} for the {\it list} argument.)

In addition to the {\it list} argument of \cd{pprint-logical-block}, the
arguments of the standard printing functions such as \cd{write},
\cd{print}, \cd{pprint}, \cd{print1}, and \cd{pprint}, as well as the
arguments of the standard \cd{format} directives such as \cd{{\Xtilde}A},
\cd{{\Xtilde}S}, (and \cd{{\Xtilde}W}) are all checked (when necessary) for
circularity and sharing.  However, such checking is not applied to the
arguments of the functions \cd{write-line}, \cd{write-string}, and
\cd{write-char} or to the literal text output by \cd{format}.  A
consequence of this is that you must use one of the latter functions if you
want to print some literal text in the output that is not supposed to be
checked for circularity or sharing.  (See the examples below.)

\beforenoterule
\begin{implementation}
Detection of circularity and sharing is supported by the pretty printer by
in essence performing the requested output twice.  On the first pass,
circularities and sharing are detected and the actual outputting of
characters is suppressed.  On the second pass, the appropriate 
``\cd{\#{\it n}=}'' and ``\cd{\#{\it n}\#}'' markers are inserted and
characters are output.

A consequence of this two-pass approach to the detection of circularity and
sharing is that the body of a \cd{pprint-logical-block} must not
perform any side-effects on the surrounding environment.  This includes not
modifying any variables that are bound outside of its scope.  Obeying this
restriction is facilitated by using \cd{pprint-pop}, instead of an ordinary
\cd{pop} when traversing a list being printed by the body of a
\cd{pprint-logical-block}.)
\end{implementation}
\afternoterule
\end{defmac}

\begin{defmac}
pprint-exit-if-list-exhausted \!!

\cd{pprint-exit-if-list-exhausted} tests whether or not the {\it list}
argument of \cd{pprint-logical-block} has been exhausted (see
\cd{pprint-pop}).  If this list has been reduced to \cd{nil},
\cd{pprint-exit-if-list-exhausted} terminates the execution of the
immediately containing \cd{pprint-logical-block} except for the printing of
the suffix.  Otherwise \cd{pprint-exit-if-list-exhausted} returns \cd{nil}.
An error message is issued if \cd{pprint-exit-if-list-exhausted} is used
anywhere other than syntactically nested within a call on
\cd{pprint-logical-block}.  It is undefined what happens if \cd{pprint-pop}
is executed outside of the dynamic extent of this
\cd{pprint-logical-block}.
\end{defmac}

\begin{defmac}
pprint-pop \!!

\cd{pprint-pop} pops elements one at a time off the {\it list} argument of
\cd{pprint-\discretionary{}{}{}logical-block}, taking care to obey \cd{*print-length*},
\cd{*print-circle*}, and \cd{*print-\discretionary{}{}{}shared*}.  An error message is issued if it is
used anywhere other than syntactically nested within a call on
\cd{pprint-logical-block}. It is undefined what happens if \cd{pprint-pop} is executed
outside of the dynamic extent of this call on \cd{pprint-\discretionary{}{}{}logical-block}.

Each time \cd{pprint-pop} is called, it pops the next value off the {\it
list} argument of \cd{pprint-logical-block} and returns it.  However,
before doing this, it performs three tests.  If the remaining list is not a
list (neither a cons nor \cd{nil}), ``\cd{.~}'' is printed
followed by the remaining list.  (This makes it easier to write printing
functions that are robust in the face of malformed arguments.)  If
\cd{*print-length*} is \cd{nil} and \cd{pprint-pop} has already been called
\cd{*print-length*} times within the immediately containing logical block,
``\cd{...}'' is printed.  (This makes it easy to write printing functions
that properly handle \cd{*print-length*}.)  If \cd{*print-circle*} (and possibly also
\cd{*print-shared*}) is not \cd{nil}, and the remaining list is a circular
(or shared) reference, then ``\cd{.~}'' is printed followed by an appropriate
``\cd{\#{\it n}\#}'' marker.  (This catches instances of cdr circularity and sharing
in lists.)

If any of the three preceding conditions occurs, the indicated output is
printed on the pretty printing stream created by the immediately containing
\cd{pprint-\discretionary{}{}{}logical-\discretionary{}{}{}block}
and the execution of the immediately containing
\cd{pprint-\discretionary{}{}{}logical-\discretionary{}{}{}block}
is terminated except for the printing of the suffix.

If \cd{pprint-logical-block} is given a {\it list} argument of
\cd{nil}---because it is not processing a list---\cd{pprint-pop} can still
be used to obtain support for \cd{*print-length*} (see the example function
\cd{pprint-vector} below).  In this situation, the first and third tests
above are disabled and \cd{pprint-pop} always returns \cd{nil}.
\end{defmac}

\begin{defun}[Function]
pprint-indent relative-to n &optional stream

\cd{pprint-indent} specifies the indentation to use in a logical block.
{\it Stream} (which defaults to \cd{*standard-output*}) follows the
standard conventions for stream arguments to printing functions.  The argument {\it
n} specifies the indentation in ems.  If {\it relative-to} is \cd{:block}, the
indentation is set to the horizontal position of the first character in the
block plus {\it n} ems.  If {\it relative-to} is \cd{:current}, the
indentation is set to the current output position plus {\it n} ems.

The argument {\it n} can be negative; however, the total indentation cannot be moved
left of the beginning of the line or left of the end of the rightmost per-line
prefix.  Changes in indentation caused by \cd{pprint-indent} do not take
effect until after the next line break.  In addition, in miser mode all
calls on \cd{pprint-indent} are ignored, forcing the lines corresponding to the
logical block to line up under the first character in the block.

An error is signaled if a value other than \cd{:block} or \cd{:current} is
supplied for {\it relative-to}.  If {\it stream} is a pretty printing
stream created by \cd{pprint-\discretionary{}{}{}logical-\discretionary{}{}{}block}, \cd{pprint-indent} sets the
indentation in the innermost dynamically enclosing logical block.
Otherwise, \cd{pprint-indent} has no effect.  The value \cd{nil} is always
returned.
\end{defun}

\begin{defun}[Function]
pprint-tab kind colnum colinc &optional stream

\cd{pprint-tab} specifies tabbing as performed by the standard \cd{format}
directive \cd{{\Xtilde}T}.  {\it Stream} (which defaults to
\cd{*standard-output*}) follows the standard conventions for stream
arguments to printing functions.  The arguments {\it colnum} and {\it
colinc} correspond to the two parameters to \cd{{\Xtilde}T} and are in
terms of ems.  The {\it kind} argument specifies the style of tabbing.  It
must be one of \cd{:line} (tab as by \cd{{\Xtilde}T}) \cd{:section} (tab as
by \cd{{\Xtilde}T}, but measuring horizontal positions relative to the
start of the dynamically enclosing section), \cd{:line-relative} (tab as by
\cd{{\Xtilde}{\Xatsign}T}), or \cd{:section-relative} (tab as by
\cd{{\Xtilde}{\Xatsign}T}, but measuring horizontal positions relative to
the start of the dynamically enclosing section).  An error is signaled if
any other value is supplied for {\it kind}.  If {\it stream} is a pretty
printing stream created by \cd{pprint-logical-block}, tabbing is performed.
Otherwise, \cd{pprint-tab} has no effect.  The value \cd{nil} is always
returned.
\end{defun}

\begin{defun}[Function]
pprint-fill stream list &optional colon? atsign? \\
pprint-linear stream list &optional colon? atsign? \\
pprint-tabular stream list &optional colon? atsign? tabsize

These three functions specify particular ways of pretty printing lists.
{\it Stream} follows the standard conventions for stream arguments to
printing functions.  Each function prints parentheses around the output if
and only if {\it colon?} (default \cd{t}) is not \cd{nil}.  Each function
ignores its {\it atsign?} argument and returns \cd{nil}.  (These two
arguments are included in this way so that these functions can be used via
\cd{{\Xtilde}/.../} and as \cd{set-pprint-dispatch} functions as well as
directly.)  Each function handles abbreviation and the detection of
circularity and sharing correctly and uses \cd{write} to print {\it list}
when given a non-list argument.

The function \cd{pprint-linear} prints a list either all on one line or with
each element on a separate line.  The function \cd{pprint-fill} prints a list
with as many elements as possible on each line.  The function
\cd{pprint-tabular} is the same as \cd{pprint-fill} except that it prints the
elements so that they line up in columns.  This function takes an
additional argument \cd{tabsize} (default 16) that specifies the column
spacing in ems.
\end{defun}

As an example of the interaction of logical blocks, conditional newlines,
and indentation, consider the function \cd{pprint-defun} below.  This
function pretty prints a list whose {\it car} is \cd{defun} in the standard way assuming
that the length of the list is exactly 4.
\begin{lisp}
;;; Pretty printer function for DEFUN forms. \\*
\\*
(defun pprint-defun (list) \\*
~~(pprint-logical-block (nil list :prefix "(" :suffix ")") \\*
~~~~(write (first list)) \\*
~~~~(write-char \#{\Xbackslash}space) \\*
~~~~(pprint-newline :miser) \\*
~~~~(pprint-indent :current 0) \\*
~~~~(write (second list)) \\*
~~~~(write-char \#{\Xbackslash}space) \\*
~~~~(pprint-newline :fill) \\*
~~~~(write (third list)) \\*
~~~~(pprint-indent :block 1) \\*
~~~~(write-char \#{\Xbackslash}space) \\*
~~~~(pprint-newline :linear) \\*
~~~~(write (fourth list))))
\end{lisp}

Suppose that one evaluates the following:
\begin{lisp}
(pprint-defun '(defun prod (x y) (* x y)))
\end{lisp}

If the line width available is greater than or equal to 26, all of the
output appears on one line.  If the width is reduced to 25,
a line break is inserted at the linear-style conditional newline before
\cd{(*~X~Y)}, producing the output shown below.  The
\cd{(pprint-indent~:block~1)} causes \cd{(*~X~Y)} to be printed at a relative
indentation of 1 in the logical block.
\begin{lisp}
(DEFUN PROD (X Y)  \\*
~~(* X Y))
\end{lisp}

If the width is 15, a line break is also inserted at the
fill-style conditional newline before the argument list.  The argument list lines
up under the function name because of the call on
\cd{(pprint-indent~:current~0)} before the printing of the function name. 
\begin{lisp}
(DEFUN PROD \\*
~~~~~~~(X Y) \\*
~~(* X Y))
\end{lisp}

If \cd{*print-miser-width*} were greater than or equal to 14,
the output would have been entirely in miser mode.
All indentation changes are
ignored in miser mode and line breaks are inserted at miser-style
conditional newlines.  The result would have been as follows:
\begin{lisp}
(DEFUN \\*
~PROD \\*
~(X Y) \\*
~(* X Y))
\end{lisp}

As an example of the use of a per-line prefix, consider that evaluating the expression
\begin{lisp}
(pprint-logical-block (nil nil :per-line-prefix ";;; ") \\*
~~(pprint-defun '(defun prod (x y) (* x y))))
\end{lisp}
produces the output
\begin{lisp}
;;; (DEFUN PROD \\*
;;;~~~~~~~~(X Y) \\*
;;;~~~(* X Y))
\end{lisp}
with a line width of 20 and \cd{nil} as the value
of the printer control variable \cd{*print-miser-width*}.

(If \cd{*print-miser-width*} were not \cd{nil} the output
\begin{lisp}
;;; (DEFUN \\*
;;; ~PROD \\*
;;; ~(X Y) \\*
;;; ~(* X Y))
\end{lisp}
might appear instead.)

As a more complex (and realistic) example, consider the function
\cd{pprint-let} below.  This specifies how to pretty print a \cd{let} in the
standard style.  It is more complex than \cd{pprint-defun} because it has
to deal with nested structure.  Also, unlike \cd{pprint-defun}, it contains
complete code to print readably any possible list that begins with the
symbol \cd{let}.  The outermost \cd{pprint-logical-block} handles the
printing of the input list as a whole and specifies that parentheses should
be printed in the output.  The second \cd{pprint-logical-block} handles the
list of binding pairs.  Each pair in the list is itself printed by the
innermost \cd{pprint-logical-block}.  (A \cd{loop} is used instead of
merely decomposing the pair into two elements so that readable output will
be produced no matter whether the list corresponding to the pair has one
element, two elements, or (being malformed) has more than two elements.)  A
space and a fill-style conditional newline are placed after each pair
except the last.  The loop at the end of the topmost
\cd{pprint-logical-block} prints out the forms in the body of the \cd{let}
separated by spaces and linear-style conditional newlines.
\begin{lisp}
;;; Pretty printer function for LET forms, \\*
;;; carefully coded to handle malformed binding pairs. \\*
\\*
(defun pprint-let (list) \\*
~~(pprint-logical-block (nil list :prefix "(" :suffix ")") \\*
~~~~(write (pprint-pop)) \\*
~~~~(pprint-exit-if-list-exhausted) \\*
~~~~(write-char \#{\Xbackslash}space) \\*
~~~~(pprint-logical-block \\*
~~~~~~~~(nil (pprint-pop) :prefix "(" :suffix ")") \\*
~~~~~~(pprint-exit-if-list-exhausted) \\*
~~~~~~(loop (pprint-logical-block \\*
~~~~~~~~~~~~~~~~(nil (pprint-pop) :prefix "(" :suffix ")") \\*
~~~~~~~~~~~~~~(pprint-exit-if-list-exhausted) \\*
~~~~~~~~~~~~~~(loop (write (pprint-pop)) \\*
~~~~~~~~~~~~~~~~~~~~(pprint-exit-if-list-exhausted) \\*
~~~~~~~~~~~~~~~~~~~~(write-char \#{\Xbackslash}space) \\*
~~~~~~~~~~~~~~~~~~~~(pprint-newline :linear))) \\*
~~~~~~~~~~~~(pprint-exit-if-list-exhausted) \\*
~~~~~~~~~~~~(write-char \#{\Xbackslash}space) \\*
~~~~~~~~~~~~(pprint-newline :fill))) \\*
~~~~(pprint-indent :block 1) \\*
~~~~(loop (pprint-exit-if-list-exhausted) \\*
~~~~~~~~~~(write-char \#{\Xbackslash}space) \\*
~~~~~~~~~~(pprint-newline :linear) \\*
~~~~~~~~~~(write (pprint-pop)))))
\end{lisp}

Suppose that the following is evaluated with \cd{*print-level*} having the value \cd{4} and
\cd{*print-circle*} having the value \cd{t}.
\begin{lisp}
(pprint-let '\#1=(let (x (*print-length* (f (g 3)))  \\*
~~~~~~~~~~~~~~~~~~~~~~(z . 2) (k (car y))) \\*
~~~~~~~~~~~~~~~~~~(setq x (sqrt z)) \#1\#))
\end{lisp}

If the line length is greater than or equal to 77, the output produced
appears on one line.  However, if the line length is 76, line breaks are
inserted at the linear-style conditional newlines separating the forms in
the body and the output below is produced.  Note that the degenerate
binding pair \cd{X} is printed readably even though it fails to be a list; a
depth abbreviation marker is printed in place of \cd{(G~3)}; the binding pair
\cd{(Z~.~2)} is printed readably even though it is not a proper list; and
appropriate circularity markers are printed.
\begin{lisp}
\#1=(LET (X (*PRINT-LENGTH* (F \#)) (Z . 2) (K (CAR Y)))  \\*
~~~~~(SETQ X (SQRT Z)) \\*
~~~~~\#1\#)
\end{lisp}

If the line length is reduced to 35, a line break is inserted at one of the
fill-style conditional newlines separating the binding pairs.
\begin{lisp}
\#1=(LET (X (*PRINT-PRETTY* (F \#)) \\*
~~~~~~~~~(Z . 2) (K (CAR Y))) \\*
~~~~~(SETQ X (SQRT Z)) \\*
~~~~~\#1\#)
\end{lisp}

Suppose that the line length is further reduced to 22 and \cd{*print-length*} is
set to 3. In this situation, line breaks are inserted after both the first
and second binding pairs.  In addition, the second binding pair is itself
broken across two lines.  Clause (b) of the description of fill-style
conditional newlines prevents the binding pair \cd{(Z~.~2)} from being printed
at the end of the third line.  Note that the length abbreviation hides the
circularity from view and therefore the printing of circularity markers
disappears.
\begin{lisp}
(LET (X \\*
~~~~~~(*PRINT-LENGTH* \\*
~~~~~~~(F \#)) \\*
~~~~~~(Z . 2) ...) \\*
~~(SETQ X (SQRT Z)) \\*
~~...)
\end{lisp}

The function \cd{pprint-tabular} could be defined as follows:
\begin{lisp}
(defun pprint-tabular (s list \&optional (c? t) a? (size 16)) \\*
~~(declare (ignore a?)) \\*
~~(pprint-logical-block \\*
~~~~~~(s list :prefix (if c? "(" "") :suffix (if c? ")" "")) \\*
~~~~(pprint-exit-if-list-exhausted) \\*
~~~~(loop (write (pprint-pop) :stream s) \\*
~~~~~~~~~~(pprint-exit-if-list-exhausted) \\*
~~~~~~~~~~(write-char \#{\Xbackslash}space s) \\*
~~~~~~~~~~(pprint-tab :section-relative 0 size s) \\*
~~~~~~~~~~(pprint-newline :fill s))))
\end{lisp}

Evaluating the following with a line length of 25 produces the output shown.
\begin{lisp}
(princ "Roads ") \\*
(pprint-tabular nil '(elm main maple center) nil nil 8) \\*
\\
Roads ELM~~~~~MAIN \\*
~~~~~~MAPLE~~~CENTER
\end{lisp}

The function below prints a vector using \cd{\#(...)} notation.
\begin{lisp}
(defun pprint-vector (v) \\*
~~(pprint-logical-block (nil nil :prefix "\#(" :suffix ")") \\*
~~~~(let ((end (length v)) (i 0)) \\*
~~~~~~(when (plusp end) \\*
~~~~~~~~(loop (pprint-pop) \\*
~~~~~~~~~~~~~~(write (aref v i)) \\*
~~~~~~~~~~~~~~(if (= (incf i) end) (return nil)) \\*
~~~~~~~~~~~~~~(write-char \#{\Xbackslash}space) \\*
~~~~~~~~~~~~~~(pprint-newline :fill))))))
\end{lisp}

Evaluating the following with a line length of 15 produces the output shown.
\begin{lisp}
(pprint-vector '\#(12 34 567 8 9012 34 567 89 0 1 23)) \\*
\\
\#(12 34 567 8  \\*
~~9012 34 567  \\*
~~89 0 1 23)
\end{lisp}

\section{Format Directive Interface}
\label{PPRINT-FORMAT-DIRECTIVES-SECTION}

The primary interface to operations for dynamically determining the
arrangement of output is provided through the functions above.  However, an
additional interface is provided via a set of format directives
because, as shown by the examples in this section and the
next, \cd{format} strings are typically a much more compact way to specify
pretty printing.  In addition, without such an interface, one would have to
abandon the use of \cd{format} when interacting with the pretty printer.

\begin{flushdesc}
\item[\cd{{\Xtilde}W}]
{\it Write.}  An {\it arg}, any Lisp object, is printed obeying {\it every}
printer control variable (as by \cd{write}).  In addition, \cd{{\Xtilde}W}
interacts correctly with depth abbreviation by not resetting the depth
counter to zero.  \cd{{\Xtilde}W} does not accept parameters.  If given the colon
modifier, \cd{{\Xtilde}W} binds \cd{*print-pretty*} to \cd{t}.  If given the atsign
modifier, \cd{{\Xtilde}W} binds \cd{*print-level*} and \cd{*print-length*} to 
\cd{nil}.

\cd{{\Xtilde}W} provides automatic support for circularity detection.  If
\cd{*print-circle*} (and possibly also \cd{*print-shared*}) is not \cd{nil} and
\cd{{\Xtilde}W} is applied to an argument that is a circular (or shared) reference,
an appropriate ``\cd{\#{\it n}\#}'' marker is inserted in the output
instead of printing the argument.

\item[\cd{{\Xtilde}{\Xunderscore}}]
{\it Conditional newline.} Without any modifiers,
\cd{{\Xtilde}{\Xunderscore}} is equivalent to
\cd{(pprint-\discretionary{}{}{}newline :linear)}.
The directive \cd{{\Xtilde}{\Xatsign}{\Xunderscore}} is
equivalent to \cd{(pprint-\discretionary{}{}{}newline :miser)}.
The directive \cd{{\Xtilde}:{\Xunderscore}}
is equivalent to \cd{(pprint-\discretionary{}{}{}newline :fill)}.
The directive \cd{{\Xtilde}:{\Xatsign}{\Xunderscore}} is
equivalent to \cd{(pprint-\discretionary{}{}{}newline :mandatory)}.


\item[\cd{{\Xtilde}<{\it str}{\Xtilde}:>}]
{\it Logical block.} If \cd{{\Xtilde}:>} is used to terminate a
\cd{{\Xtilde}<...} directive, the directive is equivalent to a call on
\cd{pprint-logical-block}.  The \cd{format} argument corresponding to the
\cd{{\Xtilde}<...{\Xtilde}:>} directive is treated in the same way as the {\it list}
argument to \cd{pprint-logical-block}, thereby providing automatic support for
non-list arguments and the detection of circularity, sharing, and depth abbreviation. 
The portion of the \cd{format} control string nested within the
\cd{{\Xtilde}<...{\Xtilde}:>} specifies the \cd{:prefix} (or \cd{:per-line-prefix}),
\cd{:suffix}, and body of the \cd{pprint-logical-block}.

The \cd{format} string portion enclosed by \cd{{\Xtilde}<...{\Xtilde}:>} can be
divided into segments \cd{{\Xtilde}<{\it prefix\/}{\Xtilde};{\it body\/}{\Xtilde};{\it
suffix\/}{\Xtilde}:>} by \cd{{\Xtilde};} directives.  If the first section is
terminated by \cd{\Xtilde\Xatsign;}, it specifies a per-line prefix rather than a
simple prefix.  The prefix and suffix cannot contain \cd{format} directives.  
An error is signaled if either the prefix or suffix fails to be a constant string
or if the enclosed portion is divided into more than three segments. 

If the enclosed portion is divided into only two segments, the suffix defaults
to the null string.  If the enclosed portion consists of only a single
segment, both the prefix and the suffix default to the null string.  If the
colon modifier is used (that is, \cd{{\Xtilde}:<...{\Xtilde}:>}), the prefix and
suffix default to \cd{"("} and \cd{")"}, respectively, instead of the null
string.  

The body segment can be any arbitrary \cd{format} control string.  This \cd{format}
control string is applied to the elements of the list corresponding to the
\cd{{\Xtilde}<...{\Xtilde}:>} directive as a whole.  Elements are extracted from this
list using \cd{pprint-pop}, thereby providing automatic support for malformed lists
and the detection of circularity, sharing, and length abbreviation.
Within the body segment, \cd{\Xtilde\Xcircumflex} acts like
\cd{pprint-exit-if-list-exhausted}.

\cd{{\Xtilde}<...{\Xtilde}:>} supports a feature not supported by
\cd{pprint-logical-block}.  If \cd{\Xtilde:\Xatsign>} is used to terminate the
directive (that is, \cd{{\Xtilde}<...{\Xtilde}:\Xatsign>}), then a fill-style
conditional newline is automatically inserted after each group of blanks
immediately contained in the body (except for blanks after a
\cd{\Xtilde<newline>} directive).  This makes it easy to achieve the equivalent
of paragraph filling.

If the atsign modifier is used with \cd{{\Xtilde}<...{\Xtilde}:>}, the
entire remaining argument list is passed to the directive as its argument.
All of the remaining arguments are always consumed by
\cd{{\Xtilde}\Xatsign<...{\Xtilde}:>}, even if they are not all used by the
\cd{format} string nested in the directive.  Other than the difference in its
argument, \cd{{\Xtilde}\Xatsign<...{\Xtilde}:>} is exactly the same as
\cd{{\Xtilde}<...{\Xtilde}:>}, except that circularity (and sharing) detection 
is not applied if the \cd{{\Xtilde}\Xatsign<...{\Xtilde}:>} is at top level
in a \cd{format} string.  This ensures that circularity detection is applied 
only to data lists and not to \cd{format} argument lists. 

To a considerable extent, the basic form of the directive
\cd{{\Xtilde}<...{\Xtilde}>} is incompatible with the dynamic control of
the arrangement of output by \cd{{\Xtilde}W}, \cd{{\Xtilde}\Xunderscore},
\cd{{\Xtilde}<...{\Xtilde}:>}, \cd{{\Xtilde}I}, and \cd{{\Xtilde}:T}.  As
a result, an error is signaled if any of these directives is nested within
\cd{{\Xtilde}<...{\Xtilde}>}.  Beyond this, an error is also signaled if
the \cd{{\Xtilde}<...{\Xtilde}:;...{\Xtilde}>} form of
\cd{{\Xtilde}<...{\Xtilde}>} is used in the same \cd{format} string with
\cd{{\Xtilde}W}, \cd{{\Xtilde}\Xunderscore},
\cd{{\Xtilde}<...{\Xtilde}:>}, \cd{{\Xtilde}I}, or \cd{{\Xtilde}:T}.


\item[\cd{{\Xtilde}I}]
{\it Indent.} \cd{{\Xtilde}{\it n}I} is equivalent to
\cd{(pprint-indent~:block~{\it n})}.  \cd{{\Xtilde}:{\it n}I} is equivalent to
\cd{(pprint-indent~:current~{\it n})}.  In both cases, {\it n} defaults to zero
if it is omitted.


\item[\cd{{\Xtilde}:T}] 
{\it Tabulate.} If the colon modifier is used with the \cd{{\Xtilde}T}
directive, the tabbing computation is done relative to the column where the
section immediately containing the directive begins, rather than with
respect to column zero.  \cd{{\Xtilde}{\it n},{\it m}:T} is equivalent to
\cd{(pprint-tab~:section~{\it n}~{\it m})}.  \cd{{\Xtilde}{\it n},{\it m}:{\Xatsign}T}
is equivalent to \cd{(pprint-tab~:section-relative~{\it n}~{\it m})}.  The numerical
parameters are both interpreted as being in units of ems and both default
to 1.

\item[\cd{{\Xtilde}/{\it name}/}]
{\it Call function.} User-defined functions can be called from within a
\cd{format} string by using the directive \cd{{\Xtilde}/{\it name}/}.  The
colon modifier, the atsign modifier, and arbitrarily many parameters can be
specified with the \cd{{\Xtilde}/{\it name}/} directive.  The {\it name}
can be any string that does not contain ``\cd{/}''.  All of the characters
in {\it name} are treated as if they were upper case.  If {\it name}
contains a ``\cd{:}'' or ``\cd{::}'', then everything up to but not
including the first ``\cd{:}'' or ``\cd{::}'' is taken to be a string that
names a package.  Everything after the first ``\cd{:}'' or ``\cd{::}'' (if
any) is taken to be a string that names a symbol.  The function
corresponding to a \cd{{\Xtilde}/{\it name}/} directive is obtained by
looking up the symbol that has the indicated name in the indicated package.
If {\it name} does not contain a ``\cd{:}'' or ``\cd{::}'', then the whole
name string is looked up in the \cd{user} package.

When a \cd{{\Xtilde}/{\it name}/} directive is encountered, the indicated
function is called with four or more arguments.  The first four arguments
are the output stream, the \cd{format} argument corresponding to the
directive, the value \cd{t} if the colon modifier was used (\cd{nil}
otherwise), and the value \cd{t} if the atsign modifier was used (\cd{nil}
otherwise).  The remaining arguments consist of any parameters specified
with the directive.  The function should print the argument appropriately.
Any values returned by the function are ignored.

The three functions \cd{pprint-linear}, \cd{pprint-fill}, and
\cd{pprint-tabular} are designed so that they can be called by
\cd{\Xtilde/.../} (that is, \cd{{\Xtilde}/pprint-linear/},
\cd{{\Xtilde}/pprint-fill/}, and \cd{{\Xtilde}/pprint-tabular/}.  In
particular they take colon and atsign arguments.
\end{flushdesc}

As examples of the convenience of specifying pretty printing with
\cd{format} strings, consider the functions \cd{pprint-defun}
and \cd{pprint-let} used as
examples in the last section.  They can be more compactly defined as follows.  The
function \cd{pprint-vector} cannot be defined using \cd{format}, because the data
structure it traverses is not a list.  The function \cd{pprint-tabular} is
inconvenient to define using \cd{format}, because of the need to pass its
\cd{tabsize} argument through to a \cd{\Xtilde:T} directive nested within
an iteration over a list.
\begin{lisp}
(defun pprint-defun (list) \\*
~~(format t
"{\Xtilde}:<{\Xtilde}W~{\Xtilde}\Xatsign\Xunderscore{\Xtilde}:I{\Xtilde}W~{\Xtilde}:\Xunderscore{\Xtilde}W{\Xtilde}1I~{\Xtilde}\Xunderscore{\Xtilde}W{\Xtilde}:>"
list))\\
 \\
(defun pprint-let (list) \\*
~~(format t "{\Xtilde}:<{\Xtilde}W{\Xtilde}{\Xcircumflex} \relax
  {\Xtilde}:<{\Xtilde}{\Xatsign}\{{\Xtilde}:<{\Xtilde}{\Xatsign}\{{\Xtilde}W{\Xtilde}{\Xcircumflex} \relax
      {\Xtilde}{\Xunderscore}{\Xtilde}\}{\Xtilde}:>{\Xtilde}{\Xcircumflex} \relax
      {\Xtilde}:{\Xunderscore}{\Xtilde}\}{\Xtilde}:>{\Xtilde}1I{\Xtilde} \\*
~~~~~~~~~~~~~~~~{\Xtilde}{\Xatsign}\{{\Xtilde}{\Xcircumflex} {\Xtilde}{\Xunderscore}{\Xtilde}W{\Xtilde}\}{\Xtilde}:>" \\*
~~~~~~~~~~list))
\end{lisp}

\section{Compiling Format Control Strings}

The control strings used by \cd{format} are essentially programs that
perform printing.  The macro \cd{formatter} provides the efficiency of
using a compiled function for printing without losing the visual compactness of
\cd{format} strings.

\begin{defmac}
formatter control-string

The {\it control-string} must be a literal string.  An error is signaled if
{\it control-string} is not a valid \cd{format} control string.  The macro
\cd{formatter} expands into an expression of the form
\cd{(function~(lambda~(stream~\&rest~args)~...))} that does the printing
specified by {\it control-string}.  The \cd{lambda} created accepts an
output stream as its first argument and zero or more data values as its
remaining arguments.  The value returned by the \cd{lambda} is the tail (if
any) of the data values that are not printed out by {\it control-string}.
(For example, if the {\it control-string} is \cd{"{\Xtilde}A{\Xtilde}A"}, the
\cd{cddr} (if any) of the data values is returned.)  The form
\cd{(formatter~"{\Xtilde}\%{\Xtilde}2\Xatsign\{{\Xtilde}S,~{\Xtilde}\}")} is
equivalent to the following:
\begin{lisp}
\#'(lambda (stream \&rest args) \\*
~~~~(terpri stream) \\*
~~~~(dotimes (n 2) \\*
~~~~~~(if (null args) (return nil)) \\*
~~~~~~(prin1 (pop args) stream) \\*
~~~~~~(write-string ", " stream)) \\*
~~~~args)
\end{lisp}

In support of the above mechanism, \cd{format} is extended so that it accepts
functions as its second argument as well as strings.  When a function is
provided, it must be a function of the form created by \cd{formatter}.  The
function is called with the appropriate output stream as its first argument
and the data arguments to \cd{format} as its remaining arguments.  The
function should perform whatever output is necessary and return the unused
tail of the arguments (if any).  The directives \cd{\Xtilde?} and
\cd{\Xtilde\{\Xtilde\}} with no body are also extended so that they accept
functions as well as control strings.  Every other standard function that
takes a \cd{format} string as an argument (for example, \cd{error} and \cd{warn})
is also extended so that it can accept functions of the form above
instead.
\end{defmac}

\section{Pretty Printing Dispatch Tables}

When \cd{*print-pretty*} is not \cd{nil}, the pprint dispatch table in the variable
\cd{*print-pprint-dispatch*} controls how objects are printed.  The information
in this table takes precedence over all other mechanisms for specifying how
to print objects.  In particular, it overrides user-defined \cd{print-object}
methods and print functions for structures.  However, if there is no
specification for how to pretty print a particular kind of object, it is then
printed using the standard mechanisms as if \cd{*print-pretty*} were \cd{nil}.

A pprint dispatch table is a mapping from keys to pairs of values.  The keys
are type specifiers.  The values are functions and numerical priorities.
Basic insertion and retrieval is done based on the keys with the equality
of keys being tested by \cd{equal}.  The function to use when pretty printing an
object is chosen by finding the highest priority function in
\cd{*print-pprint-dispatch*} that is associated with a type specifier that
matches the object.

\begin{defun}[Function]
copy-pprint-dispatch &optional table 

A copy is made of {\it table}, which defaults to the current pprint dispatch
table.  If {\it table} is \cd{nil}, a copy is returned of the initial value of
\cd{*print-pprint-dispatch*}.
\end{defun}

\begin{defun}[Function]
pprint-dispatch object &optional table 

This retrieves the highest priority function from a pprint table that is
associated with a type specifier in the table that matches {\it object}.
The function is chosen by finding all the type specifiers in {\it table}
that match the object and selecting the highest priority function
associated with any of these type specifiers.  If there is more than one
highest priority function, an arbitrary choice is made.  If no type
specifiers match the object, a function is returned that prints object with
\cd{*print-pretty*} bound to \cd{nil}.

As a second return value, \cd{pprint-dispatch} returns a flag that is \cd{t} if a
matching type specifier was found in {\it table} and \cd{nil} if not.

{\it Table} (which defaults to \cd{*print-pprint-dispatch*}) must be a
pprint dispatch table.  {\it Table} can be \cd{nil}, in which case
retrieval is done in the initial value of \cd{*print-pprint-dispatch*}.

When \cd{*print-pretty*} is \cd{t}, \cd{(write~object~:stream~s)} is equivalent to
\cd{(funcall~(pprint-dispatch~object)~s~object)}.
\end{defun}

\begin{defun}[Function]
set-pprint-dispatch type function &optional priority table 

This puts an entry into a pprint dispatch table and returns \cd{nil}.  The {\it
type} must be a valid type specifier and is the key of the entry.
The first action of \cd{set-pprint-dispatch} is to remove any pre-existing
entry associated with {\it type}.  This guarantees that there
will never be two entries associated with the same type specifier in a
given pprint dispatch table.  Equality of type specifiers is tested by
\cd{equal}.

Two values are associated with each type specifier in a pprint dispatch
table: a function and a priority.  The {\it function} must accept two
arguments:  the stream to send output to and the object to be printed.
The {\it function} should pretty print the object on the stream.  The {\it
function} can assume that object satisfies {\it type}.  The {\it function}
should obey \cd{*print-readably*}.  Any values returned by the {\it function}
are ignored.

The {\it priority} (which defaults to 0) must be a non-complex number.
This number is used as a
priority to resolve conflicts when an object matches more than one entry.  An error
is signaled if priority fails to be a non-complex number.

The {\it table} (which defaults to the value of \cd{*print-pprint-dispatch*}) must be a pprint
dispatch table.  The specified entry is placed in this table.

It is permissible for {\it function} to be \cd{nil}.  In this situation,
there will be no {\it type} entry in {\it table} after
\cd{set-pprint-dispatch} is evaluated.

To facilitate the use of pprint dispatch tables for controlling the pretty
printing of Lisp code, the {\it type-specifier} argument of the function
\cd{set-pprint-dispatch} is allowed to contain the form \cd{(cons}~{\it
car-type~cdr-type}\cd{)}.  This form indicates that the corresponding object must be
a cons whose {\it car} satisfies the type specifier {\it car-type} and whose
{\it cdr} satisfies
the type specifier {\it cdr-type}.  The {\it cdr-type} can be omitted, in which case
it defaults to \cd{t}.
\end{defun}

The initial value of \cd{*print-pprint-dispatch*} is implementation-dependent.
However, the initial entries all use a special class of priorities that
are less than every priority that can be
specified using \cd{set-pprint-dispatch}.  This guarantees that pretty printing
functions specified by users will override everything in the initial value of
\cd{*print-pprint-dispatch*}.

Consider the following examples.  The first form restores
\cd{*print-\discretionary{}{}{}pprint-\discretionary{}{}{}dispatch*} to its initial value.
The next two forms then specify a special way of pretty printing ratios.  Note that the more specific type
specifier has to be associated with a higher priority.
\begin{lisp}
(setq *print-pprint-dispatch* \\
~~~~~~(copy-pprint-dispatch nil)) \\*
\\
(defun div-print (s r colon? atsign?) \\*
~~(declare (ignore colon? atsign?)) \\*
~~(format s "(/ {\Xtilde}D {\Xtilde}D)" (numerator (abs r)) (denominator r))) \\*
\\
(set-pprint-dispatch 'ratio (formatter "\#.\Xtilde/div-print/")) \\*
\\
(set-pprint-dispatch '(and ratio (satisfies minusp)) \\*
~~(formatter "\#.(- \Xtilde/div-print/)") \\*
~~5) \\*
\\
(pprint '(1/3 -2/3)) {\rm prints:} (\#.(/ 1 3) \#.(- (/ 2 3)))
\end{lisp}

The following two forms illustrate the specification of pretty printing
functions for particular types of Lisp code.  The first form illustrates how to
specify the traditional method for printing quoted objects using ``\cd{'}''
syntax.  Note the care taken to ensure that data lists that happen to begin
with \cd{quote} will be printed readably.  The second form specifies that lists
beginning with the symbol \cd{my-let} should print the same way that lists
beginning with \cd{let} print when the initial pprint dispatch table is in effect.
\begin{lisp}
(set-pprint-dispatch '(cons (member quote)) \\*
~~\#'(lambda (s list) \\*
~~~~~~(if (and (consp (cdr list)) (null (cddr list))) \\*
~~~~~~~~~~(funcall (formatter "'{\Xtilde}W") s (cadr list)) \\*
~~~~~~~~~~(pprint-fill s list))))) \\*
\\
(set-pprint-dispatch '(cons (member my-let)) \\*
~~(pprint-dispatch '(let) nil)) \\*
\end{lisp}

The next example specifies a default method for printing lists that do not
correspond to function calls.  Note that, as shown in the definition of
\cd{pprint-tabular} above, \cd{pprint-linear}, \cd{pprint-fill}, and
\cd{pprint-tabular} are defined with optional colon and atsign arguments so that
they can be used as pprint dispatch functions as well as \cd{\Xtilde/.../} functions.
\begin{lisp}
(set-pprint-dispatch \\*
~~'(cons (not (and symbol (satisfies fboundp)))) \\*
~~\#'pprint-fill \\*
~~-5)
\end{lisp}
With a line length of 9, \cd{(pprint '(0 b c d e f g h i j k))} prints:
\begin{lisp}
(0 b c d \\*
~e f g h \\*
~i j k)
\end{lisp}

This final example shows how to define a pretty printing function for a
user defined data structure.
\begin{lisp}
(defstruct family mom kids) \\*
\\
(set-pprint-dispatch 'family \\*
~~\#'(lambda (s f) \\*
~~~~~~(format s "{\Xtilde}\Xatsign<\#<{\Xtilde};{\Xtilde}W and \relax
 {\Xtilde}2I{\Xtilde}\Xunderscore{\Xtilde}/pprint-fill/{\Xtilde};>{\Xtilde}:>" \\*
~~~~~~~~~~~~~~(family-mom f) (family-kids f))))
\end{lisp}

The pretty printing function for the structure \cd{family} specifies how to
adjust the layout of the output so that it can fit aesthetically into a
variety of line widths.  In addition, it obeys the printer control
variables \cd{*print-level*}, \cd{*print-length*}, \cd{*print-lines*},
\cd{*print-circle*}, \cd{*print-shared*}, and \cd{*print-escape*}, and can tolerate
several different kinds of malformity in the data structure.  The output below
shows what is printed out with a right margin of 25, \cd{*print-pretty*}
\cd{t}, \cd{*print-escape*} \cd{nil}, and a malformed \cd{kids} list.
\begin{lisp}
(write (list 'principal-family \\*
~~~~~~~~~~~~~(make-family :mom "Lucy" \\*
~~~~~~~~~~~~~~~~~~~~~~~~~~:kids '("Mark" "Bob" . "Dan"))) \\*
~~~~~~~:right-margin 25 :pretty T :escape nil :miser-width nil) \\*
\\
(PRINCIPAL-FAMILY \\*
~\#<Lucy and \\*
~~~~~Mark Bob . Dan>)
\end{lisp}

Note that a pretty printing function for a structure is different from the
structure's print function.  While print functions are permanently
associated with a structure, pretty printing functions are stored in pprint
dispatch tables and can be rapidly changed to reflect different printing
needs.  If there is no pretty printing function for a structure in the
current print dispatch table, the print function (if any) is used instead.

      % Pretty-printing
%Part{CLOS, Root = "CLM.MSS"}
%%% Chapter of Common Lisp Manual.  Copyright 1984, 1987, 1988, 1989 Guy L. Steele Jr.

\clearpage\def\pagestatus{FINAL PROOF}


%\begingroup

\def\CLOS{Common Lisp Object System}
\def\OS{Object System}
\def\bit{\it}                             % NOT \let\bit\it !!!
\def\sub{_}

\chapterauthor{Daniel G.~Bobrow, Linda G.~DeMichiel,
Richard P.~Gabriel,\hfil\break Sonya E.~Keene, Gregor Kiczales,
and David A.~Moon}
\chapter{Common Lisp Object System}
\label{CLOS}

\prefaceword
\begin{new}
X3J13 voted in June 1988
\issue{CLOS}
to adopt the first two chapters (of three) of the
Common Lisp Object System specification
as a part of the forthcoming draft Common Lisp standard.
\end{new}
This chapter presents the bulk of the first two chapters of the
Common Lisp Object System specification; it is substantially
identical to these two specification chapters as previously published elsewhere
\cite{SIGPLAN-CLOS,LASC-CLOS-PART-1,LASC-CLOS-PART-2}.
I have edited the material only very lightly
to conform to the overall style of this book and to save a substantial
number of pages by using a typographically condensed presentation.
I have inserted a small
number of bracketed remarks, identified by the initials GLS.
The chapter divisions of the original specification have become
section divisions in this chapter; references to the three chapters
of the original specification now refer to the three ``parts'' of the
specification.
(See the Acknowledgments to this second edition for
acknowledgments to others who contributed to the Common Lisp Object System specification.)
This is not the last word on CLOS;
X3J13 may well refine this material further.
Keene has written a good tutorial introduction to CLOS~\cite{KEENE}.

\noindent\hbox to \textwidth{\hss---Guy L. Steele Jr.}
\vskip 8pt plus 3pt minus 2pt

\section{Programmer Interface Concepts}

The \CLOS\ (CLOS) is an object-oriented extension to Common Lisp. It is based on
generic functions, multiple inheritance, declarative method
combination, and a meta-object protocol.

The first two parts of this specification describe
the standard Programmer Interface for the \CLOS.  The first part,
Programmer Interface Concepts,
contains a description of the concepts of the \CLOS, and the second part,
Functions in the Programmer Interface,
contains a description of the functions and macros in the \CLOS\
Programmer Interface.  The third part, The \CLOS\ Meta-Object
Protocol, explains how the \CLOS\ can be customized.  [The third part
has not yet been approved by X3J13 for inclusion in the forthcoming
Common Lisp standard and is not included in this book.---GLS]

The fundamental objects of the \CLOS\ are classes, instances,
generic functions, and methods. 

A {\bit class\/} object determines the structure and behavior of a set
of other objects, which are called its {\bit instances}. 
Every Common Lisp object is an {\bit
instance\/} of a class.  The class of an object determines the set of
operations that can be performed on the object.

A {\bit generic function\/} is a function whose behavior depends on the
classes or identities of the arguments supplied to it.  A generic
function object contains a set of methods, a lambda-list, a
method combination type, and other information.  The {\bit methods} define
the class-specific behavior and operations of the generic function; a
method is said to {\bit specialize\/} a generic function.  When invoked,
a generic function executes a subset of its methods based on the
classes of its arguments.

A generic function can be used in 
the same ways as an ordinary function in Common Lisp; in
particular, a generic function can be used as an argument to 
\cdf{funcall} and \cdf{apply} and can be given a global or a local name.

A {\bit method\/} is an object that contains a method function, a sequence of
{\bit parameter specializers\/} that specify when the given method is
applicable, and a sequence of {\bit qualifiers\/} that is used by the
{\bit method combination\/} facility to distinguish among methods.  Each
required formal parameter of each method has an associated parameter
specializer, and the method will be invoked only on arguments that
satisfy its parameter specializers.

The method combination facility controls the selection of methods, the
order in which they are run, and the values that are returned by the
generic function.  The \CLOS\ offers a default method combination type
and provides a facility for declaring new types of method combination.


\subsection{Error Terminology}
\label{Error-Terminology-SECTION}

The terminology used in this chapter to describe erroneous
situations differs from the terminology used in the first edition.
The new terminology involves {\bit situations};
a situation is the evaluation of an expression in some
specific context. For example, a situation might be the invocation of
a function on arguments that fail to satisfy some specified
constraints.

In the specification of the \CLOS, the behavior of programs in all situations
is described, and the options available to the implementor are defined. No
implementation is allowed to extend the syntax or semantics of the \OS\ except
as explicitly defined in the \OS\ specification. In particular, no
implementation is allowed to extend the syntax of the \OS\ in such a way that
ambiguity between the specified syntax of the \OS\ and those extensions is
possible.

\begin{flushdesc}
\item[``When situation {\it S} occurs, an error is signaled.'']

This terminology has the following meaning:

\begin{itemize}

\item  If this situation occurs, an error will be signaled in
the interpreter and in code compiled under all compiler safety
optimization levels.

\item  Valid programs may rely on the fact that an error will be
signaled in the interpreter and in code compiled under all compiler
safety optimization levels.

\item  Every implementation is required to detect such an error
in the interpreter and in code compiled under all compiler safety
optimization levels.

\end{itemize}

\item[``When situation {\it S} occurs, an error should be signaled.'']

This terminology has the following meaning:

\begin{itemize}

\item  If this situation occurs, an error will be signaled at
least in the interpreter and in code compiled under the safest
compiler safety optimization level.

\item  Valid programs may not rely on the fact that an error will be
signaled.

\item  Every implementation is required to detect such an error
at least in the interpreter and in code compiled under the safest
compiler safety optimization level.

\item  When an error is not signaled, the results are undefined (see
below).

\end{itemize}

\item[``When situation {\it S} occurs, the results are undefined.'']

This terminology has the following meaning:

\begin{itemize}

\item  If this situation occurs, the results are unpredictable.  The
results may range from harmless to fatal.

\item  Implementations are allowed to detect this situation and
signal an error, but no implementation is required to detect the
situation.

\item  No valid program may depend on the effects of this
situation, and all valid programs are required to treat the effects 
of this situation as unpredictable.

\end{itemize}

\item[``When situation {\it S} occurs, the results are unspecified.'']

This terminology has the following meaning:

\nobreak
\begin{itemize}

\item  The effects of this situation are not specified in
the \OS, but the effects are harmless.

\item  Implementations are allowed to specify the effects of
this situation.

\item  No portable program can depend on the effects of this
situation, and all portable programs are required to treat the situation
as unpredictable but harmless.

\end{itemize}

\item[``The \CLOS\ may be extended to cover situation {\it S}.'']

The meaning of this terminology is that an implementation is free to treat
situation {\it S} in one of three ways:

\begin{itemize}

\item  When situation {\it S} occurs, an error is signaled at least
in the interpreter and in code compiled under the safest compiler
safety optimization level.

\item  When situation {\it S} occurs, the results are undefined.

\item  When situation {\it S} occurs, the results are defined and
specified.

\end{itemize}

\noindent
In addition, this terminology has the following meaning:

\begin{itemize}

\item  No portable program can depend on the effects of this
situation, and all portable programs are required to treat the situation
as undefined.

\end{itemize}

\item[``Implementations are free to extend the syntax {\it S}.'']

This terminology has the following meaning:

\begin{itemize}

\item  Implementations are allowed to define unambiguous extensions
to syntax {\it S}.

\item  No portable program can depend on this extension, and
all portable programs are required to treat the syntax
as meaningless.

\end{itemize}
\end{flushdesc}

The \CLOS\ specification may disallow certain extensions while allowing others.


\subsection{Classes}
\label{Classes-SECTION}

A {\bit class\/} is an object that determines the structure and behavior 
of a set of other objects, which are called its {\bit instances}.   

A class can inherit structure and behavior from other classes.  
A class whose definition refers to other classes for the purpose of
inheriting from them is said to be a {\bit subclass\/} of each of
those classes.  The classes that are designated for purposes of
inheritance are said to be {\bit superclasses\/}
of the inheriting class.

A class can have a {\bit name}. The function \cdf{class-name} takes a
class object and returns its name. The name of an anonymous class is
\cdf{nil}.  A symbol can {\bit name\/} a class.  The function 
\cdf{find-class} takes a symbol and returns the class that the symbol
names. A class has a {\bit proper name\/} if the name is a symbol
and if the name of the class
names that class.  That is, a class~{\it C} has the {\bit proper
name\/}~{\it S} if {\it S}~$=$ \cd{(class-name {\it C})} and {\it C}~$=$ \cd{(find-class
{\it S})}.  Notice that it is possible for \cd{(find-class ${\it S}\sub 1$)}
$=$ \cd{(find-class ${\it S}\sub 2$)} and ${\it S}\sub 1\neq {\it S}\sub 2$.
If {\it C}~$=$ \cd{(find-class {\it S})}, we say that {\it C} is the {\bit class named}
{\it S}.

A class ${\it C}\sub{1}$ is a {\bit direct superclass\/} of a class
${\it C}\sub{2}$ if ${\it C}\sub{2}$ explicitly designates ${\it C}\sub{1}$ as a
superclass in its definition.  In this case, ${\it C}\sub{2}$ is a {\bit
direct subclass\/} of ${\it C}\sub{1}$.  A class ${\it C}\sub{\hbox{\scriptsize\it n}}$ is a {\bit
superclass\/} of a class ${\it C}\sub{1}$ if there exists a series of
classes ${\it C}\sub{2},\ldots,C\sub{\hbox{\scriptsize\it n}-1}$
such that ${\it C}\sub{\hbox{\scriptsize\it i}+1}$ is a
direct superclass of ${\it C}\sub{\hbox{\scriptsize\it i}}$ for $1 \leq {\it i} < {\it n}$.  In this case, 
${\it C}\sub{1}$ is a {\bit subclass\/} of ${\it C}\sub{\hbox{\scriptsize\it n}}$.  A class is
considered neither a superclass nor a subclass of itself.  That is, if
${\it C}\sub{1}$ is a superclass of ${\it C}\sub{2}$, then ${\it C}\sub{1} \neq
C\sub{2}$.  The set of classes consisting of some given
class {\it C} along with all of its superclasses is called ``{\it C} and its
superclasses.''

Each class has a {\bit class precedence list}, which is a total ordering
on the set of the given class and its superclasses.  The total ordering
is expressed as a list ordered from most specific to least specific.
The class precedence list is used in several ways.  In general, more
specific classes can {\bit shadow}, or override, features that would
otherwise be inherited from less specific classes.  The method selection
and combination process uses the class precedence list to order methods
from most specific to least specific. 
 
When a class is defined, the order in which its direct superclasses
are mentioned in the defining form is important.  Each class has a
{\bit local precedence order\/}, which is a list consisting of the
class followed by its direct superclasses in the order mentioned
in the defining form.

A class precedence list is always consistent with the local precedence
order of each class in the list.  The classes in each local precedence
order appear within the class precedence list in the same order.  If
the local precedence orders are inconsistent with each other, no class
precedence list can be constructed, and an error is signaled.
The class precedence list and its computation is discussed
in section \ref{Determining-the-Class-Precedence-List-SECTION}.

Classes are organized into a {\bit directed acyclic graph}.  There are
two distinguished classes, named \cdf{t} and \cdf{standard-object}.
The class named \cdf{t} has no superclasses.  It is a superclass of
every class except itself.  The class named \cdf{standard-object} is
an instance of the class \cdf{standard-class} and is a superclass of
every class that is an instance of \cdf{standard-class} except itself.

There is a mapping from the Common Lisp Object System class space into
the Common Lisp type space.  Many of the standard Common Lisp types
have a corresponding
class that has the same name as the type.  Some Common Lisp types do
not have a corresponding class.  The integration of the type and class
systems is discussed in section~\ref{Integrating-Types-and-Classes-SECTION}.

Classes are represented by objects that are themselves
instances of classes.  The class of the class of an object is termed
the {\bit metaclass\/} of that object.  When no misinterpretation is
possible, the term {\bit metaclass\/} will be used to refer to a class
that has instances that are themselves classes.  The metaclass
determines the form of inheritance used by the classes that are its
instances and the representation of the instances of those classes.
The \CLOS\ provides a default metaclass, \cdf{standard-class}, that is
appropriate for most programs.  The meta-object protocol provides
mechanisms for defining and using new metaclasses.

Except where otherwise specified, all classes mentioned in this
chapter are instances of the class \cdf{standard-class}, all generic
functions are instances of the class \cdf{standard-generic-function},
and all methods are instances of the class \cdf{standard-method}.

\subsubsection{Defining Classes}

The macro \cdf{defclass} is used to define a new named class.
The definition of a class includes the following:

\begin{itemize}

\item  The name of the new class. For newly defined classes
this is a proper name.

\item  The list of the direct superclasses of the new class. 

\item  A set of {\bit slot specifiers}.  Each slot specifier
includes the name of the slot and zero or more {\bit slot options}.  A
slot option pertains only to a single slot. If a class definition
contains two slot specifiers with the same name, an error is signaled.

\item  A set of {\bit class options}.  Each class option pertains 
to the class as a whole.  
\end{itemize}
The slot options and class options of the \cdf{defclass} form provide
mechanisms for the following:

\begin{itemize}

\item  Supplying a default initial value form for a given slot.  

\item  Requesting that methods for generic functions
be automatically generated for reading or writing slots. 

\item  Controlling whether a given slot is shared by instances
of the class or whether each instance of the class has its own slot.

\item  Supplying a set of initialization arguments and initialization
argument defaults to be used in instance creation.

%\item  Requesting that a constructor function be automatically
%generated for making instances of the new class.

\item  Indicating that the metaclass is to be other than the default.

\item  Indicating the expected type for the value stored in the slot.

\item  Indicating the documentation string for the slot.

\end{itemize} 

\subsubsection{Creating Instances of Classes}

The generic function \cdf{make-instance} creates and returns a new
instance of a class.  The \OS\ provides several mechanisms for
specifying how a new instance is to be initialized.  For example, it
is possible to specify the initial values for slots in newly created
instances either by giving arguments to \cdf{make-instance} or by
providing default initial values.

Further initialization activities
can be performed by methods written for generic functions that are
part of the initialization protocol.  The complete initialization
protocol is described in section~\ref{Object-Creation-and-Initialization-SECTION}.

\subsubsection{Slots}

An object that has \cdf{standard-class} as its metaclass has zero or
more named slots.  The slots of an object are determined by the class
of the object.  Each slot can hold one value.  The name of a slot is a
symbol that is syntactically valid for use as a variable
name.

When a slot does not have a value, the slot is said to be {\bit
unbound}.  When an unbound slot is read, the generic
function \cdf{slot-unbound} is invoked. The system-supplied primary method
for \cdf{slot-unbound} signals an error.

The default initial value form for a slot is defined by the \hbox{
\cd{:initform}} slot option. When the \cd{:initform} form is used to
supply a value, it is evaluated in the lexical environment in which
the \cdf{defclass} form was evaluated. The \cd{:initform} along with
the lexical environment in which the \cdf{defclass} form was evaluated
is called a {\bit captured\/} \cd{:initform}.
See section~\ref{Object-Creation-and-Initialization-SECTION}.

A {\bit local slot\/} is defined to be a slot that is visible to exactly
one instance, namely the one in which the slot is allocated.  A {\bit
shared slot\/} is defined to be a slot that is visible to more than one
instance of a given class and its subclasses.

A class is said to {\bit define\/} a slot with a given name when
the \cdf{defclass} form for that class contains a slot specifier with
that name.  Defining a local slot does not immediately create a slot;
it causes a slot to be created each time an instance of the class is
created.  Defining a shared slot immediately creates a slot.

The \cd{:allocation} slot option to \cdf{defclass} controls the kind
of slot that is defined.  If the value of the \cd{:allocation} slot
option is \cd{:instance}, a local slot is created.  If the value of
\cd{:allocation} is \cd{:class}, a shared slot is created.

A slot is said to be {\bit accessible\/} in an instance of a class if
the slot is defined by the class of the instance or is inherited from\vadjust{\penalty-10000}
a superclass of that class.  At most one slot of a given name can be
accessible in an instance.  A shared slot defined by a class is
accessible in all instances of that class.  A detailed explanation of
the inheritance of slots is given in
section~\ref{Inheritance-of-Slots-and-Slot-Options-SECTION}.

\subsubsection{Accessing Slots}

Slots can be accessed in two ways: by use of the primitive function
\cdf{slot-value} and by use of generic functions generated by the 
\cdf{defclass} form.

The function \cdf{slot-value} can be used with any slot name
specified in the \cdf{defclass} form to access a specific slot
accessible in an instance of the given class.

The macro \cdf{defclass} provides syntax for generating methods to
read and write slots.  If a {\bit reader\/} is requested, a method is
automatically generated for reading the value of the slot, but no
method for storing a value into it is generated.  If a {\bit writer\/}
is requested, a method is automatically generated for storing a value
into the slot, but no method for reading its value is generated.  If
an {\bit accessor\/} is requested, a method for reading the value of
the slot and a method for storing a value into the slot are
automatically generated.  Reader and writer methods are implemented
using \cdf{slot-value}.

When a reader or writer is specified for a slot, the name of the
generic function to which the generated method belongs is directly
specified.  If the name specified for the writer option is the symbol
{\it name}, the name of the generic function for writing the slot
is the symbol {\it name}, and the generic function takes two
arguments: the new value and the instance, in that order.  If the name
specified for the accessor option is the symbol {\it name}, the
name of the generic function for reading the slot is the symbol {\it
name\/}, and the name of the generic function for writing the slot is
the list \cd{(setf {\it name\/})}.

A generic function created or modified by supplying reader, writer, or
accessor slot options can be treated exactly as an ordinary generic
function.

Note that \cdf{slot-value} can be used to read or write the value of a
slot whether or not reader or writer methods exist for that slot.
When \cdf{slot-value} is used, no reader or writer methods are
invoked.

The macro \cdf{with-slots} can be used to establish a lexical
environment in which specified slots are lexically available as if they
were variables.  The macro \cdf{with-slots} invokes the function 
\cdf{slot-value} to access the specified slots.

The macro \cdf{with-accessors} can be used to establish a lexical
environment in which specified slots are lexically available through
their accessors as if they were variables.  The macro 
\cdf{with-accessors} invokes the appropriate accessors to access the
specified slots. Any accessors specified by \cdf{with-accessors} must
already have been defined before they are used.

\penalty-10000 %required

\subsection{Inheritance}
\label{Inheritance-SECTION}

A class can inherit methods, slots, and some \cdf{defclass} options
from its superclasses.  The following sections describe the inheritance of
methods, the inheritance of slots and slot options, and the inheritance of
class options.
 
\subsubsection{Inheritance of Methods}
\label{Inheritance-of-Methods-SECTION}

A subclass inherits methods in the sense that any method applicable to
all instances of a class is also applicable to all instances of any
subclass of that class.

The inheritance of methods acts the same way regardless of whether the
method was created by using one of the method-defining forms or by
using one of the \cdf{defclass} options that causes methods to be
generated automatically.

The inheritance of methods is described in detail in
section~\ref{Method-Selection-and-Combination-SECTION}.


\subsubsection{Inheritance of Slots and Slot Options}
\label{Inheritance-of-Slots-and-Slot-Options-SECTION}

The set of names of all slots accessible in an instance of a class
{\it C} is the union of the sets of names of slots defined by {\it C} and its
superclasses. The {\bit structure} of an instance is the set of names
of local slots in that instance.

In the simplest case, only one class among {\it C} and its superclasses
defines a slot with a given slot name.  If a slot is defined by a
superclass of {\it C}, the slot is said to be {\bit
inherited}.  The characteristics of the slot are determined by the
slot specifier of the defining class.  Consider the defining class for
a slot {\it S}.  If the value of the \cd{:allocation} slot
option is \cd{:instance}, then {\it S} is a local slot and each instance
of {\it C} has its own slot named {\it S} that stores its own value.  If the
value of the \cd{:allocation} slot option is \cd{:class}, then {\it S}
is a shared slot, the class that defined {\it S} stores the value, and all
instances of {\it C} can access that single slot.  If the 
\cd{:allocation} slot option is omitted, \cd{:instance} is used.

In general, more than one class among {\it C} and its superclasses can
define a slot with a given name.  In such cases, only one slot with
the given name is accessible in an instance of {\it C}, and
the characteristics of that slot are a combination of the several slot
specifiers, computed as follows:

\begin{itemize}

\item  All the slot specifiers for a given slot name are ordered
from most specific to least specific, according to the order in {\it C\/}'s
class precedence list of the classes that define them. All references
to the specificity of slot specifiers immediately following refer to this
ordering.

\item  The allocation of a slot is controlled by the most specific
slot specifier.  If the most specific slot specifier does not contain an
\cd{:allocation} slot option, \cd{:instance} is used.  Less specific
slot specifiers do not affect the allocation.

\item  The default initial value form for a
slot is the value of the \cd{:initform} slot option in the most
specific slot specifier that contains one.  If no slot specifier
contains an \cd{:initform} slot option, the slot has no default
initial value form.

\item  The contents of a slot will always be of type 
\cd{(and ${\it T}\sub 1$ $\ldots$ ${\it T}\sub {\hbox{\scriptsize\it n}}$)}
where ${\it T}\sub 1, \ldots, T\sub {\hbox{\scriptsize\it n}}$ are
the values of the \cd{:type} slot options contained in all of the slot
specifiers.  If no slot specifier contains the \cd{:type} slot option, the
contents of the slot will always be of type \cdf{t}. The result
of attempting to store in a slot
a value that does not satisfy the type of the slot is undefined.

\item  The set of initialization arguments that initialize a given
slot is the union of the initialization arguments declared in the 
\cd{:initarg} slot options in all the slot specifiers.

\item  The documentation string for a slot is the value of the
\cd{:documentation} slot option in the most specific slot specifier
that contains one.  If no slot specifier contains a 
\cd{:documentation} slot option, the slot has no documentation string.
\end{itemize}

A consequence of the allocation rule is that a shared slot can be
shadowed.  For example, if a class ${\it C}\sub 1$ defines a slot named {\it S}
whose value for the \cd{:allocation} slot option is \cd{:class},
that slot is accessible in instances of ${\it C}\sub 1$ and all of its
subclasses.  However, if ${\it C}\sub 2$ is a subclass of ${\it C}\sub 1$ and also
defines a slot named {\it S}, ${\it C}\sub 1$'s slot is not shared
by instances of ${\it C}\sub 2$ and its subclasses. When a class
${\it C}\sub 1$ defines a shared slot, any subclass ${\it C}\sub 2$ of ${\it C}\sub
1$ will share this single slot unless the \cdf{defclass} form for
${\it C}\sub 2$ specifies a slot of the same name or there is a superclass
of ${\it C}\sub 2$ that precedes ${\it C}\sub 1$ in the class precedence list of
${\it C}\sub 2$ that defines a slot of the same name.

A consequence of the type rule is that the value of a slot satisfies
the type constraint of each slot specifier that contributes to that
slot.  Because the result of attempting to store in a slot a value
that does not satisfy the type constraint for the slot is undefined,
the value in a slot might fail to satisfy its type constraint.

The \cd{:reader}, \cd{:writer}, and \cd{:accessor} slot options
create methods rather than define the characteristics of a slot.
Reader and writer  methods are inherited in the sense described in
section~\ref{Inheritance-of-Methods-SECTION}. 

Methods that access slots use only the name of the slot and the type
of the slot's value.  Suppose a superclass provides a method that
expects to access a shared slot of a given name, and a subclass defines
a local slot with the same name.  If the method provided by the
superclass is used on an instance of the subclass, the method accesses
the local slot.

\subsubsection{Inheritance of Class Options}

The \cd{:default-initargs} class option is inherited.  The set of
defaulted initialization arguments for a class is the union of the
sets of initialization arguments specified in the 
\cd{:default-initargs} class options of the class and its superclasses.
When more than one default initial value form is supplied for a given
initialization argument, the default initial value form that is used
is the one supplied by the class that is most specific according to
the class precedence list.


If a given \cd{:default-initargs} class option specifies an
initialization argument of the same name more than once, an
error is signaled.

\subsubsection{Examples}

\begin{lisp}
(defclass C1 () \\*
~~((S1 :initform 5.4 :type number) \\*
~~~(S2 :allocation :class))) \\
\\
(defclass C2 (C1) \\*
~~((S1 :initform 5 :type integer)\\*
~~~(S2 :allocation :instance)\\*
~~~(S3 :accessor C2-S3)))
\end{lisp}

Instances of the class \cd{C1} have a local slot named \cd{S1}, whose default
initial value is 5.4 and whose value should always be a number.
The class \cd{C1} also has a shared slot named \cd{S2}.

There is a local slot named \cd{S1} in instances of \cd{C2}.  The
default initial value of \cd{S1} is 5.  The value of \cd{S1} will be
of type \cd{(and integer number)}.  There are also local slots named
\cd{S2} and \cd{S3} in instances of \cd{C2}.  The class \cd{C2}
has a method for \cd{C2-S3} for reading the value of slot \cd{S3};
there is also a method for \cd{(setf C2-S3)} that writes the
value of \cd{S3}.


\subsection{Integrating Types and Classes} 
\label{Integrating-Types-and-Classes-SECTION} 

The \CLOS\ maps the space of classes into the Common Lisp type space.
Every class that has a proper name has a corresponding type with the same 
name.  

The proper name of every class is a valid type specifier.  In
addition, every class object is a valid type specifier.  Thus the
expression \cd{(typep {\it object class\/})} evaluates to true if the
class of {\it object\/} is {\it class\/} itself or a subclass of {\it
class}.  The evaluation of the expression \cd{(subtypep {\it class1
class2\/})} returns the values \cdf{t}~and~\cdf{t} if {\it class1\/} is a
subclass of {\it class2\/} or if they are the same class; otherwise it
returns the values \cdf{nil}~and~\cdf{t}.  If {\it I} is an instance of some class
{\it C} named {\it S} and {\it C} is an instance of \cdf{standard-class}, the
evaluation of the expression \cd{(type-of {\it I\/})} will return {\it S} if
{\it S} is the proper name of {\it C\/}; if {\it S} is not the proper
name of {\it C}, the expression \cd{(type-of {\it I\/})} will
return {\it C}.

Because the names of classes and class objects are type specifiers, they may
be used in the special form \cdf{the} and in type declarations.

Many but not all of the predefined Common Lisp type specifiers have a
corresponding class with the same proper name as the type.  These type
specifiers are listed in table~\ref{CLOS-PRECEDENCE-TABLE}.  For example, the
type \cdf{array} has a corresponding class named \cdf{array}.  No type specifier
that is a list, such as \cd{(vector double-float 100)}, has a corresponding
class. The form \cdf{deftype} does not create any classes.

Each class that corresponds to a predefined Common Lisp type specifier
can be implemented in one of three ways, at the discretion of each
implementation.  It can be a {\bit standard class\/} (of the kind
defined by \cdf{defclass}), a {\bit structure class\/} (defined
by \cdf{defstruct}), or a {\bit built-in class\/} (implemented in
a special, non-extensible way).

A built-in class is one whose instances have restricted capabilities or
special representations.  Attempting to use \cdf{defclass} to define 
subclasses of a built-in class signals an error.  Calling 
\cdf{make-instance} to create an instance of a built-in class signals an error.
Calling \cdf{slot-value} on an instance of a built-in class signals an
error.  Redefining a built-in class or using \cdf{change-class} to change
the class of an instance to or from a built-in class signals an error.
However, built-in classes can be used as parameter specializers in
methods.

%The \OS\ specifies that all predefined Common Lisp type specifiers
%listed in table~\ref{CLOS-PRECEDENCE-TABLE} are built-in classes, but a particular
%implementation is allowed to extend the \OS\ to define some of them as
%standard classes or as structure classes.

It is possible to determine whether a class is a built-in class by
checking the metaclass.  A standard class is an instance of 
\cdf{standard-class}, a built-in class is an instance of 
\cdf{built-in-class}, and a structure class is an instance of 
\cdf{structure-class}.

Each structure type created by \cdf{defstruct} without using the 
\cd{:type} option has a corresponding class.  This class is an instance of
\cdf{structure-class}.  
%A portable program must assume that 
%\cdf{structure-class} is a subclass of \cdf{built-in-class} and has the
%same restrictions as built-in classes.  Whether \cdf{structure-class}
%in fact is a subclass of \cdf{built-in-class} is
%implementation-dependent. 
The \cd{:include} option of \cdf{defstruct} creates a direct
subclass of the class that corresponds to the included structure.

The purpose of specifying that many of the standard Common Lisp type
specifiers have a corresponding class is to enable users to write methods that
discriminate on these types.  
Method selection requires that a class precedence list can be
determined for each class. 

The hierarchical relationships among the Common Lisp type specifiers
are mirrored by relationships among the classes corresponding to those
types.  The existing type hierarchy is used for determining the
class precedence list for each class that corresponds to a predefined
Common Lisp type.  In some cases, the first edition
did not specify a local precedence order for two supertypes of a
given type specifier.  For example, \cdf{null} is a subtype of both
\cdf{symbol} and \cdf{list}, but the first edition
did not specify whether \cdf{symbol} is more specific or less
specific than \cdf{list}.  The CLOS specification defines those
relationships for all such classes.

Table~\ref{CLOS-PRECEDENCE-TABLE} lists the set of classes required by the \OS\
that correspond to predefined Common Lisp type specifiers.  The
superclasses of each such class are presented in order from most
specific to most general, thereby defining the class precedence list
for the class. The local precedence order for each class that
corresponds to a Common Lisp type specifier can be derived from this
table.

Individual implementations may be extended to define other type
specifiers to have a corresponding class.  Individual implementations
can be extended to add other subclass relationships and to add other
elements to the class precedence lists in the above table as long as
they do not violate the type relationships and disjointness
requirements specified in section~\ref{DATA-TYPE-RELATIONSHIPS}.
A standard class defined with no direct superclasses is guaranteed to
be disjoint from all of the classes in the table, except for the
class named \cdf{t}.

[At this point the original CLOS report specified that certain Common Lisp
types were to appear in table~\ref{CLOS-PRECEDENCE-TABLE} if and only if
X3J13 voted to make them disjoint from
\cdf{cons}, \cdf{symbol}, \cdf{array}, \cdf{number}, and \cdf{character}.
X3J13 voted to do so in June 1988
\issue{DATA-TYPES-HIERARCHY-UNDERSPECIFIED}.  I have added these types
and their class precedence lists to the table; the new types are indicated
by asterisks.---GLS]


\begin{table}[t]
\caption{Class Precedence Lists for Predefined Types}
\label{CLOS-PRECEDENCE-TABLE}
\begin{flushleft}
\cf
\begin{tabular}{@{}ll@{}}
{\rm Predefined Common Lisp Type}&{\rm Class Precedence List for Corresponding Class} \\
\hlinesp
array&(array t)\\
bit-vector&(bit-vector vector array sequence t)\\
character&(character t)\\
complex&(complex number t)\\
cons&(cons list sequence t)\\
float&(float number t)\\
function {\rm *}&(function t) \\
hash-table {\rm *}&(hash-table t) \\
integer&(integer rational number t)\\
list&(list sequence t)\\
null&(null symbol list sequence t)\\
number&(number t)\\
package {\rm *}&(package t) \\
pathname {\rm *}&(pathname t) \\
random-state {\rm *}&(random-state t) \\
ratio&(ratio rational number t)\\
rational&(rational number t)\\
readtable {\rm *}&(readtable t) \\
sequence&(sequence t)\\
stream {\rm *}&(stream t) \\
string&(string vector array sequence t)\\
symbol&(symbol t)\\
t&(t)\\
vector&(vector array sequence t)
\end{tabular}
\end{flushleft}
[An asterisk indicates a type added to this table as a consequence
of a portion of the CLOS specification that was conditional on X3J13 voting
to make that type disjoint from certain other built-in types
\issue{DATA-TYPES-HIERARCHY-UNDERSPECIFIED}.---GLS]
\end{table}


\subsection{Determining the Class Precedence List}
\label{Determining-the-Class-Precedence-List-SECTION}

The \cdf{defclass} form for a class provides a total ordering on that
class and its direct superclasses.  This ordering is called the {\bit
local precedence order}.  It is an ordered list of the class and its
direct superclasses. The {\bit class precedence list\/} for a
class {\it C} is a total ordering on {\it C} and its superclasses that is consistent
with the local precedence orders for {\it C} and its superclasses.

A class precedes its direct superclasses, and a
direct superclass precedes all other direct superclasses specified to
its right in the superclasses list of the \cdf{defclass} form.  For
every class {\it C}, define
$${\it R}\sub {\it C}=\{({\it C},{\it C}\sub 1),({\it C}\sub 1,{\it C}\sub 2),\ldots,({\it C}\sub {\hbox{\scriptsize\it n}-1},{\it C}\sub {\hbox{\scriptsize\it n}})\}$$
where ${\it C}\sub 1,\ldots,{\it C}\sub {\hbox{\scriptsize\it n}}$ are
the direct superclasses of {\it C} in the order in which
they are mentioned in the \cdf{defclass} form. These ordered pairs
generate the total ordering on the class {\it C} and its direct
superclasses.

Let ${\it S}\sub {\it C}$ be the set of {\it C} and its superclasses. Let {\it R} be
$${\it R}=\bigcup\sub{\textstyle {\it c}\in {{\it S}\sub {\hbox{\scriptsize\it C}}}} {\it R}\sub {\hbox{\scriptsize\it c}}$$

The set {\it R} may or may not generate a partial ordering, depending on
whether the ${\it R}\sub {\hbox{\scriptsize\it c}}$, ${\it c}\in {\it S}\sub {\hbox{\scriptsize\it C}}$,
are consistent; it is assumed
that they are consistent and that {\it R} generates a partial ordering.
When the ${\it R}\sub {\hbox{\scriptsize\it c}}$ are not consistent, it is said that {\it R} is inconsistent.


To compute the class precedence list for~{\it C},
topologically sort the elements of ${\it S}\sub {\hbox{\scriptsize\it C}}$ with respect to the
partial ordering generated by {\it R}.  When the topological
sort must select a class from a set of two or more classes, none of
which are preceded by other classes with respect to~{\it R},
the class selected is chosen deterministically, as described below.
If {\it R} is inconsistent, an error is signaled.

\penalty-10000 %required


\subsubsection{Topological Sorting}

Topological sorting proceeds by finding a class {\it C} in~${\it S}\sub {\hbox{\scriptsize\it C}}$ such
that no other class precedes that element according to the elements
in~{\it R}.  The class {\it C} is placed first in the result.
Remove {\it C} from ${\it S}\sub {\hbox{\scriptsize\it C}}$, and remove all pairs of the form
$({\it C},{\it D})$,
${\it D}\in {\it S}\sub {\hbox{\scriptsize\it C}}$, from {\it R}. Repeat the process, adding
classes with no predecessors to the end of the result.  Stop when no
element can be found that has no predecessor.

If ${\it S}\sub {\hbox{\scriptsize\it C}}$ is not empty and the process has stopped, the set {\it R} is
inconsistent. If every class in the finite set of classes is preceded
by another, then {\it R} contains a loop. That is, there is a chain of
classes ${\it C}\sub 1,\ldots,C\sub {\hbox{\scriptsize\it n}}$
such that ${\it C}\sub {\hbox{\scriptsize\it i}}$ precedes
${\it C}\sub{\hbox{\scriptsize\it i}+1}$, $1\leq {\it i}<{\it n}$,
and ${\it C}\sub {\hbox{\scriptsize\it n}}$ precedes ${\it C}\sub 1$.

Sometimes there are several classes from ${\it S}\sub {\hbox{\scriptsize\it C}}$ with no
predecessors.  In this case select the one that has a direct
subclass rightmost in the class precedence list computed so far.
%%%%%%% RPG said to take out the following sentence 10/5/89.
%Because a direct superclass precedes all other direct superclasses to
%its right, there can be only one such candidate class.
If there is no
such candidate class, {\it R} does not generate a partial ordering---the
${\it R}\sub {\hbox{\scriptsize\it c}}$, ${\it c}\in {\it S}\sub {\hbox{\scriptsize\it C}}$, are inconsistent.

In more precise terms, let $\{{\it N}\sub 1,\ldots,{\it N}\sub {\hbox{\scriptsize\it m}}\}$,
${\it m}\geq 2$, be
the classes from ${\it S}\sub {\hbox{\scriptsize\it C}}$ with no predecessors.  Let $({\it C}\sub
1\ldots {\it C}\sub {\hbox{\scriptsize\it n}})$, ${\it n}\geq 1$, be the class precedence list
constructed so far.  ${\it C}\sub 1$ is the most specific class, and ${\it C}\sub
{\it n}$ is the least specific.  Let $1\leq {\it j}\leq {\it n}$ be the largest number
such that there exists an {\it i} where $1\leq {\it i}\leq {\it m}$ and
${\it N}\sub {\hbox{\scriptsize\it i}}$
is a direct superclass of ${\it C}\sub {\hbox{\scriptsize\it j}}$;
${\it N}\sub {\hbox{\scriptsize\it i}}$ is placed next.

The effect of this rule for selecting from a set of classes with no
predecessors is that classes in a simple superclass chain are
adjacent in the class precedence list and that classes in each
relatively separated subgraph are adjacent in the class
precedence list. For example, let ${\it T}\sub 1$ and ${\it T}\sub 2$ be subgraphs
whose only element in common is the class {\it J}. Suppose
that no superclass of {\it J} appears in either ${\it T}\sub 1$ or ${\it T}\sub 2$.
Let ${\it C}\sub 1$ be the bottom of ${\it T}\sub 1$; and let ${\it C}\sub 2$ be the
bottom of ${\it T}\sub 2$.  Suppose {\it C} is a class whose direct superclasses
are ${\it C}\sub 1$ and ${\it C}\sub 2$ in that order; then the class precedence
list for {\it C} will start with {\it C} and will be followed by all classes
in ${\it T}\sub 1$ except {\it J}. All the classes of ${\it T}\sub 2$ will be next.
The class {\it J} and its superclasses will appear last.


\subsubsection{Examples}

This example determines a class precedence list for the
class \cdf{pie}.  The following classes are defined:

\begin{lisp}
(defclass pie (apple cinnamon) ()) \\*
(defclass apple (fruit) ()) \\*
(defclass cinnamon (spice) ()) \\
(defclass fruit (food) ()) \\
(defclass spice (food) ()) \\*
(defclass food () ())
\end{lisp}

\begin{flushleft}\parindent 1em\relax
\newbox\Qhyphbox
\setbox\Qhyphbox\hbox{\cdf{-}}
\def\Qhyphen{\copy\Qhyphbox}

\setbox0\hbox{\hskip\textwidth\hskip 1pt\vrule height 10pt depth 490pt width 0.25pt}
\ht0=0pt \dp0=0pt \wd0=0pt \relax
\leavevmode\box0 The set
${\it S}=\{\cdf{pie},\discretionary{}{}{}
\cdf{apple},\discretionary{}{}{}
\cdf{cinnamon},\discretionary{}{}{}
\cdf{fruit},\discretionary{}{}{}
\cdf{spice},\discretionary{}{}{}
\cdf{food},\discretionary{}{}{}
\cd{standard{\Qhyphen}\discretionary{}{}{}object},\discretionary{}{}{}
\cdf{t}\}$.
The set ${\it R}=\{(\cdf{pie},\discretionary{}{}{}
\cdf{apple}),\discretionary{}{}{}
(\cdf{apple},\discretionary{}{}{}
\cdf{cinnamon}),\discretionary{}{}{}
(\cdf{cinnamon},\discretionary{}{}{}
\cd{standard{\Qhyphen}\discretionary{}{}{}object}),\discretionary{}{}{}
(\cdf{apple},\discretionary{}{}{}
\cdf{fruit}),\discretionary{}{}{}
(\cdf{fruit},\discretionary{}{}{}
\cd{standard{\Qhyphen}\discretionary{}{}{}object}),\discretionary{}{}{}
(\cdf{cinnamon},\discretionary{}{}{}
\cdf{spice}),\discretionary{}{}{}
(\cdf{spice},\discretionary{}{}{}
\cd{standard{\Qhyphen}\discretionary{}{}{}object}),\discretionary{}{}{}
(\cdf{fruit},\discretionary{}{}{}
\cdf{food}),\discretionary{}{}{}
(\cdf{food},\discretionary{}{}{}
\cd{standard{\Qhyphen}\discretionary{}{}{}object}),\discretionary{}{}{}
(\cdf{spice},\discretionary{}{}{}
\cdf{food}),\discretionary{}{}{}
(\cd{standard{\Qhyphen}\discretionary{}{}{}object},\discretionary{}{}{}
\cdf{t})\}$.

[The original CLOS specification~\cite{SIGPLAN-CLOS,LASC-CLOS-PART-1}
contained a minor error in this example: the pairs
$(\cdf{cinnamon},\discretionary{}{}{}
\cd{standard{\Qhyphen}\discretionary{}{}{}object})$,
$(\cdf{fruit},\discretionary{}{}{}
\cd{standard{\Qhyphen}\discretionary{}{}{}object})$, and
$(\cdf{spice},\discretionary{}{}{}
\cd{standard{\Qhyphen}\discretionary{}{}{}object})$
were inadvertently omitted from {\it R} in the preceding paragraph.
It is important to understand that \cdf{defclass} implicitly appends the
class \cdf{standard-object} to the list of superclasses when the metaclass
is \cdf{standard-class} (the normal situation),
in order to insure that \cdf{standard-object} will be a superclass
of every instance of \cdf{standard-class} except \cdf{standard-object} itself
(see section~\ref{Classes-SECTION}).
${\it R}\sub {\hbox{\scriptsize\it c}}$ is then generated from this augmented list of superclasses;
this is where the extra pairs come from.  I~have corrected the example
by adding these pairs as appropriate throughout the example.  The final result,
the class precedence list for \cdf{pie},
is unchanged.---GLS]

The class \cdf{pie} is not preceded by anything, so it comes first;
the result so far is \cd{(pie)}.  Remove \cdf{pie} from {\it S} and pairs
mentioning \cdf{pie} from {\it R} to get
${\it S}=\{\cdf{apple},\discretionary{}{}{}
\cdf{cinnamon},\discretionary{}{}{}
\cdf{fruit},\discretionary{}{}{}
\cdf{spice},\discretionary{}{}{}
\cdf{food},\discretionary{}{}{}
\cd{standard{\Qhyphen}\discretionary{}{}{}object},\discretionary{}{}{}
\cdf{t}\}$ and ${\it R}=\{
(\cdf{apple},\discretionary{}{}{}
\cdf{cinnamon}),\discretionary{}{}{}
(\cdf{cinnamon},\discretionary{}{}{}
\cd{standard{\Qhyphen}\discretionary{}{}{}object}),\discretionary{}{}{}
(\cdf{apple},\discretionary{}{}{}
\cdf{fruit}),\discretionary{}{}{}
(\cdf{fruit},\discretionary{}{}{}
\cd{standard{\Qhyphen}\discretionary{}{}{}object}),\discretionary{}{}{}
(\cdf{cinnamon},\discretionary{}{}{}
\cdf{spice}),\discretionary{}{}{}
(\cdf{spice},\discretionary{}{}{}
\cd{standard{\Qhyphen}\discretionary{}{}{}object}),\discretionary{}{}{}
(\cdf{fruit},\discretionary{}{}{}
\cdf{food}),\discretionary{}{}{}
(\cdf{food},\discretionary{}{}{}
\cd{standard{\Qhyphen}\discretionary{}{}{}object}),\discretionary{}{}{}
(\cdf{spice},\discretionary{}{}{}
\cdf{food}),\discretionary{}{}{}
(\cd{standard{\Qhyphen}\discretionary{}{}{}object},\discretionary{}{}{}
\cdf{t})\}$.

The class \cdf{apple} is not preceded by anything, so it is next; the
result is \cd{(pie~apple)}. Removing \cdf{apple} and the relevant
pairs results in ${\it S}=\{\cdf{cinnamon},\discretionary{}{}{}
\cdf{fruit},\discretionary{}{}{}
\cdf{spice},\discretionary{}{}{}
\cdf{food},\discretionary{}{}{}
\cd{standard{\Qhyphen}\discretionary{}{}{}object},\discretionary{}{}{}
\cdf{t}\}$ and ${\it R}=\{(\cdf{cinnamon},\discretionary{}{}{}
\cd{standard{\Qhyphen}\discretionary{}{}{}object}),\discretionary{}{}{}
(\cdf{fruit},\discretionary{}{}{}
\cd{standard{\Qhyphen}\discretionary{}{}{}object}),\discretionary{}{}{}
(\cdf{cinnamon},\discretionary{}{}{}
\cdf{spice}),\discretionary{}{}{}
(\cdf{spice},\discretionary{}{}{}
\cd{standard{\Qhyphen}\discretionary{}{}{}object}),\discretionary{}{}{}
(\cdf{fruit},\discretionary{}{}{}
\cdf{food}),\discretionary{}{}{}
(\cdf{food},\discretionary{}{}{}
\cd{standard{\Qhyphen}\discretionary{}{}{}object}),\discretionary{}{}{}
(\cdf{spice},\discretionary{}{}{}
\cdf{food}),\discretionary{}{}{}
(\cd{standard{\Qhyphen}\discretionary{}{}{}object},\discretionary{}{}{}
\cdf{t})\}$.

The classes \cdf{cinnamon} and \cdf{fruit} are not preceded by
anything, so the one with a direct subclass rightmost in the class
precedence list computed so far goes next.  The class \cdf{apple} is a
direct subclass of \cdf{fruit}, and the class \cdf{pie} is a direct
subclass of \cdf{cinnamon}.  Because \cdf{apple} appears to the right
of \cdf{pie} in the precedence list, \cdf{fruit} goes next, and the
result so far is \cd{(pie apple fruit)}.  ${\it S}=\{\cdf{cinnamon},\discretionary{}{}{}
\cdf{spice},\discretionary{}{}{}
\cdf{food},\discretionary{}{}{}
\cd{standard{\Qhyphen}\discretionary{}{}{}object},\discretionary{}{}{}
\cdf{t}\}$; ${\it R}=\{(\cdf{cinnamon},\discretionary{}{}{}
\cd{standard{\Qhyphen}\discretionary{}{}{}object}),\discretionary{}{}{}
(\cdf{cinnamon},\discretionary{}{}{}
\cdf{spice}),\discretionary{}{}{}
(\cdf{spice},\discretionary{}{}{}
\cd{standard{\Qhyphen}\discretionary{}{}{}object}),\discretionary{}{}{}
(\cdf{food},\discretionary{}{}{}
\cd{standard{\Qhyphen}\discretionary{}{}{}object}),\discretionary{}{}{}
(\cdf{spice},\discretionary{}{}{}
\cdf{food}),\discretionary{}{}{}
(\cd{standard{\Qhyphen}\discretionary{}{}{}object},\discretionary{}{}{}
\cdf{t})\}$.

The class \cdf{cinnamon} is next, giving the result so far as 
\cd{(pie apple fruit cinnamon)}.  At this point ${\it S}=\{\cdf{spice},\discretionary{}{}{}
\cdf{food},\discretionary{}{}{}
\cd{standard{\Qhyphen}\discretionary{}{}{}object},\discretionary{}{}{}
\cdf{t}\}$; ${\it R}=\{(\cdf{spice},\discretionary{}{}{}
\cd{standard{\Qhyphen}\discretionary{}{}{}object}),\discretionary{}{}{}
(\cdf{food},\discretionary{}{}{}
\cd{standard{\Qhyphen}\discretionary{}{}{}object}),\discretionary{}{}{}
(\cdf{spice},\discretionary{}{}{}
\cdf{food}),\discretionary{}{}{}
(\cd{standard{\Qhyphen}\discretionary{}{}{}object},\discretionary{}{}{}
\cdf{t})\}$.
\end{flushleft}

The classes \cdf{spice}, \cdf{food}, \cdf{standard-object}, and 
\cdf{t} are then added in that order, and the final class precedence list for \cdf{pie} is
\begin{lisp}
(pie apple fruit cinnamon spice food standard-object t)
\end{lisp}

It is possible to write a set of class definitions that cannot be 
ordered.   For example: 

\begin{lisp}
(defclass new-class (fruit apple) ()) \\
(defclass apple (fruit) ())
\end{lisp}

The class \cdf{fruit} must precede \cdf{apple} because the local
ordering of superclasses must be preserved.  The class \cdf{apple} must
precede \cdf{fruit} because a class always precedes its own
superclasses.  When this situation occurs, an error is signaled when
the system tries to compute the class precedence list.

The following might appear to be a conflicting set of definitions:

\begin{lisp}
(defclass pie (apple cinnamon) ()) \\
(defclass pastry (cinnamon apple) ()) \\
(defclass apple () ()) \\
(defclass cinnamon () ())
\end{lisp}

The class precedence list for \cdf{pie} is
\begin{lisp}
(pie apple cinnamon standard-object t)
\end{lisp}

The class precedence list for \cdf{pastry} is
\begin{lisp}
(pastry cinnamon apple standard-object t)
\end{lisp}

It is not a problem for \cdf{apple} to precede \cdf{cinnamon} in the
ordering of the superclasses of \cdf{pie} but not in the ordering for
\cdf{pastry}.  However, it is not possible to build a new class that
has both \cdf{pie} and \cdf{pastry} as superclasses.


\subsection{Generic Functions and Methods}

A {\bit generic function\/} is a function whose behavior depends on
the classes or identities of the arguments supplied to it.  The {\bit
methods} define the class-specific behavior and operations of the
generic function. The following sections describe generic functions
and methods.

\subsubsection{Introduction to Generic Functions}

A generic function object contains a set of methods, a
lambda-list, a method combination type, and other information.

Like an ordinary Lisp function, a generic function takes arguments,
performs a series of operations, and perhaps returns useful values.
An ordinary function has a single body of code that is always executed
when the function is called.  A generic function has a set of bodies
of code of which a subset is selected for execution. The selected
bodies of code and the manner of their combination are determined by
the classes or identities of one or more of the arguments to the
generic function and by its method combination type.

Ordinary functions and generic functions are called with identical function-call
syntax.
 
Generic functions are true functions that can be passed as arguments, returned as values,
used as the first argument to \cdf{funcall} and \cdf{apply}, and otherwise used in all the ways
an ordinary function may be used.

A name can be given to an ordinary function in one of
two ways: a {\bit global\/} name can be given to a function using the
\cdf{defun} construct; a {\bit local\/} name can be given using the
\cdf{flet} or \cdf{labels} special forms.  A generic function can be
given a global name using the \cdf{defmethod} or \cdf{defgeneric}
construct.  A generic function can be given a local name using the
\cdf{generic-flet}, \cdf{generic-labels}, or \cdf{with-added-methods}
special forms.  The name of a generic function, like the name of an
ordinary function, can be either a symbol or a two-element list whose
first element is \cdf{setf} and whose second element is a symbol.
This is true for both local and global names.

The \cdf{generic-flet} special form creates new local generic
functions using the set of methods specified by the method definitions
in the \cdf{generic-flet} form.  The scoping of generic function names
within a \cdf{generic-flet} form is the same as for \cdf{flet}.

The \cdf{generic-labels} special form creates a set of new mutually
recursive local generic functions using the set of methods specified
by the method definitions in the \cdf{generic-labels} form.  The
scoping of generic function names within a \cdf{generic-labels} form
is the same as for \cdf{labels}.

The \cdf{with-added-methods} special form creates new local generic
functions by adding the set of methods specified by the method
definitions with a given name in the \cdf{with-added-methods} form to
copies of the methods of the lexically visible generic function of the
same name. If there is a lexically visible ordinary function of the
same name as one of the specified generic functions, that function
becomes the method function of the default method for the new generic
function of that name.

The \cdf{generic-function} macro creates an anonymous generic
function with the set of methods specified by the method definitions that appear
in the \cd{generic-\discretionary{}{}{}function} form.

When a \cdf{defgeneric} form is evaluated, one of three actions
is taken:

\begin{itemize}

\item  If a generic function of the given name already exists,
the existing generic function object is modified.  Methods specified
by the current \cdf{defgeneric} form are added, and any methods in the
existing generic function that were defined by a previous 
\cdf{defgeneric} form are removed.  Methods added by the current 
\cdf{defgeneric} form might replace methods defined by \cdf{defmethod} or
\cdf{defclass}.  No other methods in the generic function are affected
or replaced.

\item  If the given name names a non-generic function, a
macro, or a special form, an error is signaled.

\item  Otherwise a generic function is created with the
methods specified by the method definitions in the \cdf{defgeneric}
form.

\end{itemize}

Some forms specify the options of a generic function,
such as the type of method combination it uses or its argument
precedence order.  They will be referred to as ``forms that
specify generic function options.'' These forms are \cdf{defgeneric},
\cdf{generic-function}, \cdf{generic-flet}, \cdf{generic-labels}, and
\cdf{with-added-methods}.

Some forms define methods for a generic function.  They will be
referred to as ``method-defining forms.'' These forms are 
\cdf{defgeneric}, \cdf{defmethod}, \cdf{generic-function}, 
\cdf{generic-flet}, \cdf{generic-labels}, \cdf{with-added-methods}, and
\cdf{defclass}. Note that all the method-defining forms except 
\cdf{defclass} and \cdf{defmethod}
are also forms that specify generic function options.

\subsubsection{Introduction to Methods}
\label{Introduction-to-Methods-SECTION}

A method object contains a method function, a sequence of {\bit
parameter specializers\/} that specify when the given method is
applicable, a lambda-list, and a sequence of {\bit qualifiers\/} that
are used by the method combination facility to distinguish among
methods.

A method object is not a function and cannot be invoked as a function. 
Various mechanisms in the \OS\ take a method object and invoke its method
function, as is the case when a generic function is invoked.  When this
occurs it is said that the method is invoked or called.

A method-defining form contains the code that is to be run when the
arguments to the generic function cause the method that it defines to
be invoked.  When a method-defining form is evaluated, a method object
is created and one of four actions is taken:

\begin{itemize}

\item  If a generic function of the given name already exists
and if a method object already exists that agrees with the new one on
parameter specializers and qualifiers, the new method object replaces
the old one.  For a definition of one method agreeing with another on
parameter specializers and qualifiers, see
section~\ref{Agreement-on-Parameter-Specializers-and-Qualifiers-SECTION}.

\item  If a generic function of the given name already exists
and if there is no method object that agrees with the new one on
parameter specializers and qualifiers, the existing generic function
object is modified to contain the new method object.

\item  If the given name names a non-generic function, a macro,
or a special form, an error is signaled.

\item  Otherwise a generic function is created with the methods
specified by the method-defining form.

\end{itemize}

If the lambda-list of a new method is not congruent with the lambda-list
of the generic function, an error is signaled.  If a
method-defining form that cannot specify generic function options
creates a new generic function, a lambda-list for that generic
function is derived from the lambda-lists of the methods in the
method-defining form in such a way as to be congruent with them.  For
a discussion of {\bit congruence}, see
section~\ref{Congruent-Lambda-Lists-for-All-Methods-of-a-Generic-Function-SECTION}.

Each method has a {\bit specialized lambda-list}, which determines
when that method can be applied.  A specialized lambda-list is like
an ordinary lambda-list except that a {\bit specialized parameter\/}
may occur instead of the name of a required parameter.  A specialized parameter
is a list \cd{({\it variable-name parameter-specializer-name\/})},
where {\it parameter-specializer-name\/} is either
a name that names a class or a list \cd{(\cdf{eql} {\it form\/})}.
A parameter specializer name denotes a parameter specializer as follows:

\begin{itemize}
\item  A name that names a class denotes that class.

\item  The list \cd{(\cdf{eql} {\it form\/})} denotes the type specifier
\cd{(\cdf{eql} {\it object\/})}, where {\it object\/} is the result of
evaluating {\it form\/}.  The form {\it form\/} is evaluated in the
lexical environment in which the method-defining form is
evaluated.  Note that {\it form\/} is evaluated only once, at the time
the method is defined, not each time the generic function is called.
\end{itemize}

Parameter specializer names are used in macros intended as the
user-level interface (\cdf{defmethod}), while parameter specializers
are used in the functional interface.

[It is very important to understand clearly the distinction made
in the preceding paragraph.  A parameter specializer name
has the form of a type specifier but is semantically quite different
from a type specifier: a parameter specializer name of the form
\cd{(\cdf{eql} {\it form\/})} is not a type specifier, for it contains
a {\it form\/} to be evaluated.   Type specifiers
never contain forms to be evaluated.  All parameter specializers
(as opposed to parameter specializer names) are valid type specifiers,
but not all type specifiers are valid parameter specializers.  Macros such as \cdf{defmethod}
take parameter specializer names and treat them as specifications for
constructing certain type specifiers (parameter specializers) that may then be used
with such functions as \cdf{find-method}.---GLS]


Only required parameters may be specialized, and there must be a
parameter specializer for each required parameter.  For notational
simplicity, if some required parameter in a specialized lambda-list in
a method-defining form is simply a variable name, its parameter
specializer defaults to the class named \cdf{t}.

Given a generic function and a set of arguments, an {\bit applicable
method\/} is a method for that generic function whose parameter
specializers are satisfied by their corresponding arguments.  The
following definition specifies what it means for a method to be
applicable and for an argument to satisfy a parameter specializer.

Let $\langle {\it A}\sub 1, \ldots, {\it A}\sub {\hbox{\scriptsize\it n}}\rangle$ be the required
arguments to a generic function in order. Let $\langle {\it P}\sub 1,
\ldots, {\it P}\sub {\hbox{\scriptsize\it n}}\rangle$ be the parameter specializers corresponding to
the required parameters of the method {\it M} in order.  The method {\it M} is
{\bit applicable\/} when each ${\it A}\sub {\hbox{\scriptsize\it i}}$
{\bit satisfies\/} ${\it P}\sub {\hbox{\scriptsize\it i}}$.
If ${\it P}\sub {\hbox{\scriptsize\it i}}$ is a class,
and if ${\it A}\sub {\hbox{\scriptsize\it i}}$ is an instance of a class
{\it C}, then it is said that ${\it A}\sub {\hbox{\scriptsize\it i}}$ {\bit satisfies\/}
${\it P}\sub {\hbox{\scriptsize\it i}}$ when $C={\it P}\sub {\hbox{\scriptsize\it i}}$ or when {\it C} is a subclass of ${\it P}\sub {\hbox{\scriptsize\it i}}$.  If
${\it P}\sub {\hbox{\scriptsize\it i}}$ is of the form
\cd{(\cdf{eql} {\it object\/})}, then it is said that
${\it A}\sub {\hbox{\scriptsize\it i}}$ satisfies ${\it P}\sub {\hbox{\scriptsize\it i}}$
when the function \cdf{eql} applied to
${\it A}\sub {\hbox{\scriptsize\it i}}$ and {\it object} is true.

Because a parameter specializer is a type specifier, the function 
\cdf{typep} can be used during method selection to determine whether an
argument satisfies a parameter specializer.  In general a
parameter specializer cannot be a type specifier list, such as 
\cd{(\cd{vector single-float})}.  The only parameter specializer that can
be a list is \cd{(\cdf{eql} {\it object\/})}.  This requires that
Common Lisp define the type specifier \cdf{eql}
as if the following were evaluated:

\begin{lisp}
(deftype eql ({\it object\/}) {\Xbq}(member ,{\it object\/}))
\end{lisp}
[See section~\ref{PREDICATING-TYPE-SPECIFIERS-SECTION}.---GLS]

A method all of whose parameter specializers are the class named 
\cdf{t} is called a {\bit default method}; it is always applicable but
may be shadowed by a more specific method.

Methods can have {\bit qualifiers}, which give the method combination
procedure a way to distinguish among methods.  A method that has one
or more qualifiers is called a {\bit qualified\/} method.
A method with no qualifiers is called an {\bit unqualified method}. 
A qualifier is any object other than a list, that is,
any non-\cdf{nil} atom.  The qualifiers defined by standard method combination
and by the built-in method combination types are symbols.

In this specification, the terms {\bit primary method\/} and {\bit
auxiliary method\/} are used to partition methods within a method
combination type according to their intended use.  In standard method
combination, primary methods are unqualified methods, and auxiliary
methods are methods with a single qualifier that is one of 
\cd{:around}, \cd{:before}, or \cd{:after}.  When a method combination
type is defined using the short form of 
\cdf{define-method-combination}, primary methods are methods qualified with
the name of the type of method combination, and auxiliary methods have
the qualifier \cd{:around}.  Thus the terms {\bit primary method\/}
and {\bit auxiliary method\/} have only a relative definition within a
given method combination type.

\subsubsection{Agreement on Parameter Specializers and Qualifiers}
\label{Agreement-on-Parameter-Specializers-and-Qualifiers-SECTION}

Two methods are said to agree with each other on parameter specializers
and qualifiers if the following conditions hold:

\begin{itemize}

\item Both methods have the same number of required parameters.
Suppose the parameter specializers of the two methods are
${\it P}\sub{1,1}\ldots P\sub{1,\hbox{\scriptsize\it n}}$
and ${\it P}\sub{2,1}\ldots P\sub{2,\hbox{\scriptsize\it n}}$.

\item For each $1\leq {\it i}\leq {\it n}$,
${\it P}\sub{1,\hbox{\scriptsize\it i}}$ agrees with ${\it P}\sub{2,\hbox{\scriptsize\it i}}$.
The parameter specializer ${\it P}\sub{1,\hbox{\scriptsize\it i}}$
agrees with ${\it P}\sub{2,\hbox{\scriptsize\it i}}$ if
${\it P}\sub{1,\hbox{\scriptsize\it i}}$ and ${\it P}\sub{2,\hbox{\scriptsize\it i}}$ are the same class or if 
${\it P}\sub{1,\hbox{\scriptsize\it i}}=\hbox{{\tt(\cdf{eql} $\hbox{{\it object}}\sub 1$)}}$,
${\it P}\sub{2,\hbox{\scriptsize\it i}}=\hbox{{\tt(\cdf{eql} $\hbox{{\it object}}\sub 2$)}}$, and
\cd{(\cdf{eql} $\hbox{{\it object}}\sub 1$ $\hbox{{\it object}}\sub 2$)}.
Otherwise ${\it P}\sub{1,\hbox{\scriptsize\it i}}$ and ${\it P}\sub{2,\hbox{\scriptsize\it i}}$ do not agree.


\item The lists of qualifiers of both methods contain the same 
non-\cdf{nil} atoms in the same order. That is, the lists are \cdf{equal}.

\end{itemize}


\subsubsection{Congruent Lambda-Lists for All Methods of a \hfil\penalty-10000\relax Generic~Function}
\label{Congruent-Lambda-Lists-for-All-Methods-of-a-Generic-Function-SECTION}

These rules define the congruence of a set of lambda-lists, including the
lambda-list of each method for a given generic function and the
lambda-list specified for the generic function itself, if given.

\begin{itemize}

\item Each lambda-list must have the same number of required
parameters.

\item Each lambda-list must have the same number of optional
parameters.  Each method can supply its own default for an optional
parameter.

\item If any lambda-list mentions \cd{\&rest} or \cd{\&key}, each
lambda-list must mention one or both of them.

\item If the generic function lambda-list mentions \cd{\&key}, each
method must accept all of the keyword names mentioned after \cd{\&key},
either by accepting them explicitly, by specifying 
\cd{\&allow-other-keys}, or by specifying \cd{\&rest} but not \cd{\&key}.
Each method can accept additional keyword arguments of its own.  The
checking of the validity of keyword names is done in the generic
function, not in each method. A method is invoked as if the keyword
argument pair whose  keyword is \cd{:allow-other-keys} and whose value
is \cdf{t} were supplied, though no such argument pair will be passed.

\item The use of \cd{\&allow-other-keys} need not be consistent
across lambda-lists.  If \cd{\&allow-other-keys} is mentioned in 
the lambda-list of any applicable method or of the generic function,
any keyword arguments may be mentioned in the call to the
generic function.

\item The use of \cd{\&aux} need not be consistent across methods.
\end{itemize}


If a method-defining form that cannot specify generic function options
creates a generic function, and if the lambda-list for the method
mentions keyword arguments, the lambda-list of the generic function
will mention \cd{\&key} (but no keyword arguments).


\subsubsection{Keyword Arguments in Generic Functions and Methods}

When a generic function or any of its methods mentions \cd{\&key} in
a lambda-list, the specific set of keyword arguments accepted by the
generic function varies according to the applicable methods.  The set
of keyword arguments accepted by the generic function for a particular
call is the union of the keyword arguments accepted by all applicable
methods and the keyword arguments mentioned after \cd{\&key} in the
generic function definition, if any.  A method that has \cd{\&rest}
but not \cd{\&key} does not affect the set of acceptable keyword
arguments.  If the lambda-list of any applicable method or of the
generic function definition contains \cd{\&allow-other-keys}, all
keyword arguments are accepted by the generic function.

The lambda-list congruence rules require that each method
accept all of the keyword arguments mentioned after \cd{\&key} in the
generic function definition, by accepting them explicitly, by
specifying \cd{\&allow-other-keys}, or by specifying \cd{\&rest} but
not \cd{\&key}.  Each method can accept additional keyword arguments
of its own, in addition to the keyword arguments mentioned in the
generic function definition.

\penalty-10000 %required

If a generic function is passed a keyword argument that no applicable
method accepts, an error is signaled.

For example, suppose there are two methods defined for \cdf{width}
as follows:

\begin{lisp}
(defmethod width ((c character-class) \&key font) ...)\\*
\\*
(defmethod width ((p picture-class) \&key pixel-size) ...)
\end{lisp}

\noindent Assume that there are no other methods and no generic
function definition for \cdf{width}. The evaluation of the
following form will signal an error because the keyword argument
\cd{:pixel-size} is not accepted by the applicable method.

\begin{lisp}
(width (make-instance 'character-class :char \#{\Xbackslash}Q) \\*
~~~~~~~:font 'baskerville :pixel-size 10)
\end{lisp}

The evaluation of the following form will signal an error.

\begin{lisp}
(width (make-instance 'picture-class :glyph (glyph \#{\Xbackslash}Q)) \\*
~~~~~~~:font 'baskerville :pixel-size 10)
\end{lisp}

The evaluation of the following form will not signal an error
if the class named \cdf{character-picture-class} is a subclass of
both \cdf{picture-class} and \cdf{character-class}.

\begin{lisp}
(width (make-instance 'character-picture-class :char \#{\Xbackslash}Q) \\*
~~~~~~~:font 'baskerville :pixel-size 10)
\end{lisp}


\subsection{Method Selection and Combination}
\label{Method-Selection-and-Combination-SECTION}

When a generic function is called with particular arguments, it must
determine the code to execute.  This code is called the {\bit effective
method\/} for those arguments.  The effective method is a {\bit
combination\/} of the applicable methods in the generic function.  A
combination of methods is a Lisp expression that contains calls to some or
all of the methods.  If a generic function is
called and no methods apply, the generic function 
\cdf{no-applicable-method} is invoked.

When the effective method has been determined, it is invoked with the same
arguments that were passed to the generic function.  Whatever values it
returns are returned as the values of the generic function.

\subsubsection{Determining the Effective Method}
\label{Determining-the-Effective-Method-SECTION}

The effective method for a set of
arguments is determined by the following three-step procedure:

\begin{enumerate}

\item Select the applicable methods.

\item Sort the applicable methods by precedence order, putting
the most specific method first.

\item Apply method combination to the sorted list of
applicable methods, producing the effective method.

\end{enumerate}

{\bf Selecting the Applicable Methods.}
This step is described in section~\ref{Introduction-to-Methods-SECTION}.


{\bf Sorting the Applicable Methods by Precedence Order.}
To compare the precedence of two methods, their parameter specializers
are examined in order.  The default examination order is from left to
right, but an alternative order may be specified by the 
\cd{:argument-precedence-order} option to \cdf{defgeneric} or to any of
the other forms that specify generic function options.

The corresponding parameter specializers from each method are
compared.  When a pair of parameter specializers are equal, the next
pair are compared for equality.  If all corresponding parameter
specializers are equal, the two methods must have different
qualifiers; in this case, either method can be selected to precede the
other.

If some corresponding parameter specializers are not equal, the first
pair of parameter specializers that are not equal determines the
precedence.  If both parameter specializers are classes, the more
specific of the two methods is the method whose parameter specializer
appears earlier in the class precedence list of the corresponding
argument.  Because of the way in which the set of applicable methods
is chosen, the parameter specializers are guaranteed to be present in
the class precedence list of the class of the argument.

If just one parameter specializer is \cd{(\cdf{eql} {\it
object\/})}, the method with that parameter specializer precedes the
other method.  If both parameter specializers are \cdf{eql}
forms, the
specializers must be the same (otherwise the two methods would
not both have been applicable to this argument).

The resulting list of applicable methods has the most specific
method first and the least specific method last.    

{\bf Applying Method Combination to the Sorted List of Applicable Methods.}
In the simple case---if standard method combination is used and all
applicable methods are primary methods---the effective method is the
most specific method.  That method can call the next most specific
method by using the function \cdf{call-next-method}.  The method that
\cdf{call-next-method} will call is referred to as the {\bit next
method}.  The predicate \cdf{next-method-p} tests whether a next
method exists.  If \cdf{call-next-method} is called and there is no
next most specific method, the generic function \cdf{no-next-method}
is invoked.

In general, the effective method is some combination of the applicable
methods.  It is defined by a Lisp form that contains calls to some or all
of the applicable methods, returns the value or values that will be
returned as the value or values of the generic function, and optionally
makes some of the methods accessible by means of \cdf{call-next-method}.
This Lisp form is the body of the effective method; it is augmented with
an appropriate lambda-list to make it a function.

The role of each method in the effective method is determined by its
method qualifiers and the specificity of the method.  A qualifier
serves to mark a method, and the meaning of a qualifier is
determined by the way that these marks are used by this step
of the procedure.  If an applicable method has an unrecognized
qualifier, this step signals an error and does not include that method
in the effective method.

When standard method combination is used together with qualified methods, 
the effective method is produced as described in
section~\ref{Standard-Method-Combination-SECTION}.

Another type of method combination can be specified by using the 
\cd{:method-\discretionary{}{}{}combination} option of \cdf{defgeneric} or of any of the other
forms that specify generic function options.  In this way this step of
the procedure can be customized.

New types of method combination can be defined by using the 
\cd{define-\discretionary{}{}{}method-\discretionary{}{}{}combination} macro. 


The meta-object level also offers a mechanism for defining new types
of method combination.  The generic function 
\cd{compute-\discretionary{}{}{}effective-\discretionary{}{}{}method} receives as arguments the generic function,
the method combination object, and the sorted list of applicable
methods.  It returns the Lisp form that defines the effective method.
A method for \cd{compute-\discretionary{}{}{}effective-\discretionary{}{}{}method} can be defined directly by
using \cdf{defmethod} or indirectly by using 
\cd{define-\discretionary{}{}{}method-\discretionary{}{}{}combination}.
A {\bit method combination object} is an
object that encapsulates the method combination type and options
specified by the \cd{:method-combination} option to forms that
specify generic function options.



\beforenoterule
\begin{implementation}
In the simplest implementation, the generic function would compute
the effective method each time it was called.  In practice, this will
be too inefficient for some implementations.  Instead, these
implementations might employ a variety of optimizations of the
three-step procedure. Some illustrative examples of such optimizations
are the following:

\begin{itemize}

\item  Use a hash table keyed by the class of the arguments to
store the effective method.

\item  Compile the effective method and save the resulting
compiled function in a table.

\item  Recognize the Lisp form as an instance of a pattern of
control structure and substitute a closure that implements
that structure.

\item  Examine the parameter specializers of all methods for the
generic function and enumerate all possible effective methods.
Combine the effective methods, together with code to select from
among them, into a single function and compile that function.  Call
that function whenever the generic function is called.
\end{itemize}
\end{implementation}
\afternoterule


\subsubsection{Standard Method Combination}
\label{Standard-Method-Combination-SECTION}

Standard method combination is supported by the class 
\cd{standard-\discretionary{}{}{}generic-\discretionary{}{}{}function}.
It is used if no other type of method
combination is specified or if the built-in method combination type
\cdf{standard} is specified. 

{\bit Primary methods\/} define the main action of the effective method,  
while {\bit auxiliary methods\/} modify that action in one of three ways.
A primary method has no method qualifiers.

An auxiliary method is a method whose method qualifier is 
\cd{:before}, \cd{:after}, or \cd{:around}.  Standard method combination
allows no more than one qualifier per method; if a method definition
specifies more than one qualifier per method, an error is signaled.

\begin{itemize}

\item 
A \cd{:before} method has the keyword \cd{:before} as its
only qualifier.  A \cd{:before} method specifies code that is to be
run before any primary method.

\item 
An \cd{:after} method has the keyword \cd{:after} as its only
qualifier.  An \cd{:after} method specifies code that is to be run
after primary methods.  

\item 
An \cd{:around} method has the keyword \cd{:around} as its only
qualifier. An \cd{:around} method specifies code that is to
be run instead of other applicable methods but that is
able to cause some of them to be run.

\end{itemize}
The semantics of standard method combination are as follows:

\begin{itemize}

\item  If there are any \cd{:around} methods, the most specific
\cd{:around} method is called.  It supplies the value or values of the
generic function.

\item  Inside the body of an \cd{:around} method, 
\cdf{call-next-method} can be used to call the next method.  When the next
method returns, the \cd{:around} method can execute more code,
perhaps based on the returned value or values.  The generic function
\cdf{no-next-method} is invoked if \cdf{call-next-method} is used and
there is no applicable method to call.  The function 
\cdf{next-method-p} may be used to determine whether a next method exists.

\item  
If an \cd{:around} method invokes \cdf{call-next-method}, the next
most specific \cd{:around} method is called, if one is applicable.
If there are no \cd{:around} methods or if 
\cdf{call-next-method} is called by the least specific \cd{:around}
method, the other methods are called as follows:

\begin{itemize}
\item  All the \cd{:before} methods are called, in
most-specific-first order.  Their values are ignored.
An error is signaled if \cdf{call-next-method} is used in a
\cd{:before} method.

\item  The most specific primary method is called.  Inside the
body of a primary method, \cdf{call-next-method} may be used to call
the next most specific primary method.  When that method returns, the
previous primary method can execute more code, perhaps based on the
returned value or values.  The generic function \cdf{no-next-method}
is invoked if \cdf{call-next-method} is used and there are no more
applicable primary methods.  The function \cdf{next-method-p} may be
used to determine whether a next method exists.  If 
\cdf{call-next-method} is not used, only the most specific primary method
is called.


\item  All the \cd{:after} methods are called in
most-specific-last order.  Their values are ignored.
An error is signaled if \cdf{call-next-method} is used in an
\cd{:after} method.
\end{itemize}

\item  If no \cd{:around} methods were invoked, the most
specific primary method supplies the value or values returned by the
generic function.  The value or values returned by the invocation of
\cdf{call-next-method} in the least specific \cd{:around} method are
those returned by the most specific primary method.

\end{itemize}

In standard method combination, if there is an applicable method
but no applicable primary method, an error is signaled.

The \cd{:before} methods are run in most-specific-first order and
the \cd{:after} methods are run in least-specific-first order.  The
design rationale for this difference can be illustrated with an
example.  Suppose class ${\it C} \sub 1$ modifies the behavior of its
superclass, ${\it C} \sub 2$, by adding \cd{:before} and \cd{:after}
methods. Whether the behavior of the class ${\it C}\sub 2$ is defined
directly by methods on ${\it C}\sub 2$ or is inherited from its superclasses
does not affect the relative order of invocation of methods on
instances of the class ${\it C}\sub 1$.  Class ${\it C} \sub 1$'s \cd{:before}
method runs before all of class ${\it C} \sub 2$'s methods.  Class ${\it C} \sub
1$'s \cd{:after} method runs after all of class ${\it C} \sub 2$'s methods.

By contrast, all \cd{:around} methods run before any other methods
run.  Thus a less specific \cd{:around} method runs before a more
specific primary method.

If only primary methods are used and if \cdf{call-next-method} is not
used, only the most specific method is invoked; that is, more specific
methods shadow more general ones. 

\subsubsection{Declarative Method Combination}

The macro \cdf{define-method-combination} defines new forms of method
combination.  It provides a mechanism for customizing the production
of the effective method. The default procedure for producing an
effective method is described in
section~\ref{Determining-the-Effective-Method-SECTION}.
There are two forms of 
\cdf{define-method-combination}.  The short form is a simple facility;
the long form is more powerful and more verbose.  The long form
resembles \cdf{defmacro} in that the body is an expression that
computes a Lisp form; it provides mechanisms for implementing
arbitrary control structures within method combination and for
arbitrary processing of method qualifiers.  The syntax and use of both
forms of \cdf{define-method-combination} are explained in
section~\ref{Functions-in-the-Programmer-Interface-SECTION}.


\subsubsection{Built-in Method Combination Types}
\label{Built-in-Method-Combination-Types-SECTION}

The \CLOS\ provides a set of built-in method combination types.  To
specify that a generic function is to use one of these method
combination types, the name of the method combination type is given as
the argument to the \cd{:method-combination} option to 
\cdf{defgeneric} or to the \cd{:method-combination} option to any of the
other forms that specify generic function options.

The names of the built-in  method combination types are
\cd{+}, \cdf{and}, \cdf{append}, \cdf{list}, \cdf{max}, \cdf{min}, 
\cdf{nconc}, \cdf{or}, \cdf{progn}, and \cdf{standard}.

The semantics of the \cdf{standard} built-in method combination type were
described in section~\ref{Standard-Method-Combination-SECTION}.  The other
built-in method combination types are called {\bit simple built-in method
combination types.}

The simple built-in method combination types act as though they were
defined by the short form of \cdf{define-method-combination}.  They
recognize two roles for methods:

\begin{itemize}

\item  An \cd{:around} method has the keyword symbol 
\cd{:around} as its sole qualifier.  The meaning of \cd{:around}
methods is the same as in standard method combination.  Use of the
functions \cdf{call-next-method} and \cdf{next-method-p} is supported
in \cd{:around} methods.

\item  A primary method has the name of the method combination
type as its sole qualifier.  For example, the built-in method
combination type \cdf{and} recognizes methods whose sole qualifier is
\cdf{and}; these are primary methods. Use of the functions 
\cdf{call-next-method} and \cdf{next-method-p} is not supported in primary
methods.

\end{itemize}
The semantics of the simple built-in method combination types are as
follows:

\begin{itemize}
\item 
If there are any \cd{:around} methods, the most specific \cd{:around}
method is called.   It supplies the value or values of the generic function. 

\item  Inside the body of an \cd{:around} method, the function
\cdf{call-next-method} can be used to call the next method.  The
generic function \cdf{no-next-method} is invoked if 
\cdf{call-next-method} is used and there is no applicable method to call.
The function \cdf{next-method-p} may be used to determine whether a
next method exists. When the next method returns, the \cd{:around}
method can execute more code, perhaps based on the returned value or
values.

\item  If an \cd{:around} method invokes 
\cdf{call-next-method}, the next most specific \cd{:around} method is
called, if one is applicable.  If there are no \cd{:around} methods
or if \cdf{call-next-method} is called by the least specific 
\cd{:around} method, a Lisp form derived from the name of the built-in
method combination type and from the list of applicable primary
methods is evaluated to produce the value of the generic function.
Suppose the name of the method combination type is {\it operator\/}
and the call to the generic function is of the form
\begin{lisp}
({\it generic-function\/} ${\it a}\sub 1$ ... ${\it a}\sub {\hbox{\scriptsize\it n}}$)
\end{lisp}
Let ${\it M}\sub 1,\ldots,{\it M}\sub {\hbox{\scriptsize\it k}}$ be the applicable primary methods
in order; then the derived Lisp form is
\begin{lisp}
({\it operator\/} $\langle {\it M}\sub 1\;{\it a}\sub 1\ldots {\it a}\sub {\hbox{\scriptsize\it n}}\rangle$
... $\langle {\it M}\sub k\;{\it a}\sub 1\ldots {\it a}\sub {\hbox{\scriptsize\it n}}\rangle$)
\end{lisp}
If the expression $\langle {\it M}\sub {\hbox{\scriptsize\it i}} \;{\it a}\sub 1\ldots {\it a}\sub
{\hbox{\scriptsize\it n}}\rangle$ is
evaluated, the method ${\it M}\sub i$ will be applied to the arguments
${\it a}\sub 1\ldots {\it a}\sub {\hbox{\scriptsize\it n}}$.  
For example,
if {\it operator\/} is \cdf{or},
the expression $\langle {\it M}\sub{\hbox{\scriptsize\it i}} \ {\it a}\sub 1\ldots {\it a}\sub {\hbox{\scriptsize\it n}}\rangle$ is
evaluated only if $\langle {\it M}\sub {\hbox{\scriptsize\it j}} \ {\it a}\sub 1\ldots {\it a}\sub {\hbox{\scriptsize\it n}}\rangle$,
$1\leq {\it j}<{\it i}$, returned \cdf{nil}.

The default order for the primary methods is 
\cd{:most-specific-first}.  However, the order can be reversed by supplying
\cd{:most-specific-last} as the second argument to the 
\cd{:method-combination} option.

\end{itemize}

The simple built-in method combination types require exactly one qualifier per
method.  An error is signaled if there are applicable methods with no
qualifiers or with qualifiers that are not supported by the method
combination type. An error is signaled if there are applicable \cd{:around}
methods and no applicable primary methods.


\subsection{Meta-objects}

The implementation of the \OS\ manipulates classes, methods, and generic
functions.  The meta-object protocol specifies a set of generic
functions defined by methods on classes; the behavior of those generic
functions defines the behavior of the \OS.  The instances of the classes
on which those methods are defined are called {\bit meta-objects}.  Programming
at the meta-object protocol level involves defining new classes of
meta-objects along with methods specialized on these classes.

\penalty-10000 %required

\subsubsection{Metaclasses}

The {\bit metaclass\/} of an object is the class of its class.  The
metaclass determines the representation of instances of its instances and
the forms of inheritance used by its instances for slot descriptions and
method inheritance.  The metaclass mechanism can be used to provide
particular forms of optimization or to tailor the \CLOS\ for particular
uses.  The protocol for defining metaclasses is discussed in the third part
of the CLOS specification, The \CLOS\ Meta-Object Protocol.
[The third part
has not yet been approved by X3J13 for inclusion in the forthcoming
Common Lisp standard and is not included in this book.---GLS]


\subsubsection{Standard Metaclasses}

The \CLOS\ provides a number of predefined metaclasses.  These include the
classes \cdf{standard-class}, \cdf{built-in-class}, and 
\cdf{structure-class}:

\begin{itemize}

\item 
The class \cdf{standard-class} is the default class of classes defined
by \cdf{defclass}.

\item  The class \cdf{built-in-class} is the class whose
instances are classes that have special implementations with
restricted capabilities.  Any class that corresponds to a standard
Common Lisp type
might be an instance of \cdf{built-in-class}.
The predefined Common Lisp type specifiers that are required to have
corresponding classes are listed in table~\ref{CLOS-PRECEDENCE-TABLE}.
It is implementation-dependent whether each of these classes is implemented as a built-in class.

\item 
All classes defined by means of \cdf{defstruct} are instances of 
\cdf{structure-class}.
\end{itemize}


\subsubsection{Standard Meta-objects}

The \OS\ supplies a standard set of meta-objects, called {\bit standard
meta-objects}. These include the class \cdf{standard-object} and
instances of the classes \cdf{standard-method}, 
\cdf{standard-generic-function}, and \cdf{method-combination}.

\begin{itemize}

\item  
The class \cdf{standard-method} is the default class of
methods that are defined by the forms \cdf{defmethod}, 
\cdf{defgeneric}, \cdf{generic-function}, \cdf{generic-flet}, 
\cdf{generic-labels}, and \cdf{with-added-methods}.

\item 
The class \cdf{standard-generic-function} is the default class of 
generic functions defined by the forms \cdf{defmethod},
\cdf{defgeneric}, \cdf{generic-function}, \cdf{generic-flet},
\cdf{generic-labels}, \cdf{with-added-methods}, and \cdf{defclass}.

\item  The class named \cdf{standard-object} is an instance of
the class \cdf{standard-class} and is a superclass of every class that
is an instance of \cdf{standard-class} except itself.

\item  Every method combination object is an instance of a
subclass of the class \cdf{method-combination}.

\end{itemize}


\subsection{Object Creation and Initialization}
\label{Object-Creation-and-Initialization-SECTION}

The generic function \cdf{make-instance} creates and returns a new
instance of a class.  The first argument is a class or the name of a
class, and the remaining arguments form an {\bit initialization argument\/}
list.  

The initialization of a new instance consists of several distinct
steps, including the following: combining the explicitly supplied
initialization arguments with default values for the unsupplied
initialization arguments, checking the validity of the initialization
arguments, allocating storage for the instance, filling slots with
values, and executing user-supplied methods that perform additional
initialization.  Each step of \cdf{make-instance} is implemented by a
generic function to provide a mechanism for customizing that step.  In
addition, \cdf{make-instance} is itself a generic function and thus
also can be customized.

The \OS\ specifies system-supplied primary methods for each step and
thus specifies a well-defined standard behavior for the entire
initialization process.  The standard behavior provides four simple
mechanisms for controlling initialization:

\begin{itemize}

\item  Declaring a symbol to be an initialization argument for a
slot.  An initialization argument is declared by using the 
\cd{:initarg} slot option to \cdf{defclass}.  This provides a mechanism
for supplying a value for a slot in a call to \cdf{make-instance}.

\item  Supplying a default value form for an initialization
argument.  Default value forms for initialization arguments are
defined by using the \cd{:default-initargs} class option to 
\cdf{defclass}.  If an initialization argument is not explicitly provided
as an argument to \cdf{make-instance}, the default value form is
evaluated in the lexical environment of the \cdf{defclass} form that
defined it, and the resulting value is used as the value of the
initialization argument.

\item  Supplying a default initial value form for a slot.  A
default initial value form for a slot is defined by using the 
\cd{:initform} slot option to \cdf{defclass}.  If no initialization
argument associated with that slot is given as an argument to 
\cdf{make-instance} or is defaulted by \cd{:default-initargs}, this
default initial value form is evaluated in the lexical environment of
the \cdf{defclass} form that defined it, and the resulting value is
stored in the slot.  The \cd{:initform} form for a local slot may be
used when creating an instance, when updating an instance to conform
to a redefined class, or when updating an instance to conform to the
definition of a different class. The \cd{:initform} form for a shared
slot may be used when defining or re-defining the class.

\item  Defining methods for \cdf{initialize-instance} and 
\cdf{shared-initialize}.  The slot-filling behavior described above is
implemented by a system-supplied primary method for 
\cdf{initialize-instance} which invokes \cdf{shared-initialize}. The
generic function \cdf{shared-initialize} implements the parts of
initialization shared by these four situations: when making an
instance, when re-initializing an instance, when updating an instance
to conform to a redefined class, and when updating an instance to
conform to the definition of a different class. The system-supplied
primary method for \cdf{shared-initialize} directly implements the
slot-filling behavior described above, and \cdf{initialize-instance}
simply invokes \cdf{shared-initialize}.

\end{itemize}

\subsubsection{Initialization Arguments}

An initialization argument controls object creation and
initialization.  It is often convenient to use keyword symbols to name
initialization arguments, but the name of an initialization argument
can be any symbol, including \cdf{nil}.  An initialization argument
can be used in two ways: to fill a slot with a value or to provide an
argument for an initialization method.  A single initialization
argument can be used for both purposes.

An {\bit initialization argument list\/} is a list of alternating
initialization argument names and values.  Its structure is identical
to a property list and also to the portion of an argument list
processed for \cd{\&key} parameters.  As in those lists, if an
initialization argument name appears more than once in an
initialization argument list, the leftmost occurrence supplies the
value and the remaining occurrences are ignored.  The arguments to
\cdf{make-instance} (after the first argument) form an initialization
argument list.  Error checking of initialization argument names is
disabled if the keyword argument pair whose keyword is 
\cd{:allow-other-keys} and whose value is non-\cdf{nil} appears in the
initialization argument list.

An initialization argument can be associated with a slot.  If the
initialization argument has a value in the initialization argument
list, the value is stored into the slot of the newly created object,
overriding any \cd{:initform} form associated with the slot.  A
single initialization argument can initialize more than one slot.  An
initialization argument that initializes a shared slot stores its
value into the shared slot, replacing any previous value.

An initialization argument can be associated with a method.  When an
object is created and a particular initialization argument is
supplied, the generic functions \cdf{initialize-instance}, 
\cdf{shared-initialize}, and \cdf{allocate-instance} are called with that
initialization argument's name and value as a keyword argument pair.
If a value for the initialization argument is not supplied in the
initialization argument list, the method's lambda-list supplies a
default value.

Initialization arguments are used in four situations: when making an
instance, when re-initializing an instance, when updating an instance to
conform to a redefined class, and when updating an instance to conform
to the definition of a different class.

Because initialization arguments are used to control the creation and
initialization of an instance of some particular class, we say that an
initialization argument is ``an initialization argument for'' that
class.


\subsubsection{Declaring the Validity of Initialization Arguments}
\label{Declaring-the-Validity-of-Initialization-Arguments-SECTION}

Initialization arguments are checked for validity in each of the four
situations that use them.  An initialization argument may be valid in
one situation and not another. For example, the system-supplied
primary method for \cdf{make-instance} defined for the class 
\cdf{standard-class} checks the validity of its initialization arguments
and signals an error if an initialization argument is supplied that is
not declared valid in that situation.


There are two means of declaring initialization arguments valid.

\begin{itemize}

\item  Initialization arguments that fill slots are declared
valid by the \cd{:initarg} slot option to \cdf{defclass}.  The 
\cd{:initarg} slot option is inherited from superclasses.  Thus the set of
valid initialization arguments that fill slots for a class is the
union of the initialization arguments that fill slots declared
valid by that class and its superclasses. Initialization arguments
that fill slots are valid in all four contexts.

\item  Initialization arguments that supply arguments to methods
are declared valid by defining those methods.  The keyword name of
each keyword parameter specified in the method's lambda-list becomes
an initialization argument for all classes for which the method is
applicable.  Thus method inheritance controls the set of valid
initialization arguments that supply arguments to methods.  The
generic functions for which method definitions serve to declare
initialization arguments valid are as follows:

\begin{itemize}
\item Making an instance of a class: \cdf{allocate-instance},
\cdf{initialize-instance}, and \cdf{shared-initialize}.
Initialization arguments declared valid by these methods are
valid when making an instance of a class.

\item  Re-initializing an instance: the functions \cdf{reinitialize-instance}
and \cd{shared-\discretionary{}{}{}initialize}.
Initialization arguments declared valid by these methods are
valid when re-initializing an instance.

\item   Updating an instance to conform to a redefined class:
\cd{update-\discretionary{}{}{}instance-\discretionary{}{}{}for-\discretionary{}{}{}redefined-\discretionary{}{}{}class}
and \cdf{shared-initialize}.
Initialization arguments declared valid by these methods are
valid when updating an instance to conform to a redefined class.

\item  Updating an instance to conform to the definition of a
different class: \cd{update-\discretionary{}{}{}instance-\discretionary{}{}{}for-\discretionary{}{}{}different-\discretionary{}{}{}class} and 
\cdf{shared-initialize}.
Initialization arguments declared valid by these methods are
valid when updating an instance to conform to the definition
of a different class.
\end{itemize}
\end{itemize}

The set of valid initialization arguments for a class is the set of
valid initialization arguments that either fill slots or supply
arguments to methods, along with the predefined initialization
argument \cd{:allow-other-keys}.  The default value for 
\cd{:allow-other-keys} is \cdf{nil}.  The meaning of 
\cd{:allow-other-keys} is the same here as when it is passed to an ordinary
function.


\subsubsection{Defaulting of Initialization Arguments}

A {\bit default value form\/} can be supplied for an initialization
argument by using the \cd{:default-initargs} class option.  If an
initialization argument is declared valid by some particular class,
its default  value form might be specified by a different class. 
In this case \cd{:default-initargs} is used to supply a default value
for an inherited initialization argument.

The \cd{:default-initargs} option is used only to provide default
values for initialization arguments; it does not declare a symbol as a
valid initialization argument name.  Furthermore, the 
\cd{:default-initargs} option is used only to provide default values for
initialization arguments when making an instance.

The argument to the \cd{:default-initargs} class option is a list of
alternating initialization argument names and forms.  Each form is the
default  value form for the corresponding initialization
argument.  The default  value form of an initialization
argument is used and evaluated only if that initialization argument
does not appear in the arguments to \cdf{make-instance} and is not
defaulted by a more specific class.  The default  value form is
evaluated in the lexical environment of the \cdf{defclass} form that
supplied it; the result is used as the initialization
argument's value.

The initialization arguments supplied to \cdf{make-instance} are combined
with defaulted initialization arguments to produce a {\bit
defaulted initialization argument list}. A defaulted initialization
argument list is a list of alternating initialization argument names and
values in which unsupplied initialization arguments are defaulted and in
which the explicitly supplied initialization arguments appear earlier in
the list than the defaulted initialization arguments.  Defaulted
initialization arguments are ordered according to the order in the class
precedence list of the classes that supplied the default values.

There is a distinction between the purposes of the 
\cd{:default-initargs} and the \cd{:initform} options with respect to the
initialization of slots.  The \cd{:default-initargs} class option
provides a mechanism for the user to give a default  value form
for an initialization argument without knowing whether the
initialization argument initializes a slot or is passed to a method.
If that initialization argument is not explicitly supplied in a call
to \cdf{make-instance}, the default  value form is used, just
as if it had been supplied in the call.  In contrast, the 
\cd{:initform} slot option provides a mechanism for the user to give a
default initial value form for a slot.  An \cd{:initform} form is
used to initialize a slot only if no initialization argument
associated with that slot is given as an argument to 
\cdf{make-instance} or is defaulted by \cd{:default-initargs}.

The order of evaluation of default value forms for initialization
arguments and the order of evaluation of \cd{:initform} forms are
undefined.  If the order of evaluation matters, use
\cdf{initialize-instance} or \cdf{shared-initialize} methods.


\subsubsection{Rules for Initialization Arguments}
\label{Rules-for-Initialization-Arguments-SECTION}

The \cd{:initarg} slot option may be specified more than
once for a given slot.
The following rules specify when initialization arguments may be
multiply defined:

\begin{itemize}

\item  A given initialization argument can be used to
initialize more than one slot if the same initialization argument name
appears in more than one \cd{:initarg} slot option.

\item  A given initialization argument name can appear 
in the lambda-list of more than one initialization method.

\item  A given initialization argument name can
appear both in an \cd{:initarg} slot option and in the lambda-list
of an initialization method.

\end{itemize}

If two or more initialization arguments that initialize
the same slot are given in the arguments to \cdf{make-instance}, the
leftmost of these initialization arguments in the initialization
argument list supplies the value, even if the initialization arguments
have different names.

If two or more different initialization arguments that
initialize the same slot have default values and none is given
explicitly in the arguments to \cdf{make-instance}, the initialization
argument that appears in a \cd{:default-initargs} class option in the
most specific of the classes supplies the value. If a single 
\cd{:default-initargs} class option specifies two or more initialization
arguments that initialize the same slot and none is given explicitly
in the arguments to \cdf{make-instance}, the leftmost argument in the 
\cd{:default-initargs} class option supplies the value, and the values of
the remaining default value forms are ignored.

Initialization arguments given explicitly in the
arguments to \cdf{make-instance} appear to the left of defaulted
initialization arguments. Suppose that the classes ${\it C}\sub 1$ and
${\it C}\sub 2$ supply the values of defaulted initialization arguments for
different slots, and suppose that ${\it C}\sub 1$ is more specific than
${\it C}\sub 2$; then the defaulted initialization argument whose value is
supplied by ${\it C}\sub 1$ is to the left of the defaulted initialization
argument whose value is supplied by ${\it C}\sub 2$ in the defaulted
initialization argument list.  If a single \cd{:default-initargs}
class option supplies the values of initialization arguments for two
different slots, the initialization argument whose value is specified
farther to the left in the \cdf{default-initargs} class option appears
farther to the left in the defaulted initialization argument list.

If a slot has both an \cd{:initform} form and an 
\cd{:initarg} slot option, and the initialization argument is defaulted
using \cd{:default-initargs} or is supplied to \cdf{make-instance},
the captured \cd{:initform} form is neither used nor evaluated.

The following is an example of the preceding rules:

\begin{lisp}
(defclass q () ((x :initarg a))) \\*
\\*
(defclass r (q) ((x :initarg b)) \\*
~~(:default-initargs a 1 b 2))
\end{lisp}

\begin{flushleft}
\begin{tabular*}{\textwidth}{@{}l@{\extracolsep{\fill}}ll@{}}
&{\rm Defaulted Initialization}&{\rm Contents} \\
{\rm Form}&{\rm Argument List}&{\rm of Slot} \\
\hlinesp
\cd{(make-instance 'r)}&\cd{(a 1 b 2)}&\cd{1}\\
\cd{(make-instance 'r 'a 3)}&\cd{(a 3 b 2)}&\cd{3}\\
\cd{(make-instance 'r 'b 4)}&\cd{(b 4 a 1)}&\cd{4}\\
\cd{(make-instance 'r 'a 1 'a 2)}&\cd{(a 1 a 2 b 2)}&\cd{1} \\
\hline
\end{tabular*}
\end{flushleft}

\subsubsection{Shared-Initialize}
\label{Shared-Initialize-SECTION}

The generic function \cdf{shared-initialize} is used to fill the slots
of an instance using initialization arguments and \cd{:initform}
forms when an instance is created, when an instance is re-initialized,
when an instance is updated to conform to a redefined class, and when
an instance is updated to conform to a different class. It uses
standard method combination. It takes the following arguments: the
instance to be initialized, a specification of a set of names of slots
accessible in that instance, and any number of initialization
arguments.  The arguments after the first two must form an initialization
argument list.

The second argument to \cdf{shared-initialize} may be one of the following:

\begin{itemize}

\item  It can be a list of slot names, which specifies
the set of those slot names. 

\item  It can be \cdf{nil}, which specifies the empty set of
slot names.

\item  It can be the symbol \cdf{t}, which specifies the set of
all of the slots.

\end{itemize}

There is a system-supplied primary method for 
\cdf{shared-initialize} whose first parameter specializer is the class 
\cdf{standard-object}.  This method behaves as follows on each slot,
whether shared or local:

\begin{itemize}

\item  If an initialization argument in the initialization
argument list specifies a value for that slot, that value is stored
into the slot, even if a value has already been stored in the slot
before the method is run.  The affected slots are independent of which
slots are indicated by the second argument to \cdf{shared-initialize}.

\item  Any slots indicated by the second argument that are still
unbound at this point are initialized according to their 
\cd{:initform} forms.  For any such slot that has an \cd{:initform} form,
that form is evaluated in the lexical environment of its defining 
\cdf{defclass} form and the result is stored into the slot.  For example,
if a \cd{:before} method stores a value in the slot, the 
\cd{:initform} form will not be used to supply a value for the slot.  If
the second argument specifies a name that does not correspond to any
slots accessible in the instance, the results are unspecified.

\item  The rules mentioned in section~\ref{Rules-for-Initialization-Arguments-SECTION} are obeyed.

\end{itemize}

The generic function \cdf{shared-initialize} is called by the
system-supplied primary methods for the generic functions
\cd{initialize-\discretionary{}{}{}instance},
\cd{reinitialize-\discretionary{}{}{}instance}, 
\cd{update-\discretionary{}{}{}instance-\discretionary{}{}{}for-\discretionary{}{}{}different-\discretionary{}{}{}class}, and
\cd{update-\discretionary{}{}{}instance-\discretionary{}{}{}for-\discretionary{}{}{}redefined-\discretionary{}{}{}class}.
Thus methods can be written for 
\cd{shared-\discretionary{}{}{}initialize} to specify actions that should be taken in all of
these contexts.


\subsubsection{Initialize-Instance}

The generic function \cdf{initialize-instance} is called by 
\cdf{make-instance} to initialize a newly created instance.  It uses
standard method combination.  Methods for 
\cdf{initialize-instance} can be defined in order to perform any
initialization that cannot be achieved with the simple slot-filling
mechanisms.

\penalty-10000 %required

During initialization, \cdf{initialize-instance} is invoked
after the following actions have been taken:

\begin{itemize} 

\item  The defaulted initialization argument list has been computed by
combining the supplied initialization argument list with any default
initialization arguments for the class.

\item  The validity of the defaulted initialization argument
list has been checked.  If any of the initialization arguments has not
been declared valid, an error is signaled.

\item  A new instance whose slots are unbound has been created.

\end{itemize}

The generic function \cdf{initialize-instance} is called with the
new instance and the defaulted initialization arguments.  There is
a system-supplied primary method for \cdf{initialize-instance}
whose parameter specializer is the class \cdf{standard-object}.  This
method calls the generic function \cdf{shared-initialize} to fill in
the slots according to the initialization arguments and the 
\cd{:initform} forms for the slots; the generic function 
\cdf{shared-initialize} is called with the following arguments: the instance,
\cdf{t}, and the defaulted initialization arguments.

Note that \cdf{initialize-instance} provides the defaulted
initialization argument list in its call to \cdf{shared-initialize},
so the first step performed by the system-supplied primary method for
\cdf{shared-initialize} takes into account both the initialization
arguments provided in the call to \cdf{make-instance} and the
defaulted initialization argument list.

Methods for \cdf{initialize-instance} can be defined to specify
actions to be taken when an instance is initialized.  If only \cd{:after}
methods for \cdf{initialize-instance} are defined, they will be
run after the system-supplied primary method for initialization and
therefore they will not interfere with the default behavior of 
\cdf{initialize-instance}.

The \OS\ provides two functions that are useful in the bodies of 
\cdf{initialize-instance} methods.  The function \cdf{slot-boundp}
returns a boolean value that indicates whether a specified slot has a
value; this provides a mechanism for writing \cd{:after} methods for
\cdf{initialize-instance} that initialize slots only if they have
not already been initialized.  The function \cdf{slot-makunbound}
causes the slot to have no value.


\subsubsection{Definitions of Make-Instance and Initialize-Instance}

The generic function \cdf{make-instance} behaves as if it were defined as
follows, except that certain optimizations are permitted:

\penalty-10000 %required

\begin{lisp}
(defmethod make-instance ((class standard-class) \&rest initargs) \\*
~~(setq initargs (default-initargs class initargs)) \\*
~~... \\*
~~(let ((instance (apply \#'allocate-instance class initargs))) \\*
~~~~(apply \#'initialize-instance instance initargs) \\*
~~~~instance)) \\
\\
(defmethod make-instance ((class-name symbol) \&rest initargs) \\*
~~(apply \#'make-instance (find-class class-name) initargs))
\end{lisp}

%This is the code:
%(defmethod make-instance ((class standard-class) &rest initargs)
%  (setq initargs (default-initargs class initargs))
%  (let* ((proto (class-prototype class))
%         (methods 
%           (append
%	      (compute-applicable-methods #'allocate-instance `(,class))
%	      (compute-applicable-methods #'initialize-instance `(,proto))
%	      (compute-applicable-methods #'shared-initialize `(,proto nil)))))
%	 (unless
%	   (subsetp
%	     (let ((keys '()))
%	       (do ((plist initargs (cddr plist)))
%		   ((null plist) keys)
%	 	 (push (car plist) keys)))
%	     (union 
%	       (class-slot-initargs class)
%	       (reduce #'union (mapcar #'function-keywords methods))))
%	   (error ...)))
%  (let ((instance (apply #'allocate-instance class initargs)))
%    (apply #'initialize-instance instance initargs)
%    instance))

The elided code in the definition of \cdf{make-instance} checks the
supplied initialization arguments to determine whether an initialization
argument was supplied that neither filled a slot nor supplied an argument
to an applicable method. This check could be implemented using the generic
functions \cdf{class-prototype}, \cdf{compute-applicable-methods}, 
\cdf{function-keywords}, and \cdf{class-slot-initargs}. See the third
part of the \CLOS\ specification for a
description of this initialization argument check.
[The third part
has not yet been approved by X3J13 for inclusion in the forthcoming
Common Lisp standard and is not included in this book.---GLS]

The generic function \cdf{initialize-instance} behaves as if it were
defined as follows, except that certain optimizations are permitted:

\begin{lisp}
(defmethod initialize-instance \\*
~~~~~~~~~~~((instance standard-object) \&rest initargs) \\*
~~(apply \#'shared-initialize instance t initargs)))
\end{lisp}

These procedures can be customized at either the Programmer Interface level,
the meta-object level, or both.  

Customizing at the Programmer Interface level includes using the 
\cd{:initform}, \cd{:initarg}, and \cd{:default-initargs} options to
\cdf{defclass}, as well as defining methods for \cdf{make-instance}
and \cdf{initialize-instance}.  It is also possible to define
methods for \cdf{shared-initialize}, which would be invoked by the
generic functions \cdf{reinitialize-instance}, 
\cd{update-\discretionary{}{}{}instance-\discretionary{}{}{}for-\discretionary{}{}{}redefined-\discretionary{}{}{}class}, 
\cd{update-\discretionary{}{}{}instance-\discretionary{}{}{}for-\discretionary{}{}{}different-\discretionary{}{}{}class}, and 
\cdf{initialize-instance}.  The meta-object level supports additional
customization by allowing methods to be defined on 
\cdf{make-instance}, \cdf{default-initargs}, and 
\cdf{allocate-instance}.  Parts~2 and~3 of the \CLOS\ specification document each of these generic
functions and the system-supplied primary methods.
[The third part
has not yet been approved by X3J13 for inclusion in the forthcoming
Common Lisp standard and is not included in this book.---GLS]

Implementations are permitted to make certain optimizations to 
\cd{initialize-\discretionary{}{}{}instance} and \cdf{shared-initialize}.  The
description of \cd{shared-\discretionary{}{}{}initialize} in
section~\ref{Functions-in-the-Programmer-Interface-SECTION}
mentions the
possible optimizations.

Because of optimization, the check for valid initialization arguments
might not be implemented using the generic functions 
\cdf{class-prototype}, \cd{compute-\discretionary{}{}{}applicable-\discretionary{}{}{}methods}, 
\cdf{function-keywords}, and \cdf{class-slot-initargs}. In addition,
methods for the generic function \cd{default-\discretionary{}{}{}initargs} and the
system-supplied primary methods for \cd{allocate-\discretionary{}{}{}instance}, 
\cd{initialize-\discretionary{}{}{}instance},
and \cd{shared-\discretionary{}{}{}initialize} might not be called on
every call to \cd{make-\discretionary{}{}{}instance} or might not receive exactly the
arguments that would be expected.


\subsection{Redefining Classes}  
\label{Redefining-Classes-SECTION}  

A class that is an instance of \cdf{standard-class} can be redefined
if the new class will also be an instance of \cdf{standard-class}.
Redefining a class modifies the existing class object to reflect the
new class definition; it does not create a new class object for the
class.  Any method object created by a \cd{:reader}, \cd{:writer}, or
\cd{:accessor} option specified by the old \cdf{defclass} form is
removed from the corresponding generic function.
Methods specified by the new \cdf{defclass} form are added.

% any function specified by the \cd{:constructor}
% option of the old \cdf{defclass} form is removed from the
% corresponding symbol function cell.

When the class {\it C} is redefined, changes are propagated to its instances
and to instances of any of its subclasses.  Updating such an
instance occurs at an implementation-dependent time, but no later than
the next time a slot of that instance is read or written.  Updating an
instance does not change its identity as defined by the \cdf{eq}
function.  The updating process may change the slots of that
particular instance, but it does not create a new instance.  Whether
updating an instance consumes storage is implementation-dependent.

Note that redefining a class may cause slots to be added or deleted.
If a class is redefined in a way that changes the set of local slots
accessible in instances, the instances will be updated.  It is
implementation-dependent whether instances are updated if a class is
redefined in a way that does not change the set of local slots
accessible in instances.

The value of a slot that is specified as shared both in the old class
and in the new class is retained.  If such a shared slot was unbound
in the old class, it will be unbound in the new class.  Slots that
were local in the old class and that are shared in the new class are
initialized.  Newly added shared slots are initialized.

Each newly added shared slot is set to the result of evaluating the
captured \cd{:initform} form for the slot that was specified in the
\cdf{defclass} form for the new class. If there is no \cd{:initform}
form, the slot is unbound.

If a class is redefined in such a way that the set of local slots
accessible in an instance of the class is changed, a two-step process
of updating the instances of the class takes place.  The process may
be explicitly started by invoking the generic function 
\cdf{make-instances-obsolete}.  This two-step process can happen in other
circumstances in some implementations.  For example, in some
implementations this two-step process will be triggered if the order
of slots in storage is changed.

The first step modifies the structure of the instance by adding new
local slots and discarding local slots that are not defined in the new
version of the class.
The second step initializes the newly added local slots and performs
any other user-defined actions. These steps are further specified
in the next two sections.

\subsubsection{Modifying the Structure of Instances}

The first step modifies the structure of instances of the redefined
class to conform to its new class definition.  Local slots specified
by the new class definition that are not specified as either local or
shared by the old class are added, and slots not specified as either
local or shared by the new class definition that are specified as
local by the old class are discarded. The names of these added and discarded
slots are passed as arguments to \cd{update-\discretionary{}{}{}instance-\discretionary{}{}{}for-\discretionary{}{}{}redefined-\discretionary{}{}{}class}
as described in the next section.

The values of local slots specified by both the new and old classes
are retained. If such a local slot was unbound, it remains unbound.

The value of a slot that is specified as shared in the old class and
as local in the new class is retained.  If such a shared slot was
unbound, the local slot will be unbound.


\subsubsection{Initializing Newly Added Local Slots}

The second step initializes the newly added local slots and performs
any other user-defined actions.  This step is implemented by the generic
function \cd{update-\discretionary{}{}{}instance-\discretionary{}{}{}for-\discretionary{}{}{}redefined-\discretionary{}{}{}class}, which is called after
completion of the first step of modifying the structure of the
instance.

The generic function \cd{update-\discretionary{}{}{}instance-\discretionary{}{}{}for-\discretionary{}{}{}redefined-\discretionary{}{}{}class} takes
four required arguments: the instance being updated after it has
undergone the first step, a list of the names of local slots that were
added, a list of the names of local slots that were discarded, and a
property list containing the slot names and values of slots that were
discarded and had values.  Included among the discarded slots are
slots that were local in the old class and that are shared in the new
class.

The generic function \cd{update-\discretionary{}{}{}instance-\discretionary{}{}{}for-\discretionary{}{}{}redefined-\discretionary{}{}{}class} also
takes any number of initialization arguments.  When it is called by
the system to update an instance whose class has been redefined, no
initialization arguments are provided.

There is a system-supplied primary method for the generic function
\cd{update-\discretionary{}{}{}instance-\discretionary{}{}{}for-\discretionary{}{}{}redefined-\discretionary{}{}{}class} whose parameter specializer for
its instance argument is the class \cdf{standard-object}.  First this
method checks the validity of initialization arguments and signals an
error if an initialization argument is supplied that is not declared
valid (see
section~\ref{Declaring-the-Validity-of-Initialization-Arguments-SECTION}.)
Then it calls the generic function
\cdf{shared-initialize} with the following arguments: the instance,
the list of names of the newly added slots, and the initialization
arguments it received.


\subsubsection{Customizing Class Redefinition}

Methods for \cd{update-\discretionary{}{}{}instance-\discretionary{}{}{}for-\discretionary{}{}{}redefined-\discretionary{}{}{}class} may be defined
to specify actions to be taken when an instance is updated.  If only
\cd{:after} methods for \cd{update-\discretionary{}{}{}instance-\discretionary{}{}{}for-\discretionary{}{}{}redefined-\discretionary{}{}{}class} are
defined, they will be run after the system-supplied primary method for
initialization and therefore will not interfere with the default
behavior of \cd{update-\discretionary{}{}{}instance-\discretionary{}{}{}for-\discretionary{}{}{}redefined-\discretionary{}{}{}class}.  Because no
initialization arguments are passed to 
\cd{update-\discretionary{}{}{}instance-\discretionary{}{}{}for-\discretionary{}{}{}redefined-\discretionary{}{}{}class} when it is called by the system,
the \cd{:initform} forms for slots that are filled by \cd{:before}
methods for \cd{update-\discretionary{}{}{}instance-\discretionary{}{}{}for-\discretionary{}{}{}redefined-\discretionary{}{}{}class} will not be
evaluated by \cdf{shared-initialize}.

Methods for \cdf{shared-initialize} may be defined to customize class
redefinition (see section~\ref{Shared-Initialize-SECTION}).


\subsubsection{Extensions}

There are two allowed extensions to class redefinition: 

\begin{itemize}

\item  The \OS\ may be extended to permit the new class
to be an instance of a metaclass other than the metaclass of the
old class.

\item  The \OS\ may be extended to support an updating process
when either the old or the new class is an instance of a
class other than \cdf{standard-class} that is not a built-in class.

\end{itemize}


\subsection{Changing the Class of an Instance}
\label{Changing-the-Class-of-an-Instance-SECTION}

The function \cdf{change-class} can be used to change the class of an
instance from its current class, ${\it C}\sub {\hbox{{\footnotesize\rm from}}}$, to a
different class, ${\it C}\sub {\hbox{{\footnotesize\rm to}}}$; it changes the
structure of the instance to conform to the definition of the class
${\it C}\sub {\hbox{{\footnotesize\rm to}}}$.

Note that changing the class of an instance may cause slots to be
added or deleted. 

When \cdf{change-class} is invoked on an instance, a two-step updating
process takes place.  The first step modifies the structure of
the instance by adding new local slots and discarding local slots that
are not specified in the new version of the instance.  The second step
initializes the newly added local slots and performs any other
user-defined actions. These steps are further described in the
following two sections.

\subsubsection{Modifying the Structure of an Instance}

In order to make an instance conform to the class ${\it C}\sub
{\hbox{{\footnotesize\it to}}}$, local slots specified by the class ${\it C}\sub
{\hbox{{\footnotesize\it to}}}$ that are not specified by the class ${\it C}\sub
{\hbox{{\footnotesize\it from}}}$ are added, and local slots not specified by
the class ${\it C}\sub {\hbox{{\footnotesize\it to}}}$ that are specified by the
class ${\it C}\sub {\hbox{{\footnotesize\it from}}}$ are discarded.

The values of local slots specified by both the class ${\it C}\sub
{\hbox{{\footnotesize\it to}}}$ and the class ${\it C}\sub {\hbox{{\footnotesize\it
from}}}$ are retained. If such a local slot was unbound, it remains
unbound.

The values of slots specified as shared in the class ${\it C}\sub
{\hbox{{\footnotesize\it from}}}$ and as local in the class ${\it C}\sub
{\hbox{{\footnotesize\it to}}}$ are retained.

This first step of the update does not affect the values of any shared
slots.


\subsubsection{Initializing Newly Added Local Slots}

The second step of the update initializes the newly added slots and
performs any other user-defined actions.  This step is implemented by
the generic function \cd{update-\discretionary{}{}{}instance-\discretionary{}{}{}for-\discretionary{}{}{}different-\discretionary{}{}{}class}.  The
generic function \cd{update-\discretionary{}{}{}instance-\discretionary{}{}{}for-\discretionary{}{}{}different-\discretionary{}{}{}class} is invoked
by \cdf{change-class} after the first step of the update has been
completed.

The generic function \cd{update-\discretionary{}{}{}instance-\discretionary{}{}{}for-\discretionary{}{}{}different-\discretionary{}{}{}class} is
invoked on two arguments computed by \cdf{change-class}.  The first
argument passed is a copy of the instance being updated and is an
instance of the class ${\it C}\sub {\hbox{{\footnotesize\it from}}}$; this copy has
dynamic extent within the generic function \cdf{change-class}.  The
second argument is the instance as updated so far by \cdf{change-class}
and is an instance of the class ${\it C}\sub {\hbox{{\footnotesize\it to}}}$.

The generic function \cd{update-\discretionary{}{}{}instance-\discretionary{}{}{}for-\discretionary{}{}{}different-\discretionary{}{}{}class} also
takes any number of initialization arguments.  When it is called by
\cdf{change-class}, no initialization arguments are provided.

There is a system-supplied primary method for the generic function
\cd{update-\discretionary{}{}{}instance-\discretionary{}{}{}for-\discretionary{}{}{}different-\discretionary{}{}{}class} that has two parameter
specializers, each of which is the class \cdf{standard-object}.  First
this method checks the validity of initialization arguments and
signals an error if an initialization argument is supplied that is not
declared valid (see section~\ref{Declaring-the-Validity-of-Initialization-Arguments-SECTION}).
Then it calls the
generic function \cdf{shared-initialize} with the following arguments:
the instance, a list of names of the newly added slots, and the
initialization arguments it received.


\subsubsection{Customizing the Change of Class of an Instance}

Methods for \cd{update-\discretionary{}{}{}instance-\discretionary{}{}{}for-\discretionary{}{}{}different-\discretionary{}{}{}class} may be defined
to specify actions to be taken when an instance is updated.  If only
\cd{:after} methods for \cd{update-\discretionary{}{}{}instance-\discretionary{}{}{}for-\discretionary{}{}{}different-\discretionary{}{}{}class} are
defined, they will be run after the system-supplied primary method for
initialization and will not interfere with the default behavior of
\cd{update-\discretionary{}{}{}instance-\discretionary{}{}{}for-\discretionary{}{}{}different-\discretionary{}{}{}class}.  Because no initialization
arguments are passed to \cd{update-\discretionary{}{}{}instance-\discretionary{}{}{}for-\discretionary{}{}{}different-\discretionary{}{}{}class} when
it is called by \cdf{change-class}, the \cd{:initform} forms for slots
that are filled by \cd{:before} methods for 
\cd{update-\discretionary{}{}{}instance-\discretionary{}{}{}for-\discretionary{}{}{}different-\discretionary{}{}{}class} will not be evaluated by 
\cdf{shared-initialize}.

Methods for \cdf{shared-initialize} may be defined to customize class
redefinition (see section~\ref{Shared-Initialize-SECTION}).

\subsection{Reinitializing an Instance}
\label{Reinitializing-an-Instance-SECTION}

The generic function \cdf{reinitialize-instance} may be used to change
the values of slots according to initialization arguments.

The process of reinitialization changes the values of some slots and
performs any user-defined actions.

Reinitialization does not modify the structure
of an instance to add or delete slots, and it does not use any 
\cd{:initform} forms to initialize slots.

The generic function \cdf{reinitialize-instance} may be called
directly.  It takes one required argument, the instance.  It also
takes any number of initialization arguments to be used by methods for
\cdf{reinitialize-instance} or for \cdf{shared-initialize}. The
arguments after the required instance must form an initialization
argument list.

There is a system-supplied primary method for 
\cdf{reinitialize-instance} whose parameter specializer is the class 
\cdf{standard-object}.  First this method checks the validity of
initialization arguments and signals an error if an initialization
argument is supplied that is not declared valid (see
section~\ref{Declaring-the-Validity-of-Initialization-Arguments-SECTION}).
Then it calls the generic function 
\cdf{shared-initialize} with the following arguments: the instance, 
\cdf{nil}, and the initialization arguments it received.

\penalty-10000 %required

\subsubsection{Customizing Reinitialization}

Methods for the generic function \cdf{reinitialize-instance} may be defined to specify
actions to be taken when an instance is updated.  If only \cd{:after}
methods for \cdf{reinitialize-instance} are defined, they will be run
after the system-supplied primary method for initialization and
therefore will not interfere with the default behavior of 
\cdf{reinitialize-instance}.

Methods for \cdf{shared-initialize} may be defined to customize class
redefinition (see section~\ref{Shared-Initialize-SECTION}).


\section{Functions in the Programmer Interface}
\label{Functions-in-the-Programmer-Interface-SECTION}

This section describes the functions, macros, special forms, and
generic functions provided by the \CLOS\ Programmer Interface.  The
Programmer Interface comprises the functions and macros that are
sufficient for writing most object-oriented programs.

This section is reference material that requires an understanding of
the basic concepts of the Common Lisp Object System.  The functions
are arranged in alphabetical order for convenient reference.

The description of each function, macro, special form,
and generic function includes its purpose, its syntax, the
semantics of its arguments and returned values, and often an example
and cross-references to related functions.

The syntax description for a function, macro, or special form
describes its parameters.
The description of a generic function includes descriptions of the
methods that are defined on that generic function by the \CLOS.  A
{\bit method signature\/} is used to describe the parameters and
parameter specializers for each method.

The following is an example of the format for
the syntax description of a generic function with the method
signature for one primary method:

\begin{defun}[Generic function][Primary method]
f x y &optional z &key :k \\
f (x class) (y t) &optional z &key :k

This description indicates that the generic function \cdf{f} 
has two required parameters, {\it x\/} and {\it y}.  In addition,
there is an optional parameter {\it z\/} and a keyword parameter \cd{:k}.

The method signature indicates that this method on the generic function
\cdf{f} has two required parameters, {\it x}, which must be an
instance of the class \cdf{class}, and {\it y}, which can be any
object. In addition, there is an optional parameter {\it z\/} and a
keyword parameter \cd{:k}.  The signature also indicates that this
method on \cdf{f} is a primary method and has no qualifiers.

The syntax description for a generic function describes the
lambda-list of the generic function itself, while the method
signatures describe the lambda-lists of the defined methods.

The generic functions described in this book are all standard
generic functions.  They all use standard method combination.

Any implementation of the \CLOS\ is allowed to provide additional methods
on the generic functions described here.


It is useful to categorize the functions and macros according to their
role in this standard:
\end{defun}   %misplaced to make better break

\begin{itemize}
\item 
{\it Tools used for simple object-oriented programming}

These tools allow for defining new classes, methods, and generic 
functions and for making instances.   Some tools used within
method bodies are also listed here.   Some of the macros listed here have 
a corresponding function that performs the same task at a lower level of
abstraction. 

\begin{flushleft}
\cf
\begin{tabular}{@{}ll@{}}
\cd{call-next-method~~~~~~~~~~~}&\cdf{initialize-instance}\\
\cdf{change-class}&\cdf{make-instance}\\
\cdf{defclass}&\cdf{next-method-p}\\
\cdf{defgeneric}&\cdf{slot-boundp}\\
\cdf{defmethod}&\cdf{slot-value}\\
\cdf{generic-flet}&\cdf{with-accessors}\\
\cdf{generic-function}&\cdf{with-added-methods}\\
\cdf{generic-labels}&\cdf{with-slots}
\end{tabular}
\end{flushleft}

\item 
{\it Functions underlying the commonly used macros}

\begin{flushleft}
\cf
\begin{tabular}{@{}ll@{}}
\cdf{add-method}&\cdf{reinitialize-instance}\\
\cdf{class-name}&\cdf{remove-method}\\
\cdf{compute-applicable-methods}&\cdf{shared-initialize}\\
\cdf{ensure-generic-function}&\cdf{slot-exists-p}\\
\cdf{find-class}&\cdf{slot-makunbound}\\
\cdf{find-method}&\cdf{slot-missing}\\
\cdf{function-keywords}&\cdf{slot-unbound}\\
\cdf{make-instances-obsolete}&\cd{update-\discretionary{}{}{}instance-\discretionary{}{}{}for-\discretionary{}{}{}different-\discretionary{}{}{}class}\\
\cdf{no-applicable-method}&\cd{update-\discretionary{}{}{}instance-\discretionary{}{}{}for-\discretionary{}{}{}redefined-\discretionary{}{}{}class}\\
\cdf{no-next-method}&
\end{tabular}
\end{flushleft}

\item 
{\it Tools for declarative method combination}

\begin{flushleft}
\cf
\begin{tabular}{@{}ll@{}}
\cdf{call-method}&\cdf{method-combination-error}\\
\cd{define-method-combination~~}&\cdf{method-qualifiers}\\
\cdf{invalid-method-error}&
\end{tabular}
\end{flushleft}

\item 
{\it General Common Lisp support tools}

\begin{flushleft}
\cf
\begin{tabular}{@{}ll@{}}
\cdf{class-of}&\cdf{print-object}\\
\cd{documentation~~~~~~~~~~~~~~}&\cdf{symbol-macrolet}
\end{tabular}
\end{flushleft}

[Note that \cdf{describe} appeared in this list in the original CLOS proposal
\cite{SIGPLAN-CLOS,LASC-CLOS-PART-2}, but X3J13 voted in March 1989 \issue{DESCRIBE-UNDERSPECIFIED}
not to make \cdf{describe} a generic function after all (see \cdf{describe-object}).---GLS]
\end{itemize}

\vskip10pt
\noindent
[At this point the original CLOS report contained a description of the
\Mchoice{ } and \Mind{} notation; that description is omitted here.
I have adopted the notation for use
throughout this book. It is described in
section~\ref{FUNCTION-HEADER-NOTATION-SECTION}.---GLS]

\begin{defun}[Generic function][Primary method]
add-method generic-function method \\
add-method~~~~~~~~~~~~~~~~~~~~~~~~~~~~~~~~ (generic-function standard-generic-function) (method method)

The generic function \cdf{add-method} adds a method to a generic
function.  It destructively modifies the generic function and returns
the modified generic function as its result.


The {\it generic-function\/} argument is a generic function
object.

The {\it method\/} argument is a method object.  The lambda-list of
the method function must be congruent with the lambda-list of the
generic function, or an error is signaled.


The modified generic function is returned.  The result of \cdf{add-method} 
is \cdf{eq} to the {\it generic-function\/} argument.


If the given method agrees with an existing method of the generic
function on parameter specializers and qualifiers, the existing method
is replaced.  See section~\ref{Agreement-on-Parameter-Specializers-and-Qualifiers-SECTION}
for a definition of agreement in this context.

If the method object is a method object of another generic function,
an error is signaled.

See section~\ref{Agreement-on-Parameter-Specializers-and-Qualifiers-SECTION}
as well as
\cdf{defmethod},
\cdf{defgeneric},
\cdf{find-method},
and \cdf{remove-method}.
\end{defun}


\begin{defmac}
call-method method next-method-list

The macro \cdf{call-method} is used in method combination.  This macro hides
the implementation-dependent details of how methods are called. It can be used only within
an effective method form, for the name \cdf{call-method} is defined only 
within the lexical scope of such a form.

The macro \cdf{call-method} invokes the specified method, supplying it
with arguments and with definitions for \cdf{call-next-method} and for
\cdf{next-method-p}.  The arguments are the arguments that were
supplied to the effective method form containing the invocation of
\cdf{call-method}.  The definitions of \cdf{call-next-method} and 
\cdf{next-method-p} rely on the list of method objects given as the second
argument to \cdf{call-method}.

The \cdf{call-next-method} function available to the method that
is the first subform will call the first method in the list that
is the second subform.  The \cdf{call-next-method} function
available in that method, in turn, will call the second
method in the list that is the second subform, and so on, until
the list of next methods is exhausted.





The {\it method\/} argument is a method object; the {\it
next-method-list\/} argument is a list of method objects.

A list whose first element is the symbol \cdf{make-method} and whose
second element is a Lisp form can be used instead of a method object
as the first subform of \cdf{call-method} or as an element of the
second subform of \cdf{call-method}.  Such a list specifies a method
object whose method function has a body that is the given form.


The result of \cdf{call-method} is the value or values returned by
the method invocation.

See \cdf{call-next-method}, \cdf{define-method-combination}, and \cdf{next-method-p}.
\end{defmac}


\begin{defun}[Function]
call-next-method &rest args

The function \cdf{call-next-method} can be used within the body of a
method defined by a method-defining form to call the next method.

The function \cdf{call-next-method} returns the value or values
returned by the method it calls.  If there is no next method, 
the generic function \cdf{no-next-method} is called.

The type of method combination used determines which 
methods can invoke \cdf{call-next-method}.  The standard method
combination type allows \cd{call-\discretionary{}{}{}next-\discretionary{}{}{}method}
to be used within primary
methods and \cd{:around} methods.

The standard method combination
type defines the next method according to the following rules:

\begin{itemize}
\item 
If \cdf{call-next-method} is used in an \cd{:around} method,
the next method is the next most specific \cd{:around} method, if one is
applicable.

\item 
If there are no \cd{:around} methods at all or if 
\cdf{call-next-method} is called by the least specific \cd{:around}
method,  other methods are called as follows:

\begin{itemize}
\item  All the \cd{:before} methods are called, in
most-specific-first order.  The function \cdf{call-next-method}
cannot be used in \cd{:before} methods.

\item 
The most specific primary method is called.  Inside the body of a
primary method, \cdf{call-next-method} may be used to pass control to
the next most specific primary method.  The generic function 
\cdf{no-next-method} is called if \cdf{call-next-method} is used and there
are no more primary methods.

\item  All the \cd{:after} methods are called in
most-specific-last order.  The function \cdf{call-next-method}
cannot be used in \cd{:after} methods.
\end{itemize}
\end{itemize}

For further discussion of the use of \cdf{call-next-method}, see
sections~\ref{Standard-Method-Combination-SECTION}
and~\ref{Built-in-Method-Combination-Types-SECTION}.




When \cdf{call-next-method} is called with no arguments, it passes the
current method's original arguments to the next method.  Neither
argument defaulting, nor using \cdf{setq}, nor rebinding variables
with the same names as parameters of the method affects the values
\cdf{call-next-method} passes to the method it calls.

When \cdf{call-next-method} is called with arguments, the next method
is called with those arguments.  When providing arguments to 
\cdf{call-next-method}, the following rule must be satisfied or an error is
signaled: The ordered set of methods applicable for a changed set of
arguments for \cdf{call-next-method} must be the same as the ordered set of
applicable methods for the original arguments to the generic function.
Optimizations of the error checking are possible, but they must 
not change the semantics of \cdf{call-next-method}.

If \cdf{call-next-method} is called with arguments but omits
optional arguments, the next method called defaults those arguments.



The function \cdf{call-next-method} returns the value or values
returned by the method it calls.  


Further computation is possible after \cdf{call-next-method} returns.

The definition of the function \cdf{call-next-method} has lexical scope (for it
is defined only within the body of a method defined by a method-defining form)
and indefinite extent.

For generic functions using a type of method combination defined by
the short form of \cdf{define-method-combination}, 
\cdf{call-next-method} can be used in \cd{:around} methods only.

The function \cdf{next-method-p} can be used to test whether or not there is
a next method.

If \cdf{call-next-method} is used in methods that do not support it,
an error is signaled.

See sections~\ref{Method-Selection-and-Combination-SECTION},
\ref{Standard-Method-Combination-SECTION}, and
\ref{Built-in-Method-Combination-Types-SECTION} as well as the functions
\cd{define-\discretionary{}{}{}method-\discretionary{}{}{}combination},
\cdf{next-method-p},
and \cdf{no-next-method}.
\end{defun}


\begin{defun}[Generic function][Primary method]
change-class instance new-class \\
change-class (instance standard-object) (new-class standard-class) \\
change-class (instance t) (new-class symbol)

The generic function \cdf{change-class} changes the class of an
instance to a new class.  It destructively modifies and returns the
instance.

If in the old class there is any slot of the same name as a local
slot in the new class, the value of that slot is retained.  This
means that if the slot has a value, the value returned by 
\cdf{slot-value} after \cdf{change-class} is invoked is \cdf{eql} to the
value returned by \cdf{slot-value} before \cdf{change-class} is
invoked.  Similarly, if the slot was unbound, it remains
unbound.  The other slots are initialized as described in
section~\ref{Changing-the-Class-of-an-Instance-SECTION}.





The {\it instance\/} argument is a Lisp object.

The {\it new-class\/} argument is a class object or a symbol that names
a class. 

If the second of the preceding methods is selected, that method
invokes \cdf{change-class} on {\it instance\/} and 
\cd{(find-class {\it new-class\/})}.


The modified instance is returned.  The result of \cdf{change-class}
is \cdf{eq} to the {\it instance} argument.

Examples:

\begin{lisp}
(defclass position () ()) \\
\\
(defclass x-y-position (position) \\*
~~((x :initform 0 :initarg :x) \\*
~~~(y :initform 0 :initarg :y))) \\
\\
(defclass rho-theta-position (position) \\*
~~((rho :initform 0) \\*
~~~(theta :initform 0))) \\
\\
(defmethod update-instance-for-different-class :before \\*
~~~~~~~~~~~((old x-y-position)  \\*
~~~~~~~~~~~~(new rho-theta-position) \\*
~~~~~~~~~~~~\&key) \\
~~;; Copy the position information from old to new to make new \\*
~~;; be a rho-theta-position at the same position as old. \\*
~~(let ((x (slot-value old 'x)) \\*
~~~~~~~~(y (slot-value old 'y))) \\*
~~~~(setf (slot-value new 'rho) (sqrt (+ (* x x) (* y y))) \\*
~~~~~~~~~~(slot-value new 'theta) (atan y x))))
\end{lisp}
\begin{lisp}
;;; At this point an instance of the class x-y-position can be \\*
;;; changed to be an instance of the class rho-theta-position \\*
;;; using change-class: \\
\\
(setq p1 (make-instance 'x-y-position :x 2 :y 0)) \\
\\
(change-class p1 'rho-theta-position) \\
\\
;;; The result is that the instance bound to p1 is now \\*
;;; an instance of the class rho-theta-position. \\*
;;; The update-instance-for-different-class method \\*
;;; performed the initialization of the rho and theta \\*
;;; slots based on the values of the x and y slots, \\*
;;; which were maintained by the old instance.
\end{lisp}


After completing all other actions, \cdf{change-class} invokes the generic
function \cd{update-\discretionary{}{}{}instance-\discretionary{}{}{}for-\discretionary{}{}{}different-\discretionary{}{}{}class}.  The generic function
\cd{update-\discretionary{}{}{}instance-\discretionary{}{}{}for-\discretionary{}{}{}different-\discretionary{}{}{}class}
can be used to assign values to slots in the transformed instance.

The generic function \cdf{change-class} has several semantic difficulties.
First, it performs a destructive operation that can be invoked within a
method on an instance that was used to select that method. When multiple
methods are involved because methods are being combined,
the methods currently executing or about to be executed
may no longer be applicable.  Second, some implementations might use compiler
optimizations of slot access, and when the class of an instance is
changed the assumptions the compiler made might be violated.
This implies that a programmer must not use 
\cdf{change-class} inside a method if any methods for that generic function 
access any slots, or the results are undefined.


See section~\ref{Changing-the-Class-of-an-Instance-SECTION} as well as
\cd{update-\discretionary{}{}{}instance-\discretionary{}{}{}for-\discretionary{}{}{}different-\discretionary{}{}{}class}.
\end{defun}



\begin{defun}[Generic function][Primary method]
class-name class \\
class-name (class class)

The generic function \cdf{class-name} takes a class object and returns its
name.
The {\it class\/} argument is a class object.
The {\it new-value\/} argument is any object.
The name of the given class is returned.

The name of an anonymous class is \cdf{nil}.

If {\it S} is a symbol such that {\it S}~$=$\cd{(class-name {\it C})} and {\it C}~$=$
\cd{(find-class {\it S})}, then {\it S} is the proper name of {\it C} (see section~\ref{Classes-SECTION}).

See also section~\ref{Classes-SECTION} and \cdf{find-class}.
\end{defun}


\begin{defun}[Generic function][Primary method]
(setf~class-name) new-value class \\
(setf~class-name) new-value (class class)

The generic function \cd{(setf class-name)} takes a class object and sets
its name.
The {\it class\/} argument is a class object.
The {\it new-value\/} argument is any object.
\end{defun}


\begin{defun}[Function]
class-of object

The function \cdf{class-of} returns the class of which
the given object is an instance.
The argument to \cdf{class-of} may be any Common Lisp object.
The function \cdf{class-of} returns the class of which
the argument is an instance.

\end{defun}


\begin{defun}[Function]
compute-applicable-methods generic-function function-arguments

Given a generic function and a set of arguments, the function
\cd{compute-\discretionary{}{}{}applicable-\discretionary{}{}{}methods} returns the set of methods
that are applicable for those arguments.

The methods are
sorted according to precedence order.
See section~\ref{Method-Selection-and-Combination-SECTION}.





The {\it generic-function\/} argument must be a generic function object.
The {\it function-arguments\/} argument is a list of the arguments to
that generic function.
The result is a list of the applicable methods in order of precedence.
See section~\ref{Method-Selection-and-Combination-SECTION}.
\end{defun}


\begin{defmac}
defclass class-name ({superclass-name}*)
         ({slot-specifier}*) <?class-option>

\begin{tabbing}
{\it class-name\/} ::= {\it symbol\/} \\
{\it superclass-name\/} ::= {\it symbol\/}\\
%\cleartabs
{\it slot-specifier\/} ::= {\it slot-name\/} {\Mor} ({\it slot-name\/}  $\lbrack\!\lbrack\downarrow\!\hbox{{\it slot-option}}\,\rbrack\!\rbrack$)\\
{\it slot-name\/} ::= {\it symbol\/}\\
\pushtabs{\it slot-option\/} ::= \=\Mstar{{\cd{:reader} {\it reader-function-name\/}}} \\
\>\hbox to 0pt{\hss\Mor~}\Mstar{{\cd{:writer} {\it writer-function-name\/}}} \\
\>\hbox to 0pt{\hss\Mor~}\Mstar{{\cd{:accessor} {\it reader-function-name\/}}} \\
\>\hbox to 0pt{\hss\Mor~}\Mgroup{\cd{:allocation} {\it allocation-type\/}} \\
\>\hbox to 0pt{\hss\Mor~}\Mstar{{\cd{:initarg} {\it initarg-name\/}}} \\
\>\hbox to 0pt{\hss\Mor~}\Mgroup{\cd{:initform} {\it form\/}} \\
\>\hbox to 0pt{\hss\Mor~}\Mgroup{\cd{:type} {\it type-specifier\/}} \\
\>\hbox to 0pt{\hss\Mor~}\Mgroup{\cd{:documentation} {\it string\/}} \poptabs
\end{tabbing}
\penalty-10000 %required
\begin{tabbing}
{\it reader-function-name\/} ::= {\it symbol\/}\\
{\it writer-function-name\/} ::= {\it function-name/}\\
{\it function-name\/} ::= \Mgroup{{\it symbol\/} {\Mor} \cd{(setf {\it symbol\/})}}\\
{\it initarg-name\/} ::= {\it symbol\/}\\
{\it allocation-type\/} ::= \cd{:instance {\Mor} :class}\\
\pushtabs{\it class-option\/} ::= \=\cd{(:default-initargs {\it initarg-list\/})} \\
\>\hbox to 0pt{\hss\Mor~}\cd{(:documentation {\it string\/})} \\
\>\hbox to 0pt{\hss\Mor~}\cd{(:metaclass {\it class-name\/})} \poptabs \\
{\it initarg-list\/} ::= \Mstar{{\it initarg-name default-initial-value-form}}
\end{tabbing}
The macro \cdf{defclass} defines a new named class.  It returns the new class
object as its result.

The syntax of \cdf{defclass} provides options for specifying
initialization arguments for slots, for specifying default
initialization values for slots, and for requesting that methods on
specified generic functions be automatically generated for reading and
writing the values of slots.  No reader or writer functions are
defined by default; their generation must be explicitly requested.

Defining a new class also causes a type of the same name to be
defined.  The predicate \cd{(typep {\it object class-name\/})} returns
true if the class of the given object is {\it class-name\/} itself or
a subclass of the class {\it class-name}.  A class object can be used
as a type specifier.  Thus \cd{(typep {\it object class\/})} returns true
if the class of the {\it object\/} is {\it class\/} itself or a
subclass of {\it class}.





The {\it class-name\/} argument is a non-\cdf{nil} symbol.  It becomes
the proper name of the new class.  If a class with the same proper
name already exists and that class is an instance of 
\cdf{standard-class}, and if the \cdf{defclass} form for the definition of the
new class specifies a class of class \cdf{standard-class}, the definition
of the existing class is replaced.

Each {\it superclass-name\/} argument is a non-\cdf{nil} symbol that
specifies a direct superclass of the new class.  The new class will
inherit slots and methods from each of its direct superclasses, from
their direct superclasses, and so on.  See
section~\ref{Inheritance-SECTION}
for a discussion of how slots and methods are inherited.

Each {\it slot-specifier\/} argument is the name of the slot or a list
consisting of the slot name followed by zero or more slot options.
The {\it slot-name\/} argument is a symbol that is syntactically valid
for use as a variable name.  If there are any duplicate
slot names, an error is signaled.

The following slot options are available:

\begin{itemize}

\item 
The \cd{:reader} slot option specifies that an unqualified method is
to be defined on the generic function named {\it
reader-function-name\/} to read the value of the given slot.
The {\it reader-function-name\/} argument is a non-\cdf{nil}
symbol.  The \cd{:reader} slot option may be specified more than once
for a given slot.

\item  
The \cd{:writer} slot option specifies that an unqualified method is
to be defined on the generic function named {\it
writer-function-name\/} to write the value of the slot.  The
{\it writer-function-name\/} argument is a function-name.
The \cd{:writer} slot option may be specified more than once for a
given slot.

\item  
The \cd{:accessor} slot option specifies that an unqualified method
is to be defined on the generic function named {\it
reader-function-name\/} to read the value of the given slot
and that an unqualified method is to be defined on the generic
function named \cd{(setf {\it reader-function-name\/})} to be
used with \cdf{setf} to modify the value of the slot.  The {\it
reader-function-name\/} argument is a non-\cdf{nil} symbol.
The \cd{:accessor} slot option may be specified more than once for a
given slot.

\item  
The \cd{:allocation} slot option is used to specify where storage is
to be allocated for the given slot.  Storage for a slot may be located
in each instance or in the class object itself, for example.  The value of the {\it
allocation-type\/} argument can be either the keyword \cd{:instance}
or the keyword \cd{:class}.  The \cd{:allocation} slot option may be
specified at most once for a given slot.  If the \cd{:allocation}
slot option is not specified, the effect is the same as specifying
\cd{:allocation :instance}.

\begin{itemize}
\item
If {\it allocation-type\/} is \cd{:instance}, a local slot of the given name
is allocated in each instance of the class.  

\item
If {\it allocation-type\/} is \cd{:class}, a shared slot of the given
name is allocated.  The value of the slot is shared by all instances of the class.
If a class ${\it C}\sub1$ defines such a shared slot, any subclass ${\it C}\sub2$ of
${\it C}\sub1$ will share this single slot unless the \cdf{defclass} form
for ${\it C}\sub2$ specifies a slot of the same name or there is a
superclass of ${\it C}\sub2$ that precedes ${\it C}\sub1$ in the class precedence
list of ${\it C}\sub2$ and that defines a slot of the same name.
\end{itemize}

\item  The \cd{:initform} slot option is used to provide a default
initial value form to be used in the initialization of the slot.  The
\cd{:initform} slot option may be specified at most once for a given
slot.  This form is evaluated every time it is used to initialize the
slot.  The lexical
environment in which this form is evaluated is the lexical environment
in which the \cdf{defclass} form was evaluated.  Note that the lexical
environment refers both to variables and to functions.  For local
slots, the dynamic environment is the dynamic environment in which
\cdf{make-instance} was called; for shared slots, the dynamic
environment is the dynamic environment in which the \cdf{defclass}
form was evaluated.  See section \ref{Object-Creation-and-Initialization-SECTION}.

No implementation is permitted to extend the syntax of \cdf{defclass}
to allow \cd{({\it slot-name form\/})} as an abbreviation for 
\cd{({\it slot-name\/} :initform {\it form\/})}.

\item 
The \cd{:initarg} slot option declares an initialization argument
named {\it initarg-name\/} and specifies that this initialization argument
initializes the given slot.  If the initialization argument has a
value in the call to \cdf{initialize-instance}, the value will be
stored into the given slot, and the slot's \cd{:initform} slot option, if
any, is not evaluated.  If none of the initialization arguments
specified for a given slot has a value, the slot is initialized
according to the \cd{:initform} slot option, if specified.  The 
\cd{:initarg} slot option can be specified more than once for a given
slot.  The {\it initarg-name\/} argument can be any symbol.

\item 
The \cd{:type} slot option specifies that the contents of the slot
will always be of the specified data type.  It effectively declares
the result type of the reader generic function when applied to an
object of this class.  The result of attempting to store in a slot a
value that does not satisfy the type of the slot is undefined.  The
\cd{:type} slot option may be specified at most once for a given
slot.  The \cd{:type} slot option is further discussed in
section~\ref{Inheritance-of-Slots-and-Slot-Options-SECTION}.

\item 
The \cd{:documentation} slot option provides a documentation string
for the slot.
\end{itemize}

Each class option is an option that refers to the class as a whole
or to all class slots.  The following class options are available:

\begin{itemize}
\item 
The \cd{:default-initargs} class option is followed by a list of
alternating initialization argument names and default initial value
forms.  If any of these initialization arguments does not appear in
the initialization argument list supplied to \cdf{make-instance}, the
corresponding default initial value form is evaluated, and the
initialization argument name and the form's value are added to the end
of the initialization argument list before the instance is created
(see section~\ref{Object-Creation-and-Initialization-SECTION}).  The default
initial value form is evaluated each time it is used.  The lexical
environment in which this form is evaluated is the lexical environment
in which the \cdf{defclass} form was evaluated.  The dynamic
environment is the dynamic environment in which \cdf{make-instance}
was called.  If an initialization argument name appears more than once
in a \cd{:default-initargs} class option, an error is signaled.  The
\cd{:default-initargs} class option may be specified at most once.


\item  
The \cd{:documentation} class option causes a documentation string to be
attached to the class name.  The documentation type for this string is
\cdf{type}.  The form \cd{(documentation {\it class-name\/} 'type)}
may be used to retrieve the documentation string.  The 
\cd{:documentation} class option may be specified at most once.

\item 
The \cd{:metaclass} class option is used to specify that instances of the
class being defined are to have a different metaclass than the default
provided by the system (the class \cdf{standard-class}).  The {\it
class-name} argument is the name of the desired metaclass.  The 
\cd{:metaclass} class option may be specified at most once.

\end{itemize}


The new class object is returned as the result.


If a class with the same proper name already exists and that class is
an instance of \cdf{standard-class}, and if the \cdf{defclass} form for
the definition of the new class specifies a class of class 
\cdf{standard-class}, the existing class is redefined, and instances of it
(and its subclasses) are updated to the new definition at the time
that they are next accessed (see section~\ref{Redefining-Classes-SECTION}).

Note the following rules of \cdf{defclass} for standard classes:

\begin{itemize}

\item 
It is not required that the superclasses of a class be defined before
the \cdf{defclass} form for that class is evaluated.

\item 
All the superclasses of a class must be defined before 
an instance of the class can be made.

\item 
A class must be defined before it can be used as a parameter
specializer in a \cdf{defmethod} form.

\end{itemize}

The \OS\ may be extended to cover situations where these rules are not
obeyed.

Some slot options are inherited by a class from its superclasses, and
some can be shadowed or altered by providing a local slot description.
No class options except \cd{:default-initargs} are inherited.  For a
detailed description of how slots and slot options are inherited, see
section~\ref{Inheritance-of-Slots-and-Slot-Options-SECTION}.

The options to \cdf{defclass} can be extended.
An implementation must signal an error if it observes a class option or
a slot option that is not implemented locally.

It is valid to specify more than one reader, writer, accessor, or
initialization argument for a slot.  No other slot option may appear
more than once in a single slot description, or an error is
signaled.

If no reader, writer, or accessor is specified for a slot, the slot
can be accessed only by the function \cdf{slot-value}.

See sections \ref{Classes-SECTION},
\ref{Inheritance-SECTION},
\ref{Redefining-Classes-SECTION},
\ref{Determining-the-Class-Precedence-List-SECTION},
\ref{Object-Creation-and-Initialization-SECTION} as well as
\cdf{slot-value},
\cdf{make-instance}, and
\cdf{initialize-instance}.
\end{defmac}



\begin{defmac}
defgeneric function-name lambda-list
           <?option | {method-description}*>

\begin{tabbing}
{\it function-name\/} ::= \Mgroup{{\it symbol\/} {\Mor} \cd{(setf {\it symbol\/})}} \\*
\pushtabs{\it lambda-list\/} ::= \cd{(}\=\Mstar{{var}}  \\*
\>\Mopt{\cd{\&optional} \Mstar{{var {\Mor} \cd{({\it var\/})}}}}  \\*
\>\Mopt{\cd{\&rest} {\it var\/}} \\*
\>\Mopt{\cd{\&key} \Mstar{keyword-parameter} \Mopt{\cd{\&allow-other-keys}}}\cd{)} \poptabs
\end{tabbing}
\begin{tabbing}
{\it keyword-parameter\/} ::= {\it var} {\Mor} \cd{(\Mgroup{var {\Mor} \cd{({\it keyword\/} {\it var\/})}})} \\[2pt]
\pushtabs{\it option\/} ::= \=\cd{(:argument-precedence-order \Mplus{parameter-name})} \\[2pt]
\>\hbox to 0pt{\hss\Mor~}\cd{(declare \Mplus{declaration})} \\
\>\hbox to 0pt{\hss\Mor~}\cd{(:documentation {\it string\/})} \\
\>\hbox to 0pt{\hss\Mor~}\cd{(:method-combination {\it symbol\/} \Mstar{{arg\/}})} \\
\>\hbox to 0pt{\hss\Mor~}\cd{(:generic-function-class {\it class-name\/})} \\
\>\hbox to 0pt{\hss\Mor~}\cd{(:method-class {\it class-name\/})} \poptabs \\[2pt]
\pushtabs{\it method-description\/} ::= \cd{(:method }\=\Mstar{{method-qualifier\/}} \\
\>{\it specialized-lambda-list\/} \\
\>\Mchoice{{\Mstar{declaration\/} {\Mor} documentation\/}} \\
\>\Mstar{{\,form\/}}\cd{)} \poptabs \\[2pt]
{\it method-qualifier\/} ::= {\it non-nil-atom} \\[2pt]
\pushtabs{\it specialized-lambda-list\/} ::= \\*
\hskip 2pc \cd{(}\=\Mstar{{var {\Mor} \cd{(}var parameter-specializer-name\/\cd{)}}}  \\
\>\Mopt{\cd{\&optional} \Mstar{{var {\Mor} \cd{(}var \Mopt{initform {\Mopt{supplied-p-parameter}}}\cd{)}}}}  \\
\>\Mopt{\cd{\&rest {\it var\/}}} \\
\>\Mopt{\cd{\&key} \Mstar{specialized-keyword-parameter} \Mopt{\cd{\&allow-other-keys}}} \\
\>\Mopt{\cd{\&aux} \Mstar{{var {\Mor} \cd{({\it var\/} \Mopt{initform})}}}}\cd{)} \poptabs \\[2pt]
{\it specialized-keyword-parameter\/} ::= \\
\hskip 2pc {\it var} {\Mor} \cd{(}\Mgroup{var {\Mor} \cd{({\it keyword\/} {\it var\/}\cd{)}}}
       \Mopt{initform \Mopt{supplied-p-parameter}}\cd{)} \\[2pt]
{\it parameter-specializer-name\/} ::= {\it symbol} {\Mor} \cd{(eql {\it eql-specializer-form\/})}
\end{tabbing}
The macro \cdf{defgeneric} is used to define a generic function or to
specify options and declarations that pertain to a generic function as
a whole.

If \cd{(fboundp {\it function-name\/})} is \cdf{nil}, a new
generic function is created.  If \cd{(fdefinition {\it
function-specifier\/})} is a generic function, that generic function
is modified.  If {\it function-name/} names a non-generic
function, a macro, or a special form, an error is signaled.

[X3J13 voted in March 1989 \issue{FUNCTION-NAME} to use \cdf{fdefinition}
in the previous paragraph, as shown, rather than \cdf{symbol-function},
as it appeared in the original report on CLOS~\cite{SIGPLAN-CLOS,LASC-CLOS-PART-2}.
The vote also changed all occurrences of {\it function-specifier} in the
original report to {\it function-name}; this change is reflected here.---GLS]

Each {\it method-description\/} defines a method on the generic function.
The lambda-list of each method must be congruent with the lambda-list
specified by the {\it lambda-list\/} option.  If this condition
does not hold, an error is signaled.
See section~\ref{Congruent-Lambda-Lists-for-All-Methods-of-a-Generic-Function-SECTION}
for a definition
of congruence in this context.

The macro \cdf{defgeneric} returns the generic function object 
as its result.




The {\it function-name} argument is a non-\cdf{nil} symbol or a
list of the form \cd{(setf {\it symbol\/})}.

\penalty-10000 %required

The {\it lambda-list\/} argument is an ordinary function lambda-list
with the following exceptions:

\begin{itemize}
\item 
The use of \cd{\&aux} is not allowed. 

\item 
Optional and keyword arguments may not have default initial value forms
nor use supplied-p parameters.
The generic function passes to the method all the argument values passed to
it, and only those; default values are not supported.
Note that optional and keyword arguments in method definitions, however,
can have default initial value forms and can use supplied-p parameters. 
\end{itemize}

The following options are provided.  A given option may occur only once,
or an error is signaled.

\begin{itemize}
 
\item  
The \cd{:argument-precedence-order} option is used to specify the
order in which the required arguments in a call to the generic
function are tested for specificity when selecting a particular
method.  Each required argument, as specified in the {\it lambda-list\/}
argument, must be included exactly once as a {\it
parameter-name} so that the full and unambiguous precedence order is
supplied.  If this condition is not met, an error is signaled.

\item 
The \cdf{declare} option is used to specify declarations that pertain
to the generic function.  The following standard Common Lisp
declaration is allowed:

\begin{itemize}
\item
An \cdf{optimize} declaration specifies whether method selection
should be optimized for speed or space, but it has no effect on
methods.  To control how a method is optimized, an \cdf{optimize}
declaration must be placed directly in the \cdf{defmethod} form or
method description.  The optimization qualities \cdf{speed} and 
\cdf{space} are the only qualities this standard requires, but an
implementation can extend the \CLOS\ to recognize other qualities.  A
simple implementation that has only one method selection technique and
ignores the \cdf{optimize} declaration is valid.
\end{itemize}

The \cdf{special}, \cdf{ftype}, \cdf{function}, \cdf{inline}, 
\cdf{notinline}, and \cdf{declaration} declarations are not permitted.
Individual implementations can extend the \cdf{declare} option to
support additional declarations.  If an implementation notices a
declaration that it does not support and that has not been proclaimed
as a non-standard declaration name in a \cdf{declaration} proclamation, it
should issue a warning.

\item  
The \cd{:documentation} argument associates a documentation string
with the generic function.  The documentation type for this string is
\cdf{function}.  The form \cd{(documentation {\it
function-name/} 'function)} may be used to retrieve this
string.

\item  
The \cd{:generic-function-class} option may be used to specify that
the generic function is to have a different class than the default
provided by the system (the class \cdf{standard-generic-function}).
The {\it class-name\/} argument is the name of a class that can be the
class of a generic function.  If {\it function-name\/} specifies
an existing generic function that has a different value for the 
\cd{:generic-function-class} argument and the new generic function class
is compatible with the old, \cdf{change-class} is called to change the
class of the generic function; otherwise an error is signaled.

\item  
The \cd{:method-class} option is used to specify that all methods on
this generic function are to have a different class from the default
provided by the system (the class \cdf{standard-method}).  The {\it
class-name\/} argument is the name of a class that is capable of being
the class of a method.

\item  
The \cd{:method-combination} option is followed by a symbol that
names a type of method combination.  The arguments (if any) that
follow that symbol depend on the type of method combination.  Note
that the standard method combination type does not support any
arguments.  However, all types of method combination defined by the
short form of \cdf{define-method-combination} accept an optional
argument named {\it order\/}, defaulting to 
\cd{:most-specific-first}, where a value of \cd{:most-specific-last} reverses
the order of the primary methods without affecting the order of the
auxiliary methods.

\end{itemize}

The {\it method-description\/} arguments define methods that will
be associated with the generic function.  The {\it method-qualifier}
and {\it specialized-lambda-list} arguments in a method description
are the same as for \cdf{defmethod}.

The {\it form\/} arguments specify the method body.  The body of the
method is enclosed in an implicit block.  If {\it
function-name\/} is a symbol, this block bears the same name as
the generic function.  If {\it function-name\/} is a list of the
form \cd{(setf {\it symbol\/})}, the name of the block is {\it
symbol}.  


The generic function object is returned as the result. 


The effect of the \cdf{defgeneric} macro is as if the following three
steps were performed: first, methods defined by previous 
\cdf{defgeneric} forms are removed; second, \cdf{ensure-generic-function}
is called; and finally, methods specified by the current 
\cdf{defgeneric} form are added to the generic function. 

If no method descriptions are specified and a generic function of the same
name does not already exist, a generic function with no methods is created.


The {\it lambda-list\/} argument of 
\cdf{defgeneric} specifies the shape of lambda-lists for the methods on
this generic function.  All methods on the resulting generic function must have
lambda-lists that are congruent with this shape.  If a 
\cdf{defgeneric} form is evaluated and some methods for that generic
function have lambda-lists that are not congruent with that given in
the \cdf{defgeneric} form, an error is signaled.  For further details
on method congruence,
see section~\ref{Congruent-Lambda-Lists-for-All-Methods-of-a-Generic-Function-SECTION}.

Implementations can extend \cdf{defgeneric} to include other options.
It is required that an implementation signal an error if
it observes an option that is not implemented locally.

See section~\ref{Congruent-Lambda-Lists-for-All-Methods-of-a-Generic-Function-SECTION}
as well as \cdf{defmethod}, \cdf{ensure-generic-function}, and \cdf{generic-function}.
\end{defmac}


\begin{defmac}
define-method-combination name <?short-form-option> \\
define-method-combination name lambda-list
    ({method-group-specifier}*)
    [(\!:arguments! \!.! lambda-list)]
    [(\!:generic-function! generic-fn-symbol)]
    <{declaration}* | doc-string>
    {\,form}*

\begin{tabbing}
\pushtabs{\it short-form-option\/} ::= \=\cd{:documentation {\it string\/}} \\
\>\hbox to 0pt{\hss\Mor~}\cd{:identity-with-one-argument {\it boolean\/}} \\
\>\hbox to 0pt{\hss\Mor~}\cd{:operator {\it operator\/}} \poptabs \\
\pushtabs{\it method-group-specifier\/} ::= \cd{(}\={\it variable\/}
    \Mgroup{\Mplus{{qualifier-pattern}} {\Mor} predicate\/} \\
\>\Mchoice{\Mind{long-form-option}}\cd{)} \poptabs \\
\pushtabs{\it long-form-option\/} ::= \=\cd{:description {\it format-string\/}} \\
\>\hbox to 0pt{\hss\Mor~}\cd{:order {\it order\/}} \\
\>\hbox to 0pt{\hss\Mor~}\cd{:required {\it boolean\/}} \poptabs
\end{tabbing}
The macro \cdf{define-method-combination} is used to define new types
of method combination.

There are two forms of \cdf{define-method-combination}.  The short
form is a simple facility for the cases that are expected
to be most commonly needed.  The long form is more powerful but more
verbose.  It resembles \cdf{defmacro} in that the body is an
expression, usually using backquote, that computes a Lisp form.  Thus
arbitrary control structures can be implemented.  The long form also
allows arbitrary processing of method qualifiers.





In both the short and long forms, {\it name\/} is a symbol.  By convention,
non-keyword, non-\cdf{nil} symbols are usually used.

\medskip

The short-form syntax of \cdf{define-method-combination} is recognized
when the second subform is a non-\cdf{nil} symbol or is not present.
When the short form is used, {\it name\/} is defined as a type of
method combination that produces a Lisp form \cd{({\it operator
method-call method-call $\ldots$ })}.  The {\it operator\/} is a symbol
that can be the name of a function, macro, or special form.  The
{\it operator\/} can be specified by a keyword option; it defaults to {\it
name}.

Keyword options for the short form are the following:

\begin{itemize}

\item 
The \cd{:documentation} option is used to document the method-combination type.

\item 
The \cd{:identity-with-one-argument} option enables an optimization
when {\it boolean\/} is true (the default is false).  If there is
exactly one applicable method and it is a primary method, that method
serves as the effective method and {\it operator\/} is not called.
This optimization avoids the need to create a new effective method and
avoids the overhead of a function call.  This option is designed to be
used with operators such as \cdf{progn}, \cdf{and}, \cd{+}, and
\cdf{max}.

\item 
The \cd{:operator} option specifies the name of the operator.  The
{\it operator\/} argument is a symbol that can be the name of a
function, macro, or special form.  By convention, {\it name\/} and
{\it operator\/} are often the same symbol.  This is the default,
but it is not required.

\end{itemize}

None of the subforms is evaluated.

These types of method combination require exactly one qualifier per
method.  An error is signaled if there are applicable methods with no
qualifiers or with qualifiers that are not supported by the method
combination type. 

A method combination procedure defined in this way recognizes two
roles for methods.  A method whose one qualifier is the symbol naming
this type of method combination is defined to be a primary method.  At
least one primary method must be applicable or an error is signaled.
A method with \cd{:around} as its one qualifier is an auxiliary
method that behaves the same as an \cd{:around} method in standard
method combination.  The function \cdf{call-next-method} can be
used only in \cd{:around} methods; it cannot be used in primary methods
defined by the short form of the \cdf{define-method-combination} macro.

A method combination procedure defined in this way accepts an optional
argument named {\it order}, which defaults to 
\cd{:most-specific-first}.  A value of \cd{:most-specific-last} reverses
the order of the primary methods without affecting the order of the
auxiliary methods.

The short form automatically includes error checking and support for
\cd{:around} methods.

For a discussion of built-in method combination types,
see section~\ref{Built-in-Method-Combination-Types-SECTION}.

\medskip

The long-form syntax of \cdf{define-method-combination} is recognized 
when the second subform is a list.  

The {\it lambda-list\/} argument is an ordinary lambda-list.  It
receives any arguments provided after the name of the method
combination type in the \cd{:method-\discretionary{}{}{}combination} option to 
\cdf{defgeneric}.

A list of method group specifiers follows.  Each specifier selects a subset
of the applicable methods to play a particular role, either by matching
their qualifiers against some patterns or by testing their qualifiers with
a predicate.   These method group specifiers define all method qualifiers
that can be used with this type of method combination.  If an applicable 
method does not fall into any method group, the system signals the error
that the method is invalid for the kind of method combination in use.

Each method group specifier names a variable.  During the execution of
the forms in the body of \cdf{define-method-combination}, this
variable is bound to a list of the methods in the method group.  The
methods in this list occur in most-specific-first order.

A qualifier pattern is a list or the symbol \cd{*}.  A method matches
a qualifier pattern if the method's list of qualifiers is \cdf{equal}
to the qualifier pattern (except that the symbol \cd{*} in a qualifier
pattern matches anything).  Thus a qualifier pattern can be one of the
following: the empty list \cd{()}, which matches unqualified methods;
the symbol \cd{*}, which matches all methods; a true list, which
matches methods with the same number of qualifiers as the length of
the list when each qualifier matches the corresponding list element;
or a dotted list that ends in the symbol \cd{*} (the \cd{*} matches
any number of additional qualifiers).


Each applicable method is tested against the qualifier patterns and
predicates in left-to-right order.  As soon as a qualifier pattern matches
or a predicate returns true, the method becomes a member of the
corresponding method group and no further tests are made.  Thus if a method
could be a member of more than one method group, it joins only the first
such group.  If a method group has more than one qualifier pattern, a
method need only satisfy one of the qualifier patterns to be a member of
the group.

The name of a predicate function can appear instead of qualifier
patterns in a method group specifier.  The predicate is called for
each method that has not been assigned to an earlier method group; it
is called with one argument, the method's qualifier list.  The
predicate should return true if the method is to be a member of the
method group.  A predicate can be distinguished from a qualifier pattern
because it is a symbol other than \cdf{nil} or \cd{*}.

If there is an applicable method whose qualifiers are not valid
for the method combination type, the function \cdf{invalid-method-error}
is called.

Method group specifiers can have keyword options following the
qualifier patterns or predicate.  Keyword options can be distinguished from
additional qualifier patterns because they are neither lists nor the symbol
\cd{*}.  The keyword options are:
\vskip 0pt plus 4pt%manual
\hrule width 0pt\relax

\begin{itemize}

\item 
The \cd{:description} option is used to provide a description of the
role of methods in the method group.  Programming environment tools
use \cd{(apply \#'format stream {\it format-string\/}
(method-qualifiers {\it method\/}))} to print this description, which
is expected to be concise.  This keyword
option allows the description of a method qualifier to be defined in
the same module that defines the meaning of the method
qualifier.  In most cases, {\it format-string\/} will not contain any
\cdf{format} directives, but they are available for generality.  If 
\cd{:description} is not specified, a default description is generated
based on the variable name and the qualifier patterns and on whether
this method group includes the unqualified methods.  The argument {\it
format-string\/} is not evaluated.

\item 
The \cd{:order} option specifies the order of methods.  The {\it
order\/} argument is a form that evaluates to 
\cd{:most-specific-first} or \cd{:most-specific-last}.  If it evaluates
to any other value, an error is signaled.  This keyword option is a
convenience and does not add any expressive power.
If \cd{:order} is not specified, it defaults to \cd{:most-specific-first}.

\item 
The \cd{:required} option specifies whether at least one method in
this method group is required.  If the {\it boolean\/} argument is
non-\cdf{nil} and the method group is empty (that is, no applicable
methods match the qualifier patterns or satisfy the predicate), an
error is signaled.  This keyword option is a convenience and does not
add any expressive power.  If \cd{:required} is not specified,
it defaults to \cdf{nil}.  The {\it boolean\/} argument is not
evaluated.

\end{itemize}

The use of method group specifiers provides a convenient syntax to
select methods, to divide them among the possible roles, and to perform the
necessary error checking.  It is possible to perform further filtering
of methods in the body forms by using normal list-processing operations
and the functions \cdf{method-qualifiers} and 
\cdf{invalid-method-error}.  It is permissible to use \cdf{setq} on the
variables named in the method group specifiers and to bind additional
variables.  It is also possible to bypass the method group specifier
mechanism and do everything in the body forms.  This is accomplished
by writing a single method group with \cd{*} as its only qualifier
pattern; the variable is then bound to a list of all of the applicable
methods, in most-specific-first order.

The body {\it forms\/} compute and return the Lisp form that specifies
how the methods are combined, that is, the effective method.  The
effective method uses the macro \cdf{call-method}.  The definition of this macro has
lexical scope and is available only in an effective method form.
Given a method object in one of the lists produced by the method group
specifiers and a list of next methods, the macro \cdf{call-method}
will invoke the method so that \cdf{call-next-method} will have available
the next methods.

When an effective method has no effect other than to call a single
method, some implementations employ an optimization that uses the
single method directly as the effective method, thus avoiding the need
to create a new effective method.  This optimization is active when
the effective method form consists entirely of an invocation of
the \cdf{call-method} macro whose first subform is a method object and
whose second subform is \cdf{nil}.  Each 
\cdf{define-method-combination} body is responsible for stripping off
redundant invocations of \cdf{progn}, \cdf{and}, 
\cd{multiple-value-prog1}, and the like, if this optimization is desired.


The list \cd{(:arguments . {\it lambda-list\/})} can appear before
any declaration or documentation string.  This form is useful when
the method combination type performs some specific behavior as part of
the combined method and that behavior needs access to the arguments to
the generic function.  Each parameter variable defined by {\it
lambda-list\/} is bound to a form that can be inserted into the
effective method.  When this form is evaluated during execution of the
effective method, its value is the corresponding argument to the
generic function.  If {\it lambda-list\/} is not congruent to the
generic function's lambda-list, additional ignored parameters are
automatic\-ally inserted until it is congruent.  Thus it is permissible
for {\it lambda-list\/} to receive fewer arguments than the number
that the generic function expects.
 
Erroneous conditions detected by the body should be reported with
\cd{method-\discretionary{}{}{}combination-\discretionary{}{}{}error} or
\cd{invalid-\discretionary{}{}{}method-\discretionary{}{}{}error}; these functions
add any necessary contextual information to the error message and will
signal the appropriate error.

The body {\it forms\/} are evaluated inside the bindings created by the
lambda-list and method group specifiers.  Declarations at the head of
the body are positioned directly inside bindings created by the
lambda-list and outside the bindings of the method group variables. 
Thus method group variables cannot be declared.

Within the body {\it forms\/}, {\it generic-function-symbol}
is bound to the generic function object.

If a {\it doc-string\/} argument is present, it provides the
documentation for the method combination type.

The functions \cdf{method-combination-error} and 
\cdf{invalid-method-error} can be called from the body {\it forms\/} or
from functions called by the body {\it forms\/}.  The actions of these
two functions can depend on implementation-dependent dynamic variables
automatically bound before the generic function 
\cd{compute-\discretionary{}{}{}effective-\discretionary{}{}{}method} is called.

Note that two methods with identical specializers, but with different
qualifiers, are not ordered by the algorithm described in step~2 of
the method selection and combination process described in
section~\ref{Method-Selection-and-Combination-SECTION}.
Normally the two methods play
different roles in the effective method because they have different
qualifiers, and no matter how they are ordered in the result of step~2
the effective method is the same.  If the two methods play the same
role and their order matters, an error is signaled.  This happens as
part of the qualifier pattern matching in 
\cdf{define-method-combination}.

\penalty-10000 %required

The value returned by the \cdf{define-method-combination} macro is the new
method combination object.


Most examples of the long form of \cdf{define-method-combination} also
illustrate the use of the related functions that are provided as part
of the declarative method combination facility.

\vskip 0pt plus 4pt
\hrule width0pt\relax

\begin{lisp}
;;; Examples of the short form of define-method-combination \\
\\
(define-method-combination and :identity-with-one-argument t) \\
\\
(defmethod func and ((x class1) y) \\
~~...) \\
\\
;;; The equivalent of this example in the long form is: \\*
\\*
(define-method-combination and \\*
~~~~~~~~(\&optional (order ':most-specific-first)) \\*
~~~~~~~~((around (:around)) \\*
~~~~~~~~~(primary (and) :order order :required t)) \\
~~(let ((form (if (rest primary) \\*
~~~~~~~~~~~~~~~~~~{\Xbq}(and ,{\Xatsign}(mapcar \#'(lambda (method) \\*
~~~~~~~~~~~~~~~~~~~~~~~~~~~~~~~~~~~~~~{\Xbq}(call-method ,method ())) \\*
~~~~~~~~~~~~~~~~~~~~~~~~~~~~~~~~~~primary)) \\*
~~~~~~~~~~~~~~~~~~{\Xbq}(call-method ,(first primary) ())))) \\
~~~~(if around \\*
~~~~~~~~{\Xbq}(call-method ,(first around) \\*
~~~~~~~~~~~~~~~~~~~~~~(,{\Xatsign}(rest around) \\*
~~~~~~~~~~~~~~~~~~~~~~~(make-method ,form))) \\*
~~~~~~~~form))) \\
\\
\\
;;; Examples of the long form of define-method-combination \\*
\\*
;;; The default method-combination technique \\*
\\*
(define-method-combination standard () \\*
~~~~~~~~((around (:around)) \\*
~~~~~~~~~(before (:before)) \\*
~~~~~~~~~(primary () :required t) \\*
~~~~~~~~~(after (:after))) \\
~~(flet ((call-methods (methods) \\*
~~~~~~~~~~~(mapcar \#'(lambda (method) \\*
~~~~~~~~~~~~~~~~~~~~~~~{\Xbq}(call-method ,method ())) \\*
~~~~~~~~~~~~~~~~~~~methods))) \\
~~~~(let ((form (if (or before after (rest primary)) \\*
~~~~~~~~~~~~~~~~~~~~{\Xbq}(multiple-value-prog1 \\*
~~~~~~~~~~~~~~~~~~~~~~~(progn ,{\Xatsign}(call-methods before) \\*
~~~~~~~~~~~~~~~~~~~~~~~~~~~~~~(call-method ,(first primary) \\*
~~~~~~~~~~~~~~~~~~~~~~~~~~~~~~~~~~~~~~~~~~~,(rest primary))) \\*
~~~~~~~~~~~~~~~~~~~~~~~,{\Xatsign}(call-methods (reverse after))) \\*
~~~~~~~~~~~~~~~~~~~~{\Xbq}(call-method ,(first primary) ())))) \\
~~~~~~(if around \\*
~~~~~~~~~~{\Xbq}(call-method ,(first around) \\*
~~~~~~~~~~~~~~~~~~~~~~~~(,{\Xatsign}(rest around) \\*
~~~~~~~~~~~~~~~~~~~~~~~~~(make-method ,form))) \\*
~~~~~~~~~~form))))
\end{lisp}
\vskip 0pt plus 10pt
\hrule width 0pt\relax
\begin{lisp}
;;; A simple way to try several methods until one returns non-nil \\*
\\*
(define-method-combination or () \\*
~~~~~~~~((methods (or))) \\*
~~{\Xbq}(or ,{\Xatsign}(mapcar \#'(lambda (method) \\*
~~~~~~~~~~~~~~~~~~~~~{\Xbq}(call-method ,method ())) \\*
~~~~~~~~~~~~~~~~~methods))) \\
\\
;;; A more complete version of the preceding \\*
\\*
(define-method-combination or  \\*
~~~~~~~~(\&optional (order ':most-specific-first)) \\*
~~~~~~~~((around (:around)) \\*
~~~~~~~~~(primary (or))) \\
~~;; Process the order argument \\*
~~(case order \\*
~~~~(:most-specific-first) \\*
~~~~(:most-specific-last (setq primary (reverse primary))) \\*
~~~~(otherwise (method-combination-error \\*
~~~~~~~~~~~~~~~~~"{\Xtilde}S is an invalid order.{\Xtilde}{\Xatsign} \\*
~~~~~~~~~~~~~~~~~~:most-specific-first and :most-specific-last {\Xtilde} \\*
~~~~~~~~~~~~~~~~~~~~are the possible values." \\*
~~~~~~~~~~~~~~~~~~~~~~~~~~~~~~~~~~~~~~~~~order))) \\
~~;; Must have a primary method \\*
~~(unless primary \\*
~~~~(method-combination-error "A primary method is required.")) \\
~~;; Construct the form that calls the primary methods \\*
~~(let ((form (if (rest primary) \\*
~~~~~~~~~~~~~~~~~~{\Xbq}(or ,{\Xatsign}(mapcar \#'(lambda (method) \\*
~~~~~~~~~~~~~~~~~~~~~~~~~~~~~~~~~~~~~{\Xbq}(call-method ,method ())) \\*
~~~~~~~~~~~~~~~~~~~~~~~~~~~~~~~~~primary)) \\*
~~~~~~~~~~~~~~~~~~{\Xbq}(call-method ,(first primary) ())))) \\
~~~~;; Wrap the around methods around that form \\*
~~~~(if around \\*
~~~~~~~~{\Xbq}(call-method ,(first around) \\*
~~~~~~~~~~~~~~~~~~~~~~(,{\Xatsign}(rest around) \\*
~~~~~~~~~~~~~~~~~~~~~~~(make-method ,form))) \\*
~~~~~~~~form)))
\end{lisp}
\vskip 0pt plus 10pt
\hrule width 0pt\relax
\begin{lisp}
;;; The same thing, using the :order and :required keyword options \\*
(define-method-combination or  \\*
~~~~~~~~(\&optional (order ':most-specific-first)) \\*
~~~~~~~~((around (:around)) \\*
~~~~~~~~~(primary (or) :order order :required t)) \\
~~(let ((form (if (rest primary) \\*
~~~~~~~~~~~~~~~~~~{\Xbq}(or ,{\Xatsign}(mapcar \#'(lambda (method) \\*
~~~~~~~~~~~~~~~~~~~~~~~~~~~~~~~~~~~~~{\Xbq}(call-method ,method ())) \\*
~~~~~~~~~~~~~~~~~~~~~~~~~~~~~~~~~primary)) \\*
~~~~~~~~~~~~~~~~~~{\Xbq}(call-method ,(first primary) ())))) \\
~~~~(if around \\*
~~~~~~~~{\Xbq}(call-method ,(first around) \\*
~~~~~~~~~~~~~~~~~~~~~~(,{\Xatsign}(rest around) \\*
~~~~~~~~~~~~~~~~~~~~~~~(make-method ,form))) \\*
~~~~~~~~form))) \\
\\
;;; This short-form call is behaviorally identical to the preceding. \\*
(define-method-combination or :identity-with-one-argument t) \\
 \\
;;; Order methods by positive integer qualifiers; note that :around \\*
;;; methods are disallowed here in order to keep the example small. \\*
\\*
(define-method-combination example-method-combination () \\*
~~~~~~~~((methods positive-integer-qualifier-p)) \\
~~{\Xbq}(progn ,{\Xatsign}(mapcar \#'(lambda (method) \\*
~~~~~~~~~~~~~~~~~~~~~~~~{\Xbq}(call-method ,method ())) \\*
~~~~~~~~~~~~~~~~~~~~(stable-sort methods \#'< \\*
~~~~~~~~~~~~~~~~~~~~~~:key \#'(lambda (method) \\*
~~~~~~~~~~~~~~~~~~~~~~~~~~~~~~~(first (method-qualifiers \\*
~~~~~~~~~~~~~~~~~~~~~~~~~~~~~~~~~~~~~~~~method))))))) \\
\\
(defun positive-integer-qualifier-p (method-qualifiers) \\*
~~(and (= (length method-qualifiers) 1) \\*
~~~~~~~(typep (first method-qualifiers) '(integer 0 *)))) \\
\\
;;; Example of the use of :arguments \\*
(define-method-combination progn-with-lock () \\*
~~~~~~~~((methods ())) \\*
~~~~~~~~(:arguments object) \\
~~{\Xbq}(unwind-protect \\*
~~~~~~~(progn (lock (object-lock ,object)) \\*
~~~~~~~~~~~~~~,{\Xatsign}(mapcar \#'(lambda (method) \\*
~~~~~~~~~~~~~~~~~~~~~~~~~~~~{\Xbq}(call-method ,method ())) \\*
~~~~~~~~~~~~~~~~~~~~~~~~methods)) \\*
~~~~~(unlock (object-lock ,object))))
\end{lisp}



The \cd{:method-combination} option of \cdf{defgeneric} is used to
specify that a generic function should use a particular method
combination type.  The argument to the \cd{:method-combination}
option is the name of a method combination type.
 
See sections~\ref{Method-Selection-and-Combination-SECTION} and
\ref{Built-in-Method-Combination-Types-SECTION} as well as
\cdf{call-method},
\cdf{method-qualifiers},
\cdf{method-combination-error},
\cdf{invalid-method-error},
and \cdf{defgeneric}.
\end{defmac}



\begin{defmac}
defmethod function-name {method-qualifier}*
          specialized-lambda-list
          <{declaration}* | doc-string> {\,form}*

\begin{tabbing}
{\it function-name\/} ::= \Mgroup{{\it symbol\/} {\Mor} \cd{(setf {\it symbol\/})}} \\*
{\it method-qualifier\/} ::= {\it non-nil-atom} \\*
{\it parameter-specializer-name\/} ::= {\it symbol} {\Mor} \cd{(eql {\it eql-specializer-form\/})}
\end{tabbing}
The macro \cdf{defmethod} defines a method on a generic function.  

If \cd{(fboundp {\it function-name\/})} is \cdf{nil}, a generic
function is created with default values for the argument precedence
order (each argument is more specific than the arguments to its right
in the argument list), for the generic function class (the class 
\cdf{standard-generic-function}), for the method class (the class 
\cdf{standard-method}), and for the method combination type (the standard
method combination type).  The lambda-list of the generic function is
congruent with the lambda-list of the method being defined; if the
\cdf{defmethod} form mentions keyword arguments, the lambda-list of
the generic function will mention \cd{\&key} (but no keyword
arguments).  If {\it function-name\/} names a non-generic
function, a macro, or a special form, an error is signaled.

If a generic function is currently named by {\it
function-name\/}, where {\it function-name\/} is a symbol or
a list of the form \cd{(setf {\it symbol\/})}, the lambda-list of the
method must be congruent with the lambda-list of the generic function.
If this condition does not hold, an error is signaled.  See
section~\ref{Congruent-Lambda-Lists-for-All-Methods-of-a-Generic-Function-SECTION}
for a definition of congruence in this context.




The {\it function-name\/} argument is a non-\cdf{nil} symbol or a
list of the form \cd{(setf {\it symbol\/})}.  It names the generic
function on which the method is defined.

Each {\it method-qualifier\/} argument is an object that is used by
method combination to identify the given method.  A method qualifier
is a non-\cdf{nil} atom.  The method combination type may further
restrict what a method qualifier may be.  The standard method
combination type allows for unqualified methods or methods whose sole
qualifier is the keyword \cd{:before}, the keyword 
\cd{:after}, or the keyword \cd{:around}.

A {\it specialized-lambda-list\/} is like an ordinary
function lambda-list except that the name of a required parameter can
be replaced by a specialized parameter, a
list of the form \cd{({\it variable-name
parameter-specializer-name\/})}.  Only required parameters may be
specialized.  A parameter specializer name is a symbol that names a
class or \cd{(eql {\it eql-specializer-form\/})}.  The parameter
specializer name \cd{(eql {\it eql-specializer-form\/})} indicates
that the corresponding argument must be \cdf{eql} to the object that
is the value of {\it eql-specializer-form\/} for the method to be
applicable.  If no parameter specializer name is specified for a given
required parameter, the parameter specializer defaults to the class
named \cdf{t}.  See section \ref{Introduction-to-Methods-SECTION}.

The {\it form\/} arguments specify the method body.
The body of the method is enclosed in an implicit block.  If
{\it function-name\/} is a symbol, this block bears the same name as the
generic function.  If {\it function-name\/} is a list of the form 
\cd{(setf {\it symbol\/})}, the name of the block is {\it symbol}.  


The result of \cdf{defmethod} is the method object.


The class of the method object that is created is that given by the 
method class option of the generic function on which the method is defined.

If the generic function already has a method that agrees with the
method being defined on parameter specializers and qualifiers, 
\cdf{defmethod} replaces the existing method with the one now being
defined.  See
section~\ref{Agreement-on-Parameter-Specializers-and-Qualifiers-SECTION}
for a definition of agreement in this context.

The parameter specializers are derived from the parameter specializer
names as described in section~\ref{Introduction-to-Methods-SECTION}.

The expansion of the \cdf{defmethod} macro refers to each
specialized parameter (see the \cdf{ignore} declaration specifier), including
parameters that
have an explicit parameter specializer name of \cdf{t}.  This means
that a compiler warning does not occur if the body of the method does
not refer to a specialized parameter.  Note that a parameter that
specializes on \cdf{t} is not synonymous with an unspecialized
parameter in this context.

See sections~\ref{Introduction-to-Methods-SECTION},
\ref{Congruent-Lambda-Lists-for-All-Methods-of-a-Generic-Function-SECTION},
and \ref{Agreement-on-Parameter-Specializers-and-Qualifiers-SECTION}.
\end{defmac}


[At this point the original CLOS report \cite{SIGPLAN-CLOS,LASC-CLOS-PART-2}
contained a specification for \cdf{describe} as a generic function.
This specification is omitted here because X3J13 voted in March 1989 \issue{DESCRIBE-UNDERSPECIFIED}
not to make \cdf{describe} a generic function after all (see \cdf{describe-object}).---GLS]


\begin{defun}[Generic function][Primary method]
documentation x &optional doc-type \\
documentation ~~~~~~~~~~~~~~~~~~~~~~~~~~~~~~~~~~~
      (method standard-method)  &optional doc-type \\
documentation ~~~~~~~~~~~~~~~~~~~~~~~~~~~~~~~~~~~
      (generic-function standard-generic-function) &optional doc-type \\
documentation (class standard-class)  &optional doc-type \\
documentation ~~~~~~~~~~~~~~~~~~~~~~~~~~~~~~~~~~~
      (method-combination method-combination) &optional doc-type \\
documentation ~~~~~~~~~~~~~~~~~~~~~~~~~~~~~~~~~~~
      (slot-description standard-slot-description) &optional doc-type \\
documentation (symbol symbol) &optional doc-type \\
documentation (list list) &optional doc-type

The ordinary function \cdf{documentation} (see section~\ref{DOCUMENTATION-SECTION})
is replaced by a generic
function.  The generic function \cdf{documentation} returns the
documentation string associated with the given object if it is
available; otherwise \cdf{documentation} returns \cdf{nil}.


The first argument of \cdf{documentation} is a symbol, a
function-name list of the form \cd{(setf {\it symbol\/})}, a
method object, a class object, a generic function object, a method
combination object, or a slot description object.
Whether a second argument should be supplied depends on the
type of the first argument.
\begin{itemize}

\item 
If the first argument is a method object, a class object, a generic
function object, a method combination object, or a slot description
object, the second argument must not be supplied, or an error is
signaled.

\item  
If the first argument is a symbol or a list of the form
\cd{(setf {\it symbol\/})}, the second argument must be
supplied.

\begin{itemize}
\item
The forms
\begin{lisp}
(documentation {\it symbol\/} 'function)
\end{lisp}
and
\begin{lisp}
(documentation '(setf {\it symbol\/}) 'function)
\end{lisp}
return the
documentation string of the function, generic function, special form, or
macro named by the symbol or list.

\item
The form \cd{(documentation {\it symbol\/} 'variable)} returns the
documentation string of the special variable or constant named by the
symbol.

\item
The form \cd{(documentation {\it symbol\/} 'structure)} returns the
documentation string of the \cdf{defstruct} structure named by the
symbol.

\item
The form \cd{(documentation {\it symbol\/} 'type)} returns the documentation
string of the class object named by the symbol, if there is such a
class.   If there is no such class, it returns the documentation string
of the type specifier named by the symbol. 

\item
The form \cd{(documentation {\it symbol\/} 'setf)} returns the documentation
string of the \cdf{defsetf} or \cdf{define-setf-method} definition
associated with the symbol.

\item
The form \cd{(documentation {\it symbol\/} 'method-combination)} returns the
documentation string of the method combination type named by the
symbol.  
\end{itemize}

\end{itemize}

An implementation may extend the set of symbols that are acceptable as
the second argument.  If a symbol is not recognized as an acceptable
argument by the implementation, an error must be signaled.


The documentation string associated with the given object is returned
unless none is available, in which case \cdf{documentation} returns
\cdf{nil}.

\end{defun}


\begin{defun}[Generic function][Primary method]
(setf~documentation) new-value x &optional doc-type \\
(setf~documentation) new-value ~~~~~~~~~~~~~~~~~~
   (method standard-method) &optional doc-type \\
(setf~documentation) new-value ~~~~~~~~~~~~~~~~~~
   (generic-function standard-generic-function) &optional doc-type \\
(setf~documentation) new-value ~~~~~~~~~~~~~~~~~~
   (class standard-class) &optional doc-type \\
(setf~documentation) new-value ~~~~~~~~~~~~~~~~~~
   (method-combination method-combination) &optional doc-type \\
(setf~documentation) new-value ~~~~~~~~~~~~~~~~~~
   (slot-description standard-slot-description) &optional doc-type \\
(setf~documentation) new-value ~~~~~~~~~~~~~~~~~~
   (symbol symbol) &optional doc-type \\
(setf~documentation) new-value ~~~~~~~~~~~~~~~~~~
   (list list) &optional doc-type

The generic function \cd{(setf documentation)} is used to update the
documentation.

The first argument of \cd{(setf documentation)} is the new documentation.

The second argument of \cdf{documentation} is a symbol, a
function-name list of the form \cd{(setf {\it symbol\/})}, a
method object, a class object, a generic function object, a method
combination object, or a slot description object.
Whether a third argument should be supplied depends on the
type of the second argument.
See \cdf{documentation}.
\end{defun}


\begin{defun}[Function]
ensure-generic-function function-name &key :lambda-list
:argument-precedence-order
:declare
:documentation
:generic-function-class
:method-combination
:method-class
:environment

\begin{tabbing}
{\it function-name\/} ::= \Mgroup{{\it symbol\/} {\Mor} \cd{(setf {\it symbol\/})}}
\end{tabbing}
The function \cdf{ensure-generic-function} is used to define a
globally named generic function with no methods or to specify or
modify options and declarations that pertain to a globally named
generic function as a whole.

If \cd{(fboundp {\it function-name\/})} is \cdf{nil}, a new
generic function is created.  If \cd{(fdefinition {\it
function-name\/})} is a non-generic function, a macro, or a
special form, an error is signaled.

[X3J13 voted in March 1989 \issue{FUNCTION-NAME} to use \cdf{fdefinition}
in the previous paragraph, as shown, rather than \cdf{symbol-function},
as it appeared in the original report on CLOS~\cite{SIGPLAN-CLOS,LASC-CLOS-PART-2}.
The vote also changed all occurrences of {\it function-specifier} in the
original report to {\it function-name}; this change is reflected here.---GLS]

If {\it function-name\/} specifies a generic function that has a
different value for any of the following arguments, the generic
function is modified to have the new value: 
\cd{:argument-precedence-order}, \cd{:declare}, \cd{:documentation},
\cd{:method-combination}.

If {\it function-name\/} specifies a generic function that has a
different value for the \cd{:lambda-list} argument, and the new value
is congruent with the lambda-lists of all existing methods or there
are no methods, the value is changed; otherwise an error is signaled.

If {\it function-name\/} specifies a generic function that has a
different value for the \cd{:generic-function-class} argument and if
the new generic function class is compatible with the old, 
\cdf{change-class} is called to change the class of the generic function;
otherwise an error is signaled.

If {\it function-name\/} specifies a generic function that has a
different \cd{:method-class} value, the value is
changed but any existing methods are not changed.




The {\it function-name\/} argument is a symbol or a list of the
form \cd{(setf {\it symbol\/})}.

The keyword arguments correspond to the {\it option\/} arguments of
\cdf{defgeneric}, except that the \cd{:method-class} and
\cd{:generic-function-class} arguments can be class objects
as well as names.


The \cd{:environment\/} argument is the same as the 
\cd{\&environment} argument to macro expansion functions.  It is typically
used to distinguish between compile-time and run-time environments.

The \cd{:method-combination} argument is a method combination object.


The generic function object is returned.
See \cdf{defgeneric}.
\end{defun}


\begin{defun}[Function]
find-class symbol &optional errorp environment

The function \cdf{find-class} returns the class object named by the
given symbol in the given environment.




The first argument to \cdf{find-class} is a symbol. 

If there is no such class and the {\it errorp\/} argument is
not supplied or is non-\cdf{nil}, \cdf{find-class} signals an error.
If there is no such class and the {\it errorp\/} argument is
\cdf{nil}, \cdf{find-class} returns \cdf{nil}.  The default value of
{\it errorp\/} is \cdf{t}.

The optional {\it environment\/} argument is the same as the 
\cd{\&environment} argument to macro expansion functions.  It is typically
used to distinguish between compile-time and run-time environments.


The result of \cdf{find-class} is the class object named by the given symbol.


The class associated with a particular symbol can be changed by using
\cdf{setf} with \cdf{find-class}.  The results are undefined if
the user attempts to change the class associated with a symbol that is
defined as a type specifier in chapter~\ref{DTSPEC}.
See section~\ref{Integrating-Types-and-Classes-SECTION}.

\end{defun}


\begin{defun}[Generic function][Primary method]
find-method generic-function method-qualifiers specializers &optional errorp \\
find-method~~~~~~~~~~~~~~~~~~~~~~~~~~~~~~~~~ (generic-function standard-generic-function)
   method-qualifiers specializers &optional errorp

The generic function \cdf{find-method} takes a generic function and
returns the method object that agrees on method qualifiers and
parameter specializers with the {\it method-qualifiers\/} and {\it
specializers\/} arguments of \cdf{find-method}.
See section~\ref{Agreement-on-Parameter-Specializers-and-Qualifiers-SECTION} for a
definition of agreement in this context.





The {\it generic-function\/} argument is a generic function.

The {\it method-qualifiers\/} argument is a list of the
method qualifiers for the method.   The order of the method qualifiers
is significant.  

The {\it specializers\/} argument is a list of the parameter
specializers for the method.  It must correspond in length to
the number of required arguments of the generic function, or
an error is signaled.  This means that to obtain the
default method on a given generic function, a list whose
elements are the class named \cdf{t} must be given.

If there is no such method and the {\it errorp\/} argument is
not supplied or is non-\cdf{nil}, \cdf{find-method} signals an error.
If there is no such method and the {\it errorp\/} argument is
\cdf{nil}, \cdf{find-method} returns \cdf{nil}.  The default value of
{\it errorp\/} is \cdf{t}.


The result of \cdf{find-method} is the method object with the given
method qualifiers and parameter specializers.

See section~\ref{Agreement-on-Parameter-Specializers-and-Qualifiers-SECTION}.

\end{defun}


\begin{defun}[Generic function][Primary method]
function-keywords method \\
function-keywords (method standard-method)

The generic function \cdf{function-keywords} is used to return the keyword
parameter specifiers for a given method.





The {\it method\/} argument is a method object.


The generic function \cdf{function-keywords} returns two values:
a list of the explicitly named keywords and a boolean that states whether
\cd{\&allow-other-keys} had been specified in the method definition.

\end{defun}


\begin{defspec}
generic-flet ({(function-name lambda-list
                <?option | {method-description}*>)}*)
            {\,form}*

The \cdf{generic-flet} special form is analogous to the
\cdf{flet} special form.  It produces new generic functions and
establishes new lexical function definition bindings.  Each generic
function is created with the set of methods specified by its method
descriptions.

The special form \cdf{generic-flet} is used to define generic functions whose
names are meaningful only locally and to execute a series of forms
with these function definition bindings.  Any number of such local
generic functions may be defined.

The names of functions defined by \cdf{generic-flet} have lexical
scope; they retain their local definitions only within the body of the
\cdf{generic-flet}.  Any references within the body of the 
\cdf{generic-flet} to functions whose names are the same as those defined
within the \cdf{generic-flet} are thus references to the local
functions instead of to any global functions of the same names.  The
scope of these generic function definition bindings, however, includes only
the body of \cdf{generic-flet}, not the definitions themselves.
Within the method bodies, local function names that match those
being defined refer to global functions defined outside the 
\cdf{generic-flet}.  It is thus not possible to define recursive functions
with \cdf{generic-flet}.




The {\it function-name\/}, {\it lambda-list\/}, {\it option}, {\it
method-qualifier}, and {\it specialized-lambda-list\/} arguments are
the same as for \cdf{defgeneric}.

A \cdf{generic-flet} local method definition is identical in form to the
method definition part of a \cdf{defmethod}.

The body of each method is enclosed in an implicit block.  If {\it
function-name\/} is a symbol, this block bears the same name as
the generic function.  If {\it function-name\/} is a list of the
form \cd{(setf {\it symbol\/})}, the name of the block is {\it
symbol}.  


The result returned by \cdf{generic-flet} is the value or values
returned by the last form executed.  If no forms are specified, 
\cdf{generic-flet} returns \cdf{nil}.

See \cdf{generic-labels}, \cdf{defmethod}, \cdf{defgeneric}, and \cdf{generic-function}.
\end{defspec}


\begin{defmac}
generic-function lambda-list <?option | {method-description}*>

\begin{tabbing}
\pushtabs{\it option\/} ::= \=\cd{(:argument-precedence-order \Mplus{parameter-name})} \\
\>\hbox to 0pt{\hss\Mor~}\cd{(declare \Mplus{declaration})} \\
\>\hbox to 0pt{\hss\Mor~}\cd{(:documentation {\it string\/})} \\
\>\hbox to 0pt{\hss\Mor~}\cd{(:method-combination {\it symbol\/} \Mstar{{arg\/}})} \\
\>\hbox to 0pt{\hss\Mor~}\cd{(:generic-function-class {\it class-name\/})} \\
\>\hbox to 0pt{\hss\Mor~}\cd{(:method-class {\it class-name\/})} \poptabs \\
\pushtabs{\it method-description\/} ::= \cd{(:method }\=\Mstar{{method-qualifier\/}} \\
\>{\it specialized-lambda-list\/} \\
\>\Mstar{{declaration\/ {\Mor} documentation\/}} \\
\>\Mstar{{\,form\/}}\cd{)} \poptabs
\end{tabbing}
The \cdf{generic-function} macro creates an anonymous generic
function. The generic function is created with the set of methods
specified by its method descriptions.




The {\it option}, {\it method-qualifier}, and {\it
specialized-lambda-list\/} arguments are the same as for 
\cdf{defgeneric}.


The generic function object is returned as the result.


If no method descriptions are specified, an anonymous generic function with no
methods is created.

See \cdf{defgeneric}, \cdf{generic-flet}, \cdf{generic-labels}, and \cdf{defmethod}.
\end{defmac}


\begin{defspec}
generic-labels ({(function-name lambda-list
                <?option | {method-description}*>)}*)
               {\,form}*

The \cdf{generic-labels} special form is analogous to the
\cdf{labels} special form.  It produces new generic functions and
establishes new lexical function definition bindings.  Each generic
function is created with the set of methods specified by its method
descriptions.

The special form \cdf{generic-labels} is used to define generic functions
whose names are meaningful only locally and to execute a series of
forms with these function definition bindings.  Any number of
such  local generic functions may be defined.  

The names of functions defined by \cdf{generic-labels} have lexical
scope; they retain their local definitions only within the body of the
\cdf{generic-labels} construct.  Any references within the body of the
\cdf{generic-labels} construct to functions whose names are the same
as those defined within the \cdf{generic-labels} form are thus
references to the local functions instead of to any global functions
of the same names.  The scope of these generic function definition bindings
includes the method bodies themselves as well as the body of the 
\cdf{generic-labels} construct.




The {\it function-name\/}, {\it lambda-list\/}, {\it option}, {\it
method-qualifier}, and {\it specialized-lambda-list\/} arguments are
the same as for \cdf{defgeneric}.

A \cdf{generic-labels} local method definition is identical in form to the
method definition part of a \cdf{defmethod}.

The body of each method is enclosed in an implicit block.  If {\it
function-name\/} is a symbol, this block bears the same name as
the generic function.  If {\it function-name\/} is a list of the
form \cd{(setf {\it symbol\/})}, the name of the block is {\it
symbol}.  


The result returned by \cdf{generic-labels} is the value or values
returned by the last form executed.  If no forms are specified, 
\cdf{generic-labels} returns \cdf{nil}.

See \cdf{generic-flet}, \cdf{defmethod}, \cdf{defgeneric}, \cdf{generic-function}.
\end{defspec}


\begin{defun}[Generic function][Primary method]
initialize-instance instance &rest initargs \\
initialize-instance (instance standard-object) &rest initargs

The generic function \cdf{initialize-instance} is called by 
\cdf{make-instance} to initialize a newly created instance.  The generic
function \cdf{initialize-instance} is called with the new instance and
the defaulted initialization arguments.

The system-supplied primary method on \cdf{initialize-instance}
initializes the slots of the instance with values according to the
initialization arguments and the \cd{:initform} forms of the slots.
It does this by calling the generic function \cdf{shared-initialize}
with the following arguments: the instance, \cdf{t} (this indicates
that all slots for which no initialization arguments are provided
should be initialized according to their \cd{:initform} forms) and
the defaulted initialization arguments.





The {\it instance\/} argument is the object to be initialized.

The {\it initargs\/} argument consists of alternating initialization
argument names and values.


The modified instance is returned as the result.


Programmers can define methods for \cdf{initialize-instance} to
specify actions to be taken when an instance is initialized.  If only
\cd{:after} methods are defined, they will be run after the
system-supplied primary method for initialization and therefore will
not interfere with the default behavior of \cdf{initialize-instance}.

See sections~\ref{Object-Creation-and-Initialization-SECTION},
\ref{Rules-for-Initialization-Arguments-SECTION}, and
\ref{Declaring-the-Validity-of-Initialization-Arguments-SECTION} as well as
\cdf{shared-initialize},
\cdf{make-instance},
\cdf{slot-boundp},
and \cdf{slot-makunbound}.
\end{defun}


\begin{defun}[Function]
invalid-method-error method format-string &rest args

The function \cdf{invalid-method-error} is used to signal an error
when there is an applicable method whose qualifiers are not valid for
the method combination type.  The error message is constructed by
using a \cdf{format} string and any arguments to it.  Because an
implementation may need to add additional contextual information to
the error message, \cdf{invalid-method-error} should be called only
within the dynamic extent of a method combination function.

The function \cdf{invalid-method-error} is called automatically when a
method fails to satisfy every qualifier pattern and predicate in a
\cd{define-\discretionary{}{}{}method-\discretionary{}{}{}combination} form.
A method combination function
that imposes additional restrictions should call 
\cdf{invalid-method-error} explicitly if it encounters a method it cannot
accept.





The {\it method\/} argument is the invalid method object.  

The {\it format-string\/} argument is a control string that can be
given to \cdf{format}, and {\it args\/} are any arguments required by
that string.


Whether \cdf{invalid-method-error} returns to its caller or exits via
\cdf{throw} is implementation-dependent.

See \cdf{define-method-combination}.

\end{defun}


\begin{defun}[Generic function][Primary method]
make-instance class &rest initargs \\
make-instance (class standard-class) &rest initargs \\
make-instance (class symbol) &rest initargs

The generic function \cdf{make-instance} creates a new
instance of the given class.

The generic function \cdf{make-instance} may be used as described in
section~\ref{Object-Creation-and-Initialization-SECTION}.





The {\it class\/} argument is a class object or a symbol that
names a class.  The remaining arguments form a list of alternating
initialization argument names and values.

If the second of the preceding methods is selected, that method invokes
\cdf{make-instance} on the arguments \cd{(find-class {\it class\/})} and
{\it initargs}.

The initialization arguments are checked within \cdf{make-instance}
(see section~\ref{Object-Creation-and-Initialization-SECTION}).


The new instance is returned.


The meta-object protocol can be used to define new methods on 
\cdf{make-instance} to replace the object-creation protocol.

See section \ref{Object-Creation-and-Initialization-SECTION} as well as
\cdf{defclass}, \cdf{initialize-instance}, and \cdf{class-of}.
\end{defun}



\begin{defun}[Generic function][Primary method]
make-instances-obsolete class \\
make-instances-obsolete (class standard-class) \\
make-instances-obsolete (class symbol)

The generic function \cdf{make-instances-obsolete} is invoked
automatically by the system when \cdf{defclass} has been used to
redefine an existing standard class and the set of local slots accessible in an
instance is changed or the order of slots in storage is changed.  It
can also be explicitly invoked by the user.

The function \cdf{make-instances-obsolete} has the effect of
initiating the process of updating the instances of the
class. During updating, the generic function 
\cd{update-\discretionary{}{}{}instance-\discretionary{}{}{}for-\discretionary{}{}{}redefined-\discretionary{}{}{}class} will be invoked.





The {\it class\/} argument is a class object symbol that names
the class whose instances are to be made obsolete.

If the second of the preceding methods is selected, that method invokes
\cdf{make-instances-obsolete} on \cd{(find-class {\it class\/})}.


The modified class is returned.  The result of \cdf{make-instances-obsolete}
is \cdf{eq} to the {\it class} argument supplied to the first of the preceding
methods.

See section~\ref{Redefining-Classes-SECTION} as well as
\cd{update-\discretionary{}{}{}instance-\discretionary{}{}{}for-\discretionary{}{}{}redefined-\discretionary{}{}{}class}.
\end{defun}



\begin{defun}[Function]
method-combination-error format-string &rest args

The function \cdf{method-combination-error} is used to signal an error
in method combination.  The error message is constructed by using a
\cdf{format} string and any arguments to it.  Because an implementation may
need to add additional contextual information to the error message,
\cdf{method-combination-error} should be called only within the
dynamic extent of a method combination function.



 

The {\it format-string\/} argument is a control string that can be
given to \cdf{format}, and {\it args\/} are any arguments required by
that string.


Whether \cdf{method-combination-error} returns to its caller or exits
via \cdf{throw} is implementation-dependent.

See \cdf{define-method-combination}.

\end{defun}



\begin{defun}[Generic function][Primary method]
method-qualifiers method \\
method-qualifiers (method standard-method)

The generic function \cdf{method-qualifiers} returns a list of the
qualifiers of the given method.





The {\it method\/} argument is a method object. 


A list of the qualifiers of the given method is returned.

Example:
\begin{lisp}
(setq methods (remove-duplicates methods \\*
~~~~~~~~~~~~~~~~~~~~~~~~~~~~~~~~~:from-end t \\*
~~~~~~~~~~~~~~~~~~~~~~~~~~~~~~~~~:key \#'method-qualifiers \\*
~~~~~~~~~~~~~~~~~~~~~~~~~~~~~~~~~:test \#'equal))
\end{lisp}

See \cdf{define-method-combination}.
\end{defun}

 

\begin{defun}[Function]
next-method-p

The locally defined function \cdf{next-method-p} can be used within
the body of a method defined by a method-defining form to determine
whether a next method exists.





The function \cdf{next-method-p} takes no arguments.


The function \cdf{next-method-p} returns true or false.


Like \cdf{call-next-method}, the function \cdf{next-method-p} has 
lexical scope (for it
is defined only within the body of a method defined by a method-defining form)
and indefinite extent.

See \cdf{call-next-method}.

\end{defun}


\begin{defun}[Generic function][Primary method]
no-applicable-method generic-function &rest function-arguments \\
no-applicable-method (generic-function t) &rest function-arguments

The generic function \cdf{no-applicable-method} is called when a
generic function of the class \cdf{standard-generic-function} is invoked
and no method on that generic function is applicable.
The default method signals an error.

The generic function \cdf{no-applicable-method} is not intended to be called
by programmers.  Programmers may write methods for it.





The {\it generic-function\/} argument of \cdf{no-applicable-method} is the
generic function object on which no applicable method was found.  

The {\it function-arguments} argument is a list of the arguments to that
generic function.

\end{defun}

\begin{defun}[Generic function][Primary method]
no-next-method generic-function method &rest args \\
no-next-method~~~~~~~~~~~~~~~~~~~~~~~~~~~ (generic-function standard-generic-function)
    (method standard-method) &rest args

The generic function \cdf{no-next-method} is called by 
\cdf{call-next-method} when there is no next method.  The system-supplied
method on \cdf{no-next-method} signals an error.

The generic function \cdf{no-next-method} is not intended to be called
by programmers.  Programmers may write methods for it.





The {\it generic-function\/} argument is the generic function object
to which the method that is the second argument belongs.

The {\it method\/} argument is the method that contains the call to
\cdf{call-next-method} for which there is no next method.

The {\it args\/} argument is a list of the arguments to
\cdf{call-next-method}.

See \cdf{call-next-method}.

\end{defun}


\begin{defun}[Generic function][Primary method]
print-object object stream \\
print-object (object standard-object) stream

The generic function \cdf{print-object} writes the printed
representation of an object to a stream.  The function 
\cdf{print-object} is called by the print system; it should not be called
by the user.

\penalty-10000 %required

Each implementation must provide a method on the class 
\cdf{standard-object} and methods on enough other classes so as to ensure
that there is always an applicable method.  Implementations are free
to add methods for other classes.  Users can write methods for 
\cdf{print-object} for their own classes if they do not wish to inherit an
implementation-supplied method.

The first argument is any Lisp object.   The second argument is a
stream; it cannot be \cdf{t} or \cdf{nil}. 

The function \cdf{print-object} returns its first argument, the object.  

Methods on \cdf{print-object} must obey the print control special
variables named \cd{*print-{\it xxx}*} for various {\it xxx}.  The
specific details are the following:

\begin{itemize}

\item 
Each method must implement \cd{*print-escape*}. 

\item  
The \cd{*print-pretty*} control variable can be ignored
by most methods other than the one for lists.

\item 
The \cd{*print-circle*} control variable is handled by the printer
and can be ignored by methods.

\item 
The printer takes care of \cd{*print-level*} automatically, provided that
each method handles exactly one level of structure and
calls \cdf{write} (or an equivalent function) recursively if
there are more structural levels.  The printer's decision
of whether an object has components (and therefore should
not be printed when the printing depth is not less than
\cd{*print-level*}) is implementation-dependent.  In some
implementations its \cdf{print-object} method is not called; in
others the method is called, and the determination that the
object has components is based on what it tries to write
to the stream.

\item 
Methods that produce output of indefinite length must obey
\cd{*print-length*}, but most methods other than the one for lists can
ignore it.

\item 
The \cd{*print-base*}, \cd{*print-radix*}, \cd{*print-case*}, 
\cd{*print-gensym*}, and \cd{*print-array*} control variables apply
to specific types of objects and are handled by the methods for those
objects.

\item X3J13 voted in June 1989 \issue{DATA-IO} to add the following point.
   All methods for \cdf{print-object} must obey \cd{*print-readably*},
   which takes precedence over all other printer control variables.  This
   includes both user-defined methods and implementation-defined methods.
\end{itemize}

If these rules are not obeyed, the results are undefined.

In general, the printer and the \cdf{print-object} methods should not
rebind the print control variables as they operate recursively through the
structure, but this is implementation-dependent.

In some implementations the stream argument passed to a 
\cdf{print-object} method is not the original stream but is an
intermediate stream that implements part of the printer.  Methods
should therefore not depend on the identity of this stream.

All of the existing printing functions (\cdf{write}, \cd{prin1}, 
\cdf{print}, \cdf{princ}, \cdf{pprint}, \cdf{write-to-string}, 
\cd{prin1-to-string}, \cdf{princ-to-string}, the \cd{{\Xtilde}S} and 
\cd{{\Xtilde}A} \cdf{format} operations, and the \cd{{\Xtilde}B}, \cd{{\Xtilde}D},
\cd{{\Xtilde}E}, \cd{{\Xtilde}F}, \cd{{\Xtilde}G}, \cd{{\Xtilde}\$}, 
\cd{{\Xtilde}O}, \cd{{\Xtilde}R}, and \cd{{\Xtilde}X} \cdf{format} operations when they
encounter a non-numeric value) are required to be changed to go
through the \cdf{print-object} generic function.  Each implementation is
required to replace its former implementation of printing with one or
more \cdf{print-object} methods.  Exactly which classes have methods for
\cdf{print-object} is not specified; it would be valid for an implementation
to have one default method that is inherited by all system-defined
classes.

\end{defun}


\begin{defun}[Generic function][Primary method]
reinitialize-instance instance &rest initargs \\
reinitialize-instance (instance standard-object) &rest initargs

The generic function \cdf{reinitialize-instance} can be used to change
the values of local slots according to initialization arguments.  This
generic function is called by the Meta-Object Protocol.   It can also be
called by users.

The system-supplied primary method for \cdf{reinitialize-instance}
checks the validity of initialization arguments and signals an error if
an initialization argument is supplied that is not declared valid.
The method then calls the generic function \cdf{shared-initialize}
with the following arguments:  the instance, \cdf{nil} (which means no slots
should be initialized according to their \cd{:initform} forms) and the
initialization arguments it received.





The {\it instance\/} argument is the object to be initialized.

The {\it initargs\/} argument consists of alternating initialization
argument names and values.


The modified instance is returned as the result.


Initialization arguments are declared valid by using the 
\cd{:initarg} option to \cdf{defclass}, or by defining methods for 
\cdf{reinitialize-instance} or \cd{shared-\discretionary{}{}{}initialize}.  The keyword name
of each keyword parameter specifier in the lambda-list of any method
defined on \cdf{reinitialize-instance} or \cd{shared-\discretionary{}{}{}initialize} is
declared a valid initialization argument name for all classes for
which that method is applicable.

See sections~\ref{Reinitializing-an-Instance-SECTION},
\ref{Rules-for-Initialization-Arguments-SECTION},
\ref{Declaring-the-Validity-of-Initialization-Arguments-SECTION} as well as
\cdf{initialize-instance}, \cdf{slot-boundp},
\cd{update-\discretionary{}{}{}instance-\discretionary{}{}{}for-\discretionary{}{}{}redefined-\discretionary{}{}{}class},
\cd{update-\discretionary{}{}{}instance-\discretionary{}{}{}for-\discretionary{}{}{}different-\discretionary{}{}{}class},
\cdf{slot-makunbound}, and \cdf{shared-initialize}.

\end{defun}


\begin{defun}[Generic function][Primary method]
remove-method generic-function method \\
remove-method~~~~~~~~~~~~~~~~~~~~~~~~~~~~~~ (generic-function standard-generic-function) method

The generic function \cdf{remove-method} removes a method from a
generic function.  It destructively modifies the specified generic
function and returns the modified generic function as its result.





The {\it generic-function\/} argument is a generic function
object.

The {\it method\/} argument is a method object.  The function 
\cdf{remove-method} does not signal an error if the method is not one of the
methods on the generic function.


The modified generic function is returned.  The result of \cdf{remove-method} 
is \cdf{eq} to the {\it generic-function\/} argument.

See \cdf{find-method}.
\end{defun}

\begin{defun}[Generic function][Primary method]
shared-initialize instance slot-names &rest initargs \\
shared-initialize (instance standard-object) slot-names &rest initargs

The generic function \cdf{shared-initialize} is used to fill the slots
of an instance using initialization arguments and \cd{:initform}
forms.  It is called when an instance is created, when an instance is
re-initialized, when an instance is updated to conform to a redefined
class, and when an instance is updated to conform to a different
class.  The generic function \cd{shared-\discretionary{}{}{}initialize} is called by the
system-supplied primary method for \cd{initialize-\discretionary{}{}{}instance}, 
\cd{reinitialize-\discretionary{}{}{}instance},
\cd{update-\discretionary{}{}{}instance-\discretionary{}{}{}for-\discretionary{}{}{}redefined-\discretionary{}{}{}class}, and
\cd{update-\discretionary{}{}{}instance-\discretionary{}{}{}for-\discretionary{}{}{}different-\discretionary{}{}{}class}.

The generic function \cdf{shared-initialize} takes the following
arguments: the instance to be initialized, a specification of a set of
names of slots accessible in that instance, and any number of initialization
arguments.  The arguments after the first two must form an initialization
argument list.  The system-supplied primary method on 
\cdf{shared-initialize} initializes the slots with values according to the
initialization arguments and specified \cd{:initform} forms.  The
second argument indicates which slots should be initialized according
to their \cd{:initform} forms if no initialization arguments are
provided for those slots. 

The system-supplied primary method behaves as follows, regardless of
whether the slots are local or shared: 

\begin{itemize}

\item  If an initialization argument in the
initialization argument list specifies a value for that slot, that
value is stored into the slot,  even if a value has
already been stored in the slot before the method is run.

\item  Any slots indicated by the second argument that are still
unbound at this point are initialized according to their 
\cd{:initform} forms.  For any such slot that has an \cd{:initform} form,
that form is evaluated in the lexical environment of its defining 
\cdf{defclass} form and the result is stored into the slot.  For example, if
a \cd{:before} method stores a value in the slot, the \cd{:initform}
form will not be used to supply a value for the slot.

\item  The rules mentioned in section~\ref{Rules-for-Initialization-Arguments-SECTION} are obeyed.

\end{itemize}





The {\it instance\/} argument is the object to be initialized.

The {\it slot-names\/} argument specifies the slots that are to be
initialized according to their \cd{:initform} forms if no
initialization arguments apply.  It is supplied in one of three forms
as follows:

\begin{itemize}

\item  It can be a list of slot names, which specifies
the set of those slot names.

\item  It can be \cdf{nil}, which specifies the empty set of
slot names.

\item  It can be the symbol \cdf{t}, which specifies the set of
all of the slots.

\end{itemize}

The {\it initargs\/} argument consists of alternating initialization 
argument names and values.


The modified instance is returned as the result.


Initialization arguments are declared valid by using the 
\cd{:initarg} option to \cdf{defclass}, or by defining methods for 
\cdf{shared-initialize}.  The keyword name of each keyword parameter
specifier in the lambda-list of any method defined on 
\cdf{shared-initialize} is declared a valid initialization argument
name for all classes for which that method is applicable.

Implementations are permitted to optimize \cd{:initform} forms that 
neither produce nor depend on side effects by evaluating these forms
and storing them into slots before running any 
\cdf{initialize-instance} methods, rather than by handling them in the
primary \cdf{initialize-instance} method.  (This optimization might
be implemented by having the \cdf{allocate-instance} method copy a
prototype instance.)

Implementations are permitted to optimize default initial value forms
for initialization arguments associated with slots by not actually
creating the complete initialization argument list when the only method
that would receive the complete list is the method on 
\cdf{standard-object}.  In this case, default initial value forms can be 
treated like \cd{:initform} forms.  This optimization has no visible
effects other than a performance improvement.

See sections~\ref{Object-Creation-and-Initialization-SECTION},
\ref{Rules-for-Initialization-Arguments-SECTION},
\ref{Declaring-the-Validity-of-Initialization-Arguments-SECTION} as well as
\cdf{initialize-instance},
\cdf{reinitialize-instance},
\cd{update-\discretionary{}{}{}instance-\discretionary{}{}{}for-\discretionary{}{}{}redefined-\discretionary{}{}{}class},
\cd{update-\discretionary{}{}{}instance-\discretionary{}{}{}for-\discretionary{}{}{}different-\discretionary{}{}{}class},
\cdf{slot-boundp},
and \cdf{slot-makunbound}.
\end{defun}


\begin{defun}[Function]
slot-boundp instance slot-name

The function \cdf{slot-boundp} tests whether a specific slot in an
instance is bound.





The arguments are the instance and the name of the slot.


The function \cdf{slot-boundp} returns true or false.


This function allows for writing \cd{:after}
methods on \cdf{initialize-instance} in order to initialize only
those slots that have not already been bound.

If no slot of the given name exists in the instance, \cdf{slot-missing}
is called as follows:
\begin{lisp}
(slot-missing (class-of {\it instance\/}) \\
~~~~~~~~~~~~~~{\it instance\/} \\
~~~~~~~~~~~~~~{\it slot-name\/} \\
~~~~~~~~~~~~~~'slot-boundp)
\end{lisp}

The function \cdf{slot-boundp} is implemented using 
\cdf{slot-boundp-using-class}.
See \cdf{slot-missing}.
\end{defun}


\begin{defun}[Function]
slot-exists-p object slot-name

The function \cdf{slot-exists-p} tests whether the specified object has
a slot of the given name.





The {\it object\/} argument is any object.  The {\it slot-name\/} argument
is a symbol.


The function \cdf{slot-exists-p} returns true or false.


The function \cdf{slot-exists-p} is implemented using 
\cd{slot-\discretionary{}{}{}exists-p-\discretionary{}{}{}using-\discretionary{}{}{}class}.

\end{defun}


\begin{defun}[Function]
slot-makunbound instance slot-name

The function \cdf{slot-makunbound} restores a slot in an instance to
the unbound state.





The arguments to \cdf{slot-makunbound} are the instance and the name of
the slot.


The instance is returned as the result.


If no slot of the given name exists in the instance, \cdf{slot-missing}
is called as follows:
\begin{lisp}
(slot-missing (class-of {\it instance\/}) \\
~~~~~~~~~~~~~~{\it instance\/} \\
~~~~~~~~~~~~~~{\it slot-name\/} \\
~~~~~~~~~~~~~~'slot-makunbound)
\end{lisp}

The function \cdf{slot-makunbound} is implemented using 
\cd{slot-\discretionary{}{}{}makunbound-\discretionary{}{}{}using-\discretionary{}{}{}class}.
See \cdf{slot-missing}.
\end{defun}



\begin{defun}[Generic function][Primary method]
slot-missing class object slot-name operation &optional new-value \\
slot-missing (class t) object slot-name operation &optional new-value

The generic function \cdf{slot-missing} is invoked when an attempt is
made to access a slot in an object whose metaclass is 
\cdf{standard-class} and the name of the slot provided is not a name of a
slot in that class.  
The default method signals an error.

The generic function \cdf{slot-missing} is not intended to be called by
programmers.  Programmers may write methods for it.





The required arguments to \cdf{slot-missing} are the class of the object
that is being accessed, the object, the slot name, and a symbol that
indicates the operation that caused  \cdf{slot-missing} to be invoked.
The optional argument to \cdf{slot-missing} is used when the operation
is attempting to set the value of the slot.


If a method written for \cdf{slot-missing} returns values, these
values get returned as the values of the original function invocation.


The generic function \cdf{slot-missing} may be called during
evaluation of \cdf{slot-value}, \cd{(setf slot-value)}, 
\cdf{slot-boundp}, and \cdf{slot-makunbound}.  For each
of these operations the corresponding symbol for the {\it operation\/}
argument is \cdf{slot-value}, \cdf{setf}, \cdf{slot-boundp}, and 
\cdf{slot-makunbound}, respectively.

The set of arguments (including the class of the instance) facilitates
defining methods on the metaclass for \cdf{slot-missing}.

\end{defun}


\begin{defun}[Generic function][Primary method]
slot-unbound class instance slot-name \\
slot-unbound (class t) instance slot-name

The generic function \cdf{slot-unbound} is called when an
unbound slot is read in an instance whose metaclass is 
\cdf{standard-class}.
The default method signals an error.

The generic function \cdf{slot-unbound} is not intended to be called
by programmers.  Programmers may write methods for it.
The function \cdf{slot-unbound} is called only by the function
\cdf{slot-value-using-class} and thus indirectly by \cdf{slot-value}.





The arguments to \cdf{slot-unbound} are the class of the instance
whose slot was accessed, the instance itself, and the name of the
slot.


If a method written for \cdf{slot-unbound} returns values, these
values get returned as the values of the original function invocation.


An unbound slot may occur if no \cd{:initform} form was
specified for the slot and the slot value has not been set, or if 
\cdf{slot-makunbound} has been called on the slot.

See \cdf{slot-makunbound}.
\end{defun}


\begin{defun}[Function]
slot-value object slot-name

The function \cdf{slot-value} returns the value contained in the slot
{\it slot-name\/} of the given object.  If there is no slot with that
name, \cdf{slot-missing} is called.  If the slot is unbound,
\cdf{slot-unbound} is called.

The macro \cdf{setf} can be used with \cdf{slot-value} to change the value
of a slot. 





The arguments are the object and the name of the given slot.


The result is the value contained in the given slot.


If an attempt is made to read a slot and no slot of the given name
exists in the instance, \cdf{slot-missing} is called as follows: 
\begin{lisp}
(slot-missing (class-of {\it instance\/}) \\
~~~~~~~~~~~~~~{\it instance} \\
~~~~~~~~~~~~~~{\it slot-name\/} \\
~~~~~~~~~~~~~~'slot-value)
\end{lisp}

If an attempt is made to write a slot and no slot of the given name
exists in the instance, \cdf{slot-missing} is called as follows: 
\begin{lisp}
(slot-missing (class-of {\it instance\/}) \\
~~~~~~~~~~~~~~{\it instance} \\
~~~~~~~~~~~~~~{\it slot-name\/} \\
~~~~~~~~~~~~~~'setf \\
~~~~~~~~~~~~~~{\it new-value\/})
\end{lisp}

The function \cdf{slot-value} is implemented using 
\cdf{slot-value-using-class}.

Implementations may optimize \cdf{slot-value} by compiling it in-line.

See \cdf{slot-missing} and \cdf{slot-unbound}.
\end{defun}


[At this point the original CLOS report \cite{SIGPLAN-CLOS,LASC-CLOS-PART-2}
contained a specification for \cdf{symbol-macrolet}.
This specification is omitted here.  Instead, a description
of \cdf{symbol-macrolet} appears with those of related constructs in chapter~\ref{CONTRL}.---GLS]


\begin{defun}[Generic function][Primary method]
update-instance-for-different-class~~~~~~~~~~ previous current &rest initargs \\
update-instance-for-different-class (previous standard-object)
     (current standard-object) &rest initargs

The generic function \cd{update-\discretionary{}{}{}instance-\discretionary{}{}{}for-\discretionary{}{}{}different-\discretionary{}{}{}class} is not
intended to be called by programmers.  Programmers may write
methods for it.  This function is called only by the function \cdf{change-class}.

The system-supplied primary method on 
\cd{update-\discretionary{}{}{}instance-\discretionary{}{}{}for-\discretionary{}{}{}different-\discretionary{}{}{}class} checks the validity of
initialization arguments and signals an error if an initialization
argument is supplied that is not declared valid.  This method then
initializes slots with values according to the initialization
arguments and initializes the newly added slots with values according
to their \cd{:initform} forms.  It does this by calling the generic
function \cdf{shared-initialize} with the following arguments: the instance,
a list of names of the newly added slots, and the initialization
arguments it received.  Newly added slots are those local slots for which
no slot of the same name exists in the previous class.

Methods for
\cd{update-\discretionary{}{}{}instance-\discretionary{}{}{}for-\discretionary{}{}{}different-\discretionary{}{}{}class}
can be defined to
specify actions to be taken when an instance is updated.  If only 
\cd{:after} methods for \cd{update-\discretionary{}{}{}instance-\discretionary{}{}{}for-\discretionary{}{}{}different-\discretionary{}{}{}class} are
defined, they will be run after the system-supplied primary method for
initialization and therefore will not interfere with the default
behavior of \cd{update-\discretionary{}{}{}instance-\discretionary{}{}{}for-\discretionary{}{}{}different-\discretionary{}{}{}class}.





The arguments to \cd{update-\discretionary{}{}{}instance-\discretionary{}{}{}for-\discretionary{}{}{}different-\discretionary{}{}{}class} are
computed by \cdf{change-class}.  When \cdf{change-class} is invoked on
an instance, a copy of that instance is made; \cdf{change-class} then
destructively alters the original instance.  The first argument to
\cd{update-\discretionary{}{}{}instance-\discretionary{}{}{}for-\discretionary{}{}{}different-\discretionary{}{}{}class}, {\it previous\/}, is that
copy; it holds the old slot values temporarily.  This argument has
dynamic extent within \cdf{change-class}; if it is referenced in any
way once \cd{update-\discretionary{}{}{}instance-\discretionary{}{}{}for-\discretionary{}{}{}different-\discretionary{}{}{}class} returns, the
results are undefined.  The second argument to 
\cd{update-\discretionary{}{}{}instance-\discretionary{}{}{}for-\discretionary{}{}{}different-\discretionary{}{}{}class}, {\it current}, is the altered
original instance.


The intended use of {\it previous\/} is to extract old slot values by using
\cdf{slot-value} or \cdf{with-slots} or by invoking a reader generic
function, or to run other methods that were applicable to instances of
the original class.

The {\it initargs\/} argument consists of alternating initialization
argument names and values.


The value returned by \cd{update-\discretionary{}{}{}instance-\discretionary{}{}{}for-\discretionary{}{}{}different-\discretionary{}{}{}class} is
ignored by \cdf{change-class}.

See the example for the function \cdf{change-class}.


Initialization arguments are declared valid by using the 
\cd{:initarg} option to \cdf{defclass}, or by defining methods for 
\cd{update-\discretionary{}{}{}instance-\discretionary{}{}{}for-\discretionary{}{}{}different-\discretionary{}{}{}class} or \cdf{shared-initialize}.  The
keyword name of each keyword parameter specifier in the lambda-list of
any method defined on \cd{update-\discretionary{}{}{}instance-\discretionary{}{}{}for-\discretionary{}{}{}different-\discretionary{}{}{}class} or 
\cdf{shared-initialize} is declared a valid initialization argument name
for all classes for which that method is applicable.

Methods on \cd{update-\discretionary{}{}{}instance-\discretionary{}{}{}for-\discretionary{}{}{}different-\discretionary{}{}{}class} can be defined to
initialize slots differently from \cdf{change-class}.  The default
behavior of \cd{change-\discretionary{}{}{}class} is described in
section~\ref{Changing-the-Class-of-an-Instance-SECTION}.


See sections~\ref{Changing-the-Class-of-an-Instance-SECTION},
\ref{Rules-for-Initialization-Arguments-SECTION}, and
\ref{Declaring-the-Validity-of-Initialization-Arguments-SECTION} as well as
\cdf{change-class} and \cdf{shared-initialize}.
\end{defun}

\penalty-10000 %required

\begin{defun}[Generic function][Primary method]
update-instance-for-redefined-class~~~~~~~~~~~~~~ instance added-slots
    discarded-slots property-list &rest initargs \\
update-instance-for-redefined-class (instance standard-object) added-slots
    discarded-slots property-list &rest initargs

\relax
\vskip 0pt plus 10pt
\noindent
The generic function \cd{update-\discretionary{}{}{}instance-\discretionary{}{}{}for-\discretionary{}{}{}redefined-\discretionary{}{}{}class} is not
intended to be called by programmers. Programmers may
write methods for it.  The generic function 
\cd{update-\discretionary{}{}{}instance-\discretionary{}{}{}for-\discretionary{}{}{}redefined-\discretionary{}{}{}class} is called by the mechanism
activated by \cdf{make-instances-obsolete}.

The system-supplied primary method on 
\cd{update-\discretionary{}{}{}instance-\discretionary{}{}{}for-\discretionary{}{}{}different-\discretionary{}{}{}class} checks the validity of
initialization arguments and signals an error if an initialization
argument is supplied that is not declared valid.  This method then
initializes slots with values according to the initialization
arguments and initializes the newly added slots with values according
to their \cd{:initform} forms.  It does this by calling the generic
function \cdf{shared-initialize} with the following arguments: the instance,
a list of names of the newly added slots, and the initialization
arguments it received.  Newly added slots are those local slots for which
no slot of the same name exists in the old version of the class.





When \cdf{make-instances-obsolete} is invoked or when a class has been
redefined and an instance is being updated, a property list is created
that captures the slot names and values of all the discarded slots with
values in the original instance.  The structure of the instance is
transformed so that it conforms to the current class definition.  The
arguments to \cd{update-\discretionary{}{}{}instance-\discretionary{}{}{}for-\discretionary{}{}{}redefined-\discretionary{}{}{}class} are this
transformed instance, a list of the names of the new slots added to the
instance, a list of the names of the old slots discarded from the
instance, and the property list containing the slot names and values for
slots that were discarded and had values.  Included in this list of
discarded slots are slots that were local in the old class and are
shared in the new class.

The {\it initargs\/} argument consists of alternating initialization
argument names and values.


The value returned by \cd{update-\discretionary{}{}{}instance-\discretionary{}{}{}for-\discretionary{}{}{}redefined-\discretionary{}{}{}class} is ignored.


Initialization arguments are declared valid by using the 
\cd{:initarg} option to \cdf{defclass} or by defining methods for 
\cd{update-\discretionary{}{}{}instance-\discretionary{}{}{}for-\discretionary{}{}{}redefined-\discretionary{}{}{}class} or \cdf{shared-initialize}.  The
keyword name of each keyword parameter specifier in the lambda-list of
any method defined on \cd{update-\discretionary{}{}{}instance-\discretionary{}{}{}for-\discretionary{}{}{}redefined-\discretionary{}{}{}class} or 
\cdf{shared-initialize} is declared a valid initialization argument name
for all classes for which that method is applicable.


See sections~\ref{Redefining-Classes-SECTION},
\ref{Rules-for-Initialization-Arguments-SECTION}, and
\ref{Declaring-the-Validity-of-Initialization-Arguments-SECTION} as well as
\cdf{shared-initialize} and \cd{make-\discretionary{}{}{}instances-\discretionary{}{}{}obsolete}.


\begin{lisp}
(defclass position () ()) \\
\\
(defclass x-y-position (position) \\*
~~((x :initform 0 :accessor position-x) \\*
~~~(y :initform 0 :accessor position-y))) \\
\\
;;; It turns out polar coordinates are used more than Cartesian  \\*
;;; coordinates, so the representation is altered and some new \\*
;;; accessor methods are added. \\
\\
(defmethod update-instance-for-redefined-class :before \\*
~~~~~~~~~~~((pos x-y-position) added deleted plist \&key) \\*
~~;; Transform the x-y coordinates to polar coordinates \\*
~~;; and store into the new slots. \\*
~~(let ((x (getf plist 'x)) \\*
~~~~~~~~(y (getf plist 'y))) \\*
~~~~(setf (position-rho pos) (sqrt (+ (* x x) (* y y))) \\*
~~~~~~~~~~(position-theta pos) (atan y x)))) \\
\\
(defclass x-y-position (position) \\*
~~~~((rho :initform 0 :accessor position-rho) \\*
~~~~~(theta :initform 0 :accessor position-theta)))
\end{lisp}
\vskip 0pt plus 10pt
\hrule width 0pt\relax
\begin{lisp}
;;; All instances of the old x-y-position class will be updated \\*
;;; automatically. \\
\\
;;; The new representation has the look and feel of the old one. \\
\\
(defmethod position-x ((pos x-y-position)) \\*
~~~(with-slots (rho theta) pos (* rho (cos theta)))) \\
\\
(defmethod (setf position-x) (new-x (pos x-y-position)) \\*
~~~(with-slots (rho theta) pos \\*
~~~~~(let ((y (position-y pos))) \\*
~~~~~~~(setq rho (sqrt (+ (* new-x new-x) (* y y))) \\*
~~~~~~~~~~~~~theta (atan y new-x)) \\*
~~~~~~~new-x))) \\
\\
(defmethod position-y ((pos x-y-position)) \\*
~~~(with-slots (rho theta) pos (* rho (sin theta))))
\end{lisp}
\begin{lisp}
(defmethod (setf position-y) (new-y (pos x-y-position)) \\*
~~~(with-slots (rho theta) pos \\*
~~~~~(let ((x (position-x pos))) \\*
~~~~~~~(setq rho (sqrt (+ (* x x) (* new-y new-y))) \\*
~~~~~~~~~~~~~theta (atan new-y x)) \\*
~~~~~~~new-y)))
\end{lisp}
\end{defun}

\newbox\hyphbox
\setbox\hyphbox\hbox{\it -}
\def\foohyphen{\copy\hyphbox}

\begin{defmac}
with-accessors ({slot-entry}*) instance-form
     {declaration}* {\,form}*

The macro \cdf{with-accessors} creates a lexical environment in which
specified slots are lexically available through their accessors as if
they were variables.  The macro \cdf{with-accessors} invokes the
appropriate accessors to access the specified slots.  Both \cdf{setf}
and \cdf{setq} can be used to set the value of the slot.

 

The result returned is that obtained by executing the forms specified
by the {\it body\/} argument.

Example:

\begin{lisp}
(with-accessors ((x position-x) (y position-y)) p1 \\*
~~(setq x y))
\end{lisp}


A \cdf{with-accessors} expression of the form
\begin{lisp}
(with-accessors ({\rm ${\it slot{\foohyphen}entry}\sub 1$} ... {\rm ${\it slot{\foohyphen}entry}\sub {\hbox{\scriptsize\it n}}$}) {\it instance\/} \\*
~~${\it declaration}\sub 1$ ... ${\it declaration}\sub {\hbox{\scriptsize\it m}}$) \\*
~~${\it form}\sub 1$ ... ${\it form}\sub {\hbox{\scriptsize\it k}}$)
\end{lisp}
expands into the equivalent of
\begin{lisp}
(let (({\it in\/} {\it instance\/})) \\
~~(symbol-macrolet (({\rm ${\it variable{\foohyphen}name}\sub 1$} ({\rm ${\it accessor{\foohyphen}name}\sub 1$} {\it in\/})) \\*
~~~~~~~~~~~~~~~~~~~~... \\*
~~~~~~~~~~~~~~~~~~~~({\rm ${\it variable{\foohyphen}name}\sub {\hbox{\scriptsize\it n}}$} ({\rm ${\it accessor{\foohyphen}name}\sub {\hbox{\scriptsize\it n}}$} {\it in\/}))) \\*
~~~~${\it declaration}\sub 1$ ... ${\it declaration}\sub {\hbox{\scriptsize\it m}}$) \\*
~~~~${\it form}\sub 1$ ... ${\it form}\sub {\hbox{\scriptsize\it k}}$)
\end{lisp}

[X3J13 voted in March 1989
\issue{SYMBOL-MACROLET-SEMANTICS}
to modify the definition of \cdf{symbol-macrolet} substantially
and also voted
\issue{SYMBOL-MACROLET-DECLARE} to allow declarations before the body
of \cdf{symbol-macrolet} but with peculiar treatment of \cdf{special}
and type declarations.  The syntactic changes are reflected in this definition
of \cdf{with-accessors}.---GLS]

See \cdf{with-slots} and \cdf{symbol-macrolet}.
\end{defmac}


\begin{defspec}
with-added-methods (function-name lambda-list
      <?option | {method-description}*>)
      {\,form}*

\relax
\vskip 0pt plus 4pt
\noindent
The \cdf{with-added-methods} special form
produces new generic functions and establishes new
lexical function definition bindings.  Each generic function is created by
adding the set of methods specified by its method definitions to a copy of the
lexically visible generic function of the same name and its methods.  If
such a generic function does not already exist, a new generic function is
created; this generic function has lexical scope.

The special form \cdf{with-added-methods} is used to define functions
whose names are meaningful only locally and to execute a series of
forms with these function definition bindings.


The names of functions defined by \cdf{with-added-methods} have lexical
scope; they retain their local definitions only within the body of the
\cdf{with-added-methods} construct.  Any references within the body of the
\cdf{with-added-methods} construct to functions whose names are the same
as those defined within the \cdf{with-added-methods} form are thus
references to the local functions instead of to any global functions
of the same names.  The scope of these generic function definition bindings
includes the method bodies themselves as well as the body of the 
\cdf{with-added-methods} construct.





The {\it function-name}, {\it option}, {\it method-qualifier}, and {\it
specialized-lambda-list\/} arguments are the same as for \cdf{defgeneric}.

The body of each method is enclosed in an implicit block.  If
{\it function-name\/} is a symbol, this block bears the same name as the
generic function.  If {\it function-name\/} is a list of the form 
\cd{(setf {\it symbol\/})}, the name of the block is {\it symbol}.  


The result returned by \cdf{with-added-methods} is the value or values
of the last form executed.  If no forms are specified, 
\cdf{with-added-methods} returns \cdf{nil}.


If a generic function with the given name already exists, the
lambda-list specified in the \cdf{with-added-methods} form must be
congruent with the lambda-lists of all existing methods on that
function as well as with the lambda-lists of all methods defined by the
\cdf{with-added-methods} form; otherwise an error is signaled.

If {\it function-name\/} specifies an existing generic function that has a
different value for any of the following {\it option\/} arguments, the
copy of that generic function is modified to have the new value: 
\cd{:argument-precedence-order}, \cdf{declare}, \cd{:documentation}, 
\cd{:generic-function-class}, \cd{:method-combination}.

If {\it function-name\/} specifies an existing generic function that has a
different value for the \cd{:method-class} {\it option\/} argument,
that value is changed in the copy of that generic function, but any
methods copied from the existing generic function are not changed.

If a function of the given name already exists, that function is copied into
the default method for a generic function of the given name.  Note that
this behavior differs from that of \cdf{defgeneric}.

If a macro or special form of the given name already exists, an error
is signaled.

If there is no existing generic function, the {\it option\/} arguments have
the same default values as the {\it option\/} arguments to \cdf{defgeneric}.

See \cdf{generic-labels},
\cdf{generic-flet},
\cdf{defmethod},
\cdf{defgeneric},
and \cd{ensure-\discretionary{}{}{}generic-\discretionary{}{}{}function}.
\end{defspec}


\begin{defmac}
with-slots ({slot-entry}*) instance-form {declaration}* {\,form}*

\begin{tabbing}
{\it slot-entry\/} ::= {\it slot-name\/} {\Mor} \cd{({\it variable-name\/} {\it slot-name\/})}
\end{tabbing}
The macro \cdf{with-slots} creates a lexical context for referring to
specified slots as though they were variables.  Within such a context
the value of the slot can be specified by using its slot name, as if
it were a lexically bound variable.  Both \cdf{setf} and \cdf{setq}
can be used to set the value of the slot.

The macro \cdf{with-slots} translates an appearance of the slot name as
a variable into a call to \cdf{slot-value}.

  



The result returned is that obtained by executing the forms specified
by the {\it body\/} argument.

Example:

\begin{lisp}
(with-slots (x y) position-1 \\*
~~(sqrt (+ (* x x) (* y y)))) \\
\\
(with-slots ((x1 x) (y1 y)) position-1 \\*
~~(with-slots ((x2 x) (y2 y)) position-2 \\*
~~~~(psetf x1 x2 \\*
~~~~~~~~~~~y1 y2)))) \\
\\
(with-slots (x y) position \\*
~~(setq x (1+ x) \\*
~~~~~~~~y (1+ y)))
\end{lisp}


A \cdf{with-slots} expression of the form:
\begin{lisp}
(with-slots ({\it ${\it slot{\foohyphen}entry}\sub 1$} ... {\it ${\it slot{\foohyphen}entry}\sub {\hbox{\scriptsize\it n}}$}) {\it instance\/} \\*
~~${\it declaration}\sub 1$ ... ${\it declaration}\sub {\hbox{\scriptsize\it m}}$) \\*
~~${\it form}\sub 1$ ... ${\it form}\sub {\hbox{\scriptsize\it k}}$)
\end{lisp}
expands into the equivalent of
\begin{lisp}
(let (({\it in\/} {\it instance\/})) \\
~~(symbol-macrolet (${\it Q}\sub 1$ ... ${\it Q}\sub {\hbox{\scriptsize\it n}}$) \\*
~~~~${\it declaration}\sub 1$ ... ${\it declaration}\sub {\hbox{\scriptsize\it m}}$) \\*
~~~~${\it form}\sub 1$ ... ${\it form}\sub {\hbox{\scriptsize\it k}}$)
\end{lisp}
where ${\hbox{{\it Q}}}\sub {\hbox{\scriptsize\it j}}$ is 
\begin{lisp}
({\rm ${\it slot{\foohyphen}entry}\sub {\hbox{\scriptsize\it j}}$} (slot-value {\it in\/} '{\it ${\it slot{\foohyphen}entry}\sub {\hbox{\scriptsize\it j}}$}))
\end{lisp}
if ${\hbox{{\it slot-entry}}}\sub {\hbox{\scriptsize\it j}}$ is a symbol and is
\begin{lisp}
({\rm ${\it variable{\foohyphen}name}\sub {\hbox{\scriptsize\it j}}$} (slot-value {\it in\/} '{\rm ${\it slot{\foohyphen}name}\sub {\hbox{\scriptsize\it j}}$}))
\end{lisp}
if ${\hbox{{\it slot-entry}}}\sub {\hbox{\scriptsize\it j}}$
is of the form \cd{({\rm ${\it variable{\foohyphen}name}\sub {\hbox{\scriptsize\it j}}$} {\rm ${\it slot{\foohyphen}name}\sub {\hbox{\scriptsize\it j}}$})}.

[X3J13 voted in March 1989
\issue{SYMBOL-MACROLET-SEMANTICS}
to modify the definition of \cdf{symbol-macrolet} substantially
and also voted
\issue{SYMBOL-MACROLET-DECLARE} to allow declarations before the body
of \cdf{symbol-macrolet} but with peculiar treatment of \cdf{special}
and type declarations.  The syntactic changes are reflected in this definition
of \cdf{with-slots}.---GLS]

See \cdf{with-accessors} and \cdf{symbol-macrolet}.

\end{defmac}
%\endgroup

        % Common Lisp Object System
%%%Chapter of Common Lisp Manual.  Copyright 1989 Guy L. Steele Jr.

\clearpage\def\pagestatus{FINAL PROOF}

\chapterauthor{Kent M. Pitman}

\def\SU#1{${}_{#1}$}

\chapter{Conditions}
\label{CONDITION}

\prefaceword
\begin{new}
The language defined by the first edition contained an enormous lacuna:
although facilities were specified for signaling errors,
no means was defined for handling errors.  This occurred not through neglect
of the issue, but because this part of the Lisp language generally
was in a state of flux.  There were several proposals at the
time.  The committee, finding that it could not agree on any one proposal,
agreed to disagree and omit error handling from Common Lisp for the time being.
This defect has now been addressed.
\end{new}

X3J13 voted in June 1988
\issue{CONDITION-SYSTEM}
to adopt the Common Lisp Condition System
as a part of the forthcoming draft Common Lisp standard.
X3J13 voted in March 1989 \issue{ZLOS-CONDITIONS}
to amend the specification of conditions to integrate them
with the Common Lisp Object System (see chapter~\ref{CLOS}). 
X3J13 voted in June 1989 \issue{CONDITION-RESTARTS} to amend the
specification of restarts in certain ways. These amendments have
been incorporated here with little further comment.

This chapter presents the bulk of the Common Lisp
Condition System proposal, written by Kent~M. Pitman
and amended by X3J13.  I have edited it only very lightly
to conform to the overall style of this book and have inserted a small
number of bracketed remarks identified by the initials GLS.
Please see the Acknowledgments to this second edition for the author's
acknowledgments to others who contributed to the Condition System proposal.

\noindent\hbox to \textwidth{\hss---Guy L. Steele Jr.}
\vskip 8pt plus 3pt minus 2pt



\section{Introduction}

Often we find it useful to describe a function in terms of its behavior in
``normal situations.'' For example, we may say informally that the function
\cdf{+} returns the sum of its arguments or that the function
\cdf{read-char} returns the next available character on a given input
stream.

Sometimes, however, an ``exceptional situation'' will arise that does not fit
neatly into such descriptions. For example, \cdf{+} might receive an argument
that is not a number, or \cdf{read-char} might receive as a single argument
a stream that has no more available characters.  This distinction between normal
and exceptional situations is in some sense arbitrary but is often very
useful in practice.

For example, suppose a function \cdf{f} were defined to allow only
integer arguments but also guaranteed to
detect and signal an error for non-integer arguments.
Such a description is in fact internally inconsistent (that is,
paradoxical) because the function's behavior is well-defined for non-integers.
Yet we would not want this annoying paradox to force description of \cdf{f}
as a function that accepts any kind of argument (just in case \cdf{f}
is being called only as a quick way to signal an error, for example).
Using the normal/exceptional distinction, we can say clearly that \cdf{f} accepts integers
in the normal situation and signals an error in exceptional situations.
Moreover, we can say that when we refer to the definition of a
function informally, it is acceptable to speak only of its normal behavior.
For example, we can speak informally about \cdf{f} as a function that accepts only
integers without feeling that we are committing some awful fraud.

Not all exceptional situations are errors.  For example, a program that is
directing the typing of a long line of text may come to an end-of-line.
It is possible that no real harm will result from failing to signal end-of-line
to its caller because the operating system will simply force a carriage
return on the output device, which will continue typing on the next line. However, it
may still be interesting to establish a protocol whereby the printing program can
inform its caller of end-of-line exceptions. The caller could
then opt to deal with these situations in interesting ways at certain times.
For example, a caller might choose to terminate printing, obtaining an end-of-line
truncation. The important thing, however, is that the failure of the
caller to provide advice about the situation need not prevent
the printer program from operating correctly.

Mechanisms for dealing with exceptional situations vary widely. When an
exceptional situation is encountered, a program may attempt to handle
it by returning a distinguished value, returning an additional value,
setting a variable, calling a function, performing a special transfer of
control, or stopping the program altogether and entering the debugger.

For the most part, the facilities described in this chapter do not introduce
any fundamentally new way of dealing with exceptional situations. Rather, they
encapsulate and formalize useful patterns of data and control flow that have
been seen to be useful in dealing with exceptional situations.

A proper conceptual approach to errors should perhaps begin from first
principles, with a discussion of {\it conditions} in general, and eventually work
up to the concept of an {\it error} as just one of the many kinds of
conditions. However, given the primitive state of error-handling
technology, a proper buildup may be as inappropriate as requiring that a
beggar learn to cook a gourmet meal before being allowed to eat.  Thus,
we deal first with the essentials---error handling---and then
go back later to fill in the missing details.

\section{Changes in Terminology}

In this section, we introduce changes to the terminology
defined in section~\ref{INTRO-ERRORS}.

A {\it condition} is an interesting situation in a program that has been
detected and announced. Later we allow this term also to refer to
objects that programs use to represent such situations.

An {\it error} is a condition in which normal program execution may not
continue without some form of intervention (either interactively by the user
or under some sort of program control, as described below).

The process by which a condition is formally announced by a program is called
{\it signaling}. The function \cdf{signal} is the primitive mechanism by which such
announcement is done. Other abstractions, such as \cdf{error} and \cdf{cerror}, are built
using \cdf{signal}.

The first edition is ambiguous about the reason why a particular program action
``is an error.'' There are two principal reasons why an action may be an error
without being required to signal an error:
\begin{itemize}
\item Detecting the error might be prohibitively expensive.

   For example, \cd{(+ nil 3)} is an error. It is likely that the designers of
   Common Lisp believed this would be an error in all implementations but
   felt it might be excessively expensive to detect the problem
   in compiled code on stock hardware, so they did not require that it signal
   an error.

\item Some implementations might implement the behavior as an extension.

   For example, \cd{(loop for x from 1 to 3 do (print x))} is an error because \cdf{loop}
   is not defined to take atoms in its body.
   In fact, however, some
   implementations offer an extension that makes this well-defined. In order
   to leave room for such extensions, the first edition used the ``is an error''
   terminology to keep implementors from being forced to signal an error in
   the extended implementations.

   [This example was written well before the vote by X3J13 in January 1989
    to add exactly this extension to the forthcoming draft standard
    (see chapter~\ref{LOOP}).---GLS]
\end{itemize}

In this chapter, we use the following terminology.
[Compare this to the terminology presented
in section~\ref{Error-Terminology-SECTION}.---GLS]
\begin{itemize}
\item
   If the signaling of a condition or error is part of a function's contract
   in all situations, we say that it ``signals'' or ``must signal'' that
   condition or error.

\item
   If the signaling of a condition or error is optional for some important
   reason (such as performance), we say that the program ``might signal''
   that condition or error. In this case, we are defining the operation to be
   illegal in all implementations, but allowing some implementations to fail to
   detect the error.

\item
   If an action is left undefined for the sake of implementation-dependent
   extension, we say that it ``is undefined'' or ``has undefined effect.''
   This means that it is not possible to depend portably upon the effects of
   that action. A program that has undefined effect may enter the debugger,
   transfer control, or modify data in unpredictable ways.

\item
   In the special case where only the return value of an operation is not
   well defined but any side effect and transfer-of-control behavior is
   well defined, we say that it has ``undefined value.'' In this case,
   the number and nature of the return values is not defined, but the
   function can reasonably be expected to return. It is worth noting that
   under this description, there are some (though not many) legitimate ways
   in which such return value(s) can be used. For example, if the function
   \cdf{foo} has no side effects and undefined value, the expression 
   \cd{(length (list (foo)))} is completely well defined even for portable code.
   However, the effect of \cd{(print (list (foo)))} is not well defined.
\end{itemize}


\section{Survey of Concepts}

This section discusses various aspects of the condition system by topic,
illustrating them with extensive examples.  The next section contains
definitions of specific functions, macros, and other facilities.

\subsection{Signaling Errors}

Conceptually, signaling an error in a program is an admission by that program
that it does not know how to continue and requires external intervention. Once
an error is signaled, any decision about how to continue must come from the
``outside.''

The simplest way to signal an error is to use the \cdf{error} function with
\cdf{format}-style arguments describing the error for the sake of the user interface.
If \cdf{error} is called and there are no active handlers (described
in sections~\ref{TRAPPING-ERRORS} and~\ref{HANDLING-CONDITIONS}), the
debugger will be entered and the error message will be typed out. For example:
\begin{lisp}
Lisp> (defun factorial (x) \\*
~~~~~~~~(cond ((or (not (typep x 'integer)) (minusp x)) \\*
~~~~~~~~~~~~~~~(error "{\Xtilde}S is not a valid argument to FACTORIAL." \\*
~~~~~~~~~~~~~~~~~~~~~~x)) \\
~~~~~~~~~~~~~~((zerop x) 1) \\
~~~~~~~~~~~~~~(t (* x (factorial (- x 1)))))) \\*
~\EV\ FACTORIAL \\
Lisp> (factorial 20) \\*
~\EV\ 2432902008176640000 \\
Lisp> (factorial -1) \\*
Error: -1 is not a valid argument to FACTORIAL. \\*
To continue, type :CONTINUE followed by an option number: \\*
~1: Return to Lisp Toplevel. \\*
Debug> 
\end{lisp}
In general, a call to \cdf{error} cannot directly return. Unless special work has
been done to override this behavior, the debugger will be entered and there
will be no option to simply continue.

The only exception may be that some implementations may provide debugger
commands for interactively returning from individual stack frames; even then,
however, such commands should never be used except by someone who has read the
erring code and understands the consequences of continuing from that point. In
particular, the programmer should feel confident
about writing code like this:
\begin{lisp}
(defun wargames:no-win-scenario () \\*
~~(when (true) (error "Pushing the button would be stupid.")) \\*
~~(push-the-button))
\end{lisp}
In this scenario, there should be no chance that the function \cdf{error} will return
and the button will be pushed.

\beforenoterule
\begin{sideremark}
It should be noted that the notion of
``no chance'' that the button will be pushed is relative only to the language
model; it assumes that the language is accurately implemented.  In practice,
compilers have bugs, computers have glitches, and users have been known
to interrupt at inopportune moments and use the debugger to return from
arbitrary stack frames.  Such violations of the language model are
beyond the scope of the condition system but not necessarily beyond the
scope of potential failures that the programmer should consider and defend against.
The possibility of such unusual failures may of course also influence the design of
code meant to handle less drastic situations,
such as maintaining a database uncorrupted.---KMP and GLS
\end{sideremark}
\afternoterule

In some cases, the programmer may have a single, well-defined idea of a
reasonable recovery strategy for this particular error. In that case, he can
use the function \cdf{cerror}, which specifies information about what would happen
if the user did simply continue from the call to \cdf{cerror}. For example:
\begin{lisp}
Lisp> (defun factorial (x) \\*
~~~~~~~~(cond ((not (typep x 'integer)) \\*
~~~~~~~~~~~~~~~(error "{\Xtilde}S is not a valid argument to FACTORIAL." \\*
~~~~~~~~~~~~~~~~~~~~~~x)) \\
~~~~~~~~~~~~~~((minusp x) \\*
~~~~~~~~~~~~~~~(let ((x-magnitude (- x))) \\*
~~~~~~~~~~~~~~~~~(cerror "Compute -({\Xtilde}D!) instead." \\*
~~~~~~~~~~~~~~~~~~~~~~~~~"(-{\Xtilde}D)! is not defined." x-magnitude) \\*
~~~~~~~~~~~~~~~~~(- (factorial x-magnitude)))) \\
~~~~~~~~~~~~~~((zerop x) 1) \\
~~~~~~~~~~~~~~(t (* x (factorial (- x 1)))))) \\*
~\EV\ FACTORIAL \\
Lisp> (factorial -3) \\*
Error: (-3)! is not defined. \\*
To continue, type :CONTINUE followed by an option number: \\*
~1: Compute -(3!) instead. \\*
~2: Return to Lisp Toplevel. \\*
Debug> :continue 1 \\
~\EV\ -6
\end{lisp}


\subsection{Trapping Errors}
\label{TRAPPING-ERRORS}

By default, a call to \cdf{error} will force entry into the debugger.  You can
override that behavior in a variety of ways. The simplest (and most blunt)
tool for inhibiting entry to the debugger on an error is to use \cdf{ignore-errors}.
In the normal situation, forms in the body of \cdf{ignore-errors} are evaluated
sequentially and the last value is returned. If a condition of type \cdf{error} is
signaled, \cdf{ignore-errors} immediately returns two values, namely \cdf{nil} and the
condition that was signaled; the debugger is not entered and no error
message is printed. For example:
\begin{lisp}
Lisp> (setq filename "nosuchfile") \\
~\EV\ "nosuchfile" \\
Lisp> (ignore-errors (open filename :direction :input)) \\
~\EV\ NIL {\rm and} \#<FILE-ERROR 3437523>
\end{lisp}
The second return value is an object that represents the kind of error. This
is explained in greater detail in section~\ref{OBJECT-0RIENTED-BASIS}.

In many cases, however, \cdf{ignore-errors} is not desirable because it deals with
too many kinds of errors. Contrary to the belief of some, a program that
does not enter the debugger is not necessarily better than one that does.
Excessive use of \cdf{ignore-errors} may keep the program out of the debugger, but it may
not increase the program's reliability, because the program may continue
to run after encountering errors other than those you meant to work past. In
general, it is better to attempt to deal only with the particular kinds of
errors that you believe could legitimately happen. That way, if an unexpected
error comes along, you will still find out about it.

\cdf{ignore-errors} is a useful special case built from a more general facility,
\cdf{handler-case}, that allows the programmer to deal with particular kinds of
conditions (including non-error conditions) without affecting what happens
when other kinds of conditions are signaled. For example, an effect 
equivalent to that of \cdf{ignore-errors} above is achieved in the following example:
\begin{lisp}
Lisp> (setq filename "nosuchfile") \\
~\EV\ "nosuchfile" \\
Lisp> (handler-case (open filename :direction :input) \\
~~~~~~~~(error (condition) \\
~~~~~~~~~~(values nil condition))) \\
~\EV\ NIL {\rm and} \#<FILE-ERROR 3437525>
\end{lisp}
However, using \cdf{handler-case}, one can indicate a more specific condition
type than just ``error.'' Condition types are explained in detail later, but the
syntax looks roughly like the following:
\begin{lisp}
Lisp> (makunbound 'filename) \\
~\EV\ FILENAME \\
Lisp> (handler-case (open filename :direction :input) \\
~~~~~~~~(file-error (condition) \\
~~~~~~~~~~(values nil condition))) \\
Error: The variable FILENAME is unbound. \\
To continue, type :CONTINUE followed by an option number: \\
~1: Retry getting the value of FILENAME. \\
~2: Specify a value of FILENAME to use this time. \\
~3: Specify a value of FILENAME to store and use. \\
~4: Return to Lisp Toplevel. \\
Debug> 
\end{lisp}


\subsection{Handling Conditions}
\label{HANDLING-CONDITIONS}

Blind transfer of control to a \cdf{handler-case} is only one possible kind
of recovery action that can be taken when a condition is signaled.  The
low-level mechanism offers great flexibility in how to continue once
a condition has been signaled. 

The basic idea behind condition handling is that a piece of code called the
{\it signaler} recognizes and announces the existence of an exceptional
situation using \cdf{signal} or some function built on \cdf{signal} (such as \cdf{error}). 

The process of signaling involves the search for and invocation of a
{\it handler}, a piece of code that will attempt to deal appropriately with
the situation. 

If a handler is found, it may either {\it handle} the situation, by performing
some non-local transfer of control, or {\it decline} to handle it, by failing to perform a
non-local transfer of control. If it declines, other handlers are sought.

Since the lexical environment of the signaler might not be available to
handlers, a data structure called a {\it condition} is created to represent
explicitly the relevant state of the situation. A condition either is created
explicitly using \cdf{make-condition} and then passed to a function such as \cdf{signal},
or is created implicitly by a function such as \cdf{signal} when given appropriate
non-condition arguments.

In order to handle the error, a handler is permitted to use any non-local
transfer of control such as \cdf{go} to a tag in a \cdf{tagbody},
\cdf{return} from a \cdf{block}, or
\cdf{throw} to a \cdf{catch}. In addition, structured abstractions of these
primitives are provided for convenience in exception handling.

A handler can be made dynamically accessible to a program by use of
\cdf{handler-bind}. For example, to create a handler for a condition of type
\cdf{arithmetic-error}, one might write:
\begingroup
\makeatletter
\def\@listi{\leftmargin\leftmargini \labelsep\leftmargin
   \parsep 3pt\relax
   \topsep 4pt plus 9pt\relax
   \itemsep\topsep}
\makeatother
\begin{lisp}
(handler-bind ((arithmetic-error {\it handler})){\it body})
\end{lisp}
The handler is a function of one argument, the condition. If a condition of
the designated type is signaled while the {\it body} is executing (and there are no
intervening handlers), the handler would be invoked on the given condition,
allowing it the option of transferring control. For example, one might write a
macro that executes a body, returning either its value(s) or the two values
\cdf{nil} and the condition:
\begin{lisp}
(defmacro without-arithmetic-errors (\&body forms) \\
~~(let ((tag (gensym))) \\
~~~~`(block ,tag \\
~~~~~~ (handler-bind ((arithmetic-error \\
~~~~~~~~~~~~~~~~~~~~~~~~~\#'(lambda (c)~~~~~;{\rm Argument \cdf{c} is a condition} \\
~~~~~~~~~~~~~~~~~~~~~~~~~~~~ (return-from ,tag (values nil c))))) \\
~~~~~~~~~,@body)))) \\
\end{lisp}
\endgroup
The handler is executed in the dynamic context of the signaler, except
that the set of available condition handlers will have been rebound to
the value that was active at the time the condition handler was made
active. If a handler declines (that is, it does not transfer control), other 
handlers are sought. If no handler is found and the condition was signaled
by \cdf{error} or \cdf{cerror} (or some function such as \cdf{assert} that behaves like
these functions), the debugger is entered, still in the dynamic context 
of the signaler.

\subsection{Object-Oriented Basis of Condition Handling}
\label{OBJECT-0RIENTED-BASIS}

Of course, the ability of the handler to usefully handle an exceptional
situation is related to the quality of the information it is provided. For
example, if all errors were signaled by
\begin{lisp}
(error "{\it some format string}")
\end{lisp}
then the only piece of information that would be accessible to the handler
would be an object of type \cdf{simple-error} that had a slot containing the
format string.

If this were done, \cdf{string-equal} would be the preferred way to tell one error
from another, and it would be very hard to allow flexibility in the
presentation of error messages because existing handlers would tend to be
broken by even tiny variations in the wording of an error message. This
phenomenon has been the major failing of most error systems previously
available in Lisp. It is fundamentally important to decouple the error
message string (the human interface) from the objects that formally
represent the error state (the program interface). We therefore have the
notion of typed conditions, and of formal operations on those conditions
that make them inspectable in a structured way.

This object-oriented approach to condition handling has the following
important advantages over a text-based approach:
\begin{itemize}
\item
   Conditions are classified according to subtype relationships, making
   it easy to test for categories of conditions.

\item
   Conditions have named slot values through which parameters are conveyed
   from the program that signals the condition to the program that handles it.

\item
   Inheritance of methods and slots reduces the amount of explicit
   specification necessary to achieve various interesting effects.
\end{itemize}

Some condition types are defined by this document, but the set of 
condition types is extensible using \cdf{define-condition}.
Common Lisp condition types are in fact CLOS classes, and condition objects
are ordinary CLOS objects; \cdf{define-condition} merely
provides an abstract interface that is a bit more convenient than
\cdf{defclass} for defining conditions.

Here, as an example,
we define a two-argument function called \cdf{divide} that is patterned after
the \cdf{/} function but does some stylized error checking:
\begin{lisp}
(defun divide (numerator denominator) \\
~~(cond ((or (not (numberp numerator)) \\
~~~~~~~~~~~~~(not (numberp denominator))) \\
~~~~~~~~~(error "(DIVIDE '{\Xtilde}S '{\Xtilde}S) - Bad arguments." \\
~~~~~~~~~~~~~~~~numerator denominator)) \\
~~~~~~~~((zerop denominator) \\
~~~~~~~~~(error 'division-by-zero \\
~~~~~~~~~~~~~~~~:operator 'divide \\
~~~~~~~~~~~~~~~~:operands (list numerator denominator))) \\
~~~~~~~~(t ...)))
\end{lisp}
Note that in the first clause we have used \cdf{error} with a string argument
and in the second clause we have named a particular condition type,
\cdf{division-by-zero}. In the case of a string argument, the condition type that
will be signaled is \cdf{simple-error}.

The particular kind of error that is signaled may be important
in cases where handlers are active. For example, \cdf{simple-error} inherits 
from type \cdf{error}, which in turn inherits from type \cdf{condition}. On the 
other hand, \cdf{division-by-zero} inherits from \cdf{arithmetic-error}, which 
inherits from \cdf{error}, which inherits from \cdf{condition}. So if a handler
existed for \cdf{arithmetic-error} while a \cdf{division-by-zero} condition was
signaled, that handler would be tried; however, if a \cdf{simple-error}
condition were signaled in the same context, the handler for type
\cdf{arithmetic-error} would not be tried.


\subsection{Restarts}
\label{RESTARTS}

In older Lisp dialects (such as MacLisp), an attempt to signal an error of a
given type often carried with it an implicit promise to support the standard
recovery strategy for that type of error. If the signaler knew the type of
error but for whatever reason was unable to deal with the standard recovery
strategy for that kind of error, it was necessary to signal an untyped error
(for which there was no defined recovery strategy). This sometimes led to
confusion when people signaled typed errors without realizing the full
implications of having done so, but more often than not it meant that users
simply avoided typed errors altogether.

The Common Lisp Condition System, which is modeled after the Zetalisp condition system,
corrects this troublesome aspect of previous Lisp dialects by creating a clear
separation between the act of signaling an error of a particular type and the
act of saying that a particular way of recovery is appropriate. In the \cdf{divide}
example above, simply signaling an error does not imply a willingness on the
part of the signaler to cooperate in any corrective action. For example, the
following sample interaction illustrates that the only recovery action
offered for this error is ``Return to Lisp Toplevel'':
\begin{lisp}
Lisp> (+ (divide 3 0) 7) \\
Error: Attempt to divide 3 by 0. \\
To continue, type :CONTINUE followed by an option number: \\
~1: Return to Lisp Toplevel. \\
Debug> :continue 1 \\
Returned to Lisp Toplevel. \\
Lisp>
\end{lisp}
When an error is detected and the function \cdf{error} is called, execution cannot
continue normally because \cdf{error} will not directly return. Control can be
transferred to other points in the program, however, by means of specially
established ``restarts.''

\subsection{Anonymous Restarts}

The simplest kind of restart involves structured transfer of control using
a macro called \cdf{restart-case}. The \cdf{restart-case} form allows execution of
a piece of code in a context where zero or more restarts are active, and
where if one of those restarts is ``invoked,'' control will be transferred
to the corresponding clause in the \cdf{restart-case} form. For example, we could
rewrite the previous \cdf{divide} example as follows.
\begin{lisp}
(defun divide (numerator denominator) \\
~~(loop \\
~~~~(restart-case \\
~~~~~~~~(return \\
~~~~~~~~~~(cond ((or (not (numberp numerator)) \\
~~~~~~~~~~~~~~~~~~~~~(not (numberp denominator))) \\
~~~~~~~~~~~~~~~~~(error "(DIVIDE '{\Xtilde}S '{\Xtilde}S) - Bad arguments." \\
~~~~~~~~~~~~~~~~~~~~~~~~~numerator denominator)) \\
~~~~~~~~~~~~~~~~((zerop denominator) \\
~~~~~~~~~~~~~~~~~(error 'division-by-zero \\
~~~~~~~~~~~~~~~~~~~~~~~~:operator 'divide \\
~~~~~~~~~~~~~~~~~~~~~~~~:operands (list numerator denominator))) \\
~~~~~~~~~~~~~~~~(t ...))) \\
~~~~~~(nil (arg1 arg2) \\
~~~~~~~~~~:report "Provide new arguments for use by DIVIDE." \\
~~~~~~~~~~:interactive \\
~~~~~~~~~~~~(lambda () \\
~~~~~~~~~~~~~~~(list (prompt-for 'number "Numerator: ") \\
~~~~~~~~~~~~~~~~~~~~~(prompt-for 'number "Denominator: "))) \\
~~~~~~~~(setq numerator arg1 denominator arg2)) \\
~~~~~~(nil (result) \\
~~~~~~~~~~:report "Provide a value to return from DIVIDE." \\
~~~~~~~~~~:interactive \\
~~~~~~~~~~~~(lambda () (list (prompt-for 'number "Result: "))) \\
~~~~~~~~(return result)))))
\end{lisp}

\beforenoterule
\begin{sideremark}
    The function \cdf{prompt-for} used in this chapter in a number of places is
    not a part of Common Lisp.  It is used in the examples in this chapter only to keep
    the presentation simple.  It is assumed to accept a type specifier
     and optionally a format string and associated arguments.  It uses the
    format string and associated arguments as part of an interactive prompt,
    and uses \cdf{read} to read a Lisp object; however, only an object of the
    type indicated by the type specifier is accepted.

    The question of whether or not \cdf{prompt-for} (or something like it) would be a
    useful addition to Common Lisp is under consideration by X3J13, but as of
    January 1989 no action has been taken. In spite of its use in a number of examples,
    nothing in the Common Lisp Condition System depends on this function.
\end{sideremark}
\afternoterule

In the example, the \cdf{nil} at the head of each clause
means that it is an ``anonymous'' restart.
Anonymous restarts are typically invoked only from within the
debugger. As we shall see later, it is possible to have ``named restarts''
that may be invoked from code without the need for user intervention.

If the arguments to anonymous restarts are not optional, then special
information must be provided about what the debugger should use as arguments.
Here the \cd{:interactive} keyword is used to specify that information.

The \cd{:report} keyword introduces information to be used when presenting the
restart option to the user (by the debugger, for example).

Here is a sample interaction that takes advantage of the restarts provided
by the revised definition of \cdf{divide}:
\begin{lisp}
Lisp> (+ (divide 3 0) 7) \\
Error: Attempt to divide 3 by 0. \\
To continue, type :CONTINUE followed by an option number: \\
~1: Provide new arguments for use by the DIVIDE function. \\
~2: Provide a value to return from the DIVIDE function. \\
~3: Return to Lisp Toplevel. \\
Debug> :continue 1 \\
1 \\
Numerator: 4 \\
Denominator: 2 \\
~\EV\ 9
\end{lisp}

\subsection{Named Restarts}

In addition to anonymous restarts, one can have named restarts, which can be invoked
by name from within code.  As a trivial example, one could write
\begin{lisp}
(restart-case (invoke-restart 'foo 3) \\
~~(foo (x) (+ x 1)))
\end{lisp}
to add \cd{3} to \cd{1}, returning \cd{4}. This trivial example is conceptually analogous to 
writing:
\begin{lisp}
(+ (catch 'something (throw 'something 3)) 1)
\end{lisp}

For a more realistic example, the code for the function \cdf{symbol-value} might signal an
unbound variable error as follows:
\begin{lisp}
(restart-case (error "The variable {\Xtilde}S is unbound." variable) \\*
~~(continue () \\*
~~~~~~:report \\*
~~~~~~~~(lambda (s)~~~~~;{\rm Argument \cdf{s} is a stream} \\*
~~~~~~~~~~(format s "Retry getting the value of {\Xtilde}S." variable)) \\*
~~~~(symbol-value variable)) \\
~~(use-value (value) \\*
~~~~~~:report \\*
~~~~~~~~(lambda (s)~~~~~;{\rm Argument \cdf{s} is a stream} \\*
~~~~~~~~~~(format s "Specify a value of {\Xtilde}S to use this time." \\*
~~~~~~~~~~~~~~~~~~variable)) \\*
~~~~value) \\
~~(store-value (value) \\*
~~~~~~:report \\*
~~~~~~~~(lambda (s)~~~~~;{\rm Argument \cdf{s} is a stream} \\*
~~~~~~~~~~(format s "Specify a value of {\Xtilde}S to store and use." \\*
~~~~~~~~~~~~~~~~~~variable)) \\*
~~~~(setf (symbol-value variable) value) \\*
~~~~value))
\end{lisp}
If this were part of the implementation of \cdf{symbol-value}, then it would be possible
for users to write a variety of automatic handlers for unbound variable
errors. For example, to make unbound variables evaluate to themselves, one
might write
\begin{lisp}
(handler-bind ((unbound-variable \\
~~~~~~~~~~~~~~~~~\#'(lambda (c)~~~~~;{\rm Argument \cdf{c} is a condition} \\
~~~~~~~~~~~~~~~~~~~~~(when (find-restart 'use-value) \\
~~~~~~~~~~~~~~~~~~~~~~~(invoke-restart 'use-value \\
~~~~~~~~~~~~~~~~~~~~~~~~~~~~~~~~~~~~~~~(cell-error-name c)))))) \\
~~{\it body})
\end{lisp}

\subsection{Restart Functions}

For commonly used restarts, it is conventional to define a program interface
that hides the use of \cdf{invoke-restart}. Such program interfaces to restarts
are called {\it restart functions}.

The normal convention is for the function to share the name of the restart.
The pre-defined functions \cdf{abort}, \cdf{continue}, \cdf{muffle-warning}, \cdf{store-value}, and
\cdf{use-value} are restart functions. With \cdf{use-value} the above example of 
\cdf{handler-bind} could have been written more concisely as
\begin{lisp}
(handler-bind ((unbound-variable \\
~~~~~~~~~~~~~~~~~~~\#'(lambda (c)~~~~~;{\rm Argument \cdf{c} is a condition} \\
~~~~~~~~~~~~~~~~~~~~~~~(use-value (cell-error-name c))))) \\
~~{\it body})
\end{lisp}

\subsection{Comparison of Restarts and Catch/Throw}
  
One important feature that \cdf{restart-case} (or \cdf{restart-bind}) offers that
\cdf{catch} does not is the ability to reason about the available points to
which control might be transferred without actually attempting the
transfer. One could, for example, write
\begin{lisp}
(ignore-errors (throw ...))
\end{lisp}
which is a sort of poor man's variation of
\begin{lisp}
(when (find-restart 'something) \\*
~~(invoke-restart 'something))
\end{lisp}
but there is no way to use \cdf{ignore-errors} and \cdf{throw} to simulate something
like
\begin{lisp}
(when (and (find-restart 'something) \\*
~~~~~~~~~~~(find-restart 'something-else)) \\*
~~(invoke-restart 'something))
\end{lisp}
or even just
\begin{lisp}
(when (and (find-restart 'something) \\
~~~~~~~~~~~(yes-or-no-p "Do something? ")) \\
~~(invoke-restart 'something))
\end{lisp}
because the degree of inspectability that comes with simply writing
\begin{lisp}
(ignore-errors (throw ...))
\end{lisp}
is too primitive---getting the desired information also forces
transfer of control, perhaps at a time when it is not desirable.

Many programmers have previously evolved strategies like the following
on a case-by-case basis:
\begin{lisp}
(defvar *foo-tag-is-available* nil) \\
\\
(defun fn-1 () \\
~~(catch 'foo \\
~~~~(let ((*foo-tag-is-available* t)) \\
~~~~~~... (fn-2) ...))) \\
\\
(defun fn-2 () \\
~~... \\
~~(if *foo-tag-is-available* (throw 'foo t)) \\
~~...)
\end{lisp}
The facility provided by \cdf{restart-case} and \cdf{find-restart} is intended to
provide a standardized protocol for this sort of information to be
communicated between programs that were developed independently so that
individual variations from program to program
do not thwart the overall modularity and debuggability of programs.

Another difference between the restart facility and the \cdf{catch}/\cdf{throw}
facility is that a \cdf{catch} with any given tag completely shadows any
outer pending \cdf{catch} that uses the same tag. Because of the presence
of \cdf{compute-restarts}, however, it is possible to see shadowed restarts,
which may be very useful in some situations (particularly in an
interactive debugger).

\penalty-10000 %required

\subsection{Generalized Restarts}
\label{LAST-RESTARTS-SECTION}

\cdf{restart-case} is a mechanism that allows only imperative transfer of control
for its associated restarts. \cdf{restart-case} is built on a lower-level mechanism
called \cdf{restart-bind}, which does not force transfer of control.

\cdf{restart-bind} is to \cdf{restart-case} as \cdf{handler-bind} is to
\cd{handler-\discretionary{}{}{}case}.
The syntax is
\begin{lisp}
(restart-bind (({\it name} {\it function} . {\it options})) . {\it body})
\end{lisp}
The {\it body} is executed in a dynamic context within which the {\it function}
will be called whenever 
\cd{(invoke-restart '{\it name})} is executed. The {\it options} are keyword-style and are
used to pass information such as that provided with the
\cd{:report} keyword in \cdf{restart-case}.

A \cdf{restart-case} expands into a call to \cdf{restart-bind} where the function
simply does an unconditional transfer of control to a particular body
of code, passing along ``argument'' information in a structured way.

It is also possible to write restarts that do not transfer control. Such
restarts may be useful in implementing various special commands for the
debugger that are of interest only in certain situations. For example,
one might imagine a situation where file space was exhausted and the
following was done in an attempt to free space in directory \cdf{dir}:
\begin{lisp}
(restart-bind ((nil \#'(lambda () (expunge-directory dir)) \\
~~~~~~~~~~~~~~~~~~~~:report-function \\
~~~~~~~~~~~~~~~~~~~~~~\#'(lambda (stream) \\
~~~~~~~~~~~~~~~~~~~~~~~~~~(format stream "Expunge {\Xtilde}A." \\
~~~~~~~~~~~~~~~~~~~~~~~~~~~~~~~~~~(directory-namestring dir))))) \\
~~(cerror "Try this file operation again." \\
~~~~~~~~~~'directory-full :directory dir))
\end{lisp}
In this case, the debugger might be entered and the user could first
perform the expunge (which would not transfer control from the debugger
context) and then retry the file operation:
\begin{lisp}
Lisp> (open "FOO" :direction :output) \\
Error: The directory PS:<JDOE> is full. \\
To continue, type :CONTINUE followed by an option number: \\
~1: Try this file operation again. \\
~2: Expunge PS:<JDOE>. \\
~3: Return to Lisp Toplevel. \\
Debug> :continue 2 \\
Expunging PS:<JDOE> ... 3 records freed. \\
Debug> :continue 1 \\
~\EV\ \#<OUTPUT-STREAM "PS:<JDOE>FOO.LSP" 2323473>
\end{lisp}


\subsection{Interactive Condition Handling}

When a program does not know how to continue, and no active handler is able to
advise it, the ``interactive condition handler,'' or ``debugger,'' can be
entered. This happens implicitly through the use of functions such as \cdf{error}
and \cdf{cerror}, or explicitly through the use of the function \cdf{invoke-debugger}.

The interactive condition handler never returns directly; it returns only
through structured non-local transfer of control to specially defined restart
points that can be set up either by the system or by user code. The
mechanisms that support the establishment of such structured restart points
for portable code are outlined
in sections~\ref{RESTARTS} through~\ref{LAST-RESTARTS-SECTION}.

Actually, implementations may also provide extended debugging facilities that
allow return from arbitrary stack frames. Although such commands are frequently
useful in practice, their effects are implementation-dependent because they
violate the Common Lisp program abstraction. The effect of using such
commands is undefined with respect to Common Lisp.


\subsection{Serious Conditions}

The \cdf{ignore-errors} macro will trap conditions of type \cdf{error}. There are,
however, conditions that are not of type \cdf{error}.

Some conditions are not considered errors but are still very serious, so
we call them {\it serious conditions} and we use the type \cdf{serious-condition} to
represent them. Conditions such as those that might be signaled for
``stack overflow'' or ``storage exhausted'' are in this category.

The type \cdf{error} is a subtype of \cdf{serious-condition}, and it would technically
be correct to use the term ``serious condition'' to refer to all serious
conditions whether errors or not. However, normally we use the term 
``serious condition'' to refer to things of type \cdf{serious-condition} but not
of type \cdf{error}.

The point of the distinction between errors and other serious conditions
is that some conditions are known to occur for reasons that are beyond the
scope of Common Lisp to specify clearly. For example, we know that a stack
will generally be used to implement function calling, and we know that stacks
tend to be of finite size and are prone to overflow. Since the available
stack size may vary from implementation to implementation, from session
to session, or from function call to function call, it would be confusing
to have expressions such as \cd{(ignore-errors (+~a~b))} return a number sometimes
and \cdf{nil} other times if \cdf{a} and \cdf{b} were always bound to numbers and the stack
just happened to overflow on a particular call. For this reason, only
conditions of type \cdf{error} and not all conditions of type \cdf{serious-condition}
are trapped by \cdf{ignore-errors}. To trap other conditions, a lower-level
facility must be used (such as \cdf{handler-bind} or \cdf{handler-case}).

By convention, the function \cdf{error} is preferred over \cdf{signal} to signal conditions
of type \cdf{serious-condition} (including those of type \cdf{error}). It is the use of
the function \cdf{error}, and not the type of the condition being signaled, that
actually causes the debugger to be entered.

\beforenoterule
\begin{incompatibility}
The Common Lisp Condition System differs from that of Zeta{}lisp in this respect.
In Zetalisp the debugger is entered for an unhandled signal if the \cdf{error}
function is used {\it or} if the condition is of type \cdf{error}.
\end{incompatibility}
\afternoterule

\subsection{Non-Serious Conditions}

Some conditions are neither errors nor serious conditions. They are signaled
to give other programs a chance to intervene, but if no action is taken,
computation simply continues normally.

For example, an implementation might choose to signal a non-serious (and
implementation-dependent) condition
called \cdf{end-of-line} when output reaches the last character position on a line
of character output. In such an implementation, the signaling of this
condition might allow a convenient way for other programs to intervene,
producing output that is truncated at the end of a line.

By convention, the function \cdf{signal} is used to signal conditions that are not
serious. It would be possible to signal serious conditions using \cdf{signal}, and
the debugger would not be entered if the condition went unhandled.  However,
by convention,
handlers will generally tend to assume that serious conditions and errors
were signaled by calling the \cdf{error} function (and will therefore
force entry to the interactive condition handler) and that they should
work to avoid this.


\subsection{Condition Types}

Some types of conditions are predefined by the system. All types of conditions
are subtypes of \cdf{condition}. That is, \cd{(typep~{\it x} 'condition)} is true if
and only if the value of {\it x} is a condition. 

Implementations supporting multiple (or non-hierarchical) type inheritance
are expressly permitted to exploit multiple inheritance in the tree of
condition types as implementation-dependent extensions, as long as such
extensions are compatible with the specifications in this chapter.
[X3J13 voted in March 1989 \issue{ZLOS-CONDITIONS}
to integrate the Condition System and the Object System,
so multiple inheritance is always available for condition types.---GLS]

In order to avoid problems in portable code that runs both in systems with
multiple type inheritance and in systems without it, programmers are explicitly
warned that while all correct Common Lisp implementations will ensure that
\cd{(typep~{\it c} 'condition)}
is true for all conditions {\it c} (and all subtype relationships indicated in this
chapter will also be true), it should {\it not} be assumed that two condition
types specified to be subtypes of the same third type are disjoint.
(In some cases,
disjoint subtypes are identified explicitly, but such disjointness is not to be assumed by
default.)  For example, it follows from the subtype descriptions contained in
this chapter that in all implementations
\cd{(typep~{\it c}~'control-error)} implies \cd{(typep~{\it c}~'error)},
but note that
\cd{(typep~{\it c}~'control-error)} does {\it not}
imply \cd{(not~(typep~{\it c}~'cell-error))}.


\subsection{Signaling Conditions}

When a condition is signaled, the system tries to locate the most appropriate
handler for the condition and to invoke that handler.

Handlers are established dynamically using \cdf{handler-bind} or abstractions built
on \cdf{handler-bind}.

If an appropriate handler is found, it is called. In some circumstances, 
the handler may {\it decline} simply by returning without performing a 
non-local transfer of control. In such cases, the search for an 
appropriate handler is picked up where it left off, as if the called 
handler had never been present.

If no handler is found, or if all handlers that were found decline,
\cdf{signal} returns \cdf{nil}.

Although it follows from the description above, it is perhaps worth noting
explicitly that the lookup procedure described here will prefer a general 
but more (dynamically) local handler over a specific but less (dynamically)
local handler. Experience with existing condition systems suggests that
this is a reasonable approach and works adequately in most situations. 
Some care should be taken when binding handlers for very general kinds of
conditions, such as is done in \cdf{ignore-errors}. Often, binding for a more
specific condition type than \cdf{error} is more appropriate.


\subsection{Resignaling Conditions}

[The contents of this section are still a subject of some debate within X3J13.
The reader may wish to take this section with a grain of salt.---GLS]

Note that signaling a condition has no side effect on that condition, and
that there is no dynamic state contained in a condition object. As such, it
may at times be reasonable and appropriate to consider caching condition
objects for repeated use, re-signaling conditions from within handlers,
or saving conditions away somewhere and re-signaling them later.

For example, it may be desirable for the system to pre-allocate objects of type
\cdf{storage-condition} so that they can be signaled when needed without
attempting to allocate more storage.


\subsection{Condition Handlers}
\label{CONDITION-HANDLERS}

A {\it handler} is a function of one argument, the condition to be handled. The
handler may inspect the object
to be sure it is ``interested'' in handling the condition.

A handler is executed in the dynamic context of the signaler, except that the
set of available condition handlers will have been rebound to the value that
was active at the time the condition handler was made active. The intent of
this is to prevent infinite recursion because of errors in a condition handler.

After inspecting the condition, the handler should take one of the following
actions:
\begin{itemize}
  \item
    It might {\it decline} to handle the condition (by simply returning). When
    this happens, the returned values are ignored and the effect is the same
    as if the handler had been invisible to the mechanism seeking to find a
    handler. The next handler in line will be tried, or if no such handler
    exists, the condition will go unhandled.

  \item
    It might {\it handle} the condition (by performing some non-local transfer
    of control). This may be done either primitively using \cdf{go}, \cdf{return}, or \cdf{throw},
    or more abstractly using a function such as \cdf{abort} or \cdf{invoke-restart}.

  \item
    It might signal another condition.

  \item
    It might invoke the interactive debugger.
\end{itemize}
In fact, the latter two actions (signaling another condition or entering the
debugger) are really just ways of putting off the decision to either handle
or decline, or trying to get someone else to make such a decision. Ultimately,
all a handler can do is to handle or decline to handle.

\subsection{Printing Conditions}

When \cd{*print-escape*} is \cdf{nil} (for example,
when the \cdf{princ} function or the \cd{{\Xtilde}A}
directive is used with \cdf{format}), the report method for the condition will be invoked. This will
be done automatically by functions such as \cdf{invoke-debugger}, \cdf{break}, and \cdf{warn},
but there may still be situations in which it is desirable to have a
condition report under explicit user control. For example,
\begin{lisp}
(let ((form '(open "nosuchfile"))) \\
~~(handler-case (eval form) \\
~~~~(serious-condition (c) \\
~~~~~~(format t "{\Xtilde}\&Evaluation of {\Xtilde}S failed:{\Xtilde}\%{\Xtilde}A" form c))))
\end{lisp}
might print something like
\begin{lisp}
Evaluation of (OPEN "nosuchfile") failed: \\
The file "nosuchfile" was not found.
\end{lisp}
Some suggestions about the form of text typed by report methods:
\begin{itemize}
 \item
    The message should generally be a complete sentence, beginning with a
   capital letter and ending with appropriate punctuation (usually a period).

 \item
    The message should {\it not} include any introductory text such as ``\cd{Error:}''
   or ``\cd{Warning:}'' and should not be followed by a trailing newline. Such
   text will be added as may be appropriate to context by the routine invoking
   the report method.

 \item
    Except where unavoidable, the tab character (which is only semi-standard anyway)
    should not be used in
   error messages. Its effect may vary from one implementation to another and may
   cause problems even within an implementation because it may do different
   things depending on the column at which the error report begins.

 \item
    Single-line messages are preferred, but newlines in the middle of long
   messages are acceptable.

 \item
   If any program (for example, the debugger) displays messages indented from the
   prevailing left margin (for example, indented seven spaces because they
   are prefixed by the seven-character herald ``\cd{Error:~}''), then that program
   will take care of inserting the appropriate indentation into the extra
   lines of a multi-line error message. Similarly, a program that prefixes
   error messages with semicolons so that they appear to be comments should
   take care of inserting a semicolon at the beginning of each line in a
   multi-line error message. (These rules are important because, even within
   a single implementation, there may be more than one program that presents
   error messages to the user, and they may use different styles of
   presentation. The caller of \cdf{error} cannot anticipate all such possible
   styles, and so it is incumbent upon the presenter of the message to make
   any necessary adjustments.)
\end{itemize}
[Note: These recommendations expand upon those in section~\ref{ERROR-SIGNALLING-FUNCTIONS}.---GLS]

When \cd{*print-escape*} is not \cdf{nil}, the object should print in some useful (but
usually fairly abbreviated) fashion according to the style of the
implementation. It is not expected that a condition will be printed in a form
suitable for \cdf{read}. Something like \cd{\#<ARITHMETIC-ERROR~1734>}
is fine.


X3J13 voted in March 1989 \issue{ZLOS-CONDITIONS} to integrate the
Condition System and the Object System.
In the original Condition System proposal,
no function was provided for directly accessing or setting the printer for
a condition type, or for invoking it; the techniques described above were
the sole interface to reporting.  The vote specified that, in CLOS terms,
condition reporting is mediated through the \cdf{print-object}
method for the condition type (that is, class) in question, with \cd{*print-escape*}
bound to \cdf{nil}.  Specifying \cd{(:report {\it fn})} to
\cdf{define-condition} when defining
condition type {\it C} is equivalent to a separate method definition:
\begin{lisp}
(defmethod print-object ((x {\it C}) stream) \\*
~~(if *print-escape* \\*
~~~~~~(call-next-method) \\*
~~~~~~(funcall \#'{\it fn} x stream)))
\end{lisp}
Note that the method uses {\it fn} to print the condition
only when \cd{*print-escape*} has the value \cdf{nil}.



\section{Program Interface to the Condition System}

This section describes functions, macros, variables, and condition
types associated with the Common Lisp Condition System.


\subsection{Signaling Conditions}
\label{SIGNALLING-CONDITIONS}


The functions in this section provide various mechanisms
for signaling warnings, breaks, continuable errors, and fatal errors.


\begin{defun}[Function]
error datum &rest arguments

   [This supersedes the description of \cdf{error}
   given in section~\ref{ERROR-SIGNALLING-FUNCTIONS}.---GLS]

  Invokes the signal facility on a condition. If the condition is not handled,
  \cd{(invoke-debugger {\it condition})} is executed. As a consequence of calling 
  \cdf{invoke-debugger}, \cdf{error} never directly returns to its caller; the only exit from this
  function can come by non-local transfer of control in a handler or by use of
  an interactive debugging command.

  If {\it datum} is a condition, then that condition is used directly. 
  In this case, it is an error for the list of {\it arguments} to be non-empty;
  that is, \cdf{error} must have been called with exactly one argument, the condition.

  If {\it datum} is a condition type (a class or class name), then the condition used is effectively the result
  of \cd{(apply \#'make-condition {\it datum} {\it arguments})}.

  If {\it datum} is a string, then the condition used is effectively the result of
\begin{lisp}
(make-condition 'simple-error \\*
~~~~~~~~~~~~~~~~:format-string {\it datum} \\*
~~~~~~~~~~~~~~~~:format-arguments {\it arguments})
\end{lisp}
\end{defun}

\begin{defun}[Function]
cerror continue-format-string datum &rest arguments

   [This supersedes the description of \cdf{cerror}
   given in section~\ref{ERROR-SIGNALLING-FUNCTIONS}.---GLS]

  The function \cdf{cerror}
  invokes the error facility on a condition. If the condition is not handled,
  \cd{(invoke-debugger {\it condition})} is executed. While signaling is going on,
  and
  while control is in the debugger (if it is reached), it is possible to continue
  program execution (thereby returning from the call to \cdf{cerror})
  using the \cdf{continue} restart.

  If {\it datum} is a condition, then that condition is used directly. 
  In this case, the list of {\it arguments} need not be empty,
  but will be used only with the {\it continue-format-string}
  and will not be used to initialize {\it datum}.

  If {\it datum} is a condition type (a class or class name), then the condition used is effectively the result
  of \cd{(apply \#'make-condition {\it datum} {\it arguments})}.

  If {\it datum} is a string, then the condition used is effectively the result of
\begin{lisp}
(make-condition 'simple-error \\*
~~~~~~~~~~~~~~~~:format-string {\it datum} \\*
~~~~~~~~~~~~~~~~:format-arguments {\it arguments})
\end{lisp}

  The {\it continue-format-string} must be a string.
  Note that if {\it datum} is not a 
  string, then the format arguments used by the {\it continue-format-string} will
  still be the list of {\it arguments} (which is in keyword format if {\it datum} is a condition
  type). In this case, some care may be necessary to set up the
  {\it continue-format-string} correctly. The \cdf{format} directive \cd{{\Xtilde}*},
  which ignores and skips over \cdf{format} arguments,
  may be particularly 
  useful in this situation.

  The value returned by \cdf{cerror} is \cdf{nil}.
\end{defun}

\begin{defun}[Function]
signal datum &rest arguments

  Invokes the signal facility on a condition. If the condition is not handled,
  \cdf{signal} returns \cdf{nil}.

  If {\it datum} is a condition, then that condition is used directly. 
  In this case, it is an error for the list of {\it arguments} to be non-empty;
  that is, \cdf{error} must have been called with exactly one argument, the condition.

  If {\it datum} is a condition type (a class or class name), then the condition used is effectively the result
  of \cd{(apply \#'make-condition {\it datum} {\it arguments})}.

  If {\it datum} is a string, then the condition used is effectively the result of
\begin{lisp}
(make-condition 'simple-error \\*
~~~~~~~~~~~~~~~~:format-string {\it datum} \\*
~~~~~~~~~~~~~~~~:format-arguments {\it arguments})
\end{lisp}

  Note that if \cd{(typep {\it condition} *break-on-signals*)} is true, then the debugger
  will be entered prior to beginning the process of signaling. The \cdf{continue}
  restart function may be used to continue with the signaling process;
the restart is associated with the signaled condition as if by
use of \cdf{with-condition-restarts}.
This is true
  also for all other functions and macros that signal conditions, such
  as \cdf{warn}, \cdf{error}, \cdf{cerror}, \cdf{assert}, and \cdf{check-type}.

During the dynamic extent of a call to \cdf{signal} with a
     particular condition, the effect of calling \cdf{signal} again on that
     condition object for a distinct abstract event is not defined.
     For example, although a handler {\it may} resignal a condition in order to
     allow outer handlers first shot at handling the condition, two
     distinct asynchronous keyboard events must not signal an the same (\cdf{eq}) condition
     object at the same time.

  For further details about signaling and handling, see the discussion of
  condition handlers in section~\ref{CONDITION-HANDLERS}.
\end{defun}


\begin{defun}[Variable]
*break-on-signals*

  This variable is intended primarily for use when the user is debugging
  programs that do signaling.
  The value of \cd{*break-on-signals*} should be suitable as a second argument to
  \cdf{typep}, that is, a type or type specifier.

  When \cd{(typep {\it condition} *break-on-signals*)} is true, then calls to
  \cdf{signal} (and to other advertised functions such as \cdf{error} that
  implicitly call \cdf{signal}) will enter the debugger prior to signaling
  that {\it condition}. The \cdf{continue} restart may be used to continue with
  the normal signaling process;
the restart is associated with the signaled condition as if by
use of \cdf{with-condition-restarts}.

  Note that \cdf{nil} is a valid type specifier.  If the value of
  \cd{*break-on-signals*} is \cdf{nil}, then \cdf{signal} will never
  enter the debugger in this implicit manner.

  When setting this variable, the user is encouraged to choose the
  most restrictive specification that suffices. Setting this flag
  effectively violates the modular handling of condition signaling
  that this chapter seeks to establish. Its complete effect may be
  unpredictable in some cases, since the user may not be aware of the
  variety or number of calls to \cdf{signal} that are used in programs
  called only incidentally.

  By default---and certainly in any ``production'' use---the value
  of this variable should be \cdf{nil}, both for reasons of performance and
  for reasons of modularity and abstraction.

\begin{newer}
X3J13 voted in March 1989
\issue{BREAK-ON-WARNINGS-OBSOLETE}
to remove \cd{*break-on-warnings*} from the language;
\cd{*break-on-signals*} offers all the power of
  \cd{*break-on-warnings*} and more.
\end{newer}

\beforenoterule
  \begin{incompatibility}
  This variable is similar to the Zetalisp
  variable \cdf{trace-conditions} except for the obvious difference that
  \cd{zl:trace-conditions} takes a type or list of types while
  \cd{*break-on-signals*} takes a single type specifier.

  [There is no loss of generality in Common Lisp
  because the \cdf{or} type specifier may be used to indicate that
  any of a set of conditions should enter the debugger.---GLS]
  \end{incompatibility}
\afternoterule
\end{defun}


\subsection{Assertions}
\label{CONDITION-ASSERTIONS}


These facilities are designed to make it convenient for the user
to insert error checks into code.

\begin{defmac}
check-type place typespec [string]

   [This supersedes the description of \cdf{check-type}
   given in section~\ref{SPECIALIZED-ERROR-SIGNALLING}.---GLS]

  A \cdf{check-type} form signals an error of type
  \cdf{type-error} if the contents of {\it place} are not of the
  desired type.

  If a condition is signaled, handlers of this condition can use the
  functions \cdf{type-error-datum} and \cdf{type-error-expected-type} to access the
  contents of {\it place} and the {\it typespec}, respectively.

  This function can return only if the \cdf{store-value} restart is invoked, either
  explicitly from a handler or implicitly as one of the options offered by the
  debugger.
The restart is associated with the signaled condition as if by
use of \cdf{with-condition-restarts}.

  If \cdf{store-value} is called, \cdf{check-type} will store the new value that is
  the argument to \cdf{store-value} (or that is prompted for interactively by
  the debugger) in {\it place} and start over, checking the type of the new value
  and signaling another error if it is still not the desired type. Subforms
  of {\it place} may be evaluated multiple times because of the implicit loop
  generated. \cdf{check-type} returns \cdf{nil}.

  The {\it place} must be a generalized variable reference acceptable to \cdf{setf}. The
  {\it typespec} must be a type specifier; it is not evaluated.  The \cdf{string} should
  be an English description of the type, starting with an indefinite article
  (``a'' or ``an''); it is evaluated. If the {\it string} is not supplied, it is computed
  automatically from the {\it typespec}. (The optional {\it string} argument is allowed 
  because some applications of \cdf{check-type} may require a more specific
  description of what is wanted than can be generated automatically from the
  type specifier.)

  The error message will mention the {\it place}, its contents, and the desired type.

  \beforenoterule
  \begin{implementation}
  An implementation may choose to generate a somewhat
  differently worded error message if it recognizes that {\it place} is of a
  particular form, such as one of the arguments to the function that called
  \cdf{check-type}.
  \end{implementation}
\afternoterule

\begin{lisp}
Lisp> (setq aardvarks '(sam harry fred)) \\*
~\EV\ (SAM HARRY FRED) \\
Lisp> (check-type aardvarks (array * (3))) \\*
Error: The value of AARDVARKS, (SAM HARRY FRED), \\*
~~~~~~~is not a 3-long array. \\
To continue, type :CONTINUE followed by an option number: \\*
~1: Specify a value to use instead. \\*
~2: Return to Lisp Toplevel. \\
Debug> :continue 1 \\*
Use Value: \#(sam fred harry) \\*
~\EV\ NIL \\
Lisp> aardvarks \\*
~\EV\ \#<ARRAY-3 13571> \\
Lisp> (map 'list \#'identity aardvarks) \\*
~\EV\ (SAM FRED HARRY) \\
Lisp> (setq aacount 'foo) \\*
~\EV\ FOO \\
Lisp> (check-type aacount (integer 0 *) "a non-negative integer") \\*
Error: The value of AACOUNT, FOO, is not a non-negative integer. \\*
To continue, type :CONTINUE followed by an option number: \\*
~1: Specify a value to use instead. \\*
~2: Return to Lisp Toplevel. \\
Debug> :continue 2 \\*
Lisp> 
\end{lisp}

\beforenoterule
  \begin{incompatibility}
  In Zetalisp, the equivalent facility is called \cdf{check-arg-type}.
  \end{incompatibility}
\afternoterule
\end{defmac}


\begin{defmac}
assert test-form [({place}*) [datum {argument}*]]

   [This supersedes the description of \cdf{assert}
   given in section~\ref{SPECIALIZED-ERROR-SIGNALLING}.---GLS]

  An \cdf{assert} form signals an error if the value of the {\it test-form} is \cdf{nil}.
  Continuing from this
  error using the \cdf{continue} restart will allow the user to alter the values
  of some variables, and \cdf{assert} will then start over, evaluating the {\it test-form}
  again.
(The restart is associated with the signaled condition as if by
use of \cdf{with-condition-restarts}.)
\cdf{assert} returns \cdf{nil}.

  The {\it test-form} may be any form. Each {\it place} (there may be any number of them, 
  or none) must be a generalized variable reference acceptable to \cdf{setf}.
  These should be variables on which {\it test-form} depends, whose values
  may sensibly be changed by the user in attempting to correct the error.
  Subforms of each {\it place} are evaluated only if an error is signaled, and
  may be re-evaluated if the error is re-signaled (after continuing without
  actually fixing the problem).

  The {\it datum} and {\it argument\/}s are evaluated only if an error is to be
  signaled, and re-evaluated if the error is to be signaled again.

  If {\it datum} is a condition, then that condition is used directly. 
  In this case, it is an error to specify any {\it argument\/}s.

  If {\it datum} is a condition type (a class or class name), then the condition used is effectively the result
  of \cd{(apply \#'make-condition {\it datum} (list \Mstar{argument}))}.

  If {\it datum} is a string, then the condition used is effectively the result of
\begin{lisp}
(make-condition 'simple-error \\*
~~~~~~~~~~~~~~~~:format-string {\it datum} \\*
~~~~~~~~~~~~~~~~:format-arguments (list \Mstar{argument}))
\end{lisp}

  If {\it datum} is omitted, then a condition of type \cdf{simple-error} is 
  constructed using the {\it test-form} as data. For example, the following
  might be used:
\begin{lisp}
(make-condition 'simple-error \\
~~:format-string "The assertion {\Xtilde}S failed." \\
~~:format-arguments '({\it test-form}))
\end{lisp}
Note that the {\it test-form} itself, and not its value, is used as the format argument.

\beforenoterule
  \begin{implementation}
  The debugger need not include the {\it test-form} in
  the error message, and any {\it places} should not be included in the message,
  but they should be made available for the user's perusal. If the user
  gives the ``continue'' command, an opportunity should be presented
  to alter the values of any or all of the references. The details of this
  depend on the implementation's style of user interface, of course.
  \end{implementation}
\afternoterule

Here is an example of the use of \cdf{assert}:
\begin{lisp}
(setq x (make-array '(3 5) :initial-element 3)) \\
(setq y (make-array '(3 5) :initial-element 7)) \\
 \\
(defun matrix-multiply (a b) \\*
~~(let ((*print-array* nil)) \\*
~~~~(assert (and (= (array-rank a) (array-rank b) 2) \\*
~~~~~~~~~~~~~~~~~(= (array-dimension a 1) \\*
~~~~~~~~~~~~~~~~~~~~(array-dimension b 0))) \\*
~~~~~~~~~~~~(a b) \\*
~~~~~~~~~~~~"Cannot multiply {\Xtilde}S by {\Xtilde}S." a b) \\*
~~~~(really-matrix-multiply a b))) \\
 \\
(matrix-multiply x y) \\
Error: Cannot multiply \#<ARRAY-3-5 12345> by \#<ARRAY-3-5 12364>. \\*
To continue, type :CONTINUE followed by an option number: \\*
~1: Specify new values. \\*
~2: Return to Lisp Toplevel. \\
Debug> :continue 1 \\*
Value for A: x \\*
Value for B: (make-array '(5 3) :initial-element 6) \\
~\EV \#2A(\=(54 54 54 54 54) \\
\>(54 54 54 54 54) \\*
\>(54 54 54 54 54) \\*
\>(54 54 54 54 54) \\*
\>(54 54 54 54 54))
\end{lisp}
\end{defmac}


\subsection{Exhaustive Case Analysis}
\label{EXHAUSTIVE-CASE-ANALYSIS-CONDITIONS}

The syntax for \cdf{etypecase} and \cdf{ctypecase} is the same as for \cdf{typecase}, except
that no \cdf{otherwise} clause is permitted. Similarly, the syntax for \cdf{ecase} and
\cdf{ccase} is the same as for \cdf{case} except for the \cdf{otherwise} clause.

\cdf{etypecase} and \cdf{ecase} are similar to \cdf{typecase} and \cdf{case}, respectively, but signal
a non-continuable error rather than returning \cdf{nil} if no clause is selected.

\cdf{ctypecase} and \cdf{ccase} are also similar to \cdf{typecase} and \cdf{case}, respectively,
but signal a
continuable error if no clause is selected.

\begin{defmac}
etypecase keyform {(type {\,form}*)}*

   [This supersedes the description of \cdf{etypecase}
   given in section~\ref{EXHAUSTIVE-CASE-ANALYSIS}.---GLS]

  This control construct is similar to \cdf{typecase}, but no explicit \cdf{otherwise}
  or \cdf{t} clause is permitted. If no clause is satisfied, \cdf{etypecase} signals 
  an error (of type \cdf{type-error}) with a message constructed from the clauses.
  It is not permissible to continue from this error. To supply an error
  message, the user should use \cdf{typecase} with an \cdf{otherwise} clause containing 
  a call to \cdf{error}. The name of this function stands for ``exhaustive type
  case'' or ``error-checking type case.''

  Example:
\begin{lisp}
Lisp> (setq x 1/3) \\*
~\EV\ 1/3 \\
Lisp> (etypecase x \\*
~~~~~~~~(integer (* x 4)) \\*
~~~~~~~~(symbol~(symbol-value x))) \\
Error: The value of X, 1/3, is neither an integer nor a symbol. \\*
To continue, type :CONTINUE followed by an option number: \\*
~1: Return to Lisp Toplevel. \\*
Debug>
\end{lisp}
\end{defmac}

\begin{defmac}
ctypecase keyplace {(type {\,form}*)}*

   [This supersedes the description of \cdf{ctypecase}
   given in section~\ref{EXHAUSTIVE-CASE-ANALYSIS}.---GLS]

  This control construct is similar to \cdf{typecase}, but no explicit 
  \cdf{otherwise} or \cdf{t} clause is permitted.

  The {\it keyplace} must be a generalized variable reference acceptable to \cdf{setf}.
  If no clause is satisfied, \cdf{ctypecase} signals an error (of type \cdf{type-error})
  with a message constructed from the clauses. This error may be continued
  using the \cdf{store-value} restart. The argument to \cdf{store-value} is stored in
  {\it keyplace} and then \cdf{ctypecase} starts over, making the type tests again. 
  Subforms of {\it keyplace} may be evaluated multiple times. If the \cdf{store-value}
  restart is invoked interactively, the user will be prompted for the value
  to be used.
  
  The name of this function is mnemonic for ``continuable (exhaustive) 
  type case.''


  Example:
\begin{lisp}
Lisp> (setq x 1/3) \\*
~\EV\ 1/3 \\
Lisp> (ctypecase x \\*
~~~~~~~~(integer (* x 4)) \\*
~~~~~~~~(symbol (symbol-value x))) \\
Error: The value of X, 1/3, is neither an integer nor a symbol. \\*
To continue, type :CONTINUE followed by an option number: \\
~1: Specify a value to use instead. \\*
~2: Return to Lisp Toplevel. \\
Debug> :continue 1 \\*
Use value: 3.7 \\
Error: The value of X, 3.7, is neither an integer nor a symbol. \\*
To continue, type :CONTINUE followed by an option number: \\*
~1: Specify a value to use instead. \\*
~2: Return to Lisp Toplevel. \\
Debug> :continue 1 \\*
Use value: 12 \\*
~\EV\ 48
\end{lisp}
\end{defmac}


\begin{defmac}
ecase keyform {({({key}*) | key} {\,form}*)}*

   [This supersedes the description of \cdf{ecase}
   given in section~\ref{EXHAUSTIVE-CASE-ANALYSIS}.---GLS]

  This control construct is similar to \cdf{case}, but no explicit \cdf{otherwise} or \cdf{t}
  clause is permitted. If no clause is satisfied, \cdf{ecase} signals an error
  (of type \cdf{type-error}) with a message constructed from the clauses. It is not
  permissible to continue from this error. To supply an error message, the
  user should use \cdf{case} with an \cdf{otherwise} clause containing a call to \cdf{error}.
  The name of this function stands for ``exhaustive case'' or ``error-checking
  case.''

Example:
\begin{lisp}
Lisp> (setq x 1/3) \\*
~\EV\ 1/3 \\
Lisp> (ecase x \\*
~~~~~~~~(alpha (foo)) \\*
~~~~~~~~(omega (bar)) \\*
~~~~~~~~((zeta phi) (baz))) \\
Error: The value of X, 1/3, is not ALPHA, OMEGA, ZETA, or PHI. \\*
To continue, type :CONTINUE followed by an option number: \\*
~1: Return to Lisp Toplevel. \\*
Debug>
\end{lisp}
\end{defmac}

\begin{defmac}
ccase keyplace {({({key}*) | key} {\,form}*)}*

   [This supersedes the description of \cdf{ccase}
   given in section~\ref{EXHAUSTIVE-CASE-ANALYSIS}.---GLS]

  This control construct is similar to \cdf{case}, but no explicit \cdf{otherwise} or
  \cdf{t} clause is permitted.

  The {\it keyplace} must be a generalized variable reference acceptable to \cdf{setf}.
  If no clause is satisfied, \cdf{ccase} signals an error (of type \cdf{type-error})
  with a message constructed from the clauses. This error may be continued
  using the \cdf{store-value} restart. The argument to \cdf{store-value} is stored in
  {\it keyplace} and then \cdf{ccase} starts over, making the type tests again. Subforms
  of {\it keyplace} may be evaluated multiple times. If the \cdf{store-value} restart is
  invoked interactively, the user will be prompted for the value to be used.

  The name of this function is mnemonic for ``continuable (exhaustive) case.''

\beforenoterule
\begin{implementation}
  The \cdf{type-error} signaled by \cdf{ccase} and \cdf{ecase} is free to
  choose any representation of the acceptable argument type that it wishes
  for placement in the expected-type slot. It will always work to use type
  \cd{(member . {\it keys})}, but in some cases it may be more efficient, for example,
  to use a type that represents an integer subrange or a type composed using the
  \cdf{or} type specifier.
\end{implementation}
\afternoterule
\end{defmac}


\subsection{Handling Conditions}

These macros allow a program to gain control when a condition is signaled.

\begin{defmac}
handler-case expression {(typespec ([var]) {\,form}*)}*

  Executes the given {\it expression} in a context where various specified handlers are active.

  Each {\it typespec} may be any type specifier. If during the execution of the \cdf{expression}
  a condition is signaled for which there is an appropriate clause---that is, one
  for which \cd{(typep {\it condition} '{\it typespec})} is true---and if there is no intervening
  handler for conditions of that type, then control is transferred to the body
  of the relevant clause (unwinding the dynamic state appropriately in the
  process) and the given variable \cdf{var} is bound to the condition that was signaled. If
  no such condition is signaled and the computation runs to completion, then
  the values resulting from the \cdf{expression} are returned by the \cdf{handler-case} form.

  If more than one case is provided, those cases are made accessible in
  parallel. That is, in
\begin{lisp}
(handler-case {\it expression} \\*
~~({\it type\SU{1}} ({\it var\SU{1}}) {\it form\SU{1}}) \\*
~~({\it type\SU{2}} ({\it var\SU{2}}) {\it form\SU{2}}))
\end{lisp}
  if the first clause (containing {\it form\SU{1}}) has been selected, the handler
  for the second is no longer visible (and vice versa).

  The cases are searched sequentially from top to bottom. If a signaled condition
  matches more than one case (possible if there is type
  overlap) the earlier of the two cases will be selected.


\penalty-10000 %required

  If the variable {\it var} is not needed, it may be omitted. That is, a clause such as
\begin{lisp}
({\it type} ({\it var}) (declare (ignore {\it var})) {\it form})
\end{lisp}
may be written using the following shorthand notation:
\begin{lisp}
({\it type} () {\it form})
\end{lisp}

  If there are no forms in a selected case, the case returns \cdf{nil}.
Note that
\begin{lisp}
(handler-case {\it expression} \\*
~~({\it type\SU{1}} ({\it var\SU{1}}) . {\it body\SU{1}}) \\*
~~({\it type\SU{2}} ({\it var\SU{2}}) . {\it body\SU{2}}) \\*
~~...)
\end{lisp}
is approximately equivalent to
\begin{lisp}
(block \#1=\#:block-1 \\*
~~(let (\#2=\#:var-2) \\*
~~~~(tagbody \\*
~~~~~~(handler-bind (({\it type\SU{1}} \pushtabs\=\#'(lambda (temp) \\*
\>~~~~(setq \#2\# temp) \\*
\>~~~~(go \#3=\#:tag-3)))\poptabs \\
~~~~~~~~~~~~~~~~~~~~~({\it type\SU{2}} \pushtabs\=\#'(lambda (temp) \\*
\>~~~~(setq \#2\# temp) \\*
\>~~~~(go \#4=\#:tag-4)))\poptabs \\*
~~~~~~~~~~~~~~~~~~~~~...) \\*
~~~~~~~~(return-from \#1\# {\it expression})) \\
~~~~~~\#3\# (return-from \#1\# (let (({\it var\SU{1}} \#2\#)) . {\it body\SU{1}})) \\*
~~~~~~\#4\# (return-from \#1\# (let (({\it var\SU{2}} \#2\#)) . {\it body\SU{2}})) \\*
~~~~~~...)))
\end{lisp}
[Note the use of ``gensyms'' such as \cd{\#:block-1}
as block names, variables, and \cdf{tagbody} tags in this example,
and the use of \cd{\#{\it n}=} and \cd{\#{\it n}\#} read-macro syntax
to indicate that the very same gensym appears in multiple places.---GLS]

  As a special case, the {\it typespec} can also be the symbol \cd{:no-error} in the last clause.
  If it is, it designates a clause that will take control if the {\it expression} returns
  normally. In that case, a completely general lambda-list may follow the symbol \cd{:no-error},
  and the arguments to which the lambda-list parameters are bound are like those for
  \cdf{multiple-value-call} on the return value of the {\it expression}.
For example,

\penalty-10000 %required

\begin{lisp}
(handler-case {\it expression} \\*
~~({\it type\SU{1}} ({\it var\SU{1}}) . {\it body\SU{1}}) \\*
~~({\it type\SU{2}} ({\it var\SU{2}}) . {\it body\SU{2}}) \\*
~~... \\*
~~({\it type\SU{\hbox{\scriptsize\it n}}} ({\it var\SU{\hbox{\scriptsize\it n}}}) . {\it body\SU{\hbox{\scriptsize\it n}}}) \\*
~~(:no-error ({\it nvar\SU{1}} {\it nvar\SU{2}} ... {\it nvar\SU{\hbox{\scriptsize\it m}}}) . {\it nbody}))
\end{lisp}
is approximately equivalent to
\begin{lisp}
(block \#1=\#:error-return \\*
~~(multiple-value-call \#'(lambda ({\it nvar\SU{1}} {\it nvar\SU{2}} ... {\it nvar\SU{\hbox{\scriptsize\it m}}}) . {\it nbody}) \\*
~~~~(block \#2=\#:normal-return \\*
~~~~~~(return-from \#1\# \\*
~~~~~~~~(handler-case (return-from \#2\# {\it expression}) \\*
~~~~~~~~~~({\it type\SU{1}} ({\it var\SU{1}}) . {\it body\SU{1}}) \\*
~~~~~~~~~~({\it type\SU{2}} ({\it var\SU{2}}) . {\it body\SU{2}}) \\*
~~~~~~~~~~... \\*
~~~~~~~~~~({\it type\SU{\hbox{\scriptsize\it n}}} ({\it var\SU{\hbox{\scriptsize\it n}}}) . {\it body\SU{\hbox{\scriptsize\it n}}}))))))
\end{lisp}



  Examples of the use of \cdf{handler-case}:
\begin{lisp}
(handler-case (/ x y) \\*
~~(division-by-zero () nil)) \\
 \\
(handler-case (open *the-file* :direction :input) \\*
~~(file-error (condition) (format t "{\Xtilde}\&Fooey: {\Xtilde}A{\Xtilde}\%" condition))) \\
 \\
(handler-case (some-user-function) \\*
~~(file-error (condition) condition) \\*
~~(division-by-zero () 0) \\*
~~((or unbound-variable undefined-function) () 'unbound)) \\
 \\
(handler-case (intern x y) \\*
~~(error (condition) condition) \\*
~~(:no-error (symbol status) \\*
~~~~(declare (ignore symbol)) \\*
~~~~status))
\end{lisp}
\end{defmac}

\begin{defmac}
ignore-errors {\,form}*

  Executes its body in a context that handles conditions of type \cdf{error} by
  returning control to this form. If no such condition is signaled, any
  values returned by the last form are returned by \cdf{ignore-errors}. Otherwise,
  two values are returned: \cdf{nil} and the \cdf{error} condition that was signaled.

\cdf{ignore-errors} could be defined by
\begin{lisp}
(defmacro ignore-errors (\&body forms) \\*
~~{\Xbq}(handler-case (progn ,{\Xatsign}forms) \\*
~~~~~(error (c) (values nil c)))
\end{lisp}
\end{defmac}


\begin{defmac}
handler-bind ({(typespec handler)}*) {\,form}*

  Executes body in a dynamic context where the given handler bindings 
  are in effect.
  Each {\it typespec} may be any type specifier.
  Each {\it handler} form should evaluate to a function to be used to handle conditions 
  of the given type(s) during execution of the {\it form\/}s. This function should
  take a single argument, the condition being signaled.

  If more than one binding is specified, the bindings are searched 
  sequentially from top to bottom in search of a match (by visual analogy
  with \cdf{typecase}). If an appropriate {\it typespec} is found, the associated handler 
  is run in a context where none of the handler bindings are visible (to avoid
  recursive errors). For example, in the case of
\begin{lisp}
(handler-bind ((unbound-variable \#'(lambda ...)) \\*
~~~~~~~~~~~~~~~(error \#'(lambda ...))) \\*
~~...)
\end{lisp}
  if an unbound variable error is signaled in the body (and not handled
  by an intervening handler), the first function will be called. If any
  other kind of error is signaled, the second function will be called.
  In either case, neither handler will be active while executing the code
  in the associated function.
\end{defmac}


\subsection{Defining Conditions}


[The contents of this section are still a subject of some debate within X3J13.
The reader may wish to take this section with a grain of salt, two aspirin
tablets, and call a hacker in the morning.---GLS]

\begin{defmac}
define-condition name ({parent-type}*)
                 [({slot-specifier}*) {option}*]

  Defines a new condition type called {\it name}, which is a subtype of each given
  {\it parent-type}.  Except as otherwise noted, the arguments are not evaluated.

  Objects of this condition type will have all of the indicated {\it slot\/}s, plus
  any additional slots inherited from the parent types (its superclasses).
  If the {\it slot\/}s list is omitted, the empty list is assumed.

  A {\it slot} must have the form
\begin{tabbing}
{\it slot-specifier\/} ::= {\it slot-name\/} {\Mor} ({\it slot-name\/}  $\lbrack\!\lbrack\downarrow\!\hbox{{\it slot-option}}\,\rbrack\!\rbrack$)
\end{tabbing}
For the syntax of a {\it slot-option}, see \cdf{defclass}.
The slots of a condition object are normal CLOS slots.
Note that \cdf{with-slots} may be used instead of accessor functions to access slots of a
condition object.

  \cdf{make-condition} will accept keywords (in the keyword package) with the
  print name of any of the designated slots, and will initialize the
  corresponding slots in conditions it creates.

  Accessors are created according to the same rules as used by \cdf{defclass}.

  The valid {\it options} are as follows:

\begin{flushdesc}
\item[\cd{(:documentation {\it doc-string})}]

     The {\it doc-string} should be either \cdf{nil} or
     a string that describes the purpose of the
     condition type. If this option is omitted, \cdf{nil} is assumed.
     Calling \cd{(documentation '{\it name} 'type)} will retrieve this information.

\item[\cd{(:report {\it exp})}]

     If {\it exp} is not a literal string, it must be a suitable argument to the
     \cdf{function} special form. The expression \cd{(function~{\it exp})} will be evaluated
     in the current lexical environment. It should produce a function of two
     arguments, a condition and a stream, that prints on the stream a
     description of the condition. This function is called whenever the
     condition is printed while \cd{*print-escape*} is \cdf{nil}.

     If {\it exp} is a literal string, it is shorthand for
\begin{lisp}
(lambda (c s) \\*
~~(declare (ignore c)) \\*
~~(write-string {\it exp} s))
\end{lisp}
     [That is, a function is provided that will simply write the given string literally
     to the stream, regardless of the particular condition object supplied.---GLS]

     The \cd{:report} option is processed {\it after} the new condition type has been defined,
     so use of the slot accessors within the report function is permitted.
     If this option is not specified, information about how to report this
     type of condition will be inherited from the {\it parent-type}.
\end{flushdesc}

[X3J13 voted in March 1989 \issue{ZLOS-CONDITIONS} to integrate the
Condition System and the Object System.
In the original Condition System proposal, \cdf{define-condition}
allowed only one {\it parent-type} (the inheritance structure was a simple
hierarchy).
Slot descriptions were much simpler, even simpler than those for \cdf{defstruct}:
\begin{tabbing}
{\it slot} ::= {\it slot-name} {\Mor} ({\it slot-name}) {\Mor} ({\it slot-name} {\it default-value})
\end{tabbing}
Similarly, \cdf{define-condition} allowed
a \cd{:conc-name} option similar to that of \cdf{defstruct}:
\begin{flushdesc}
\item[\cd{(:conc-name {\it symbol-or-string})}]

     {\bf Not now part of Common Lisp.}
     As with \cdf{defstruct}, this sets up automatic prefixing of the names 
     of slot accessors. Also as in \cdf{defstruct}, the default behavior 
     is to use the name of the new type, {\it name}, followed by a hyphen.
     (Generated names are interned in the package that is current at the time that the 
     \cdf{define-condition} is processed).
\end{flushdesc}
One consequence of the vote was to make \cdf{define-condition} slot descriptions
like those of \cdf{defclass}.---GLS]

Here are some examples of the use of \cdf{define-condition}.

\vskip 6pt plus 3 pt minus 2 pt

  The following form defines a condition of type \cd{peg/hole-mismatch} that
  inherits from a condition type called \cdf{blocks-world-error}:
\begin{lisp}
(define-condition peg/hole-mismatch (blocks-world-error) \\*
~~~~~~~~~~~~~~~~~~(peg-shape hole-shape) \\*
~~(:report \\
~~~~(lambda (condition stream) \\*
~~~~~~(with-slots (peg-shape hole-shape) condition \\*
~~~~~~~~(format stream "A {\Xtilde}A peg cannot go in a {\Xtilde}A hole." \\*
~~~~~~~~~~~~~~~~peg-shape hole-shape))))
\end{lisp}
  The new type has slots \cdf{peg-shape} and \cdf{hole-shape}, so \cdf{make-condition} will
  accept \cd{:peg-shape} and \cd{:hole-shape} keywords. The \cdf{with-slots} macro 
  may be used to access the \cdf{peg-shape} and \cdf{hole-shape} slots,
  as illustrated in the \cd{:report} information.

  Here is another example. This defines a condition called \cdf{machine-error}
  that inherits from \cdf{error}:
\begin{lisp}
(define-condition machine-error (error) \\*
~~~~~~~~~~~~~~~~~~((machine-name \\*
~~~~~~~~~~~~~~~~~~~~:reader machine-error-machine-name)) \\
~~(:report (lambda (condition stream) \\*
~~~~~~~~~~~~~(format stream "There is a problem with {\Xtilde}A." \\*
~~~~~~~~~~~~~~~~~~~~~(machine-error-machine-name condition)))))
\end{lisp}
  Building on this definition, we can define a new error condition that
  is a subtype of \cdf{machine-error} for use when machines are not available: 
\begin{lisp}
(define-condition machine-not-available-error (machine-error) () \\*
~~(:report (lambda (condition stream) \\*
~~~~~~~~~~~~~(format stream "The machine {\Xtilde}A is not available." \\*
~~~~~~~~~~~~~~~~~~~~~(machine-error-machine-name condition)))))
\end{lisp}
  We may now define a still more specific condition, built upon 
  \cd{machine-\discretionary{}{}{}not-\discretionary{}{}{}available-\discretionary{}{}{}error},
  that provides a default for \cdf{machine-name}
  but does not provide any new slots or report information. It just
  gives the \cdf{machine-name} slot a default initialization:
\begin{lisp}
(define-condition my-favorite-machine-not-available-error \\*
~~~~~~~~~~~~~~~~~~(machine-not-available-error) \\*
~~~~~~~~~~~~~~~~~~((machine-name :initform "MC.LCS.MIT.EDU")))
\end{lisp}
  Note that since no \cd{:report} clause was given, the information inherited from
  \cdf{machine-not-available-error} will be used to report this type of condition.
\end{defmac}


\subsection{Creating Conditions}

The function \cdf{make-condition} is the basic means for
creating condition objects.

\begin{defun}[Function]
make-condition type &rest slot-initializations

   Constructs a condition object of the given {\it type} using {\it slot-initializations}
   as a specification of the initial value of the slots. The newly created
   condition is returned.

   The {\it slot-initializations} are alternating keyword/value pairs.
   For example:
\begin{lisp}
(make-condition 'peg/hole-mismatch \\*
~~~~~~~~~~~~~~~~:peg-shape 'square :hole-shape 'round)
\end{lisp}
\end{defun}


\subsection{Establishing Restarts}

The lowest-level form that creates restart points is called \cdf{restart-bind}.
The \cdf{restart-case} macro is an abstraction that addresses many common needs for
\cd{restart-\discretionary{}{}{}bind} while offering a more\vadjust{\penalty-10000}
palatable syntax. See also 
\cd{with-\discretionary{}{}{}simple-\discretionary{}{}{}restart}.
The function that transfers control to a restart
point established by one of these macros is called \cd{invoke-\discretionary{}{}{}restart}.

All restarts have dynamic extent; a restart does not survive execution of the form
that establishes it.

\begin{defmac}
with-simple-restart (name format-string {\,format-argument}*)
                    {\,form}*

  This is shorthand for one of the most common uses of \cdf{restart-case}.

  If the restart designated by {\it name} is not invoked while executing the {\it form\/}s,
  all values returned by the last {\it form} are returned. If that
  restart is invoked, control is transferred to the \cdf{with-simple-restart}
  form, which immediately returns the two values \cdf{nil} and \cdf{t}.

  The {\it name} may be \cdf{nil}, in which case an anonymous restart
  is established.

  \cdf{with-simple-restart} could be defined by
\begin{lisp}
(defmacro with-simple-restart ((restart-name format-string \\*
~~~~~~~~~~~~~~~~~~~~~~~~~~~~~~~~\&rest format-arguments) \\*
~~~~~~~~~~~~~~~~~~~~~~~~~~~~~~~\&body forms) \\
~~{\Xbq}(restart-case (progn ,{\Xatsign}forms) \\*
~~~~~(,restart-name () \\*
~~~~~~~:report \\*
~~~~~~~~~(lambda (stream) \\*
~~~~~~~~~~~(format stream ,format-string ,{\Xatsign}format-arguments)) \\*
~~~~~~~(values nil t))))
\end{lisp}

Here is an example of the use of \cdf{with-simple-restart}.
\begin{lisp}
Lisp> (defun read-eval-print-loop (level) \\*
~~~~~~~~(with-simple-restart \\*
~~~~~~~~~~~~(abort "Exit command level {\Xtilde}D." level) \\*
~~~~~~~~~~(loop \\*
~~~~~~~~~~~~(with-simple-restart \\*
~~~~~~~~~~~~~~~~(abort "Return to command level {\Xtilde}D." level) \\*
~~~~~~~~~~~~~~(let ((form (prog2 (fresh-line) \\*
~~~~~~~~~~~~~~~~~~~~~~~~~~~~~~~~~(read) \\*
~~~~~~~~~~~~~~~~~~~~~~~~~~~~~~~~~(fresh-line)))) \\*
~~~~~~~~~~~~~~~~(prin1 (eval form))))))) \\*
~\EV\ READ-EVAL-PRINT-LOOP \\
Lisp> (read-eval-print-loop 1) \\*
(+ 'a 3) \\
Error: The argument, A, to the function + was of the wrong type. \\*
~~~~~~~The function expected a number. \\*
To continue, type :CONTINUE followed by an option number: \\*
~1: Specify a value to use this time. \\*
~2: Return to command level 1. \\*
~3: Exit command level 1. \\*
~4: Return to Lisp Toplevel. \\*
Debug> 
\end{lisp}

\beforenoterule
\begin{incompatibility}
In contrast to the way that Zetalisp has traditionally
  defined \cdf{abort} as a kind of condition to be handled,
  the Common Lisp Condition System defines \cdf{abort} as a
  way to restart (``proceed'' in Zetalisp terms).
\end{incompatibility}
\betweennoterule
\begin{sideremark}
    Some readers may wonder what ought to be done by the ``abort'' key (or whatever
    the implementation's interrupt key is---Control-C or Control-G, 
    for example). Such interrupts, whether synchronous or asynchronous
    in nature, are beyond the scope of this chapter and indeed are not currently
    addressed by Common Lisp at all. This may be a topic
    worth standardizing under separate cover. Here is some speculation
    about some possible things that might happen.

    An implementation might simply call \cdf{abort} or \cdf{break} directly
    without signaling any condition.

    Another implementation might signal some condition related to
    the fact that a key had been pressed rather than to the action that
    should be taken. This is one way to allow user customization.
    Perhaps there would be an implementation-dependent \cdf{keyboard-interrupt}
    condition type with a slot containing the key that was pressed---or
    perhaps there would be such a condition type, but rather than its having
    slots, different subtypes of that type with names like \cdf{keyboard-abort},
    \cdf{keyboard-break}, and so on might be signaled. That implementation would
    then document the action it would take if user programs failed
    to handle the condition, and perhaps ways for user programs to
    usefully dismiss the interrupt.
\end{sideremark}
\betweennoterule
\begin{implementation}
  Implementors are encouraged to make sure that there
  is always a restart named \cdf{abort} around any user code so that user code 
  can call \cdf{abort} at any time and expect something reasonable to happen;
  exactly what the reasonable thing is may vary somewhat. Typically, in an
  interactive program, invoking \cdf{abort} should return the user to top level,
  though in some batch or multi-processing situations
  killing the running process might be
  more appropriate.
\end{implementation}
\afternoterule
\end{defmac}


\begin{defmac}
restart-case expression {(case-name arglist
                         {keyword value}*
                         {\,form}*)}*

  The {\it expression} is evaluated in a dynamic context where the clauses have 
  special meanings as points to which control may be transferred. If the {\it expression}
  finishes executing and returns any values, all such values are simply
  returned by the \cdf{restart-case} form. While the {\it expression} is running, any code may
  transfer control to one of the clauses (see \cdf{invoke-restart}). If a transfer
  occurs, the {\it form\/}s in the body of that clause will be evaluated and any values
  returned by the last such {\it form} will be returned by the \cdf{restart-case} form.

As a special case,
if the {\it expression} is a list whose {\it car} is \cdf{signal}, \cdf{error},
     \cdf{cerror}, or \cdf{warn}, then \cdf{with-condition-restarts} is implicitly
     used to associate the restarts with the condition to be signaled.
For example,
\begin{lisp}
(restart-case (signal weird-error) \\*
~~(become-confused ...) \\*
~~(rewind-line-printer ...) \\*
~~(halt-and-catch-fire ...))
\end{lisp}
     is equivalent to
\begin{lisp}
(restart-case (with-condition-restarts \\*
~~~~~~~~~~~~~~~~weird-error  \\*
~~~~~~~~~~~~~~~~(list (find-restart 'become-confused)  \\*
~~~~~~~~~~~~~~~~~~~~~~(find-restart 'rewind-line-printer) \\*
~~~~~~~~~~~~~~~~~~~~~~(find-restart 'halt-and-catch-fire)) \\*
~~~~~~~~~~~~~~~~(signal weird-error)) \\
~~(become-confused ...) \\*
~~(rewind-line-printer ...) \\*
~~(halt-and-catch-fire ...))
\end{lisp}

  If there are no {\it form\/}s in a selected clause, \cdf{restart-case} returns \cdf{nil}.

  The {\it case-name} may be \cdf{nil} or a symbol naming this restart.

  It is possible to have more than one clause use the same {\it case-name}.
  In this case, the first clause with that name will be found by
  \cdf{find-restart}. The other clauses are accessible using \cdf{compute-restarts}.
  [In this respect, \cdf{restart-case} is rather different from \cdf{case}!---GLS]

  Each {\it arglist} is a normal lambda-list containing parameters
  to be bound during the execution of 
  its corresponding {\it form\/}s. These parameters are used to pass any necessary 
  data from a call to \cdf{invoke-restart} to the \cdf{restart-case} clause.

  By default, \cdf{invoke-restart-interactively} will pass no arguments and
  all parameters must be optional in order to accommodate interactive
  restarting. However, the parameters need not be optional if the
  \cd{:interactive} keyword has been used to inform \cdf{invoke-restart-interactively}
  about how to compute a proper argument list.

  The valid {\it keyword value} pairs are the following:
\begin{flushdesc}
\item[\cd{:test {\it fn}}]

    The {\it fn} must be a suitable argument for the \cdf{function} special form. The
    expression \cd{(function~{\it fn})} will be evaluated in the current lexical
    environment. It should produce a function of one argument, a condition.
    If this function returns \cdf{nil} when given some condition, functions such as
\cdf{find-restart}, \cdf{compute-restart}, and \cdf{invoke-restart}
will not consider this restart when searching for restarts associated with
that condition.  If this pair is not supplied, it is as if
\begin{lisp}
(lambda (c) (declare (ignore c)) t)
\end{lisp}
were used for the {\it fn}.

\item[\cd{:interactive {\it fn}}]

    The {\it fn} must be a suitable argument for the \cdf{function} special form. The
    expression \cd{(function~{\it fn})} will be evaluated in the current lexical
    environment. It should produce a function of no arguments that 
    returns arguments to be used by \cdf{invoke-restart-interactively} when
    invoking this function. This function will be called in the dynamic
    environment available prior to any restart attempt. It may interact with the user
    on the stream in \cd{*query-io*}.

    If a restart is invoked interactively but no \cd{:interactive} option
    was supplied, the argument list used in the invocation is the empty
    list.

\item[\cd{:report {\it exp}}]

    If {\it exp} is not a literal string, it must be a suitable argument to the
    \cdf{function} special form. The expression \cd{(function~{\it exp})} will be evaluated
    in the current lexical environment. It should produce a function of one
    argument, a stream, that prints on the stream a description of the
    restart. This function is called whenever the restart is printed while
    \cd{*print-escape*} is \cdf{nil}.

    If {\it exp} is a literal string, it is shorthand for
\begin{lisp}
(lambda (s) (write-string {\it exp} s))
\end{lisp}
[That is, a function is provided that will simply write the given string literally
to the stream.---GLS]

    If a named restart is asked to report but no report information has been
    supplied, the name of the restart is used in generating default report text.

    When \cd{*print-escape*} is \cdf{nil}, the printer will use the report information for
    a restart. For example, a debugger might announce the action of typing
    ``\cd{:continue}'' by executing the equivalent of
\begin{lisp}
(format *debug-io* "{\Xtilde}\&{\Xtilde}S -- {\Xtilde}A{\Xtilde}\%" ':continue some-restart)
\end{lisp}
    which might then display as something like
\begin{lisp}
:CONTINUE -- Return to command level.
\end{lisp}

    It is an error if an unnamed restart is used and no report information
    is provided.

\beforenoterule
    \begin{rationale}
    Unnamed restarts are required to have report information
    on the grounds that they are generally only useful interactively,
    and an interactive option that has no description is of little value.
    \end{rationale}
\betweennoterule
\begin{implementation}
      Implementations are encouraged to warn about this error at compilation time.

      At run time, this error might be noticed when entering
      the debugger. Since signaling an error would probably cause recursive
      entry into the debugger (causing yet another recursive error, and so on), it is
      suggested that the debugger print some indication of such problems when
      they occur, but not actually signal errors.
\end{implementation}
\afternoterule
\end{flushdesc}

Note that 
\begin{lisp}
(restart-case {\it expression} \\*
~~({\it name\SU{1}} {\it arglist\SU{1}} {\it options\SU{1}} . {\it body\SU{1}}) \\*
~~({\it name\SU{2}} {\it arglist\SU{2}} {\it options\SU{2}} . {\it body\SU{2}}) \\*
~~...)
\end{lisp}
is essentially equivalent to
\begin{lisp}
(block \#1=\#:block-1 \\*
~~(let ((\#2=\#:var-2 nil)) \\*
~~~~(tagbody \\*
~~~~~~(restart-bind (({\it name\SU{1}} \pushtabs\=\#'(lambda (\&rest temp) \\*
\>~~~~(setq \#2\# temp) \\*
\>~~~~(go \#3=\#:tag-3)) \\*
\>{\rm $\langle$slightly transformed {\it options\SU{1}}$\rangle$})\poptabs \\
~~~~~~~~~~~~~~~~~~~~~({\it name\SU{2}} \pushtabs\=\#'(lambda (\&rest temp) \\*
\>~~~~(setq \#2\# temp) \\*
\>~~~~(go \#4=\#:tag-4)) \\*
\>{\rm $\langle$slightly transformed {\it options\SU{2}}$\rangle$})\poptabs \\*
~~~~~~~~~~~~~~~~~~~~~...) \\*
~~~~~~~~(return-from \#1\# expression)) \\
~~~~~~\#3\# (return-from \#1\# \\*
~~~~~~~~~~~~~~~~(apply \#'(lambda {\it arglist\SU{1}} . {\it body\SU{1}}) \#2\#)) \\
~~~~~~\#4\# (return-from \#1\# \\*
~~~~~~~~~~~~~~~~(apply \#'(lambda {\it arglist\SU{2}} . {\it body\SU{2}}) \#2\#)) \\*
~~~~~~...)))
\end{lisp}
[Note the use of ``gensyms'' such as \cd{\#:block-1} as block names,
variables, and \cdf{tagbody} tags in this example,
and the use of \cd{\#{\it n}=} and \cd{\#{\it n}\#} read-macro syntax
to indicate that the very same gensym appears in multiple places.---GLS]


Here are some examples of the use of \cdf{restart-case}.
\begin{lisp}
(loop \\*
~~(restart-case (return (apply function some-args)) \\*
~~~~(new-function (new-function) \\*
~~~~~~~~:report "Use a different function." \\*
~~~~~~~~:interactive \\*
~~~~~~~~~~(lambda () \\*
~~~~~~~~~~~~(list (prompt-for 'function "Function: "))) \\*
~~~~~~(setq function new-function)))) \\
 \\
(loop \\*
~~(restart-case (return (apply function some-args)) \\*
~~~~(nil (new-function) \\*
~~~~~~~~:report "Use a different function." \\*
~~~~~~~~:interactive \\*
~~~~~~~~~~(lambda () \\*
~~~~~~~~~~~~(list (prompt-for 'function "Function: "))) \\*
~~~~~~(setq function new-function)))) \\
 \\
(restart-case (a-command-loop) \\*
~~(return-from-command-level () \\*
~~~~~~:report \\*
~~~~~~~~(lambda (s)~~~~~;{\rm Argument \cdf{s} is a stream} \\*
~~~~~~~~~~(format s "Return from command level {\Xtilde}D." level)) \\*
~~~~nil)) \\
 \\
(loop  \\*
~~(restart-case (another-random-computation) \\*
~~~~(continue () nil)))
\end{lisp}
  The first and second examples are equivalent from the point of view of someone
  using the interactive debugger, but they differ in one important aspect for 
  non-interactive handling. If a handler ``knows about'' named restarts, as in, for example,
\begin{lisp}
(when (find-restart 'new-function) \\*
~~(invoke-restart 'new-function the-replacement))
\end{lisp}
  then only the first example, and not the second, will have control
  transferred to its correction clause, since only the first example uses
  a restart named \cdf{new-function}.

\vskip 6pt plus 3pt minus 2pt

  Here is a more complete example:
\begin{lisp}
(let ((my-food 'milk) \\*
~~~~~~(my-color 'greenish-blue)) \\*
~~(do () \\*
~~~~~~((not (bad-food-color-p my-food my-color))) \\*
~~~~(restart-case (error 'bad-food-color \\*
~~~~~~~~~~~~~~~~~~~~~~~~~:food my-food :color my-color) \\*
~~~~~~(use-food (new-food) \\*
~~~~~~~~~~:report "Use another food." \\*
~~~~~~~~(setq my-food new-food)) \\
~~~~~~(use-color (new-color) \\*
~~~~~~~~~~:report "Use another color." \\*
~~~~~~~~(setq my-color new-color)))) \\*
~~;; We won't get to here until MY-FOOD \\*
~~;; and MY-COLOR are compatible. \\*
~~(list my-food my-color))
\end{lisp}
Assuming that \cdf{use-food} and \cdf{use-color} have been defined as
\begin{lisp}
(defun use-food (new-food) \\*
~~(invoke-restart 'use-food new-food)) \\
\\
(defun use-color (new-color) \\*
~~(invoke-restart 'use-color new-color))
\end{lisp}
a handler can then restart from the error in either of two ways.
It may correct the color or correct the food. For example:
\begin{lisp}
\#'(lambda (c) ... (use-color 'white) ...)~~~;{\rm Corrects \cdf{color}} \\
\\
\#'(lambda (c) ... (use-food 'cheese) ...)~~~;{\rm Corrects \cdf{food}}
\end{lisp}

  Here is an example using \cdf{handler-bind} and \cdf{restart-case} that refers to a
  condition type \cdf{foo-error}, presumably defined elsewhere:
\begin{lisp}
(handler-bind ((foo-error \#'(lambda (ignore) (use-value 7)))) \\*
~~(restart-case (error 'foo-error) \\*
~~~~(use-value (x) (* x x)))) \\*
~\EV\ 49
\end{lisp}
\end{defmac}


\begin{defmac}
restart-bind ({(name function {keyword value}*)}*) {\,form}*

  Executes a body of forms in a dynamic context where the given restart
  bindings are in effect.

  Each {\it name} may be \cdf{nil} to indicate an anonymous restart, or some other symbol
  to indicate a named restart.

  Each {\it function} is a form that
  should evaluate to a function to be used to perform the restart.
  If invoked, this function may either perform a non-local transfer of control
  or it may return normally. The function may take whatever arguments the
  programmer feels are appropriate; it will be invoked only if \cdf{invoke-restart}
  is used from a program, or if a user interactively asks the debugger to
  invoke it. In the case of interactive invocation, the \cd{:interactive-function}
  option is used.

\vskip 6pt plus 3pt minus 2pt

  The valid {\it keyword value} pairs are as follows:
\begin{flushdesc}
\item[\cd{:test-function {\it form}}]

    The {\it form} will be evaluated in the current lexical environment and
     should return a function of one argument, a condition.
    If this function returns \cdf{nil} when given some condition, functions such as
\cdf{find-restart}, \cdf{compute-restart}, and \cdf{invoke-restart}
will not consider this restart when searching for restarts associated with
that condition.  If this pair is not supplied, it is as if
\begin{lisp}
\#'(lambda (c) (declare (ignore c)) t)
\end{lisp}
were used for the {\it form}.

\item[\cd{:interactive-function {\it form}}]

     The {\it form} will be evaluated in the current lexical environment and
     should return a function of no arguments that constructs a list
     of arguments to be used by \cdf{invoke-restart-interactively} when invoking
     this restart. The function may prompt interactively using \cd{*query-io*}
     if necessary.

\item[\cd{:report-function {\it form}}]

     The {\it form} will be evaluated in the current lexical environment and
     should return a function of one argument, a stream, that prints on
     the stream a summary of the action this restart will take. This
     function is called whenever the restart is printed while \cd{*print-escape*}
     is \cdf{nil}.
\end{flushdesc}
\end{defmac}

\begin{defmac}
with-condition-restarts condition-form restarts-form
  {declaration}* {\,form}*

The value of {\it condition-form} should be a condition {\it C} and
the value of {\it restarts-form} should be a list of restarts \cd{({\it R1} {\it R2} ...)}.
      The {\it form\/}s of the body are evaluated as an implicit \cdf{progn}.
      While in the dynamic context of the body,
      an attempt to find a restart associated with a particular
      condition ${\it C}'$ will 
      consider the restarts {\it R1}, {\it R2}, $\ldots$ if ${\it C}'$ is \cdf{eq} to {\it C}.

     Usually this macro is not used explicitly in code, because
     \cdf{restart-case} handles most of the common uses in a way that is
     syntactically more concise.

[The X3J13 vote \issue{CONDITION-RESTARTS} left it unclear whether \cdf{with-condition-restarts}
permits declarations to appear at the heads of its body.
I believe that was the intent, but this is only my interpretation.---GLS]

\end{defmac}


\subsection{Finding and Manipulating Restarts}

The following functions determine what restarts are
active and invoke restarts.

\begin{defun}[Function]
compute-restarts &optional condition

  Uses the dynamic state of the program to compute a list of the restarts
  that are currently active. See \cdf{restart-bind}.

If {\it condition} is \cdf{nil} or not supplied, all outstanding restarts
are returned.
If {\it condition} is not \cdf{nil}, only restarts associated
with that condition are returned.

  Each restart represents a function that can be called to perform some
  form of recovery action, usually a transfer of control to an outer point
  in the running program. Implementations are free to implement these objects
  in whatever manner is most convenient; the objects need have only dynamic
  extent (relative to the scope of the binding form that instantiates them).

  The list that results from a call to \cdf{compute-restarts} is ordered so that
  the inner (that is, more recently established) restarts are nearer the head
  of the list.

  Note, too, that \cdf{compute-restarts} returns all valid restarts, including
  anonymous ones, even if some of them have the same name as others and
  would therefore not be found by \cdf{find-restart} when given a symbol argument.

  Implementations are permitted, but not required, to return different 
  (that is, non-\cdf{eq}) lists from repeated calls to \cdf{compute-restarts} while in
  the same dynamic environment. It is an error to modify the list that
  is returned by \cdf{compute-restarts}.
\end{defun}


\begin{defun}[Function]
restart-name restart

  Returns the name of the given {\it restart}, or \cdf{nil} if it is not named.
\end{defun}

\begin{defun}[Function]
find-restart restart-identifier &optional condition

  Searches for a particular restart in the current dynamic environment.

If {\it condition} is \cdf{nil} or not supplied, all outstanding restarts
are considered.
If {\it condition} is not \cdf{nil}, only restarts associated
with that condition are considered.

  If the {\it restart-identifier} is a non-\cdf{nil}
  symbol, then the innermost (that is, most recently
  established) restart with that name is returned;  \cdf{nil} is returned if no
  such restart is found.

  If {\it restart-identifier} is a restart object, then it is simply returned,
  unless it is not currently active, in which case \cdf{nil} is returned.

  Although anonymous restarts have a name of \cdf{nil}, it is an error for
  the symbol \cdf{nil} to be given as the {\it restart-identifier}.  Applications that 
  would seem to require this should be rewritten to make appropriate use 
  of \cdf{compute-restarts} instead.
\end{defun}

\begin{defun}[Function]
invoke-restart restart-identifier &rest arguments

  Calls the function associated with the given {\it restart-identifier}, passing any given
  {\it arguments}. The {\it restart-identifier} must be a restart or the non-null name of a
  restart that is valid in the current dynamic context. If the argument
  is not valid, an error of type \cdf{control-error} will be signaled.

  \beforenoterule
  \begin{implementation}
  Restart functions call this function, not vice versa.
  \end{implementation}
  \afternoterule
\end{defun}

\begin{defun}[Function]
invoke-restart-interactively restart-identifier

  Calls the function associated with the given {\it restart-identifier}, prompting for any
  necessary arguments. The {\it restart-identifier} must be a restart or the non-null name
  of a restart that is valid in the current dynamic context. If the
  argument is not valid, an error of type \cdf{control-error} will be signaled.

  The function \cdf{invoke-restart-interactively} will prompt for arguments by executing
  the code provided in the \cd{:interactive} keyword to \cdf{restart-case} or 
  \cd{:interactive-function} keyword to \cdf{restart-bind}.

  If no \cd{:interactive} or \cd{:interactive-function}
  option has been supplied in the corresponding
  \cdf{restart-case} or \cdf{restart-bind}, then it is an error if the restart takes
  required arguments. If the arguments are optional, an empty argument list
  will be used in this case.

  Once \cdf{invoke-restart-interactively} has calculated the arguments, it simply
  performs
  \cd{(apply~\#'invoke-restart {\it restart-identifier} {\it arguments})}.

  \cdf{invoke-restart-interactively} is used internally by the debugger and may also be useful
  in implementing other portable, interactive debugging tools.
\end{defun}


\subsection{Warnings}
\label{WARNING-CONDITIONS}

Warnings are a subclass of errors that are conventionally regarded as ``mild.''

\begin{defun}[Function]
warn datum &rest arguments

   [This supersedes the description of \cdf{warn}
   given in section~\ref{ERROR-SIGNALLING-FUNCTIONS}.---GLS]

  Warns about a situation, by signaling a condition of type \cdf{warning}.

  If {\it datum} is a condition, then that condition is used directly.
  In this case, if the condition is not of type \cdf{warning} or arguments
  is non-\cdf{nil}, an error of type \cdf{type-error} is signaled.

  If {\it datum} is a condition type (a class or class name), then the condition used is effectively the result
  of \cd{(apply \#'make-condition {\it datum} {\it arguments})}. This result
  must be of type \cdf{warning} or an error of type \cdf{type-error} is signaled.

  If {\it datum} is a string, then the condition used is effectively the result of
\begin{lisp}
(make-condition 'simple-error \\*
~~~~~~~~~~~~~~~~:format-string {\it datum} \\*
~~~~~~~~~~~~~~~~:format-arguments {\it arguments})
\end{lisp}

  The precise mechanism for warning is as follows.
\begin{enumerate}

%%% *break-on-warnings* deleted March 1989
%\item If \cd{*break-on-warnings*} is true, then the debugger will be entered.
%     This feature is primarily for compatibility with old code; use of
%     \cd{*break-on-signals*} is preferred. See \cd{*break-on-warnings*} below.
%     If the break is continued using the \cdf{continue} restart, \cdf{warn} proceeds
%     with the following step.
%
%\item The warning condition is then signaled.
\item The warning condition is signaled.

     While the \cdf{warning} condition is being signaled, the \cdf{muffle-warning}
     restart is established for use by a handler to bypass further action
     by \cdf{warn} (that is, to cause \cdf{warn} to immediately return \cdf{nil}).

     As part of the signaling process, if
     \cd{(typep {\it condition} *break-on-signals*)}
     is true, then a \cdf{break} will occur prior to beginning the signaling
     process.

\item If no handlers for the warning condition are found, or if all such
     handlers decline, then the condition will be reported to \cd{*error-output*}
     by the \cdf{warn} function (with possible implementation-specific extra
     output such as motion to a fresh line before or after the display
     of the warning, or supplying some introductory text mentioning
     the name of the function that called \cdf{warn} or the fact that this
     is a warning).

\item The value returned by \cdf{warn} (if it returns) is \cdf{nil}.
\end{enumerate}
\end{defun}

%%% *break-on-warnings* deleted March 1989
%
%\begin{defun}[Variable]
%*break-on-warnings*
%
%   [This supersedes the description of \cd{*break-on-warnings*}
%   given in section~\ref{ERROR-SIGNALLING-FUNCTIONS}.---GLS]
%
%  If \cd{*break-on-warnings*} is true, then the function \cdf{warn} will enter the
%  debugger \cdf{break} before signaling the \cdf{warning} condition. If the \cdf{continue}
%  restart is used, \cdf{warn} continues normally with the warning process.
%
%  It is intended primarily for use when the user is debugging programs that
%  issue warnings; in ``production'' use, the value of \cd{*break-on-warnings*}
%  should be \cdf{nil}. 
%
%  Note: The variable \cd{*break-on-warnings*} is still part
%        of Common Lisp but is considered obsolete.
%        See \cd{*break-on-signals*}.
%\end{defun}

\subsection{Restart Functions}

Common Lisp has the following restart functions built in.

\begin{defun}[Function]
abort &optional condition

  This function transfers control to the restart named \cdf{abort}. If no such
  restart exists, \cdf{abort} signals an error of type \cdf{control-error}.

 If {\it condition} is \cdf{nil} or not supplied, all outstanding restarts
are considered.
If {\it condition} is not \cdf{nil}, only restarts associated
with that condition are considered.

  The purpose of the \cdf{abort} restart is generally to allow control to return to the
  innermost ``command level.''
\end{defun}

\begin{defun}[Function]
continue &optional condition

  This function transfers control to the restart named \cdf{continue}. If no such
  restart exists, \cdf{continue} returns \cdf{nil}.

 If {\it condition} is \cdf{nil} or not supplied, all outstanding restarts
are considered.
If {\it condition} is not \cdf{nil}, only restarts associated
with that condition are considered.

  The \cdf{continue} restart is generally part of simple protocols where there is
  a single ``obvious'' way to continue, as with \cdf{break} and \cdf{cerror}. Some
  user-defined protocols may also wish to incorporate it for similar reasons.
  In general, however, it is more reliable to design a special-purpose restart
  with a name that better suits the particular application.
\end{defun}

\begin{defun}[Function]
muffle-warning &optional condition

  This function transfers control to the restart named \cdf{muffle-warning}.
  If no such restart exists, \cdf{muffle-warning} signals an error of type 
  \cdf{control-error}.
 
 If {\it condition} is \cdf{nil} or not supplied, all outstanding restarts
are considered.
If {\it condition} is not \cdf{nil}, only restarts associated
with that condition are considered.

  \cdf{warn} sets up this restart so that handlers of \cdf{warning} conditions have
  a way to tell \cdf{warn} that a \cdf{warning} has already been dealt with and
  that no further action is warranted.
\end{defun}

\begin{defun}[Function]
store-value value &optional condition

  This function transfers control (and one value) to the restart named
  \cdf{store-value}. If no such restart exists, \cdf{store-value} returns \cdf{nil}.

 If {\it condition} is \cdf{nil} or not supplied, all outstanding restarts
are considered.
If {\it condition} is not \cdf{nil}, only restarts associated
with that condition are considered.

  The \cdf{store-value} restart is generally used by handlers trying to recover
  from errors of types such as \cdf{cell-error} or \cdf{type-error}, where the handler
  may wish to supply a replacement datum to be stored permanently.
\end{defun}

\begin{defun}[Function]
use-value value &optional condition

  This function transfers control (and one value) to the restart named
  \cdf{use-value}. If no such restart exists, \cdf{use-value} returns \cdf{nil}.
 
 If {\it condition} is \cdf{nil} or not supplied, all outstanding restarts
are considered.
If {\it condition} is not \cdf{nil}, only restarts associated
with that condition are considered.

  The \cdf{use-value} restart is generally used by handlers trying to recover
  from errors of types such as \cdf{cell-error}, where the handler may wish to
  supply a replacement datum for one-time use.
\end{defun}


\subsection{Debugging Utilities}
\label{DEBUGGING-UTILITIES}

Common Lisp does not specify exactly what a debugger is or does,
but it does provide certain means for indicating intent to transfer control
to a supervisory or debugging facility.

\begin{defun}[Function]
break &optional format-string &rest format-arguments

   [This supersedes the description of \cdf{break}
   given in section~\ref{ERROR-SIGNALLING-FUNCTIONS}.---GLS]

  The function \cdf{break} prints the message described by the
  {\it format-string} and {\it format-arguments} and then
  goes directly into the debugger without allowing any possibility of
  interception by programmed error-handling facilities.

  If no {\it format-string} is supplied, a suitable default will be generated.

  If continued, \cdf{break} returns \cdf{nil}.

  Note that \cdf{break} is presumed to be used as a way of inserting temporary debugging
  ``breakpoints'' in a program, not as a way of signaling errors; it is
  expected that continuing from a \cdf{break} will not trigger any unusual recovery
  action. For this reason, \cdf{break} does not take the additional format control
  string that \cdf{cerror} takes as its first argument. This and the lack of any
  possibility of interception by programmed error handling are the only
  program-visible differences between \cdf{break} and \cdf{cerror}. The user interface
  aspects of these functions are permitted to vary more widely; for example,
  it is permissible for a read-eval-print loop to be entered by \cdf{break} rather
  than by the conventional debugger.

  \cdf{break} could be defined by
\begin{lisp}
(defun break (\&optional (format-string "Break") \\*
~~~~~~~~~~~~~~\&rest format-arguments) \\*
~~(with-simple-restart (continue "Return from BREAK.") \\*
~~~~(invoke-debugger \\*
~~~~~~(make-condition 'simple-condition \\*
~~~~~~~~~~~~~~~~~~~~~~:format-string format-string \\*
~~~~~~~~~~~~~~~~~~~~~~:format-arguments format-arguments))) \\*
~~nil)
\end{lisp}
\end{defun}

\begin{defun}[Function]
invoke-debugger condition

  Attempts interactive handling of its argument, which must be a condition.

  If the variable \cd{*debugger-hook*} is not \cdf{nil}, it will be called as a function on two
  arguments: the {\it condition} being handled and the value of \cd{*debugger-hook*}.
  If a hook function returns normally, the standard debugger will be tried.

  The standard debugger will never directly return. Return can occur only by a
  special transfer of control, such as the use of a restart.

\beforenoterule
\begin{sideremark}
    The exact way in which the debugger interacts with users is expected to
    vary considerably from system to system. For example, some systems may
    use a keyboard interface, while others may use a mouse interface. Of those
    systems using keyboard commands, some may use single-character commands
    and others may use parsed line-at-a-time commands. The exact set of commands
    will vary as well. The important properties of a debugger are that
it makes information about the error accessible and that
it makes the set of apparent restarts easily accessible.

    It is desirable to have a mode where the debugger allows other features,
    such as the ability to inspect data, stacks, etc. However, it may
    sometimes be appropriate to have this kind of information hidden from
    users. Experience on the Lisp Machines has shown that some users who are
    not programmers develop a terrible phobia of debuggers. The reason for
    this usually may be traced to the fact that the debugger is very foreign to them
    and provides an overwhelming amount of information of
    interest only to programmers. With the advent of restarts, there is a clear
    mechanism for the construction of ``friendly'' debuggers. Programmers can
    be taught how to get to the information they need for debugging, but it
    should be possible to construct user interfaces to the debugger that are
    natural, convenient, intelligible, and friendly even to non-programmers.
\end{sideremark}
\afternoterule
\end{defun}

\begin{defun}[Variable]
*debugger-hook*

  This variable should hold either \cdf{nil} or a function of two arguments, a
  condition and the value of \cd{*debugger-hook*}. This function may either
  handle the condition (transfer control) or return normally (allowing the
  standard debugger to run).

  Note that, to minimize recursive errors while debugging, \cd{*debugger-hook*} is
  bound to \cdf{nil} when calling this function. When evaluating code typed in
  by the user interactively, the hook function may want to bind
  \cd{*debugger-hook*} to the function that was its second argument so that
  recursive errors can be handled using the same interactive facility.
\end{defun}



\section{Predefined Condition Types}        
\label{PREDEFINED-CONDITIONS-SECTION}
[The proposal for the Common Lisp Condition System introduced
a new notation for documenting types, treating them in the
same syntactic manner as functions and variables.  This notation
is used in this section but is not reflected
throughout the entire book.---GLS]


X3J13 voted in March 1989 \issue{ZLOS-CONDITIONS} to integrate
the Condition System and the Object System.  All condition types
are CLOS classes and all condition objects are ordinary CLOS objects.

\begin{defun}[Type]
restart

  This is the data type used to represent a restart.
\end{defun}

\begin{table}[t]
\caption{Condition Type Hierarchy}
\label{CONDITION-HIERARCHY-TABLE}
\begin{lisp}
condition \\
~~~~simple-condition \\
~~~~serious-condition \\
~~~~~~~~error \\
~~~~~~~~~~~~simple-error \\
~~~~~~~~~~~~arithmetic-error \\
~~~~~~~~~~~~~~~~division-by-zero \\
~~~~~~~~~~~~~~~~floating-point-overflow \\
~~~~~~~~~~~~~~~~floating-point-underflow \\
~~~~~~~~~~~~~~~~... \\
~~~~~~~~~~~~cell-error \\
~~~~~~~~~~~~~~~~unbound-variable \\
~~~~~~~~~~~~~~~~undefined-function \\
~~~~~~~~~~~~~~~~... \\
~~~~~~~~~~~~control-error \\
~~~~~~~~~~~~file-error \\
~~~~~~~~~~~~package-error \\
~~~~~~~~~~~~program-error \\
~~~~~~~~~~~~stream-error \\
~~~~~~~~~~~~~~~~end-of-file \\
~~~~~~~~~~~~~~~~... \\
~~~~~~~~~~~~type-error \\
~~~~~~~~~~~~~~~~simple-type-error \\
~~~~~~~~~~~~~~~~... \\
~~~~~~~~~~~~... \\
~~~~~~~~storage-condition \\
~~~~~~~~... \\
~~~~warning \\
~~~~~~~~simple-warning \\
~~~~~~~~... \\
~~~~...
\end{lisp}
\vfill
\end{table}

The Common Lisp condition type hierarchy is illustrated in table~\ref{CONDITION-HIERARCHY-TABLE}.

The types that are not leaves in the hierarchy (that is, \cdf{condition}, \cdf{warning},
\cdf{storage-condition}, \cdf{error}, \cdf{arithmetic-error}, \cdf{control-error},
and so on) are provided
primarily for type inclusion purposes. Normally they would not be directly
instantiated. 

Implementations are permitted to support non-portable synonyms for these
types, as well as to introduce other types that are above, below, or between
the types shown in this tree as long as the indicated subtype relationships
are not violated.

The types \cdf{simple-condition}, \cdf{serious-condition}, and \cdf{warning} are pairwise
disjoint. The type \cdf{error} is also disjoint from types \cdf{simple-condition} and
\cdf{warning}.

\begin{defun}[Type]
condition

  All types of conditions, whether error or non-error, must inherit from 
  this type.
\end{defun}

\begin{defun}[Type]
warning

  All types of warnings should inherit from this type. 
  This is a subtype of \cdf{condition}.
\end{defun}

\begin{defun}[Type]
serious-condition

  All serious conditions (conditions serious enough to require interactive
  intervention if not handled) should inherit from this type. This is a
  subtype of \cdf{condition}.

  This condition type is provided primarily for terminological convenience.
  In fact, signaling a condition that inherits from \cdf{serious-condition} does
  not force entry into the debugger. Rather, it is conventional
  to use \cdf{error} (or something built on \cdf{error}) to signal conditions that are
  of this type, and to use \cdf{signal} to signal conditions that are not of this
  type.
\end{defun}

\begin{defun}[Type]
error

  All types of error conditions inherit from this condition.
  This is a subtype of \cdf{serious-condition}.
\end{defun}


The default condition type for \cdf{signal} and \cdf{warn} is \cdf{simple-condition}.
The default condition type for \cdf{error} and \cdf{cerror} is \cdf{simple-error}.

\begin{defun}[Type]
simple-condition

  Conditions signaled by \cdf{signal} when given a format string as a first
  argument are of this type. This is a subtype of \cdf{condition}.
  The initialization keywords \cd{:format-string} and \cd{:format-arguments} are supported
  to initialize the slots, which can be accessed using
  \cdf{simple-condition-format-string} and \cdf{simple-condition-format-arguments}.
  If \cd{:format-arguments} is not supplied to \cdf{make-condition}, the 
  format-arguments slot defaults to \cdf{nil}.
\end{defun}

\begin{defun}[Type]
simple-warning

  Conditions signaled by \cdf{warn} when given a format string as a first
  argument are of this type. This is a subtype of \cdf{warning}.
  The initialization keywords \cd{:format-string} and \cd{:format-arguments} are supported
  to initialize the slots, which can be accessed using
  \cdf{simple-condition-format-string} and \cdf{simple-condition-format-arguments}.
  If \cd{:format-arguments} is not supplied to \cdf{make-condition}, the 
  format-arguments slot defaults to \cdf{nil}.

  In implementations supporting multiple inheritance, this type will also be
  a subtype of \cdf{simple-condition}.
\end{defun}

\begin{defun}[Type]
simple-error

  Conditions signaled by \cdf{error} and \cdf{cerror} when given a format string 
  as a first argument are of this type. This is a subtype of \cdf{error}.
  The initialization keywords \cd{:format-string} and \cd{:format-arguments} are supported
  to initialize the slots, which can be accessed using
  \cdf{simple-condition-format-string} and \cdf{simple-condition-format-arguments}.
  If \cd{:format-arguments} is not supplied to \cdf{make-condition}, the 
  format-arguments slot defaults to \cdf{nil}.

  In implementations supporting multiple inheritance, this type will also be
  a subtype of \cdf{simple-condition}.
\end{defun}

\begin{defun}[Function]
simple-condition-format-string condition

  Accesses the format-string slot of a given {\it condition}, which must be
  of type \cdf{simple-condition}, \cdf{simple-warning}, \cdf{simple-error}, or 
  \cdf{simple-type-error}.
\end{defun}

\begin{defun}[Function]
simple-condition-format-arguments condition

  Accesses the format-arguments slot of a given {\it condition}, which must
  be of type \cdf{simple-condition}, \cdf{simple-warning}, \cdf{simple-error}, or
  \cdf{simple-type-error}.
\end{defun}


\begin{defun}[Type]
storage-condition

  Conditions that relate to storage overflow should inherit from this type.
  This is a subtype of \cdf{serious-condition}.
\end{defun}


\begin{defun}[Type]
type-error

  Errors in the transfer of data in a program should inherit from
  this type. This is a subtype of \cdf{error}. For example, conditions to
  be signaled by \cdf{check-type} should inherit from this type. The
  initialization keywords \cd{:datum} and \cd{:expected-type} are supported to 
  initialize the slots, which can be accessed using
  \cdf{type-error-datum} and \cdf{type-error-expected-type}.
\end{defun}

\begin{defun}[Function]
type-error-datum condition

  Accesses the datum slot of a given {\it condition}, which must be of
  type \cdf{type-error}.
\end{defun}

\begin{defun}[Function]
type-error-expected-type condition

  Accesses the expected-type slot of a given {\it condition}, which must be
  of type \cdf{type-error}. Users of \cdf{type-error} conditions are expected to
  fill this slot with an object that is a valid Common Lisp type specifier.
\end{defun}

\begin{defun}[Type]
simple-type-error

  Conditions signaled by facilities similar to \cdf{check-type} may want to
  use this type. The initialization keywords \cd{:format-string} and \cd{:format-arguments}
  are supported to initialize the slots, which can be accessed using 
  \cd{simple-\discretionary{}{}{}condition-\discretionary{}{}{}format-\discretionary{}{}{}string} and
  \cd{simple-\discretionary{}{}{}condition-\discretionary{}{}{}format-\discretionary{}{}{}arguments}.
  If \cd{:format-\discretionary{}{}{}arguments} is not supplied to
  \cd{make-\discretionary{}{}{}condition}, the 
  format-arguments slot defaults to \cdf{nil}.

  In implementations supporting multiple inheritance, this type will also be
  a subtype of \cdf{simple-condition}.
\end{defun}

\begin{defun}[Type]
program-error

  Errors relating to incorrect program syntax that are statically
  detectable should inherit from this type (regardless of whether they
  are in fact statically detected). This is a subtype of \cdf{error}. This is
  {\it not} a subtype of \cdf{control-error}.
\end{defun}

\begin{defun}[Type]
control-error

  Errors in the dynamic transfer of control in a program should inherit
  from this type. This is a subtype of \cdf{error}. This is {\it not} a subtype of
  \cdf{program-error}.

  The errors that result from giving \cdf{throw} a tag that is not
  active or from giving \cdf{go} or \cdf{return-from} a tag that is no longer
  dynamically available are control errors.

  On the other hand, the errors that result from naming a \cdf{go} tag or \cdf{return-from} tag that
  is not lexically apparent are not control errors. They are program
  errors. See \cdf{program-error}.
\end{defun}


\begin{defun}[Type]
package-error

  Errors that occur during operations on packages should inherit from
  this type. This is a subtype of \cdf{error}. The initialization keyword \cd{:package}
  is supported to initialize the slot, which can be accessed using
  \cdf{package-error-package}.
\end{defun}

\begin{defun}[Function]
package-error-package condition

  Accesses the package (or package name) that was being modified or
  manipulated in a {\it condition} of type \cdf{package-error}.
\end{defun}


\begin{defun}[Type]
stream-error

  Errors that occur during input from, output to, or closing a stream
  should inherit from this type. This is a subtype of \cdf{error}. The initialization
  keyword \cd{:stream} is supported to initialize the slot, which can be
  accessed using \cdf{stream-error-stream}.
\end{defun}

\begin{defun}[Function]
stream-error-stream condition

  Accesses the offending stream of a {\it condition} of type \cdf{stream-error}.
\end{defun}

\begin{defun}[Type]
end-of-file

  The error that results when a read operation is done on a stream that has
  no more tokens or characters should inherit from this type. This is a subtype of 
  \cdf{stream-error}.
\end{defun}

\begin{defun}[Type]
file-error

  Errors that occur during an attempt to open a file, or during some low-level
  transaction with a file system, should inherit from this type. This is a
  subtype of \cdf{error}. The initialization keyword \cd{:pathname} is supported to initialize the
  slot, which can be accessed using \cdf{file-error-pathname}.
\end{defun}

\begin{defun}[Function]
file-error-pathname condition

  Accesses the offending pathname of a {\it condition} of type \cdf{file-error}.
\end{defun}

\begin{defun}[Type]
cell-error

  Errors that occur while accessing a location should inherit from this
  type. This is a subtype of \cdf{error}.  The initialization keyword \cd{:name} is supported to 
  initialize the slot, which can be accessed using \cdf{cell-error-name}.
\end{defun}

\begin{defun}[Function]
cell-error-name condition

  Accesses the offending cell name of a {\it condition} of type \cdf{cell-error}.
\end{defun}

\begin{defun}[Type]
unbound-variable

  The error that results from trying to access the value of an unbound
  variable should inherit from this type. This is a subtype of \cdf{cell-error}.
\end{defun}

\begin{defun}[Type]
undefined-function

  The error that results from trying to access the value of an undefined
  function should inherit from this type. This is a subtype of \cdf{cell-error}.
\end{defun}

\beforenoterule
\begin{sideremark}
    [Note: This remark was written well before the vote by X3J13 in June 1988 \issue{CLOS}
    to add the Common Lisp Object System to the forthcoming draft standard
    (see chapter~\ref{CLOS}) and the vote to integrate the Condition System
    and the Object System.  I have retained the remark here for reasons
    of historical interest.---GLS]

    Some readers may wonder why \cdf{undefined-function} is not defined to inherit
    from some condition such as \cdf{control-error}. The answer is that any such
    arrangement would require the presence of multiple inheritance---a 
    luxury we do not currently have (without resorting to \cdf{deftype}, which 
    we are currently avoiding). When the Common Lisp Object System comes
    into being, we might want to consider issues like this. 
    Multiple inheritance makes a lot of things in a condition system much
    more flexible to deal with.

   
\end{sideremark}
\afternoterule

\begin{defun}[Type]
arithmetic-error

  Errors that occur while doing arithmetic type operations should inherit
  from this type. This is a subtype of \cdf{error}. The initialization keywords \cd{:operation}
  and \cd{:operands} are supported to initialize the slots, which can be accessed
  using \cdf{arithmetic-error-operation} and \cdf{arithmetic-error-operands}.
\end{defun}

\begin{defun}[Function]
arithmetic-error-operation condition

  Accesses the offending operation of a condition of type \cdf{arithmetic-error}.
\end{defun}

\begin{defun}[Function]
arithmetic-error-operands condition

  Accesses a list of the offending operands in a condition of type 
  \cdf{arithmetic-error}.
\end{defun}

\begin{defun}[Type]
division-by-zero

  Errors that occur because of division by zero should inherit from this type.
  This is a subtype of \cdf{arithmetic-error}.
\end{defun}

\begin{defun}[Type]
floating-point-overflow

  Errors that occur because of floating-point overflow should inherit from
  this type. This is a subtype of \cdf{arithmetic-error}.
\end{defun}

\begin{defun}[Type]
floating-point-underflow

  Errors that occur because of floating-point underflow should inherit from
  this type. This is a subtype of \cdf{arithmetic-error}.
\end{defun}


   % Common Lisp Condition System

\appendix
%%%Chapter of Common Lisp Manual.  Copyright 1989 Guy L. Steele Jr.

%  +++  Final version of chapter  +++

\clearpage\def\pagestatus{FINAL PROOF}

\chapter{Series}
\label{SERIES}

\def\SU#1{${}_{#1}$}

\def\fooprime#1{#1'}

Author: Richard C. Waters

\begin{new}
\prefaceword A series is a data structure much like a sequence, with similar
kinds of operations.  The difference is that in many situations, operations
on series may be composed functionally and yet execute iteratively, without
the need to construct intermediate series values explicitly.  In this
manner, series provide both the clarity of a functional programming style
and the efficiency of an iterative programming style.

The remainder of this chapter consists of a description by Richard
C.~Waters of his work on an existing implementation of series.
This is the culmination of many years of design and use of this approach,
during which some 100,000 lines of application code have been written (by
about half a dozen people over the course of seven years) using the series
facility in nearly all iteration situations.  This includes one large
system (\cdf{KBEmacs}) of over 40,000 lines of code.

I have edited the chapter only very lightly to conform to the overall style
of this book.  Please see the Preface to this book for more information
about the genesis of the series approach and its relationship to the work
of X3J13.
\end{new}


\noindent\hbox to \textwidth{\hss---Guy L. Steele Jr.}
\vskip 8pt plus 3pt minus 2pt

\section{Introduction}

Series combine aspects of sequences, streams, and loops.  Like sequences,
series represent totally ordered multi-sets.  In addition, the series
functions have the same flavor as the sequence functions---namely, they
operate on whole series, rather than extracting elements to be
processed by other functions.  For instance, the series expression below
computes the sum of the positive elements in a list.
\begin{lisp}
(collect-sum (choose-if \#'plusp (scan '(1 -2 3 -4)))) {\EV} 4
\end{lisp}

Like streams, series can represent unbounded sets of elements and are
supported by lazy evaluation: each element of a series is not
computed until it is needed.  For instance, the series expression below
returns a list of the first five even natural numbers and their sum.  The
call on \cdf{scan-range} returns a series of all the even natural numbers.
However, since no elements beyond the first five are ever used, no elements
beyond the first five are ever computed.
\begin{lisp}
(let ((x (subseries (scan-range :from 0 :by 2) 0 5))) \\*
~~(values (collect x) (collect-sum x))) \\*
~~{\EV} (0 2 4 6 8) {\rm and} 20
\end{lisp}

Like sequences and unlike streams, a series is not altered
when its elements are accessed.  For instance, both users of \cdf{x}
above receive the same elements.

A totally ordered multi-set of elements can be represented in a loop by the
successive values of a variable.  This is extremely efficient, because it
avoids the need to store the elements as a group in any kind of data
structure.  In most situations, series expressions achieve this same high
level of efficiency, because they are automatically transformed into loops
before being evaluated or compiled.  For instance, the first expression
above is transformed into a loop like the following.
\begin{lisp}
(let ((sum 0)) \\*
~~(dolist (i '(1 -2  3 -4) sum) \\*
~~~~(when (plusp i) (setq sum (+ sum i))))) {\EV} 4
\end{lisp}

A wide variety of algorithms can be expressed clearly and succinctly with
series expressions.  In particular, at least 90 percent of the loops
programmers typically write can be replaced by series expressions that are
much easier to understand and modify, and just as efficient.   From this
perspective, the key feature of series is that they are supported by a rich
set of functions.  These functions more or less correspond to the union of
the operations provided by the sequence functions, the \cdf{loop} clauses,
and the vector operations of APL.

Some series expressions cannot be transformed into loops.
This is unfortunate, because while transformable series expressions are much more
efficient than equivalent expressions involving sequences or streams,
non-transformable series expressions are much less efficient.  Whenever a
problem comes up that blocks the transformation of a series expression, a
warning message is issued.  On the basis of information in the message, it is
usually easy to provide an efficient fix for the problem (see
section~\ref{SERIES-E-SECTION}).

Fortunately, most series expressions can be transformed into loops.  In
particular, pure expressions (ones that do not store series in variables)
can always be transformed.  As a result, the best approach for programmers
to take is simply to write series expressions without worrying about
transformability.  When problems come up, they can be ignored (since they
cannot lead to the computation of incorrect results) or dealt with on an
individual basis.

\beforenoterule
\begin{implementation}
The series functions and the theory
underlying them are described in greater detail
in~\cite{WATERS-SERIES-DESIGN,WATERS-SERIES-IMPLEMENTATION}.
These reports also discuss the algorithms required to transform series
expressions into loops and explain how to obtain a portable implementation.
\end{implementation}
\afternoterule

\section{Series Functions}
\label{SERIES-F-SECTION}

Throughout this chapter the notation \cd{S\SU{j}} is used to
denote the \emph{j\/}th element of the series \cdf{S}.  As in a list or
vector, the first element of a series has the subscript zero.

The \cd{\#} macro character syntax \cd{\#Z\emph{list}} denotes a series that contains
the elements of \emph{list}.  This syntax is also used when series are printed.
\begin{lisp}
(choose-if \#'symbolp \#Z(a 2 b)) {\EV} \#Z(a b)
\end{lisp}
Series are self-evaluating objects and the series data type is disjoint
from all other types.


\begin{defun}[Type specifier]
series element-type

The type specifier \cd{(series \emph{element-type})}
denotes the set of series whose elements are all
members of the type \emph{element-type}.
\end{defun}


\begin{defun}[Function]
series arg &rest args

The function \cdf{series} returns an unbounded series that endlessly repeats the
values of the arguments.  The second example below shows the preferred
method for constructing a bounded series.
\begin{lisp}
(series 'b 'c) {\EV} \#Z(b c b c b c ...) \\
(scan (list 'a 'b 'c)) {\EV} \#Z(a b c)
\end{lisp}
\end{defun}

\subsection{Scanners}

Scanners create series outputs based on non-series inputs.  Either they
operate based on some formula (for example, scanning a range of integers) or they
enumerate the elements in an aggregate data structure (for example, scanning the
elements in a list or array).


\begin{defun}[Function]
scan-range &key (:start 0) (:by 1) (:type 'number)
    :upto :below :downto :above :length

The function \cdf{scan-range} returns a series of numbers starting with the
\cd{:start} argument
(default integer \cd{0}) and counting up by the \cd{:by} argument (default
integer \cd{1}).  The \cd{:type} argument (default \cdf{number}) is
a type specifier indicating the type of numbers in the series
produced.  The \cd{:type} argument must be a (not necessarily proper) subtype of
\cdf{number}.  The \cd{:start} and \cd{:by} arguments must be of that type.

One of the last five arguments may be used
to specify the kind of end test to be used;
these are called \emph{termination arguments}.
If \cd{:upto} is specified, counting continues only so long as the
numbers generated are less than or equal to \cd{:upto}.  If 
\cd{:below} is specified, counting continues only so long as the numbers
generated are less than \cd{:below}.  If \cd{:downto} is specified,
counting continues only so long as the numbers generated are greater
than or equal to \cd{:downto}.  If \cd{:above} is specified,
counting continues only so long as the numbers generated are greater
than \cd{:above}.  If \cd{:length} is specified, it must be a
non-negative integer and the output series has this length.

If none
of the termination arguments are specified, the output has unbounded
length.  If more than one termination argument is specified, it is an error.

\begin{lisp}
(scan-range :upto 4) {\EV} \#Z(0 1 2 3 4) \\*
(scan-range :from 1 :by -1 :above -4) {\EV} \#Z(1 0 -1 -2 -3) \\
(scan-range :from .5 :by .1 :type 'float) {\EV} \#Z(.5 .6 .7 ...) \\*
(scan-range) {\EV} \#Z(0 1 2 3 4 5 6 ...)
\end{lisp}
\end{defun}

\begin{defun}[Function]
scan sequence \\
scan type sequence

\cdf{scan} returns a series containing the elements of \emph{sequence} in
order.  The \emph{type} argument is a type specifier indicating the type of
sequence to be scanned; it must be a (not necessarily proper) subtype of
\cdf{sequence}.  If \emph{type} is omitted, it defaults to \cdf{list}.
(This function exhibits an argument pattern that is unusual for Common
Lisp:  an ``optional'' argument preceding a required argument.  This
pattern cannot be expressed in the usual manner with \cd{\&optional}.  It
is indicated above by two definition lines, showing the two possible
argument patterns.)

If the \emph{sequence} is a list, it must be a proper list ending in \cdf{nil}.
Scanning is significantly more efficient if it can be determined at compile
time whether \emph{type} is a subtype of \cdf{list} or \cdf{vector} and for
vectors what the length of the vector is.
\begin{lisp}
(scan '(a b c)) {\EV} \#Z(a b c) \\*
(scan 'string "BAR") {\EV} \#Z(\#{\Xbackslash}B \#{\Xbackslash}A \#{\Xbackslash}R)
\end{lisp}
\end{defun}

\begin{defun}[Function]
scan-sublists list

\cdf{scan-sublists} returns a series containing the successive sublists of
\emph{list}.  The \emph{list} must be a proper list ending in \cdf{nil}.
\begin{lisp}
(scan-sublists '(a b c)) {\EV} \#Z((a b c) (b c) (c))
\end{lisp}
\end{defun}

\begin{defun}[Function]
scan-multiple type first-sequence &rest more-sequences

Several sequences can be scanned at once by using several calls on
\cdf{scan}.  Each call on \cdf{scan} will test to see when its sequence runs
out of elements and execution will stop as soon as any of the sequences are
exhausted.  Although very robust, this approach to scanning can be
inefficient.  In situations where it is known in
advance which sequence is the shortest, \cdf{scan-multiple} can be used to
obtain the same results more rapidly.
  
\cdf{scan-multiple} is similar to \cdf{scan} except that several sequences
can be scanned at once.  If there are \emph{n} sequence inputs,
\cdf{scan-multiple} returns \emph{n} series containing the elements of these
sequences.  It must be the case that none of the sequence inputs is shorter
than the first sequence.  All of the output series are the same length as
the first input sequence.  Extra elements in the other input sequences are
ignored.  Using \cdf{scan-multiple} is more efficient than using multiple
instances of \cdf{scan}, because \cdf{scan-multiple} only has to check for
the first input running out of elements.

If \emph{type} is of the form \cd{(values~$\emph{t}_1$~$\ldots$~$\emph{t}x_m$)}, then
there must be $\emph{m}$ sequence inputs and the \emph{i\/}th sequence must have type
$\emph{t}_{i}$.  Otherwise there can be any number of sequence inputs, each of which
must have type \emph{type}.
\begin{lisp}
(multiple-value-bind (data weights) \\*
~~~~(scan-multiple 'list '(1 6 3 2 8) '(2 3 3 3 2)) \\*
~~(collect (map-fn t \#'* data weights))) \\*
~~{\EV} (2 18 9 6 16)
\end{lisp}
\end{defun}

\begin{defun}[Function]
scan-lists-of-lists lists-of-lists &optional leaf-test \\
scan-lists-of-lists-fringe lists-of-lists &optional leaf-test

The argument \emph{lists-of-lists} is viewed as a tree where each
internal node is a non-empty list and the elements of the list are the
children of the node.  \cdf{scan-lists-of-lists} and
\cdf{scan-lists-of-lists-fringe} each scan \emph{lists-of-lists} in preorder
and return a series of its nodes.  \cdf{scan-lists-of-lists} returns every
node in the tree.  \cdf{scan-lists-of-lists-fringe} returns only the leaf
nodes.  

The scan proceeds as follows.  The argument \emph{lists-of-lists} can be any
Lisp object.  If \emph{lists-of-lists} is an atom or satisfies the predicate
\emph{leaf-test} (if present), it is a leaf node.  (The predicate can count
on being applied only to conses.) Otherwise, \emph{lists-of-lists} is a (not
necessarily proper) list.  The first element of \emph{lists-of-lists} is
recursively scanned in full, followed by the second and so on until a
non-cons \emph{cdr} is encountered.  Whether or not this final \emph{cdr} is
\cdf{nil}, it is ignored.
\begin{lisp}
(scan-lists-of-lists '((2) (nil))) \\*
~~{\EV} \#Z(((2) (nil)) (2) 2 (nil) nil) \\*
(scan-lists-of-lists-fringe '((2) (nil))) {\EV} \#Z(2 nil) \\*
(scan-lists-of-lists-fringe '((2) (nil)) \\*
~~~~~~~~~~~~~~~~~~~~~~~~~~~~\#'(lambda (e) (numberp (car e)))) \\*
~~{\EV} \#Z((2) nil)
\end{lisp}
\end{defun}

\begin{defun}[Function]
scan-alist a-list &optional (test \#'eql) \\
scan-plist plist \\
scan-hash table

When given an association list, a property list, or a hash table
(respectively), each of these functions produces two outputs:  a series of keys
\emph{K} and a series of the corresponding values \emph{V}.  Each key in the
input appears exactly once in the output, even if it appears more than once
in the input.  (The \emph{test} argument of \cdf{scan-alist} specifies the
equality test between keys; it defaults to \cdf{eql}.)
The two outputs have the same length.  Each
\emph{V}\SU{j} is the value returned by the appropriate accessing function
(\cdf{cdr} of \cdf{assoc}, \cdf{getf}, or \cdf{gethash}, respectively)
when given \emph{K}\SU{j}.  \cdf{scan-alist} and \cdf{scan-plist} scan keys
in the order
they appear in the underlying structure.  \cdf{scan-hash} scans keys in no
particular order.
\begin{lisp}
(scan-plist '(a 1 b 3)) {\EV} \#Z(a b) {\rm and} \#Z(1 3) \\*
(scan-alist '((a . 1) nil (a . 3) (b . 2))) \\*
~~{\EV} \#Z(a b) {\rm and} \#Z(1 2)
\end{lisp}
\end{defun}

\begin{defun}[Function]
scan-symbols &optional (package *package*)

\cdf{scan-symbols} returns a series, in no particular order, and possibly
containing duplicates, of the symbols accessible in \emph{package} (which
defaults to the current package).
\end{defun}

\begin{defun}[Function]
scan-file file-name &optional (reader \#'read)

\cdf{scan-file} opens the file named by the string \emph{file-name}
and applies the function \emph{reader} to it repeatedly until the end of the
file is reached.  \emph{Reader} must accept the standard input function
arguments \emph{input-stream}, \emph{eof-error-p}, and \emph{eof-value} as its
arguments.  (For instance, \emph{reader} can be \cdf{read},
\cdf{read-preserving-white-space}, \cdf{read-line}, or
\cdf{read-char}.) If omitted, \emph{reader} defaults to \cdf{read}.
\cdf{scan-file} returns a series of the values returned
by \emph{reader}, up to but not including the value returned
when the end of the file is reached.  The
file is correctly closed, even if an abort occurs. \end{defun}

\begin{defun}[Function]
scan-fn type init step &optional test

The higher-order function \cdf{scan-fn} supports the general concept of
scanning.  The \emph{type} argument is a type specifier indicating
the type of values returned by \emph{init} and \emph{step}.  The \cdf{values}
type specifier can be used for this argument
to indicate multiple types; however, \emph{type} cannot
indicate zero values.  If \emph{type} indicates $\emph{m}$ types
$\emph{t}_1, \ldots\,, \emph{t}_{m}$,
then \cdf{scan-fn} returns $\emph{m}$ series
\emph{T1}, $\ldots\,$, \emph{Tm}, where \emph{Ti} has
the type \cd{(series $\emph{t}_{i}$)}.
The arguments \emph{init}, \emph{step}, and \emph{test} are functions.

The \emph{init} must be of type 
\cd{(function () (values $\emph{t}_1$ ... $\emph{t}_{m}$))}.

The \emph{step} must be of type 
\cd{(function ($\emph{t}_1$ ... $\emph{t}_{m}$) (values $\emph{t}_1$ ... $\emph{t}_{m}$))}.

The \emph{test} (if present) must be of type 
\cd{(function ($\emph{t}_1$ ... $\emph{t}_{m}$) t)}.

The elements of the \emph{Ti} are computed as follows:
\begin{lisp}
(values \emph{T1}\SU{0} ... \emph{Tm}\SU{0}) = (funcall \emph{init}) \\*
(values \emph{T1}\SU{j} ... \emph{Tm}\SU{j}) = (funcall \emph{step} \emph{T1}\SU{(j-1)} ... \emph{Tm}\SU{(j-1)})
\end{lisp}

The outputs all have the same length.  If there is no \emph{test}, the
outputs have unbounded length.  If there is a \emph{test}, the outputs
consist of the elements up to, but not including, the first elements (with
index \emph{j}, say) for which the following termination test is not \cdf{nil}.
\begin{lisp}
(funcall \emph{test} \emph{T1}\SU{j} ... \emph{Tm}\SU{j})
\end{lisp}
It is guaranteed that \emph{step} will
not be applied to the elements that pass this termination test.

If \emph{init}, \emph{step}, or \emph{test} has side effects when
invoked, it can count on being called in the order indicated by the
equations above, with \emph{test} called just before \emph{step} on each
cycle.  However, given the lazy evaluation nature of series, these
functions will not be called until their outputs are actually used (if
ever).  In addition, no assumptions can be made about the relative order of
evaluation of these calls with regard to execution in other parts of a
given series expression.  The first example below scans down a list
stepping two elements at a time.  The second example generates two unbounded
series: the integers counting up from 1 and the sequence of partial
sums of the first \emph{i} integers.
\begin{lisp}
(scan-fn t \#'(lambda () '(a b c d)) \#'cddr \#'null) \\*
~~{\EV} \#Z((a b c d) (c d)) \\
\\
(scan-fn '(values integer integer) \\*
~~~~~~~~~\#'(lambda () (values 1 0)) \\*
~~~~~~~~~\#'(lambda (i sum) (values (+ i 1) (+ sum i)))) \\*
~~{\EV} \#Z(1 2 3 4 ...) {\rm and} \#Z(0 1 3 6 ...)
\end{lisp}
\end{defun}

\begin{defun}[Function]
scan-fn-inclusive type init step test

The higher-order function \cdf{scan-fn-inclusive} is the same as 
\cdf{scan-fn} except that the first set of elements for which \emph{test}
returns a non-null value is included in the output.  As with
\cdf{scan-fn}, it is guaranteed that \emph{step} will not be applied to the
elements for which \emph{test} is non-null.
\end{defun}

\subsection{Mapping}

By far the most common kind of series operation is mapping.  In cognizance
of this fact, four different ways are provided for specifying mapping:  one
fundamental form (\cdf{map-fn}) and three shorthand forms that are more
convenient in particular common situations.

\begin{defun}[Function]
map-fn type function &rest series-inputs

The higher-order function \cdf{map-fn} supports the general concept of
mapping.   The \emph{type} argument is a type specifier indicating
the type of values returned by \emph{function}.  The \cdf{values}
construct can be used to indicate multiple types; however, \emph{type}
cannot indicate zero values.  If \emph{type} indicates $m$ types
$t_1, \ldots\,, t_m$,
then \cdf{map-fn} returns $m$ series
\emph{T1},~$\ldots\,$,~\emph{Tm}, where \emph{Ti} has the
type \cd{(series~$t_i$)}.
The argument
\emph{function} is a function.   The remaining arguments (if any) are all
series.  Let these series be \emph{S1},~$\ldots\,$,~\emph{Sn} and suppose that
\emph{Si} has the type \cd{(series~$\emph{s}_{i}$)}.

The \emph{function} must be of type
\begin{lisp}
(function ($\emph{s}_1$ ... $\emph{s}_{n}$) (values $\emph{t}_1$ ... $\emph{t}_{m}$))
\end{lisp}

The length of each output is the same as the length of the shortest input.
If there are no bounded series inputs, the outputs are unbounded.
The elements of the \emph{Ti} are the results of applying \emph{function} to
the corresponding elements of the series inputs.
\begin{lisp}
(values \emph{T1}\SU{j} ... \emph{Tm}\SU{j}) {\EQ} (funcall \emph{function} \emph{S1}\SU{j} ... \emph{Sn}\SU{j})
\end{lisp}

If \emph{function} has side effects, it can count on being called first on
the \emph{Si}\SU{0}, then on the \emph{Si}\SU{1}, and so on.  However, given
the lazy evaluation nature of series, \emph{function} will not be called on
any group of input elements until the result is actually used (if ever).
In addition, no assumptions can be made about the relative order of
evaluation of the calls on \emph{function} with regard to execution in other parts of a
given series expression.
\begin{lisp}
(map-fn 'integer \#'+ \#Z(1 2 3) \#Z(4 5)) {\EV} \#Z(5 7) \\*
(map-fn t \#'gensym) {\EV} \#Z(\#:G3 \#:G4 \#:G5 ...) \\
(map-fn '(values integer rational) \#'floor \#Z(1/4 9/5 12/3)) \\*
~~{\EV} \#Z(0 1 4) {\rm and} \#Z(1/4 4/5 0)
\end{lisp}

The \cd{\#} macro character syntax \cd{\#M} makes it easy to specify uses of \cdf{map-fn}
where \emph{type} is \cdf{t} and the \emph{function} is a named
function.  The notation \cd{(\#M\emph{function}~...)} is an
abbreviation for \cd{(map-fn~t~\#'\emph{function}~...)}.  The form \emph{function} can
be the printed representation of any Lisp object.  The notation
\cd{\#M}\emph{function} can appear only in the
function position of a list.
\begin{lisp}
(collect (\#M1+ (scan '(1 2 3)))) {\EV} (2 3 4)
\end{lisp}
\end{defun}

\begin{defmac}
mapping ({({var | ({var}*)} value)}*) {declaration}* {form}*

The macro \cdf{mapping} makes it easy to specify uses of \cdf{map-fn}
where \emph{type} is \cdf{t} and the \emph{function} is a literal 
\cdf{lambda}.  The syntax of \cdf{mapping} is analogous to that of \cdf{let}.
The binding list specifies zero or more variables that are bound in parallel to
successive values of series.  The \emph{value} part of each pair is
an expression that must produce a
series.  The \emph{declarations} and \emph{forms} are
treated as the body of a \cdf{lambda} expression
that is mapped over the series values.  A series of the first values
returned by this \cdf{lambda} expression is returned as the result of 
\cdf{mapping}.
\begin{lisp}
(mapping ((x r) (y s)) ...) {\EQ} \\*
~~(map-fn t \#'(lambda (x y) ...) r s) \\
\\
(mapping ((x (scan '(2 -2 3)))) \\*
~~(expt (abs x) 3)) \\*
~~{\EV} \#Z(8 8 27)
\end{lisp}

The form \cdf{mapping} supports a special syntax that facilitates the
use of series functions returning multiple values.  Instead of being
a single variable, the variable part of a \emph{var-value} pair can be a list of
variables.  This list is treated the same way as the first argument to
\cdf{multiple-value-bind} and can be used to access the elements of
multiple series returned by a series function.
\begin{lisp}
(mapping (((i v) (scan-plist '(a 1 b 2)))) \\*
~~(list i v)) \\*
~~{\EV} \#Z((a 1) (b 2))
\end{lisp}
\end{defmac}

\begin{defmac}
iterate ({({var | ({var}*)} value)}*) {declaration}* {form}*

The form \cdf{iterate} is the same as \cdf{mapping}, except that after
mapping the \emph{forms} over the \emph{values}, the results are discarded and
\cdf{nil} is returned.
\begin{lisp}
(let ((item (scan '((1) (-2) (3))))) \\*
~~(iterate ((x (\#Mcar item))) \\*
~~~~(if (plusp x) (prin1 x)))) \\*
~~{\EV} nil {\rm (after printing ``\cd{13}'')}
\end{lisp}

To a first approximation, \cdf{iterate} and \cdf{mapping} differ in the same
way as \cdf{mapc} and \cdf{mapcar}.  In particular, like \cdf{mapc},
\cdf{iterate} is intended to be used in situations where the \emph{forms} are
being evaluated for side effects rather than for their results.  However, given
the lazy evaluation semantics of series, the difference between
\cdf{iterate} and \cdf{mapping} is more than just a question of efficiency.

If \cdf{mapcar} is used in a situation where the output is not used, time is
wasted unnecessarily creating the output list.  However, if \cdf{mapping} is
used in a situation where the output is not used, no computation is
performed, because series elements are not computed until they are used.
Thus \cdf{iterate} can be thought of as a declaration that the indicated
computation is to be performed even though the output is not used for
anything.
\end{defmac}

\subsection{Truncation and Other Simple Transducers}

Transducers compute series from series and form the heart of most series
expressions.  Mapping is by far the most common transducer.   This section
presents a number of additional simple transducers.


\begin{defun}[Function]
cotruncate &rest series-inputs \\
until bools &rest series-inputs \\
until-if pred &rest series-inputs

Each of these functions accepts one or more series inputs {\it
S1},~$\ldots\,$,~\emph{Sn} as its \cd{\&rest} argument and returns $\emph{n}$ series
outputs \emph{T1},~$\ldots\,$,~\emph{Tn} that contain the same elements in the same
order---that is, \emph{Ti\SU{j}=Si\SU{j}}.
Let $\emph{k}$ be the length of the
shortest input \emph{Si}.  \cdf{cotruncate} truncates the series so that
each output has length $\emph{k}$.  Let $\emph{k}'$ be the position of the first element
in the boolean series \emph{bools} that is not \cdf{nil} or, if every
element is \cdf{nil}, the length of \emph{bools}.  \cdf{until} truncates the
series so that each output has length \cd{(min~\emph{k}~$\fooprime{\emph{k}}$)}.
Let ${it k}''$ be the position of the first element in \emph{S1} such that 
\cd{(\emph{pred}~\emph{S1\SU{\fooprime{\fooprime{k}}}})}
is not \cdf{nil} or, if there is no such
element, the length of \emph{S1}.  \cdf{until-if} truncates the series so
that each output has length \cd{(min~\emph{k}~$\fooprime{\fooprime{\emph{k}}}$)}.
\begin{lisp}
(cotruncate \#Z(1 2 -3 4) \#Z(a b c)) \\*
~~{\EV} \#Z(1 2 -3) {\rm and} \#Z(a b c) \\
(until \#Z(nil nil t nil) \#Z(1 2 -3 4) \#Z(a b c)) \\*
~~{\EV} \#Z(1 2) {\rm and} \#Z(a b) \\
(until-if \#'minusp \#Z(1 2 -3 4) \#Z(a b c)) \\*
~~{\EV} \#Z(1 2) {\rm and} \#Z(a b)
\end{lisp}
\end{defun}

\begin{defun}[Function]
previous items &optional (default nil) (amount 1)

The series returned by \cdf{previous} is the same as the input series
\emph{items} except that it is shifted to the right by the positive
integer \emph{amount}.  The shifting is done by inserting \emph{amount}
copies of \emph{default} before \emph{items} and discarding \emph{amount}
elements from the end of \emph{items}.
\begin{lisp}
(previous \#Z(10 11 12) 0) {\EV} \#Z(0 10 11)
\end{lisp}
\end{defun}

\begin{defun}[Function]
latch items &key :after :before :pre :post

The series returned by \cdf{latch} is the same as the input series
\emph{items} except that some of the elements are replaced by other
values.  \cdf{latch} acts like a \emph{latch} electronic circuit
component.  Each input element causes the creation of a corresponding
output element.  After a specified number of non-null input elements
have been encountered, the latch is triggered and the output mode is
permanently changed.

The \cd{:after} and \cd{:before} arguments specify the latch point.
The latch point is just after the \cd{:after}-th non-null element in
\emph{items} or just before the \cd{:before}-th non-null element.  If
neither \cd{:after} nor \cd{:before} is specified, an \cd{:after}
of \cd{1} is assumed.  If both are specified, it is an error.

If a \cd{:pre} is specified, every element prior to the latch point
is replaced by this value.  If a \cd{:post} is specified, every element
after the latch point is replaced by this value.  If neither is
specified, a \cd{:post} of \cdf{nil} is assumed.
\begin{lisp}
(latch \#Z(nil c nil d e)) {\EV} \#Z(nil c nil nil nil) \\*
(latch \#Z(nil c nil d e) :before 2 :post t) {\EV} \#Z(nil c nil t t)
\end{lisp}
\end{defun}

\begin{defun}[Function]
collecting-fn type init function &rest series-inputs

The higher-order function \cdf{collecting-fn} supports the general concept of
a simple transducer with internal state.  The \emph{type} argument is a type
specifier indicating the type of values returned by \emph{function}.
The \cdf{values} construct can be used to indicate multiple types; however,
\emph{type} cannot indicate zero values.  If \emph{type} indicates $\emph{m}$ types
$\emph{t}_1, \ldots\,, \emph{t}_{m}$,
then \cdf{collecting-fn} returns $\emph{m}$ series {\it
T1},~$\ldots\,$,~\emph{Tm}, where \emph{Ti} has the
type \cd{(series~$\emph{t}_{i}$)}.  The
arguments \emph{init} and \emph{function} are functions.  The remaining
arguments (if any) are all series.  Let these series be {\it
S1},~$\ldots\,$,~\emph{Sn} and suppose that \emph{Si} has the type
\cd{(series~$\emph{s}_{i}$)}.

The \emph{init} must be of type 
\cd{(function () (values $\emph{t}_1$ ... $\emph{t}_{m}$))}.

The \emph{function} must be of type
\begin{lisp}
(function ($\emph{t}_1$ ... $\emph{t}_{m}$ $\emph{s}_1$ ... $\emph{s}_{n}$) (values $\emph{t}_1$ ... $\emph{t}_{m}$))
\end{lisp}

The length of each output is the same as the length of the shortest input.
If there are no bounded series inputs, the outputs are unbounded.
The elements of the \emph{Ti} are computed as follows:
\begin{lisp}
(values \emph{T1}\SU{0} ... \emph{Tm}\SU{0}) {\EQ} \\*
~~(multiple-value-call \emph{function} (funcall  \emph{init}) \emph{S1}\SU{0} ... \emph{Sn}\SU{0}) \\
\\
(values \emph{T1}\SU{j} ... \emph{Tm}\SU{j}) {\EQ} \\*
~~(funcall \emph{function} \emph{T1}\SU{(j-1)} ... \emph{Tm}\SU{(j-1)} \emph{S1}\SU{j} ... \emph{Sn}\SU{j}) 
\end{lisp}

If \emph{init} or \emph{function} has side effects, it can count on
being called in the order indicated by the equations above.  However,
given the lazy evaluation nature of series, these functions will not be called
until their outputs are actually used (if ever).  In addition, no
assumptions can be made about the relative order of evaluation of these
calls with regard to execution in other parts of a given series expression.
The second example below computes a series of partial sums of the numbers in
an input series.  The third example computes two output series: the
partial sums of its first input and the partial products of its second
input.
\begin{lisp}
(defun running-averages (float-list) \\*
~~(multiple-value-call \#'map-fn \\*
~~~~'float \#'/ \\*
~~~~(collecting-fn '(values float integer) \\*
~~~~~~~~~~~~~~~~~~~\#'(lambda () (values 0.0 0) \\*
~~~~~~~~~~~~~~~~~~~\#'(lambda (s n x) (values (+ s x) (+ n 1)))) \\*
~~~~~~~~~~~~~~~~~~~float-list)))
\end{lisp}
\begin{lisp}
(collecting-fn 'integer \#'(lambda () 0) \#'+ \#Z(1 2 3)) \\*
~~{\EV} \#Z(1 3 6) \\
\\
(collecting-fn '(values integer integer) \\*
~~~~~~~~~~~~~~~\#'(lambda () (values 0 1)) \\*
~~~~~~~~~~~~~~~\#'(lambda (sum prod x y) \\*
~~~~~~~~~~~~~~~~~~~(values (+ sum x) (* prod y))) \\*
~~~~~~~~~~~~~~~\#Z(4 6 8)  \\*
~~~~~~~~~~~~~~~\#Z(1 2 3)) \\*
~~{\EV} \#Z(4 10 18) {\rm and} \#Z(1 2 6)
\end{lisp}
\end{defun}

\subsection{Conditional and Other Complex Transducers}
\label{SERIES-OL-SECTION}

This section presents a number of complex transducers, including ones that
support conditional computation.


\begin{defun}[Function]
choose bools &optional (items bools) \\
choose-if pred items

Each of these functions takes in a series of elements (\emph{items}) and
returns a series containing the same elements in the same order, but with
some elements removed.  \cdf{choose} removes \emph{items}\SU{j} if {\it
bools}\SU{j} is \cdf{nil} or $\emph{j}$ is beyond the end of \emph{bools}.  If {\it
items} is omitted, \cdf{choose} returns the non-null elements of {\it
bools}.  \cdf{choose-if} removes \emph{items}\SU{j} if 
\cd{(\emph{pred}~\emph{items}\SU{j})} is \cdf{nil}.
\begin{lisp}
(choose \#Z(t nil t nil) \#Z(a b c d)) {\EV} \#Z(a c) \\*
(collect-sum (choose-if \#'plusp \#Z(-1 2 -3 4))) {\EV} 6
\end{lisp}
\end{defun}

\begin{defun}[Function]
expand bools items &optional (default nil)

\cdf{expand} is a quasi-inverse of \cdf{choose}.  The output contains the
elements of the input series \emph{items} spread out into the positions
specified by the non-null elements
in \emph{bools}---that is, \emph{items}\SU{j}
is in the position occupied by the \emph{j\/}th non-null element in \emph{bools}.
The other positions in the output are occupied by \emph{default}.  The
output stops as soon as \emph{bools} runs out of elements or a non-null
element in \emph{bools} is encountered for which there is no corresponding
element in \emph{items}.
\begin{lisp}
(expand \#Z(nil t nil t t) \#Z(a b c)) {\EV} \#Z(nil a nil b c) \\*
(expand \#Z(nil t nil t t) \#Z(a)) {\EV} \#Z(nil a nil)
\end{lisp}
\end{defun}

\begin{defun}[Function]
split items &rest test-series-inputs \\
split-if items &rest test-predicates

These functions are like \cdf{choose} and \cdf{choose-if} except that
instead of producing one restricted output, they partition the input series
\emph{items} between several outputs.  If there are $\emph{n}$ test inputs
following \emph{items}, then there are $\emph{n}+1$ outputs.  Each input element is
placed in exactly one output series, depending on the outcome of a sequence
of tests.  If the element \emph{items}\SU{j} fails the first $\emph{k}-1$ tests and
passes the \emph{k\/}h test, it is put in the \emph{k\/}th output.
If \emph{items}\SU{j}
fails every test, it is placed in the last output.  In addition, all output
stops as soon as any series input runs out of elements.  The test inputs to
\cdf{split} are series of values; \emph{items}\SU{j} passes the \emph{k\/}th test
if the \emph{j\/}th element of the \emph{k\/}th test series is not \cdf{nil}.  The test
inputs to \cdf{split-if} are predicates; \emph{items}\SU{j} passes the \emph{k\/}th
test if the \emph{k\/}th test predicate returns non-null when applied to {\it
items}\SU{j}.
\begin{lisp}
(split \#Z(-1 2 3 -4) \#Z(t nil nil t)) \\*
~~{\EV} \#Z(-1 -4) {\rm and} \#Z(2 3) \\
(multiple-value-bind (+x -x) (split-if  \#Z(-1 2 3 -4) \#'plusp) \\*
~~(values (collect-sum +x) (collect-sum -x))) \\*
~~{\EV} 5 {\rm and} -5
\end{lisp}
\end{defun}

\begin{defun}[Function]
catenate &rest series-inputs

\cdf{catenate} combines two or more series into one long series by appending
them end to end.  The length of the output is the sum of the lengths of the
inputs.
\begin{lisp}
(catenate \#Z(b c) \#Z() \#Z(d)) {\EV} \#Z(b c d)
\end{lisp}
\end{defun}

\begin{defun}[Function]
subseries items start &optional below

\cdf{subseries} returns a series containing the elements of the input
series \emph{items} indexed by the non-negative integers from \emph{start} up
to, but not including, \emph{below}.  If \emph{below} is omitted or greater
than the length of \emph{items}, the output goes all the way to the end
of \emph{items}.
\begin{lisp}
(subseries \#Z(a b c d) 1) {\EV} \#Z(b c d) \\*
(subseries \#Z(a b c d) 1 3) {\EV} \#Z(b c)
\end{lisp}
\end{defun}

\begin{defun}[Function]
positions bools

\cdf{positions} returns a series of the indices of the non-null elements in
the series input \emph{bools}.
\begin{lisp}
(positions \#Z(t nil t 44)) {\EV} \#Z(0 2 3)
\end{lisp}
\end{defun}

\begin{defun}[Function]
mask monotonic-indices

\cdf{mask} is a quasi-inverse of \cdf{positions}.  The series input {\it
monotonic-indices} must be a strictly increasing series of non-negative
integers.  The output, which is always unbounded, contains \cdf{t} in the
positions specified by \emph{monotonic-indices} and \cdf{nil} everywhere else.
\begin{lisp}
(mask \#Z(0 2 3)) {\EV} \#Z(t nil t t nil nil ...) \\*
(mask \#Z()) {\EV} \#Z(nil nil ...) \\*
(mask (positions \#Z(nil a nil b nil))) \\*
~~{\EV} \#Z(nil t nil t nil ...)
\end{lisp}
\end{defun}


\begin{defun}[Function]
mingle items1 items2 comparator

The series returned by \cdf{mingle} contains all and only the elements of
the two input series.  The length of the output is the sum of the lengths
of the inputs and is unbounded if either input is unbounded.  The order of
the elements remains unchanged; however, the elements from the two inputs
are stably intermixed under the control of the \emph{comparator}.

The \emph{comparator} must accept two arguments and return non-null if and only
if its first argument is strictly less than its second argument (in some
appropriate sense).  At each step, the \emph{comparator} is used to compare
the current elements in the two series.  If the current element from {\it
items2} is strictly less than the current element from \emph{items1}, the
current element is removed from \emph{items2} and transferred to the output.
Otherwise, the next output element comes from \emph{items1}.
\begin{lisp}
(mingle \#Z(1 3 7 9) \#Z(4 5 8) \#'<) {\EV} \#Z(1 3 4 5 7 8 9) \\*
(mingle \#Z(1 7 3 9) \#Z(4 5 8) \#'<) {\EV} \#Z(1 4 5 7 3 8 9)
\end{lisp}
\end{defun}

\begin{defun}[Function]
chunk m n items

This function has the effect of breaking up the input series \emph{items} into
(possibly overlapping) chunks of length \emph{m}.  The starting positions of successive chunks differ
by \emph{n}.  The inputs \emph{m} and \emph{n} must both be positive integers.

\cdf{chunk} produces \emph{m} output series.  The \emph{i\/}th chunk provides
the \emph{i\/}th element for
each of  the \emph{m} outputs.  Suppose that the length of \emph{items} is \emph{l}.
The length of
each output is $\lfloor1+(\emph{l}-\emph{m})/\emph{n}\rfloor$.
The \emph{i\/}th element of the \emph{k\/}th output is the
$(\emph{i}*\emph{n}+\emph{k})$th element of \emph{items} (\emph{i} and $\emph{k}$ counting from zero).  

Note that if $\emph{l}<\emph{m}$, there will be no
output elements, and if $\emph{l}-\emph{m}$ is not a multiple of \emph{n},
the last few input elements will
not appear in the output.  If $\emph{m}\ge \emph{n}$,
one can guarantee that the last chunk will contain the last
element of \emph{items} by catenating $\emph{n}-1$
copies of an appropriate padding value to the end of \emph{items}.

The first example below shows \cdf{chunk}
being used to compute a moving average.  The second example shows
\cdf{chunk} being used to convert a property list into an association list.
\begin{lisp}
(mapping (((xi xi+1 xi+2) (chunk 3 1 \#Z(1 5 3 4 5 6)))) \\* 
~~(/ (+ xi xi+1 xi+2) 3)) \\*
~~{\EV} \#Z(3 4 4 5)
\\
(collect \\*
~~(mapping (((prop val) (chunk 2 2 (scan '(a 2 b 5 c 8))))) \\*
~~~~(cons prop val))) \\*
~~{\EV} ((a . 2) (b . 5) (c . 8))
\end{lisp}
\end{defun}

\subsection{Collectors}

Collectors produce non-series outputs based on series inputs.  They either
create a summary value based on some formula (the sum, for example) or collect the
elements of a series in an aggregate data structure (such as a list).

\begin{defun}[Function]
collect-first items &optional (default nil) \\
collect-last items &optional (default nil) \\
collect-nth n items &optional (default nil)

Given a series \emph{items}, these functions return the first element, the
last element, and the \emph{n\/}th element, respectively.  If \emph{items} has
no elements (or no \emph{n\/}th element), \emph{default} is returned.
If \emph{default} is not specified, then \cdf{nil} is used for \emph{default}.
\begin{lisp}
(collect-first \#Z() 'z) {\EV} z \\*
(collect-last \#Z(a b c)) {\EV} c \\*
(collect-nth 1 \#Z(a b c)) {\EV} b
\end{lisp}
\end{defun}

\begin{defun}[Function]
collect-length items

\cdf{collect-length} returns the number of elements in a series.
\begin{lisp}
(collect-length \#Z(a b c)) {\EV} 3
\end{lisp}
\end{defun}


\begin{defun}[Function]
collect-sum numbers &optional (type 'number)

\cdf{collect-sum} returns the sum of the elements in a series of numbers.
The \emph{type} is a type specifier that indicates the type of sum
to be created.  If \emph{type} is not specified, then \cdf{number} is used for
the \emph{type}.
If there are no elements in the input, a zero (of the
appropriate type) is returned.
\begin{lisp}
(collect-sum \#Z(1.1 1.2 1.3)) {\EV} 3.6 \\*
(collect-sum \#Z() 'complex) {\EV} \#C(0 0)
\end{lisp}
\end{defun}


\begin{defun}[Function]
collect-max numbers \\
collect-min numbers

Given a series of non-complex numbers, these functions compute the maximum
element and the minimum element, respectively.  If there are no elements in
the input, \cdf{nil} is returned.
\begin{lisp}
(collect-max \#Z(2 1 4 3)) {\EV} 4 \\*
(collect-min \#Z(1.2 1.1 1.4 1.3)) {\EV} 1.1 \\*
(collect-min \#Z()) {\EV} nil
\end{lisp}
\end{defun}


\begin{defun}[Function]
collect-and bools

\cdf{collect-and} returns the \cdf{and} of the elements in a series.  As
with the macro \cdf{and}, \cdf{nil} is returned if any element of {\it
bools} is \cdf{nil}.  Otherwise, the last element of \emph{bools} is
returned.  The value \cdf{t} is returned if there are no elements in {\it
bools}.
\begin{lisp}
(collect-and \#Z(a b c)) {\EV} c \\*
(collect-and \#Z(a nil c)) {\EV} nil
\end{lisp}
\end{defun}

\begin{defun}[Function]
collect-or bools

\cdf{collect-or} returns the \cdf{or} of the elements in a series.  As with
the macro \cdf{or}, \cdf{nil} is returned if every element of {\it
bools} is \cdf{nil}.  Otherwise, the first non-null element of \emph{bools}
is returned.  The value \cdf{nil} is returned if there are no elements in
\emph{bools}.
\begin{lisp}
(collect-or \#Z(nil b c)) {\EV} b \\*
(collect-or \#Z()) {\EV} nil
\end{lisp}
\end{defun}

\begin{defun}[Function]
collect items \\
collect type items

\cdf{collect} returns a sequence containing the elements of the series {\it
items}.  The \emph{type} is a type specifier indicating the type of sequence
to be created.  It must be either a proper subtype of \cdf{sequence} or the
symbol \cdf{bag}.  If \emph{type} is omitted, it defaults to \cdf{list}.
(This function exhibits an argument pattern that is unusual for Common
Lisp:  an ``optional'' argument preceding a required argument.  This
pattern cannot be expressed in the usual manner with \cd{\&optional}.  It
is indicated above by two definition lines, showing the two possible
argument patterns.)

If the \emph{type} is \cdf{bag}, a list is created with the elements in
whatever order can be most efficiently obtained.  Otherwise, the order of
the elements in the sequence is the same as the order in \emph{items}.  If
\emph{type} specifies a length (that is, of a vector) this length must be
greater than or equal to the length of \emph{items}.

The \emph{n\/}th element of \emph{items} is
placed in the \emph{n\/}th slot of the sequence produced.  Any unneeded slots are
left in their initial state.  Collecting is significantly more efficient if
it can be determined at compile time whether \emph{type} is a subtype of
\cdf{list} or \cdf{vector} and for vectors what the length of the vector is.
\begin{lisp}
(collect \#Z(a b c)) {\EV} (a b c) \\*
(collect 'bag \#Z(a b c)) {\EV} (c a b) {\rm or} (b a c) {\rm or $\ldots$} \\*
(collect '(vector integer 3) \#Z(1 2 3)) {\EV} \#(1 2 3)
\end{lisp}
\end{defun}

\begin{defun}[Function]
collect-append sequences \\
collect-append type sequences

Given a series of sequences, \cdf{collect-append} returns a new sequence by
concatenating these sequences together in order.  The \emph{type} is a type
specifier indicating the type of sequence created and must be a proper
subtype of \cdf{sequence}.  If \emph{type} is omitted, it defaults to
\cdf{list}.  (This function exhibits an argument pattern that is unusual for Common
Lisp:  an ``optional'' argument preceding a required argument.  This
pattern cannot be expressed in the usual manner with \cd{\&optional}.  It
is indicated above by two definition lines, showing the two possible
argument patterns.)

It must be possible for every element of every sequence in the input series
to be an element of a sequence of type \emph{type}.  The result does not
share any structure with the sequences in the input.
\begin{lisp}
(collect-append \#Z((a b) nil (c d))) {\EV} (a b c d) \\*
(collect-append 'string \#Z("a " "big " "cat")) {\EV} "a big cat"
\end{lisp}
\end{defun}

\begin{defun}[Function]
collect-nconc lists

\cdf{collect-nconc} \cdf{nconc}s the elements of the series {\it
lists} together in order and returns the result.  This is the same as
\cdf{collect-append} except that the input must be a series of lists,
the output is always a list, the concatenation is done rapidly by
destructively modifying the input elements, and therefore the output
shares all of its structure with the input elements.
\end{defun}

\begin{defun}[Function]
collect-alist keys values \\
collect-plist keys values \\
collect-hash keys values &key :test :size :rehash-size :rehash-threshold

\noindent Given a series of keys and a series of corresponding values, these
functions return an association list, a property list, and a hash table,
respectively.  Following the order of the input, each \emph{keys}\SU{j}-{\it
values}\SU{j} pair is entered into the output so that it overrides all
earlier associations.  If one of the input series is longer than the other,
the extra elements are ignored.  The keyword arguments of 
\cdf{collect-hash} specify attributes of the hash table produced and have the
same meanings as the arguments to \cdf{make-hash-table}.
\begin{lisp}
(collect-alist \#Z(a b c) \#Z(1 2)) {\EV} ((b . 2) (a . 1)) \\*
(collect-plist \#Z(a b c) \#Z(1 2)) {\EV} (b 2 a 1) \\*
(collect-hash \#Z() \#Z(1 2) :test \#'eq) {\EV} {\rm $\langle$an empty hash table$\rangle$}
\end{lisp}
\end{defun}

\begin{defun}[Function]
collect-file file-name items &optional (printer \#'print)

This creates a file named \emph{file-name} and writes the
elements of the series \emph{items} into it using the function {\it
printer}.  \emph{Printer} must accept two inputs: an object
and an output stream.  (For instance, \emph{printer} can be \cdf{print},
\cd{prin1}, \cdf{princ}, \cdf{pprint}, \cdf{write-char},
\cdf{write-string}, or \cdf{write-line}.)
If omitted, \emph{printer} defaults to \cdf{print}.  The value \cdf{t} is
returned.  The file is correctly closed, even if an abort occurs.
\end{defun}


\begin{defun}[Function]
collect-fn type init function &rest series-inputs

The higher-order function \cdf{collect-fn} supports the general concept of
collecting.  It is identical to \cdf{collecting-fn} except that it
returns only the last element of each series computed.  If there are no elements
in these series, the values returned by \emph{init} are passed on directly
as the output of \cdf{collect-fn}.
\begin{lisp}
(collect-fn 'integer \#'(lambda () 0) \#'+ \#Z(1 2 3)) {\EV} 6 \\*
(collect-fn 'integer \#'(lambda () 0) \#'+ \#Z()) {\EV} 0 \\*
(collect-fn 'integer \#'(lambda () 1) \#'* \#Z(1 2 3 4 5)) {\EV} 120
\end{lisp}
\end{defun}

\subsection{Alteration of Series}

Series that come from scanning data structures such as lists and vectors
are closely linked to these structures.  The function 
\cdf{alter} can be used to modify the underlying data structure with
reference to the series derived from it. (Conversely, it is possible to
modify a series by destructively modifying the data structure it is derived
from.  However, given the lazy evaluation nature of series, the effects of
such modifications can be very hard to predict.  As a result, this kind of
modification is inadvisable.)

\begin{defun}[Function]
alter destinations items

\cdf{alter} changes the series \emph{destinations} so that it contains
the elements in the series \emph{items}.  More importantly, in the
manner of \cdf{setf}, the data structure that underlies {\it
destinations} is changed so that if the series \emph{destinations} were
to be regenerated, the new values would be obtained.  The alteration
process stops as soon as either input runs out of elements.  The value
\cdf{nil} is always returned. In the example below each negative element in
a list is replaced with its square.
\begin{lisp}
(let* ((data (list 1 -2 3 4 -5 6)) \\*
~~~~~~~(x (choose-if \#'minusp (scan data)))) \\*
~~(alter x (\#M* x x)) \\*
~~data) \\*
~~{\EV} (1 4 3 4 25 6)
\end{lisp}

\cdf{alter} can be applied only to series that are \emph{alterable}.  
\cdf{scan}, \cdf{scan-alist}, \cdf{scan-multiple}, \cdf{scan-plist}, and
\cdf{scan-lists-of-lists-fringe} produce alterable series.  
However, the alterability of
the output of
\cdf{scan-lists-of-lists-fringe}
is incomplete.  If
\cdf{scan-lists-of-lists-fringe}
is applied to an object that is a leaf,
altering the output series does not change the object.

In general, the output of a transducer is alterable as long as the elements
of the output come directly from the elements of an input that is
alterable.  In particular, the outputs of \cdf{choose}, \cdf{choose-if},
\cdf{split}, \cdf{split-if}, \cdf{cotruncate}, \cdf{until}, \cdf{until-if},
and \cdf{subseries} are alterable as long as the corresponding inputs are
alterable.
\end{defun}


\begin{defun}[Function]
to-alter items alter-fn &rest args

Given a series \emph{items}, \cdf{to-alter} returns an alterable series {\it
A} containing the same elements.  The argument \emph{alter-fn} is a
function.  The remaining arguments are all series.  Let these series be
\emph{S1},~$\ldots\,$,~\emph{Sn}.  If there are $\emph{n}$ arguments after \emph{alter-fn},
\emph{alter-fn} must accept $\emph{n}+1$ inputs.  If \cd{(alter~\emph{A}~\emph{B})} is
later encountered, the expression
\cd{(map-fn~t~\emph{alter-fn}~\emph{B}~\emph{S1}~...~\emph{Sn})} is implicitly
evaluated.  For each
element in \emph{B}, \emph{alter-fn} should make appropriate changes in the
data structure underlying \emph{A}.

As an example, consider the following definition of a series function
that scans the elements of a list.  Alteration is performed by
changing cons cells in the list being scanned.
\begin{lisp}
(defun scan-list (list) \\*
~~(declare (optimizable-series-function)) \\*
~~(let ((sublists (scan-sublists list))) \\*
~~~~(to-alter (\#Mcar sublists) \\*
~~~~~~~~~~~~~~\#'(lambda (new parent) (setf (car parent) new)) \\*
~~~~~~~~~~~~~~sublists)))
\end{lisp}
\end{defun}

\section{Optimization}\label{SERIES-E-SECTION}

Series expressions are transformed into loops by pipelining them---the
computation is converted from a form where entire series are computed one
after the other to a form where the series are incrementally computed in
parallel.  In the resulting loop, each individual element is computed just
once, used, and then discarded before the next element is computed.  For
this pipelining to be possible, a number of restrictions have to be
satisfied.  Before these restrictions are explained, it will be useful to consider
a related issue.

The composition of two series functions cannot be pipelined unless the
destination function consumes series elements in the same order that the source
function produces them.  Taken together, the series functions guarantee
that this will always be true, because they all follow the same fixed
processing order.  In particular, they are all \emph{preorder\/}
functions---they process the elements of their series inputs and outputs in
ascending order starting with the first element.  Further, while it is easy
for users to define new series functions, it is impossible to define one
that is not a preorder function.

It turns out that most series operations can easily be implemented in a
preorder fashion, the most notable exceptions being reversal and sorting.  As
a result, little is lost by outlawing non-preorder series functions.  If some
non-preorder operation has to be applied to a series, the series can be
collected into a list or vector and the operation applied to this new data
structure.  (This is inefficient, but no less efficient than what would be
required if non-preorder series functions were supported.)

\subsection{Basic Restrictions}

The transformation of series expressions into loops is required to occur at
some time before compiled code is actually run.  Optimization may or may
not be applied to interpreted code.  If any of the restrictions described
below are violated, optimization is not possible.  In this situation, a
warning message is issued at the time optimization is attempted and the
code is left unoptimized.  This is not a fatal error and does not prevent
the correct results from being computed.  However, given the large
improvements in efficiency to be gained, it is well worth fixing any
violations that occur.  This is usually easy to do.

\begin{defun}[Variable]
*suppress-series-warnings*

If this variable is set (or bound) to anything other than its default
value of \cdf{nil}, warnings about conditions that block the optimization
of series expressions are suppressed.
\end{defun}

Before the restrictions on series expressions are discussed, it will be useful to
define precisely what is meant by the term \emph{series expression}.  This
term is semantic rather than syntactic in nature. Imagine a program
converted from Lisp code into a data flow graph.  In a data flow graph,
functions are represented as boxes, and both control flow and data flow are
represented as arrows between the boxes.  Constructs such as \cdf{let} and
\cdf{setq} are converted into patterns of data flow arcs.  Control
constructs such as \cdf{if} and \cdf{loop} are converted into patterns of
control flow arcs.  Suppose further that all loops have been converted
into tail recursions so that the graph is acyclic.

A series expression is a subgraph of the data flow graph for a program that
contains a group of interacting series functions.  More specifically, given
a call \emph{f} on a series function, the series expression \emph{E} containing it is
defined as follows.  \emph{E} contains \emph{f}.  Every function using a series
created by a function in \emph{E} is in \emph{E}.  Every function computing a series
used by a function in \emph{E} is in \emph{E}.  Finally, suppose that two functions
\emph{g} and \emph{h} are in \emph{E} and that there is a data flow path consisting of
series and/or non-series data flow arcs from \emph{g} to \emph{h}.  Every function
touched by this path (be it a series function or not) is in~\emph{E}.

{\bf For optimization to be possible, series expressions have to be
statically analyzable}.  As with most other optimization processes, a
series expression cannot be transformed into a loop at compile time, unless
it can be determined at compile time exactly what computation is being
performed.  This places a number of relatively minor limits on what can be
written.  For example, for optimization to be possible the type arguments
to higher-order functions such as \cdf{map-fn} and \cdf{collecting-fn} have
to be quoted constants.  Similarly, the numeric arguments to \cdf{chunk}
have to be constants.  In addition, if \cdf{funcall} is used to call a
series function, the function called has to be of the
form \cd{(function~...)}.

{\bf For optimization to be possible, every series created within a series
expression must be used solely inside the expression}.  If a series is
transmitted outside of the expression that creates it, it has to be
physically represented as a whole.  This is incompatible with the
transformations required to pipeline the creating expression. To avoid this
problem, a series must not be returned as a result of a series expression
as a whole, assigned to a free variable, assigned to a special variable, or
stored in a data structure.  A corollary of the last point is that when
defining new optimizable series functions, series cannot be passed into
\cd{\&rest} arguments.  Further, optimization is blocked if a series is
passed as an argument to an ordinary Lisp function.  Series can be
passed only to the series functions in section~\ref{SERIES-F-SECTION} and to new series
functions defined using the declaration \cdf{optimizable-series-function}.

{\bf For optimization to be possible, series expressions must correspond to
straight-line computations}.  That is to say, the data flow graph
corresponding to a series expression cannot contain any conditional
branches.  (Complex control flow is incompatible with pipelining.)
Optimization is possible in the presence of standard straight-line forms
such as \cdf{progn}, \cdf{funcall}, \cdf{setq}, \cdf{lambda}, \cdf{let},
\cdf{let*}, and \cd{multiple-{\allowbreak}value-{\allowbreak}bind} as long
as none of the variables bound are special.  There is also no problem with
macros as long as they expand into series functions and straight-line forms.
However, optimization is blocked by forms that specify complex control flow
(i.e., conditionals \cdf{if}, \cdf{cond}, etc., looping constructs \cdf{loop},
\cdf{do}, etc., or branching constructs \cdf{tagbody}, \cdf{go}, \cdf{catch},
etc.).

In the first example below, optimization is blocked, because the \cdf{if}
form is inside the series expression.  In the second example, however,
optimization is possible, because although the \cdf{if} feeds data to the
series expression, it is not inside the corresponding subgraph.  Both of
the expressions below produce the same value, but the second one is
much more efficient.
\begin{lisp}
(collect (if flag (scan x) (scan y)))~~;{\rm Warning message issued} \\*
(collect (scan (if flag x y))) 
\end{lisp}

\subsection{Constraint Cycles}

Even if a series expression satisfies all of the restrictions above, it
still may not be possible to transform the expression into a loop.  The
sole remaining problem is that if a series is used in two places, the
two uses may place incompatible constraints on the times at which series
elements should be produced.

The series expression below shows a situation where this problem arises.
The expression creates a series \cdf{x} of the elements in a list. It then
creates a normalized series by dividing each element of \cdf{x} by the sum
of the elements in \cdf{x}.  Finally, the expression returns the maximum of
the normalized elements.
\begin{lisp}
(let ((x (scan '(1 2 5 2))))~~~~~~~~~~~;{\rm Warning message issued} \\*
~~(collect-max (\#M/ x (series (collect-sum x))))) {\EV} 1/2
\end{lisp}

The two uses of \cdf{x} in the expression place contradictory
constraints on the way pipelined evaluation must proceed;  \cdf{collect-sum}
requires that all of the elements of \cdf{x} be produced before the sum can
be returned, and \cdf{series} requires that its input be available before it
can start to produce its output.  However, \cd{\#M/} requires that the
first element of \cdf{x} be available at the same time as the first element
of the output of \cdf{series}.  For pipelining to work,
the first element of the output of \cdf{series} (and therefore the output
of \cdf{collect-sum}) must be available before the second element of 
\cdf{x} is produced.  Unfortunately, this is impossible.

The essence of the inconsistency above is the cycle of constraints used in
the argument.  This in turn stems from a cycle in the data flow graph
underlying the expression.  In
figure~\ref{SERIES-F1-FIGURE} function calls are represented by boxes and data
flow is represented by arrows.  Simple arrows indicate the flow of series
values and cross-hatched arrows indicate the flow of non-series values.

\begin{figure}[t]
\caption{A Constraint Cycle in a Series Expression}\label{SERIES-F1-FIGURE}
\vskip 5pc
\PostScriptFile{series-plot.ps}\relax
\hbox{\relax
\def\foo#1#2#3{\vbox to 0pt{\vskip #2\vskip 3pt\hbox to 0pt{\hskip #1\hskip -3pt\vbox to 0pt{\vss
   \hbox to 0pt{\hss \tt #3\hss}\vss}\hss}\vss}}
\foo{2.5pc}{-3.5pc}{scan}\relax
\foo{8.5pc}{-2.5pc}{sum}\relax
\foo{14pc}{-2.5pc}{series}\relax
\foo{19.5pc}{-3.5pc}{\#M/}\relax
\foo{25pc}{-3.5pc}{max}}
\end{figure}

Given a data flow graph corresponding to a series expression, a {\it
constraint cycle} is a closed oriented loop of data flow arcs such
that each arc is traversed exactly once and no non-series arc
is traversed backward.  (Series data flow arcs can be traversed in either
direction.)  A constraint cycle is said to \emph{pass through} an input or
output port when exactly one of the arcs in the cycle touches the port.  In
figure~\ref{SERIES-F1-FIGURE} the data flow arcs touching \cdf{scan}, \cdf{sum},
\cdf{series}, and \cd{\#M/} form a constraint cycle.  Note that if the
output of \cdf{scan} were not a series, this loop would not be a constraint
cycle, because there would be no valid way to traverse it.  Also note that
while the constraint cycle passes through all the other ports it touches,
it does not pass through the output of \cdf{scan}.

Whenever a constraint cycle passes through a non-series output, an argument
analogous to the one above can be constructed and therefore pipelining will be
impossible.  When this situation arises, a warning message is issued
identifying the problematical port and the cycle passing through it.  For
instance, the warning triggered by the example above states that the
constraint cycle associated with \cdf{scan}, \cdf{collect-sum}, 
\cdf{series}, and \cd{\#M/} passes through the non-series output of 
\cdf{collect-sum}.

Given this kind of detailed information, it is easy to alleviate the
problem.  To start with, every cycle must contain at least one function
that has two series data flows leaving it.  At worst, the cycle can be broken by
duplicating this function (and any functions computing series used by it).
For instance, the example above can be
rewritten as shown below.
\begin{lisp}
(let ((x (scan '(1 2 5 2))) \\*
~~~~~~(sum (collect-sum (scan '(1 2 5 2))))) \\*
~~(collect-max (\#M/ x (series sum)))) \\*
~~{\EV} 1/2
\end{lisp}

It would be easy enough to automatically apply code copying to break
problematical constraint cycles.  However, this is not done for two
reasons.  First, there is considerable virtue in maintaining the property
that each function in a series expression turns into one piece of
computation in the loop produced.  Users can be confident that series
expressions that look simple and efficient actually are simple and
efficient.  Second, with a little creativity, constraint problems can often
be resolved in ways that are much more efficient than copying code.  In the
example above, the conflict can be eliminated efficiently by interchanging
the operation of computing the maximum with the operation of normalizing an
element.
\begin{lisp}
(let ((x (scan '(1 2 5 2)))) \\*
~~(/ (collect-max x) (collect-sum x))) {\EV} 1/2
\end{lisp}

The restriction that optimizable series expressions cannot contain
constraint cycles that pass through non-series outputs places limitations on
the qualitative character of optimizable series expressions.  In particular,
they all must have the general form of creating some number of series using
scanners, computing various intermediate series using transducers, and then
computing one or more summary results using collectors.  The output of a
collector cannot be used in the intermediate computation unless it is the
output of a separate subexpression.

It is worthy of note that the last expression above fixes the constraint
conflict by moving the non-series output out of the cycle, rather than by
breaking the cycle.  This illustrates the fact that constraint cycles that
do not pass through non-series outputs do not necessarily cause problems.
They cause problems only if they pass through \emph{off-line} ports.

A series input port or series output port of a series function is \emph{on-line}
if and only if it is processed in lockstep with all the other on-line
ports as follows:  the initial element of each on-line input is
read, then the initial element of each on-line output is written, then the
second element of each on-line input is read, then the second element of
each on-line output is written, and so on.  Ports that are not on-line are
off-line.  If all of the series ports of a function are on-line, the
function is said to be on-line; otherwise, it is off-line.  (The above
extends the standard definition of the term \emph{on-line} so that it applies
to individual ports as well as whole functions.)

If all of the ports a cycle passes through are on-line, the lockstep
processing of these ports guarantees that there cannot be any conflicts
between the constraints associated with the cycle.  However, passing
through an off-line port leads to the same kinds of problems as passing
through a non-series output.

Most of the series functions are on-line.  In particular, scanners and
collectors are all on-line as are many transducers.  However, the
transducers in section~\ref{SERIES-OL-SECTION} are off-line.  In particular, the
series inputs of \cdf{catenate}, \cdf{choose-if}, \cdf{chunk}, \cdf{expand}, \cdf{mask},
\cdf{mingle}, \cdf{positions}, and \cdf{subseries} along with the
series outputs of \cdf{choose}, \cdf{split}, and \cdf{split-if} are off-line.

In summary, the fourth and final restriction is that {\bf for optimization
to be possible, a series expression cannot contain a constraint cycle that
passes through a non-series output or an off-line port}.  Whenever this
restriction is violated a warning message is issued.  Violations can be
fixed either by breaking the cycle or restructuring the computation so that
the offending port is removed from the cycle.

\subsection{Defining New Series Functions}

New functions operating on series can be defined just as easily as new
functions operating on any other data type.  However, expressions
containing these new functions cannot be transformed into loops unless a
complete analysis of the functions is available.  Among other things,
this implies that the definition of a new series function must appear
before its first use.


\begin{defun}[Declaration specifier]
optimizable-series-function

The declaration specifier \cd{(optimizable-series-function~\emph{integer})} indicates
that the function being defined is a series function that needs to be
analyzed so that it can be optimized when it appears in series expressions.
(A warning is issued if the function being defined neither takes a series
as input nor produces a series as output.)  \emph{Integer} (default 1)
specifies the number of values returned by the function being defined.
(This cannot necessarily be determined by local analysis.)  The only place
\cdf{optimizable-series-function} is allowed to appear is in a declaration
immediately inside a \cdf{defun}.  As an example, the following shows how a
simplified version of \cdf{collect-sum} could be defined.
\begin{lisp}
(defun simple-collect-sum (numbers) \\*
~~(declare (optimizable-series-function 1)) \\*
~~(collect-fn 'number \#'(lambda () 0) \#'+ numbers))
\end{lisp}
\end{defun}

\begin{defun}[Declaration specifier]
off-line-port

The declaration specifier
\cd{(off-line-port~\emph{port-spec1}~\emph{port-spec2}~...)} specifies that the
indicated inputs and outputs are off-line.  This declaration
specifier is only allowed in a \cdf{defun} that contains the declaration 
\cdf{optimizable-series-function}.  Each \emph{port-spec} must either be a symbol
that is one of the inputs of the function or an integer \emph{j} indicating the
\emph{j\/}th output (counting from zero).  For example, \cd{(off-line-port~x~1)}
indicates that the input \cdf{x} and the second output are off-line.
Every port that is not mentioned in an \cdf{off-line-port}
declaration is assumed to be on-line.  A warning is issued whenever a
port's actual on-line/off-line status does not agree with its declared
status.  This makes it easier to keep track of which ports are off-line and
which are not.  Note that off-line ports virtually never arise when
defining scanners or reducers.
\end{defun}

\subsection{Declarations}

A key feature of Lisp is that variable declarations are strictly optional.
Nevertheless, it is often the case that they are necessary in situations
where efficiency matters.  Therefore, it is important that it be {\it
possible} for programmers to provide declarations for every variable in a
program.  The transformation of series expressions into loops presents
certain problems in this regard, because the loops created contain
variables not evident in the original code.  However, if the information
described below is supplied by the user, appropriate declarations can be
generated for all of the loop variables created.

All the explicit variables that are bound in a series expression (for example, by a 
\cdf{let} that is part of the expression) should be given informative
declarations making use of the type specifier \cd{(series~\emph{element-type})}
where appropriate.

Informative types should be supplied to series functions (such as 
\cdf{scan} and \cdf{map-fn}) that have type arguments.  When using 
\cdf{scan} it is important to specify the type of element in the sequence as
well as the sequence itself (for example, by using \cd{(vector~*~integer)} as
opposed to merely \cdf{vector}).  The form \cd{(list~\emph{element-type})}
can be used to specify the type of elements in a list.

If it is appropriate to have a type more specific than \cd{(series~t)}
associated with the output of \cd{\#M}, \cd{\#Z}, \cdf{scan-alist}, 
\cdf{scan-file}, \cdf{scan-hash}, \cdf{scan-lists-of-lists-fringe}, 
\cdf{scan-lists-of-lists}, \cdf{scan-plist},
\cdf{series}, \cdf{latch}, or \cdf{catenate}, then the form
\cdf{the} must be used to specify this type.

Finally, if the expression computing a non-series argument to a series
variable is neither a variable nor a constant, \cdf{the} must be used to
specify the type of its result.

For example, the declarations in the series expressions below are
sufficient to ensure that every loop variable will have an accurate
declaration.
\begin{lisp}
(collect-last (choose-if \#'plusp (scan '(list integer) data))) \\
\\
(collect '(vector * float) \\*
~~~~~~~~~(map-fn 'float \#'/ \\*
~~~~~~~~~~~~~~~~~(series (the integer (car data))) \\*
~~~~~~~~~~~~~~~~~(the (series integer) (scan-file f))))
\end{lisp}

The amount of information the user has to provide is reduced by the fact
that this information can be propagated from place to place.  For instance,
the variable holding the output of \cdf{choose-if} holds a subset of the
elements held by the input variable.  As a result, it is appropriate for it
to have the same type.  When defining a new series function, the type
specifier \cdf{series-element-type} can be used to indicate where type
propagation should occur.

\begin{defun}[Type specifier]
series-element-type

The type specifier \cd{(series-element-type~\emph{variable})} denotes the
type of elements in the series held in \emph{variable}.  \emph{Variable} must
be a variable carrying a series value (for example, a series argument of a series
function).  \cdf{series-element-type} can be used only in three places: in
a declaration in a \cdf{let}, \cdf{mapping}, \cdf{producing}, or other
binding form in a series expression; in a declaration in a \cdf{defun}
being used to define a series function; or in a type argument to a series
function.  As an example, consider that \cdf{collect-last} could have been
defined as follows.  The use of \cdf{series-element-type} ensures that the
internal variable keeping track of the most recent item has the correct
type.
\begin{lisp}
(defun collect-last (items \&optional (default nil)) \\*
~~(declare (optimizable-series-function)) \\*
~~(collect-fn '(series-element-type items) \\*
~~~~~~~~~~~~~~\#'(lambda () default) \\*
~~~~~~~~~~~~~~\#'(lambda (old new) new) \\*
~~~~~~~~~~~~~~items))
\end{lisp}
\end{defun}

\section{Primitives}

A large number of series functions are provided, because there are a
large number of useful operations that can be performed on series.
However, this functionality can be boiled down to a small
number of primitive constructs.

\cdf{collecting-fn} embodies the fundamental idea of series computations
that utilize internal state.  It can be used as the basis for defining any
on-line transducer.

\cdf{until} embodies the fundamental idea of producing a series that is
shorter than the shortest input series.  In particular, it embodies the
idea of computing a bounded series from non-series inputs.  Together with
\cdf{collecting-fn}, \cdf{until} can be used to define \cdf{scan-fn}, which
can be used as the basis for defining all the other scanners.

\cdf{collect-last} embodies the fundamental idea of producing a
non-series value from a series.  Together with \cdf{collecting-fn}, it
can be used to define \cdf{collect-fn}, which (with the occasional
assistance of \cdf{until}) can be used as the basis for defining all the other
collectors. 

\cdf{producing} embodies the fundamental idea of preorder computation.  It
can be used as the basis for defining all the other series functions,
including the off-line transducers.

In addition to the above, four primitives support
various specialized aspects of series functions.  Alterability is
supported by the function \cdf{to-alter} and the declaration 
\cdf{propagate-alterability}.  The propagation of type information is
supported by the type specifier \cdf{series-element-type}.  The best
implementation of certain series functions requires the form 
\cdf{encapsulated}.

\begin{defmac}
producing output-list input-list {declaration}* {form}*

\cdf{producing} computes and returns a group of series and non-series
outputs given a group of series and non-series inputs.  The key feature of
\cdf{producing} is that some or all of the series inputs and outputs can be
processed in an off-line way.  To support this, the processing in the
body (consisting of the \emph{forms}) is performed from the perspective
of generators and gatherers (see
appendix~\ref{GENERATORS}).  Each series input is converted to a generator
before being used in the body.  Each series output is associated with
a gatherer in the body.

The \emph{output-list} has the same syntax as the binding list of a 
\cdf{let}.  The names of the variables must be distinct from each other and
from the names of the variables in the \cdf{input-list}.  If there are \emph{n}
variables in the \emph{output-list}, \cdf{producing} computes \emph{n}
outputs.  There must be at least one output variable.  The variables act as
the names for the outputs and can be used in either of two ways.  First, if
an output variable has a value associated with it in the \emph{output-list},
then the variable is treated as holding a non-series value.  The variable
is initialized to the indicated value and can be used in any way desired in
the body. The eventual output value is whatever value is in the variable
when the execution of the body terminates.  Second, if an output variable
does not have a value associated with it in the \emph{output-list}, the
variable is given as its value a gatherer that collects elements.  The only
valid way to use the variable in the body is in a call on \cdf{next-out}.
The output returned is a series containing these elements.  If the body
never terminates, this series is unbounded.

The \emph{input-list} also has the same syntax as the binding list of a 
\cdf{let}.   The names of the variables must be distinct from each other and
the names of the variables in the \emph{output-list}.  The values can be
series or non-series.  If the value is not explicitly specified, it
defaults to \cdf{nil}.  The variables act logically both as inputs and
state variables and can be used in one of two ways.  First, if an input
variable is associated with a non-series value, then it is given this value
before the evaluation of the body begins and can be used in any way desired
in the body.   Second, if an input variable is associated with a series,
then the variable is given a generator corresponding to this series as its
initial value.  The only valid way to use the variable in the body is in a
call on \cdf{next-in}.

There can be declarations at the start of the body.  However,
the only declarations allowed are \cdf{ignore} declarations, type
declarations, and \cdf{propagate-alterability} declarations (see
below).  In particular, it is an error for any of the input or output
variables to be special.

In conception, the body can contain arbitrary Lisp expressions.
After the appropriate generators and gatherers have been set up, the
body is executed until it terminates.  If the body never
terminates, the series outputs (if any) are unbounded in length and
the non-series outputs (if any) are never produced.

Although easy to understand, this view of what can happen in the
body presents severe difficulties when optimizing (and even when
evaluating) series expressions that contain calls on \cdf{producing}.
As a result, several limitations are imposed on the form of the
body to simplify the processing required.

The first limitation is that, exclusive of any declarations, the
body must have the form \cd{(loop~(tagbody~...))}.  The following
example shows how \cdf{producing} could be used to implement a
scanner creating an unbounded series of integers.
\begin{lisp}
(producing (nums) ((num 0)) \\*
~~(declare (integer num) (type (series integer) nums)) \\*
~~(loop \\*
~~~~(tagbody \\*
~~~~~~(setq num (1+ num)) \\*
~~~~~~(next-out nums num)))) \\*
~~{\EV} \#Z(1 2 3 4 ...)
\end{lisp}

The second limitation is that the form \cdf{terminate-producing} must be
used to terminate the execution of the body.  Any other method of
terminating the body (with \cdf{return}, for example) is an error.
The following example shows how \cdf{producing} could be used to
implement the operation of summing a series.  The function 
\cdf{terminate-producing} is used to stop the computation when \cdf{numbers}
runs out of elements.
\begin{lisp}
(producing ((sum 0)) ((numbers \#Z(1 2 3)) num) \\* 
~~(loop \\*
~~~~(tagbody \\*
~~~~~~(setq num (next-in numbers (terminate-producing))) \\*
~~~~~~(setq sum (+ sum num))))) \\*
~~{\EV} 6
\end{lisp}

The third limitation is that calls on \cdf{next-out} associated with
output variables must appear at top level in the \cdf{tagbody} in the
body.  They cannot be nested in other forms.  In addition, an
output variable can be the destination of at most one call on 
\cdf{next-out} and if it is the destination of a \cdf{next-out}, it cannot
be used in any other way.

If the call on \cdf{next-out} for a given output appears in the
final part of the \cdf{tagbody} in the body, after everything
other than other calls on \cdf{next-out}, then the output is an
on-line output---a new value is written on every cycle of the
body.  Otherwise the output is off-line.

The following example shows how \cdf{producing} could be used to split
a series into two parts.  Items are read in one at a time and tested.
Depending on the test, they are written to one of two outputs.  Note
the use of labels and branches to keep the calls on 
\cdf{next-out} at top level.  Both outputs are off-line.  The first example
above shows an on-line output.
\begin{lisp}
(producing (items-1 items-2) ((items \#Z(1 -2 3 -4)) item) \\*
~~(loop \\*
~~~~(tagbody (setq item (next-in items (terminate-producing))) \\*
~~~~~~~~~~~~~(if (not (plusp item)) (go D)) \\*
~~~~~~~~~~~~~(next-out items-1 item) \\*
~~~~~~~~~~~~~(go F) \\*
~~~~~~D~~~~~~(next-out items-2 item) \\*
~~~~~~F~~~~~~))) \\*
~~{\EV} \#Z(1 3) {\rm and} \#Z(-2 -4)
\end{lisp}

The fourth limitation is that the calls on \cdf{next-in} associated with an
input variable \cdf{v} must appear at top level in the \cdf{tagbody} in the
body, nested in assignments of the form 
\cd{(setq~\emph{var}~(next-in~v~...))}.  They cannot be nested in other
forms.  In addition, an input variable can be the source for at most one
call on \cdf{next-in} and if it is the source for a \cdf{next-in}, it
cannot be used in any other way.

If the call on \cdf{next-in} for a given input has as its sole
termination action \cd{(terminate-producing)} and
appears in the initial part of the \cdf{tagbody} in the body,
before anything other than similar calls on \cdf{next-in}, then the
input is an on-line input---a new value is read on every cycle of the
body.  Otherwise the input is off-line.

The example below shows how \cdf{producing} could be used to
concatenate two series.  To start with, elements are read from the
first input series.  When this runs out, a flag is set and reading
begins from the second input.  Both inputs are off-line.
(Compare this to the example
above, which shows an on-line input.)
\begin{lisp}
(producing (items) ((item-1 \#Z(1 2)) \\*
~~~~~~~~~~~~~~~~~~~~(item-2 \#Z(3 4)) \\*
~~~~~~~~~~~~~~~~~~~~(in-2 nil) \\*
~~~~~~~~~~~~~~~~~~~~item) \\*
~~(loop \\*
~~~~(tagbody (if in-2 (go D)) \\*
~~~~~~~~~~~~~(setq item (next-in item-1 (setq in-2 t) (go D))) \\*
~~~~~~~~~~~~~(go F) \\*
~~~~~~D~~~~~~(setq item (next-in item-2 (terminate-producing))) \\*
~~~~~~F~~~~~~(next-out items item)))) \\*
~~{\EV} \#Z(1 2 3 4)
\end{lisp}
\end{defmac}

\begin{defmac}
terminate-producing \!!

This form (which takes no arguments) is used to terminate execution of
(the expansion of) the \cdf{producing} macro.

As with the form \cdf{go},
\cdf{terminate-producing} does not return any values; rather, control
immediately leaves the current context.

The form \cdf{terminate-producing}
is allowed to appear only in a \cdf{producing} body and causes the
termination of the enclosing call on \cdf{producing}.
\end{defmac}

\begin{defun}[Declaration specifier]
propagate-alterability

The declaration specifier
\cd{(propagate-alterability~\emph{input}~\emph{output})}
indicates that attempts to alter an element of \emph{output} should be
satisfied by altering the corresponding element of \emph{input}.    (The
corresponding element of \emph{input} is the one most recently read at the
moment when the output element is written.)

This declaration may
appear only in a call on \cdf{producing}.  The \emph{input} and \emph{output} arguments must be
an input and an output, respectively, of the \cdf{producing} macro.  The example below shows how
the propagation of alterability could be supported in a simplified version
of \cdf{until}.
\begin{lisp}
(defun simple-until (bools items) \\*
~~(declare (optimizable-series-function)) \\*
~~(producing (z) ((x bools) (y items) bool item) \\*
~~~~(declare (propagate-alterability y z)) \\*
~~~~(loop \\*
~~~~~~(tagbody \\*
~~~~~~~~(setq bool (next-in x (terminate-producing))) \\*
~~~~~~~~(setq item (next-in y (terminate-producing))) \\*
~~~~~~~~(if bool (terminate-producing)) \\*
~~~~~~~~(next-out z item)))))
\end{lisp}
\end{defun}

\begin{defmac}
encapsulated encapsulating-fn scanner-or-collector

Some of the features provided by Common Lisp are supported solely by encapsulating forms.
For example, there is no way to specify a cleanup expression that will always be run, even
when an abort occurs, without using \cdf{unwind-protect}.  \cdf{encapsulated} makes it possible
to take advantage of forms such as \cdf{unwind-protect} when defining a series function.

\cdf{encapsulated} specifies a function that places an encapsulating
form around the computation performed by its second argument.  The first argument must be a
quoted function that takes a Lisp expression and wraps the appropriate encapsulating form
around it, returning the resulting code.
The second input must be a literal call on \cdf{scan-fn}, 
\cdf{scan-fn-inclusive}, or \cdf{collect-fn}.  The second argument can count on being evaluated in the
scope of the encapsulating form.  The values returned by the second argument are returned as the
values of \cdf{encapsulated}.  The following shows how 
\cdf{encapsulated} could be used to define a simplified version of \cdf{collect-file}.
\begin{lisp}
(defun collect-file-wrap (file name body) \\*
~~`(with-open-file (,file ,name :direction :output) ,body)) \\
\\
(defmacro simple-collect-file (name items) \\*
~~(let ((file (gensym))) \\*
~~~~{\Xbq}(encapsulated \#'(lambda (body) \\*
~~~~~~~~~~~~~~~~~~~~~~~(collect-file-wrap ',file ',name body)) \\*
~~~~~~~~~~~~~~~~~~~(collect-fn t \#'(lambda () t) \\*
~~~~~~~~~~~~~~~~~~~~~~~~~~~~~~~\#'(lambda (state item) \\*
~~~~~~~~~~~~~~~~~~~~~~~~~~~~~~~~~~~(print item ,file) \\*
~~~~~~~~~~~~~~~~~~~~~~~~~~~~~~~~~~~state) \\*
~~~~~~~~~~~~~~~~~~~~~~~~~~~~~~~,items))))
\end{lisp}
\end{defmac}
\newpage%manual
      % Series
%%%Chapter of Common Lisp Manual.  Copyright 1989 Guy L. Steele Jr.

%  +++  Final version of chapter  +++

\clearpage\def\pagestatus{FINAL PROOF}

\chapterauthor{Crispin Perdue and Richard C. Waters}
\chapter{Generators and Gatherers}
\label{GENERATORS}

\begin{new}
\prefaceword  Generators and gatherers are yet another
approach, closely related to series,
to providing iteration in a functional style.

The remainder of this chapter consists of a description by Crispin Perdue
and Richard C.~Waters of their work on an existing implementation of
generators and gatherers.  I have edited the chapter only very lightly to
conform to the overall style of this book.  Please see the Preface to this
book for more information about the genesis of the generators/gatherers
approach and its relationship to the work of X3J13.

\noindent\hbox to \textwidth{\hss---Guy L. Steele Jr.}
\vskip 8pt plus 3pt minus 2pt

\section{Introduction}

Generators are generalized input streams in the sense of
Smalltalk~\cite{SMALLTALK-80-BOOK}.  A generator can produce a potentially
unbounded number of elements of any type.  Individual elements are not
computed until requested by \cdf{next-in}.  When an element is taken from
a generator, it is removed by side effect.  Subsequent uses of 
\cdf{next-in} obtain later elements.

There is a close relationship between a generator and a series of the
elements it produces.  In particular, any series can be converted into
a generator.  As a result, all the scanner functions used for
creating series (see appendix~\ref{SERIES}) can be used to create
generators as well.  There is no need to have a separate
set of functions for creating generators.

Gatherers are generalized output streams.  Elements of any type can be
entered into a gatherer using \cdf{next-out}.  The gatherer combines the
elements together in time-sequence order into a net result.  This result can
be retrieved using \cdf{result-of}.

There is a close relationship between a gatherer and a collector function
that combines elements in the same way.  In particular, any one-input
one-output collector can be converted into a gatherer.  As a result, all
the collectors used for computing summary results from series can be used to
create gatherers.  There is no need to have a separate set of functions for
creating gatherers.


\section{Generators}

These functions create and process generators.

\begin{defun}[Function]
generator series

Given a series, \cdf{generator} returns a generator containing the same
elements.
\end{defun}


\begin{defmac}
next-in generator {action}*

\cdf{next-in} returns the next element in the generator {\it generator}.
The {\it actions} can be any Lisp expressions.  They are evaluated if and
only if no more elements can be retrieved from {\it generator}.  If there
are no more elements and no actions, it is an error.  It is also an error
to apply \cdf{next-in} to a generator a second time after the generator has
run out of elements.  As an example of generators, consider the following.
\begin{lisp}
(let ((x (generator (scan '(1 2 3 4))))) \\*
~~(with-output-to-string (s) \\*
~~~~(loop (prin1 (next-in x (return)) s) \\*
~~~~~~~~~~(prin1 (next-in x (return)) s) \\*
~~~~~~~~~~(princ "," s)))) \\*
~{\EV} "12,34,"
\end{lisp}
\end{defmac}

\section{Gatherers}

These functions create and process gatherers.

\begin{defun}[Function]
gatherer collector

The {\it collector} must be a function of type 
\cd{(function~((series~$t_1$))~$t_2$)}.  Given this function, \cdf{gatherer}
returns a gatherer that accepts elements of type $t_1$ and returns a final
result of type $t_2$.  The method for combining elements used by the
gatherer is the same as the one used by the {\it collector}.
\end{defun}


\begin{defun}[Function]
next-out gatherer item

Given a gatherer and a value, \cdf{next-out} enters the value into the
gatherer.
\end{defun}


\begin{defun}[Function]
result-of gatherer

\cdf{result-of} retrieves the net result from a gatherer.  \cdf{result-of}
can be applied at any time.  However, it is an error to apply 
\cdf{result-of} twice to the same gatherer or to apply \cdf{next-out} to a
gatherer once \cdf{result-of} has been applied.
\begin{lisp}
(let ((g (gatherer \#'collect-sum))) \\*
~~(dolist (i '(1 2 3 4)) \\*
~~~~(next-out g i) \\*
~~~~(if (evenp i) (next-out g (* 10 i)))) \\*
~~(result-of g)) \\*
~{\EV} 70
\end{lisp}
\end{defun}

\begin{defmac}
gathering ({(var fn)}*) {\,form}*

The first subform must be a list of pairs.  The first
element of each pair, {\it var}, must be a variable name.
The second element of each pair, {\it fn},
must be a form that when wrapped in \cd{(function~...)} is
acceptable as an argument to \cdf{gatherer}.  Each symbol is bound to a
gatherer constructed from the corresponding collector.  The body
(consisting of the {\it forms}) is evaluated in the scope of these bindings. 
When this evaluation is complete, \cdf{gathering} returns the \cdf{result-of} each
gatherer.  If there are $n$ pairs in the binding list,
\cdf{gathering} returns $n$ values.  For example:
\begin{lisp}
(defun examp (data) \\*
~~(gathering ((x collect) (y collect-sum)) \\*
~~~~(iterate ((i (scan data))) \\*
~~~~~~(case (first i) \\*
~~~~~~~~(:slot (next-out x (second i))) \\*
~~~~~~~~(:part (dolist (j (second i)) (next-out x j)))) \\*
~~~~~~(next-out y (third i))))) \\
\\
(examp '((:slot a 10) (:part (c d) 40))) {\EV} (a c d) {\rm and} 50
\end{lisp}

As a further illustration of gatherers, consider the following definition for a
simplified version of \cdf{gathering} that handles only one binding pair.
\begin{lisp}
(defmacro simple-gathering (((var collector)) \&body body) \\*
~~{\Xbq}(let ((,var (gatherer (function ,collector)))) \\*
~~~~~,{\Xatsign}body \\*
~~~~~(result-of ,var)))
\end{lisp}
The full capabilities of 
\cdf{gathering} can be supported in much the same way.
\end{defmac}

\section{Discussion}

The idea of generators and gatherers was first proposed by Pavel
Curtis.  A key aspect of his proposal was the realization that
generators and gatherers can be implemented simply and elegantly as
closures and that these closures can be compiled very
efficiently if certain conditions are met.

First, the compiler must support an optimization Curtis calls
``\cdf{let} eversion'' in addition to the optimization methods presented
in~\cite{RABBIT}.  If a closure is created and used entirely within a
limited lexical scope, the scopes of any bound variables nested in the
closure can be enlarged (everted) to enclose all the uses of the
closure.  This allows the variables to be allocated on the stack
rather than the heap.

Second, for a generator/gatherer closure to be compiled efficiently,
it must be possible to determine at compile time exactly what closure
is involved and exactly what the scope of use of the closure is.
There are several aspects to this.  The expression creating the
generator/gatherer cannot refer to a free series variable.  The
generator/gatherer must be stored in a local variable.  This
variable must be used only in calls of \cdf{next-in}, 
\cdf{next-out}, and \cdf{result-of}, and not inside a closure.  In
particular the generator/gatherer cannot be stored in a data
structure, stored in a special variable, or returned as a result
value.

All of the examples above satisfy these restrictions.  For instance,
once the uses of \cdf{gathering} and \cdf{iterate} have been
optimized, the body of \cdf{examp} is as efficient as any loop
performing the same computation.

The implementation discussed in~\cite{WATERS-SERIES-DESIGN} includes a
portable Common Lisp implementation of generators and gatherers.  Although
the implementation does not support optimizations of the kind discussed
in~\cite{RABBIT}, it fully optimizes uses of \cdf{gathering}.
\end{new}
  % Generators
\clearpage\def\pagestatus{ROUGH PAGES}

\begingroup
\makeatletter
\def\@listi{\leftmargin\leftmargini \labelsep\leftmargin
   \parsep 3pt\relax
   \topsep 4pt plus 10pt\relax
   \itemsep\topsep}
\makeatother

\chapter{Backquote}
\label{BACKQUOTE-SIMULATOR}

\begin{new}
Here is the code for an implementation of backquote syntax
(see section~\ref{BACKQUOTE}) that I have found quite useful
in explaining to myself the behavior of nested backquotes.
It implements the formal rules for backquote processing
and optionally applies a code simplifier to the result.
One must be very careful in choosing the simplification rules;
the rules given here work, but some Common Lisp implementations
have run into trouble at one time or another by using a
simplification rule that does not work in all cases.
Code transformations that are plausible when single forms
are involved are likely to fail in the presence of splicing.

At the end of this appendix are some samples of
nested backquote syntax with commentary.

\begin{lisp}
;;; Common Lisp backquote implementation, written in Common Lisp. \\*
;;; Author: Guy L. Steele Jr.~~~~~Date: 27 December 1985 \\*
;;; Tested under Symbolics Common Lisp and Lucid Common Lisp. \\*
;;; This software is in the public domain.
\end{lisp}
\begin{lisp}
;;; \$ is pseudo-backquote and \% is pseudo-comma.~~This makes it \\*
;;; possible to test this code without interfering with normal \\*
;;; Common Lisp syntax.
\end{lisp}
\begin{lisp}
;;; The following are unique tokens used during processing. \\*
;;; They need not be symbols; they need not even be atoms.
\end{lisp}
\begin{lisp}
(defvar *comma* (make-symbol "COMMA")) \\*
(defvar *comma-atsign* (make-symbol "COMMA-ATSIGN")) \\
(defvar *comma-dot* (make-symbol "COMMA-DOT")) \\
(defvar *bq-list* (make-symbol "BQ-LIST")) \\
(defvar *bq-append* (make-symbol "BQ-APPEND")) \\
(defvar *bq-list** (make-symbol "BQ-LIST*")) \\
(defvar *bq-nconc* (make-symbol "BQ-NCONC")) \\
(defvar *bq-clobberable* (make-symbol "BQ-CLOBBERABLE")) \\
(defvar *bq-quote* (make-symbol "BQ-QUOTE")) \\*
(defvar *bq-quote-nil* (list *bq-quote* nil))
\end{lisp}
\begin{lisp}
;;; Reader macro characters: \\*
;;;~~~~\$foo is read in as (BACKQUOTE foo) \\*
;;;~~~~\%foo is read in as (\#:COMMA foo) \\*
;;;~~~~\%{\Xatsign}foo is read in as (\#:COMMA-ATSIGN foo) \\*
;;;~~~~\%.foo is read in as (\#:COMMA-DOT foo) \\*
;;; where \#:COMMA is the value of the variable *COMMA*, etc.
\end{lisp}
\begin{lisp}
;;; BACKQUOTE is an ordinary macro (not a read-macro) that \\*
;;; processes the expression foo, looking for occurrences of \\*
;;; \#:COMMA, \#:COMMA-ATSIGN, and \#:COMMA-DOT.~~It constructs code \\*
;;; in strict accordance with the rules on pages 349-350 of \\*
;;; the first edition (pages 528-529 of this second edition). \\*
;;; It then optionally applies a code simplifier.
\end{lisp}
\begin{lisp}
(set-macro-character \#{\Xbackslash}\$ \\*
~~\#'(lambda (stream char) \\*
~~~~~~(declare (ignore char)) \\*
~~~~~~(list 'backquote (read stream t nil t))))
\end{lisp}
\begin{lisp}
(set-macro-character \#{\Xbackslash}\% \\*
~~\#'(lambda (stream char) \\*
~~~~~~(declare (ignore char)) \\*
~~~~~~~~(case (peek-char nil stream t nil t) \\*
~~~~~~~~~~(\#{\Xbackslash}{\Xatsign} (read-char stream t nil t) \\*
~~~~~~~~~~~~~~~(list *comma-atsign* (read stream t nil t))) \\
~~~~~~~~~~(\#{\Xbackslash}. (read-char stream t nil t) \\*
~~~~~~~~~~~~~~~(list *comma-dot* (read stream t nil t))) \\*
~~~~~~~~~~(otherwise (list *comma* (read stream t nil t))))))
\end{lisp}
\begin{lisp}
 \\*
;;; If the value of *BQ-SIMPLIFY* is non-NIL, then BACKQUOTE \\*
;;; processing applies the code simplifier.~~If the value is NIL, \\*
;;; then the code resulting from BACKQUOTE is exactly that \\*
;;; specified by the official rules.
\end{lisp}
\begin{lisp}
(defparameter *bq-simplify* t)
\end{lisp}
\begin{lisp}
(defmacro backquote (x) \\*
~~(bq-completely-process x))
\end{lisp}
\begin{lisp}
;;; Backquote processing proceeds in three stages: \\*
;;; \\*
;;; (1) BQ-PROCESS applies the rules to remove occurrences of \\*
;;; \#:COMMA, \#:COMMA-ATSIGN, and \#:COMMA-DOT corresponding to \\*
;;; this level of BACKQUOTE.~~(It also causes embedded calls to \\
;;; BACKQUOTE to be expanded so that nesting is properly handled.) \\
;;; Code is produced that is expressed in terms of functions \\
;;; \#:BQ-LIST, \#:BQ-APPEND, and \#:BQ-CLOBBERABLE.~~This is done \\
;;; so that the simplifier will simplify only list construction \\
;;; functions actually generated by BACKQUOTE and will not involve \\
;;; any user code in the simplification.~~\#:BQ-LIST means LIST, \\
;;; \#:BQ-APPEND means APPEND, and \#:BQ-CLOBBERABLE means IDENTITY \\
;;; but indicates places where "\%." was used and where NCONC may \\*
;;; therefore be introduced by the simplifier for efficiency. \\*
;;; \\*
;;; (2) BQ-SIMPLIFY, if used, rewrites the code produced by \\*
;;; BQ-PROCESS to produce equivalent but faster code.~~The \\
;;; additional functions \#:BQ-LIST* and \#:BQ-NCONC may be \\*
;;; introduced into the code. \\*
;;; \\*
;;; (3) BQ-REMOVE-TOKENS goes through the code and replaces \\*
;;; \#:BQ-LIST with LIST, \#:BQ-APPEND with APPEND, and so on. \\
;;; \#:BQ-CLOBBERABLE is simply eliminated (a call to it being \\
;;; replaced by its argument).~~\#:BQ-LIST* is replaced by either \\
;;; LIST* or CONS (the latter is used in the two-argument case, \\*
;;; purely to make the resulting code a tad more readable).
\end{lisp}
\begin{lisp}
(defun bq-completely-process (x) \\*
~~(let ((raw-result (bq-process x))) \\*
~~~~(bq-remove-tokens (if *bq-simplify* \\*
~~~~~~~~~~~~~~~~~~~~~~~~~~(bq-simplify raw-result) \\*
~~~~~~~~~~~~~~~~~~~~~~~~~~raw-result))))
\end{lisp}
\begin{lisp}
(defun bq-process (x) \\*
~~(cond ((atom x) \\*
~~~~~~~~~(list *bq-quote* x)) \\
~~~~~~~~((eq (car x) 'backquote) \\*
~~~~~~~~~(bq-process (bq-completely-process (cadr x)))) \\
~~~~~~~~((eq (car x) *comma*) (cadr x)) \\
~~~~~~~~((eq (car x) *comma-atsign*) \\*
~~~~~~~~~(error ",{\Xatsign}{\Xtilde}S after {\Xbq}" (cadr x))) \\
~~~~~~~~((eq (car x) *comma-dot*) \\*
~~~~~~~~~(error ",.{\Xtilde}S after {\Xbq}" (cadr x))) \\
~~~~~~~~(t (do ((p x (cdr p)) \\*
~~~~~~~~~~~~~~~~(q '() (cons (bracket (car p)) q))) \\*
~~~~~~~~~~~~~~~((atom p) \\*
~~~~~~~~~~~~~~~~(cons *bq-append* \\*
~~~~~~~~~~~~~~~~~~~~~~(nreconc q (list (list *bq-quote* p))))) \\
~~~~~~~~~~~~~(when (eq (car p) *comma*) \\*
~~~~~~~~~~~~~~~(unless (null (cddr p)) (error "Malformed ,{\Xtilde}S" p)) \\*
~~~~~~~~~~~~~~~(return (cons *bq-append* \\*
~~~~~~~~~~~~~~~~~~~~~~~~~~~~~(nreconc q (list (cadr p)))))) \\
~~~~~~~~~~~~~(when (eq (car p) *comma-atsign*) \\*
~~~~~~~~~~~~~~~(error "Dotted ,{\Xatsign}{\Xtilde}S" p)) \\
~~~~~~~~~~~~~(when (eq (car p) *comma-dot*) \\*
~~~~~~~~~~~~~~~(error "Dotted ,.{\Xtilde}S" p))))))
\end{lisp}
\begin{lisp}
;;; This implements the bracket operator of the formal rules.
\end{lisp}
\begin{lisp}
(defun bracket (x) \\*
~~(cond ((atom x) \\*
~~~~~~~~~(list *bq-list* (bq-process x))) \\
~~~~~~~~((eq (car x) *comma*) \\*
~~~~~~~~~(list *bq-list* (cadr x))) \\
~~~~~~~~((eq (car x) *comma-atsign*) \\*
~~~~~~~~~(cadr x)) \\
~~~~~~~~((eq (car x) *comma-dot*) \\*
~~~~~~~~~(list *bq-clobberable* (cadr x))) \\*
~~~~~~~~(t (list *bq-list* (bq-process x)))))
\end{lisp}
\begin{lisp}
;;; This auxiliary function is like MAPCAR but has two extra \\*
;;; purposes: (1) it handles dotted lists; (2) it tries to make \\*
;;; the result share with the argument x as much as possible.
\end{lisp}
\begin{lisp}
(defun maptree (fn x) \\*
~~(if (atom x) \\*
~~~~~~(funcall fn x) \\*
~~~~~~(let ((a (funcall fn (car x))) \\*
~~~~~~~~~~~~(d (maptree fn (cdr x)))) \\
~~~~~~~~(if (and (eql a (car x)) (eql d (cdr x))) \\*
~~~~~~~~~~~~x \\*
~~~~~~~~~~~~(cons a d)))))
\end{lisp}
\begin{lisp}
;;; This predicate is true of a form that when read looked \\*
;;; like \%{\Xatsign}foo or \%.foo.
\end{lisp}
\begin{lisp}
(defun bq-splicing-frob (x) \\*
~~(and (consp x) \\*
~~~~~~~(or (eq (car x) *comma-atsign*) \\*
~~~~~~~~~~~(eq (car x) *comma-dot*))))
\end{lisp}
\begin{lisp}
 \\*
;;; This predicate is true of a form that when read \\*
;;; looked like \%{\Xatsign}foo or \%.foo or just plain \%foo.
\end{lisp}
\begin{lisp}
(defun bq-frob (x) \\*
~~(and (consp x) \\*
~~~~~~~(or (eq (car x) *comma*) \\*
~~~~~~~~~~~(eq (car x) *comma-atsign*) \\*
~~~~~~~~~~~(eq (car x) *comma-dot*))))
\end{lisp}
\begin{lisp}
;;; The simplifier essentially looks for calls to \#:BQ-APPEND and \\*
;;; tries to simplify them.~~The arguments to \#:BQ-APPEND are \\*
;;; processed from right to left, building up a replacement form. \\*
;;; At each step a number of special cases are handled that, \\*
;;; loosely speaking, look like this: \\*
;;; \\
;;;~~(APPEND (LIST a b c) foo) => (LIST* a b c foo) \\*
;;;~~~~~~~provided a, b, c are not splicing frobs \\*
;;;~~(APPEND (LIST* a b c) foo) => (LIST* a b (APPEND c foo)) \\*
;;;~~~~~~~provided a, b, c are not splicing frobs \\*
;;;~~(APPEND (QUOTE (x)) foo) => (LIST* (QUOTE x) foo) \\*
;;;~~(APPEND (CLOBBERABLE x) foo) => (NCONC x foo)
\end{lisp}
\begin{lisp}
(defun bq-simplify (x) \\*
~~(if (atom x) \\*
~~~~~~x \\*
~~~~~~(let ((x (if (eq (car x) *bq-quote*) \\*
~~~~~~~~~~~~~~~~~~~x \\*
~~~~~~~~~~~~~~~~~~~(maptree \#'bq-simplify x)))) \\
~~~~~~~~(if (not (eq (car x) *bq-append*)) \\*
~~~~~~~~~~~~x \\*
~~~~~~~~~~~~(bq-simplify-args x)))))
\end{lisp}
\begin{lisp}
(defun bq-simplify-args (x) \\*
~~(do ((args (reverse (cdr x)) (cdr args)) \\*
~~~~~~~(result \\*
~~~~~~~~~nil \\*
~~~~~~~~~(cond ((atom (car args)) \\*
~~~~~~~~~~~~~~~~(bq-attach-append *bq-append* (car args) result)) \\
~~~~~~~~~~~~~~~((and (eq (caar args) *bq-list*) \\*
~~~~~~~~~~~~~~~~~~~~~(notany \#'bq-splicing-frob (cdar args))) \\*
~~~~~~~~~~~~~~~~(bq-attach-conses (cdar args) result)) \\
~~~~~~~~~~~~~~~((and (eq (caar args) *bq-list**) \\*
~~~~~~~~~~~~~~~~~~~~~(notany \#'bq-splicing-frob (cdar args))) \\
~~~~~~~~~~~~~~~~(bq-attach-conses \\*
~~~~~~~~~~~~~~~~~~(reverse (cdr (reverse (cdar args)))) \\*
~~~~~~~~~~~~~~~~~~(bq-attach-append *bq-append* \\*
~~~~~~~~~~~~~~~~~~~~~~~~~~~~~~~~~~~~(car (last (car args))) \\*
~~~~~~~~~~~~~~~~~~~~~~~~~~~~~~~~~~~~result))) \\
~~~~~~~~~~~~~~~((and (eq (caar args) *bq-quote*) \\*
~~~~~~~~~~~~~~~~~~~~~(consp (cadar args)) \\*
~~~~~~~~~~~~~~~~~~~~~(not (bq-frob (cadar args))) \\*
~~~~~~~~~~~~~~~~~~~~~(null (cddar args))) \\
~~~~~~~~~~~~~~~~(bq-attach-conses (list (list *bq-quote* \\*
~~~~~~~~~~~~~~~~~~~~~~~~~~~~~~~~~~~~~~~~~~~~~~(caadar args))) \\*
~~~~~~~~~~~~~~~~~~~~~~~~~~~~~~~~~~result)) \\
~~~~~~~~~~~~~~~((eq (caar args) *bq-clobberable*) \\*
~~~~~~~~~~~~~~~~(bq-attach-append *bq-nconc* (cadar args) result)) \\
~~~~~~~~~~~~~~~(t (bq-attach-append *bq-append* \\*
~~~~~~~~~~~~~~~~~~~~~~~~~~~~~~~~~~~~(car args) \\*
~~~~~~~~~~~~~~~~~~~~~~~~~~~~~~~~~~~~result))))) \\*
~~~~~~((null args) result)))
\end{lisp}
\begin{lisp}
(defun null-or-quoted (x) \\*
~~(or (null x) (and (consp x) (eq (car x) *bq-quote*))))
\end{lisp}
\begin{lisp}
;;; When BQ-ATTACH-APPEND is called, the OP should be \#:BQ-APPEND \\*
;;; or \#:BQ-NCONC.~~This produces a form (op item result) but \\*
;;; some simplifications are done on the fly: \\*
;;; \\
;;;~~(op '(a b c) '(d e f g)) => '(a b c d e f g) \\*
;;;~~(op item 'nil) => item, provided item is not a splicable frob \\*
;;;~~(op item 'nil) => (op item), if item is a splicable frob \\*
;;;~~(op item (op a b c)) => (op item a b c)
\end{lisp}
\begin{lisp}
(defun bq-attach-append (op item result) \\*
~~(cond ((and (null-or-quoted item) (null-or-quoted result)) \\*
~~~~~~~~~(list *bq-quote* (append (cadr item) (cadr result)))) \\
~~~~~~~~((or (null result) (equal result *bq-quote-nil*)) \\*
~~~~~~~~~(if (bq-splicing-frob item) (list op item) item)) \\
~~~~~~~~((and (consp result) (eq (car result) op)) \\*
~~~~~~~~~(list* (car result) item (cdr result))) \\*
~~~~~~~~(t (list op item result))))
\end{lisp}
\begin{lisp}
;;; The effect of BQ-ATTACH-CONSES is to produce a form as if by \\*
;;; {\Xbq}(LIST* ,{\Xatsign}items ,result) but some simplifications are done \\*
;;; on the fly. \\*
;;; \\
;;;~~(LIST* 'a 'b 'c 'd) => '(a b c . d) \\*
;;;~~(LIST* a b c 'nil) => (LIST a b c) \\*
;;;~~(LIST* a b c (LIST* d e f g)) => (LIST* a b c d e f g) \\*
;;;~~(LIST* a b c (LIST d e f g)) => (LIST a b c d e f g)
\end{lisp}
\begin{lisp}
(defun bq-attach-conses (items result) \\*
~~(cond ((and (every \#'null-or-quoted items) \\*
~~~~~~~~~~~~~~(null-or-quoted result)) \\
~~~~~~~~~(list *bq-quote* \\*
~~~~~~~~~~~~~~~(append (mapcar \#'cadr items) (cadr result)))) \\
~~~~~~~~((or (null result) (equal result *bq-quote-nil*)) \\*
~~~~~~~~~(cons *bq-list* items)) \\
~~~~~~~~((and (consp result) \\*
~~~~~~~~~~~~~~(or (eq (car result) *bq-list*) \\*
~~~~~~~~~~~~~~~~~~(eq (car result) *bq-list**))) \\*
~~~~~~~~~(cons (car result) (append items (cdr result)))) \\*
~~~~~~~~(t (cons *bq-list** (append items (list result))))))
\end{lisp}
\begin{lisp}
;;; Removes funny tokens and changes (\#:BQ-LIST* a b) into \\*
;;; (CONS a b) instead of (LIST* a b), purely for readability.
\end{lisp}
\begin{lisp}
(defun bq-remove-tokens (x) \\*
~~(cond ((eq x *bq-list*) 'list) \\*
~~~~~~~~((eq x *bq-append*) 'append) \\
~~~~~~~~((eq x *bq-nconc*) 'nconc) \\
~~~~~~~~((eq x *bq-list**) 'list*) \\
~~~~~~~~((eq x *bq-quote*) 'quote) \\
~~~~~~~~((atom x) x) \\
~~~~~~~~((eq (car x) *bq-clobberable*) \\*
~~~~~~~~~(bq-remove-tokens (cadr x))) \\
~~~~~~~~((and (eq (car x) *bq-list**) \\*
~~~~~~~~~~~~~~(consp (cddr x)) \\*
~~~~~~~~~~~~~~(null (cdddr x))) \\*
~~~~~~~~~(cons 'cons (maptree \#'bq-remove-tokens (cdr x)))) \\
~~~~~~~~(t (maptree \#'bq-remove-tokens x))))
\end{lisp}

Suppose that we first make the following definitions:

\begin{lisp}
(setq q '(r s)) \\*
(defun r (x) (reduce \#'* x)) \\*
(setq r '(3 5)) \\*
(setq s '(4 6))
\end{lisp}

Without simplification, the notation
\cd{\$\$(\%\%q)} (which stands for \cd{{\Xbq}{\Xbq}(,,q)})
is read as the expression
\begin{lisp}
(APPEND (LIST 'APPEND) (LIST (APPEND (LIST 'LIST) (LIST Q))))
\end{lisp}
The value of this expression is
\begin{lisp}
(APPEND (LIST (R S)))
\end{lisp}
and the value of this value is \cd{(24)}.  We conclude
that the net effect
of twice-evaluating \cd{{\Xbq}{\Xbq}(,,q)} is to take
the value \cd{24} of the value \cd{(r~s)} of \cd{q}
and plug it into the template \cd{(~)} to produce \cd{(24)}.

With simplification, the notation
\cd{\$\$(\%\%q)}
is read as the expression
\begin{lisp}
(LIST 'LIST Q)
\end{lisp}
The value of this expression is
\begin{lisp}
(LIST (R S))
\end{lisp}
and the value of this value is \cd{(24)}.
Thus the two ways of reading \cd{\$\$(\%\%q)} do not produce the
same expression---this we expected---but the values of the two ways are
different as well.  Only the values of the values are the same.
In general, Common Lisp guarantees the result of
an expression with backquotes nested to depth {\it k} only after
{\it k} successive evaluations have been performed; the results after
fewer than {\it k} evaluations are implementation-dependent.

(Note that in the expression \cd{`(foo ,(process `(bar ,x)))}
the backquotes are {\it not} doubly nested.  The inner backquoted
expression occurs within the textual scope of a comma belonging
to the outer backquote.  The correct way to determine the backquote
nesting level of any subexpression is to start a count at zero and
proceed up the S-expression tree, adding one for each backquote
and subtracting one for each comma.  This is similar to the rule
for determining nesting level with respect to parentheses by scanning
a character string linearly, adding or subtracting one as parentheses
are passed.)

It is convenient to extend the ``\EQ'' notation to handle multiple evaluation:
{\it x}~\EQ\EQ~{\it y} means that the expressions {\it x} and {\it y} may have
different results but they have the same results when twice evaluated.
Similarly, {\it x}~\EQ\EQ\EQ~{\it y} means that the values of the values of the
values of {\it x} and {\it y} are the same, and so on.

We can illustrate the differences between non-splicing and splicing
backquote inclusions quite concisely:
\begin{lisp}
\$\$(\%\%q)~~\EQ \\*
~~(APPEND (LIST 'APPEND) (LIST (APPEND (LIST 'LIST) (LIST Q)))) \\*
~~\EQ\EQ\ (LIST 'LIST Q) \EV\ (LIST (R S)) \EV\ (24)
\end{lisp}
\begin{lisp}
\$\$(\%{\Xatsign}\%q) \EQ \\*
~~(APPEND (LIST 'APPEND) (LIST Q)) \\*
~~\EQ\EQ\ Q \EV\ (R S) \EV\ 24
\end{lisp}
\begin{lisp}
\$\$(\%\%{\Xatsign}q) \EQ \\*
~~(APPEND (LIST 'APPEND) (LIST (APPEND (LIST 'LIST) Q))) \\*
~~\EQ\EQ\ (CONS 'LIST Q) \EV\ (LIST R S) \EV\ ((3 5) (4 6))
\end{lisp}
\begin{lisp}
\$\$(\%{\Xatsign}\%{\Xatsign}q) \EQ \\*
~~(APPEND (LIST 'APPEND) Q) \\*
~~\EQ\EQ\ (CONS 'APPEND Q) \EV\ (APPEND R S) \EV\ (3 5 4 6)
\end{lisp}
In each case I have shown both the unsimplified and simplified forms
and then traced the intermediate evaluations of the simplified form.
(Actually, the unsimplified forms do contain one simplification
without which they would be unreadable:
the \cd{nil} that terminates each list has been systematically suppressed,
so that one sees \cd{(append~{\it x}~{\it y})} rather than
\cd{(append~{\it x}~{\it y}~'nil)}.)

The following driver function is useful for tracing the behavior
of nested backquote syntax through multiple evaluations.
The argument \cd{ls} is a list of strings; each string
will be processed by the reader (\cd{read-from-string}).
The argument \cd{n} is the number of evaluations desired.
\begin{lisp}
(defun try (ls \&optional (n 0)) \\*
~~(dolist (x ls) \\*
~~~~(format t "{\Xtilde}\&{\Xtilde}A" \\*
~~~~~~~~~~~~(substitute \#{\Xbackslash}{\Xbq} \#{\Xbackslash}\$ (substitute \#{\Xbackslash}, \#{\Xbackslash}\% x))) \\
~~~~(do ((form (macroexpand (read-from-string x)) (eval form)) \\*
~~~~~~~~~(str " = " "{\Xtilde}\% => ") \\*
~~~~~~~~~(j 0 (+ j 1))) \\
~~~~~~~~((>= j n) \\*
~~~~~~~~~(format t str) \\*
~~~~~~~~~(write form :pretty t)) \\
~~~~~~(format t str) \\*
~~~~~~(write form :pretty t))) \\*
~~(format t "{\Xtilde}\&"))
\end{lisp}
This driver routine makes it easdy to explore a large number of cases
systematically.  Here is a list of examples that illustrate not only
the differences between \cd{,} and \cd{,{\Xatsign}} but also their
interaction with \cd{'}.
\begin{lisp}
(setq fools2 '( \\*
"\$\$(foo \%\%p)" \\*
"\$\$(foo \%\%{\Xatsign}q)" \\*
"\$\$(foo \%'\%r)" \\
"\$\$(foo \%'\%{\Xatsign}s)" \\
"\$\$(foo \%{\Xatsign}\%p)" \\
"\$\$(foo \%{\Xatsign}\%{\Xatsign}q)" \\
"\$\$(foo \%{\Xatsign}'\%r)" \\*
"\$\$(foo \%{\Xatsign}'\%{\Xatsign}s)" \\*
))
\end{lisp}

Consider this set of sample values:
\begin{lisp}
(setq p '(union x y)) \\*
(setq q '((union x y) (list 'sqrt 9))) \\*
(setq r '(union x y)) \\*
(setq s '((union x y)))
\end{lisp}

Here is what happened when I executed \cd{(try fools2 2)} with
a non-\cd{nil} value for the variable \cd{*bq-simplify*} (to see
simplified forms).  I have interpolated some remarks.
\begin{lisp}
{\Xbq}{\Xbq}(foo ,,p) = (LIST 'LIST ''FOO P) \\*
 => (LIST 'FOO (UNION X Y)) \\*
 => (FOO (A B C))
\end{lisp}
So \cd{,,p} means ``the value of \cd{p} is a form;
use the value of the value of \cd{p}.''
\begin{lisp}
{\Xbq}{\Xbq}(foo ,,{\Xatsign}q) = (LIST* 'LIST ''FOO Q) \\*
 => (LIST 'FOO (UNION X Y) (LIST 'SQRT 9)) \\*
 => (FOO (A B C) (SQRT 9))
\end{lisp}
So \cd{,,{\Xatsign}q} means ``the value of \cd{q} is a list of forms;
splice the list of values of the elements of the value of \cd{q}.''
\begin{lisp}
{\Xbq}{\Xbq}(foo ,',r) = (LIST 'LIST ''FOO (LIST 'QUOTE R)) \\*
 => (LIST 'FOO '(UNION X Y)) \\*
 => (FOO (UNION X Y))
\end{lisp}
So \cd{,',r} means ``the value of \cd{r} may be any object;
use the value of \cd{r}
that is available at the time of first evaluation,
that is, when the outer backquote is evaluated.''
(To use the value of \cd{r} that is available at the time of second evaluation,
that is, when the inner backquote is evaluated,
just use \cd{,r}.)
\begin{lisp}
{\Xbq}{\Xbq}(foo ,',{\Xatsign}s) = (LIST 'LIST ''FOO (CONS 'QUOTE S)) \\*
 => (LIST 'FOO '(UNION X Y)) \\*
 => (FOO (UNION X Y))
\end{lisp}
So \cd{,',{\Xatsign}s} means ``the value of \cd{s} must be a singleton list of any object;
use the element of the value of \cd{s}
that is available at the time of first evaluation,
that is, when the outer backquote is evaluated.''
Note that \cd{s} must be a singleton list because it will be spliced
into a form \cd{(quote~)}, and the \cd{quote} special form requires exactly
one subform to appear; this is generally true of the sequence \cd{',{\Xatsign}}.
(To use the value of \cd{s} that is available at the time of second evaluation,
that is, when the inner backquote is evaluated,
just use \cd{,{\Xatsign}s},in which case the list \cd{s} is not restricted to be singleton,
or \cd{,(car~s)}.)
\begin{lisp}
{\Xbq}{\Xbq}(foo ,{\Xatsign},p) = (LIST 'CONS ''FOO P) \\*
 => (CONS 'FOO (UNION X Y)) \\*
 => (FOO A B C)
\end{lisp}
So \cd{,{\Xatsign},p} means ``the value of \cd{p} is a form;
splice in the value of the value of \cd{p}.''
\begin{lisp}
{\Xbq}{\Xbq}(foo ,{\Xatsign},{\Xatsign}q) = (LIST 'CONS ''FOO (CONS 'APPEND Q)) \\*
 => (CONS 'FOO (APPEND (UNION X Y) (LIST 'SQRT 9))) \\*
 => (FOO A B C SQRT 9)
\end{lisp}
So \cd{,{\Xatsign},{\Xatsign}q} means ``the value of \cd{q} is a list of forms;
splice each of the values of the elements of the value of \cd{q},
so that many splicings occur.''
\begin{lisp}
{\Xbq}{\Xbq}(foo ,{\Xatsign}',r) = (LIST 'CONS ''FOO (LIST 'QUOTE R)) \\*
 => (CONS 'FOO '(UNION X Y)) \\*
 => (FOO UNION X Y)
\end{lisp}
So \cd{,{\Xatsign}',r} means ``the value of \cd{r} must be a list;
splice in the value of \cd{r}
that is available at the time of first evaluation,
that is, when the outer backquote is evaluated.''
(To splice the value of \cd{r} that is available at the time of second evaluation,
that is, when the inner backquote is evaluated,
just use \cd{,{\Xatsign}r}.)
\begin{lisp}
{\Xbq}{\Xbq}(foo ,{\Xatsign}',{\Xatsign}s) = (LIST 'CONS ''FOO (CONS 'QUOTE S)) \\*
 => (CONS 'FOO '(UNION X Y)) \\*
 => (FOO UNION X Y)
\end{lisp}
So \cd{,{\Xatsign}',{\Xatsign}s} means ``the value of \cd{s} must be a singleton list whose
element is a list;
splice in the list that is the element of the value of \cd{s}
that is available at the time of first evaluation,
that is, when the outer backquote is evaluated.''
(To splice the element of the value of \cd{s} that is available at the time of second evaluation,
that is, when the inner backquote is evaluated,
just use \cd{,{\Xatsign}(car~s)}.)

I leave it to the reader to explore the possibilities of triply nested backquotes.
\begin{lisp}
(setq fools3 '( \\*
"\$\$\$(foo \%\%\%p)" ~~~~"\$\$\$(foo \%\%\%{\Xatsign}q)" \\*
"\$\$\$(foo \%\%'\%r)" ~~~"\$\$\$(foo \%\%'\%{\Xatsign}s)" \\
"\$\$\$(foo \%\%{\Xatsign}\%p)" ~~~"\$\$\$(foo \%\%{\Xatsign}\%{\Xatsign}q)" \\
"\$\$\$(foo \%\%{\Xatsign}'\%r)" ~~"\$\$\$(foo \%\%{\Xatsign}'\%{\Xatsign}s)" \\
"\$\$\$(foo \%'\%\%p)" ~~~"\$\$\$(foo \%'\%\%{\Xatsign}q)" \\
"\$\$\$(foo \%'\%'\%r)" ~~"\$\$\$(foo \%'\%'\%{\Xatsign}s)" \\
"\$\$\$(foo \%'\%{\Xatsign}\%p)" ~~"\$\$\$(foo \%'\%{\Xatsign}\%{\Xatsign}q)" \\
"\$\$\$(foo \%'\%{\Xatsign}'\%r)" ~"\$\$\$(foo \%'\%{\Xatsign}'\%{\Xatsign}s)" \\
"\$\$\$(foo \%{\Xatsign}\%\%p)" ~~~"\$\$\$(foo \%{\Xatsign}\%\%{\Xatsign}q)" \\
"\$\$\$(foo \%{\Xatsign}\%'\%r)" ~~"\$\$\$(foo \%{\Xatsign}\%'\%{\Xatsign}s)" \\
"\$\$\$(foo \%{\Xatsign}\%{\Xatsign}\%p)" ~~"\$\$\$(foo \%{\Xatsign}\%{\Xatsign}\%{\Xatsign}q)" \\
"\$\$\$(foo \%{\Xatsign}\%{\Xatsign}'\%r)" ~"\$\$\$(foo \%{\Xatsign}\%{\Xatsign}'\%{\Xatsign}s)" \\
"\$\$\$(foo \%{\Xatsign}'\%\%p)" ~~"\$\$\$(foo \%{\Xatsign}'\%\%{\Xatsign}q)" \\
"\$\$\$(foo \%{\Xatsign}'\%'\%r)" ~"\$\$\$(foo \%{\Xatsign}'\%'\%{\Xatsign}s)" \\
"\$\$\$(foo \%{\Xatsign}'\%{\Xatsign}\%p)" ~"\$\$\$(foo \%{\Xatsign}'\%{\Xatsign}\%{\Xatsign}q)" \\*
"\$\$\$(foo \%{\Xatsign}'\%{\Xatsign}'\%r)" "\$\$\$(foo \%{\Xatsign}'\%{\Xatsign}'\%{\Xatsign}s)" \\*
))
\end{lisp}
It is a pleasant exercise to construct values for \cd{p}, \cd{q}, \cd{r}, and \cd{s}
that will allow execution of \cd{(try~fools3~3)} without error.

\end{new}
\endgroup


%%% Bibliography for Common Lisp book.
% Copyright 1984, 1989 Guy L. Steele Jr.  All rights reserved.

% Real data is in file clm.bbl

\bibliography{books}
         % Bibliography
%% 
\clearpage\def\pagestatus{FINAL PROOF}


\chapter*{Index of X3J13 Votes\markboth
    {Index of X3J13 Votes}{Index of X3J13 Votes}}

\label{ISSUES}

This is an index of issues voted upon by X3J13.  For the benefit of
those readers who may wish to cross-reference to the X3J13 working documents
or to the minutes of the X3J13 meetings, each vote is identified below
by the (sometimes whimsical) descriptive label used in X3J13 discussions.
Each label consists of the name of an issue and the name
of the solution that was approved (many issues had more than one proposed solution)
separated by a colon.  A few solutions had no explicit name.
Page numbers indicate where each issue is cited in the text;
a following number in parentheses indicates that the issue is cited
that many times on the page.

\begingroup
\small
\list{$\langle$\arabic{enumi}$\rangle$}{\settowidth
  \labelwidth{$\langle$999$\rangle$}\leftmargin\labelwidth
  \advance\leftmargin\labelsep \itemsep 0pt plus 2pt \parsep 0pt\usecounter{enumi}}
\raggedright

\issueitem{ADJUST-ARRAY-DISPLACEMENT}[RULES] {\footnotesize 458}
\issueitem{ADJUST-ARRAY-FILL-POINTER}[MINIMAL] {\footnotesize 456}
\issueitem{ADJUST-ARRAY-NOT-ADJUSTABLE}[IMPLICIT-COPY] {\footnotesize 32, 444, 445, 452, 455, 457}
\issueitem{ALLOW-LOCAL-INLINE}[INLINE-NOTINLINE] {\footnotesize 230}
\issueitem{APPLYHOOK-ENVIROMENT}[REMOVE-ENV] {\footnotesize 491, 493}
\issueitem{AREF-1D}[ROW-MAJOR-AREF] {\footnotesize 125, 450}
\issueitem{ARGUMENTS-UNDERSPECIFIED}[SPECIFY] {\footnotesize 122, 360, 394, 416, 437, 464, 541, 546, 567}
\issueitem{ARRAY-TYPE-ELEMENT-TYPE-SEMANTICS}[UNIFY-UPGRADING] {\footnotesize 53, 54, 57, 67, 96, 98, 443}
\issueitem{ASSOC-RASSOC-IF-KEY}[YES] {\footnotesize 432, 433}
\issueitem{BREAK-ON-WARNINGS-OBSOLETE}[REMOVE] {\footnotesize 668, 669, 889}
\issueitem{CHARACTER-PROPOSAL} {\footnotesize 23, 26(2), 33, 39, 40(2), 49, 50(2), 56, 61, 64(2), 126(2), 
                               134, 238, 243, 266, 371, 374, 375(2), 376, 379, 381, 382(3), 
                               383, 384(2), 385, 386, 387(2), 394, 442, 460, 461, 464, 466, 515,
                               533, 540, 588}
\issueitem{CLOS} {\footnotesize 15, 51, 73, 140, 155, 216, 472, 695, 770, 921}
\issueitem{CLOS-MACRO-COMPILATION}[MINIMAL] {\footnotesize 690}
\issueitem{CLOSE-CONSTRUCTED-STREAM}[ARGUMENT-STREAM-ONLY] {\footnotesize 506}
\issueitem{CLOSED-STREAM-OPERATIONS}[ALLOW-INQUIRY] {\footnotesize 504, 505, 638, 639, 641, 644, 645, 646, 654, 663}
\issueitem{COLON-NUMBER}[UNDEFINED] {\footnotesize 521, 522}
\issueitem{COMMON-TYPE}[REMOVE] {\footnotesize 12, 41, 49, 50, 51, 103}
\issueitem{COMPILE-ARGUMENT-PROBLEMS}[CLARIFY] {\footnotesize 677}
\issueitem{COMPILE-ENVIRONMENT-CONSISTENCY}[CLARIFY] {\footnotesize 685}
\issueitem{COMPILE-FILE-HANDLING-OF-TOP-LEVEL-FORMS}[CLARIFY] {\footnotesize 687}
\issueitem{COMPILE-FILE-PACKAGE}[REBIND] {\footnotesize 262, 678}
\issueitem{COMPILE-FILE-SYMBOL-HANDLING}[NEW-REQUIRE-CONSISTENCY] {\footnotesize 678, 692}
\issueitem{COMPILED-FUNCTION-REQUIREMENTS}[TIGHTEN] {\footnotesize 685}
\issueitem{COMPILER-DIAGNOSTICS}[USE-HANDLER] {\footnotesize 677, 678, 683}
\issueitem{COMPILER-LET-CONFUSION}[ELIMINATE] {\footnotesize 73, 151}
\issueitem{COMPILER-VERBOSITY}[LIKE-LOAD] {\footnotesize 657, 658, 678, 680(2)}
\issueitem{COMPILER-WARNING-STREAM}[ERROR-OUTPUT] {\footnotesize 683}
\issueitem{COMPLEX-ATAN-BRANCH-CUT}[TWEAK] {\footnotesize 307, 309, 312}
\issueitem{COMPLEX-RATIONAL-RESULT}[EXTEND] {\footnotesize 299, 300}
\issueitem{CONDITION-SYSTEM} {\footnotesize 14, 664, 666, 667, 669, 670, 671, 672, 673, 674(3), 865}
\issueitem{CONDITION-RESTARTS}[PERMIT-ASSOCIATION] {\footnotesize 216, 865, 910}
\issueitem{CONSTANT-CIRCULAR-COMPILATION}[YES] {\footnotesize 115, 694}
\issueitem{CONSTANT-COLLAPSING}[GENERALIZE] {\footnotesize 694}
\issueitem{CONSTANT-COMPILABLE-TYPES}[SPECIFY] {\footnotesize 115, 691}
\issueitem{CONSTANT-FUNCTION-COMPILATION}[NO] {\footnotesize 693}
\issueitem{CONSTANT-MODIFICATION}[DISALLOW] {\footnotesize 70, 115, 694}
\issueitem{CONTAGION-ON-NUMERICAL-COMPARISONS}[TRANSITIVE] {\footnotesize 109, 290, 437}
\issueitem{COPY-SYMBOL-COPY-PLIST}[COPY-LIST] {\footnotesize 244}
\issueitem{COPY-SYMBOL-PRINT-NAME}[EQUAL] {\footnotesize 244}
\issueitem{DATA-IO}[ADD-SUPPORT] {\footnotesize 216, 524, 534, 539, 551, 552, 553(2), 554(2), 
                                 555, 556(2), 557(2), 565, 577, 579, 580, 851}
\issueitem{DATA-TYPES-HIERARCHY-UNDERSPECIFIED}[DISJOINT] {\footnotesize 38, 41, 479, 782, 783}
\issueitem{DECLARATION-SCOPE}[NO-HOISTING] {\footnotesize 219}
\issueitem{DECLARE-ARRAY-TYPE-ELEMENT-REFERENCES}[RESTRICTIVE] {\footnotesize 55}
\issueitem{DECLARE-FUNCTION-AMBIGUITY}[DELETE-FTYPE-ABBREVIATION] {\footnotesize 228}
\issueitem{DECLARE-MACROS}[FLUSH] {\footnotesize 217}
\issueitem{DECLARE-TYPE-FREE}[LEXICAL] {\footnotesize 219, 222, 224}
\issueitem{DECODE-UNIVERSAL-TIME-DAYLIGHT}[LIKE-ENCODE] {\footnotesize 704}
\issueitem{DEFCONSTANT-SPECIAL}[DOESNT-MATTER] {\footnotesize 87}
\issueitem{DEFINE-COMPILER-MACRO}[NEW-FACILITY] {\footnotesize 125, 205, 260}
\issueitem{DEFINING-MACROS-NON-TOP-LEVEL}[ALLOW] {\footnotesize 63, 84(2), 139, 143, 153, 195, 207, 472}
\issueitem{DEFMACRO-LAMBDA-LIST}[TIGHTEN-DESCRIPTION] {\footnotesize 197}
\issueitem{DEFPACKAGE}[ADDITION] {\footnotesize 269, 280}
\issueitem{DEFSTRUCT-CONSTRUCTOR-KEY-MIXTURE}[ALLOW-KEY] {\footnotesize 483}
\issueitem{DEFSTRUCT-DEFAULT-VALUE-EVALUATION}[IFF-NEEDED] {\footnotesize 472, 474}
\issueitem{DEFSTRUCT-PRINT-FUNCTION-INHERITANCE}[YES] {\footnotesize 480}
\issueitem{DEFSTRUCT-REDEFINITION}[ERROR] {\footnotesize 473}
\issueitem{DEFSTRUCT-SLOTS-CONSTRAINTS-NAME}[DUPLICATES-ERROR] {\footnotesize 472}
\issueitem{DEFSTRUCT-SLOTS-CONSTRAINTS-NUMBER}[ALLOW-ZERO] {\footnotesize 471}
\issueitem{DEFVAR-DOCUMENTATION}[UNEVALUATED] {\footnotesize 87}
\issueitem{DEFVAR-INIT-TIME}[NOT-DELAYED] {\footnotesize 86}
\issueitem{DEFVAR-INITIALIZATION}[CONSERVATIVE] {\footnotesize 86}
\issueitem{DESCRIBE-INTERACTIVE}[EXPLICITLY-VAGUE] {\footnotesize 697}
\issueitem{DESCRIBE-UNDERSPECIFIED}[DESCRIBE-OBJECT] {\footnotesize 697, 698, 817, 840}
\issueitem{DESTRUCTURING-BIND}[NEW-MACRO] {\footnotesize 204, 207}
\issueitem{DISASSEMBLE-SIDE-EFFECT}[DO-NOT-INSTALL] {\footnotesize 682}
\issueitem{DO-SYMBOLS-DUPLICATES}[ALLOWED] {\footnotesize 274}
\issueitem{DOTTED-MACRO-FORMS}[ALLOW] {\footnotesize 74}
\issueitem{DRIBBLE-TECHNIQUE}[MAKE-EXPLICITLY-VAGUE] {\footnotesize 700}
\issueitem{DYNAMIC-EXTENT}[NEW-DECLARATION] {\footnotesize 232}
\issueitem{DYNAMIC-EXTENT-FUNCTION}[EXTEND] {\footnotesize 232}
\issueitem{EQUAL-STRUCTURE}[MAYBE-STATUS-QUO] {\footnotesize 107, 108}
\issueitem{EVAL-OTHER}[SELF-EVALUATE] {\footnotesize 70}
\issueitem{EVAL-WHEN-NON-TOP-LEVEL}[GENERALIZE-EVAL-NEW-KEYWORDS] {\footnotesize 89, 207}
\issueitem{EXIT-EXTENT}[MINIMAL] {\footnotesize 189}
\issueitem{EXPT-RATIO}[P.211] {\footnotesize 301}
\issueitem{FIXNUM-NON-PORTABLE}[TIGHTEN-DEFINITION] {\footnotesize 16, 39, 368, 446}
\issueitem{FLET-DECLARATIONS}[ALLOW] {\footnotesize 154}
\issueitem{FLET-IMPLICIT-BLOCK}[YES] {\footnotesize 63, 139, 143, 154, 196, 206}
\issueitem{FLOAT-UNDERFLOW}[ADD-CONTROLS] {\footnotesize 289, 369(2)}
\issueitem{FORMAT-ATSIGN-COLON}[OK] {\footnotesize 582}
\issueitem{FORMAT-COLON-UPARROW-SCOPE}[TEST-FOR-REMAINING-SUBLISTS] {\footnotesize 606}
\issueitem{FORMAT-COMMA-INTERVAL}[YES] {\footnotesize 585}
\issueitem{FORMAT-E-EXPONENT-SIGN}[FORCE-SIGN] {\footnotesize 592}
\issueitem{FORMAT-OP-C}[WRITE-CHAR] {\footnotesize 588}
\issueitem{FORMAT-PRETTY-PRINT}[YES] {\footnotesize 583, 584(2), 585(2), 586(4), 588, 590, 592, 594, 596}
\issueitem{FUNCTION-CALL-EVALUATION-ORDER}[UNSPECIFIED] {\footnotesize 75}
\issueitem{FUNCTION-COMPOSITION}[JAN89-X3J13] {\footnotesize 391}
\issueitem{FUNCTION-DEFINITION}[JAN89-X3J13] {\footnotesize 682}
\issueitem{FUNCTION-NAME}[SMALL] {\footnotesize 84, 114, 116, 120(2), 123, 125(2), 126(2), 127, 128(3), 154, 
                                 227, 230(2), 677, 682, 695, 696, 699, 827, 842}
\issueitem{FUNCTION-TYPE}[X3J13-MARCH-88] {\footnotesize 14, 36, 38, 53, 65, 102, 116, 119, 120, 145, 146, 
                                           173, 203, 389, 391, 492, 503, 504}
\issueitem{FUNCTION-TYPE-ARGUMENT-TYPE-SEMANTICS}[RESTRICTIVE] {\footnotesize 58, 227, 228}
\issueitem{FUNCTION-TYPE-KEY-NAME}[SPECIFY-KEYWORD] {\footnotesize 57}
\issueitem{FUNCTION-TYPE-REST-LIST-ELEMENT}[USE-ACTUAL-ARGUMENT-TYPE] {\footnotesize 57}
\issueitem{GENSYM-NAME-STICKINESS}[LIKE-TEFLON] {\footnotesize 245, 246}
\issueitem{GET-MACRO-CHARACTER-READTABLE}[NIL-STANDARD] {\footnotesize 542, 548}
\issueitem{GET-SETF-METHOD-ENVIRONMENT}[ADD-ARG] {\footnotesize 137, 142, 144, 145(2)}
\issueitem{HASH-TABLE-ACCESS}[X3J13-MAR-89] {\footnotesize 440}
\issueitem{HASH-TABLE-PACKAGE-GENERATORS}[ADD-WITH-WRAPPER] {\footnotesize 275, 439}
\issueitem{HASH-TABLE-SIZE}[INTENDED-ENTRIES] {\footnotesize 436, 437(2)}
\issueitem{HASH-TABLE-TESTS}[ADD-EQUALP] {\footnotesize 437}
\issueitem{IEEE-ATAN-BRANCH-CUT}[SPLIT] {\footnotesize 302(2), 303, 306, 309, 310(2), 311}
\issueitem{IMPORT-SETF-SYMBOL-PACKAGE}[YES] {\footnotesize 268}
\issueitem{IN-PACKAGE-FUNCTIONALITY}[MAR89-X3J13] {\footnotesize 261, 263, 690}
\issueitem{IN-SYNTAX}[MINIMAL] {\footnotesize 658, 679}
\issueitem{KEYWORD-ARGUMENT-NAME-PACKAGE}[ANY] {\footnotesize 76, 79}
\issueitem{LAST-N}[ALLOW-OPTIONAL-ARGUMENT] {\footnotesize 416}
\issueitem{LCM-NO-ARGUMENTS}[1] {\footnotesize 299}
\issueitem{LISP-PACKAGE-NAME}[COMMON-LISP] {\footnotesize 258, 262, 263(2), 278, 280}
\issueitem{LISP-SYMBOL-REDEFINITION}[MAR89-X3J13] {\footnotesize 260}
\issueitem{LOAD-OBJECTS}[MAKE-LOAD-FORM] {\footnotesize 659, 694}
\issueitem{LOAD-TIME-EVAL}[R**3-NEW-SPECIAL-FORM] {\footnotesize 73, 680}
\issueitem{LOAD-TRUENAME}[NEW-PATHNAME-VARIABLES] {\footnotesize 658(2), 659, 680(3)}
\issueitem{LOCALLY-TOP-LEVEL}[SPECIAL-FORM] {\footnotesize 73, 90, 221}
\issueitem{LOOP-AND-DISCREPANCY}[NO-REITERATION] {\footnotesize 716}
\issueitem{LOOP-FACILITY} {\footnotesize 163, 709}
\issueitem{MACRO-CACHING}[DISALLOW] {\footnotesize 203}
\issueitem{MACRO-ENVIRONMENT-EXTENT}[DYNAMIC] {\footnotesize 197, 203, 204}
\issueitem{MACRO-FUNCTION-ENVIRONMENT}[YES] {\footnotesize 194}
\issueitem{MAKE-PACKAGE-USE-DEFAULT}[IMPLEMENTATION-DEPENDENT] {\footnotesize 263, 271}
\issueitem{MAP-INTO}[ADD-FUNCTION] {\footnotesize 395}
\issueitem{MAPPING-DESTRUCTIVE-INTERACTION}[EXPLICITLY-VAGUE] {\footnotesize 169, 173, 178, 275(3), 277, 395, 397,
           398, 400, 401, 402, 403, 404(2), 405, 407(3), 409, 410, 413, 425(2), 426(3), 427, 
           429(2), 430, 431(2), 433, 434, 439(2), 491}
\issueitem{MORE-CHARACTER-PROPOSAL} {\footnotesize 373(2), 374, 502, 504, 508, 646, 647, 651, 656}
\issueitem{NTH-VALUE}[ADD] {\footnotesize 184}
\issueitem{OPTIMIZE-DEBUG-INFO}[NEW-QUALITY] {\footnotesize 231}
\issueitem{PACKAGE-CLUTTER}[REDUCE] {\footnotesize 259}
\issueitem{PACKAGE-DELETION}[NEW-FUNCTION] {\footnotesize 264, 265}
\issueitem{PACKAGE-FUNCTION-CONSISTENCY}[MORE-PERMISSIVE] {\footnotesize 250, 264(3), 265(4), 267(3), 268(4), 
               269(3), 274, 275(2)}
\issueitem{PATHNAME-COMPONENT-CASE}[KEYWORD-ARGUMENT] {\footnotesize 617, 625, 641, 643, 644}
\issueitem{PATHNAME-COMPONENT-VALUE}[SPECIFY] {\footnotesize 615, 623}
\issueitem{PATHNAME-LOGICAL}[ADD] {\footnotesize 628, 639, 640, 641, 646, 652, 653, 654(2), 655(2), 658, 
            663, 678, 699(2)}
\issueitem{PATHNAME-PRINT-READ}[SHARPSIGN-P] {\footnotesize 531, 537, 556}
\issueitem{PATHNAME-STREAM}[FILES-OR-SYNONYM] {\footnotesize 278, 638(2), 639, 640, 641, 644, 645, 646, 651, 653(2),
           654, 655(2), 658, 663, 678}
\issueitem{PATHNAME-SUBDIRECTORY-LIST}[NEW-REPRESENTATION] {\footnotesize 615, 617, 620, 644}
\issueitem{PATHNAME-SYMBOL}[NO] {\footnotesize 278, 637, 638, 639, 640, 641, 644, 645}
\issueitem{PATHNAME-SYNTAX-ERROR-TIME}[PATHNAME-CREATION] {\footnotesize 642, 643, 645}
\issueitem{PATHNAME-UNSPECIFIC-COMPONENT}[NEW-TOKEN] {\footnotesize 613}
\issueitem{PATHNAME-WILD}[NEW-FUNCTIONS] {\footnotesize 623, 639, 646, 652, 653, 654(2), 655(2), 658, 678}
\issueitem{PEEK-CHAR-READ-CHAR-ECHO}[FIRST-READ-CHAR] {\footnotesize 501, 570, 571, 573(2), 574}
\issueitem{PRETTY-PRINT-INTERFACE}[XP] {\footnotesize 24, 556, 559, 577, 578, 579, 598, 599, 605, 607, 748}
\issueitem{PRINC-CHARACTER}[WRITE-CHAR] {\footnotesize 578}
\issueitem{PRINT-CASE-PRINT-ESCAPE-INTERACTION}[VERTICAL-BAR-RULE-NO-UPCASE] {\footnotesize 552, 560}
\issueitem{PRINT-CIRCLE-SHARED}[RESPECT-PRINT-CIRCLE] {\footnotesize 559}
\issueitem{PRINT-CIRCLE-STRUCTURE}[USER-FUNCTIONS-WORK] {\footnotesize 480, 559}
\issueitem{PROCLAIM-ETC-IN-COMPILE-FILE}[NEW-MACRO] {\footnotesize 215, 223, 232, 689}
\issueitem{PROCLAIM-INLINE-WHERE}[BEFORE] {\footnotesize 229}
\issueitem{PUSH-EVALUATION-ORDER}[ITEM-FIRST] {\footnotesize 132, 242(2), 297, 420, 421, 422, 671, 672, 674(2)}
\issueitem{QUOTE-SEMANTICS}[NO-COPYING] {\footnotesize 105, 115}
\issueitem{RANGE-OF-COUNT-KEYWORD}[NIL-OR-INTEGER] {\footnotesize 400(2), 403}
\issueitem{RANGE-OF-START-AND-END-PARAMETERS}[INTEGER-AND-INTEGER-NIL] {\footnotesize 390}
\issueitem{READ-CASE-SENSITIVITY}[READTABLE-KEYWORDS] {\footnotesize 11, 28, 513(2), 515, 549, 552, 561}
\issueitem{REAL-NUMBER-TYPE}[X3J13-MAR-89] {\footnotesize 15, 38, 49, 50, 61, 101}
\issueitem{REDUCE-ARGUMENT-EXTRACTION}[KEY] {\footnotesize 398}
\issueitem{REMF-DESTRUCTION-UNSPECIFIED}[X3J13-MAR-89] {\footnotesize 241(2), 242(2), 393, 401, 402, 404, 419, 
            420, 429(2), 431}
\issueitem{REQUIRE-PATHNAME-DEFAULTS}[ELIMINATE] {\footnotesize 277, 280, 637}
\issueitem{REST-LIST-ALLOCATION}[MAY-SHARE] {\footnotesize 77}
\issueitem{RETURN-VALUES-UNSPECIFIED}[SPECIFY] {\footnotesize 221, 263, 265, 278, 542, 696, 698}
\issueitem{ROOM-DEFAULT-ARGUMENT}[NEW-VALUE] {\footnotesize 699}
\issueitem{SEQUENCE-TYPE-LENGTH}[MUST-MATCH] {\footnotesize 64, 394, 395(2), 410}
\issueitem{SETF-MULTIPLE-STORE-VARIABLES}[ALLOW] {\footnotesize 129(2), 131(2), 672}
\issueitem{SETF-SUB-METHODS}[DELAYED-ACCESS-STORES] {\footnotesize 134}
\issueitem{SHADOW-ALREADY-PRESENT}[WORKS] {\footnotesize 269}
\issueitem{SHARP-COMMA-CONFUSION}[REMOVE] {\footnotesize 523, 535, 539, 676}
\issueitem{SHARPSIGN-PLUS-MINUS-PACKAGE}[KEYWORD] {\footnotesize 539, 707}
\issueitem{SPECIAL-TYPE-SHADOWING}[CLARIFY] {\footnotesize 222}
\issueitem{STANDARD-INPUT-INITIAL-BINDING}[DEFINED-CONTRACTS] {\footnotesize 499}
\issueitem{STEP-ENVIRONMENT}[CURRENT] {\footnotesize 696, 697}
\issueitem{STREAM-ACCESS}[ADD-TYPES-ACCESSORS] {\footnotesize 35, 41, 500(2), 501(4), 502(2), 503, 504, 505, 
                507, 581, 646}
\issueitem{STREAM-CAPABILITIES}[INTERACTIVE-STREAM-P] {\footnotesize 507}
\issueitem{STRING-COERCION}[MAKE-CONSISTENT] {\footnotesize 462(2), 463(2), 465, 466, 467}
\issueitem{SUBSEQ-OUT-OF-BOUNDS}[IS-AN-ERROR] {\footnotesize 390}
\issueitem{SUBTYPEP-TOO-VAGUE}[CLARIFY-MORE] {\footnotesize 97}
\issueitem{SYMBOL-MACROLET-DECLARE}[ALLOW] {\footnotesize 155, 216, 861, 864}
\issueitem{SYMBOL-MACROLET-SEMANTICS}[SPECIAL-FORM] {\footnotesize 121, 122, 128, 155, 156, 184, 204, 861, 864}
\issueitem{SYNTACTIC-ENVIRONMENT-ACCESS}[SMALL] {\footnotesize 207}
\issueitem{TAILP-NIL}[T] {\footnotesize 427}
\issueitem{TEST-NOT-IF-NOT}[FLUSH-ALL] {\footnotesize 391}
\issueitem{THE-AMBIGUITY}[FOR-DECLARATION] {\footnotesize 237}
\issueitem{TIME-ZONE-NON-INTEGER}[ALLOW] {\footnotesize 703}
\issueitem{TYPE-OF-UNDERCONSTRAINED}[ADD-CONSTRAINTS] {\footnotesize 66}
\issueitem{UNDEFINED-VARIABLES-AND-FUNCTIONS}[COMPROMISE] {\footnotesize 71}
\issueitem{UNREAD-CHAR-AFTER-PEEK-CHAR}[DONT-ALLOW] {\footnotesize 573}
\issueitem{VARIABLE-LIST-ASYMMETRY}[SYMMETRIZE] {\footnotesize 150, 151, 164}
\issueitem{WITH-COMPILATION-UNIT}[NEW-MACRO] {\footnotesize 683}
\issueitem{WITH-OPEN-FILE-DOES-NOT-EXIST}[STREAM-IS-NIL] {\footnotesize 652}
\issueitem{WITH-OUTPUT-TO-STRING-APPEND-STYLE}[VECTOR-PUSH-EXTEND] {\footnotesize 504}

% The following issue was noticed to late to be put into the
% alphabetical order.  The simplest fix was to misspell it!

\issueitem{ZLOS-CONDITIONS}[INTEGRATE] {\footnotesize 865, 883, 886, 900, 916}

\endlist
\endgroup

\clearpage
\pdfbookmark{Index}{INDEX}
\label{INDEX}
\printindex


\chapter*{Colophon\markboth
    {Colophon}{Colophon}}


Camera-ready copy for this book was created by the author (using \TeX, \LaTeX,
and \TeX\ macros written by the author), proofed on an Apple LaserWriter II, and
printed on a Linotron 300 at Advanced Computer Graphics.  The text of the first
edition was converted from Scribe format to \TeX\ format by a throwaway program
written in Common Lisp.  The diagrams in chapter 12 were generated automatically
as PostScript code (by a program written in Common Lisp) and integrated into the
text by Textures, an implementation of \TeX\ by Blue Sky Research for the Apple
Macintosh computer.

The body type is 10-point Times Roman. Chapter titles are in ITC Eras Demi;
running heads and chapter subtitles are in ITC Eras Book.  The monospace typeface used for program
code in both displays and running text is 8.5-point Letter Gothic Bold, somewhat
modified by the author through TEX macros for improved legibility. The accent
grave (\cd{\Xbq}), accent acute(\cd{\Xquote}),
circumflex (\cd{\Xcircumflex}), and tilde (\cd{\Xtilde})
characters are in 10-point Letter Gothic
Bold and adjusted vertically to match the height of the 8.5-point characters.  The
hyphen (\cd{\char45}) was replaced by an en dash (\cd{-}).
The equals sign (\cd{\char61}) was replaced by a construction of two em
dashes (\cd{=}), one raised and one lowered, the better to match the other
relational characters.  The sharp sign (\cd{\char35})
is overstruck with two hyphens,
one raised and one lowered, to eliminate the vertical gap (\cd{\#}).  Special mathematical
characters such as square-root signs are in Computer Modern Math. The typefaces
used in this book were digitized by Adobe Systems Incorporated, except for
Computer Modern Math, which was designed by Donald E. Knuth.
      % Index to ANSI issues

% index

\clearpage\def\pagestatus{FINAL PROOF}
%\pdfbookmark{Index of X3J13 Votes}{ISSUES}
%\label{ISSUES}
\ifx \rulang\Undef
\printindex{issues}{Index of X3J13 Votes}
\else
\printindex{issues}{Указатель голосования X3J13}
\fi

\clearpage\def\pagestatus{FINAL PROOF}
%\pdfbookmark{Symbols Index}{INDEX}
%\label{INDEX}
\ifx \rulang\Undef
\printindex{symbols}{Symbols Index}
\else
\printindex{symbols}{Указатель символов}
\fi

%Part{Dtypes, Root = "CLM.MSS"}
% Chapter of Common Lisp Manual.  Copyright 1984, 1988, 1989 Guy L. Steele Jr.

\clearpage\def\pagestatus{FINAL PROOF}

%\chapter*{Colophon\markboth{Colophon}{Colophon}}
\chapter*{Colophon}


Camera-ready copy for this book was created by the author (using \TeX, \LaTeX,
and \TeX\ macros written by the author), proofed on an Apple LaserWriter II, and
printed on a Linotron 300 at Advanced Computer Graphics.  The text of the first
edition was converted from Scribe format to \TeX\ format by a throwaway program
written in Common Lisp.  The diagrams in chapter 12 were generated automatically
as PostScript code (by a program written in Common Lisp) and integrated into the
text by Textures, an implementation of \TeX\ by Blue Sky Research for the Apple
Macintosh computer.

The body type is 10-point Times Roman. Chapter titles are in ITC Eras Demi;
running heads and chapter subtitles are in ITC Eras Book.  The monospace typeface used for program
code in both displays and running text is 8.5-point Letter Gothic Bold, somewhat
modified by the author through TEX macros for improved legibility. The accent
grave (\cd{\Xbq}), accent acute(\cd{\Xquote}),
circumflex (\cd{\Xcircumflex}), and tilde (\cd{\Xtilde})
characters are in 10-point Letter Gothic
Bold and adjusted vertically to match the height of the 8.5-point characters.  The
hyphen (\cd{\char45}) was replaced by an en dash (\cdf{-}).
The equals sign (\cd{\char61}) was replaced by a construction of two em
dashes (\cd{=}), one raised and one lowered, the better to match the other
relational characters.  The sharp sign (\cd{\char35})
is overstruck with two hyphens,
one raised and one lowered, to eliminate the vertical gap (\cd{\#}).  Special mathematical
characters such as square-root signs are in Computer Modern Math. The typefaces
used in this book were digitized by Adobe Systems Incorporated, except for
Computer Modern Math, which was designed by Donald E. Knuth.

%%% Local Variables: 
%%% mode: latex
%%% TeX-master: "clm"
%%% End: 


\end{document}
