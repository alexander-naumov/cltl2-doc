%Part{Symbol, Root = "CLM.MSS"}
%Chapter of Common Lisp Manual.  Copyright 1984, 1988, 1989 Guy L. Steele Jr.

\clearpage\def\pagestatus{ULTIMATE}

\ifx \rulang\Undef

\chapter{Symbols}
\label{symbol}

A Lisp symbol is a data object that has three user-visible
components:
\begin{itemize}
\item
The \emph{property list} is a list that effectively provides each symbol
with many modifiable named components.

\item
The \emph{print name} must be a string, which is the sequence of
characters used to identify the symbol.  Symbols are of great use
because a symbol can be located once its name is given
(typed, say, on a keyboard).
One may ordinarily not alter a symbol's print name.
\end{itemize}

It is an error to alter a print name.

\begin{itemize}
\item
The \emph{package cell} must refer to a package object.
A package is a data structure
used to locate a symbol once given the symbol's name.
A symbol is uniquely identified
by its name only when considered relative to a package.  A symbol may
appear in many packages, but it can be \emph{owned} by at most one package.
The package cell points to the owner, if any.
Package cells are discussed along with packages in chapter~\ref{XPACK}.
\end{itemize}

A symbol may actually have other components for use by the
implementation.  One of the more important uses of symbols is as
names for program variables; it is frequently desirable for the
implementor to use certain components of a symbol to implement
the semantics of variables.  See \cdf{symbol-value}
and \cdf{symbol-function}.
However, there are several possible
implementation strategies, and so such possible components are not
described here.

\section{The Property List}

Since its inception, Lisp has associated with each symbol
a kind of tabular data structure
called a \emph{property list} (\emph{plist} for short).  A property list contains
zero or more entries; each entry associates with a key
(called the \emph{indicator}), which is typically
a symbol, an arbitrary Lisp object (called the \emph{value} or,
sometimes, the \emph{property}).
There are no duplications among the indicators; a property list may only
have one property at a time with a given name.  In this way, given
a symbol and an indicator (another symbol), an associated value can be
retrieved.

A property
list is very similar in purpose to an association list. The difference
is that a property list is an object with a unique identity; the
operations for adding and removing property-list entries are destructive
operations that alter the property list rather than making a new one.
Association lists, on the other hand, are normally augmented
non-destructively (without side effects) by adding new entries to the
front (see \cdf{acons} and \cdf{pairlis}).

A property list is implemented as a memory cell
containing a list with an even number (possibly zero) of elements.
(Usually this memory cell is the property-list cell of a symbol,
but any memory cell acceptable to \cdf{setf} can be used
if \cdf{getf} and \cdf{remf} are used.)
Each pair of elements in the list constitutes an entry;
the first item is the indicator, and the second is the
value.  Because property-list functions are given the symbol
and not the list itself, modifications to the property list
can be recorded by storing back into the property-list cell of the symbol.

When a symbol is created, its property list is initially empty.
Properties are created by using \cdf{get} within a \cdf{setf} form.

Common Lisp does not use a symbol's property list as extensively as earlier
Lisp implementations did.  Less-used data, such as compiler,
debugging, and documentation information, is kept on property lists
in Common Lisp.

\begin{defun}[Function]
get symbol indicator &optional default

\cdf{get} searches the property list of
\emph{symbol} for an indicator \cdf{eq} to \emph{indicator}.
The first argument must be a symbol.
If one is found, then the corresponding value is returned;
otherwise \emph{default} is returned.

If \emph{default} is not specified,
then {\false} is used for \emph{default}.

Note that there is no way to distinguish an absent property from
one whose value is \emph{default}.
\begin{lisp}
(get x y) \EQ\ (getf (symbol-plist x) y)
\end{lisp}
Suppose that the property list of \cdf{foo} is \cd{(bar t baz 3 hunoz "Huh?")}.
Then, for example:
\begin{lisp}
(get 'foo 'baz) \EV\ 3 \\
(get 'foo 'hunoz) \EV\ "Huh?" \\
(get 'foo 'zoo) \EV\ {\false}
\end{lisp}

\cdf{setf} may be used with \cdf{get} to create a new property-value
pair, possibly replacing an old pair with the same property name.
For example:
\begin{lisp}
(get 'clyde 'species) \EV\ {\false} \\
(setf (get 'clyde 'species) 'elephant) \EV\ elephant \\
\textrm{and now} (get 'clyde 'species) \EV\ elephant
\end{lisp}
The \emph{default} argument may be
specified to \cdf{get} in this context; it is ignored by \cdf{setf} but
may be useful in such macros as \cdf{push} that are related to \cdf{setf}:
\begin{lisp}
(push item (get sym 'token-stack '(initial-item)))
\end{lisp}
means approximately the same as
\begin{lisp}
(setf (get sym 'token-stack '(initial-item)) \\
~~~~~~(cons item (get sym 'token-stack '(initial-item))))
\end{lisp}
which in turn would be treated as simply
\begin{lisp}
(setf (get sym 'token-stack) \\
~~~~~~(cons item (get sym 'token-stack '(initial-item))))
\end{lisp}

\begin{newer}
X3J13 voted in March 1989 \issue{REMF-DESTRUCTION-UNSPECIFIED}
to clarify the permissible side effects of certain operations;
\cd{(setf (get \emph{symbol} \emph{indicator}) \emph{newvalue})}
is required to behave exactly the same as
\cd{(setf (getf (symbol-plist \emph{symbol}) \emph{indicator}) \emph{newvalue})}.
\end{newer}
\end{defun}

\begin{defun}[Function]
remprop symbol indicator

This removes from \emph{symbol} the property with an indicator \cdf{eq}
to \emph{indicator}.  The property indicator and the corresponding
value are removed by destructively splicing the property
list.  It returns {\false} if no such property was found,
or non-{\false} if a property was found.
\begin{lisp}
(remprop x y) \EQ\ (remf (symbol-plist x) y)
\end{lisp}
For example, if the property list of \cdf{foo} is initially
\begin{lisp}
(color blue height 6.3 near-to bar)
\end{lisp}
then the call
\begin{lisp}
(remprop 'foo 'height)
\end{lisp}
returns a non-{\nil} value after altering \cdf{foo}'s property list to be
\begin{lisp}
(color blue near-to bar)
\end{lisp}

\begin{newer}
X3J13 voted in March 1989 \issue{REMF-DESTRUCTION-UNSPECIFIED}
to clarify the permissible side effects of certain operations;
\cd{(remprop \emph{symbol} \emph{indicator})}
is required to behave exactly the same as
\cd{(remf (symbol-plist \emph{symbol}) \emph{indicator})}.
\end{newer}
\end{defun}

\begin{defun}[Function]
symbol-plist symbol

This returns the list that contains the property pairs of \emph{symbol};
the contents of the property-list cell are extracted and returned.

Note that using \cdf{get} on the result of \cdf{symbol-plist} does \emph{not} work.
One must give the symbol itself to \cdf{get} or else
use the function \cdf{getf}.

\cdf{setf} may be used with \cdf{symbol-plist} to destructively replace
the entire property list of a symbol.  This is a relatively dangerous
operation, as it may destroy important information that
the implementation may happen to store in property lists.
Also, care must be taken that the new
property list is in fact a list of even length.
\beforenoterule
\begin{incompatibility}
In MacLisp, this function is called \cdf{plist};
in Interlisp, it is called \cdf{getproplist}.
\end{incompatibility}
\afternoterule
\end{defun}

\begin{defun}[Function]
getf place indicator &optional default

\cdf{getf} searches the property list stored in \emph{place}
for an indicator \cdf{eq} to \emph{indicator}.
If one is found, then the corresponding value is returned;
otherwise \emph{default} is returned.  If \emph{default} is not specified,
then {\false} is used for \emph{default}.
Note that there is no way to distinguish an absent property from
one whose value is \emph{default}.
Often \emph{place} is computed from
a generalized variable acceptable to \cdf{setf}.

\cdf{setf} may be used with \cdf{getf}, in which case the \emph{place} must
indeed be acceptable as a \emph{place} to \cdf{setf}.  The effect is to
add a new property-value pair, or update an existing pair,
in the property list kept in the \emph{place}.
The \emph{default} argument may be
specified to \cdf{getf} in this context; it is ignored by \cdf{setf} but
may be useful in such macros as \cdf{push} that are related to \cdf{setf}.
See the description of \cdf{get} for an example of this.

\begin{newer}
X3J13 voted in March 1989 \issue{REMF-DESTRUCTION-UNSPECIFIED}
to clarify the permissible side effects of certain operations;
\cdf{setf} used with \cdf{getf} is permitted to perform a \cdf{setf}
on the \emph{place} or on any part, \emph{car} or \emph{cdr}, of the
top-level list structure held by that \emph{place}.
\end{newer}

\begin{newer}
X3J13 voted in March 1988 \issue{PUSH-EVALUATION-ORDER}
to clarify order of evaluation (see section~\ref{SETF-SECTION}).
\end{newer}
\end{defun}

\begin{defmac}
remf place indicator

This removes from the property list stored in \emph{place}
the property with an indicator \cdf{eq}
to \emph{indicator}.  The property indicator and the corresponding
value are removed by destructively splicing the property
list.  \cdf{remf} returns {\false} if no such property was found,
or some non-{\nil} value if a property was found.
The form \emph{place} may be any generalized variable acceptable to \cdf{setf}.
See \cdf{remprop}.

\begin{newer}
X3J13 voted in March 1989 \issue{REMF-DESTRUCTION-UNSPECIFIED}
to clarify the permissible side effects of certain operations;
\cdf{remf} is permitted to perform a \cdf{setf}
on the \emph{place} or on any part, \emph{car} or \emph{cdr}, of the
top-level list structure held by that \emph{place}.
\end{newer}

\begin{newer}
X3J13 voted in March 1988 \issue{PUSH-EVALUATION-ORDER}
to clarify order of evaluation (see section~\ref{SETF-SECTION}).
\end{newer}
\end{defmac}

\begin{defun}[Function]
get-properties place indicator-list

\cdf{get-properties} is like \cdf{getf}, except that the second
argument is a list of indicators.  \cdf{get-properties} searches the
property list stored in \emph{place} for any of the indicators in
\emph{indicator-list} until it finds the first property
in the property list whose indicator is one of
the elements of \emph{indicator-list}.  Normally \emph{place} is computed from
a generalized variable acceptable to \cdf{setf}.

\cdf{get-properties} returns three values.
If any property was found, then
the first two values are the indicator
and value for the first property whose indicator was in \emph{indicator-list},
and the third is that tail of the property list whose \emph{car}
was the indicator (and whose \emph{cadr} is therefore the value).
If no property was found, all three values are {\nil}.
Thus the third value serves as a flag indicating success or failure
and also allows the search to be restarted, if desired, after the property
was found.
\end{defun}

\section{The Print Name}

Every symbol has an associated string called the \emph{print name}.
This string is used as the external representation of the symbol:
if the characters in the string are typed in to \cdf{read}
(with suitable escape conventions for certain characters),
it is interpreted as a reference to that symbol
(if it is interned); and if the symbol is printed, \cdf{print} types out the
print name.
For more information, see the sections on the \emph{reader}
(section~\ref{READER})
and \emph{printer} (section~\ref{PRINTER}).

\begin{defun}[Function]
symbol-name sym

This returns the print name of the symbol \emph{sym}.
For example:
\begin{lisp}
(symbol-name 'xyz) \EV\ "XYZ"
\end{lisp}
It is an extremely bad idea to modify a string being used as the print name of
a symbol.  Such a modification may tremendously confuse
the function \cdf{read} and the package system.

\begin{newer}
X3J13 voted in March 1989 \issue{CHARACTER-PROPOSAL}
to specify that it is an error to modify a string being used
as the print name of a symbol.
\end{newer}
\end{defun}

\section{Creating Symbols}

Symbols can be used in two rather different ways.
An \emph{interned} symbol is one that is indexed by its print name
in a catalogue called a \emph{package}.
A request to locate a symbol with that print name results
in the same (\cdf{eq}) symbol.  Every time input is read with the
function \cdf{read},
and that print name appears, it is read as the same symbol.
This property of symbols makes them appropriate to use as names for
things and as hooks on which to hang permanent data objects
(using the property list, for example).

Interned symbols are normally created automatically; the first time
something (such as the function \cdf{read})
asks the package system for a symbol with a given print name,
that symbol is automatically created.  The function used to ask for
an interned symbol is \cdf{intern}, or one of the functions
related to \cdf{intern}.

Although interned symbols are the most commonly
used, they will not be discussed further here.  For more information,
see chapter~\ref{XPACK}.

An \emph{uninterned} symbol is a symbol used simply as a data object,
with no special cataloguing (it belongs to no particular package).
An uninterned symbol is printed as \cd{\#:} followed by its
print name.
The following are some functions for creating uninterned symbols.

\begin{defun}[Function]
make-symbol print-name

\cd{(make-symbol \emph{print-name})} creates a new uninterned symbol, whose
print name is the string \emph{print-name}.  The value and function bindings will
be unbound and the property list will be empty.

The string actually installed in the symbol's print-name component
may be the given string \emph{print-name} or may be a copy of it,
at the implementation's discretion.  The user should not assume
that \cd{(symbol-name (make-symbol x))} is \cdf{eq} to \cdf{x}, but also
should not alter a string once it has been given as an argument
to \cdf{make-symbol}.

\beforenoterule
\begin{implementation}
An implementation might choose, for example,
to copy the string to some read-only area, in the expectation that
it will never be altered.
\end{implementation}
\afternoterule
\end{defun}

\begin{defun}[Function]
copy-symbol sym &optional copy-props

This returns a new uninterned symbol with the same print name
as \emph{sym}.

\begin{newer}
X3J13 voted in March 1989 \issue{COPY-SYMBOL-PRINT-NAME}
that the print name of the new symbol is required to be
the same only in the sense of \cdf{string=}; in other words,
an implementation is permitted (but not required)
to make a copy of the print name.
User programs should not assume that the print names of the old and new symbols
will be \cdf{eq}, although they may happen to be \cdf{eq} in some implementations.
\end{newer}

If \emph{copy-props} is non-{\nil}, then the initial
value and function definition of the new symbol will
be the same as those of \emph{sym}, and the property list of
the new symbol will be a copy of \emph{sym}'s.

\begin{newer}
X3J13 voted in March 1989 \issue{COPY-SYMBOL-COPY-PLIST}
to clarify that only the top-level conses of the
property list are copied; it is as if \cd{(copy-list (symbol-plist \emph{sym}))}
were used as the property list of the new symbol.
\end{newer}

If \emph{copy-props}
is {\nil} (the default), then the new symbol will be unbound and undefined, and
its property list will be empty.
\end{defun}

\begin{defun}[Function]
gensym &optional x

\cdf{gensym} invents a print name and creates a new symbol with that print name.
It returns the new, uninterned symbol.

The invented print name consists of a prefix
(which defaults to \cdf{G}), followed by the decimal representation of a number.

\cdf{gensym} is usually used to create a symbol that should not normally
be seen by the user and whose print name is unimportant except to
allow easy distinction by eye between two such symbols.
The optional argument is rarely supplied.
The name comes from ``generate symbol,'' and the symbols produced by it
are often called ``gensyms.''

If it is desirable
for the generated symbols to be interned, and yet guaranteed to be
symbols distinct from all others,
then the function \cdf{gentemp}
may be more appropriate to use.

\begin{newer}
X3J13 voted in March 1989 \issue{GENSYM-NAME-STICKINESS}
to alter the specification of \cdf{gensym} so that supplying an
optional argument (whether a string or a number) does \emph{not} alter
the internal state maintained by \cdf{gensym}.
Instead, the internal
counter is made explicitly available as a variable named \cd{*gensym-counter*}.

If a string argument is given to \cdf{gensym}, that string is used as the prefix;
otherwise ``\cdf{G}'' is used.  If a number is provided, its decimal
representation is used, but the internal counter is unaffected.
X3J13 deprecates the use of a number as an argument.
\end{newer}
\end{defun}

\begin{defun}[Variable]
*gensym-counter*

\cdf{*gensym-counter*}
holds the state of the \cdf{gensym} counter; that is, \cdf{gensym}
uses the decimal representation of its value as part of the generated name
and then increments its value.

The initial value of this variable is implementation-dependent
but will be a non-negative integer.

The user may assign to or bind this variable at any time, but its value
must always be a non-negative integer.
\end{defun}

\begin{defun}[Function]
gentemp &optional prefix package

\cdf{gentemp}, like \cdf{gensym}, creates and returns a new symbol.
\cdf{gentemp} differs from \cdf{gensym} in that it interns the symbol
(see \cdf{intern}) in the \emph{package} (which defaults to the current
package; see \cdf{*package*}).  \cdf{gentemp} guarantees the symbol
will be a new one not already existing in the package.  It does this
by using a counter as \cdf{gensym} does, but if the generated symbol
is not really new, then the process is repeated until a new one is created.
There is no provision for resetting the \cdf{gentemp} counter.
Also, the prefix for \cdf{gentemp} is not remembered from one call
to the next; if \emph{prefix} is omitted, the default prefix \cdf{T} is used.
\end{defun}

\begin{defun}[Function]
symbol-package sym

Given a symbol \emph{sym}, \cdf{symbol-package} returns the contents of the
package cell of that symbol.  This will be a package object or {\nil}.
\end{defun}

\begin{defun}[Function]
keywordp object

The argument may be any Lisp object.  The predicate \cdf{keywordp} is true
if the argument is a symbol and that
symbol is a keyword (that is, belongs to the keyword
package).  Keywords are those symbols that are written with
a leading colon.  Every keyword is a constant, in the sense
that it always evaluates to itself.  See \cdf{constantp}.
\end{defun}

%RUSSIAN
\else

\chapter{Символы}
\label{symbol}

Lisp'овые символы является объектами данных, которые имеют три элемента, видимых
для пользователя:
\begin{itemize}
\item
\emph{Список свойств} является списком, который позволяет хранить в символе
именованные изменяемые данные.

\item
\emph{Выводимое имя} должно быть строкой, которая является последовательностью
строковых символов, идентифицирующей символ. Символы несут большую пользу, так
как они могут быть обозначены просто заданным именем (например, напечатанным на
клавиатуре).
Выводимое имя изменять нельзя.

\item
\emph{Ячейка пакета} должна ссылаться на объект пакета.
Пакет является структурой данных, используемой для группирования имен символов.
Символ уникально идентифицируется по имени, только когда рассматривается
относительно пакета. Символ может встречаться в нескольких пакетах, но
\emph{родительским} пакетом может быть как максимум только один.
Ячейка пакета ссылается на родительский пакет, если он есть.
Ячейки пакетов обсуждаются в главе~\ref{XPACK}.
\end{itemize}

Символ может также содержать другие элементы, которые используются
реализацией. Ещё одна важная функция, это использование символов в качестве
имен переменных.  Желательно, чтобы разработчик использовал такие элементы
символа для реализации семантики переменных. Смотрите \cdf{symbol-value} и
\cdf{symbol-function}. Однако, существует несколько стратегий реализации, и
такие возможные элементы символов здесь не описаны.

\section{Список свойств}

Начиная с самого создания, Lisp для каждого символа ассоциирует табличную
структуру данных, называемую \emph{список свойств} (для краткости \emph{plist}).
Список свойств содержит ноль и более элементов. Каждый элемент содержит ключ
(называемым \emph{индикатором}), который чаще всего является символом, и
ассоциированным с ним значением (называемым иногда \emph{свойством}), которое
может быть любым Lisp'овым объектом.
Среди индикаторов не может быть дубликатов. Список свойств может иметь только
одно свойство для данного имени. Таким образом, значение может получено с
помощью двух символов: исходного символа и индикатора.

Список свойств по целевому назначению очень похож на ассоциативный
список. Различие в том, что список свойств является единственно подлинным
объектом. Операции добавления и удаления элементов деструктивны, то есть при их
использовании изменяется старый список, и нового списка свойств не создаётся.
Ассоциативные список, наоборот, обычно изменяются недеструктивно (без побочных
эффектов) с помощью добавления в начало новых элементов (смотрите \cdf{acons} и
\cdf{pairlis}).

Список свойств реализуется, как ячейка памяти, содержащая список с чётным
(возможно нулевым) количеством аргументов.
(Обычно эта ячейка памяти является ячейкой списка свойств в символе, но в
принципе подходит любая ячейка памяти, к которой можно применить \cdf{setf})
Каждая пара элементов в списке составляет строку.
Первый в паре это индикатор, а второй --- значение. Так как функции для списка
свойств используют символ, а не сам список, то изменения этого списка свойств
может быть записаны сохранением обратно в ячейку списка свойств символа. FIXME

Когда создаётся символ, его список свойств пуст. Свойства создаются с помощью
\cdf{get} внутри формы \cdf{setf}.

Common Lisp не использует список свойств символа так интенсивно, как это делали
ранние реализации Lisp'а. В Common Lisp'е нечасто используемые данные, такие как
отладочная информация, информация для компилятора и документация, хранятся в списках
свойств.

\begin{defun}[Функция]
get symbol indicator &optional default

\cdf{get} ищет в списке свойств символа \emph{symbol} индикатор равный \cdf{eq}
индикатору \emph{indicator}.
Первый аргумент должен быть символом.
Если такое свойство найдено, возвращается её значение. Иначе возвращается
значение \emph{default}.

Если \emph{default} не указано, тогда для значения по-умолчанию используется
{\false}.

Следует отметить, что способа отличить значения по-умолчанию и такое же значение
свойства нет:
\begin{lisp}
(get x y) \EQ\ (getf (symbol-plist x) y)
\end{lisp}
Допустим, что список свойств символа \cd{foo} является \cd{(bar t baz 3 hunoz
  "Huh?")}.
Тогда, например:
\begin{lisp}
(get 'foo 'baz) \EV\ 3 \\
(get 'foo 'hunoz) \EV\ "Huh?" \\
(get 'foo 'zoo) \EV\ {\false}
\end{lisp}

\cdf{setf} может использоваться вместе с \cdf{get} для создания новой пары
свойства, возможно замещая старую пару с тем же именем.
Например:
\begin{lisp}
(get 'clyde 'species) \EV\ {\false} \\
(setf (get 'clyde 'species) 'elephant) \EV\ elephant \\
\textrm{и теперь} (get 'clyde 'species) \EV\ elephant
\end{lisp}
В данном контексте может быть указан аргумент \emph{default}. Он игнорируется в
\cdf{setf}, но может быть полезен в таких макросах, как \cdf{push}, которые
связаны с \cdf{setf}:
\begin{lisp}
(push item (get sym 'token-stack '(initial-item)))
\end{lisp}
означает то же, что и
\begin{lisp}
(setf (get sym 'token-stack '(initial-item)) \\
~~~~~~(cons item (get sym 'token-stack '(initial-item))))
\end{lisp}
а если упростить, то
\begin{lisp}
(setf (get sym 'token-stack) \\
~~~~~~(cons item (get sym 'token-stack '(initial-item))))
\end{lisp}
\end{defun}

\begin{defun}[Функция]
remprop symbol indicator

Эта функция удаляет из символа \emph{symbol} свойство с индикатором, равным
\cdf{eq} индикатору \emph{indicator}. Индикатор свойства и соответствующее
значение удаляется из списка деструктивной склейкой списка свойств. Она
возвращает {\false}, если указанного свойства не было, или не-{\false}, если
свойство было.
\begin{lisp}
(remprop x y) \EQ\ (remf (symbol-plist x) y)
\end{lisp}
Например, если список свойств \cd{foo} равен
\begin{lisp}
(color blue height 6.3 near-to bar)
\end{lisp}
тогда вызов
\begin{lisp}
(remprop 'foo 'height)
\end{lisp}
вернёт значение не-{\nil}, после изменения списка свойств символа \cd{foo} на
\begin{lisp}
(color blue near-to bar)
\end{lisp}
\end{defun}

\begin{defun}[Функция]
symbol-plist symbol

Эта функция возвращает список, который содержит пары свойств для символа
\emph{symbol}. Извлекается содержимое ячейки списка свойств и возвращается в
качестве результата.

Следует отметить, что использование \cdf{get} с результатом \cdf{symbol-plist}
\emph{не} будет работать. Необходимо передавать в \cdf{get} символ, или же
использовать \cdf{getf}.

С \cdf{symbol-plist} может использоваться \cdf{setf} для деструктивной замены
списка свойств символа. Это относительно опасная операция, так как может
уничтожить важную информацию, которую, возможно, хранила там реализация.
Также, позаботьтесь о том, чтобы новый список свойств содержал чётное количество
элементов.
\end{defun}

\begin{defun}[Функция]
getf place indicator &optional default

\cdf{getf} ищет индикатор равный \cdf{eq} \emph{indicator} в списке свойств,
находящимся в \emph{place}. Если он найден, тогда возвращается соответствующее
значение. Иначе возвращается \emph{default}. Если \emph{default} не задан, то
значение по-умолчанию используется {\false}.
Следует отметить, что метода определения, вернулось ли значение по умолчанию или
это значение свойства, нет.
Часто \emph{place} вычисляется из обобщённой переменной, принимаемой функцией
\cdf{setf}.

\cdf{setf} может использоваться вместе \cdf{getf}, и в этом случае \emph{place}
должно быть обобщённым, чтобы его можно было передать в \cdf{setf}. Целью
является добавление новой пары свойства, или изменению уже существующей пары, в
списке свойств, хранящимся в \emph{place}.
В данном контексте может быть использован аргумент \emph{default}. Он
игнорируется функцией \cdf{setf}, но может быть полезен в таких макросах, как
\cdf{push}, которые связаны с \cdf{setf}.
Смотрите описание \cdf{get} для примера.
\end{defun}

\begin{defmac}
remf place indicator

Эта функция в списке свойств, хранящимся в \emph{place}, удаляет свойства,
индикатор которого равен \cdf{eq} аргументу \emph{indicator}. Индикатор свойства
и соответствующее значение удаляется из списка деструктивной
склейкой. \cdf{remf} возвращает {\false}, если свойства не было в списке, и
некоторое не-{\nil} значение, если свойство было.
Форма \emph{place} может быть любой обобщённой переменной, принимаемой
\cdf{setf}.
Смотрите \cdf{remprop}.
\end{defmac}

\begin{defun}[Функция]
get-properties place indicator-list

\cdf{get-properties} похожа на \cdf{getf} за исключением того, что второй
аргумент является списком индикаторов. \cdf{get-properties} ищет первый
встретившийся индикатор, который есть в списке индикаторов. Обычно \emph{place}
вычисляется из обобщённой переменной, которая может использоваться в \cdf{setf}.

\cdf{get-properties} возвращает три значения.
Если было найдено свойство, то первые два значения являются индикатором и
значением для первого свойства, чей индикатор присутствовал в списке
\emph{indicator-list}, и третье значение является остатком списка свойств,
\emph{car} элемент которого был индикатором (и соответственно \emph{cadr}
элемент --- значением).
Если ни одного свойства не было найдено, то все три значения равны {\nil}.
Третье значение является флагом успешности или неудачи, и позволяет продолжить
поиск свойств в оставшемся списке.
\end{defun}

\section{Выводимое имя}

Каждый символ имеет ассоциированную строку, называемую \emph{выводимое имя}.
Строка используется для вывода отображения символа:
если символы в строке будут напечатаны в функцию \cdf{read} (с необходимым, где
надо, экранированием), они будут интерпретированы как ссылка на этот символ
(если он интернирован). Если символ выводится куда-либо, \cdf{print} печатает
его выводимое имя.
Для подробностей, смотрите раздел о \emph{читателе} (раздел~\ref{READER}) и
\emph{писателе} (раздел~\ref{PRINTER}).

\begin{defun}[Функция]
symbol-name sym

Данная функция возвращает выводимое имя для символа \emph{sym}.
Например:
\begin{lisp}
(symbol-name 'xyz) \EV\ "XYZ"
\end{lisp}
Это не очень хорошая идея, изменять строку, которая используется, как выводимое
имя символа. Такое изменение может сильно запутать функцию \cdf{read} и систему
пакетов.
\end{defun}

\section{Создание символов}

Символы могут использоваться двумя различными способами.
\emph{Интернированный} символ один из них, который определяется его выводимым
именем и каталогом, называемым \emph{пакет}.
A request to locate a symbol with that print name results
in the same (\cdf{eq}) symbol.  Every time input is read with the
function \cdf{read},
and that print name appears, it is read as the same symbol.
This property of symbols makes them appropriate to use as names for
things and as hooks on which to hang permanent data objects
(using the property list, for example).

Интернированные символы обычно создаются автоматически. Во время первого
обращения к символу в системе пакетов (с помощью \cdf{read}, например), данный
символ создаётся автоматически. Для обращения к
интернированному символу используется функция \cdf{inter}, или другая с ней
связанная.

Несмотря на то, что интернированные символы используются чаще всего, они не
будут больше здесь рассматриваться. Для подробной информации смотрите
главу~\ref{XPACK}.

\emph{Неинтернированный} символ является символом, используемым в качестве
объекта данных, без связи с каталогом (он не имеет родительского пакета).
Неинтернированный символ выводится как \cd{\#:} с последующим выводимым именем.

\begin{defun}[Функция]
make-symbol print-name

\cd{(make-symbol \emph{print-name})} создаёт новый неинтернированный символ, у
которого выводимое имя является строкой \emph{print-name}. Полученный символ не
имеет значения функции (unbound) и пустой список свойств.

Строка, переданная в аргументе, может использоваться сразу, либо сначала
копироваться. Это зависит от реализации.
Пользователь не может полагаться на то, что \cd{(symbol-name
  (make-symbol x))} равно \cdf{eq} \cdf{x}, но и также не может изменять строку
переданную в качестве аргумента для \cdf{make-symbol}.
\end{defun}

\begin{defun}[Функция]
copy-symbol sym &optional copy-props

Эта функция возвращает новый неинтернированный символ с тем же выводимым именем,
что \emph{sym}.

Если \emph{copy-props} не-{\nil}, тогда начальное значение и определение функции
нового символа будут те же, что и в переданном \emph{sym}, а список свойств будет
скопирован из исходного символа.

Если \emph{copy-props} является {\nil} (по-умолчанию), тогда новый символ не
будет связан, не будет иметь определения функции, и список свойств будет пуст.
\end{defun}

\begin{defun}[Функция]
gensym &optional x

\cdf{gensym} создаёт выводимое имя и создаёт новый символ с этим именем.
Она возвращает новый неинтернированный символ.

Созданное имя содержит префикс (по-умолчанию \cd{G}), с последующим десятичным
представлением числа.

\cdf{gensym} обычно используется для создания символа, который не виден
пользователю, и его имя не имеет важности.
Необязательный аргумент используется нечасто. Имя образовано от <<генерация
символа>>, и символы созданные, таким образом, часто называются <<gensyms>>.

Если необходимо, чтобы сгенерированные символы были интернированными и отличными
от существующих символов, тогда удобно использовать функцию \cdf{gentemp}.
\end{defun}

\begin{defun}[Переменная]
*gensym-counter*

Переменная хранит состояние счётчика для функции \cdf{gensym}. \cdf{gensym}
использует десятичное представление этого значения в качестве части имени
генерируемого символа, а затем наращивает этот счётчик.

Первоначальное значение этой переменной зависит от реализации, но должно быть
неотрицательным целым.

Пользователь в любое время может присваивать или связывать это переменную, но
значение должно быть неотрицательным целым.
\end{defun}

\begin{defun}[Функция]
gentemp &optional prefix package

\cdf{gentemp}, как и \cdf{gensym}, создаёт и возвращает новый символ.
\cdf{gentemp} отличается от \cdf{gensym} в том, что возвращает интернированный
символ (смотрите \cdf{intern}) в пакете \emph{package} (который по-умолчанию
является текущим, смотрите \cdf{*package*}). \cdf{gentemp} гарантирует, что
символ будет новым, и не существовал ранее в указанном пакете. Она также
использует счётчик, однако если полученный символ уже существует счётчик
наращивается, и действия повторяются, пока не будет найдено имя ещё не
существующего символа.
Сбросить счётчик невозможно.
Кроме того, префикс для \cdf{gentemp} не сохраняется между вызовами. Если
аргумент \emph{prefix} опущен, то используется значение по-умолчанию \cdf{T}.
\end{defun}

\begin{defun}[Функция]
symbol-package sym

Для заданного символа возвращает содержимое ячейки пакета. Результат может быть
объектом пакета или {\nil}.
\end{defun}

\begin{defun}[Функция]
keywordp object

Аргумент может быть любым Lisp'овым объектом. Предикат \cdf{keywordp} истинен,
если аргумент является символом и этот символ является ключевым (значит, что
принадлежит пакету ключевых символов). Ключевые символы --- это символы, которые
записываются с двоеточием в начала. Каждый ключевой символ является константой,
в смысле, что выполняются сами в себя. Смотрите \cdf{constantp}.
\end{defun}

\fi