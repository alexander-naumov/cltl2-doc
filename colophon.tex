%Part{Dtypes, Root = "CLM.MSS"}
% Chapter of Common Lisp Manual.  Copyright 1984, 1988, 1989 Guy L. Steele Jr.

\clearpage\def\pagestatus{FINAL PROOF}

%\chapter*{Colophon\markboth{Colophon}{Colophon}}
\chapter*{Colophon}


Camera-ready copy for this book was created by the author (using \TeX, \LaTeX,
and \TeX\ macros written by the author), proofed on an Apple LaserWriter II, and
printed on a Linotron 300 at Advanced Computer Graphics.  The text of the first
edition was converted from Scribe format to \TeX\ format by a throwaway program
written in Common Lisp.  The diagrams in chapter 12 were generated automatically
as PostScript code (by a program written in Common Lisp) and integrated into the
text by Textures, an implementation of \TeX\ by Blue Sky Research for the Apple
Macintosh computer.

The body type is 10-point Times Roman. Chapter titles are in ITC Eras Demi;
running heads and chapter subtitles are in ITC Eras Book.  The monospace typeface used for program
code in both displays and running text is 8.5-point Letter Gothic Bold, somewhat
modified by the author through TEX macros for improved legibility. The accent
grave (\cd{\Xbq}), accent acute(\cd{\Xquote}),
circumflex (\cd{\Xcircumflex}), and tilde (\cd{\Xtilde})
characters are in 10-point Letter Gothic
Bold and adjusted vertically to match the height of the 8.5-point characters.  The
hyphen (\cd{\char45}) was replaced by an en dash (\cdf{-}).
The equals sign (\cd{\char61}) was replaced by a construction of two em
dashes (\cdf{=}), one raised and one lowered, the better to match the other
relational characters.  The sharp sign (\cd{\char35})
is overstruck with two hyphens,
one raised and one lowered, to eliminate the vertical gap (\cd{\#}).  Special mathematical
characters such as square-root signs are in Computer Modern Math. The typefaces
used in this book were digitized by Adobe Systems Incorporated, except for
Computer Modern Math, which was designed by Donald E. Knuth.

%%% Local Variables: 
%%% mode: latex
%%% TeX-master: "clm"
%%% End: 
